
\documentclass[11pt]{report}
\usepackage[margin=1in]{geometry}
\usepackage{amsmath,amssymb,amsthm}
\usepackage{hyperref}
\usepackage{bookmark}
\usepackage{enumitem}
\usepackage{xcolor}
\hypersetup{colorlinks=true,linkcolor=blue!60!black}
\setlist[itemize]{leftmargin=*}

\newcommand{\phase}[1]{\chapter{#1}}
\newcommand{\authorblock}[1]{\section{#1}}
\newcommand{\chapterblock}[1]{\subsection{#1}}
\newcommand{\readingsection}{\subsubsection*{Assigned Reading}}
\newcommand{\notessection}{\subsubsection*{Notes}}

\begin{document}

\begin{titlepage}
\centering
\vspace*{2.5cm}
{\Huge\bfseries My Mathematical Journey\par}
\vspace{1.2cm}
{\Large Phase 1 Bundle: Worksheets include Proofs\par}
\vfill
{\large ``Proof is the price of admission.''}
\end{titlepage}

\tableofcontents
\clearpage

\phase{Phase 1: Foundations (Logic, Sets, Real Numbers)}


\authorblock{Terence Tao — \textit{Analysis I} (3rd Edition)}
\chapterblock{Chapter 1}
\hypertarget{P1-TAO-C01}{}

\readingsection
\begin{itemize}
  \item (Fill in exact section/page ranges you assign for Chapter 1 in this text)
\end{itemize}

\notessection
\begin{itemize}
  \item (Your notes here)
\end{itemize}

\subsubsection*{Worksheet (included)}
\noindent\textbf{Worksheet file:} \texttt{\detokenize{worksheets/phase1/tao/ch01/TAO_P1_C01_WS_001.tex}}\\
% Fragment (no \documentclass)

\subsubsection*{TAO-P1-C01-WS-001 (10 problems)}\n\hypertarget{TAO-P1-C01-WS-001}{}\n\n\noindent\textbf{Navigation: }
\hyperlink{P1-TAO-C01}{Back to chapter}\n\n\begin{enumerate}
\item \textbf{Contrapositive equivalence.} Prove that $P \Rightarrow Q$ is logically equivalent to $\neg Q \Rightarrow \neg P$. (\hyperlink{TAO-P1-C01-001}{proof})
\item \textbf{Implication as disjunction.} Prove that $P \Rightarrow Q$ is logically equivalent to $\neg P \lor Q$. (\hyperlink{TAO-P1-C01-002}{proof})
\item \textbf{Negation of forall.} Prove that $\neg(\forall x\, P(x)) \iff \exists x\, \neg P(x)$. (\hyperlink{TAO-P1-C01-003}{proof})
\item \textbf{Negation of exists.} Prove that $\neg(\exists x\, P(x)) \iff \forall x\, \neg P(x)$. (\hyperlink{TAO-P1-C01-004}{proof})
\item \textbf{Transitivity of implication.} Give a formal proof that if $P \Rightarrow Q$ and $Q \Rightarrow R$, then $P \Rightarrow R$. (\hyperlink{TAO-P1-C01-005}{proof})
\item \textbf{Logical equivalence is an equivalence relation.} Prove that logical equivalence is an equivalence relation on the set of propositions. (\hyperlink{TAO-P1-C01-006}{proof})
\item \textbf{False implies anything.} Prove that if $P$ is false, then $P \Rightarrow Q$ is true for any $Q$. (\hyperlink{TAO-P1-C01-007}{proof})
\item \textbf{Equivalence preserves truth value.} Prove that if $P \Leftrightarrow Q$, then $P$ and $Q$ have the same truth value. (\hyperlink{TAO-P1-C01-008}{proof})
\item \textbf{Induction: sum 1..n.} Using induction, prove that for all $n \in \mathbb{N}$, $1+2+\cdots+n=\frac{n(n+1)}{2}$. (\hyperlink{TAO-P1-C01-009}{proof})
\item \textbf{Quantifier dependency.} Prove that a statement of the form $\forall \varepsilon>0\,\exists \delta>0$ does not assert the existence of a universal $\delta$. (\hyperlink{TAO-P1-C01-010}{proof})
\end{enumerate}

\subsubsection*{Proofs (included)}
% Fragment (no \documentclass)

\hypertarget{TAO-P1-C01-001}{}\n\subsubsection*{TAO-P1-C01-001: Contrapositive equivalence}\n\n\noindent\textbf{Navigation: }
\hyperlink{TAO-P1-C01-WS-001}{Back to worksheet}\n\quad|\quad
\hyperlink{P1-TAO-C01}{Back to chapter}\n\n\textbf{Claim.} Prove that $P \Rightarrow Q$ is logically equivalent to $\neg Q \Rightarrow \neg P$.\n\n\textbf{Proof.}
Write your proof here.
\qed

% Fragment (no \documentclass)

\hypertarget{TAO-P1-C01-002}{}\n\subsubsection*{TAO-P1-C01-002: Implication as disjunction}\n\n\noindent\textbf{Navigation: }
\hyperlink{TAO-P1-C01-WS-001}{Back to worksheet}\n\quad|\quad
\hyperlink{P1-TAO-C01}{Back to chapter}\n\n\textbf{Claim.} Prove that $P \Rightarrow Q$ is logically equivalent to $\neg P \lor Q$.\n\n\textbf{Proof.}
Write your proof here.
\qed

% Fragment (no \documentclass)

\hypertarget{TAO-P1-C01-003}{}\n\subsubsection*{TAO-P1-C01-003: Negation of forall}\n\n\noindent\textbf{Navigation: }
\hyperlink{TAO-P1-C01-WS-001}{Back to worksheet}\n\quad|\quad
\hyperlink{P1-TAO-C01}{Back to chapter}\n\n\textbf{Claim.} Prove that $\neg(\forall x\, P(x)) \iff \exists x\, \neg P(x)$.\n\n\textbf{Proof.}
Write your proof here.
\qed

% Fragment (no \documentclass)

\hypertarget{TAO-P1-C01-004}{}\n\subsubsection*{TAO-P1-C01-004: Negation of exists}\n\n\noindent\textbf{Navigation: }
\hyperlink{TAO-P1-C01-WS-001}{Back to worksheet}\n\quad|\quad
\hyperlink{P1-TAO-C01}{Back to chapter}\n\n\textbf{Claim.} Prove that $\neg(\exists x\, P(x)) \iff \forall x\, \neg P(x)$.\n\n\textbf{Proof.}
Write your proof here.
\qed

% Fragment (no \documentclass)

\hypertarget{TAO-P1-C01-005}{}\n\subsubsection*{TAO-P1-C01-005: Transitivity of implication}\n\n\noindent\textbf{Navigation: }
\hyperlink{TAO-P1-C01-WS-001}{Back to worksheet}\n\quad|\quad
\hyperlink{P1-TAO-C01}{Back to chapter}\n\n\textbf{Claim.} Give a formal proof that if $P \Rightarrow Q$ and $Q \Rightarrow R$, then $P \Rightarrow R$.\n\n\textbf{Proof.}
Write your proof here.
\qed

% Fragment (no \documentclass)

\hypertarget{TAO-P1-C01-006}{}\n\subsubsection*{TAO-P1-C01-006: Logical equivalence is an equivalence relation}\n\n\noindent\textbf{Navigation: }
\hyperlink{TAO-P1-C01-WS-001}{Back to worksheet}\n\quad|\quad
\hyperlink{P1-TAO-C01}{Back to chapter}\n\n\textbf{Claim.} Prove that logical equivalence is an equivalence relation on the set of propositions.\n\n\textbf{Proof.}
Write your proof here.
\qed

% Fragment (no \documentclass)

\hypertarget{TAO-P1-C01-007}{}\n\subsubsection*{TAO-P1-C01-007: False implies anything}\n\n\noindent\textbf{Navigation: }
\hyperlink{TAO-P1-C01-WS-001}{Back to worksheet}\n\quad|\quad
\hyperlink{P1-TAO-C01}{Back to chapter}\n\n\textbf{Claim.} Prove that if $P$ is false, then $P \Rightarrow Q$ is true for any $Q$.\n\n\textbf{Proof.}
Write your proof here.
\qed

% Fragment (no \documentclass)

\hypertarget{TAO-P1-C01-008}{}\n\subsubsection*{TAO-P1-C01-008: Equivalence preserves truth value}\n\n\noindent\textbf{Navigation: }
\hyperlink{TAO-P1-C01-WS-001}{Back to worksheet}\n\quad|\quad
\hyperlink{P1-TAO-C01}{Back to chapter}\n\n\textbf{Claim.} Prove that if $P \Leftrightarrow Q$, then $P$ and $Q$ have the same truth value.\n\n\textbf{Proof.}
Write your proof here.
\qed

% Fragment (no \documentclass)

\hypertarget{TAO-P1-C01-009}{}\n\subsubsection*{TAO-P1-C01-009: Induction: sum 1..n}\n\n\noindent\textbf{Navigation: }
\hyperlink{TAO-P1-C01-WS-001}{Back to worksheet}\n\quad|\quad
\hyperlink{P1-TAO-C01}{Back to chapter}\n\n\textbf{Claim.} Using induction, prove that for all $n \in \mathbb{N}$, $1+2+\cdots+n=\frac{n(n+1)}{2}$.\n\n\textbf{Proof.}
Write your proof here.
\qed

% Fragment (no \documentclass)

\hypertarget{TAO-P1-C01-010}{}\n\subsubsection*{TAO-P1-C01-010: Quantifier dependency}\n\n\noindent\textbf{Navigation: }
\hyperlink{TAO-P1-C01-WS-001}{Back to worksheet}\n\quad|\quad
\hyperlink{P1-TAO-C01}{Back to chapter}\n\n\textbf{Claim.} Prove that a statement of the form $\forall \varepsilon>0\,\exists \delta>0$ does not assert the existence of a universal $\delta$.\n\n\textbf{Proof.}
Write your proof here.
\qed



\authorblock{Thomson--Bruckner — \textit{Elementary Real Analysis}}
\chapterblock{Chapter 1}
\hypertarget{P1-TB-C01}{}

\readingsection
\begin{itemize}
  \item (Fill in exact section/page ranges you assign for Chapter 1 in this text)
\end{itemize}

\notessection
\begin{itemize}
  \item (Your notes here)
\end{itemize}

\subsubsection*{Worksheet (included)}
\noindent\textbf{Worksheet file:} \texttt{\detokenize{worksheets/phase1/thomson_bruckner/ch01/TB_P1_C01_WS_001.tex}}\\
% Fragment (no \documentclass)

\subsubsection*{TB-P1-C01-WS-001 (10 problems)}\n\hypertarget{TB-P1-C01-WS-001}{}\n\n\noindent\textbf{Navigation: }
\hyperlink{P1-TB-C01}{Back to chapter}\n\n\begin{enumerate}
\item \textbf{De Morgan's law.} Prove $(A\cup B)^c = A^c \cap B^c$ (complements relative to a fixed universe). (\hyperlink{TB-P1-C01-001}{proof})
\item \textbf{Set difference distributes over union.} Prove $A\setminus (B\cup C) = (A\setminus B)\cap (A\setminus C)$. (\hyperlink{TB-P1-C01-002}{proof})
\item \textbf{Symmetry of symmetric difference.} Prove that $A \triangle B = B \triangle A$. (\hyperlink{TB-P1-C01-003}{proof})
\item \textbf{Zero symmetric difference iff equality.} Prove that $A \triangle B=\varnothing$ if and only if $A=B$. (\hyperlink{TB-P1-C01-004}{proof})
\item \textbf{Preimage preserves union.} Let $f:X\to Y$. Prove $f^{-1}(E\cup F)=f^{-1}(E)\cup f^{-1}(F)$. (\hyperlink{TB-P1-C01-005}{proof})
\item \textbf{Image of preimage subset.} Prove $f(f^{-1}(E))\subseteq E$ and give a condition for equality. (\hyperlink{TB-P1-C01-006}{proof})
\item \textbf{Injective preserves intersections of images.} If $f$ is injective, prove $f(A\cap B)=f(A)\cap f(B)$. (\hyperlink{TB-P1-C01-007}{proof})
\item \textbf{Equivalence classes partition.} Define an equivalence relation and prove that equivalence classes form a partition. (\hyperlink{TB-P1-C01-008}{proof})
\item \textbf{Intersecting classes equal.} Prove that if $[x]\cap [y]\neq\varnothing$, then $[x]=[y]$. (\hyperlink{TB-P1-C01-009}{proof})
\item \textbf{Set equality is an equivalence relation.} Prove that equality of sets is an equivalence relation. (\hyperlink{TB-P1-C01-010}{proof})
\end{enumerate}

\subsubsection*{Proofs (included)}
% Fragment (no \documentclass)

\hypertarget{TB-P1-C01-001}{}\n\subsubsection*{TB-P1-C01-001: De Morgan's law}\n\n\noindent\textbf{Navigation: }
\hyperlink{TB-P1-C01-WS-001}{Back to worksheet}\n\quad|\quad
\hyperlink{P1-TB-C01}{Back to chapter}\n\n\textbf

{Claim.} Prove $(A\cup B)^c = A^c \cap B^c$ (complements relative to a fixed universe).\n\n\textbf

{Proof.}



\begin{proof}
Let $U$ be the fixed universe and let $A,B\subseteq U$. We prove equality by showing mutual inclusion.

\medskip
\noindent\textbf{($\subseteq$)} Let $x\in (A\cup B)^c$. By definition of complement, this means $x\notin A\cup B$.
By definition of union, $x\notin A\cup B$ implies $(x\notin A)\ \wedge\ (x\notin B)$.
Equivalently, $x\in A^c$ and $x\in B^c$. Hence $x\in A^c\cap B^c$.

\medskip
\noindent\textbf{($\supseteq$)} Let $x\in A^c\cap B^c$. Then $x\in A^c$ and $x\in B^c$, so $x\notin A$ and $x\notin B$.
Therefore $x\notin A\cup B$ (since membership in a union would require membership in at least one of the sets).
Thus $x\in (A\cup B)^c$.

\medskip
Since $(A\cup B)^c\subseteq A^c\cap B^c$ and $A^c\cap B^c\subseteq (A\cup B)^c$, we conclude
\[
(A\cup B)^c = A^c\cap B^c.\qedhere
\]
\end{proof}




\qed

% Fragment (no \documentclass)

\hypertarget{TB-P1-C01-002}{}\n\subsubsection*{TB-P1-C01-002: Set difference distributes over union}\n\n\noindent\textbf{Navigation: }
\hyperlink{TB-P1-C01-WS-001}{Back to worksheet}\n\quad|\quad
\hyperlink{P1-TB-C01}{Back to chapter}\n\n\textbf{Claim.} Prove $A\setminus (B\cup C) = (A\setminus B)\cap (A\setminus C)$.\n\n\textbf{Proof.}
Write your proof here.
\qed

% Fragment (no \documentclass)

\hypertarget{TB-P1-C01-003}{}\n\subsubsection*{TB-P1-C01-003: Symmetry of symmetric difference}\n\n\noindent\textbf{Navigation: }
\hyperlink{TB-P1-C01-WS-001}{Back to worksheet}\n\quad|\quad
\hyperlink{P1-TB-C01}{Back to chapter}\n\n\textbf{Claim.} Prove that $A \triangle B = B \triangle A$.\n\n\textbf{Proof.}
Write your proof here.
\qed

% Fragment (no \documentclass)

\hypertarget{TB-P1-C01-004}{}\n\subsubsection*{TB-P1-C01-004: Zero symmetric difference iff equality}\n\n\noindent\textbf{Navigation: }
\hyperlink{TB-P1-C01-WS-001}{Back to worksheet}\n\quad|\quad
\hyperlink{P1-TB-C01}{Back to chapter}\n\n\textbf{Claim.} Prove that $A \triangle B=\varnothing$ if and only if $A=B$.\n\n\textbf{Proof.}
Write your proof here.
\qed

% Fragment (no \documentclass)

\hypertarget{TB-P1-C01-005}{}\n\subsubsection*{TB-P1-C01-005: Preimage preserves union}\n\n\noindent\textbf{Navigation: }
\hyperlink{TB-P1-C01-WS-001}{Back to worksheet}\n\quad|\quad
\hyperlink{P1-TB-C01}{Back to chapter}\n\n\textbf{Claim.} Let $f:X\to Y$. Prove $f^{-1}(E\cup F)=f^{-1}(E)\cup f^{-1}(F)$.\n\n\textbf{Proof.}
Write your proof here.
\qed

% Fragment (no \documentclass)

\hypertarget{TB-P1-C01-006}{}\n\subsubsection*{TB-P1-C01-006: Image of preimage subset}\n\n\noindent\textbf{Navigation: }
\hyperlink{TB-P1-C01-WS-001}{Back to worksheet}\n\quad|\quad
\hyperlink{P1-TB-C01}{Back to chapter}\n\n\textbf{Claim.} Prove $f(f^{-1}(E))\subseteq E$ and give a condition for equality.\n\n\textbf{Proof.}
Write your proof here.
\qed

% Fragment (no \documentclass)

\hypertarget{TB-P1-C01-007}{}\n\subsubsection*{TB-P1-C01-007: Injective preserves intersections of images}\n\n\noindent\textbf{Navigation: }
\hyperlink{TB-P1-C01-WS-001}{Back to worksheet}\n\quad|\quad
\hyperlink{P1-TB-C01}{Back to chapter}\n\n\textbf{Claim.} If $f$ is injective, prove $f(A\cap B)=f(A)\cap f(B)$.\n\n\textbf{Proof.}
Write your proof here.
\qed

% Fragment (no \documentclass)

\hypertarget{TB-P1-C01-008}{}\n\subsubsection*{TB-P1-C01-008: Equivalence classes partition}\n\n\noindent\textbf{Navigation: }
\hyperlink{TB-P1-C01-WS-001}{Back to worksheet}\n\quad|\quad
\hyperlink{P1-TB-C01}{Back to chapter}\n\n\textbf{Claim.} Define an equivalence relation and prove that equivalence classes form a partition.\n\n\textbf{Proof.}
Write your proof here.
\qed

% Fragment (no \documentclass)

\hypertarget{TB-P1-C01-009}{}\n\subsubsection*{TB-P1-C01-009: Intersecting classes equal}\n\n\noindent\textbf{Navigation: }
\hyperlink{TB-P1-C01-WS-001}{Back to worksheet}\n\quad|\quad
\hyperlink{P1-TB-C01}{Back to chapter}\n\n\textbf{Claim.} Prove that if $[x]\cap [y]\neq\varnothing$, then $[x]=[y]$.\n\n\textbf{Proof.}
Write your proof here.
\qed

% Fragment (no \documentclass)

\hypertarget{TB-P1-C01-010}{}\n\subsubsection*{TB-P1-C01-010: Set equality is an equivalence relation}\n\n\noindent\textbf{Navigation: }
\hyperlink{TB-P1-C01-WS-001}{Back to worksheet}\n\quad|\quad
\hyperlink{P1-TB-C01}{Back to chapter}\n\n\textbf{Claim.} Prove that equality of sets is an equivalence relation.\n\n\textbf{Proof.}
Write your proof here.
\qed



\authorblock{Kolmogorov--Fomin — \textit{Introductory Real Analysis}}
\chapterblock{Chapter 1}
\hypertarget{P1-KF-C01}{}

\readingsection
\begin{itemize}
  \item (Fill in exact section/page ranges you assign for Chapter 1 in this text)
\end{itemize}

\notessection
\begin{itemize}
  \item (Your notes here)
\end{itemize}

\subsubsection*{Worksheet (included)}
\noindent\textbf{Worksheet file:} \texttt{\detokenize{worksheets/phase1/kolmogorov_fomin/ch01/KF_P1_C01_WS_001.tex}}\\
% Fragment (no \documentclass)

\subsubsection*{KF-P1-C01-WS-001 (10 problems)}\n\hypertarget{KF-P1-C01-WS-001}{}\n\n\noindent\textbf{Navigation: }
\hyperlink{P1-KF-C01}{Back to chapter}\n\n\begin{enumerate}
\item \textbf{Equality via mutual inclusion.} Prove that $A=B$ iff $A\subseteq B$ and $B\subseteq A$. (\hyperlink{KF-P1-C01-001}{proof})
\item \textbf{Distributive law.} Prove $(A\cap B)\cup C = (A\cup C)\cap (B\cup C)$. (\hyperlink{KF-P1-C01-002}{proof})
\item \textbf{De Morgan for families.} Prove De Morgan's laws for an arbitrary family of sets. (\hyperlink{KF-P1-C01-003}{proof})
\item \textbf{Preimage of union.} Let $f:M\to N$. Prove $f^{-1}(\bigcup_\alpha B_\alpha)=\bigcup_\alpha f^{-1}(B_\alpha)$. (\hyperlink{KF-P1-C01-004}{proof})
\item \textbf{Image of intersection subset.} Prove $f(A\cap B)\subseteq f(A)\cap f(B)$ and give an example where equality fails. (\hyperlink{KF-P1-C01-005}{proof})
\item \textbf{Injective: preimage of image.} If $f$ is injective, prove $f^{-1}(f(A))=A$. (\hyperlink{KF-P1-C01-006}{proof})
\item \textbf{Congruence mod n.} Prove that congruence modulo $n$ on $\mathbb{Z}$ is an equivalence relation. (\hyperlink{KF-P1-C01-007}{proof})
\item \textbf{Classes equal or disjoint.} Prove that equivalence classes are either equal or disjoint. (\hyperlink{KF-P1-C01-008}{proof})
\item \textbf{Countable union of countable sets.} Prove that a countable union of countable sets is countable. (\hyperlink{KF-P1-C01-009}{proof})
\item \textbf{Cantor--Bernstein theorem.} Prove the Cantor--Bernstein theorem. (\hyperlink{KF-P1-C01-010}{proof})
\end{enumerate}

\subsubsection*{Proofs (included)}
% Fragment (no \documentclass)

\hypertarget{KF-P1-C01-001}{}\n\subsubsection*{KF-P1-C01-001: Equality via mutual inclusion}\n\n\noindent\textbf{Navigation: }
\hyperlink{KF-P1-C01-WS-001}{Back to worksheet}\n\quad|\quad
\hyperlink{P1-KF-C01}{Back to chapter}\n\n\textbf{Claim.} Prove that $A=B$ iff $A\subseteq B$ and $B\subseteq A$.\n\n\textbf{Proof.}
Write your proof here.
\qed

% Fragment (no \documentclass)

\hypertarget{KF-P1-C01-002}{}\n\subsubsection*{KF-P1-C01-002: Distributive law}\n\n\noindent\textbf{Navigation: }
\hyperlink{KF-P1-C01-WS-001}{Back to worksheet}\n\quad|\quad
\hyperlink{P1-KF-C01}{Back to chapter}\n\n\textbf{Claim.} Prove $(A\cap B)\cup C = (A\cup C)\cap (B\cup C)$.\n\n\textbf{Proof.}
Write your proof here.
\qed

\input{proofs/phase1/kolmogorov_fomin/ch01/KF_P1_C01_003.tex}
% Fragment (no \documentclass)

\hypertarget{KF-P1-C01-004}{}\n\subsubsection*{KF-P1-C01-004: Preimage of union}\n\n\noindent\textbf{Navigation: }
\hyperlink{KF-P1-C01-WS-001}{Back to worksheet}\n\quad|\quad
\hyperlink{P1-KF-C01}{Back to chapter}\n\n\textbf{Claim.} Let $f:M\to N$. Prove $f^{-1}(\bigcup_\alpha B_\alpha)=\bigcup_\alpha f^{-1}(B_\alpha)$.\n\n\textbf{Proof.}
Write your proof here.
\qed

% Fragment (no \documentclass)

\hypertarget{KF-P1-C01-005}{}\n\subsubsection*{KF-P1-C01-005: Image of intersection subset}\n\n\noindent\textbf{Navigation: }
\hyperlink{KF-P1-C01-WS-001}{Back to worksheet}\n\quad|\quad
\hyperlink{P1-KF-C01}{Back to chapter}\n\n\textbf{Claim.} Prove $f(A\cap B)\subseteq f(A)\cap f(B)$ and give an example where equality fails.\n\n\textbf{Proof.}
Write your proof here.
\qed

% Fragment (no \documentclass)

\hypertarget{KF-P1-C01-006}{}\n\subsubsection*{KF-P1-C01-006: Injective: preimage of image}\n\n\noindent\textbf{Navigation: }
\hyperlink{KF-P1-C01-WS-001}{Back to worksheet}\n\quad|\quad
\hyperlink{P1-KF-C01}{Back to chapter}\n\n\textbf{Claim.} If $f$ is injective, prove $f^{-1}(f(A))=A$.\n\n\textbf{Proof.}
Write your proof here.
\qed

% Fragment (no \documentclass)

\hypertarget{KF-P1-C01-007}{}\n\subsubsection*{KF-P1-C01-007: Congruence mod n}\n\n\noindent\textbf{Navigation: }
\hyperlink{KF-P1-C01-WS-001}{Back to worksheet}\n\quad|\quad
\hyperlink{P1-KF-C01}{Back to chapter}\n\n\textbf{Claim.} Prove that congruence modulo $n$ on $\mathbb{Z}$ is an equivalence relation.\n\n\textbf{Proof.}
Write your proof here.
\qed

% Fragment (no \documentclass)

\hypertarget{KF-P1-C01-008}{}\n\subsubsection*{KF-P1-C01-008: Classes equal or disjoint}\n\n\noindent\textbf{Navigation: }
\hyperlink{KF-P1-C01-WS-001}{Back to worksheet}\n\quad|\quad
\hyperlink{P1-KF-C01}{Back to chapter}\n\n\textbf{Claim.} Prove that equivalence classes are either equal or disjoint.\n\n\textbf{Proof.}
Write your proof here.
\qed

% Fragment (no \documentclass)

\hypertarget{KF-P1-C01-009}{}\n\subsubsection*{KF-P1-C01-009: Countable union of countable sets}\n\n\noindent\textbf{Navigation: }
\hyperlink{KF-P1-C01-WS-001}{Back to worksheet}\n\quad|\quad
\hyperlink{P1-KF-C01}{Back to chapter}\n\n\textbf{Claim.} Prove that a countable union of countable sets is countable.\n\n\textbf{Proof.}
Write your proof here.
\qed

\input{proofs/phase1/kolmogorov_fomin/ch01/KF_P1_C01_010.tex}


\authorblock{Rinaldo B. Schinazi — \textit{From Classical to Modern Analysis}}
\chapterblock{Chapter 1}
\hypertarget{P1-SCH-C01}{}

\readingsection
\begin{itemize}
  \item (Fill in exact section/page ranges you assign for Chapter 1 in this text)
\end{itemize}

\notessection
\begin{itemize}
  \item (Your notes here)
\end{itemize}

\subsubsection*{Worksheet (included)}
\noindent\textbf{Worksheet file:} \texttt{\detokenize{worksheets/phase1/schinazi/ch01/SCH_P1_C01_WS_001.tex}}\\
% Fragment (no \documentclass)

\subsubsection*{SCH-P1-C01-WS-001 (10 problems)}\n\hypertarget{SCH-P1-C01-WS-001}{}\n\n\noindent\textbf{Navigation: }
\hyperlink{P1-SCH-C01}{Back to chapter}\n\n\begin{enumerate}
\item \textbf{Uniqueness of additive identity.} Using only field axioms, prove uniqueness of the additive identity. (\hyperlink{SCH-P1-C01-001}{proof})
\item \textbf{Uniqueness of additive inverses.} Using only field axioms, prove uniqueness of additive inverses. (\hyperlink{SCH-P1-C01-002}{proof})
\item \textbf{Zero times a.} Using only field axioms, prove that $0\cdot a = 0$ for all $a\in\mathbb{R}$. (\hyperlink{SCH-P1-C01-003}{proof})
\item \textbf{(-1)a = -a.} Using only field axioms, prove that $(-1)a=-a$ for all $a\in\mathbb{R}$. (\hyperlink{SCH-P1-C01-004}{proof})
\item \textbf{Order translation.} Assuming order axioms, prove that if $a<b$ then $a+c<b+c$ for any $c$. (\hyperlink{SCH-P1-C01-005}{proof})
\item \textbf{Order scaling.} Assuming order axioms, prove that if $a<b$ and $0<c$, then $ac<bc$. (\hyperlink{SCH-P1-C01-006}{proof})
\item \textbf{Inverse preserves positivity.} Prove that if $a>0$ then $a^{-1}>0$. (\hyperlink{SCH-P1-C01-007}{proof})
\item \textbf{Supremum uniqueness.} Prove that a bounded nonempty set has at most one supremum. (\hyperlink{SCH-P1-C01-008}{proof})
\item \textbf{Epsilon characterization.} Prove the $\varepsilon$-characterization of supremum. (\hyperlink{SCH-P1-C01-009}{proof})
\item \textbf{Archimedean property.} Prove the Archimedean property of $\mathbb{R}$. (\hyperlink{SCH-P1-C01-010}{proof})
\end{enumerate}

\subsubsection*{Proofs (included)}
% Fragment (no \documentclass)

\hypertarget{SCH-P1-C01-001}{}\n\subsubsection*{SCH-P1-C01-001: Uniqueness of additive identity}\n\n\noindent\textbf{Navigation: }
\hyperlink{SCH-P1-C01-WS-001}{Back to worksheet}\n\quad|\quad
\hyperlink{P1-SCH-C01}{Back to chapter}\n\n\textbf{Claim.} Using only field axioms, prove uniqueness of the additive identity.\n\n\textbf{Proof.}
Write your proof here.
\qed

% Fragment (no \documentclass)

\hypertarget{SCH-P1-C01-002}{}\n\subsubsection*{SCH-P1-C01-002: Uniqueness of additive inverses}\n\n\noindent\textbf{Navigation: }
\hyperlink{SCH-P1-C01-WS-001}{Back to worksheet}\n\quad|\quad
\hyperlink{P1-SCH-C01}{Back to chapter}\n\n\textbf{Claim.} Using only field axioms, prove uniqueness of additive inverses.\n\n\textbf{Proof.}
Write your proof here.
\qed

% Fragment (no \documentclass)

\hypertarget{SCH-P1-C01-003}{}\n\subsubsection*{SCH-P1-C01-003: Zero times a}\n\n\noindent\textbf{Navigation: }
\hyperlink{SCH-P1-C01-WS-001}{Back to worksheet}\n\quad|\quad
\hyperlink{P1-SCH-C01}{Back to chapter}\n\n\textbf{Claim.} Using only field axioms, prove that $0\cdot a = 0$ for all $a\in\mathbb{R}$.\n\n\textbf{Proof.}
Write your proof here.
\qed

% Fragment (no \documentclass)

\hypertarget{SCH-P1-C01-004}{}\n\subsubsection*{SCH-P1-C01-004: (-1)a = -a}\n\n\noindent\textbf{Navigation: }
\hyperlink{SCH-P1-C01-WS-001}{Back to worksheet}\n\quad|\quad
\hyperlink{P1-SCH-C01}{Back to chapter}\n\n\textbf{Claim.} Using only field axioms, prove that $(-1)a=-a$ for all $a\in\mathbb{R}$.\n\n\textbf{Proof.}
Write your proof here.
\qed

% Fragment (no \documentclass)

\hypertarget{SCH-P1-C01-005}{}\n\subsubsection*{SCH-P1-C01-005: Order translation}\n\n\noindent\textbf{Navigation: }
\hyperlink{SCH-P1-C01-WS-001}{Back to worksheet}\n\quad|\quad
\hyperlink{P1-SCH-C01}{Back to chapter}\n\n\textbf{Claim.} Assuming order axioms, prove that if $a<b$ then $a+c<b+c$ for any $c$.\n\n\textbf{Proof.}
Write your proof here.
\qed

% Fragment (no \documentclass)

\hypertarget{SCH-P1-C01-006}{}\n\subsubsection*{SCH-P1-C01-006: Order scaling}\n\n\noindent\textbf{Navigation: }
\hyperlink{SCH-P1-C01-WS-001}{Back to worksheet}\n\quad|\quad
\hyperlink{P1-SCH-C01}{Back to chapter}\n\n\textbf{Claim.} Assuming order axioms, prove that if $a<b$ and $0<c$, then $ac<bc$.\n\n\textbf{Proof.}
Write your proof here.
\qed

% Fragment (no \documentclass)

\hypertarget{SCH-P1-C01-007}{}\n\subsubsection*{SCH-P1-C01-007: Inverse preserves positivity}\n\n\noindent\textbf{Navigation: }
\hyperlink{SCH-P1-C01-WS-001}{Back to worksheet}\n\quad|\quad
\hyperlink{P1-SCH-C01}{Back to chapter}\n\n\textbf{Claim.} Prove that if $a>0$ then $a^{-1}>0$.\n\n\textbf{Proof.}
Write your proof here.
\qed

% Fragment (no \documentclass)

\hypertarget{SCH-P1-C01-008}{}\n\subsubsection*{SCH-P1-C01-008: Supremum uniqueness}\n\n\noindent\textbf{Navigation: }
\hyperlink{SCH-P1-C01-WS-001}{Back to worksheet}\n\quad|\quad
\hyperlink{P1-SCH-C01}{Back to chapter}\n\n\textbf{Claim.} Prove that a bounded nonempty set has at most one supremum.\n\n\textbf{Proof.}
Write your proof here.
\qed

% Fragment (no \documentclass)

\hypertarget{SCH-P1-C01-009}{}\n\subsubsection*{SCH-P1-C01-009: Epsilon characterization}\n\n\noindent\textbf{Navigation: }
\hyperlink{SCH-P1-C01-WS-001}{Back to worksheet}\n\quad|\quad
\hyperlink{P1-SCH-C01}{Back to chapter}\n\n\textbf{Claim.} Prove the $\varepsilon$-characterization of supremum.\n\n\textbf{Proof.}
Write your proof here.
\qed

\input{proofs/phase1/schinazi/ch01/SCH_P1_C01_010.tex}


\authorblock{Robert Magnus — \textit{Metric Spaces}}
\chapterblock{Chapter 1}
\hypertarget{P1-MAG-C01}{}

\readingsection
\begin{itemize}
  \item (Fill in exact section/page ranges you assign for Chapter 1 in this text)
\end{itemize}

\notessection
\begin{itemize}
  \item (Your notes here)
\end{itemize}

\subsubsection*{Worksheet (included)}
\noindent\textbf{Worksheet file:} \texttt{\detokenize{worksheets/phase1/magnus/ch01/MAG_P1_C01_WS_001.tex}}\\
% Fragment (no \documentclass)

\subsubsection*{MAG-P1-C01-WS-001 (10 problems)}\n\hypertarget{MAG-P1-C01-WS-001}{}\n\n\noindent\textbf{Navigation: }
\hyperlink{P1-MAG-C01}{Back to chapter}\n\n\begin{enumerate}
\item \textbf{Inclusion is partial order.} Prove that $\subseteq$ is a partial order on $\mathcal{P}(X)$ for any set $X$. (\hyperlink{MAG-P1-C01-001}{proof})
\item \textbf{Equality is equivalence relation.} Prove that equality is an equivalence relation. (\hyperlink{MAG-P1-C01-002}{proof})
\item \textbf{Power set poset.} Prove that $(\mathcal{P}(X),\subseteq)$ is a partially ordered set. (\hyperlink{MAG-P1-C01-003}{proof})
\item \textbf{Preimages preserve unions.} Let $f:X\to Y$. Prove that preimages preserve unions. (\hyperlink{MAG-P1-C01-004}{proof})
\item \textbf{Preimages preserve intersections.} Let $f:X\to Y$. Prove that preimages preserve intersections. (\hyperlink{MAG-P1-C01-005}{proof})
\item \textbf{Preimages preserve complements.} Let $f:X\to Y$. Prove that preimages preserve complements. (\hyperlink{MAG-P1-C01-006}{proof})
\item \textbf{Intersection of equivalence relations?.} Give an example showing the intersection of two equivalence relations need not be an equivalence relation, or prove it always is (whichever is true). (\hyperlink{MAG-P1-C01-007}{proof})
\item \textbf{Composition not symmetric.} Give an example showing that the composition of relations need not be symmetric. (\hyperlink{MAG-P1-C01-008}{proof})
\item \textbf{Uniqueness of limits.} Prove uniqueness of limits in a metric space. (\hyperlink{MAG-P1-C01-009}{proof})
\item \textbf{Convergent implies Cauchy.} Prove that every convergent sequence in a metric space is Cauchy. (\hyperlink{MAG-P1-C01-010}{proof})
\end{enumerate}

\subsubsection*{Proofs (included)}
% Fragment (no \documentclass)

\hypertarget{MAG-P1-C01-001}{}\n\subsubsection*{MAG-P1-C01-001: Inclusion is partial order}\n\n\noindent\textbf{Navigation: }
\hyperlink{MAG-P1-C01-WS-001}{Back to worksheet}\n\quad|\quad
\hyperlink{P1-MAG-C01}{Back to chapter}\n\n\textbf{Claim.} Prove that $\subseteq$ is a partial order on $\mathcal{P}(X)$ for any set $X$.\n\n\textbf{Proof.}
Write your proof here.
\qed

\input{proofs/phase1/magnus/ch01/MAG_P1_C01_002.tex}
% Fragment (no \documentclass)

\hypertarget{MAG-P1-C01-003}{}\n\subsubsection*{MAG-P1-C01-003: Power set poset}\n\n\noindent\textbf{Navigation: }
\hyperlink{MAG-P1-C01-WS-001}{Back to worksheet}\n\quad|\quad
\hyperlink{P1-MAG-C01}{Back to chapter}\n\n\textbf{Claim.} Prove that $(\mathcal{P}(X),\subseteq)$ is a partially ordered set.\n\n\textbf{Proof.}
Write your proof here.
\qed

\input{proofs/phase1/magnus/ch01/MAG_P1_C01_004.tex}
% Fragment (no \documentclass)

\hypertarget{MAG-P1-C01-005}{}\n\subsubsection*{MAG-P1-C01-005: Preimages preserve intersections}\n\n\noindent\textbf{Navigation: }
\hyperlink{MAG-P1-C01-WS-001}{Back to worksheet}\n\quad|\quad
\hyperlink{P1-MAG-C01}{Back to chapter}\n\n\textbf{Claim.} Let $f:X\to Y$. Prove that preimages preserve intersections.\n\n\textbf{Proof.}
Write your proof here.
\qed

% Fragment (no \documentclass)

\hypertarget{MAG-P1-C01-006}{}\n\subsubsection*{MAG-P1-C01-006: Preimages preserve complements}\n\n\noindent\textbf{Navigation: }
\hyperlink{MAG-P1-C01-WS-001}{Back to worksheet}\n\quad|\quad
\hyperlink{P1-MAG-C01}{Back to chapter}\n\n\textbf{Claim.} Let $f:X\to Y$. Prove that preimages preserve complements.\n\n\textbf{Proof.}
Write your proof here.
\qed

% Fragment (no \documentclass)

\hypertarget{MAG-P1-C01-007}{}\n\subsubsection*{MAG-P1-C01-007: Intersection of equivalence relations?}\n\n\noindent\textbf{Navigation: }
\hyperlink{MAG-P1-C01-WS-001}{Back to worksheet}\n\quad|\quad
\hyperlink{P1-MAG-C01}{Back to chapter}\n\n\textbf{Claim.} Give an example showing the intersection of two equivalence relations need not be an equivalence relation, or prove it always is (whichever is true).\n\n\textbf{Proof.}
Write your proof here.
\qed

% Fragment (no \documentclass)

\hypertarget{MAG-P1-C01-008}{}\n\subsubsection*{MAG-P1-C01-008: Composition not symmetric}\n\n\noindent\textbf{Navigation: }
\hyperlink{MAG-P1-C01-WS-001}{Back to worksheet}\n\quad|\quad
\hyperlink{P1-MAG-C01}{Back to chapter}\n\n\textbf{Claim.} Give an example showing that the composition of relations need not be symmetric.\n\n\textbf{Proof.}
Write your proof here.
\qed

\input{proofs/phase1/magnus/ch01/MAG_P1_C01_009.tex}
% Fragment (no \documentclass)

\hypertarget{MAG-P1-C01-010}{}\n\subsubsection*{MAG-P1-C01-010: Convergent implies Cauchy}\n\n\noindent\textbf{Navigation: }
\hyperlink{MAG-P1-C01-WS-001}{Back to worksheet}\n\quad|\quad
\hyperlink{P1-MAG-C01}{Back to chapter}\n\n\textbf{Claim.} Prove that every convergent sequence in a metric space is Cauchy.\n\n\textbf{Proof.}
Write your proof here.
\qed




\appendix
\chapter{Proof Grading Rubric}
\section*{Rubric (used for grading/feedback)}
\begin{enumerate}
\item \textbf{Correctness}: Every inference is valid; no hidden assumptions.
\item \textbf{Quantifiers \& scope}: ``arbitrary'' vs ``chosen'' elements are handled correctly.
\item \textbf{Definitions}: Key definitions are stated and used precisely.
\item \textbf{Structure}: Clear start, logical progression, explicit conclusion.
\item \textbf{Rigor}: No handwaving; lemmas invoked properly.
\item \textbf{Notation}: Consistent symbols; domains/codomains specified when needed.
\item \textbf{Minimality}: Proof is not bloated; each step has a purpose.
\end{enumerate}

\end{document}
