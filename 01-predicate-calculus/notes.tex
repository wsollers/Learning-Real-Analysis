% =========================================================
% Predicate Calculus — Reorganized Notes
% =========================================================


% ---------------------------------------------------------
\subsection{Syntax of First-Order Logic}
% ---------------------------------------------------------

\subsubsection{Terms}

\begin{definition}[Variable]
A \emph{variable} is a syntactic symbol that ranges over elements of a fixed
domain of discourse.

Variables serve as placeholders in formulas and do not refer to specific objects
until they are assigned values or bound by quantifiers.
\end{definition}

\begin{definition}[Term]
A \emph{term} is a syntactic expression intended to denote an object in the
domain of discourse.

The set of terms of a formal language is defined recursively as follows:
\begin{enumerate}
  \item \textbf{Variables.}
  Every variable is a term.

  \item \textbf{Constants.}
  Every constant symbol is a term.

  \item \textbf{Function application.}
  If $f$ is an $n$-ary function symbol and
  $t_1, \dots, t_n$ are terms, then
  \[
  f(t_1, \dots, t_n)
  \]
  is a term.

  \item \textbf{Closure.}
  No expression is a term unless it can be obtained by finitely many applications
  of rules (1)--(3).
\end{enumerate}
\end{definition}

\begin{remark}
Terms are purely syntactic objects. Under an interpretation, each term denotes
an element of the domain of discourse, but the term itself is not an object of
the domain.
\end{remark}

\subsubsection{Formulas}

\begin{definition}[Atomic Formula]
An \emph{atomic formula} is a well-formed formula obtained by applying an
$n$-ary predicate symbol to $n$ terms.

If $P$ is an $n$-ary predicate symbol and $t_1, \dots, t_n$ are terms, then
\[
P(t_1, \dots, t_n)
\]
is an atomic formula.

Atomic formulas contain no logical connectives or quantifiers and serve as the
base case for the recursive definition of well-formed formulas.
\end{definition}

\begin{definition}[Well-Formed Formula]
The set of \emph{well-formed formulas} (wffs) of a first-order language is
defined recursively as follows:
\begin{enumerate}
  \item \textbf{Atomic formulas.}
  Every atomic formula is a well-formed formula.

  \item \textbf{Negation.}
  If $\varphi$ is a formula, then $\neg\varphi$ is a formula.

  \item \textbf{Binary connectives.}
  If $\varphi$ and $\psi$ are formulas and
  $\circ \in \{\wedge, \vee, \rightarrow, \leftrightarrow\}$, then
  $(\varphi \circ \psi)$ is a formula.

  \item \textbf{Quantification.}
  If $\varphi$ is a formula and $x$ is a variable, then
  $\forall x\,\varphi$ and $\exists x\,\varphi$ are formulas.

  \item \textbf{Closure.}
  No expression is a formula unless it can be obtained by finitely many
  applications of rules (1)--(4).
\end{enumerate}
\end{definition}

\begin{definition}[Molecular Formula]
A \emph{molecular formula} is a well-formed formula that is not atomic.

Equivalently, a formula is molecular if it is formed from one or more atomic
formulas by the application of logical connectives or quantifiers.
\end{definition}

\begin{remark}
Atomic formulas express basic properties or relations, while molecular formulas
express compound statements built from atomic formulas using the logical
apparatus of the language.
\end{remark}

\subsubsection{Free and Bound Variables}

\begin{definition}[Scope of a Quantifier]
Let $\varphi$ be a formula of a first-order language.

If $\varphi$ is of the form $\forall x\,\psi$ or $\exists x\,\psi$, then the
formula $\psi$ is called the \emph{scope} of the quantifier.

The quantifier is said to \emph{bind} all occurrences of the variable $x$ that
appear within its scope.
\end{definition}

\begin{definition}[Bound and Free Occurrences]
An occurrence of a variable $x$ in a formula $\varphi$ is \emph{bound} if it lies
within the scope of a quantifier $\forall x$ or $\exists x$.

An occurrence of $x$ is \emph{free} if it is not bound by any quantifier in
$\varphi$.
\end{definition}

\begin{definition}[Free Variable]
A variable $x$ is said to be \emph{free} in a formula $\varphi$ if there exists
an occurrence of $x$ in $\varphi$ that is not within the scope of a quantifier
binding $x$.

Equivalently, $x$ is free in $\varphi$ if the truth value of $\varphi$ under a
structure depends on the value assigned to $x$ by a variable assignment.
\end{definition}

\begin{definition}[Free Variables of a Formula]
The set $\mathrm{FV}(\varphi)$ of free variables of a formula $\varphi$ is defined
recursively as follows:
\begin{enumerate}
  \item If $\varphi$ is an atomic formula $P(t_1,\dots,t_n)$, then
  \[
  \mathrm{FV}(\varphi) = \bigcup_{i=1}^n \mathrm{Var}(t_i),
  \]
  where $\mathrm{Var}(t_i)$ denotes the set of variables occurring in the term
  $t_i$.

  \item $\mathrm{FV}(\neg \varphi) = \mathrm{FV}(\varphi)$.

  \item $\mathrm{FV}(\varphi \circ \psi)
  = \mathrm{FV}(\varphi) \cup \mathrm{FV}(\psi)$,
  where $\circ$ is any binary connective.

  \item $\mathrm{FV}(\forall x\,\varphi)
  = \mathrm{FV}(\varphi) \setminus \{x\}$.

  \item $\mathrm{FV}(\exists x\,\varphi)
  = \mathrm{FV}(\varphi) \setminus \{x\}$.
\end{enumerate}
\end{definition}

\begin{definition}[Sentence]
A \emph{sentence} (or \emph{closed formula}) is a formula with no free variables.

Formally, $\varphi$ is a sentence if and only if $\mathrm{FV}(\varphi) = \varnothing$.
\end{definition}

\begin{remark}
Terms themselves contain only free variables.
Variables become bound only through quantification in formulas.
The interpretation of a formula depends exactly on the values assigned to its
free variables.
\end{remark}

\begin{example}
In the formula $\forall x\,(P(x) \rightarrow Q(x))$, the scope of $\forall x$ is
the formula $(P(x) \rightarrow Q(x))$.

In the formula $(\forall x\,P(x)) \wedge Q(x)$, the scope of $\forall x$ is $P(x)$
only; the occurrence of $x$ in $Q(x)$ is free.
\end{example}

\subsubsection{Substitution}

\begin{definition}[Substitution Notation]
Let $\varphi$ be a formula, $x$ a variable, and $t$ a term.

The expression $\varphi[t/x]$ denotes the formula obtained from $\varphi$ by
replacing every \emph{free} occurrence of $x$ with the term $t$, leaving bound
occurrences of $x$ unchanged.
\end{definition}

\begin{definition}[Free for Substitution]
A term $t$ is \emph{free for} $x$ in $\varphi$ if no free occurrence of $x$ in
$\varphi$ lies within the scope of a quantifier $\forall y$ or $\exists y$ where
$y$ is a variable occurring in $t$.

Equivalently, $t$ is free for $x$ in $\varphi$ if the substitution $\varphi[t/x]$
does not result in any variable in $t$ becoming bound.
\end{definition}

\begin{remark}[Capture-Avoiding Substitution]
A substitution $\varphi[t/x]$ is admissible only if $t$ is free for $x$ in
$\varphi$. If this condition is violated, a variable occurring in $t$ may become
bound after substitution, changing the meaning of the formula.
\end{remark}

\begin{definition}[Alpha-Equivalence]
Formulas that differ only by the names of bound variables are logically
equivalent.
\[
\forall x\,\varphi \;\equiv\; \forall y\,\varphi[y/x]
\quad\text{(provided $y$ is not free in $\varphi$)}
\]
This equivalence is called \emph{alpha-equivalence} or \emph{alphabetic variance}.
\end{definition}

\subsubsection{Formula Complexity}

\begin{definition}[Formula Depth]
The \emph{depth} (or \emph{complexity}) of a formula $\varphi$, denoted
$\mathrm{depth}(\varphi)$, is a natural number defined recursively as follows:
\begin{enumerate}
  \item \textbf{Atomic case.}
  If $\varphi$ is an atomic formula, then
  $\mathrm{depth}(\varphi) = 0$.

  \item \textbf{Negation.}
  If $\varphi$ is of the form $\neg \psi$, then
  $\mathrm{depth}(\varphi) = \mathrm{depth}(\psi) + 1$.

  \item \textbf{Binary connectives.}
  If $\varphi$ is of the form $(\psi \circ \chi)$, where
  $\circ \in \{\wedge,\vee,\rightarrow,\leftrightarrow\}$, then
  \[
  \mathrm{depth}(\varphi)
  =
  \max\{\mathrm{depth}(\psi),\mathrm{depth}(\chi)\} + 1.
  \]

  \item \textbf{Quantifiers.}
  If $\varphi$ is of the form $\forall x\,\psi$ or $\exists x\,\psi$, then
  $\mathrm{depth}(\varphi) = \mathrm{depth}(\psi) + 1$.
\end{enumerate}
\end{definition}

\begin{remark}
The depth of a formula measures the maximum number of logical formation steps
required to build the formula from atomic formulas. It corresponds to the height
of the formula's syntactic parse tree.
\end{remark}

% ---------------------------------------------------------
\subsection{Semantics of First-Order Logic}
% ---------------------------------------------------------

\subsubsection{Structures and Interpretations}

\begin{definition}[First-Order Language]
A \emph{first-order language} $\mathcal{L}$ consists of:
\begin{itemize}
  \item a set of constant symbols,
  \item a set of function symbols, each with a specified arity, and
  \item a set of predicate symbols, each with a specified arity.
\end{itemize}
In addition, every first-order language includes a countable set of variables and
the logical symbols $\neg, \wedge, \vee, \rightarrow, \leftrightarrow, \forall,
\exists$, and (optionally) $=$.
\end{definition}

\begin{definition}[Structure]
Let $\mathcal{L}$ be a first-order language.
A \emph{structure} (or \emph{interpretation}) for $\mathcal{L}$ is an ordered pair
\[
\mathcal{M} = \langle D, I \rangle,
\]
where:
\begin{enumerate}
  \item $D$ is a nonempty set called the \emph{domain of discourse} (or
  \emph{universe}), and
  \item $I$ is an interpretation function that assigns:
  \begin{itemize}
    \item to each constant symbol $c$ of $\mathcal{L}$ an element
          $I(c) \in D$,
    \item to each $n$-ary function symbol $f$ of $\mathcal{L}$ a function
          $I(f) : D^n \to D$,
    \item to each $n$-ary predicate symbol $P$ of $\mathcal{L}$ a relation
          $I(P) \subseteq D^n$.
  \end{itemize}
\end{enumerate}
\end{definition}

\begin{remark}[Nonempty Domain Convention]
Throughout these notes, we assume that the domain of discourse is nonempty. This
is the standard convention in classical first-order logic.

If empty domains were permitted, some standard equivalences would fail. In
particular, the inference $\forall x\,\varphi \Rightarrow \exists x\,\varphi$
(subalternation) would be invalid, since the universal statement is vacuously
true in an empty domain while the existential statement is false.
\end{remark}

\begin{definition}[Variable Assignment]
Let $\mathcal{M} = \langle D, I \rangle$ be a structure for a language
$\mathcal{L}$.
A \emph{variable assignment} is a function
\[
s : \mathsf{Var} \to D
\]
that assigns to each variable an element of the domain.
\end{definition}

\begin{definition}[Modified Assignment]
Let $s$ be a variable assignment, $x$ a variable, and $d \in D$.
The \emph{modified assignment} $s[x \mapsto d]$ is defined by
\[
s[x \mapsto d](y) =
\begin{cases}
d & \text{if } y = x, \\
s(y) & \text{if } y \neq x.
\end{cases}
\]
\end{definition}

\begin{remark}
A structure assigns meaning only to the non-logical symbols of the language.
The interpretation of variables is provided separately by a variable
assignment. Together, a structure and a variable assignment determine the
truth or falsity of formulas.
\end{remark}

\subsubsection{Interpretation of Terms}

\begin{definition}[Interpretation of a Term]
Let $\mathcal{M} = \langle D, I \rangle$ be a structure for a language
$\mathcal{L}$, and let $s : \mathsf{Var} \to D$ be a variable assignment.

The \emph{interpretation} (or \emph{value}) of a term $t$ in $\mathcal{M}$ under
$s$, denoted $\llbracket t \rrbracket_{\mathcal{M},s}$, is defined recursively
as follows:
\begin{enumerate}
  \item If $t$ is a variable $x$, then
  \[
  \llbracket x \rrbracket_{\mathcal{M},s} = s(x).
  \]

  \item If $t$ is a constant symbol $c$, then
  \[
  \llbracket c \rrbracket_{\mathcal{M},s} = I(c).
  \]

  \item If $t$ is of the form $f(t_1,\dots,t_n)$, where $f$ is an $n$-ary function
  symbol, then
  \[
  \llbracket f(t_1,\dots,t_n) \rrbracket_{\mathcal{M},s}
  =
  I(f)\bigl(\llbracket t_1 \rrbracket_{\mathcal{M},s},\dots,
            \llbracket t_n \rrbracket_{\mathcal{M},s}\bigr).
  \]
\end{enumerate}
\end{definition}

\subsubsection{Satisfaction and Truth}

\begin{definition}[Satisfaction]
Let $\mathcal{M} = \langle D, I \rangle$ be a structure, $s$ a variable
assignment, and $\varphi$ a formula.

The relation $\mathcal{M}, s \models \varphi$ (read ``$\mathcal{M}$ satisfies
$\varphi$ under $s$'' or ``$\varphi$ is true in $\mathcal{M}$ under $s$'') is
defined recursively as follows:

\begin{enumerate}
  \item \textbf{Atomic formulas.}
  \[
  \mathcal{M}, s \models P(t_1, \dots, t_n)
  \quad\Longleftrightarrow\quad
  (\llbracket t_1 \rrbracket_{\mathcal{M},s}, \dots,
   \llbracket t_n \rrbracket_{\mathcal{M},s}) \in I(P).
  \]

  \item \textbf{Equality} (if $=$ is in the language).
  \[
  \mathcal{M}, s \models (t_1 = t_2)
  \quad\Longleftrightarrow\quad
  \llbracket t_1 \rrbracket_{\mathcal{M},s} =
  \llbracket t_2 \rrbracket_{\mathcal{M},s}.
  \]

  \item \textbf{Negation.}
  \[
  \mathcal{M}, s \models \neg\varphi
  \quad\Longleftrightarrow\quad
  \mathcal{M}, s \not\models \varphi.
  \]

  \item \textbf{Conjunction.}
  \[
  \mathcal{M}, s \models (\varphi \wedge \psi)
  \quad\Longleftrightarrow\quad
  \mathcal{M}, s \models \varphi \text{ and } \mathcal{M}, s \models \psi.
  \]

  \item \textbf{Disjunction.}
  \[
  \mathcal{M}, s \models (\varphi \vee \psi)
  \quad\Longleftrightarrow\quad
  \mathcal{M}, s \models \varphi \text{ or } \mathcal{M}, s \models \psi.
  \]

  \item \textbf{Implication.}
  \[
  \mathcal{M}, s \models (\varphi \rightarrow \psi)
  \quad\Longleftrightarrow\quad
  \mathcal{M}, s \not\models \varphi \text{ or } \mathcal{M}, s \models \psi.
  \]

  \item \textbf{Biconditional.}
  \[
  \mathcal{M}, s \models (\varphi \leftrightarrow \psi)
  \quad\Longleftrightarrow\quad
  (\mathcal{M}, s \models \varphi \Leftrightarrow \mathcal{M}, s \models \psi).
  \]

  \item \textbf{Universal quantification.}
  \[
  \mathcal{M}, s \models \forall x\,\varphi
  \quad\Longleftrightarrow\quad
  \text{for all } d \in D,\; \mathcal{M}, s[x \mapsto d] \models \varphi.
  \]

  \item \textbf{Existential quantification.}
  \[
  \mathcal{M}, s \models \exists x\,\varphi
  \quad\Longleftrightarrow\quad
  \text{there exists } d \in D \text{ such that }
  \mathcal{M}, s[x \mapsto d] \models \varphi.
  \]
\end{enumerate}
\end{definition}

\begin{definition}[Truth in a Structure]
A sentence $\varphi$ is \emph{true} in a structure $\mathcal{M}$, written
$\mathcal{M} \models \varphi$, if $\mathcal{M}, s \models \varphi$ for every
(equivalently, for some) variable assignment $s$.

A sentence $\varphi$ is \emph{false} in $\mathcal{M}$ if $\mathcal{M} \not\models \varphi$.
\end{definition}

\begin{definition}[Validity and Satisfiability]
A formula $\varphi$ is:
\begin{itemize}
  \item \emph{valid} (or a \emph{logical truth}) if $\mathcal{M}, s \models \varphi$
  for every structure $\mathcal{M}$ and every assignment $s$;
  \item \emph{satisfiable} if $\mathcal{M}, s \models \varphi$ for some structure
  $\mathcal{M}$ and some assignment $s$;
  \item \emph{unsatisfiable} (or a \emph{contradiction}) if it is not satisfiable.
\end{itemize}
\end{definition}

\begin{remark}
For sentences (closed formulas), truth depends only on the structure, not on the
variable assignment, since there are no free variables whose values could affect
the truth value.
\end{remark}

\subsubsection{Models and Theories}

\begin{definition}[Model]
Let $\varphi$ be a sentence and $\mathcal{M}$ a structure.
We say that $\mathcal{M}$ is a \emph{model} of $\varphi$ if $\mathcal{M} \models \varphi$.

More generally, if $\Gamma$ is a set of sentences, then $\mathcal{M}$ is a
\emph{model} of $\Gamma$ if $\mathcal{M} \models \gamma$ for every $\gamma \in \Gamma$.
In this case, we write $\mathcal{M} \models \Gamma$.
\end{definition}

\begin{definition}[Theory]
A \emph{theory} is a set of sentences.

A theory $T$ is \emph{satisfiable} (or \emph{consistent}) if it has at least one
model.

A theory $T$ is \emph{complete} if for every sentence $\varphi$ in the language,
either $T \models \varphi$ or $T \models \neg\varphi$.
\end{definition}

\begin{remark}
In some contexts, a theory is required to be closed under logical consequence:
if $T \models \varphi$, then $\varphi \in T$. In other contexts, a theory is
simply any set of sentences (the axioms), and consequences are derived from it.
\end{remark}

\begin{definition}[Logical Consequence]
Let $\Gamma$ be a set of formulas and $\varphi$ a formula.
We say that $\varphi$ is a \emph{logical consequence} of $\Gamma$, written
\[
\Gamma \models \varphi,
\]
if for every structure $\mathcal{M}$ and every variable assignment $s$:
\[
\text{if } \mathcal{M}, s \models \gamma \text{ for all } \gamma \in \Gamma,
\text{ then } \mathcal{M}, s \models \varphi.
\]

Equivalently, every model of $\Gamma$ is a model of $\varphi$.
\end{definition}

\begin{definition}[Logical Equivalence]
Two formulas $\varphi$ and $\psi$ are \emph{logically equivalent}, written
$\varphi \equiv \psi$, if each is a logical consequence of the other:
\[
\varphi \equiv \psi
\quad\Longleftrightarrow\quad
(\varphi \models \psi \text{ and } \psi \models \varphi).
\]

Equivalently, $\varphi$ and $\psi$ have the same truth value in every structure
under every variable assignment.
\end{definition}

\begin{remark}
The notation $\models \varphi$ (with empty left-hand side) means that $\varphi$
is valid: it is true in all structures under all assignments.
\end{remark}

\subsubsection{Predicates and Relations}

\begin{definition}[Predicate]
A \emph{predicate} is an expression containing one or more variables that
represents a property or relation and becomes a proposition when all of its
variables are instantiated.

Formally, a predicate may be viewed in two equivalent ways:
\begin{enumerate}
  \item \textbf{Syntactic view.}
  A predicate is an open formula $\varphi(x_1,\dots,x_n)$ of a formal language,
  which does not have a truth value until specific objects are substituted for
  its variables.

  \item \textbf{Semantic view.}
  Given a domain of discourse $D$, an $n$-ary predicate determines a function
  \[
  P : D^n \to \{\mathsf{T}, \mathsf{F}\},
  \]
  assigning a truth value to each ordered $n$-tuple of elements of $D$.
\end{enumerate}

Under an interpretation, these two views coincide: for any
$a_1,\dots,a_n \in D$, the instantiated formula
$\varphi(a_1,\dots,a_n)$ is true if and only if
$P(a_1,\dots,a_n) = \mathsf{T}$.
\end{definition}

\begin{remark}
A predicate is not itself a proposition, since it does not have a truth value
until its variables are instantiated. Once all variables are replaced by
elements of the domain, the resulting expression is a proposition.
\end{remark}

\begin{remark}[Predicates vs.\ Relations]
Predicates and relations are closely related but conceptually distinct.

A \emph{predicate} is a syntactic or semantic device: syntactically, it is an
open formula or predicate symbol; semantically, it determines a truth-valued
function on a domain.

A \emph{relation}, by contrast, is a purely set-theoretic object. An $n$-ary
relation on a domain $D$ is a subset $R \subseteq D^n$.

Under an interpretation, a predicate symbol corresponds to a relation by
identifying the tuples for which the predicate is true:
\[
P(a_1,\dots,a_n) \text{ is true } \iff (a_1,\dots,a_n) \in R.
\]

Thus, predicates belong to the \emph{language} of logic, while relations belong
to the \emph{structures} used to interpret that language.
\end{remark}

\subsubsection{Substitution Lemmas}

\begin{lemma}[Substitution Lemma for Terms]
Let $\mathcal{M}$ be a structure and $s$ a variable assignment.
For any term $t$, variable $x$, and $d \in D$,
\[
\llbracket t \rrbracket_{\mathcal{M},\,s[x\mapsto d]}
=
\llbracket t[d/x] \rrbracket_{\mathcal{M},\,s}.
\]
\end{lemma}

\begin{remark}
Evaluating a term after modifying an assignment is equivalent to substituting
the value directly into the term.
\end{remark}

\begin{lemma}[Substitution Lemma for Formulas]
Let $\varphi$ be a formula, $t$ a term free for $x$ in $\varphi$, and $x$ a variable.
Then for any structure $\mathcal{M}$ and assignment $s$,
\[
\mathcal{M},s \models \varphi[t/x]
\quad\Longleftrightarrow\quad
\mathcal{M},\,s[x\mapsto \llbracket t \rrbracket_{\mathcal{M},s}] \models \varphi.
\]
\end{lemma}

\begin{remark}
This lemma formally connects syntactic substitution with semantic evaluation and
is essential for soundness proofs.
\end{remark}

% ---------------------------------------------------------
\subsection{Quantifiers}
% ---------------------------------------------------------

\subsubsection{Universal and Existential Quantifiers}

\begin{definition}[Universal Quantifier]
The \emph{universal quantifier}, denoted by $\forall$, is a logical operator that
binds a variable and asserts that a formula holds for all elements of the domain
of discourse.

If $\varphi$ is a formula and $x$ is a variable, then
\[
\forall x\,\varphi
\]
is a formula, read as ``for all $x$, $\varphi$.''
\end{definition}

\begin{definition}[Existential Quantifier]
The \emph{existential quantifier}, denoted by $\exists$, is a logical operator that
binds a variable and asserts that a formula holds for at least one element of the
domain of discourse.

If $\varphi$ is a formula and $x$ is a variable, then
\[
\exists x\,\varphi
\]
is a formula, read as ``there exists an $x$ such that $\varphi$.''
\end{definition}

\begin{remark}
Quantifiers do not assert the existence or universality of symbols, but of
elements in the domain of discourse under an interpretation. Their semantic
content is defined via satisfaction in a structure.
\end{remark}

\begin{remark}
The universal quantifier expresses a global condition over the domain, while the
existential quantifier expresses a local condition. Both quantifiers bind
variables and thereby control the scope of variable dependence in formulas.
\end{remark}

\subsubsection{Unique Existential Quantifier}

\begin{definition}[Unique Existential Quantifier]
The \emph{unique existential quantifier}, denoted by $\exists!$, asserts that
there exists exactly one element satisfying a given property.

If $\varphi$ is a formula and $x$ is a variable, then
\[
\exists! x\,\varphi
\]
is an abbreviation for
\[
\exists x\,\bigl(\varphi \wedge \forall y\,(\varphi[y/x] \rightarrow y = x)\bigr),
\]
where $y$ is a variable not occurring in $\varphi$.

Equivalently,
\[
\exists! x\,\varphi
\;\equiv\;
\exists x\,\varphi \;\wedge\; \forall x\,\forall y\,
\bigl((\varphi \wedge \varphi[y/x]) \rightarrow x = y\bigr).
\]
\end{definition}

\begin{remark}
The unique existential quantifier combines existence and uniqueness:
$\exists! x\,\varphi$ asserts that there is at least one $x$ satisfying $\varphi$
(existence) and at most one such $x$ (uniqueness).
\end{remark}

\begin{example}
The statement ``there is exactly one even prime number'' can be formalized as
\[
\exists! x\,(P(x) \wedge E(x)),
\]
where $P(x)$ means ``$x$ is prime'' and $E(x)$ means ``$x$ is even.''
\end{example}

\subsubsection{Bounded Quantifiers}

\begin{definition}[Bounded Quantifiers]
Let $A$ be a set (or a predicate defining a set).
The \emph{bounded universal quantifier} and \emph{bounded existential quantifier}
are abbreviations defined as follows:
\begin{align*}
\forall x \in A\,\varphi &\;:=\; \forall x\,(x \in A \rightarrow \varphi), \\
\exists x \in A\,\varphi &\;:=\; \exists x\,(x \in A \wedge \varphi).
\end{align*}
\end{definition}

\begin{remark}
Bounded quantifiers restrict the domain of quantification to a specified set.
They are common in mathematical practice and often improve readability.

Note the asymmetry: the bounded universal uses implication ($\rightarrow$),
while the bounded existential uses conjunction ($\wedge$). This ensures the
correct logical meaning when the restriction is empty.
\end{remark}

\begin{example}
The statement ``every natural number has a successor'' can be written as
\[
\forall n \in \mathbb{N}\,\exists m \in \mathbb{N}\,(m = n + 1).
\]
\end{example}

\subsubsection{Quantifier Negation}

\begin{theorem}[Quantifier Negation Laws]
Let $\varphi$ be any formula. The negation of quantified formulas is governed by
the following equivalences:
\begin{align*}
\neg \forall x\,\varphi &\;\equiv\; \exists x\,\neg \varphi, \\
\neg \exists x\,\varphi &\;\equiv\; \forall x\,\neg \varphi.
\end{align*}
These equivalences preserve logical meaning while reversing the quantifier.
\end{theorem}

\begin{remark}[Procedure for Negating Quantifiers]
To negate a quantified statement:
\begin{enumerate}
  \item Move the negation symbol inward past the quantifier.
  \item Replace $\forall$ with $\exists$, or $\exists$ with $\forall$.
  \item Negate the formula within the scope of the quantifier.
\end{enumerate}
\end{remark}

\begin{example}
\[
\neg(\forall x\,P(x)) \;\equiv\; \exists x\,\neg P(x).
\]
\[
\neg(\exists x\,P(x)) \;\equiv\; \forall x\,\neg P(x).
\]
\end{example}

\begin{example}
\[
\neg\bigl(\forall x\,\exists y\,R(x,y)\bigr)
\;\equiv\;
\exists x\,\forall y\,\neg R(x,y).
\]
\end{example}

\begin{remark}
It is incorrect to negate a quantified statement by negating the predicate
without changing the quantifier. For example,
\[
\neg(\forall x\,P(x)) \not\equiv \forall x\,\neg P(x).
\]
\end{remark}

\begin{remark}[Negating Statements with Multiple Quantifiers]
Negating a statement with multiple quantifiers requires systematically pushing
the negation inward while reversing each quantifier encountered.

For example, consider the formula
\[
\forall x\,\exists y\,P(x,y).
\]
Its negation is computed as follows:
\[
\begin{aligned}
\neg\bigl(\forall x\,\exists y\,P(x,y)\bigr)
&\equiv \exists x\,\neg\bigl(\exists y\,P(x,y)\bigr) \\
&\equiv \exists x\,\forall y\,\neg P(x,y).
\end{aligned}
\]
\end{remark}

\begin{remark}[Negating a Statement with Three Quantifiers]
Consider the formula
\[
\forall x\,\exists y\,\forall z\,R(x,y,z).
\]
Negating this statement proceeds by successively pushing the negation inward:
\[
\begin{aligned}
\neg\bigl(\forall x\,\exists y\,\forall z\,R(x,y,z)\bigr)
&\equiv \exists x\,\neg\bigl(\exists y\,\forall z\,R(x,y,z)\bigr) \\
&\equiv \exists x\,\forall y\,\neg\bigl(\forall z\,R(x,y,z)\bigr) \\
&\equiv \exists x\,\forall y\,\exists z\,\neg R(x,y,z).
\end{aligned}
\]
\end{remark}

\begin{remark}[Negating Quantified Implications]
Consider the quantified formula
\[
\forall x\,\exists y\,\bigl((A(x) \rightarrow B(y)) \wedge (D(x,y) \rightarrow F(y))\bigr).
\]

We negate this statement by systematically pushing the negation inward.

\textbf{Step 1: Negate the outer quantifiers.}
\[
\exists x\,\forall y\,\neg\bigl((A(x) \rightarrow B(y)) \wedge (D(x,y) \rightarrow F(y))\bigr).
\]

\textbf{Step 2: Apply De Morgan's law to the conjunction.}
\[
\exists x\,\forall y\,
\bigl(
\neg(A(x) \rightarrow B(y))
\;\vee\;
\neg(D(x,y) \rightarrow F(y))
\bigr).
\]

\textbf{Step 3: Negate the implications.}
Using $\neg(P \rightarrow Q) \equiv P \wedge \neg Q$:
\[
\exists x\,\forall y\,
\bigl(
(A(x) \wedge \neg B(y))
\;\vee\;
(D(x,y) \wedge \neg F(y))
\bigr).
\]
\end{remark}

\begin{remark}[General Schema for Negating Quantified Implications]
Let $\varphi$ and $\psi$ be formulas.
Consider a quantified implication of the form
\[
Q_1 x_1\, Q_2 x_2 \cdots Q_n x_n \bigl(\varphi \rightarrow \psi\bigr),
\]
where each $Q_i \in \{\forall,\exists\}$.

The negation is:
\[
\neg\Bigl(Q_1 x_1 \cdots Q_n x_n (\varphi \rightarrow \psi)\Bigr)
\;\equiv\;
Q_1' x_1 \cdots Q_n' x_n (\varphi \wedge \neg \psi),
\]
where each $Q_i'$ is the dual quantifier:
$\forall' = \exists$ and $\exists' = \forall$.
\end{remark}

\begin{theorem}[Negation of Bounded Quantifiers]
Let $A$ be a set and $\varphi$ a formula. Then:
\begin{align*}
\neg(\forall x \in A\,\varphi) &\;\equiv\; \exists x \in A\,\neg\varphi, \\
\neg(\exists x \in A\,\varphi) &\;\equiv\; \forall x \in A\,\neg\varphi.
\end{align*}
\end{theorem}

\begin{proof}
We derive each equivalence by expanding the bounded quantifier and applying 
standard logical laws.

\textbf{Universal case.}
\begin{align*}
\neg(\forall x \in A\,\varphi) 
&\equiv \neg\forall x\,(x \in A \rightarrow \varphi) 
    && \text{(definition of bounded $\forall$)} \\
&\equiv \exists x\,\neg(x \in A \rightarrow \varphi) 
    && \text{(quantifier negation)} \\
&\equiv \exists x\,(x \in A \wedge \neg\varphi) 
    && \text{(since $\neg(P \rightarrow Q) \equiv P \wedge \neg Q$)} \\
&\equiv \exists x \in A\,\neg\varphi 
    && \text{(definition of bounded $\exists$)}
\end{align*}

\textbf{Existential case.}
\begin{align*}
\neg(\exists x \in A\,\varphi) 
&\equiv \neg\exists x\,(x \in A \wedge \varphi) 
    && \text{(definition of bounded $\exists$)} \\
&\equiv \forall x\,\neg(x \in A \wedge \varphi) 
    && \text{(quantifier negation)} \\
&\equiv \forall x\,(x \in A \rightarrow \neg\varphi) 
    && \text{(since $\neg(P \wedge Q) \equiv P \rightarrow \neg Q$)} \\
&\equiv \forall x \in A\,\neg\varphi 
    && \text{(definition of bounded $\forall$)}
\end{align*}
\end{proof}

\begin{remark}
The domain restriction $x \in A$ is preserved under negation because it 
functions as a constraint on \emph{which} elements are quantified over, 
not as part of the claim being negated. The negation passes through the 
quantifier and affects only the inner formula $\varphi$.
\end{remark}


\subsubsection{Quantifier Commutation and Distribution}

\begin{theorem}[Quantifier Commutation]
Quantifiers of the same type commute:
\begin{align*}
\forall x\,\forall y\,\varphi &\;\equiv\; \forall y\,\forall x\,\varphi, \\
\exists x\,\exists y\,\varphi &\;\equiv\; \exists y\,\exists x\,\varphi.
\end{align*}
However, quantifiers of different types do not commute in general:
\[
\forall x\,\exists y\,\varphi \;\not\equiv\; \exists y\,\forall x\,\varphi.
\]
\end{theorem}

\begin{theorem}[Quantifier Distribution]
Let $x$ not be free in $\psi$. Then:
\begin{align*}
\forall x\,(\varphi \wedge \psi) &\;\equiv\; (\forall x\,\varphi) \wedge \psi, \\
\forall x\,(\psi \wedge \varphi) &\;\equiv\; \psi \wedge (\forall x\,\varphi), \\
\exists x\,(\varphi \vee \psi) &\;\equiv\; (\exists x\,\varphi) \vee \psi, \\
\exists x\,(\psi \vee \varphi) &\;\equiv\; \psi \vee (\exists x\,\varphi).
\end{align*}
\end{theorem}

\begin{theorem}[Distribution over Conjunction and Disjunction]
\begin{align*}
\forall x\,(\varphi \wedge \psi) &\;\equiv\; (\forall x\,\varphi) \wedge (\forall x\,\psi), \\
\exists x\,(\varphi \vee \psi) &\;\equiv\; (\exists x\,\varphi) \vee (\exists x\,\psi).
\end{align*}
However:
\begin{align*}
\forall x\,(\varphi \vee \psi) &\;\not\equiv\; (\forall x\,\varphi) \vee (\forall x\,\psi), \\
\exists x\,(\varphi \wedge \psi) &\;\not\equiv\; (\exists x\,\varphi) \wedge (\exists x\,\psi).
\end{align*}
\end{theorem}

\subsubsection{Vacuous Quantification}

\begin{theorem}[Vacuous Quantification]
If $x$ does not occur free in $\varphi$, then:
\begin{align*}
\forall x\,\varphi &\;\equiv\; \varphi, \\
\exists x\,\varphi &\;\equiv\; \varphi.
\end{align*}
\end{theorem}

\begin{remark}
A quantifier is \emph{vacuous} if it binds a variable that does not occur free
in its scope. Vacuous quantifiers may be added or removed without changing the
meaning of a formula.
\end{remark}

\subsubsection{Renaming Bound Variables}

\begin{theorem}[Renaming Bound Variables]
If $y$ does not occur in $\varphi$, then:
\begin{align*}
\forall x\,\varphi &\;\equiv\; \forall y\,\varphi[y/x], \\
\exists x\,\varphi &\;\equiv\; \exists y\,\varphi[y/x].
\end{align*}
This is a consequence of alpha-equivalence.
\end{theorem}

\subsubsection{Prenex Normal Form}

\begin{definition}[Prenex Normal Form]
A first-order formula is in \emph{prenex normal form} (PNF) if it has the shape
\[
Q_1 x_1\, Q_2 x_2 \cdots Q_n x_n \;\; \psi,
\]
where each $Q_i \in \{\forall,\exists\}$ and $\psi$ contains no quantifiers
(i.e.\ $\psi$ is quantifier-free). The prefix
$Q_1 x_1 \cdots Q_n x_n$ is called the \emph{quantifier prefix}, and $\psi$ is
the \emph{matrix}.
\end{definition}

\begin{theorem}[Prenex Normal Form Theorem]
Every first-order formula is logically equivalent to a formula in prenex normal
form.
\end{theorem}

\begin{remark}[Procedure for Converting to Prenex Normal Form]
Given a formula $\varphi$, the following steps produce an equivalent formula in
prenex normal form:

\begin{enumerate}
  \item \textbf{Eliminate $\leftrightarrow$ and $\rightarrow$.}
  Rewrite using only $\neg,\wedge,\vee$:
  \[
  (A \rightarrow B) \equiv (\neg A \vee B),\qquad
  (A \leftrightarrow B) \equiv (A \rightarrow B)\wedge(B \rightarrow A).
  \]

  \item \textbf{Push negations inward (Negation Normal Form).}
  Use De Morgan's laws and quantifier-negation laws until $\neg$ applies only
  to atomic formulas.

  \item \textbf{Standardize bound variables apart.}
  Rename bound variables so that no variable is quantified twice and no bound
  variable coincides with any free variable.

  \item \textbf{Pull quantifiers outward.}
  Use the quantifier-pulling rules to move all quantifiers to the front.
\end{enumerate}
\end{remark}

\begin{center}
\renewcommand{\arraystretch}{1.25}
\begin{tabular}{|p{6.5cm}|p{6.5cm}|}
\hline
\textbf{Equivalence} & \textbf{Side Condition} \\
\hline
$(\forall x\,\phi)\wedge \psi \;\equiv\; \forall x\,(\phi\wedge \psi)$
& $x \notin \mathrm{FV}(\psi)$ \\
\hline
$\psi \wedge (\forall x\,\phi) \;\equiv\; \forall x\,(\psi\wedge \phi)$
& $x \notin \mathrm{FV}(\psi)$ \\
\hline
$(\exists x\,\phi)\vee \psi \;\equiv\; \exists x\,(\phi\vee \psi)$
& $x \notin \mathrm{FV}(\psi)$ \\
\hline
$\psi \vee (\exists x\,\phi) \;\equiv\; \exists x\,(\psi\vee \phi)$
& $x \notin \mathrm{FV}(\psi)$ \\
\hline
$(\forall x\,\phi)\vee \psi \;\equiv\; \forall x\,(\phi\vee \psi)$
& $x \notin \mathrm{FV}(\psi)$ \\
\hline
$(\exists x\,\phi)\wedge \psi \;\equiv\; \exists x\,(\phi\wedge \psi)$
& $x \notin \mathrm{FV}(\psi)$ \\
\hline
\end{tabular}
\end{center}

\begin{example}[Prenex Normal Form Conversion]
Normalize the scope of
\[
\varphi := \neg\Bigl(\forall x\,P(x) \rightarrow \exists y\,Q(y)\Bigr).
\]

\textbf{Step 1 (Eliminate $\rightarrow$).}
\[
\varphi \equiv \neg\Bigl(\neg\forall x\,P(x)\;\vee\;\exists y\,Q(y)\Bigr).
\]

\textbf{Step 2 (Push $\neg$ inward).}
\[
\varphi \equiv \bigl(\forall x\,P(x)\bigr)\wedge \bigl(\forall y\,\neg Q(y)\bigr).
\]

\textbf{Step 3 (Standardize apart).}
Variables are already distinct.

\textbf{Step 4 (Pull quantifiers outward).}
\[
\varphi \equiv \forall x\,\forall y\,(P(x)\wedge \neg Q(y)).
\]
\end{example}

% ---------------------------------------------------------
\subsection{Inference Rules for Quantifiers}
% ---------------------------------------------------------

\subsubsection{Universal Instantiation and Generalization}

\begin{definition}[Universal Instantiation (UI)]
From a universally quantified statement, one may infer any instance obtained by
substituting a term for the bound variable:
\[
\forall x\,\varphi \;\Rightarrow\; \varphi[t/x].
\]
\end{definition}

\begin{remark}[Side Condition for UI]
The term $t$ must be free for $x$ in $\varphi$; that is, no variable occurring in
$t$ may become bound as a result of the substitution.
\end{remark}

\begin{definition}[Universal Generalization (UG)]
From a formula, one may infer a universally quantified statement:
\[
\varphi \;\Rightarrow\; \forall x\,\varphi.
\]
\end{definition}

\begin{remark}[Restriction on UG]
The variable $x$ must not occur free in any undischarged assumption on which
$\varphi$ depends.
\end{remark}

\subsubsection{Existential Instantiation and Generalization}

\begin{definition}[Existential Instantiation (EI)]
From an existentially quantified statement, one may infer an instance with a
fresh constant symbol (witness):
\[
\exists x\,\varphi \;\Rightarrow\; \varphi[c/x].
\]
\end{definition}

\begin{remark}[Witness Discipline for EI]
The constant $c$ must be a \emph{new} constant symbol that does not appear in:
\begin{itemize}
  \item the formula $\varphi$,
  \item any undischarged assumption, or
  \item the final conclusion of the proof.
\end{itemize}
The constant $c$ represents an arbitrary but fixed witness to the existential
claim.
\end{remark}

\begin{definition}[Existential Generalization (EG)]
From a formula containing a term, one may infer an existential statement:
\[
\varphi[t/x] \;\Rightarrow\; \exists x\,\varphi.
\]
\end{definition}

\begin{remark}
EG has no side conditions: any term $t$ may be replaced by an existentially
quantified variable.
\end{remark}

\subsubsection{Summary of Quantifier Rules}

\begin{center}
\renewcommand{\arraystretch}{1.3}
\begin{tabular}{|c|c|c|}
\hline
\textbf{Rule} & \textbf{Form} & \textbf{Key Restriction} \\
\hline
UI & $\forall x\,\varphi \Rightarrow \varphi[t/x]$ & $t$ free for $x$ \\
EI & $\exists x\,\varphi \Rightarrow \varphi[c/x]$ & $c$ fresh witness \\
EG & $\varphi[t/x] \Rightarrow \exists x\,\varphi$ & none \\
UG & $\varphi \Rightarrow \forall x\,\varphi$ & $x$ not free in assumptions \\
\hline
\end{tabular}
\end{center}

\subsubsection{Derived Inference Patterns}

\begin{remark}[Arbitrary Element Argument]
To prove $\forall x\,\varphi(x)$, fix an arbitrary element $x$ and prove
$\varphi(x)$ without making any special assumptions about $x$. Then apply UG.
\end{remark}

\begin{remark}[Witness Argument]
To prove $\exists x\,\varphi(x)$, explicitly exhibit a term $t$ such that
$\varphi(t)$ holds. Then apply EG.
\end{remark}

\begin{remark}[Counterexample Argument]
To refute $\forall x\,\varphi(x)$, it suffices to produce a term $t$ such that
$\neg\varphi(t)$ holds.

To refute $\exists x\,\varphi(x)$, one must show that $\varphi(t)$ fails for all
terms $t$.
\end{remark}


\subsubsection{Logical Strength and Quantifier Structure}



\begin{definition}[Logical entailment]
Let $A$ and $B$ be statements (sentences in predicate logic).
We write
\[
A \models B
\]
and say that \emph{$A$ entails $B$} if, in every structure/interpretation in which $A$ is true,
$B$ is also true.
Equivalently, $A \models B$ means that $A \rightarrow B$ is valid.
\end{definition}

\begin{definition}[Logical strength]
Let $A$ and $B$ be statements.
We say that \emph{$A$ is (logically) stronger than $B$} if
\[
A \models B
\quad\text{but}\quad
B \not\models A.
\]
We say that \emph{$A$ and $B$ are logically equivalent} if
\[
A \models B \quad\text{and}\quad B \models A.
\]
\end{definition}

\begin{remark}
``Stronger'' here means ``harder to satisfy'' (fewer models).
If $A$ is stronger than $B$, then $A$ rules out more possibilities than $B$ does.
\end{remark}

\begin{remark}[Strength vs.\ information]
If $A \models B$, then $A$ contains at least as much information as $B$.
If $A$ is strictly stronger than $B$, then $A$ contains strictly more information.
\end{remark}



\begin{definition}[Model set viewpoint]
For a statement $A$, let $\mathrm{Mod}(A)$ denote the class of all interpretations in which $A$ is true.
Then
\[
A \models B \quad\Longleftrightarrow\quad \mathrm{Mod}(A) \subseteq \mathrm{Mod}(B).
\]
\end{definition}

\begin{remark}
Thus ``$A$ is stronger than $B$'' is literally the statement
\[
\mathrm{Mod}(A) \subsetneq \mathrm{Mod}(B).
\]
\end{remark}

\begin{remark}[Quantifier blocks]
A typical quantified statement has the schematic form
\[
Q_1 x_1\; Q_2 x_2\; \cdots\; Q_k x_k \;\; \Phi(x_1,\dots,x_k),
\]
where each $Q_i \in \{\forall,\exists\}$ and $\Phi$ is quantifier-free.
The \emph{quantifier ordering} is the sequence $Q_1,Q_2,\dots,Q_k$.
\end{remark}

\begin{definition}[Quantifier alternation]
A \emph{quantifier alternation} occurs when $Q_i \ne Q_{i+1}$ for some $i$.
For example, $\forall x\,\exists y\,\Phi(x,y)$ has one alternation, while $\forall x\,\forall y\,\Phi(x,y)$ has none.
\end{definition}

\begin{remark}[Why alternation increases ``strength demands'']
Statements of the form $\forall x\,\exists y\,\Phi(x,y)$ require a \emph{rule} assigning, to each $x$, some (possibly dependent) witness $y$.
In contrast, $\exists y\,\forall x\,\Phi(x,y)$ requires a \emph{single} uniform witness $y$ that works for all $x$.
These are fundamentally different demands.
\end{remark}

\begin{remark}[Order matters: in general you cannot swap $\forall$ and $\exists$]
In general,
\[
\forall x\,\exists y\,\Phi(x,y) \not\equiv \exists y\,\forall x\,\Phi(x,y).
\]
The first allows $y$ to depend on $x$; the second forbids that dependence.
\end{remark}

\begin{example}[A concrete swap failure]
Let the domain be $\mathbb{R}$ and let $\Phi(x,y)$ be the formula $(y>x)$.
Then
\[
\forall x\,\exists y\,(y>x)
\]
is true (given any $x$, take $y=x+1$), but
\[
\exists y\,\forall x\,(y>x)
\]
is false (no single real number is larger than all real numbers).
\end{example}

\begin{example}[Pure propositional entailment]
Let
\[
A := (P \land Q),
\qquad
B := P.
\]
Then
\[
A \models B,
\]
because whenever $P \land Q$ is true, $P$ must be true.
However,
\[
B \not\models A,
\]
since $P$ may be true while $Q$ is false.

Thus $P \land Q$ is strictly stronger than $P$.
\end{example}

\begin{example}[Divisibility over $\mathbb{Z}$]
Let
\[
A := \text{``$x$ is divisible by $4$''},
\qquad
B := \text{``$x$ is even''}.
\]
Then
\[
A \models B,
\]
because every integer divisible by $4$ is even.

But
\[
B \not\models A,
\]
since $6$ is even but not divisible by $4$.

Thus ``divisible by $4$'' is strictly stronger than ``even.''
\end{example}

\begin{example}[Quantifier strength: swapping order]
Let the domain be $\mathbb{R}$ and define
\[
A := \forall x \in \mathbb{R}\; \exists y \in \mathbb{R}\ (y > x),
\]
\[
B := \exists y \in \mathbb{R}\; \forall x \in \mathbb{R}\ (y > x).
\]

Statement $A$ is true: for any $x$, take $y = x + 1$.

Statement $B$ is false: no single real number exceeds all real numbers.

Hence
\[
B \models A,
\]
but
\[
A \not\models B.
\]

Thus $\exists y\,\forall x$ is strictly stronger than $\forall x\,\exists y$.
\end{example}

\begin{example}[Universal vs.\ existential strength]
Let the domain be $\mathbb{R}$ and define
\[
A := \forall x\; (x^2 \ge 0),
\]
\[
B := \exists x\; (x^2 \ge 0).
\]

Then
\[
A \models B,
\]
since if every $x$ satisfies $x^2 \ge 0$, then certainly some $x$ does.

But
\[
B \not\models A,
\]
since the existence of one nonnegative square does not imply all squares are nonnegative.

Thus universal statements are generally stronger than corresponding existential ones.
\end{example}

% =========================================================
\subsubsection*{A Strength Hierarchy for Common Logical Forms}

\begin{remark}
Logical strength can often be visualized by arranging statements so that
stronger statements appear above weaker ones.
If $A$ is above $B$, then $A \models B$.
\end{remark}

\paragraph{Model-set interpretation.}
Recall that
\[
A \models B
\quad \Longleftrightarrow \quad
\mathrm{Mod}(A) \subseteq \mathrm{Mod}(B).
\]
Thus a stronger statement has a \emph{smaller} class of models.

\bigskip

\paragraph{Basic propositional hierarchy (fixed predicates $P,Q$).}

\[
\begin{array}{c}
P \land Q \\[4pt]
\Downarrow \\[4pt]
P \\[4pt]
\Downarrow \\[4pt]
P \lor Q
\end{array}
\]

\begin{itemize}
\item $P \land Q$ is strongest (most restrictive).
\item $P$ is weaker.
\item $P \lor Q$ is weakest (least restrictive).
\end{itemize}

\bigskip

\paragraph{Quantifier strength hierarchy (same predicate $\Phi(x)$).}

\[
\begin{array}{c}
\forall x\, \Phi(x) \\[6pt]
\Downarrow \\[6pt]
\exists x\, \Phi(x)
\end{array}
\]

\begin{itemize}
\item Universal statements are generally stronger than existential ones.
\item If something holds for all $x$, then it certainly holds for some $x$.
\item The converse fails in general.
\end{itemize}

\bigskip

\paragraph{Alternating quantifier hierarchy (two-variable case).}

\[
\begin{array}{c}
\exists y\, \forall x\, \Phi(x,y) \\[6pt]
\Downarrow \\[6pt]
\forall x\, \exists y\, \Phi(x,y)
\end{array}
\]

\begin{itemize}
\item $\exists y\,\forall x$ requires a \emph{single uniform witness}.
\item $\forall x\,\exists y$ allows the witness to depend on $x$.
\item Uniformity is strictly stronger than dependence.
\end{itemize}

\begin{remark}[General Pattern]
Logical strength increases when you:
\begin{itemize}
\item Add conjuncts ($A \land B$ stronger than $A$),
\item Replace $\exists$ with $\forall$,
\item Move existential quantifiers outward (requiring uniform witnesses),
\item Introduce quantifier alternation.
\end{itemize}
\end{remark}

\begin{remark}[Caution]
Quantifier order cannot generally be swapped.
The hierarchy above holds for fixed predicates.
Strength comparisons may fail if the internal formula changes.
\end{remark}










































\subsubsection{Soundness and Completeness}

\begin{definition}[Soundness]
A proof system is \emph{sound} if every provable formula is valid.

Formally, if $\Gamma \vdash \varphi$ (i.e., $\varphi$ is provable from $\Gamma$),
then $\Gamma \models \varphi$ (i.e., $\varphi$ is a logical consequence of $\Gamma$).
\end{definition}

\begin{definition}[Completeness]
A proof system is \emph{complete} if every valid formula is provable.

Formally, if $\Gamma \models \varphi$, then $\Gamma \vdash \varphi$.
\end{definition}

\begin{theorem}[Soundness of First-Order Logic]
The standard inference rules for first-order logic (including UI, UG, EI, EG,
and the propositional rules) are sound: they preserve truth in all structures.

If $\Gamma \vdash \varphi$, then $\Gamma \models \varphi$.
\end{theorem}

\begin{theorem}[G\"{o}del's Completeness Theorem]
First-order logic is complete: every logically valid formula is provable.

If $\Gamma \models \varphi$, then $\Gamma \vdash \varphi$.

Equivalently, if a set of sentences $\Gamma$ is consistent (has no proof of
contradiction), then $\Gamma$ has a model.
\end{theorem}

\begin{remark}
Soundness ensures that proofs do not lead us astray: we cannot prove false
statements from true premises. Completeness ensures that proofs are powerful
enough: every true statement (in the sense of being true in all models) can
be established by a proof.

Together, soundness and completeness show that syntactic provability ($\vdash$)
and semantic entailment ($\models$) coincide for first-order logic.
\end{remark}

\begin{remark}
G\"{o}del's completeness theorem should not be confused with his incompleteness
theorems, which concern the limitations of formal systems capable of expressing
arithmetic.
\end{remark}

% ---------------------------------------------------------
\subsection{Equality}
% ---------------------------------------------------------

\subsubsection{Equality Introduction}

\begin{definition}[Equality Introduction (Reflexivity)]
For any term $t$, one may infer
\[
t = t.
\]
\end{definition}

\begin{remark}
Reflexivity requires no premises and holds for all terms under all
interpretations.
\end{remark}

\subsubsection{Equality Elimination}

\begin{definition}[Equality Elimination (Substitution of Equals)]
Let $\varphi$ be a formula and let $x$ be a variable.
From
\[
t_1 = t_2
\quad\text{and}\quad
\varphi[t_1/x],
\]
one may infer
\[
\varphi[t_2/x].
\]
\end{definition}

\begin{remark}
Equality elimination permits the replacement of a term by an equal term in any
formula position, provided the substitution is capture-avoiding.
\end{remark}

\subsubsection{Derived Equality Rules}

\begin{theorem}[Symmetry of Equality]
From $t_1 = t_2$, one may infer $t_2 = t_1$.
\end{theorem}

\begin{theorem}[Transitivity of Equality]
From $t_1 = t_2$ and $t_2 = t_3$, one may infer $t_1 = t_3$.
\end{theorem}

\begin{theorem}[Term Substitution under Equality]
If $t_1 = t_2$, then for any function symbol $f$,
\[
f(\dots,t_1,\dots) = f(\dots,t_2,\dots).
\]
\end{theorem}

\begin{theorem}[Predicate Substitution under Equality]
If $t_1 = t_2$ and $P$ is an $n$-ary predicate symbol, then
\[
P(\dots,t_1,\dots) \;\Leftrightarrow\; P(\dots,t_2,\dots).
\]
\end{theorem}

\begin{remark}
These derived rules express that functions and predicates respect equality.
\end{remark}

\subsubsection{Summary of Equality Rules}

\begin{center}
\renewcommand{\arraystretch}{1.3}
\begin{tabular}{|c|c|c|}
\hline
\textbf{Rule} & \textbf{Form} & \textbf{Status} \\
\hline
Reflexivity & $t=t$ & Inference rule \\
Equality Elimination & $t_1=t_2,\;\varphi[t_1/x]\Rightarrow\varphi[t_2/x]$ & Inference rule \\
Symmetry & $t_1=t_2 \Rightarrow t_2=t_1$ & Derived \\
Transitivity & $t_1=t_2,\;t_2=t_3 \Rightarrow t_1=t_3$ & Derived \\
Term Substitution & $f(\dots,t_1,\dots)=f(\dots,t_2,\dots)$ & Derived \\
Predicate Substitution & $P(\dots,t_1,\dots)\Leftrightarrow P(\dots,t_2,\dots)$ & Derived \\
\hline
\end{tabular}
\end{center}

% ---------------------------------------------------------
\subsection{Translation and Interpretation}
% ---------------------------------------------------------

\subsubsection{English to Logic}

\begin{remark}[Translation Discipline]
When translating a symbolic formula into natural language, logical equivalences
may be used to analyze or verify meaning. However, the English translation must
respect the original quantifier nesting of the given formula, unless explicitly
instructed to normalize or rewrite the expression.
\end{remark}

\begin{center}
\renewcommand{\arraystretch}{1.35}
\begin{tabular}{|p{5cm}|p{4.5cm}|p{4.5cm}|}
\hline
\textbf{Original Statement (English)}
& \textbf{Logical Form}
& \textbf{Negation (Logical Form)}
\\ \hline

Everyone has property $P$
&
$\forall x\,P(x)$
&
$\exists x\,\neg P(x)$
\\ \hline

Someone has property $P$
&
$\exists x\,P(x)$
&
$\forall x\,\neg P(x)$
\\ \hline

Everyone likes something
&
$\forall x\,\exists y\,L(x,y)$
&
$\exists x\,\forall y\,\neg L(x,y)$
\\ \hline

Someone likes everything
&
$\exists x\,\forall y\,L(x,y)$
&
$\forall x\,\exists y\,\neg L(x,y)$
\\ \hline

Every student passed every exam
&
$\forall x\,\forall y\,P(x,y)$
&
$\exists x\,\exists y\,\neg P(x,y)$
\\ \hline

Some student passed every exam
&
$\exists x\,\forall y\,P(x,y)$
&
$\forall x\,\exists y\,\neg P(x,y)$
\\ \hline

\end{tabular}
\end{center}

\subsubsection{Quantifier Scope Ambiguity}

\begin{remark}[Scope Determines Meaning]
Natural language often leaves quantifier scope ambiguous.
Formal logic resolves this ambiguity by fixing a precise quantifier order.
Changing the order of quantifiers generally changes the meaning of the statement.
\end{remark}

\begin{center}
\renewcommand{\arraystretch}{1.35}
\begin{tabular}{|p{4.8cm}|p{4.8cm}|p{6.2cm}|}
\hline
\textbf{English Sentence}
& \textbf{Logical Form}
& \textbf{Meaning}
\\ \hline

Everyone loves someone
&
$\forall x\,\exists y\,L(x,y)$
&
Each person may love a different person.
\\ \cline{2-3}
&
$\exists y\,\forall x\,L(x,y)$
&
There is one person whom everyone loves.
\\ \hline

Every student passed an exam
&
$\forall x\,\exists y\,P(x,y)$
&
Each student passed at least one (possibly different) exam.
\\ \cline{2-3}
&
$\exists y\,\forall x\,P(x,y)$
&
There is a single exam that all students passed.
\\ \hline

A teacher knows every student
&
$\exists x\,\forall y\,K(x,y)$
&
There is one teacher who knows all students.
\\ \cline{2-3}
&
$\forall y\,\exists x\,K(x,y)$
&
Every student is known by at least one teacher.
\\ \hline

Every function has a zero
&
$\forall f\,\exists x\,Z(f,x)$
&
Each function has at least one (possibly different) zero.
\\ \cline{2-3}
&
$\exists x\,\forall f\,Z(f,x)$
&
There is a single point that is a zero of every function.
\\ \hline

\end{tabular}
\end{center}

\subsubsection{Square of Opposition}

\begin{definition}[Quantified Opposition Forms]
Let $P(x)$ be a formula with one free variable.
The four standard quantified forms are:
\begin{align*}
\textbf{Universal Affirmative (A):} &\quad \forall x\,P(x) \\
\textbf{Universal Negative (E):} &\quad \forall x\,\neg P(x) \\
\textbf{Existential Affirmative (I):} &\quad \exists x\,P(x) \\
\textbf{Existential Negative (O):} &\quad \exists x\,\neg P(x)
\end{align*}
\end{definition}

\begin{center}
\begin{tikzpicture}[
  every node/.style={draw, rectangle, align=center, minimum width=4cm, minimum height=1.2cm},
  node distance=3.8cm
]

\node (A) {$\forall x\,P(x)$\\\small Universal Affirmative (A)};
\node (E) [right=of A] {$\forall x\,\neg P(x)$\\\small Universal Negative (E)};
\node (I) [below=of A] {$\exists x\,P(x)$\\\small Existential Affirmative (I)};
\node (O) [below=of E] {$\exists x\,\neg P(x)$\\\small Existential Negative (O)};

% Contradictories
\draw[<->, thick] (A) -- (O);
\draw[<->, thick] (E) -- (I);

% Subalternation
\draw[->, thick] (A) -- (I);
\draw[->, thick] (E) -- (O);

% Contraries
\draw[<->, dashed] (A) -- (E);

% Subcontraries
\draw[<->, dashed] (I) -- (O);

\end{tikzpicture}
\end{center}

\begin{remark}[Logical Relations in First-Order Logic]
In first-order logic with nonempty domains:
\begin{itemize}
  \item \textbf{Contradictories:}
  $\forall x\,P(x)$ and $\exists x\,\neg P(x)$;
  $\forall x\,\neg P(x)$ and $\exists x\,P(x)$.
  These pairs cannot both be true and cannot both be false.

  \item \textbf{Subalternation:}
  $\forall x\,P(x) \Rightarrow \exists x\,P(x)$ and
  $\forall x\,\neg P(x) \Rightarrow \exists x\,\neg P(x)$.

  \item \textbf{Contraries and Subcontraries:}
  These relations do \emph{not} generally hold in first-order logic without
  existential presuppositions.
\end{itemize}
\end{remark}

\begin{remark}[Modern Status of the Square]
In classical first-order logic, only contradiction and subalternation are
logically valid relations. The traditional notions of contrariety and
subcontrariety rely on existential assumptions and are not preserved in general
model-theoretic semantics.
\end{remark}

% ---------------------------------------------------------
\subsection{Common Errors and Fallacies}
% ---------------------------------------------------------

\subsubsection{Quantifier Fallacy Checklist}

\begin{center}
\renewcommand{\arraystretch}{1.4}
\begin{tabular}{|p{6cm}|p{7.5cm}|}
\hline
\textbf{Fallacy}
& \textbf{Diagnostic Question}
\\ \hline

\textbf{Failure to Flip Quantifier Under Negation}
&
Was $\forall$ changed to $\exists$ (or vice versa) when negating a quantified
statement?
\\ \hline

\textbf{Negating Predicate Only}
&
Was only the predicate negated while the quantifier was left unchanged?
\\ \hline

\textbf{Partial Quantifier Negation}
&
When negating nested quantifiers, were \emph{all} quantifiers crossed by the
negation and flipped?
\\ \hline

\textbf{Quantifier Order Confusion}
&
Was the order of quantifiers changed without justification?
\\ \hline

\textbf{Illicit Quantifier Swap}
&
Were $\forall$ and $\exists$ commuted when they are not of the same type?
\\ \hline

\textbf{Variable Capture}
&
Did substitution introduce a variable that became bound unintentionally?
\\ \hline

\textbf{Illegal Existential Instantiation}
&
Was a witness chosen before an existential statement was established?
\\ \hline

\textbf{Illicit Universal Generalization}
&
Was a universally quantified conclusion drawn from a statement depending on a
specific object?
\\ \hline

\textbf{Scope Ambiguity}
&
Was the scope of a quantifier unclear or implicitly extended beyond its
syntactic bounds?
\\ \hline

\textbf{Vacuous Quantification Misuse}
&
Was a quantifier added or removed even though the variable does not occur free
in the formula?
\\ \hline

\end{tabular}
\end{center}

\begin{remark}
Quantifier errors almost always stem from ignoring scope or treating quantifiers
as informal linguistic modifiers. Every quantifier introduces a binding context
that must be respected syntactically before semantic reasoning is applied.
\end{remark}

\subsubsection{Common Incorrect Negations}

\begin{center}
\renewcommand{\arraystretch}{1.35}
\begin{tabular}{|p{4.5cm}|p{4.5cm}|p{4.5cm}|}
\hline
\textbf{Original Statement}
& \textbf{Incorrect Negation}
& \textbf{Why It Is Wrong}
\\ \hline

$\forall x\,P(x)$
&
$\forall x\,\neg P(x)$
&
Fails to flip the quantifier; this asserts that \emph{everyone} fails $P$.
\\ \hline

$\exists x\,P(x)$
&
$\exists x\,\neg P(x)$
&
Negates the predicate but not the existential claim.
\\ \hline

$\forall x\,\exists y\,P(x,y)$
&
$\exists x\,\exists y\,\neg P(x,y)$
&
Only the outer quantifier was flipped.
\\ \hline

$\exists x\,\forall y\,P(x,y)$
&
$\forall x\,\forall y\,\neg P(x,y)$
&
Overstrengthened the negation.
\\ \hline

$\forall x\,(P(x)\rightarrow Q(x))$
&
$\forall x\,(P(x)\wedge \neg Q(x))$
&
Negates incorrectly; the correct negation is
$\exists x\,(P(x)\wedge \neg Q(x))$.
\\ \hline

\end{tabular}
\end{center}

\begin{remark}[Negation Discipline]
When negating a quantified statement, the negation must be pushed inward across
\emph{every} quantifier, flipping each one, until it reaches the atomic
predicate. Negating only the predicate or only one quantifier changes the
logical claim.
\end{remark}

\subsubsection{Common Errors with Inference Rules}

\begin{remark}[Common Errors]
\begin{itemize}
  \item Choosing a witness before establishing an existential premise.
  \item Using a witness constant introduced by EI outside its allowed scope.
  \item Generalizing universally over a variable that depends on a special
        assumption.
  \item Substituting inside the scope of a quantifier without checking capture.
  \item Replacing a term by an equal term in one occurrence but not another.
\end{itemize}
\end{remark}

\begin{remark}
If a proof step would still make sense after replacing $\forall$ with ``for most''
or $\exists$ with ``maybe'', it is almost certainly invalid.
\end{remark}

% ---------------------------------------------------------
\subsection{Summary Tables}
% ---------------------------------------------------------

\subsubsection{Comparison: Propositional vs.\ Predicate Logic}

\begin{center}
\renewcommand{\arraystretch}{1.3}
\begin{tabular}{|p{4cm}|p{5cm}|p{5cm}|}
\hline
\textbf{Aspect}
& \textbf{Propositional Logic}
& \textbf{Predicate Logic}
\\ \hline

\textbf{Atomic formulas}
&
Propositional variables (e.g.\ $P$, $Q$, $R$)
&
Predicate symbols applied to terms (e.g.\ $P(x)$, $R(x,y)$)
\\ \hline

\textbf{Internal structure}
&
No internal structure
&
Contain terms, variables, constants, and functions
\\ \hline

\textbf{Dependence on variables}
&
None
&
May contain free variables
\\ \hline

\textbf{Semantic interpretation}
&
Assigned a truth value directly
&
True or false relative to a structure and assignment
\\ \hline

\textbf{Use of quantifiers}
&
Not available
&
Essential component
\\ \hline

\textbf{Semantic evaluation}
&
Truth tables
&
Satisfaction in a structure
\\ \hline

\end{tabular}
\end{center}

\subsubsection{Quantifier Rules Summary}

\begin{center}
\renewcommand{\arraystretch}{1.35}
\begin{tabular}{|p{4cm}|p{6cm}|p{5cm}|}
\hline
\textbf{Rule Name}
& \textbf{Schema}
& \textbf{Conditions / Notes}
\\ \hline

\textbf{Universal Instantiation}
&
$\forall x\,\varphi \;\Rightarrow\; \varphi[t/x]$
&
$t$ must be free for $x$ in $\varphi$
\\ \hline

\textbf{Existential Generalization}
&
$\varphi[t/x] \;\Rightarrow\; \exists x\,\varphi$
&
$t$ may be any term
\\ \hline

\textbf{Existential Instantiation}
&
$\exists x\,\varphi \;\Rightarrow\; \varphi[c/x]$
&
$c$ is a new constant symbol
\\ \hline

\textbf{Universal Generalization}
&
$\varphi \;\Rightarrow\; \forall x\,\varphi$
&
$x$ not free in any undischarged assumption
\\ \hline

\textbf{Quantifier Negation}
&
$\neg\forall x\,\varphi \;\equiv\; \exists x\,\neg\varphi$
&
\\ \cline{2-3}
&
$\neg\exists x\,\varphi \;\equiv\; \forall x\,\neg\varphi$
&
\\ \hline

\textbf{Quantifier Commutation}
&
$\forall x\,\forall y\,\varphi \;\equiv\; \forall y\,\forall x\,\varphi$
&
Same quantifier type only
\\ \cline{2-3}
&
$\exists x\,\exists y\,\varphi \;\equiv\; \exists y\,\exists x\,\varphi$
&
\\ \hline

\textbf{Vacuous Quantification}
&
$\forall x\,\varphi \;\equiv\; \varphi$
&
$x$ not free in $\varphi$
\\ \cline{2-3}
&
$\exists x\,\varphi \;\equiv\; \varphi$
&
\\ \hline

\textbf{Renaming Bound Variables}
&
$\forall x\,\varphi \;\equiv\; \forall y\,\varphi[y/x]$
&
$y$ not free in $\varphi$
\\ \hline

\end{tabular}
\end{center}

\subsubsection{Syntax to Semantics Overview}

\begin{center}
\resizebox{\linewidth}{!}{%
\begin{tikzpicture}[
  node distance=3.2cm,
  every node/.style={draw, rectangle, rounded corners, align=center, inner sep=8pt},
  arrow/.style={->, thick}
]

\node (var) {\textbf{Variables}\\ $x_1,\dots,x_n$};

\node (term) [right=of var] {\textbf{Terms}\\ $t_1,\dots,t_n$};

\node (atom) [right=of term] {\textbf{Atomic Formula}\\ $P(t_1,\dots,t_n)$};

\node (form) [right=of atom] {\textbf{Formula}\\ $\forall x\,\varphi,\;\exists x\,\varphi$};

\node (rel) [right=of form] {\textbf{Truth in Structure}\\ $\mathcal{M}, s \models \varphi$};

\draw[arrow] (var) -- node[above]{\small substitution} (term);
\draw[arrow] (term) -- node[above]{\small formation} (atom);
\draw[arrow] (atom) -- node[above]{\small logical construction} (form);
\draw[arrow] (form) -- node[above]{\small satisfaction} (rel);

\end{tikzpicture}
}
\end{center}

\begin{remark}
Variables and terms belong purely to syntax.
Predicates form atomic formulas.
Logical connectives and quantifiers build complex formulas.
A structure and variable assignment then determine whether a formula is satisfied,
with quantifiers ranging over the domain of discourse.
\end{remark}