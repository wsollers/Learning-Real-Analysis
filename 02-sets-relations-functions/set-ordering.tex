\subsection{Ordered Sets}

\begin{definition}[Ordered Set]
An \emph{ordered set} is a pair $(A,\leq)$ consisting of a set $A$ together with
a binary relation $\leq$ on $A$ that is a partial order.

That is, $\leq$ satisfies:
\begin{enumerate}[label=(\roman*)]
\item \emph{Reflexivity:} $\forall a \in A,\; a \leq a$;
\item \emph{Antisymmetry:} $\forall a,b \in A,\;
(a \leq b \land b \leq a) \Rightarrow a=b$;
\item \emph{Transitivity:} $\forall a,b,c \in A,\;
(a \leq b \land b \leq c) \Rightarrow a \leq c$.
\end{enumerate}
\end{definition}

\begin{remark}
An ordered set is therefore a set equipped with a specified partial order.
Different orders on the same underlying set generally produce different ordered
sets.
\end{remark}

\begin{definition}[Strict Order]
Let $(A,\leq)$ be an ordered set.
The associated \emph{strict order} $<$ on $A$ is defined by
\[
a < b
\;\;\Longleftrightarrow\;\;
(a \leq b \land a \neq b).
\]
\end{definition}

\begin{remark}
The relation $<$ is irreflexive and transitive.
Conversely, given a strict partial order $<$ on $A$, one may define a non-strict
order $\leq$ by
\[
a \leq b \;\Longleftrightarrow\; (a < b \lor a=b).
\]
Thus strict and non-strict orders carry the same information.
\end{remark}

\begin{definition}[Comparable Elements]
Let $(A,\leq)$ be an ordered set.
Two elements $a,b \in A$ are said to be \emph{comparable} if either
\[
a \leq b \quad\text{or}\quad b \leq a.
\]
\end{definition}

\begin{definition}[Noncomparable Elements]
Let $(A,\leq)$ be an ordered set.
Two elements $a,b \in A$ are said to be \emph{noncomparable} (or \emph{incomparable})
if neither precedes the other under the order.

Formally,
\[
a \nleq b \;\;\text{and}\;\; b \nleq a.
\]
Equivalently,
\[
\neg(a \leq b) \land \neg(b \leq a).
\]
\end{definition}

\begin{remark}
Noncomparable elements can occur only in partially ordered sets.
In a totally ordered set, every pair of elements is comparable, so
noncomparable elements do not exist.
\end{remark}


\begin{definition}[Total (Linear) Order]
An ordered set $(A,\leq)$ is called a \emph{total order} (or \emph{linear order})
if every pair of elements is comparable:
\[
\forall a,b \in A,\; a \leq b \lor b \leq a.
\]
\end{definition}

\begin{remark}
Total orders generalize the familiar ordering of the real numbers.
Partial orders allow incomparable elements and arise naturally in algebra,
topology, and logic.
\end{remark}

\begin{definition}[Upper and Lower Bounds]
Let $(A,\leq)$ be an ordered set and let $S \subseteq A$.

\begin{itemize}
\item An element $u \in A$ is an \emph{upper bound} of $S$ if
\[
\forall s \in S,\; s \leq u.
\]
\item An element $\ell \in A$ is a \emph{lower bound} of $S$ if
\[
\forall s \in S,\; \ell \leq s.
\]
\end{itemize}
\end{definition}

\begin{remark}
The notions of upper and lower bounds depend on the chosen order.
They will play a central role in the study of completeness, supremum and infimum,
and ordered topological and metric structures.
\end{remark}

\begin{definition}[Minimal and Maximal Elements]
Let $(A,\leq)$ be an ordered set and let $S \subseteq A$.

\begin{itemize}
\item An element $m \in S$ is a \emph{minimal element} of $S$ if there is no
$s \in S$ such that $s < m$.
\item An element $M \in S$ is a \emph{maximal element} of $S$ if there is no
$s \in S$ such that $M < s$.
\end{itemize}
\end{definition}

\begin{remark}
Minimal and maximal elements need not be unique, and need not exist.
They should not be confused with least or greatest elements, which are stronger
notions.
\end{remark}

\begin{definition}[Least and Greatest Elements]
Let $(A,\leq)$ be an ordered set and let $S \subseteq A$.

\begin{itemize}
\item An element $\ell \in S$ is the \emph{least element} of $S$ if
\[
\forall s \in S,\; \ell \leq s.
\]
\item An element $g \in S$ is the \emph{greatest element} of $S$ if
\[
\forall s \in S,\; s \leq g.
\]
\end{itemize}
\end{definition}

\begin{remark}
If a least or greatest element exists, it is unique.
Every least element is minimal, and every greatest element is maximal, but the
converses need not hold.
\end{remark}

\begin{remark}[Transition]
Ordered sets provide the abstract framework for order structures on the real
numbers, function spaces, and metric spaces.
In later sections, order will interact with topology and analysis through
intervals, monotone functions, and completeness properties.
\end{remark}

\begin{definition}[Order-Preserving Mapping]
Let $(M,\leq)$ and $(M',\leq')$ be partially ordered sets, and let
\[
f : M \to M'
\]
be a function.
We say that $f$ is \emph{order-preserving} (or \emph{monotone}) if
\[
\forall a,b \in M,\quad a \leq b \;\Longrightarrow\; f(a) \leq' f(b).
\]
\end{definition}

\begin{definition}[Order Isomorphism]
Let $(M,\leq)$ and $(M',\leq')$ be partially ordered sets.
A function
\[
f : M \to M'
\]
is called an \emph{order isomorphism} if it is bijective and satisfies
\[
\forall a,b \in M,\quad
a \leq b \;\Longleftrightarrow\; f(a) \leq' f(b).
\]
\end{definition}

\begin{remark}
An order isomorphism preserves and reflects the order structure exactly.
In particular, it preserves all order-theoretic properties such as
comparability, minimal and maximal elements, bounds, and chains.

Two partially ordered sets are said to be \emph{order-isomorphic} if there
exists an order isomorphism between them.
In this case, the two ordered sets are considered structurally identical
from the standpoint of order theory.
\end{remark}


\begin{definition}[Well-Ordered Set]
An ordered set $(A,<)$ is said to be \emph{well-ordered} if every nonempty subset
$S \subseteq A$ has a least element.

That is,
\[
\forall S \subseteq A,\;
\bigl(S \neq \varnothing \;\Rightarrow\; \exists m \in S \text{ such that }
m \leq s \text{ for all } s \in S \bigr).
\]
\end{definition}

\begin{remark}
Every well-ordered set is a totally ordered set.
In particular, any two elements of a well-ordered set are comparable.

The defining property is stronger than totality: it requires the existence of
a least element in \emph{every} nonempty subset, not just the whole set.
\end{remark}

\begin{example}
\begin{enumerate}
\item $(\mathbb{N}, \leq)$ is well-ordered.
\item $(\mathbb{Z}, \leq)$ is \emph{not} well-ordered, since the subset
$\{\dots,-3,-2,-1\}$ has no least element.
\item $(\mathbb{R}, \leq)$ is not well-ordered; for example, the interval $(0,1)$
has no least element.
\end{enumerate}
\end{example}

\begin{remark}[Connection to Ordinal Numbers]
Well-ordered sets admit a powerful classification.
Two well-ordered sets are said to be \emph{order-isomorphic} if there exists an
order-preserving bijection between them.

An \emph{ordinal number} is defined (in set theory) as an equivalence class of
well-ordered sets under order isomorphism.
Thus, ordinal numbers measure the \emph{order type} of well-ordered sets rather
than their size.

Ordinal theory will be developed later; at this stage, well-ordering is the
fundamental structural concept.
\end{remark}


\begin{definition}[Chain]
A subset $C \subseteq A$ of a partially ordered set $(A,\le)$ is called a
\emph{chain} if every pair of elements in $C$ is comparable.
\end{definition}

\begin{definition}[Antichain]
A subset $S \subseteq A$ is called an \emph{antichain} if no two distinct elements
of $S$ are comparable.
\end{definition}

\begin{definition}[Initial Segment]
Let $(A,\le)$ be an ordered set.
A subset $I \subseteq A$ is an \emph{initial segment} if
\[
a \in I \text{ and } b \le a \;\Rightarrow\; b \in I.
\]
\end{definition}











