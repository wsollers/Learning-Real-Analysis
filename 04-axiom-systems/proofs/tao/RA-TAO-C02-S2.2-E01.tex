% =========================================================
% Proof Sheets — Tao (Analysis I) — Induction / Natural Numbers
% Sections 2.2 and 2.3
% =========================================================

% --------------------------------------
% Proof: RA-TAO-C02-S2.2-E01
% --------------------------------------
\clearpage
\phantomsection
\hypertarget{proof-RA-TAO-C02-S2.2-E01}{}

\section*{Proof — RA-TAO-C02-S2.2-E01}

\noindent
\hyperlink{ws-RA-TAO-C02-S2.2-E01}{\textbf{← Back to worksheet}}

\noindent
\textbf{Source.}
\srccite{TaoAnalysisI}{Chapter~2, \S2.2, Exercise~2.2.1}.

\vspace{0.75em}

\noindent
\textbf{Goal.}
Prove Tao's Proposition~2.2.5 (Addition is associative), namely: for all
$a,b,c\in\mathbb{N}$,
\[
(a+b)+c \;=\; a+(b+c).
\]
\emph{(Hint: use induction.)}

\vspace{0.75em}

\noindent
\textbf{Proof.}

\vspace{1em}
\noindent
\hyperlink{ws-RA-TAO-C02-S2.2-E01}{\textbf{← Back to worksheet}}

