% Fragment (no \documentclass)

\hypertarget{ABB-P1-C01-WS-001}{}

\subsubsection*{ABB-P1-C01-WS-001: Abbott — Chapter 1, Section 1.3 (Axiom of Completeness)}

\noindent\textbf{Navigation: }
\hyperlink{P1-ABB-C01}{Back to chapter}
\quad|\quad
\hyperlink{P1-ABB}{Back to author}
\quad|\quad
\hyperlink{Phase1}{Back to Phase 1}

\readingsection
\begin{itemize}
  \item Abbott, \textit{Understanding Analysis} (2nd ed.), Chapter 1, Section 1.3.
  \item Exercises: 1.3.1, 1.3.3, 1.3.4, 1.3.5, 1.3.6, 1.3.7, 1.3.8, 1.3.9, 1.3.10.
\end{itemize}

\notessection
\begin{itemize}
  \item Write complete proofs unless the problem explicitly says ``without proofs.''
  \item Track where completeness is used (as opposed to order/field axioms).
\end{itemize}

\subsubsection*{Problems}

\begin{enumerate}[label=\textbf{Exercise 1.3.\arabic*:}, leftmargin=*]

\item
\begin{enumerate}[label=(\alph*)]
\item Write a formal definition, in the style of Definition 1.2.2, for the
\emph{infimum} (or greatest lower bound) of a set.
\item Now, state and prove a version of Lemma 1.3.8 for greatest lower bounds.
\end{enumerate}

\item[\textbf{Exercise 1.3.3.}]
Let $A$ be nonempty and bounded below, and define
\[
B=\{\,b\in\R: b \text{ is a lower bound for } A\,\}.
\]
\begin{enumerate}[label=(\alph*)]
\item Show that $\sup B=\inf A$.
\item Use (a) to explain why there is no need to assert that greatest lower bounds exist
as part of the Axiom of Completeness.
\end{enumerate}

\item[\textbf{Exercise 1.3.4.}]
Let $A_1,A_2,A_3,\dots$ be a collection of nonempty sets, each of which is bounded above.
\begin{enumerate}[label=(\alph*)]
\item Find a formula for $\sup(A_1\cup A_2)$. Extend this to $\sup\!\left(\bigcup_{k=1}^{n}A_k\right)$.
\item Consider $\sup\!\left(\bigcup_{k=1}^{\infty}A_k\right)$. Does the formula in (a) extend to the infinite case?
\end{enumerate}

\item[\textbf{Exercise 1.3.5.}]
As in Example 1.3.7, let $A\subseteq\R$ be nonempty and bounded above, and let $c\in\R$.
Define $cA=\{ca: a\in A\}$.
\begin{enumerate}[label=(\alph*)]
\item If $c\ge 0$, show that $\sup(cA)=c\sup A$.
\item Postulate a similar type of statement for $\sup(cA)$ for the case $c<0$.
\end{enumerate}

\item[\textbf{Exercise 1.3.6.}]
Given sets $A$ and $B$, define
\[
A+B=\{a+b: a\in A \text{ and } b\in B\}.
\]
Follow these steps to prove that if $A$ and $B$ are nonempty and bounded above then
$\sup(A+B)=\sup A+\sup B$.
\begin{enumerate}[label=(\alph*)]
\item Let $s=\sup A$ and $t=\sup B$. Show $s+t$ is an upper bound for $A+B$.
\item Now let $u$ be an arbitrary upper bound for $A+B$, and temporarily fix $a\in A$. Show $t\le u-a$.
\item Finally, show $\sup(A+B)=s+t$.
\item Construct another proof of this same fact using Lemma 1.3.8.
\end{enumerate}

\item[\textbf{Exercise 1.3.7.}]
Prove that if $a$ is an upper bound for $A$, and if $a$ is also an element of $A$,
then it must be that $a=\sup A$.

\item[\textbf{Exercise 1.3.8.}]
Compute, \emph{without proofs}, the suprema and infima (if they exist) of the following sets:
\begin{enumerate}[label=(\alph*)]
\item $\{m/n: m,n\in\N \text{ with } m<n\}$.
\item $\{(-1)^m/n: m,n\in\N\}$.
\item $\{n/(3n+1): n\in\N\}$.
\item $\{m/(m+n): m,n\in\N\}$.
\end{enumerate}

\item[\textbf{Exercise 1.3.9.}]
\begin{enumerate}[label=(\alph*)]
\item If $\sup A<\sup B$, show that there exists an element $b\in B$ that is an upper bound for $A$.
\item Give an example to show that this is not always the case if we only assume $\sup A\le \sup B$.
\end{enumerate}

\item[\textbf{Exercise 1.3.10 (Cut Property).}]
If $A$ and $B$ are nonempty, disjoint sets with $A\cup B=\R$ and $a<b$ for all $a\in A$ and $b\in B$,
then there exists $c\in\R$ such that $x\le c$ whenever $x\in A$ and $x\ge c$ whenever $x\in B$.
\begin{enumerate}[label=(\alph*)]
\item Use the Axiom of Completeness to prove the Cut Property.
\end{enumerate}

\end{enumerate}

\subsubsection*{Proof Stub Pages}
\begin{itemize}
  \item \hyperlink{ABB-P1-C01-PR-001}{ABB-P1-C01-PR-001 (Proof Pages)}
\end{itemize}

% Fragment (no \documentclass)

\hypertarget{ABB-P1-C01-PR-001}{}
\subsubsection*{ABB-P1-C01-PR-001: Abbott Exercise 1.3.1}

\noindent\textbf{Navigation: }
\hyperlink{ABB-P1-C01-WS-001}{Back to worksheet}
\quad|\quad
\hyperlink{P1-ABB-C01}{Back to chapter}
\quad|\quad
\hyperlink{Phase1}{Back to Phase 1}

\medskip
\hrule
\medskip

\noindent\textbf{Problem (Abbott 1.3.1).}
\begin{enumerate}[label=(\alph*)]
\item Write a formal definition, in the style of Definition 1.2.2, for the
\emph{infimum} (or greatest lower bound) of a set.
\item Now state and prove a version of Lemma 1.3.8 for greatest lower bounds.
\end{enumerate}

\medskip
\noindent\textbf{Proof.}
\begin{proof}
\vspace{16cm}
\end{proof}

\clearpage

