% ============================================================
% Abbott — Understanding Analysis (2nd ed.)
% Section 1.3: The Axiom of Completeness
% Selected Exercises Worksheet
% INCLUDE FILE — to be used with \input{...}
% ============================================================

\section*{Abbott — Understanding Analysis (2nd ed.)}
\addcontentsline{toc}{section}{Abbott — Understanding Analysis (2nd ed.)}

\subsection*{Section 1.3: The Axiom of Completeness — Selected Exercises}
\addcontentsline{toc}{subsection}{Section 1.3: The Axiom of Completeness (Exercises)}

\noindent
\textbf{Instructions.}
Write complete and rigorous proofs unless otherwise indicated.  
You may use only definitions and results established up to Section~1.3.

\bigskip

\begin{enumerate}[label=\textbf{Exercise 1.3.\arabic*:}, leftmargin=*]

% ---------------- 1.3.1 ----------------
\item
\begin{enumerate}[label=(\alph*)]
\item Write a formal definition, in the style of Definition~1.2.2, for the
\emph{infimum} (or greatest lower bound) of a set.
\item Now state and prove a version of Lemma~1.3.8 for greatest lower bounds.
\end{enumerate}

% ---------------- 1.3.3 ----------------
\item
Let $A$ be nonempty and bounded below, and define
\[
B = \{\, b \in \R : b \text{ is a lower bound for } A \,\}.
\]
\begin{enumerate}[label=(\alph*)]
\item Show that $\sup B = \inf A$.
\item Use part (a) to explain why there is no need to assert that greatest lower
bounds exist as part of the Axiom of Completeness.
\end{enumerate}

% ---------------- 1.3.4 ----------------
\item
Let $A_1, A_2, A_3, \dots$ be a collection of nonempty sets, each of which is
bounded above.
\begin{enumerate}[label=(\alph*)]
\item Find a formula for $\sup(A_1 \cup A_2)$.
\item Extend this formula to $\sup\!\left(\bigcup_{k=1}^{n} A_k\right)$.
\item Does the formula in part (b) extend to the infinite case
$\sup\!\left(\bigcup_{k=1}^{\infty} A_k\right)$?
\end{enumerate}

% ---------------- 1.3.5 ----------------
\item
As in Example~1.3.7, let $A \subseteq \R$ be nonempty and bounded above, and let
$c \in \R$. This time define the set
\[
cA = \{ ca : a \in A \}.
\]
\begin{enumerate}[label=(\alph*)]
\item If $c \ge 0$, show that $\sup(cA) = c \sup A$.
\item Postulate a similar type of statement for $\sup(cA)$ in the case $c < 0$.
\end{enumerate}

% ---------------- 1.3.6 ----------------
\item
Given sets $A$ and $B$, define
\[
A + B = \{ a + b : a \in A \text{ and } b \in B \}.
\]
Follow the steps below to prove that if $A$ and $B$ are nonempty and bounded above,
then
\[
\sup(A + B) = \sup A + \sup B.
\]
\begin{enumerate}[label=(\alph*)]
\item Let $s = \sup A$ and $t = \sup B$. Show that $s + t$ is an upper bound for
$A + B$.
\item Now let $u$ be an arbitrary upper bound for $A + B$, and temporarily fix
$a \in A$. Show that $t \le u - a$.
\item Finally, show that $\sup(A + B) = s + t$.
\item Construct another proof of this same fact using Lemma~1.3.8.
\end{enumerate}

% ---------------- 1.3.7 ----------------
\item
Prove that if $a$ is an upper bound for $A$, and if $a$ is also an element of $A$,
then it must be that $a = \sup A$.

% ---------------- 1.3.8 ----------------
\item
Compute, \emph{without proofs}, the suprema and infima (if they exist) of the
following sets:
\begin{enumerate}[label=(\alph*)]
\item $\{\, m/n : m,n \in \N \text{ with } m < n \,\}$
\item $\{\, (-1)^m / n : m,n \in \N \,\}$
\item $\{\, n/(3n+1) : n \in \N \,\}$
\item $\{\, m/(m+n) : m,n \in \N \,\}$
\end{enumerate}

% ---------------- 1.3.9 ----------------
\item
\begin{enumerate}[label=(\alph*)]
\item If $\sup A < \sup B$, show that there exists an element $b \in B$ that is an
upper bound for $A$.
\item Give an example to show that this is not always the case if we only assume
$\sup A \le \sup B$.
\end{enumerate}

% ---------------- 1.3.10 ----------------
\item \textbf{(Cut Property).}
The Cut Property of the real numbers is the following:

\medskip
If $A$ and $B$ are nonempty, disjoint sets with $A \cup B = \R$ and $a < b$ for all
$a \in A$ and $b \in B$, then there exists $c \in \R$ such that
\[
x \le c \quad \text{whenever } x \in A,
\qquad
x \ge c \quad \text{whenever } x \in B.
\]

\begin{enumerate}[label=(\alph*)]
\item Use the Axiom of Completeness to prove the Cut Property.
\end{enumerate}

\end{enumerate}

\bigskip
\hrule
\bigskip

\noindent
\textbf{Note.}
These exercises are foundational for understanding the logical content of the
Axiom of Completeness and its consequences for supremum and infimum arguments.
