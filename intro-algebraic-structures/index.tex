% =========================================================
% Algebraic Structures — Structural Roadmap
% Driver Text: Joseph Gallian, Contemporary Abstract Algebra
% =========================================================
\subsection{Algebraic Structures}

% =========================================================
% Breadcrumb
% =========================================================
\begin{tcolorbox}[
  colback=gray!6,
  colframe=gray!40,
  arc=2pt,
  left=8pt, right=8pt, top=6pt, bottom=6pt,
  title={\small\textbf{Where You Are in the Journey}},
  fonttitle=\small\bfseries
]
\begin{center}
\small
Propositional Logic
$\;\to\;$ Predicate Calculus
$\;\to\;$ Sets \& Functions
$\;\to\;$ Proof Techniques
$\;\to\;$ Real Analysis
$\;\to\;$ \textbf{Algebraic Structures}
$\;\to\;$ Linear Algebra
$\;\to\;$ Topology
$\;\to\;$ $\cdots$
\end{center}

\medskip
\noindent\textbf{How we got here.}
Sets and functions gave us the language of mathematical structure,
and proof techniques gave us the tools to reason about it.
Real analysis showed what rigorous reasoning looks like on a concrete
number system. Algebraic structures now ask the deeper question:
what is the \emph{minimal} set of axioms needed for the most important
algebraic phenomena --- identity, inverses, and arithmetic --- to occur?

\medskip
\noindent\textbf{What this chapter builds.}
We construct the hierarchy of algebraic structures: groups, rings,
and fields. Each is defined by progressively richer axiom systems.
Fields are the scalars over which vector spaces are built,
and groups are the additive backbone of every vector space.

\medskip
\noindent\textbf{Where this leads.}
Linear algebra inherits the field and group axioms directly:
a vector space is an abelian group equipped with a scalar multiplication
by a field, and every proof about vector spaces draws on both simultaneously.
\end{tcolorbox}
\vspace{1em}

% =========================================================
% Structural Roadmap
% =========================================================
\subsubsection*{Structural Roadmap}

The development of algebraic structures in this project follows
the definition--theorem--structure architecture used
throughout the analysis volumes.

The primary driver is \textit{Contemporary Abstract Algebra}
by Joseph Gallian. The emphasis is on axiom systems,
structural consequences, and the hierarchy of algebraic objects
rather than computational techniques.

Each major topic is organized as:
\begin{center}
\textbf{Definitions $\longrightarrow$ Main Theorems
$\longrightarrow$ Consequences and Structural Insight}
\end{center}

The global progression is:
\begin{enumerate}
  \item Groups and abelian groups
  \item Subgroups
  \item Rings
  \item Integral domains
  \item Fields and their properties
\end{enumerate}

\vspace{1em}

\begin{remark}[Structural Position]
The structures developed here are not studied for their own sake alone.
Groups provide the additive structure of every vector space.
Fields provide the scalars. The interaction between them
is the subject of linear algebra.
\end{remark}

\begin{remark}[Dependency Note]
The uniqueness theorems proved here --- uniqueness of identity,
uniqueness of inverses --- are the same theorems invoked in
vector space proofs. They are proved once here and cited by
theorem number thereafter.
\end{remark}

% =========================================================
% I. Groups
% =========================================================
% =========================================================
% Groups
% =========================================================
\subsubsection{Groups}

% =========================================================
% Binary Operations
% =========================================================
\paragraph{Binary Operations}

\begin{definition}[Binary Operation]
Let $G$ be a set.
A \emph{binary operation} on $G$ is a function
\[
\star : G \times G \to G.
\]
For $a, b \in G$, we write $a \star b$ for the image of $(a,b)$ under $\star$.
\end{definition}

\begin{remark}
The codomain of $\star$ is $G$ itself. This means that for any $a, b \in G$,
the result $a \star b$ is again an element of $G$.
This property is called \emph{closure} and is built into the definition of a
binary operation.

Intuitively: a binary operation takes two elements of a set and produces
a third element of the \emph{same} set.
\end{remark}

\begin{definition}[Associativity]
A binary operation $\star$ on $G$ is \emph{associative} if
\[
(a \star b) \star c = a \star (b \star c)
\quad \text{for all } a, b, c \in G.
\]
\end{definition}

\begin{definition}[Commutativity]
A binary operation $\star$ on $G$ is \emph{commutative} if
\[
a \star b = b \star a
\quad \text{for all } a, b \in G.
\]
\end{definition}

\begin{remark}
Associativity and commutativity are independent properties.
Addition on $\mathbb{Z}$ is both; matrix multiplication is associative
but not commutative; subtraction on $\mathbb{Z}$ is neither.
\end{remark}
% =========================================================
% Group Definition and Axioms
% =========================================================
\paragraph{Definition of a Group}

\begin{definition}[Group]
A \emph{group} is a pair $(G, \star)$ where $G$ is a set and
$\star$ is a binary operation on $G$ satisfying the following axioms:

\begin{enumerate}[label=\textbf{G\arabic*.}]
  \item \textbf{Associativity.}
        $(a \star b) \star c = a \star (b \star c)$
        for all $a, b, c \in G$.

  \item \textbf{Identity.}
        There exists an element $e \in G$ such that
        \[
          e \star a = a \star e = a
          \quad \text{for all } a \in G.
        \]
        The element $e$ is called the \emph{identity element} of $G$.

  \item \textbf{Inverses.}
        For each $a \in G$, there exists an element $a^{-1} \in G$ such that
        \[
          a \star a^{-1} = a^{-1} \star a = e.
        \]
        The element $a^{-1}$ is called the \emph{inverse} of $a$.
\end{enumerate}
\end{definition}

\begin{remark}
Closure is not listed as a separate axiom because it is already
encoded in the requirement that $\star : G \times G \to G$ is a
binary operation --- the codomain forces the result to stay in $G$.

Intuitively: a group is a set where you can combine elements,
undo combinations, and the order of grouping never matters.
\end{remark}

\begin{remark}[Axiom Count]
Some treatments list four axioms (closure, associativity, identity, inverses).
Here closure is absorbed into the definition of binary operation,
leaving three axioms. Both presentations define the same object.
\end{remark}

\begin{remark}[Notation]
When the operation is understood from context, we write $ab$ instead
of $a \star b$, and call $G$ itself a group rather than the pair $(G, \star)$.
For groups whose operation is addition, we write $a + b$, use $0$ for the
identity, and $-a$ for the inverse of $a$.
\end{remark}

\begin{definition}[Order of a Group]
The \emph{order} of a group $G$, denoted $|G|$, is the cardinality of $G$
as a set. If $|G|$ is finite, $G$ is called a \emph{finite group};
otherwise it is an \emph{infinite group}.
\end{definition}

\begin{example}[Canonical Examples of Groups]
\begin{enumerate}[label=(\roman*)]
  \item $(\mathbb{Z}, +)$: the integers under addition.
        Identity: $0$. Inverse of $n$: $-n$. Infinite group.

  \item $(\mathbb{Q} \setminus \{0\}, \cdot)$: nonzero rationals under multiplication.
        Identity: $1$. Inverse of $q$: $1/q$. Infinite group.

  \item $(\mathbb{Z}/n\mathbb{Z}, +)$: integers modulo $n$ under addition.
        Identity: $[0]$. Inverse of $[k]$: $[n-k]$. Finite group of order $n$.

  \item $(GL_n(\mathbb{R}), \cdot)$: invertible $n \times n$ real matrices
        under multiplication.
        Identity: $I_n$. Inverse: matrix inverse. Infinite group.
\end{enumerate}
\end{example}

\begin{remark}
Note what fails to be a group:
$(\mathbb{Z}, \cdot)$ is not a group because $2$ has no multiplicative
inverse in $\mathbb{Z}$.
$(\mathbb{N}, +)$ is not a group because positive integers have no
additive inverse in $\mathbb{N}$.
These failures illustrate why each axiom is necessary.
\end{remark}
% =========================================================
% Basic Theorems of Groups
% =========================================================
\paragraph{Basic Theorems}

\begin{remark}[Why These Theorems Matter]
The group axioms guarantee existence of an identity and inverses,
but say nothing about uniqueness. The following theorems establish
that both are unique. This is essential: without uniqueness, we
cannot speak of \emph{the} identity or \emph{the} inverse of an element,
and proofs that equate two objects via the identity or inverse would
be invalid.

These same uniqueness theorems reappear in vector space proofs,
where they are cited by name. They are proved once here.
\end{remark}

% ---------------------------------------------------------
\begin{proposition}[Uniqueness of the Identity]
\label{prop:group-identity-unique}
Let $G$ be a group. The identity element of $G$ is unique.
\end{proposition}


\begin{remark}
The proof strategy is standard for uniqueness arguments:
assume two identities exist, then show they must be equal.
This pattern recurs throughout algebra whenever a definition
asserts existence of a distinguished element.

Intuitively: if two elements both act as an identity,
applying one to the other forces them to coincide.
\end{remark}

% ---------------------------------------------------------
\begin{proposition}[Uniqueness of Inverses]
\label{prop:group-inverse-unique}
Let $G$ be a group. For each $a \in G$, the inverse of $a$ is unique.
\end{proposition}


\begin{remark}
Intuitively: if two elements both undo $a$, then they must be
the same element --- because each can be obtained from the other
by cancellation.
\end{remark}

% ---------------------------------------------------------
\begin{proposition}[Cancellation Laws]
\label{prop:group-cancellation}
Let $G$ be a group and let $a, b, c \in G$. Then:
\begin{enumerate}[label=(\roman*)]
  \item \textbf{Left cancellation:} $ab = ac \implies b = c$.
  \item \textbf{Right cancellation:} $ba = ca \implies b = c$.
\end{enumerate}
\end{proposition}


\begin{remark}
Cancellation is what makes group equations solvable.
It does \emph{not} hold in general for rings or monoids without inverses.

Intuitively: multiply both sides by $a^{-1}$ and the common factor disappears.
\end{remark}

% ---------------------------------------------------------
\begin{proposition}[Socks-Shoes Property]
\label{prop:group-socks-shoes}
Let $G$ be a group and let $a, b \in G$. Then
\[
(ab)^{-1} = b^{-1} a^{-1}.
\]
\end{proposition}


\begin{remark}
The name comes from the observation that to undo putting on socks
then shoes, you must first remove the shoes, then the socks ---
in reverse order.

This reversal of order is characteristic of non-abelian groups and
becomes important in the theory of group homomorphisms and
in matrix algebra, where $(AB)^{-1} = B^{-1}A^{-1}$.
\end{remark}
% =========================================================
% Abelian Groups
% =========================================================
\paragraph{Abelian Groups}

\begin{definition}[Abelian Group]
A group $(G, \star)$ is called \emph{abelian} (or \emph{commutative}) if
\[
a \star b = b \star a \quad \text{for all } a, b \in G.
\]
\end{definition}

\begin{remark}
An abelian group is a group with one additional axiom: commutativity
of the operation. Every abelian group is a group, but not every group
is abelian.

Intuitively: in an abelian group, the order in which you combine
elements is irrelevant. This is the familiar arithmetic of addition
on $\mathbb{Z}$, $\mathbb{Q}$, $\mathbb{R}$, and $\mathbb{C}$.
\end{remark}

\begin{remark}[Structural Position]
Abelian groups are the additive backbone of every vector space.
The four vector space axioms governing addition --- associativity,
commutativity, existence of zero, existence of additive inverses ---
are precisely the axioms that make $(V, +)$ an abelian group.

This is why the vector space definition can be stated compactly as:
\emph{a vector space over $\mathbb{F}$ is an abelian group $(V,+)$
equipped with a scalar multiplication by $\mathbb{F}$}.
The abelian group structure is not an analogy; it is the literal
algebraic content of the first four vector space axioms.
\end{remark}

\begin{example}[Abelian and Non-Abelian Groups]
\begin{enumerate}[label=(\roman*)]
  \item $(\mathbb{Z}, +)$, $(\mathbb{R}, +)$, $(\mathbb{C}, +)$:
        all abelian. These are the additive groups underlying the
        standard vector spaces.

  \item $(\mathbb{Z}/n\mathbb{Z}, +)$: abelian for all $n \geq 1$.

  \item $(GL_n(\mathbb{R}), \cdot)$ for $n \geq 2$: \emph{not} abelian,
        since matrix multiplication does not commute in general.

  \item $(S_n, \circ)$ for $n \geq 3$: the symmetric group on $n$ symbols
        under composition is not abelian.
\end{enumerate}
\end{example}

\begin{remark}[Additive Notation Convention]
For abelian groups, it is standard to write the operation as $+$,
the identity as $0$, and the inverse of $a$ as $-a$.
This additive notation is used throughout linear algebra, where
$(V, +)$ is always an abelian group.
\end{remark}













\clearpage
\section{Worksheets}
% worksheets.tex

% worksheets.tex

% =========================================================
% Worksheet: Abbott — Axiom of Completeness (Bounds / Sup / Inf)
% File: 01-real-analysis/worksheets/abbott.tex
% =========================================================

\subsection{Abbott}

\noindent\textbf{Source.} Stephen Abbott, \textit{Understanding Analysis} (2nd ed.).

\vspace{0.75em}
\begin{center}
\begin{tabular}{|p{5.0cm}|p{9.0cm}|}
\hline
\textbf{Problem ID} & \textbf{Exercise (descriptor)} \\
\hline

% -------------------------
% Chapter 1, §1.3 — The Axiom of Completeness
% -------------------------

\phantomsection
\hypertarget{ws-RA-ABB-C01-S1-3-E02A}{}
\hyperlink{proof-RA-ABB-C01-S1-3-E02A}{\texttt{RA-ABB-C01-S1-3-E02A}}
&
Abbott, Chapter 1, \S1-3, Ex.~1.3.2(a) — Example of $B$ with $\inf B \ge \sup B$ (or impossible).
\\ \hline

\phantomsection
\hypertarget{ws-RA-ABB-C01-S1-3-E02B}{}
\hyperlink{proof-RA-ABB-C01-S1-3-E02B}{\texttt{RA-ABB-C01-S1-3-E02B}}
&
Abbott, Chapter 1, \S1-3, Ex.~1.3.2(b) — Finite set containing $\inf$ but not $\sup$ (or impossible).
\\ \hline

\phantomsection
\hypertarget{ws-RA-ABB-C01-S1-3-E02C}{}
\hyperlink{proof-RA-ABB-C01-S1-3-E02C}{\texttt{RA-ABB-C01-S1-3-E02C}}
&
Abbott, Chapter 1, \S1-3, Ex.~1.3.2(c) — Bounded subset of $\mathbb{Q}$ containing $\sup$ but not $\inf$.
\\ \hline

\phantomsection
\hypertarget{ws-RA-ABB-C01-S1-3-E03A}{}
\hyperlink{proof-RA-ABB-C01-S1-3-E03A}{\texttt{RA-ABB-C01-S1-3-E03A}}
&
Abbott, Chapter 1, \S1-3, Ex.~1.3.3(a) — If $B=\{b\in\mathbb{R}: b\text{ lower bound for }A\}$, show $\sup B=\inf A$.
\\ \hline

\phantomsection
\hypertarget{ws-RA-ABB-C01-S1-3-E03B}{}
\hyperlink{proof-RA-ABB-C01-S1-3-E03B}{\texttt{RA-ABB-C01-S1-3-E03B}}
&
Abbott, Chapter 1, \S1-3, Ex.~1.3.3(b) — Explain why GLB existence need not be an axiom (derive from LUB).
\\ \hline

\phantomsection
\hypertarget{ws-RA-ABB-C01-S1-3-E04}{}
\hyperlink{proof-RA-ABB-C01-S1-3-E04}{\texttt{RA-ABB-C01-S1-3-E04}}
&
Abbott, Chapter 1, \S1-3, Ex.~1.3.4 — Suprema of unions: $\sup(A_1\cup A_2)$, $\sup(\bigcup_{k=1}^n A_k)$, and infinite case.
\\ \hline

\phantomsection
\hypertarget{ws-RA-ABB-C01-S1-3-E05A}{}
\hyperlink{proof-RA-ABB-C01-S1-3-E05A}{\texttt{RA-ABB-C01-S1-3-E05A}}
&
Abbott, Chapter 1, \S1-3, Ex.~1.3.5(a) — If $c\ge 0$, show $\sup(cA)=c\,\sup(A)$ for $cA=\{ca:a\in A\}$.
\\ \hline

\phantomsection
\hypertarget{ws-RA-ABB-C01-S1-3-E05B}{}
\hyperlink{proof-RA-ABB-C01-S1-3-E05B}{\texttt{RA-ABB-C01-S1-3-E05B}}
&
Abbott, Chapter 1, \S1-3, Ex.~1.3.5(b) — Postulate a formula for $\sup(cA)$ when $c<0$.
\\ \hline

\phantomsection
\hypertarget{ws-RA-ABB-C01-S1-3-E06A}{}
\hyperlink{proof-RA-ABB-C01-S1-3-E06A}{\texttt{RA-ABB-C01-S1-3-E06A}}
&
Abbott, Chapter 1, \S1-3, Ex.~1.3.6(a) — For $A+B=\{a+b:a\in A,b\in B\}$, show $s+t$ is an upper bound of $A+B$.
\\ \hline

\phantomsection
\hypertarget{ws-RA-ABB-C01-S1-3-E06B}{}
\hyperlink{proof-RA-ABB-C01-S1-3-E06B}{\texttt{RA-ABB-C01-S1-3-E06B}}
&
Abbott, Chapter 1, \S1-3, Ex.~1.3.6(b) — If $u$ upper bounds $A+B$ and $a\in A$ is fixed, show $t\le u-a$.
\\ \hline

\phantomsection
\hypertarget{ws-RA-ABB-C01-S1-3-E06C}{}
\hyperlink{proof-RA-ABB-C01-S1-3-E06C}{\texttt{RA-ABB-C01-S1-3-E06C}}
&
Abbott, Chapter 1, \S1-3, Ex.~1.3.6(c) — Conclude $\sup(A+B)=s+t$.
\\ \hline

\phantomsection
\hypertarget{ws-RA-ABB-C01-S1-3-E06D}{}
\hyperlink{proof-RA-ABB-C01-S1-3-E06D}{\texttt{RA-ABB-C01-S1-3-E06D}}
&
Abbott, Chapter 1, \S1-3, Ex.~1.3.6(d) — Re-prove $\sup(A+B)=\sup A+\sup B$ using Lemma~1.3.8.
\\ \hline

\phantomsection
\hypertarget{ws-RA-ABB-C01-S1-3-E07}{}
\hyperlink{proof-RA-ABB-C01-S1-3-E07}{\texttt{RA-ABB-C01-S1-3-E07}}
&
Abbott, Chapter 1, \S1-3, Ex.~1.3.7 — If $a$ is an upper bound for $A$ and $a\in A$, prove $a=\sup A$.
\\ \hline

\phantomsection
\hypertarget{ws-RA-ABB-C01-S1-3-E08A}{}
\hyperlink{proof-RA-ABB-C01-S1-3-E08A}{\texttt{RA-ABB-C01-S1-3-E08A}}
&
Abbott, Chapter 1, \S1-3, Ex.~1.3.8(a) — Compute $\sup/\inf$ of $\{m/n:m,n\in\mathbb{N},\ m<n\}$.
\\ \hline

\phantomsection
\hypertarget{ws-RA-ABB-C01-S1-3-E08B}{}
\hyperlink{proof-RA-ABB-C01-S1-3-E08B}{\texttt{RA-ABB-C01-S1-3-E08B}}
&
Abbott, Chapter 1, \S1-3, Ex.~1.3.8(b) — Compute $\sup/\inf$ of $\{(-1)^m/n:m,n\in\mathbb{N}\}$.
\\ \hline

\phantomsection
\hypertarget{ws-RA-ABB-C01-S1-3-E08C}{}
\hyperlink{proof-RA-ABB-C01-S1-3-E08C}{\texttt{RA-ABB-C01-S1-3-E08C}}
&
Abbott, Chapter 1, \S1-3, Ex.~1.3.8(c) — Compute $\sup/\inf$ of $\{n/(3n+1):n\in\mathbb{N}\}$.
\\ \hline

\phantomsection
\hypertarget{ws-RA-ABB-C01-S1-3-E08D}{}
\hyperlink{proof-RA-ABB-C01-S1-3-E08D}{\texttt{RA-ABB-C01-S1-3-E08D}}
&
Abbott, Chapter 1, \S1-3, Ex.~1.3.8(d) — Compute $\sup/\inf$ of $\{m/(m+n):m,n\in\mathbb{N}\}$.
\\ \hline

\phantomsection
\hypertarget{ws-RA-ABB-C01-S1-3-E09A}{}
\hyperlink{proof-RA-ABB-C01-S1-3-E09A}{\texttt{RA-ABB-C01-S1-3-E09A}}
&
Abbott, Chapter 1, \S1-3, Ex.~1.3.9(a) — If $\sup A<\sup B$, show $\exists b\in B$ that is an upper bound for $A$.
\\ \hline

\phantomsection
\hypertarget{ws-RA-ABB-C01-S1-3-E09B}{}
\hyperlink{proof-RA-ABB-C01-S1-3-E09B}{\texttt{RA-ABB-C01-S1-3-E09B}}
&
Abbott, Chapter 1, \S1-3, Ex.~1.3.9(b) — Counterexample if only $\sup A\le \sup B$ is assumed.
\\ \hline

\phantomsection
\hypertarget{ws-RA-ABB-C01-S1-3-E10A}{}
\hyperlink{proof-RA-ABB-C01-S1-3-E10A}{\texttt{RA-ABB-C01-S1-3-E10A}}
&
Abbott, Chapter 1, \S1-3, Ex.~1.3.10(a) — Use completeness to prove the Cut Property.
\\ \hline

\end{tabular}
\end{center}


% =========================================================
% Worksheet: Ross — Bounds / Sup / Inf / Density (Chapter 4)
% File: 05-real-line-foundations/worksheets/ross.tex
% =========================================================

\subsection{Ross}

\noindent\textbf{Source.} Kenneth A. Ross, \textit{Elementary Analysis: The Theory of Calculus} (2nd ed.).

\vspace{0.75em}
\begin{center}
\begin{tabular}{|p{5.0cm}|p{9.0cm}|}
\hline
\textbf{Problem ID} & \textbf{Exercise (descriptor)} \\
\hline

% -------------------------
% Chapter 4 — Suprema, Infima, and Density
% -------------------------

\phantomsection
\hypertarget{ws-RA-ROS-C04-E045}{}
\hyperlink{proof-RA-ROS-C04-E045}{\texttt{RA-ROS-C04-E045}}
&
Ross, Chapter 4, Ex.~4.5 — If $\sup S\in S$, prove $\sup S=\max S$ (for nonempty $S\subseteq\mathbb{R}$ bounded above).
\\ \hline

\phantomsection
\hypertarget{ws-RA-ROS-C04-E046A}{}
\hyperlink{proof-RA-ROS-C04-E046A}{\texttt{RA-ROS-C04-E046A}}
&
Ross, Chapter 4, Ex.~4.6(a) — For nonempty bounded $S\subseteq\mathbb{R}$, prove $\inf S \le \sup S$.
\\ \hline

\phantomsection
\hypertarget{ws-RA-ROS-C04-E046B}{}
\hyperlink{proof-RA-ROS-C04-E046B}{\texttt{RA-ROS-C04-E046B}}
&
Ross, Chapter 4, Ex.~4.6(b) — Describe $S$ if $\inf S=\sup S$.
\\ \hline

\phantomsection
\hypertarget{ws-RA-ROS-C04-E047A}{}
\hyperlink{proof-RA-ROS-C04-E047A}{\texttt{RA-ROS-C04-E047A}}
&
Ross, Chapter 4, Ex.~4.7(a) — If $S\subseteq T$, prove $\inf T \le \inf S \le \sup S \le \sup T$.
\\ \hline

\phantomsection
\hypertarget{ws-RA-ROS-C04-E047B}{}
\hyperlink{proof-RA-ROS-C04-E047B}{\texttt{RA-ROS-C04-E047B}}
&
Ross, Chapter 4, Ex.~4.7(b) — Prove $\sup(S\cup T)=\max\{\sup S,\sup T\}$ (do not assume $S\subseteq T$).
\\ \hline

\phantomsection
\hypertarget{ws-RA-ROS-C04-E048A}{}
\hyperlink{proof-RA-ROS-C04-E048A}{\texttt{RA-ROS-C04-E048A}}
&
Ross, Chapter 4, Ex.~4.8(a) — If $s\le t$ for all $s\in S$, $t\in T$, observe $S$ bounded above and $T$ bounded below.
\\ \hline

\phantomsection
\hypertarget{ws-RA-ROS-C04-E048B}{}
\hyperlink{proof-RA-ROS-C04-E048B}{\texttt{RA-ROS-C04-E048B}}
&
Ross, Chapter 4, Ex.~4.8(b) — Under Ex.~4.8 hypotheses, prove $\sup S \le \inf T$.
\\ \hline

\phantomsection
\hypertarget{ws-RA-ROS-C04-E048C}{}
\hyperlink{proof-RA-ROS-C04-E048C}{\texttt{RA-ROS-C04-E048C}}
&
Ross, Chapter 4, Ex.~4.8(c) — Give an example with $S\cap T\neq\varnothing$.
\\ \hline

\phantomsection
\hypertarget{ws-RA-ROS-C04-E048D}{}
\hyperlink{proof-RA-ROS-C04-E048D}{\texttt{RA-ROS-C04-E048D}}
&
Ross, Chapter 4, Ex.~4.8(d) — Give an example with $\sup S=\inf T$ and $S\cap T=\varnothing$.
\\ \hline

\phantomsection
\hypertarget{ws-RA-ROS-C04-E049}{}
\hyperlink{proof-RA-ROS-C04-E049}{\texttt{RA-ROS-C04-E049}}
&
Ross, Chapter 4, Ex.~4.9 — Complete the proof that $\inf S=-\sup(-S)$ (Cor.~4.5), by proving (1) and (2).
\\ \hline

\phantomsection
\hypertarget{ws-RA-ROS-C04-E0410}{}
\hyperlink{proof-RA-ROS-C04-E0410}{\texttt{RA-ROS-C04-E0410}}
&
Ross, Chapter 4, Ex.~4.10 — If $a>0$, show $\exists n\in\mathbb{N}$ with $\frac1n<a<n$.
\\ \hline

\phantomsection
\hypertarget{ws-RA-ROS-C04-E0411}{}
\hyperlink{proof-RA-ROS-C04-E0411}{\texttt{RA-ROS-C04-E0411}}
&
Ross, Chapter 4, Ex.~4.11 — If $a<b$, use density of $\mathbb{Q}$ to show infinitely many rationals lie between $a$ and $b$.
\\ \hline

\phantomsection
\hypertarget{ws-RA-ROS-C04-E0412}{}
\hyperlink{proof-RA-ROS-C04-E0412}{\texttt{RA-ROS-C04-E0412}}
&
Ross, Chapter 4, Ex.~4.12 — If $a<b$, prove $\exists x\in\mathbb{I}$ with $a<x<b$ (irrationals; hint uses $\{r+\sqrt2:r\in\mathbb{Q}\}$).
\\ \hline

\end{tabular}
\end{center}


% =========================================================
% Worksheet: Johar — Sequences and Limits
% File: real-line-foundations/worksheets/johar.tex
% =========================================================

\subsection{Johar}

\noindent\textbf{Source.} Syafiq Johar, \textit{The Big Book of Real Analysis}.

\vspace{0.75em}
\begin{center}
\begin{tabular}{|p{5.0cm}|p{9.0cm}|}
\hline
\textbf{Problem ID} & \textbf{Exercise (descriptor)} \\
\hline

% -------------------------------------------------
% Chapter 5 — Exercise 5-1
% -------------------------------------------------

\phantomsection
\hypertarget{ws-RA-JOH-C05-S5-1-E01A}{}
\hyperlink{proof-RA-JOH-C05-S5-1-E01A}{\texttt{RA-JOH-C05-S5-1-E01A}}
&
Ex.~5.1(a) — Show $a_n\to 0$ for $a_n=\dfrac{1}{n^2+3}$.
\\ \hline

\phantomsection
\hypertarget{ws-RA-JOH-C05-S5-1-E01B}{}
\hyperlink{proof-RA-JOH-C05-S5-1-E01B}{\texttt{RA-JOH-C05-S5-1-E01B}}
&
Ex.~5.1(b) — Show $a_n\to 0$ for $a_n=\dfrac{1}{n-\frac{5}{2}}$.
\\ \hline

\phantomsection
\hypertarget{ws-RA-JOH-C05-S5-1-E01C}{}
\hyperlink{proof-RA-JOH-C05-S5-1-E01C}{\texttt{RA-JOH-C05-S5-1-E01C}}
&
Ex.~5.1(c) — Show $a_n\to 0$ for $a_n=\dfrac{1}{n\left(n-\frac{1}{2}\right)}$.
\\ \hline

\phantomsection
\hypertarget{ws-RA-JOH-C05-S5-1-E01D}{}
\hyperlink{proof-RA-JOH-C05-S5-1-E01D}{\texttt{RA-JOH-C05-S5-1-E01D}}
&
Ex.~5.1(d) — Show $a_n\to 0$ for $a_n=\dfrac{1}{\sqrt{5n}-1}$.
\\ \hline

\phantomsection
\hypertarget{ws-RA-JOH-C05-S5-1-E01E}{}
\hyperlink{proof-RA-JOH-C05-S5-1-E01E}{\texttt{RA-JOH-C05-S5-1-E01E}}
&
Ex.~5.1(e) — Show $a_n\to 0$ for $a_n=\dfrac{\sin(n)}{n}$.
\\ \hline

\phantomsection
\hypertarget{ws-RA-JOH-C05-S5-1-E01F}{}
\hyperlink{proof-RA-JOH-C05-S5-1-E01F}{\texttt{RA-JOH-C05-S5-1-E01F}}
&
Ex.~5.1(f) — Prime/non-prime definition.
\\ \hline

\phantomsection
\hypertarget{ws-RA-JOH-C05-S5-1-E01G}{}
\hyperlink{proof-RA-JOH-C05-S5-1-E01G}{\texttt{RA-JOH-C05-S5-1-E01G}}
&
Ex.~5.1(g) — $a_n=\sqrt{n+1}-\sqrt{n}$.
\\ \hline

\phantomsection
\hypertarget{ws-RA-JOH-C05-S5-1-E01H}{}
\hyperlink{proof-RA-JOH-C05-S5-1-E01H}{\texttt{RA-JOH-C05-S5-1-E01H}}
&
Ex.~5.1(h) — $a_n=n-\sqrt{n^2+\sqrt{n}}$.
\\ \hline


% -------------------------------------------------
% Additional Chapter 5 Exercises
% -------------------------------------------------

\phantomsection
\hypertarget{ws-RA-JOH-C05-S5-7}{}
\hyperlink{proof-RA-JOH-C05-S5-7}{\texttt{RA-JOH-C05-S5-7}}
&
Ex.~5.7 — Convergence from even/odd and modular subsequences.
\\ \hline

\phantomsection
\hypertarget{ws-RA-JOH-C05-S5-8}{}
\hyperlink{proof-RA-JOH-C05-S5-8}{\texttt{RA-JOH-C05-S5-8}}
&
Ex.~5.8 — $|a_n-b_n|\to 0$ and counterexample.
\\ \hline

\phantomsection
\hypertarget{ws-RA-JOH-C05-S5-17}{}
\hyperlink{proof-RA-JOH-C05-S5-17}{\texttt{RA-JOH-C05-S5-17}}
&
Ex.~5.17 — Limits of $r^{1/n}$ and related sequences.
\\ \hline

\phantomsection
\hypertarget{ws-RA-JOH-C05-S5-18}{}
\hyperlink{proof-RA-JOH-C05-S5-18}{\texttt{RA-JOH-C05-S5-18}}
&
Ex.~5.18 — Applications of $n^{1/n}\to 1$.
\\ \hline

\phantomsection
\hypertarget{ws-RA-JOH-C05-S5-34}{}
\hyperlink{proof-RA-JOH-C05-S5-34}{\texttt{RA-JOH-C05-S5-34}}
&
Ex.~5.34 — limsup and liminf algebra rules.
\\ \hline

\end{tabular}
\end{center}



\section{Proofs}algebraic-structures
% =========================================================
% Integers — Notes
% =========================================================

% =========================================================
% The Integers — Chapter Header
% (Drop this at the top of integers/notes/index.tex,
%  before \subsubsection{Tao Construction})
% =========================================================

% ---------------------------------------------------------
% "Where You Are in the Journey" breadcrumb box
% ---------------------------------------------------------
\begin{tcolorbox}[colback=gray!6, colframe=gray!50, arc=2pt,
  left=6pt, right=6pt, top=4pt, bottom=4pt,
  title={\small\textbf{Where You Are in the Journey}},
  fonttitle=\small\bfseries]
\small
Propositional Logic $\to$ Predicate Calculus $\to$ Sets \& Functions $\to$
Proof Techniques $\to$ Axiom Systems \& Natural Numbers $\to$
\textbf{The Integers} $\to$ Rationals \& Ordered Fields
$\to$ Real Line Foundations $\to$ \(\cdots\)

\medskip
\textbf{How we got here.}
The natural numbers gave us addition, multiplication, and a well-founded
order, but subtraction is not available: for $a < b$ in $\mathbb{N}$,
the expression $a - b$ is meaningless.
Every subsequent number system depends on fixing this, so we pause here
to do it rigorously.

\medskip
\textbf{What this chapter builds.}
We follow two complementary sources.
Tao's \textit{Analysis~I}, Chapter~4 constructs $\mathbb{Z}$ from
$\mathbb{N}\times\mathbb{N}$ via formal pairs and proves all arithmetic
properties directly.
Mendelson's \textit{Number Systems and the Foundations of Analysis},
Chapters~3--4 performs the same construction but frames the result in
the language of abstract algebra: rings, integral domains, and ordered
integral domains.
Together they give both the computational fluency and the structural
vocabulary that real analysis and abstract algebra require.

\medskip
\textbf{Where this leads.}
The ring and integral-domain structure established here is not special
to $\mathbb{Z}$: the same axioms, the same order properties, and the
same absolute-value toolkit reappear verbatim for $\mathbb{Q}$ and
$\mathbb{R}$.
Everything algebraic built later inherits the patterns established here.
\end{tcolorbox}

\bigskip

% ---------------------------------------------------------
% Structural Roadmap
% ---------------------------------------------------------
\noindent\textbf{Structural Roadmap}

\medskip
\noindent
Both sources follow the same logical spine:

\begin{center}
\textbf{Equivalence Relation}
$\longrightarrow$ \textbf{Definitions}
$\longrightarrow$ \textbf{Well-Definedness}
$\longrightarrow$ \textbf{Ring Laws}
$\longrightarrow$ \textbf{Order}
$\longrightarrow$ \textbf{Absolute Value}
\end{center}

\noindent
Nothing is assumed beyond what has been proved for $\mathbb{N}$.
The available toolkit at each step is exactly what precedes it ---
most critically, the cancellation law for natural-number addition, which
carries the entire construction forward.

\medskip
\noindent The global progression is:

\begin{enumerate}
  \item \textbf{The equivalence relation.}
        Integers are equivalence classes of pairs $(a,b)\in\mathbb{N}
        \times\mathbb{N}$ under $(a,b)\sim(c,d)\iff a+d=c+b$.
        Transitivity requires cancellation in $\mathbb{N}$.

  \item \textbf{Addition and multiplication.}
        Defined on representatives; well-definedness verified separately
        for each operation (Tao L4.1.3; Mendelson Lemmas 2.1, 2.3).

  \item \textbf{Ring laws.}
        The nine identities of Tao's P4.1.6 (= Mendelson Thms 2.2,~2.4)
        are exactly the axioms of a commutative ring with unit.
        All proofs expand pairs and reduce to arithmetic in $\mathbb{N}$.

  \item \textbf{Integral domain.}
        No zero divisors (P4.1.8 / Thm~2.4(v)) and the cancellation law
        (C4.1.9 / Thm~3.3) elevate $\mathbb{Z}$ from a ring to an
        integral domain.
        Mendelson's \S3.3 names and defines these structures abstractly.

  \item \textbf{Order.}
        Tao defines order directly via $\mathbb{N}$-differences
        (Def~4.1.10).
        Mendelson axiomatises an ordered integral domain (Def~4.1,
        axioms O1--O5) and instantiates it for $\mathbb{Z}$ via a
        positivity set (Thm~4.3, Cor~4.6).
        The same five axioms recur unchanged for $\mathbb{R}$.

  \item \textbf{Absolute value.}
        Mendelson's Theorem~4.8 (11~parts, for any ordered integral
        domain) provides a permanent toolkit ---
        triangle inequality, reverse triangle inequality, multiplicativity
        --- that requires no re-proof when we reach $\mathbb{Q}$ and
        $\mathbb{R}$.
        Tao defers absolute value to the rational chapter.

  \item \textbf{Induction boundary.}
        Induction fails for $\mathbb{Z}$ (Tao Ex~4.1.8): there is no
        smallest element to anchor a descent below $0$.
        Strong induction on $\mathbb{N}$ remains available for the
        positive integers inside $\mathbb{Z}$; Mendelson Thm~4.7 recovers
        a Peano system from $\mathcal{P}_{\mathbb{Z}}$ to make this
        precise.
\end{enumerate}

\subsubsection{Tao Construction}
% =========================================================
% The Integers — Tao Construction (Analysis I, §4.1)
% =========================================================

% ---------------------------------------------------------
% Motivation
% ---------------------------------------------------------
\begin{remark}[Why the integers are needed]
The natural number system $\mathbb{N}$ with addition and multiplication
has reached its limits: subtraction is not available.
Given $a, b \in \mathbb{N}$ with $a < b$, the expression $a - b$ has
no value in $\mathbb{N}$. To fix this, we pass to a larger system $\mathbb{Z}$
in which every natural number has an additive inverse.
\end{remark}

\begin{remark}[The circularity problem]
The naive approach — define an integer as a difference $a - b$ of two
natural numbers — is circular, because the symbol $-$ is exactly what we
are trying to construct.
Tao resolves this by introducing a \textbf{placeholder symbol} $--$
with no arithmetic meaning, using $a {-\!\!-} b$ purely as notation for a
formal pair $(a, b)$.
Once subtraction is defined at the end of the construction, we verify that
$a {-\!\!-} b$ coincides with $a - b$ and discard the scaffolding.
\end{remark}

\begin{remark}[Why not define integers as signed naturals?]
One might define an integer as either a positive natural number, zero,
or the negation of a positive natural number.
Lemma~4.1.5 (trichotomy) shows this classification is correct, but
\emph{using it as the definition} forces case-splits for every arithmetic
operation: negative $\times$ negative, positive $+$ negative of different
sizes, etc. The verification of the ring laws becomes enormously messy.
The equivalence-class construction pays one upfront cost and avoids
all subsequent case-explosion.
\end{remark}

% ---------------------------------------------------------
% Definition 4.1.1 — Integers
% ---------------------------------------------------------
\begin{tcolorbox}[colback=propbox, colframe=propborder, arc=2pt,
  left=6pt, right=6pt, top=4pt, bottom=4pt,
  title={\small\textbf{Definition 4.1.1 (Integers)}},
  fonttitle=\small\bfseries]
An \textbf{integer} is an expression of the form $a {-\!\!-} b$, where
$a$ and $b$ are natural numbers.

Two integers are \textbf{equal}:
\[
a {-\!\!-} b \;=\; c {-\!\!-} d
\quad\iff\quad
a + d \;=\; c + b.
\]
The set of all integers is denoted $\mathbb{Z}$.
\end{tcolorbox}

\begin{remark}[Reading the equality condition]
The condition $a + d = c + b$ is the cross-addition criterion that would
hold for genuine differences: $a - b = c - d \iff a + d = c + b$.
It uses only addition in $\mathbb{N}$, which is already available,
and avoids any reference to subtraction.
\end{remark}

\begin{remark}[Equality must be verified as a legitimate equivalence relation]
Four axioms must be checked: reflexivity, symmetry, transitivity,
and substitution.
Reflexivity and symmetry are immediate from the definition.
Transitivity requires the cancellation law for natural numbers
(Proposition~2.2.6): the key load-bearing result from Chapter~2.
Substitution cannot be verified until operations on $\mathbb{Z}$ are
defined, and must be re-checked for each operation.
\end{remark}

% ---------------------------------------------------------
% Definition 4.1.2 — Addition and Multiplication
% ---------------------------------------------------------
\begin{tcolorbox}[colback=propbox, colframe=propborder, arc=2pt,
  left=6pt, right=6pt, top=4pt, bottom=4pt,
  title={\small\textbf{Definition 4.1.2 (Addition and Multiplication)}},
  fonttitle=\small\bfseries]
\[
(a {-\!\!-} b) + (c {-\!\!-} d) \;:=\; (a+c) {-\!\!-} (b+d)
\]
\[
(a {-\!\!-} b) \times (c {-\!\!-} d) \;:=\; (ac+bd) {-\!\!-} (ad+bc)
\]
\end{tcolorbox}

\begin{remark}[Motivation for the multiplication formula]
Expanding the product of genuine differences:
$(a-b)(c-d) = ac - ad - bc + bd = (ac+bd) - (ad+bc)$.
The formula is forced by this foreknowledge.
\end{remark}

% ---------------------------------------------------------
% Lemma 4.1.3 — Well-definedness
% ---------------------------------------------------------
\begin{lemma}[Tao 4.1.3 --- Addition and multiplication are well-defined]
\label{lem:int-well-defined}
Let $a, b, a', b', c, d \in \mathbb{N}$.
If $(a {-\!\!-} b) = (a' {-\!\!-} b')$, then:
\begin{align*}
(a {-\!\!-} b) + (c {-\!\!-} d) &= (a' {-\!\!-} b') + (c {-\!\!-} d), \\
(a {-\!\!-} b) \times (c {-\!\!-} d) &= (a' {-\!\!-} b') \times (c {-\!\!-} d),
\end{align*}
and similarly when replacing $(c {-\!\!-} d)$ by an equal integer.
\end{lemma}

\begin{remark}[Proof]
See \hyperref[prf:int-well-defined]{Proof $\to$ INT-TAO-L4.1.3 (Well-Definedness of Integer Arithmetic)}.
\end{remark}

\begin{remark}[Why well-definedness must be checked]
The operations are defined in terms of representatives $(a, b)$ of an
equivalence class.
If equal inputs (i.e., equal integers with different representatives)
gave different outputs, the operation would not be a function on
$\mathbb{Z}$ — it would be a function on representations, which is
meaningless.
This check is the substitution axiom for each operation.
\end{remark}

% ---------------------------------------------------------
% Embedding N in Z
% ---------------------------------------------------------
\begin{remark}[Embedding $\mathbb{N}$ in $\mathbb{Z}$]
The integers of the form $n {-\!\!-} 0$ behave identically to the
natural numbers:
\[
(n {-\!\!-} 0) + (m {-\!\!-} 0) = (n+m) {-\!\!-} 0,
\qquad
(n {-\!\!-} 0) \times (m {-\!\!-} 0) = nm {-\!\!-} 0.
\]
Furthermore $(n {-\!\!-} 0) = (m {-\!\!-} 0)$ if and only if $n = m$.
We therefore identify $n \equiv n {-\!\!-} 0$, embedding $\mathbb{N}$
into $\mathbb{Z}$.
In particular $0 = 0 {-\!\!-} 0$ and $1 = 1 {-\!\!-} 0$.
\end{remark}

% ---------------------------------------------------------
% Definition 4.1.4 — Negation
% ---------------------------------------------------------
\begin{tcolorbox}[colback=propbox, colframe=propborder, arc=2pt,
  left=6pt, right=6pt, top=4pt, bottom=4pt,
  title={\small\textbf{Definition 4.1.4 (Negation)}},
  fonttitle=\small\bfseries]
$-(a {-\!\!-} b) \;:=\; b {-\!\!-} a$.

In particular, for a positive natural number $n = n {-\!\!-} 0$:
\[
-n \;:=\; 0 {-\!\!-} n.
\]
\end{tcolorbox}

\begin{remark}[Well-definedness of negation]
Negation must also be checked as well-defined: if
$(a {-\!\!-} b) = (a' {-\!\!-} b')$ then $-(a {-\!\!-} b) = -(a' {-\!\!-} b')$.
This is Exercise~4.1.2.
\end{remark}

% ---------------------------------------------------------
% Lemma 4.1.5 --- Trichotomy of integers
% ---------------------------------------------------------
\begin{lemma}[Tao 4.1.5 --- Trichotomy of integers]
\label{lem:int-trichotomy}
Let $x$ be an integer. Then exactly one of the following holds:
\begin{enumerate}
  \item[(a)] $x = 0$.
  \item[(b)] $x = n$ for some positive natural number $n$.
  \item[(c)] $x = -n$ for some positive natural number $n$.
\end{enumerate}
If (b) holds we call $x$ a \textbf{positive integer};
if (c) holds we call $x$ a \textbf{negative integer}.
\end{lemma}

\begin{remark}[Proof]
See \hyperref[prf:int-trichotomy]{Proof $\to$ INT-TAO-L4.1.5 (Trichotomy of Integers)}.
\end{remark}

\begin{remark}[Proof sketch]
Existence of one case: $x = a {-\!\!-} b$.
By trichotomy of $\mathbb{N}$ (P2.2.13), either $a > b$, $a = b$,
or $a < b$.
Each case yields (b), (a), or (c) respectively.
Mutual exclusion uses P2.2.8 and Proposition~2.2.6.
\end{remark}

% ---------------------------------------------------------
% Proposition 4.1.6 — Laws of algebra
% ---------------------------------------------------------
\begin{tcolorbox}[colback=propbox, colframe=propborder, arc=2pt,
  left=6pt, right=6pt, top=4pt, bottom=4pt,
  title={\small\textbf{Proposition 4.1.6 (Laws of Algebra for $\mathbb{Z}$)}},
  fonttitle=\small\bfseries]
Let $x, y, z \in \mathbb{Z}$. Then:
\begin{align*}
&x + y = y + x &\quad& \text{(commutativity of addition)}\\
&(x+y)+z = x+(y+z) &\quad& \text{(associativity of addition)}\\
&x + 0 = 0 + x = x &\quad& \text{(additive identity)}\\
&x + (-x) = (-x) + x = 0 &\quad& \text{(additive inverse)}\\
&xy = yx &\quad& \text{(commutativity of multiplication)}\\
&(xy)z = x(yz) &\quad& \text{(associativity of multiplication)}\\
&x \cdot 1 = 1 \cdot x = x &\quad& \text{(multiplicative identity)}\\
&x(y+z) = xy + xz &\quad& \text{(left distributive law)}\\
&(y+z)x = yx + zx &\quad& \text{(right distributive law)}
\end{align*}
\end{tcolorbox}

\begin{remark}[This is the commutative ring axioms]
These nine identities are exactly the definition of a
\textbf{commutative ring}.
Without the identity $xy = yx$ the remaining eight would give a ring.
Note: these properties were proved for $\mathbb{N}$, but
$\mathbb{Z} \supsetneq \mathbb{N}$, so the proofs must be redone.
\end{remark}

\begin{remark}[Proof strategy]
Write $x = (a {-\!\!-} b)$, $y = (c {-\!\!-} d)$, $z = (e {-\!\!-} f)$
and expand both sides in terms of natural number arithmetic.
This is far cleaner than case-splitting on sign via Lemma~4.1.5.
Tao demonstrates the method on associativity of multiplication.
\end{remark}

% ---------------------------------------------------------
% Subtraction
% ---------------------------------------------------------
\begin{definition}[Subtraction]
\label{def:int-subtraction}
For integers $x, y$:
\[
x - y \;:=\; x + (-y).
\]
\end{definition}

\begin{remark}[Recovering the $--$ notation]
Once subtraction is defined, one checks that for natural numbers
$a, b$:
\[
a - b = (a {-\!\!-} 0) + (0 {-\!\!-} b) = a {-\!\!-} b.
\]
The placeholder $--$ is now equal to genuine subtraction; the
scaffolding is removed and $--$ is discarded.
\end{remark}

% ---------------------------------------------------------
% Proposition 4.1.8 — No zero divisors
% ---------------------------------------------------------
\begin{proposition}[Tao 4.1.8 — No zero divisors]
\label{prop:int-no-zero-div}
Let $a, b \in \mathbb{Z}$ with $ab = 0$. Then $a = 0$ or $b = 0$ (or both).
\end{proposition}

% ---------------------------------------------------------
% Corollary 4.1.9 — Cancellation law
% ---------------------------------------------------------
\begin{corollary}[Tao 4.1.9 — Cancellation law for $\mathbb{Z}$]
\label{cor:int-cancel}
Let $a, b, c \in \mathbb{Z}$ with $ac = bc$ and $c \neq 0$. Then $a = b$.
\end{corollary}

% ---------------------------------------------------------
% Definition 4.1.10 — Order
% ---------------------------------------------------------
\begin{tcolorbox}[colback=propbox, colframe=propborder, arc=2pt,
  left=6pt, right=6pt, top=4pt, bottom=4pt,
  title={\small\textbf{Definition 4.1.10 (Ordering of $\mathbb{Z}$)}},
  fonttitle=\small\bfseries]
Let $n, m \in \mathbb{Z}$.
\[
n \geq m \quad\iff\quad n = m + a \text{ for some } a \in \mathbb{N}.
\]
\[
n > m \quad\iff\quad n \geq m \text{ and } n \neq m.
\]
\end{tcolorbox}

\begin{remark}[Consistency with $\mathbb{N}$]
This definition is verbatim the same as Definition~2.2.11 for $\mathbb{N}$.
Since the embedding $n \equiv n {-\!\!-} 0$ is consistent with addition,
the two orderings agree on natural numbers.
\end{remark}

% ---------------------------------------------------------
% Lemma 4.1.11 — Properties of order
% ---------------------------------------------------------
\begin{lemma}[Tao 4.1.11 — Properties of order on $\mathbb{Z}$]
\label{lem:int-order}
Let $a, b, c \in \mathbb{Z}$. Then:
\begin{enumerate}
  \item[(a)] $a > b$ if and only if $a - b$ is a positive natural number.
  \item[(b)] \textbf{(Addition preserves order.)} If $a > b$, then $a + c > b + c$.
  \item[(c)] \textbf{(Positive multiplication preserves order.)}
    If $a > b$ and $c$ is a positive integer, then $ac > bc$.
  \item[(d)] \textbf{(Negation reverses order.)} If $a > b$, then $-a < -b$.
  \item[(e)] \textbf{(Transitivity.)} If $a > b$ and $b > c$, then $a > c$.
  \item[(f)] \textbf{(Trichotomy.)} Exactly one of $a > b$,\ $a < b$,\ $a = b$ holds.
\end{enumerate}
\end{lemma}

\begin{remark}[Strategy: derive from part (a)]
Part (a) reformulates $>$ in terms of positivity.
Parts (b)--(f) all follow from (a) by translating into statements about
positive natural numbers, where the analogous results are already known.
\end{remark}

% ---------------------------------------------------------
% Exercise 4.1.8 — Induction fails for Z
% ---------------------------------------------------------
\begin{remark}[Exercise 4.1.8 — Induction does not apply to $\mathbb{Z}$]
Axiom P5 (induction) does not carry over to the integers.
Specifically: there exists a property $P(n)$ of integers such that
$P(0)$ is true and $P(n) \Rightarrow P(n\pp)$ for all $n \in \mathbb{Z}$,
yet $P(n)$ fails for some $n \in \mathbb{Z}$.
This is because $\mathbb{Z}$ has no smallest element — there is nothing
to anchor a descent below $0$.
The situation becomes worse for $\mathbb{Q}$ and $\mathbb{R}$.
\end{remark}

% ---------------------------------------------------------
% Toolkit reference table
% ---------------------------------------------------------
\begin{tcolorbox}[colback=gray!6, colframe=gray!40, arc=2pt,
  left=6pt, right=6pt, top=4pt, bottom=4pt,
  title={\small\textbf{Integer Toolkit — Tao §4.1 Quick Reference}},
  fonttitle=\small\bfseries]
\small
\begin{tabular}{l l l}
\toprule
\textbf{Label} & \textbf{Statement} & \textbf{Proof method} \\
\midrule
Def~4.1.1   & $a{-\!\!-}b = c{-\!\!-}d \iff a+d=c+b$ & Definition \\
Def~4.1.2   & Addition and multiplication formulas & Definition \\
L4.1.3      & Operations are well-defined & Expand; use $a+b'=a'+b$ \\
Def~4.1.4   & $-(a{-\!\!-}b) := b{-\!\!-}a$ & Definition \\
L4.1.5      & Trichotomy of $\mathbb{Z}$ & Cases $a>b$, $a=b$, $a<b$ \\
P4.1.6      & Nine ring laws & Expand via representatives \\
Def~4.1.7   & $x - y := x + (-y)$ & Definition \\
P4.1.8      & No zero divisors & Uses L2.3.3 \\
C4.1.9      & Cancellation: $ac=bc, c\neq 0 \Rightarrow a=b$ & Uses P4.1.8 or C2.3.7+L4.1.5 \\
Def~4.1.10  & Order: $n \geq m \iff n = m+a$, $a \in \mathbb{N}$ & Definition \\
L4.1.11     & Six order properties & Use part (a) as bridge \\
\bottomrule
\end{tabular}
\end{tcolorbox}


\subsubsection{Mendelson Construction}
% =========================================================
% The Integers — Mendelson Construction
% Source: Mendelson, \textit{Number Systems and the
%   Foundations of Analysis}, Chapters 3--4 (core);
%   Chapters 5--9 (extended theory).
% =========================================================

% ---------------------------------------------------------
% §3.1  The equivalence relation on P × P
% ---------------------------------------------------------
\subsubsection*{§3.1 \quad The Equivalence Relation on $P \times P$}

\begin{remark}[Starting point]
Mendelson begins from $P \times P$, the set of all ordered pairs of
positive integers (his notation for $\mathbb{N} \setminus \{0\}$).
Tao uses $\mathbb{N} \times \mathbb{N}$ (including zero).
The idea is identical: a pair $(n,j)$ represents the
``formal difference'' $n - j$.
\end{remark}

\begin{tcolorbox}[colback=propbox, colframe=propborder, arc=2pt,
  left=6pt, right=6pt, top=4pt, bottom=4pt,
  title={\small\textbf{Definition 3.1 (Equivalence relation on $P \times P$)}},
  fonttitle=\small\bfseries]
For natural numbers $n, k, j, i$, define a relation $\sim$ on
ordered pairs by
\[
(n,j) \sim (k,i)
\quad\Longleftrightarrow\quad
n + i = k + j.
\]
\end{tcolorbox}

\begin{theorem}[Mendelson 1.1 --- $\sim$ is an equivalence relation]
\label{thm:men-equiv}
For all natural numbers $h,i,j,k,m,n$:
\begin{enumerate}
  \item[\normalfont(R)] $(h,i) \sim (h,i)$ \hfill\textit{(Reflexivity)}
  \item[\normalfont(S)] $(h,i) \sim (j,k) \Rightarrow (j,k) \sim (h,i)$
    \hfill\textit{(Symmetry)}
  \item[\normalfont(T)] $\bigl[(h,i)\sim(j,k)\wedge(j,k)\sim(m,n)\bigr]
    \Rightarrow (h,i)\sim(m,n)$ \hfill\textit{(Transitivity)}
\end{enumerate}
\end{theorem}

\begin{remark}[Transitivity uses cancellation]
As in Tao, transitivity of $\sim$ requires the cancellation law for
addition in $\mathbb{N}$.
This is the same load-bearing step in both constructions.
\end{remark}

\begin{tcolorbox}[colback=propbox, colframe=propborder, arc=2pt,
  left=6pt, right=6pt, top=4pt, bottom=4pt,
  title={\small\textbf{Definition 3.2 (The integers $\mathbb{Z}$)}},
  fonttitle=\small\bfseries]
$\mathbb{Z}$ is the set of all equivalence classes of $P \times P$
under $\sim$.  Elements of $\mathbb{Z}$ are called \textbf{integers}.

Distinguished elements:
$0_{\mathbb{Z}} = [(1,1)]$, \quad $1_{\mathbb{Z}} = [(2,1)]$.
\end{tcolorbox}

\begin{remark}[Notation comparison]
Mendelson writes $[(n,j)]$ for an equivalence class;
Tao writes $a {-\!\!-} b$ for the same object.
Mendelson's notation makes the set-theoretic content explicit.
Tao's notation is cleaner for computation.
\end{remark}

% ---------------------------------------------------------
% §3.2  Addition and Multiplication
% ---------------------------------------------------------
\subsubsection*{§3.2 \quad Addition and Multiplication}

\begin{lemma}[Mendelson 2.1 --- Addition well-defined]
\label{lem:men-add-welld}
If $(n,j) \sim (n_1,j_1)$ and $(k,i) \sim (k_1,i_1)$, then
$(n+k,\; j+i) \sim (n_1+k_1,\; j_1+i_1)$.
\end{lemma}

\begin{tcolorbox}[colback=propbox, colframe=propborder, arc=2pt,
  left=6pt, right=6pt, top=4pt, bottom=4pt,
  title={\small\textbf{Definition 3.3 (Addition)}},
  fonttitle=\small\bfseries]
For $\alpha, \beta \in \mathbb{Z}$ with representatives
$(n,j) \in \alpha$ and $(k,i) \in \beta$:
\[
\alpha +_{\mathbb{Z}} \beta = [(n+k,\; j+i)].
\]
Lemma~2.1 guarantees independence of representatives.
\end{tcolorbox}

\begin{theorem}[Mendelson 2.2 --- Properties of addition]
\label{thm:men-add}
For all $\alpha, \beta, \gamma \in \mathbb{Z}$:
\begin{enumerate}
  \item[(i)] Commutativity: $\alpha +_{\mathbb{Z}} \beta
    = \beta +_{\mathbb{Z}} \alpha$.
  \item[(ii)] Associativity: $\alpha +_{\mathbb{Z}}
    (\beta +_{\mathbb{Z}} \gamma)
    = (\alpha +_{\mathbb{Z}} \beta) +_{\mathbb{Z}} \gamma$.
  \item[(iii)] Additive identity: $\alpha +_{\mathbb{Z}} 0_{\mathbb{Z}}
    = \alpha$.
  \item[(iv)] Unique additive inverse: $\exists!\,\delta \in \mathbb{Z}$
    such that $\alpha +_{\mathbb{Z}} \delta = 0_{\mathbb{Z}}$.
\end{enumerate}
\end{theorem}

\begin{remark}[Uniqueness of the inverse]
Part (iv) asserts both existence \emph{and} uniqueness of $-\alpha$.
Tao's P4.1.6 asserts existence only; uniqueness follows later from
C4.1.9.  Mendelson builds uniqueness into the statement.
\end{remark}

\begin{lemma}[Mendelson 2.3 --- Multiplication well-defined]
\label{lem:men-mul-welld}
If $(n,j) \sim (n_1,j_1)$ and $(k,i) \sim (k_1,i_1)$, then
$(nk+ji,\; jk+ni) \sim (n_1k_1+j_1i_1,\; j_1k_1+n_1i_1)$.
\end{lemma}

\begin{tcolorbox}[colback=propbox, colframe=propborder, arc=2pt,
  left=6pt, right=6pt, top=4pt, bottom=4pt,
  title={\small\textbf{Definition 3.4 (Multiplication)}},
  fonttitle=\small\bfseries]
For $\alpha, \beta \in \mathbb{Z}$ with representatives
$(n,j) \in \alpha$ and $(k,i) \in \beta$:
\[
\alpha \times_{\mathbb{Z}} \beta = [(nk+ji,\; jk+ni)].
\]
Lemma~2.3 guarantees well-definedness.
\end{tcolorbox}

\begin{theorem}[Mendelson 2.4 --- Properties of multiplication]
\label{thm:men-mul}
For all $\alpha, \beta, \gamma \in \mathbb{Z}$:
\begin{enumerate}
  \item[(i)] Commutativity: $\alpha \times_{\mathbb{Z}} \beta
    = \beta \times_{\mathbb{Z}} \alpha$.
  \item[(ii)] Associativity: $\alpha \times_{\mathbb{Z}}
    (\beta \times_{\mathbb{Z}} \gamma)
    = (\alpha \times_{\mathbb{Z}} \beta) \times_{\mathbb{Z}} \gamma$.
  \item[(iii)] Distributivity: $\alpha \times_{\mathbb{Z}}
    (\beta +_{\mathbb{Z}} \gamma)
    = (\alpha \times_{\mathbb{Z}} \beta)
    +_{\mathbb{Z}} (\alpha \times_{\mathbb{Z}} \gamma)$.
  \item[(iv)] Multiplicative identity: $\alpha \times_{\mathbb{Z}}
    1_{\mathbb{Z}} = \alpha$.
  \item[(v)] No zero divisors: $\alpha \neq 0_{\mathbb{Z}} \wedge
    \beta \neq 0_{\mathbb{Z}} \Rightarrow
    \alpha \times_{\mathbb{Z}} \beta \neq 0_{\mathbb{Z}}$.
\end{enumerate}
\end{theorem}

\begin{remark}[Structural consequence]
Theorems 2.2 and 2.4 establish that
$(\mathbb{Z}, +_\mathbb{Z}, \times_\mathbb{Z})$ is an
\textbf{integral domain}: a commutative ring with unit and no zero divisors.
Mendelson makes this explicit in \S3.3 via abstract definitions.
\end{remark}

% ---------------------------------------------------------
% §3.3  Rings and Integral Domains (abstract)
% ---------------------------------------------------------
\subsubsection*{§3.3 \quad Rings and Integral Domains (Abstract Theory)}

\begin{remark}[Abstract vs concrete]
Mendelson develops ring theory abstractly before applying it to $\mathbb{Z}$.
Tao works concretely throughout, naming the ring structure only in
Remark 4.1.7.
\end{remark}

\begin{tcolorbox}[colback=propbox, colframe=propborder, arc=2pt,
  left=6pt, right=6pt, top=4pt, bottom=4pt,
  title={\small\textbf{Definition 3.5 (Ring; integral domain)}},
  fonttitle=\small\bfseries]
A \textbf{ring} $(R,+,\times)$: $(R,+)$ is an abelian group;
$\times$ is associative; $\times$ distributes over $+$ on both sides.
A ring is \textbf{commutative} if $xy=yx$;
has a \textbf{unit} if $\exists\,1$ with $x\cdot 1=1\cdot x=x$.
A nonzero $x$ is a \textbf{zero divisor} if $\exists$ nonzero $y$
with $xy=0$.
An \textbf{integral domain} is a commutative ring with unit,
$0 \neq 1$, and no zero divisors.
\end{tcolorbox}

\begin{theorem}[Mendelson 3.3 --- Cancellation $\Leftrightarrow$ no zero divisors]
\label{thm:men-cancel-equiv}
In a commutative ring with unit:
\[
(\forall x,y,z:\; xy=xz \wedge x\neq 0 \Rightarrow y=z)
\;\Longleftrightarrow\;
\text{no zero divisors.}
\]
\end{theorem}

\begin{theorem}[Mendelson 3.4 --- Trivial ring]
\label{thm:men-trivial}
In a ring with unit: $0=1 \Longleftrightarrow R$ is a singleton.
\end{theorem}

% ---------------------------------------------------------
% §4  Order
% ---------------------------------------------------------
\subsubsection*{§4 \quad Ordered Integral Domains and Order on $\mathbb{Z}$}

\begin{remark}[Strategy]
Mendelson proves order theory for any integral domain first, then
instantiates it for $\mathbb{Z}$.
This gives order on $\mathbb{Q}$ and $\mathbb{R}$ for free later.
\end{remark}

\begin{tcolorbox}[colback=propbox, colframe=propborder, arc=2pt,
  left=6pt, right=6pt, top=4pt, bottom=4pt,
  title={\small\textbf{Definition 4.1 (Ordered integral domain)}},
  fonttitle=\small\bfseries]
$(R,+,\times,<)$ is an \textbf{ordered integral domain} if
$(R,+,\times)$ is an integral domain ($0\neq 1$) and $<$ satisfies:
\begin{enumerate}
  \item[(O1)] $x \not< x$ \hfill\textit{Irreflexivity}
  \item[(O2)] $x<y \wedge y<z \Rightarrow x<z$ \hfill\textit{Transitivity}
  \item[(O3)] $x<y \vee x=y \vee y<x$ \hfill\textit{Trichotomy}
  \item[(O4)] $x<y \Rightarrow x+z<y+z$ \hfill\textit{Addition-monotone}
  \item[(O5)] $x<y \wedge 0<z \Rightarrow xz<yz$ \hfill\textit{Positive-mult-monotone}
\end{enumerate}
$x$ is \textbf{positive} if $0<x$; \textbf{negative} if $x<0$.
\end{tcolorbox}

\begin{theorem}[Mendelson 4.1 --- Consequences of order axioms]
In any ordered integral domain:
(i)~exactly one of $x<y$, $x=y$, $y<x$ holds;
(ii)~$x<y \wedge u<v \Rightarrow x+u<y+v$;
(iii)~$0<z \wedge xz<yz \Rightarrow x<y$.
\end{theorem}

\begin{theorem}[Mendelson 4.2 --- Order and positivity]
In any ordered integral domain:
(i)~$x<y \iff y-x$ positive;
(ii)~$x<y \iff x-y$ negative;
(iii)~$x<y \iff -y<-x$;
(iv)~sum and product of positives are positive;
(v)~product of two negatives is positive.
\end{theorem}

\begin{theorem}[Mendelson 4.3 --- Positivity set construction]
\label{thm:men-pos-set}
Let $(R,+,\times)$ be an integral domain, $0\neq 1$.
If $\mathcal{P}\subseteq R$ satisfies:
$0\notin\mathcal{P}$;
$\forall x:\; x\in\mathcal{P} \vee x=0 \vee {-x}\in\mathcal{P}$;
$\mathcal{P}$ closed under $+$ and $\times$;
then $(R,+,\times,<)$ with $x<y \iff y-x\in\mathcal{P}$ is an
ordered integral domain and $\mathcal{P}$ is its positive set.
\end{theorem}

\begin{remark}[Why this matters]
Theorem 4.3 reduces verifying five order axioms to four positivity
conditions.  Mendelson uses it to order $\mathbb{Z}$ via
$\mathcal{P}_\mathbb{Z}$ without checking (O1)--(O5) directly.
\end{remark}

\begin{lemma}[Mendelson 4.4 --- Positivity is class-invariant]
For $\alpha\in\mathbb{Z}$ and any two representatives
$(n,j),(k,i)\in\alpha$: $j<n \iff i<k$.
\end{lemma}

\begin{tcolorbox}[colback=propbox, colframe=propborder, arc=2pt,
  left=6pt, right=6pt, top=4pt, bottom=4pt,
  title={\small\textbf{Definition 4.5 (Positivity set $\mathcal{P}_\mathbb{Z}$)}},
  fonttitle=\small\bfseries]
\[
\mathcal{P}_{\mathbb{Z}}
= \bigl\{\alpha\in\mathbb{Z}
: \forall(n,j)\in\alpha,\; j<n \bigr\}.
\]
Lemma 4.4 ensures this is well-defined.
Intuitively: $\alpha$ is positive iff its representative has
second coordinate strictly less than first, i.e.\ $n-j > 0$.
\end{tcolorbox}

\begin{lemma}[Mendelson 4.5]
$\mathcal{P}_\mathbb{Z}$ satisfies the four conditions of Theorem~4.3.
\end{lemma}

\begin{corollary}[Mendelson 4.6]
$(\mathbb{Z},+_\mathbb{Z},\times_\mathbb{Z},<_\mathbb{Z})$ is an
ordered integral domain with positive set $\mathcal{P}_\mathbb{Z}$.
\end{corollary}

\begin{theorem}[Mendelson 4.7 --- Recovery of Peano system]
Let $T(x)=x+_\mathbb{Z}1_\mathbb{Z}$ for $x\in\mathcal{P}_\mathbb{Z}$.
Then $(\mathcal{P}_\mathbb{Z},T,1_\mathbb{Z})$ is a Peano system.
\end{theorem}

\begin{remark}[Significance]
Starting from a Peano system for $\mathbb{N}$, we construct $\mathbb{Z}$,
and then prove the positive integers \emph{inside} $\mathbb{Z}$ form a
new Peano system consistent with the original.
Tao does not prove this explicitly.
\end{remark}

% ---------------------------------------------------------
% §4  Absolute Value
% ---------------------------------------------------------
\subsubsection*{§4 \quad Absolute Value}

\begin{tcolorbox}[colback=propbox, colframe=propborder, arc=2pt,
  left=6pt, right=6pt, top=4pt, bottom=4pt,
  title={\small\textbf{Definition 4.6 (Absolute value)}},
  fonttitle=\small\bfseries]
In any ordered integral domain:
$|x| = x$ if $0\leq x$; $|x|=-x$ if $x<0$.
\end{tcolorbox}

\begin{theorem}[Mendelson 4.8 --- Properties of absolute value]
In any ordered integral domain, $|{\cdot}|$ satisfies:
(1)~$|x|\geq 0$;
(2)~$|x|=0\iff x=0$;
(3)~$|-x|=|x|$;
(4)~$|x-y|=|y-x|$;
(5)~$|xy|=|x||y|$;
(6)~$-|x|\leq x\leq|x|$;
(7)~$|z|<u\iff -u<z<u$;
(8)~$|z|\leq u\iff -u\leq z\leq u$;
(9)~$u\geq v \wedge u\geq -v \Rightarrow u\geq|v|$;
(10)~$|x+y|\leq|x|+|y|$ \textit{(triangle inequality)};
(11)~$|x-y|\geq\bigl||x|-|y|\bigr|$ \textit{(reverse triangle inequality)}.
\end{theorem}

% ---------------------------------------------------------
% §5  Number theory
% ---------------------------------------------------------
\subsubsection*{§5 \quad Division, Divisibility, Primes}

\begin{remark}[Scope]
This material is not in Tao Ch~4 and is not a prerequisite for real
analysis.  It is valuable background for number theory and abstract algebra.
See the comparison table for coverage decisions.
\end{remark}

\begin{theorem}[Mendelson 5.1 --- Euclidean division]
For any integer $\alpha>1$ and any integer $\beta$, there exist unique
integers $q,r$ with $\beta=q\alpha+r$ and $0\leq r<\alpha$.
\end{theorem}

\begin{definition}[Divisibility]
$\alpha\mid\beta \iff \exists\,\gamma\in\mathbb{Z}:\;\beta=\alpha\gamma$.
\end{definition}

\begin{theorem}[Mendelson 5.2--5.9]
Standard divisibility properties, existence and Bézout form of gcd,
characterisation of relative primeness, infinitude of primes
(Euclid), Euclid's lemma ($\rho\mid\alpha\beta\Rightarrow\rho\mid\alpha$
or $\rho\mid\beta$), and the fundamental theorem of arithmetic
(unique prime factorisation up to order and sign).
\end{theorem}

% ---------------------------------------------------------
% §6  Modular arithmetic
% ---------------------------------------------------------
\subsubsection*{§6 \quad Integers Modulo $n$ (Optional)}

\begin{remark}[Scope]
Congruence mod $n$ as an equivalence relation, $\mathbb{Z}_n$ as a
commutative ring with unit; integral domain iff $n$ is prime.
Not required for the real analysis track.
\end{remark}

% ---------------------------------------------------------
% §7--9  Structural theory
% ---------------------------------------------------------
\subsubsection*{§7--9 \quad Integer Action, Embedding Theory,
  Uniqueness of $\mathbb{Z}$}

\begin{remark}[Scope]
Chapters 7--9 are graduate-level algebra.
Key results:
\textbf{Thm~8.8}: every characteristic-$0$ integral domain contains
a copy of $\mathbb{Z}$.
\textbf{Thm~9.5}: $\mathcal{D}\cong\mathbb{Z}$ iff characteristic~$0$
and no proper subdomains.
\textbf{Thm~9.8}: any two well-ordered integral domains are isomorphic
--- $\mathbb{Z}$ is unique up to isomorphism.
These results are beyond what is needed for real analysis but are
worth knowing as statements.
\end{remark}

\begin{definition}[Well-ordered integral domain]
An ordered integral domain is \textbf{well-ordered} if every nonempty
subset of its positive elements contains a least element.
\end{definition}

\begin{theorem}[Mendelson 9.8 --- Uniqueness of $\mathbb{Z}$]
Any two well-ordered integral domains are isomorphic.
In particular, $\mathbb{Z}$ is, up to isomorphism, the unique
well-ordered integral domain.
\end{theorem}


\subsubsection{Tao vs.\ Mendelson: Comparison Table}
% =========================================================
% Tao vs. Mendelson — Integer Theory Comparison Table
% For use in planning coverage decisions.
% =========================================================

\clearpage
\section*{Tao vs.\ Mendelson: Integer Theory — Statement-by-Statement Comparison}

\begin{remark*}[How to read this table]
Each row is one mathematical concept.
\textbf{Coverage} recommendations: \textbf{Core} = do both;
\textbf{Tao only} = Tao suffices, Mendelson adds abstraction only;
\textbf{Men. only} = Mendelson covers this, Tao does not;
\textbf{Skip} = beyond real analysis scope for now.
\end{remark*}

\bigskip

% ------------------------------------------------------------------
% Block 1: The construction
% ------------------------------------------------------------------
\noindent\textbf{Block 1: The Construction of $\mathbb{Z}$}

\smallskip
\begin{tabular}{p{0.22\textwidth} p{0.21\textwidth} p{0.21\textwidth} p{0.10\textwidth} p{0.14\textwidth}}
\toprule
\textbf{Concept} & \textbf{Tao} & \textbf{Mendelson} & \textbf{Same idea?} & \textbf{Coverage} \\
\midrule

Formal pairs as integers
& Def 4.1.1: $a{-\!\!-}b$, $a{-\!\!-}b=c{-\!\!-}d\iff a+d=c+b$
& Def 3.1--3.2: $(n,j)\sim(k,i)\iff n+i=k+j$; $\mathbb{Z}=$ equiv.\ classes
& Yes — identical relation, different notation
& \textbf{Core}
\\[6pt]

Equality is an equivalence relation
& Ex 4.1.1 (reflexivity, symmetry); transitivity proved in text
& Thm 1.1 (all three parts)
& Yes
& \textbf{Core}
\\[6pt]

Distinguished elements $0,1$
& $0=0{-\!\!-}0$, $1=1{-\!\!-}0$ (implicit in Def 4.1.1)
& $0_\mathbb{Z}=[(1,1)]$, $1_\mathbb{Z}=[(2,1)]$ (explicit definition)
& Yes — same objects
& \textbf{Core}
\\[6pt]

Embedding $\mathbb{N}\hookrightarrow\mathbb{Z}$
& Remark after Def 4.1.2: $n\equiv n{-\!\!-}0$
& Thm 4.7 + Cor 9.7 (recovered as Peano system)
& Tao informal; Mendelson proves formally
& \textbf{Core}
\\

\bottomrule
\end{tabular}

\bigskip

% ------------------------------------------------------------------
% Block 2: Addition
% ------------------------------------------------------------------
\noindent\textbf{Block 2: Addition}

\smallskip
\begin{tabular}{p{0.22\textwidth} p{0.21\textwidth} p{0.21\textwidth} p{0.10\textwidth} p{0.14\textwidth}}
\toprule
\textbf{Concept} & \textbf{Tao} & \textbf{Mendelson} & \textbf{Same?} & \textbf{Coverage} \\
\midrule

Addition formula
& Def 4.1.2: $(a{-\!\!-}b)+(c{-\!\!-}d):=(a+c){-\!\!-}(b+d)$
& Def 3.3 (after Lem 2.1): $\alpha+_\mathbb{Z}\beta=[(n+k,j+i)]$
& Yes
& \textbf{Core}
\\[6pt]

Well-definedness of addition
& L4.1.3 (combines add.\ and mult.)
& Lem 2.1 (addition separately)
& Yes — Mendelson separates the two
& \textbf{Core}
\\[6pt]

Commutativity
& P4.1.6 (i)
& Thm 2.2 (i)
& Yes
& \textbf{Core}
\\[6pt]

Associativity
& P4.1.6 (ii)
& Thm 2.2 (ii)
& Yes
& \textbf{Core}
\\[6pt]

Additive identity
& P4.1.6 (iii): $x+0=0+x=x$
& Thm 2.2 (iii): $\alpha+0_\mathbb{Z}=\alpha$
& Yes
& \textbf{Core}
\\[6pt]

Additive inverse (negation)
& Def 4.1.4: $-(a{-\!\!-}b):=b{-\!\!-}a$; P4.1.6 (iv) existence
& Thm 2.2 (iv): unique $\delta$ with $\alpha+\delta=0$
& Tao: existence; Men.: existence + uniqueness in one statement
& \textbf{Core}
\\[6pt]

Negation well-defined
& Ex 4.1.2
& (Implicit in Lem 2.1 / Thm 2.2)
& Yes — Tao makes it an explicit exercise
& \textbf{Core}
\\

\bottomrule
\end{tabular}

\bigskip

% ------------------------------------------------------------------
% Block 3: Multiplication
% ------------------------------------------------------------------
\noindent\textbf{Block 3: Multiplication}

\smallskip
\begin{tabular}{p{0.22\textwidth} p{0.21\textwidth} p{0.21\textwidth} p{0.10\textwidth} p{0.14\textwidth}}
\toprule
\textbf{Concept} & \textbf{Tao} & \textbf{Mendelson} & \textbf{Same?} & \textbf{Coverage} \\
\midrule

Multiplication formula
& Def 4.1.2: $(a{-\!\!-}b)(c{-\!\!-}d):=(ac+bd){-\!\!-}(ad+bc)$
& Def 3.4 (after Lem 2.3): $\alpha\times_\mathbb{Z}\beta=[(nk+ji,jk+ni)]$
& Yes
& \textbf{Core}
\\[6pt]

Well-definedness of multiplication
& L4.1.3
& Lem 2.3
& Yes
& \textbf{Core}
\\[6pt]

Commutativity
& P4.1.6 (v)
& Thm 2.4 (i)
& Yes
& \textbf{Core}
\\[6pt]

Associativity
& P4.1.6 (vi); proved in text as model calculation
& Thm 2.4 (ii)
& Yes
& \textbf{Core}
\\[6pt]

Multiplicative identity
& P4.1.6 (vii)
& Thm 2.4 (iv)
& Yes
& \textbf{Core}
\\[6pt]

Distributive law
& P4.1.6 (viii, ix)
& Thm 2.4 (iii)
& Yes
& \textbf{Core}
\\[6pt]

$(-1)\times a = -a$
& Ex 4.1.3
& Not stated separately (follows from ring axioms)
& Tao only as exercise
& \textbf{Core (Tao)}
\\[6pt]

No zero divisors
& P4.1.8
& Thm 2.4 (v)
& Yes
& \textbf{Core}
\\[6pt]

Cancellation law
& C4.1.9
& Thm 3.3 (as equivalence, in any comm.\ ring with unit)
& Mendelson more general
& \textbf{Core}
\\[6pt]

$\mathbb{Z}$ is a commutative ring / integral domain
& Rem 4.1.7 (named but not defined abstractly)
& Def 3.5 + Thms 2.2, 2.4 (built up formally)
& Mendelson makes it explicit
& \textbf{Core}
\\

\bottomrule
\end{tabular}

\bigskip

% ------------------------------------------------------------------
% Block 4: Order
% ------------------------------------------------------------------
\noindent\textbf{Block 4: Order}

\smallskip
\begin{tabular}{p{0.22\textwidth} p{0.21\textwidth} p{0.21\textwidth} p{0.10\textwidth} p{0.14\textwidth}}
\toprule
\textbf{Concept} & \textbf{Tao} & \textbf{Mendelson} & \textbf{Same?} & \textbf{Coverage} \\
\midrule

Order definition
& Def 4.1.10: $n\geq m\iff n=m+a$, $a\in\mathbb{N}$
& Def 4.1 (abstract OID axioms) + $<_\mathbb{Z}$ via $\mathcal{P}_\mathbb{Z}$ (Cor 4.6)
& Same result; Mendelson builds via positivity set
& \textbf{Core}
\\[6pt]

Order irreflexivity
& (Implicit in Def 4.1.10 and L4.1.11(f))
& Explicitly O1 in Def 4.1
& Men.\ more explicit
& \textbf{Core}
\\[6pt]

Trichotomy of integers
& L4.1.5: every integer is positive, zero, or negative (one only)
& Thm 4.1(i) + Def 4.1 O3
& Yes
& \textbf{Core}
\\[6pt]

$a>b \iff a-b$ positive
& L4.1.11(a)
& Thm 4.2(i)
& Yes
& \textbf{Core}
\\[6pt]

Addition preserves order
& L4.1.11(b)
& Def 4.1 O4 (axiom); Thm 4.1(ii) (addition of inequalities)
& Yes
& \textbf{Core}
\\[6pt]

Positive mult.\ preserves order
& L4.1.11(c)
& Def 4.1 O5 (axiom)
& Yes
& \textbf{Core}
\\[6pt]

Negation reverses order
& L4.1.11(d)
& Thm 4.2(iii)
& Yes
& \textbf{Core}
\\[6pt]

Transitivity
& L4.1.11(e)
& Def 4.1 O2 (axiom)
& Yes
& \textbf{Core}
\\[6pt]

Order trichotomy (strict)
& L4.1.11(f)
& Thm 4.1(i)
& Yes
& \textbf{Core}
\\[6pt]

Positivity set construction (Thm 4.3)
& ---
& Thm 4.3: $\mathcal{P}$ satisfying 4 conditions $\Rightarrow$ OID
& Men.\ only
& \textbf{Men.\ only} (worth knowing)
\\[6pt]

Product of negatives is positive
& (Derivable from P4.1.6 + L4.1.11)
& Thm 4.2(v) (stated explicitly)
& Tao implicit; Men.\ explicit
& \textbf{Core}
\\[6pt]

Recovery of Peano system in $\mathbb{Z}$
& ---
& Thm 4.7: $(\mathcal{P}_\mathbb{Z},T,1_\mathbb{Z})$ is a Peano system
& Men.\ only
& \textbf{Men.\ only} (important conceptually)
\\

\bottomrule
\end{tabular}

\bigskip

% ------------------------------------------------------------------
% Block 5: Absolute value
% ------------------------------------------------------------------
\noindent\textbf{Block 5: Absolute Value}

\smallskip
\begin{tabular}{p{0.22\textwidth} p{0.21\textwidth} p{0.21\textwidth} p{0.10\textwidth} p{0.14\textwidth}}
\toprule
\textbf{Concept} & \textbf{Tao} & \textbf{Mendelson} & \textbf{Same?} & \textbf{Coverage} \\
\midrule

Definition of $|x|$
& (Defined for rationals in Def 4.3.1, not integers separately)
& Def 4.6 (for any ordered integral domain)
& Men.\ gives it for $\mathbb{Z}$ directly; Tao defers to $\mathbb{Q}$
& \textbf{Men.\ only} (do it here)
\\[6pt]

Triangle inequality $|x+y|\leq|x|+|y|$
& P4.3.3(b) (for rationals)
& Thm 4.8(10)
& Yes — Mendelson proves it for $\mathbb{Z}$
& \textbf{Core}
\\[6pt]

Multiplicativity $|xy|=|x||y|$
& P4.3.3(d) (for rationals)
& Thm 4.8(5)
& Yes
& \textbf{Core}
\\[6pt]

Reverse triangle inequality
& (Implicit)
& Thm 4.8(11) explicitly
& Men.\ explicit
& \textbf{Core}
\\[6pt]

Full 11-part absolute value theorem
& Spread across P4.3.3 (for $\mathbb{Q}$)
& Thm 4.8 (complete, for any OID)
& Men.\ more systematic
& \textbf{Men.\ only} (do once, use forever)
\\

\bottomrule
\end{tabular}

\bigskip

% ------------------------------------------------------------------
% Block 6: Induction failure / structural notes
% ------------------------------------------------------------------
\noindent\textbf{Block 6: Induction and Structural Observations}

\smallskip
\begin{tabular}{p{0.22\textwidth} p{0.21\textwidth} p{0.21\textwidth} p{0.10\textwidth} p{0.14\textwidth}}
\toprule
\textbf{Concept} & \textbf{Tao} & \textbf{Mendelson} & \textbf{Same?} & \textbf{Coverage} \\
\midrule

Induction fails for $\mathbb{Z}$
& Ex 4.1.8: explicit counterexample
& Implied by lack of least element; Thm 9.6 addresses positives only
& Tao makes it an explicit exercise
& \textbf{Core (Tao)}
\\[6pt]

Subtraction definition
& $x-y:=x+(-y)$ (after Def 4.1.4)
& Standard (follows from additive inverse)
& Yes
& \textbf{Core}
\\[6pt]

$0\neq 1$ in $\mathbb{Z}$
& Implicit (P2.2.8 + L4.1.5)
& Thm 3.4 (explicit: $0=1\iff$ trivial ring)
& Men.\ explicit
& \textbf{Core}
\\

\bottomrule
\end{tabular}

\bigskip

% ------------------------------------------------------------------
% Block 7: Beyond Tao (Mendelson only)
% ------------------------------------------------------------------
\noindent\textbf{Block 7: Material in Mendelson Only --- Coverage Decision Required}

\smallskip
\begin{tabular}{p{0.22\textwidth} p{0.29\textwidth} p{0.10\textwidth} p{0.28\textwidth}}
\toprule
\textbf{Concept} & \textbf{Mendelson ref.} & \textbf{Prereq for RA?} & \textbf{Recommendation} \\
\midrule

Abstract ring and integral domain definitions
& §3.3, Def 3.5
& No (but clarifying)
& Include notes; no proof sheets needed
\\[6pt]

Cancellation $\Leftrightarrow$ no zero divisors (abstract)
& Thm 3.3
& No
& Notes only; proof optional
\\[6pt]

Trivial ring theorem ($0=1\Rightarrow$ singleton)
& Thm 3.4
& No
& Skip proof; note the statement
\\[6pt]

Abstract ordered integral domain axioms
& Def 4.1
& Indirectly (same axioms used for $\mathbb{R}$)
& \textbf{Include} — these axioms recur throughout real analysis
\\[6pt]

Positivity set construction (Thm 4.3)
& Thm 4.3
& No (used internally)
& Notes only — elegant but not exercised directly
\\[6pt]

Peano recovery (Thm 4.7)
& Thm 4.7
& No
& Notes only; important for logical completeness
\\[6pt]

Euclidean division (§5, Thm 5.1)
& Thm 5.1
& No
& \textbf{Proof sheet} — classical and useful
\\[6pt]

Divisibility properties (§5, Thm 5.2)
& Thm 5.2 (11 parts)
& No
& \textbf{Proof sheet} — good algebraic practice
\\[6pt]

GCD existence and Bézout (Thm 5.3)
& Thm 5.3
& No
& \textbf{Proof sheet} — important
\\[6pt]

Infinitude of primes (Thm 5.6)
& Thm 5.6
& No
& \textbf{Proof sheet} — Euclid's classic argument
\\[6pt]

Fundamental theorem of arithmetic (Thm 5.9)
& Thm 5.9
& No
& \textbf{Proof sheet} — core number theory
\\[6pt]

Modular arithmetic (§6)
& Thms 6.1--6.6
& No
& \textbf{Skip} for now; revisit in algebra track
\\[6pt]

Integer action on domains (§7, Thms 7.1--7.4)
& Thms 7.1--7.4
& No
& \textbf{Skip} — graduate algebra
\\[6pt]

$N_\mathcal{D}$, $Z_\mathcal{D}$ embedding theory (§8)
& Thms 8.1--8.9
& No
& \textbf{Skip} — graduate algebra
\\[6pt]

Subdomain theory (§9, Thms 9.1--9.4)
& Thms 9.1--9.4
& No
& \textbf{Skip} — graduate algebra
\\[6pt]

Characterisation of $\mathbb{Z}$ (Thm 9.5)
& Thm 9.5
& No
& Notes only — worth knowing as a statement
\\[6pt]

Well-ordered integral domain (Def, Cor 9.7)
& Def + Cor 9.7
& No
& Notes only
\\[6pt]

Uniqueness of $\mathbb{Z}$ up to isomorphism (Thm 9.8)
& Thm 9.8
& No
& \textbf{Note the statement} — philosophically important
\\

\bottomrule
\end{tabular}

\bigskip

% ------------------------------------------------------------------
% Summary
% ------------------------------------------------------------------
\begin{tcolorbox}[colback=gray!6, colframe=gray!40, arc=2pt,
  left=6pt, right=6pt, top=4pt, bottom=4pt,
  title={\small\textbf{Summary: Recommended Coverage}},
  fonttitle=\small\bfseries]
\small
\textbf{Core (both sources):}
Equality, addition, multiplication, well-definedness, ring laws,
no zero divisors, cancellation, order, trichotomy, six order properties.

\medskip
\textbf{Mendelson adds (do notes, consider proof sheets):}
Abstract ring/integral domain definitions (\S3.3);
abstract OID axioms (Def~4.1) --- these recur in real analysis;
absolute value on $\mathbb{Z}$ (Thm~4.8) --- Tao defers to $\mathbb{Q}$;
positivity set construction (Thm~4.3) --- elegant technique;
Peano recovery (Thm~4.7) --- logical closure;
number theory: Euclidean division, divisibility, GCD, primes (§5).

\medskip
\textbf{Skip for now (graduate algebra):}
Modular arithmetic (§6), integer action on domains (§7),
$N_\mathcal{D}$/$Z_\mathcal{D}$ embedding theory (§8--9).
Know the statements of Thms~9.5 and~9.8 (uniqueness of $\mathbb{Z}$).
\end{tcolorbox}



\section{Capstone}
% ======================================================
% Capstone Assessment — Sets, Relations, and Functions
% ======================================================

\section{Capstone Assessment: Sets, Relations, and Functions}

\noindent
\textbf{Purpose.}
This capstone assesses mastery of elementary set theory,
relations, and functions as used in analysis and abstract mathematics.
All proofs must be written using precise definitions and logical reasoning.

\vspace{1em}

\noindent
\textbf{Instructions.}
Each problem requires a complete proof.
You must explicitly invoke definitions (e.g.\ function, equivalence relation)
when they are used.
No appeal to diagrams or intuition is permitted.

\vspace{1.5em}

% --------------------------------------------------
\subsection*{Problem 1 — Equivalence Relations}

Let $\sim$ be a relation on a set $A$.
Prove that $\sim$ is an equivalence relation if and only if
its equivalence classes form a partition of $A$.

Your proof must establish both directions.

\vspace{1.5em}

% --------------------------------------------------
\subsection*{Problem 2 — Images and Preimages}

Let $f : A \to B$ be a function and let $S \subseteq A$.
Prove that
\[
S \subseteq f^{-1}(f(S)).
\]

Give an example where equality does not hold.

\vspace{1.5em}

% --------------------------------------------------
\subsection*{Problem 3 — Injectivity and Left Inverses}

Prove that a function $f : A \to B$ is injective
if and only if there exists a function $g : B \to A$
such that
\[
g \circ f = \mathrm{id}_A.
\]

Your proof must clearly indicate where injectivity is used.

\vspace{1.5em}

% --------------------------------------------------
\subsection*{Problem 4 — Composition of Relations}

Let $R \subseteq A \times B$ and $S \subseteq B \times C$
be relations.
Prove that if both $R$ and $S$ are transitive relations
(on their respective domains), then their composition
need not be transitive.

Your proof must include a concrete counterexample.

\vspace{1.5em}

% --------------------------------------------------
\subsection*{Problem 5 — Order Relations}

Let $(A,\le)$ be a partially ordered set.
Prove that $\le$ is antisymmetric if and only if
\[
(x \le y \;\land\; y \le x) \;\rightarrow\; x = y
\]
holds for all $x,y \in A$.

Your proof must explicitly use the definition of antisymmetry.

\vspace{1em}

\noindent
\textbf{Completion Criterion.}
You have mastered sets, relations, and functions if all five proofs:
\begin{itemize}
  \item correctly invoke definitions,
  \item handle element-wise reasoning rigorously,
  \item distinguish relations from functions,
  \item and avoid implicit assumptions.
\end{itemize}

\noindent
Successful completion certifies readiness to proceed to
foundations of the real line and completeness.

