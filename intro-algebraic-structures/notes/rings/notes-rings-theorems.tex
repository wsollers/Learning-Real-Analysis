% =========================================================
% Basic Theorems of Rings
% =========================================================
\paragraph{Basic Theorems}

\begin{remark}[What Needs Proving]
The ring axioms say nothing explicitly about how multiplication
interacts with the additive identity $0$ or with additive inverses.
These must be derived. The key tool in both proofs is distributivity
--- the bridge between the two operations.

Notice that these proofs make no assumption about which ring you are in.
They work in $\mathbb{Z}$, in $M_n(\mathbb{R})$, in $\mathbb{Z}[x]$,
in any ring simultaneously.
\end{remark}

\begin{proposition}[Multiplication by Zero]
\label{prop:ring-mult-zero}
Let $R$ be a ring. For all $a \in R$,
\[
a \cdot 0 = 0 \cdot a = 0.
\]
\end{proposition}
\begin{proof}
We show $a \cdot 0 = 0$; the proof that $0 \cdot a = 0$ is symmetric.

\noindent
\begin{tabular}{T S J}
\toprule
\textbf{Tag} & \multicolumn{1}{p{0.44\textwidth}}{\textbf{Step}} & \textbf{Justification} \\
\midrule
\addlinespace[4pt]

\tagDU
&
0 + 0 = 0
&
$0$ is the additive identity (Axiom R1).
\\[10pt]

\tagAM
&
a \cdot 0 = a \cdot (0 + 0)
&
Substituting $0 = 0 + 0$.
\\[10pt]

\tagDU
&
a \cdot (0 + 0) = a \cdot 0 + a \cdot 0
&
Left distributivity (Axiom R3).
\\[10pt]

\tagAM
&
a \cdot 0 + a \cdot 0 = a \cdot 0
&
Combining the chain: $a \cdot 0 = a \cdot 0 + a \cdot 0$.
\\[10pt]

\tagTA
&
a \cdot 0 = 0
&
Add $-(a \cdot 0)$ to both sides.
Left side: $(a \cdot 0 + a \cdot 0) + (-(a \cdot 0)) = a \cdot 0$.
Right side: $a \cdot 0 + (-(a \cdot 0)) = 0$.
By additive cancellation
(Proposition~\ref{prop:group-cancellation}).
\hfill$\blacksquare$
\\[6pt]

\bottomrule
\end{tabular}
\end{proof}


\begin{remark}
The proof uses distributivity to produce the equation
$a \cdot 0 + a \cdot 0 = a \cdot 0$, then applies additive
cancellation to conclude $a \cdot 0 = 0$.

This is not circular: $0$ on the left side is the additive identity
of the ring; the $0$ on the right is the same element being derived
as a consequence of cancellation. The proof works because additive
cancellation was already proved for all abelian groups
(Proposition~\ref{prop:group-cancellation}).
\end{remark}

\begin{proposition}[Multiplication by Additive Inverse]
\label{prop:ring-mult-neg}
Let $R$ be a ring. For all $a, b \in R$,
\[
a \cdot (-b) = -(a \cdot b)
\quad\text{and}\quad
(-a) \cdot b = -(a \cdot b).
\]
In any ring with unity, $(-1) \cdot a = -a$.
\end{proposition}
\begin{proof}
We show $a \cdot (-b) = -(a \cdot b)$;
the proof for $(-a) \cdot b$ is symmetric.

It suffices to show $a \cdot (-b)$ satisfies the defining property
of the additive inverse of $a \cdot b$, then invoke uniqueness.

\noindent
\begin{tabular}{T S J}
\toprule
\textbf{Tag} & \multicolumn{1}{p{0.44\textwidth}}{\textbf{Step}} & \textbf{Justification} \\
\midrule
\addlinespace[4pt]

\tagAM
&
(a \cdot b) + (a \cdot (-b)) = a \cdot (b + (-b))
&
Left distributivity (Axiom R3).
\\[10pt]

\tagDU
&
a \cdot (b + (-b)) = a \cdot 0
&
$b + (-b) = 0$ by definition of additive inverse (Axiom R1).
\\[10pt]

\tagTA
&
a \cdot 0 = 0
&
Proposition~\ref{prop:ring-mult-zero}.
\\[10pt]

\tagDU
&
(a \cdot b) + (a \cdot (-b)) = 0
&
Combining: $a \cdot (-b)$ is an additive inverse of $a \cdot b$.
\\[10pt]

\tagTA
&
a \cdot (-b) = -(a \cdot b)
&
Additive inverses are unique
(Proposition~\ref{prop:group-inverse-unique}).
\hfill$\blacksquare$
\\[6pt]

\bottomrule
\end{tabular}
\end{proof}


\begin{remark}
The proof strategy is: show $a \cdot (-b)$ satisfies the defining
property of the additive inverse of $a \cdot b$, then invoke
uniqueness of additive inverses
(Proposition~\ref{prop:group-inverse-unique}).

This is the same strategy used in the vector space proof that
$(-1)v = -v$ --- because that proof is just this theorem applied
to the scalar field $\mathbb{F}$ acting on $V$.
\end{remark}
