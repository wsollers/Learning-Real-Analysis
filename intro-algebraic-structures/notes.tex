% =========================================================
% Breadcrumb
% =========================================================
\begin{tcolorbox}[
  colback=gray!6,
  colframe=gray!40,
  arc=2pt,
  left=8pt, right=8pt, top=6pt, bottom=6pt,
  title={\small\textbf{Where You Are in the Journey}},
  fonttitle=\small\bfseries
]
\begin{center}
\small
Propositional Logic
$\;\to\;$ Predicate Calculus
$\;\to\;$ Sets \& Functions
$\;\to\;$ Proof Techniques
$\;\to\;$ Real Analysis
$\;\to\;$ \textbf{Intro to Algebraic Structures}
$\;\to\;$ Linear Algebra
$\;\to\;$ Topology
$\;\to\;$ $\cdots$
\end{center}

\medskip
\noindent\textbf{How we got here.}
Sets and functions gave us the language of mathematical structure,
and proof techniques gave us the tools to reason about it.
Real analysis showed what rigorous reasoning looks like on a concrete
number system. Algebraic structures now ask the deeper question:
what is the \emph{minimal} set of axioms needed for the most important
algebraic phenomena --- identity, inverses, and arithmetic --- to occur?

\medskip
\noindent\textbf{What this chapter builds.}
We construct the hierarchy of algebraic structures: groups, rings,
and fields. Each is defined by progressively richer axiom systems.
Fields are the scalars over which vector spaces are built,
and groups are the additive backbone of every vector space.

\medskip
\noindent\textbf{Where this leads.}
Linear algebra inherits the field and group axioms directly:
a vector space is an abelian group equipped with a scalar multiplication
by a field, and every proof about vector spaces draws on both simultaneously.
\end{tcolorbox}
\vspace{1em}

% =========================================================
% Structural Roadmap
% =========================================================
\subsection*{Structural Roadmap}

The development of algebraic structures in this project follows
the definition--theorem--structure architecture used
throughout the analysis volumes.

The primary driver is \textit{Contemporary Abstract Algebra}
by Joseph Gallian. The emphasis is on axiom systems,
structural consequences, and the hierarchy of algebraic objects
rather than computational techniques.

Each major topic is organized as:
\begin{center}
\textbf{Definitions $\longrightarrow$ Main Theorems
$\longrightarrow$ Consequences and Structural Insight}
\end{center}

The global progression is:
\begin{enumerate}
  \item Groups and abelian groups
  \item Rings and integral domains
  \item Fields and their properties
  \item Bridge to linear algebra
\end{enumerate}

\vspace{1em}

\begin{remark}[Structural Position]
The structures developed here are not studied for their own sake alone.
Groups provide the additive structure of every vector space.
Fields provide the scalars. The interaction between them
is the subject of linear algebra.
\end{remark}

\begin{remark}[Dependency Note]
The uniqueness theorems proved here --- uniqueness of identity,
uniqueness of inverses --- are the same theorems invoked in
vector space proofs. They are proved once here and cited by
theorem number thereafter.
\end{remark}

% =========================================================
% Content
% =========================================================
% =========================================================
% Groups
% =========================================================
\subsubsection{Groups}

% =========================================================
% Binary Operations
% =========================================================
\paragraph{Binary Operations}

\begin{definition}[Binary Operation]
Let $G$ be a set.
A \emph{binary operation} on $G$ is a function
\[
\star : G \times G \to G.
\]
For $a, b \in G$, we write $a \star b$ for the image of $(a,b)$ under $\star$.
\end{definition}

\begin{remark}
The codomain of $\star$ is $G$ itself. This means that for any $a, b \in G$,
the result $a \star b$ is again an element of $G$.
This property is called \emph{closure} and is built into the definition of a
binary operation.

Intuitively: a binary operation takes two elements of a set and produces
a third element of the \emph{same} set.
\end{remark}

\begin{definition}[Associativity]
A binary operation $\star$ on $G$ is \emph{associative} if
\[
(a \star b) \star c = a \star (b \star c)
\quad \text{for all } a, b, c \in G.
\]
\end{definition}

\begin{definition}[Commutativity]
A binary operation $\star$ on $G$ is \emph{commutative} if
\[
a \star b = b \star a
\quad \text{for all } a, b \in G.
\]
\end{definition}

\begin{remark}
Associativity and commutativity are independent properties.
Addition on $\mathbb{Z}$ is both; matrix multiplication is associative
but not commutative; subtraction on $\mathbb{Z}$ is neither.
\end{remark}
% =========================================================
% Group Definition and Axioms
% =========================================================
\paragraph{Definition of a Group}

\begin{definition}[Group]
A \emph{group} is a pair $(G, \star)$ where $G$ is a set and
$\star$ is a binary operation on $G$ satisfying the following axioms:

\begin{enumerate}[label=\textbf{G\arabic*.}]
  \item \textbf{Associativity.}
        $(a \star b) \star c = a \star (b \star c)$
        for all $a, b, c \in G$.

  \item \textbf{Identity.}
        There exists an element $e \in G$ such that
        \[
          e \star a = a \star e = a
          \quad \text{for all } a \in G.
        \]
        The element $e$ is called the \emph{identity element} of $G$.

  \item \textbf{Inverses.}
        For each $a \in G$, there exists an element $a^{-1} \in G$ such that
        \[
          a \star a^{-1} = a^{-1} \star a = e.
        \]
        The element $a^{-1}$ is called the \emph{inverse} of $a$.
\end{enumerate}
\end{definition}

\begin{remark}
Closure is not listed as a separate axiom because it is already
encoded in the requirement that $\star : G \times G \to G$ is a
binary operation --- the codomain forces the result to stay in $G$.

Intuitively: a group is a set where you can combine elements,
undo combinations, and the order of grouping never matters.
\end{remark}

\begin{remark}[Axiom Count]
Some treatments list four axioms (closure, associativity, identity, inverses).
Here closure is absorbed into the definition of binary operation,
leaving three axioms. Both presentations define the same object.
\end{remark}

\begin{remark}[Notation]
When the operation is understood from context, we write $ab$ instead
of $a \star b$, and call $G$ itself a group rather than the pair $(G, \star)$.
For groups whose operation is addition, we write $a + b$, use $0$ for the
identity, and $-a$ for the inverse of $a$.
\end{remark}

\begin{definition}[Order of a Group]
The \emph{order} of a group $G$, denoted $|G|$, is the cardinality of $G$
as a set. If $|G|$ is finite, $G$ is called a \emph{finite group};
otherwise it is an \emph{infinite group}.
\end{definition}

\begin{example}[Canonical Examples of Groups]
\begin{enumerate}[label=(\roman*)]
  \item $(\mathbb{Z}, +)$: the integers under addition.
        Identity: $0$. Inverse of $n$: $-n$. Infinite group.

  \item $(\mathbb{Q} \setminus \{0\}, \cdot)$: nonzero rationals under multiplication.
        Identity: $1$. Inverse of $q$: $1/q$. Infinite group.

  \item $(\mathbb{Z}/n\mathbb{Z}, +)$: integers modulo $n$ under addition.
        Identity: $[0]$. Inverse of $[k]$: $[n-k]$. Finite group of order $n$.

  \item $(GL_n(\mathbb{R}), \cdot)$: invertible $n \times n$ real matrices
        under multiplication.
        Identity: $I_n$. Inverse: matrix inverse. Infinite group.
\end{enumerate}
\end{example}

\begin{remark}
Note what fails to be a group:
$(\mathbb{Z}, \cdot)$ is not a group because $2$ has no multiplicative
inverse in $\mathbb{Z}$.
$(\mathbb{N}, +)$ is not a group because positive integers have no
additive inverse in $\mathbb{N}$.
These failures illustrate why each axiom is necessary.
\end{remark}
% =========================================================
% Basic Theorems of Groups
% =========================================================
\paragraph{Basic Theorems}

\begin{remark}[Why These Theorems Matter]
The group axioms guarantee existence of an identity and inverses,
but say nothing about uniqueness. The following theorems establish
that both are unique. This is essential: without uniqueness, we
cannot speak of \emph{the} identity or \emph{the} inverse of an element,
and proofs that equate two objects via the identity or inverse would
be invalid.

These same uniqueness theorems reappear in vector space proofs,
where they are cited by name. They are proved once here.
\end{remark}

% ---------------------------------------------------------
\begin{proposition}[Uniqueness of the Identity]
\label{prop:group-identity-unique}
Let $G$ be a group. The identity element of $G$ is unique.
\end{proposition}


\begin{remark}
The proof strategy is standard for uniqueness arguments:
assume two identities exist, then show they must be equal.
This pattern recurs throughout algebra whenever a definition
asserts existence of a distinguished element.

Intuitively: if two elements both act as an identity,
applying one to the other forces them to coincide.
\end{remark}

% ---------------------------------------------------------
\begin{proposition}[Uniqueness of Inverses]
\label{prop:group-inverse-unique}
Let $G$ be a group. For each $a \in G$, the inverse of $a$ is unique.
\end{proposition}


\begin{remark}
Intuitively: if two elements both undo $a$, then they must be
the same element --- because each can be obtained from the other
by cancellation.
\end{remark}

% ---------------------------------------------------------
\begin{proposition}[Cancellation Laws]
\label{prop:group-cancellation}
Let $G$ be a group and let $a, b, c \in G$. Then:
\begin{enumerate}[label=(\roman*)]
  \item \textbf{Left cancellation:} $ab = ac \implies b = c$.
  \item \textbf{Right cancellation:} $ba = ca \implies b = c$.
\end{enumerate}
\end{proposition}


\begin{remark}
Cancellation is what makes group equations solvable.
It does \emph{not} hold in general for rings or monoids without inverses.

Intuitively: multiply both sides by $a^{-1}$ and the common factor disappears.
\end{remark}

% ---------------------------------------------------------
\begin{proposition}[Socks-Shoes Property]
\label{prop:group-socks-shoes}
Let $G$ be a group and let $a, b \in G$. Then
\[
(ab)^{-1} = b^{-1} a^{-1}.
\]
\end{proposition}


\begin{remark}
The name comes from the observation that to undo putting on socks
then shoes, you must first remove the shoes, then the socks ---
in reverse order.

This reversal of order is characteristic of non-abelian groups and
becomes important in the theory of group homomorphisms and
in matrix algebra, where $(AB)^{-1} = B^{-1}A^{-1}$.
\end{remark}
% =========================================================
% Abelian Groups
% =========================================================
\paragraph{Abelian Groups}

\begin{definition}[Abelian Group]
A group $(G, \star)$ is called \emph{abelian} (or \emph{commutative}) if
\[
a \star b = b \star a \quad \text{for all } a, b \in G.
\]
\end{definition}

\begin{remark}
An abelian group is a group with one additional axiom: commutativity
of the operation. Every abelian group is a group, but not every group
is abelian.

Intuitively: in an abelian group, the order in which you combine
elements is irrelevant. This is the familiar arithmetic of addition
on $\mathbb{Z}$, $\mathbb{Q}$, $\mathbb{R}$, and $\mathbb{C}$.
\end{remark}

\begin{remark}[Structural Position]
Abelian groups are the additive backbone of every vector space.
The four vector space axioms governing addition --- associativity,
commutativity, existence of zero, existence of additive inverses ---
are precisely the axioms that make $(V, +)$ an abelian group.

This is why the vector space definition can be stated compactly as:
\emph{a vector space over $\mathbb{F}$ is an abelian group $(V,+)$
equipped with a scalar multiplication by $\mathbb{F}$}.
The abelian group structure is not an analogy; it is the literal
algebraic content of the first four vector space axioms.
\end{remark}

\begin{example}[Abelian and Non-Abelian Groups]
\begin{enumerate}[label=(\roman*)]
  \item $(\mathbb{Z}, +)$, $(\mathbb{R}, +)$, $(\mathbb{C}, +)$:
        all abelian. These are the additive groups underlying the
        standard vector spaces.

  \item $(\mathbb{Z}/n\mathbb{Z}, +)$: abelian for all $n \geq 1$.

  \item $(GL_n(\mathbb{R}), \cdot)$ for $n \geq 2$: \emph{not} abelian,
        since matrix multiplication does not commute in general.

  \item $(S_n, \circ)$ for $n \geq 3$: the symmetric group on $n$ symbols
        under composition is not abelian.
\end{enumerate}
\end{example}

\begin{remark}[Additive Notation Convention]
For abelian groups, it is standard to write the operation as $+$,
the identity as $0$, and the inverse of $a$ as $-a$.
This additive notation is used throughout linear algebra, where
$(V, +)$ is always an abelian group.
\end{remark}













% =========================================================
% Rings
% =========================================================
\subsubsection{Rings}

% =========================================================
% Ring Definition and Axioms
% =========================================================
\paragraph{Definition of a Ring}

\begin{remark}[Motivation]
A group has one binary operation. A ring has two: addition and
multiplication. The addition makes the ring an abelian group.
Multiplication is layered on top, connected to addition through
the distributive laws.

Rings are the natural algebraic home of arithmetic. The integers
$\mathbb{Z}$, polynomials $\mathbb{Z}[x]$, and square matrices
$M_n(\mathbb{R})$ are all rings. What they share is not the
specific objects but the axiom structure --- and every theorem
proved here holds simultaneously in all of them.
\end{remark}

\begin{definition}[Ring]
A \emph{ring} is a triple $(R, +, \cdot)$ where $R$ is a set and
$+$ and $\cdot$ are binary operations on $R$ satisfying:

\begin{enumerate}[label=\textbf{R\arabic*.}]
  \item \textbf{Additive abelian group.}
        $(R, +)$ is an abelian group:
        \begin{itemize}
          \item $(a + b) + c = a + (b + c)$ for all $a,b,c \in R$.
          \item $a + b = b + a$ for all $a, b \in R$.
          \item There exists $0 \in R$ such that $a + 0 = a$ for all $a \in R$.
          \item For each $a \in R$, there exists $-a \in R$ such that
                $a + (-a) = 0$.
        \end{itemize}

  \item \textbf{Multiplicative associativity.}
        $(a \cdot b) \cdot c = a \cdot (b \cdot c)$
        for all $a, b, c \in R$.

  \item \textbf{Distributivity.}
        \begin{itemize}
          \item $a \cdot (b + c) = a \cdot b + a \cdot c$
                for all $a,b,c \in R$.
          \item $(a + b) \cdot c = a \cdot c + b \cdot c$
                for all $a,b,c \in R$.
        \end{itemize}
\end{enumerate}
\end{definition}

\begin{remark}[What a Ring Does and Does Not Require]
A ring does \emph{not} require:
\begin{itemize}
  \item commutativity of multiplication ($ab = ba$ need not hold),
  \item a multiplicative identity ($1$ need not exist),
  \item multiplicative inverses ($a^{-1}$ need not exist).
\end{itemize}
Each additional requirement produces a richer structure.
A ring with a multiplicative identity is a \emph{ring with unity}.
A commutative ring with unity where every nonzero element has a
multiplicative inverse is a \emph{field} --- covered in the next section.

Intuitively: a ring is the minimal structure needed for addition,
subtraction, and multiplication to coexist and interact sensibly.
Division is not guaranteed.
\end{remark}

\begin{definition}[Commutative Ring]
A ring $(R, +, \cdot)$ is \emph{commutative} if
\[
a \cdot b = b \cdot a \quad \text{for all } a, b \in R.
\]
\end{definition}

\begin{definition}[Ring with Unity]
A ring $(R, +, \cdot)$ is a \emph{ring with unity} if there exists
$1 \in R$ such that
\[
1 \cdot a = a \cdot 1 = a \quad \text{for all } a \in R.
\]
The element $1$ is called the \emph{multiplicative identity} or \emph{unity}.
When it exists, it is unique (proof identical to
Proposition~\ref{prop:group-identity-unique} applied to $(R, \cdot)$).
\end{definition}

\begin{example}[Canonical Examples of Rings]
\begin{enumerate}[label=(\roman*)]
  \item $(\mathbb{Z}, +, \cdot)$: integers.
        Commutative ring with unity $1$.
        No multiplicative inverses for $|n| \neq 1$.

  \item $(\mathbb{Z}/n\mathbb{Z}, +, \cdot)$: integers modulo $n$.
        Commutative ring with unity $[1]$.

  \item $(M_n(\mathbb{R}), +, \cdot)$: $n \times n$ real matrices.
        Ring with unity $I_n$. \emph{Not} commutative for $n \geq 2$.

  \item $(\mathbb{Z}[x], +, \cdot)$: polynomials with integer coefficients.
        Commutative ring with unity $1$.

  \item The trivial ring $\{0\}$, where $0 = 1$, is the only ring
        in which the additive and multiplicative identities coincide.
\end{enumerate}
\end{example}

% =========================================================
% Basic Theorems of Rings
% =========================================================
\paragraph{Basic Theorems}

\begin{remark}[What Needs Proving]
The ring axioms say nothing explicitly about how multiplication
interacts with the additive identity $0$ or with additive inverses.
These must be derived. The key tool in both proofs is distributivity
--- the bridge between the two operations.

Notice that these proofs make no assumption about which ring you are in.
They work in $\mathbb{Z}$, in $M_n(\mathbb{R})$, in $\mathbb{Z}[x]$,
in any ring simultaneously.
\end{remark}

\begin{proposition}[Multiplication by Zero]
\label{prop:ring-mult-zero}
Let $R$ be a ring. For all $a \in R$,
\[
a \cdot 0 = 0 \cdot a = 0.
\]
\end{proposition}
\begin{proof}
We show $a \cdot 0 = 0$; the proof that $0 \cdot a = 0$ is symmetric.

\noindent
\begin{tabular}{T S J}
\toprule
\textbf{Tag} & \multicolumn{1}{p{0.44\textwidth}}{\textbf{Step}} & \textbf{Justification} \\
\midrule
\addlinespace[4pt]

\tagDU
&
0 + 0 = 0
&
$0$ is the additive identity (Axiom R1).
\\[10pt]

\tagAM
&
a \cdot 0 = a \cdot (0 + 0)
&
Substituting $0 = 0 + 0$.
\\[10pt]

\tagDU
&
a \cdot (0 + 0) = a \cdot 0 + a \cdot 0
&
Left distributivity (Axiom R3).
\\[10pt]

\tagAM
&
a \cdot 0 + a \cdot 0 = a \cdot 0
&
Combining the chain: $a \cdot 0 = a \cdot 0 + a \cdot 0$.
\\[10pt]

\tagTA
&
a \cdot 0 = 0
&
Add $-(a \cdot 0)$ to both sides.
Left side: $(a \cdot 0 + a \cdot 0) + (-(a \cdot 0)) = a \cdot 0$.
Right side: $a \cdot 0 + (-(a \cdot 0)) = 0$.
By additive cancellation
(Proposition~\ref{prop:group-cancellation}).
\hfill$\blacksquare$
\\[6pt]

\bottomrule
\end{tabular}
\end{proof}


\begin{remark}
The proof uses distributivity to produce the equation
$a \cdot 0 + a \cdot 0 = a \cdot 0$, then applies additive
cancellation to conclude $a \cdot 0 = 0$.

This is not circular: $0$ on the left side is the additive identity
of the ring; the $0$ on the right is the same element being derived
as a consequence of cancellation. The proof works because additive
cancellation was already proved for all abelian groups
(Proposition~\ref{prop:group-cancellation}).
\end{remark}

\begin{proposition}[Multiplication by Additive Inverse]
\label{prop:ring-mult-neg}
Let $R$ be a ring. For all $a, b \in R$,
\[
a \cdot (-b) = -(a \cdot b)
\quad\text{and}\quad
(-a) \cdot b = -(a \cdot b).
\]
In any ring with unity, $(-1) \cdot a = -a$.
\end{proposition}
\begin{proof}
We show $a \cdot (-b) = -(a \cdot b)$;
the proof for $(-a) \cdot b$ is symmetric.

It suffices to show $a \cdot (-b)$ satisfies the defining property
of the additive inverse of $a \cdot b$, then invoke uniqueness.

\noindent
\begin{tabular}{T S J}
\toprule
\textbf{Tag} & \multicolumn{1}{p{0.44\textwidth}}{\textbf{Step}} & \textbf{Justification} \\
\midrule
\addlinespace[4pt]

\tagAM
&
(a \cdot b) + (a \cdot (-b)) = a \cdot (b + (-b))
&
Left distributivity (Axiom R3).
\\[10pt]

\tagDU
&
a \cdot (b + (-b)) = a \cdot 0
&
$b + (-b) = 0$ by definition of additive inverse (Axiom R1).
\\[10pt]

\tagTA
&
a \cdot 0 = 0
&
Proposition~\ref{prop:ring-mult-zero}.
\\[10pt]

\tagDU
&
(a \cdot b) + (a \cdot (-b)) = 0
&
Combining: $a \cdot (-b)$ is an additive inverse of $a \cdot b$.
\\[10pt]

\tagTA
&
a \cdot (-b) = -(a \cdot b)
&
Additive inverses are unique
(Proposition~\ref{prop:group-inverse-unique}).
\hfill$\blacksquare$
\\[6pt]

\bottomrule
\end{tabular}
\end{proof}


\begin{remark}
The proof strategy is: show $a \cdot (-b)$ satisfies the defining
property of the additive inverse of $a \cdot b$, then invoke
uniqueness of additive inverses
(Proposition~\ref{prop:group-inverse-unique}).

This is the same strategy used in the vector space proof that
$(-1)v = -v$ --- because that proof is just this theorem applied
to the scalar field $\mathbb{F}$ acting on $V$.
\end{remark}

% =========================================================
% Integral Domains
% =========================================================
\paragraph{Integral Domains}

\begin{remark}[Motivation]
In $\mathbb{Z}$, if $ab = 0$ then $a = 0$ or $b = 0$.
This feels obvious but it is not an axiom of rings --- it fails in
$\mathbb{Z}/6\mathbb{Z}$, where $[2] \cdot [3] = [0]$ even though
neither $[2]$ nor $[3]$ is zero. Elements that behave like $[2]$
and $[3]$ are called zero divisors, and rings without them are
integral domains.

This property matters because it is exactly what is needed for
multiplicative cancellation --- and for the zero product argument
that appears throughout linear algebra.
\end{remark}

\begin{definition}[Zero Divisor]
Let $R$ be a commutative ring with unity.
A nonzero element $a \in R$ is a \emph{zero divisor} if there exists
a nonzero $b \in R$ such that $a \cdot b = 0$.
\end{definition}

\begin{definition}[Integral Domain]
A commutative ring with unity $R$ is an \emph{integral domain} if
$R$ has no zero divisors. Equivalently,
\[
a \cdot b = 0 \;\Longrightarrow\; a = 0 \;\text{ or }\; b = 0
\quad \text{for all } a, b \in R.
\]
\end{definition}

\begin{remark}[Connection to Linear Algebra]
The zero product property is the theorem cited in the vector space proof
that $av = \mathbf{0} \Rightarrow a = 0$ or $v = \mathbf{0}$.
Specifically, the case $a \neq 0$ uses the fact that the scalar field
$\mathbb{F}$ has no zero divisors --- which holds because every field
is an integral domain (proved in the next section).

The proof is not about vectors at all. It is about the scalar field.
\end{remark}

\begin{proposition}[Cancellation in Integral Domains]
\label{prop:domain-cancellation}
Let $R$ be an integral domain and $a, b, c \in R$ with $a \neq 0$.
Then
\[
ab = ac \;\Longrightarrow\; b = c.
\]
\end{proposition}


\begin{example}[Integral Domains and Non-Examples]
\begin{enumerate}[label=(\roman*)]
  \item $\mathbb{Z}$, $\mathbb{Q}$, $\mathbb{R}$, $\mathbb{C}$:
        all integral domains.
  \item $\mathbb{Z}[x]$: integral domain.
  \item $\mathbb{Z}/p\mathbb{Z}$ for prime $p$: integral domain
        (in fact a field, as shown in the next section).
  \item $\mathbb{Z}/6\mathbb{Z}$: \emph{not} an integral domain,
        since $[2][3] = [0]$ with $[2],[3] \neq [0]$.
  \item $M_2(\mathbb{R})$: \emph{not} an integral domain ---
        not commutative, and admits zero divisors.
\end{enumerate}
\end{example}


% =========================================================
% Fields
% =========================================================
\subsubsection{Fields}

% =========================================================
% Field Definition and Axioms
% =========================================================
\paragraph{Definition of a Field}

\begin{remark}[Motivation]
A ring allows addition, subtraction, and multiplication.
A field adds division: every nonzero element has a multiplicative inverse.
This is the structure of $\mathbb{Q}$, $\mathbb{R}$, and $\mathbb{C}$ ---
the number systems where you can always solve $ax = b$ for $a \neq 0$.

Fields are the scalars of linear algebra. Axler's $\mathbb{F}$ denotes
an arbitrary field, meaning every theorem in linear algebra holds
simultaneously over $\mathbb{R}$, $\mathbb{C}$, $\mathbb{Q}$,
and any other field --- because the proofs use only field axioms,
not properties specific to real or complex numbers.
\end{remark}

\begin{definition}[Field]
A \emph{field} is a commutative ring with unity $(F, +, \cdot)$
satisfying:

\begin{enumerate}[label=\textbf{F\arabic*.}]
  \item \textbf{Additive abelian group.}
        $(F, +)$ is an abelian group with identity $0$.

  \item \textbf{Multiplicative abelian group on nonzero elements.}
        $(F \setminus \{0\}, \cdot)$ is an abelian group with identity $1$.
        Explicitly: for each $a \in F$ with $a \neq 0$, there exists
        $a^{-1} \in F$ such that $a \cdot a^{-1} = 1$.

  \item \textbf{Distributivity.}
        $a \cdot (b + c) = a \cdot b + a \cdot c$
        for all $a, b, c \in F$.

  \item \textbf{Non-triviality.}
        $0 \neq 1$.
\end{enumerate}
\end{definition}

\begin{remark}[Unpacking the Definition]
A field is simultaneously:
\begin{itemize}
  \item $(F, +)$: an abelian group (additive structure),
  \item $(F \setminus \{0\}, \cdot)$: an abelian group (multiplicative structure),
  \item connected by distributivity.
\end{itemize}
The non-triviality axiom $0 \neq 1$ rules out the trivial ring $\{0\}$,
which would otherwise technically satisfy the other axioms.
\end{remark}

\begin{example}[Fields]
\begin{enumerate}[label=(\roman*)]
  \item $\mathbb{Q}$, $\mathbb{R}$, $\mathbb{C}$: the standard fields.
  \item $\mathbb{Z}/p\mathbb{Z}$ for any prime $p$: a finite field
        with $p$ elements, denoted $\mathbb{F}_p$.
  \item $\mathbb{Q}(\sqrt{2}) = \{a + b\sqrt{2} : a, b \in \mathbb{Q}\}$:
        a field extending $\mathbb{Q}$.
\end{enumerate}
\end{example}

\begin{example}[Non-Fields]
\begin{enumerate}[label=(\roman*)]
  \item $\mathbb{Z}$: not a field. The element $2$ has no multiplicative
        inverse in $\mathbb{Z}$.
  \item $\mathbb{Z}/6\mathbb{Z}$: not a field. $[2]$ has no inverse
        since $\gcd(2,6) \neq 1$. (Also not an integral domain.)
  \item $M_n(\mathbb{R})$ for $n \geq 2$: not a field.
        Not commutative, and singular matrices have no inverse.
\end{enumerate}
\end{example}

% =========================================================
% Basic Theorems of Fields
% =========================================================
\paragraph{Basic Theorems}

\begin{remark}[Why These Theorems Matter]
These are the theorems cited in every vector space proof that crosses
from the scalar side to the vector side. They live here, in the field,
and are cited by number when needed in linear algebra.
\end{remark}

\begin{proposition}[Every Field is an Integral Domain]
\label{prop:field-is-domain}
Every field is an integral domain.
\end{proposition}


\begin{remark}
This is the theorem behind the linear algebra argument:
if $av = \mathbf{0}$ and $a \neq 0$, then $v = \mathbf{0}$.
The scalar $a$ lives in a field $\mathbb{F}$, and the key step
is that $\mathbb{F}$ has no zero divisors.
\end{remark}

\begin{proposition}[Zero Product Property in a Field]
\label{prop:field-zero-product}
Let $\mathbb{F}$ be a field and $a, b \in \mathbb{F}$. Then
\[
ab = 0 \;\Longrightarrow\; a = 0 \;\text{ or }\; b = 0.
\]
\end{proposition}

\begin{proof}
Immediate from Proposition~\ref{prop:field-is-domain},
since every field is an integral domain.
\end{proof}

\begin{proposition}[Nonzero Scalars Have Inverses]
\label{prop:field-inverse-exists}
Let $\mathbb{F}$ be a field and $a \in \mathbb{F}$ with $a \neq 0$.
Then there exists a unique $a^{-1} \in \mathbb{F}$ such that
$a \cdot a^{-1} = 1$.
\end{proposition}

\begin{proof}
Existence is Axiom F2. Uniqueness follows from
Proposition~\ref{prop:group-inverse-unique} applied to the
multiplicative group $(F \setminus \{0\}, \cdot)$.
\end{proof}

\begin{remark}
This is the theorem cited in the vector space proof that
$av = \mathbf{0}$ with $a \neq 0$ implies $v = \mathbf{0}$:
the step ``multiply both sides by $a^{-1}$'' is valid
precisely because $a^{-1}$ exists and is unique.
\end{remark}

\begin{proposition}[Characteristic of a Field]
\label{prop:field-characteristic}
Let $\mathbb{F}$ be a field. The \emph{characteristic} of $\mathbb{F}$
is the smallest positive integer $n$ such that
\[
\underbrace{1 + 1 + \cdots + 1}_{n} = 0,
\]
or $0$ if no such $n$ exists.
The characteristic of a field is either $0$ or a prime $p$.
\end{proposition}

\begin{remark}
$\mathbb{Q}$, $\mathbb{R}$, and $\mathbb{C}$ all have characteristic $0$.
$\mathbb{Z}/p\mathbb{Z}$ has characteristic $p$.

In Axler, $\mathbb{F}$ denotes $\mathbb{R}$ or $\mathbb{C}$,
both of characteristic $0$. Some results (such as the existence
of eigenvalues) require characteristic $0$ or algebraic closure
and would fail over $\mathbb{F}_p$.
\end{remark}

% =========================================================
% Bridge to Linear Algebra
% =========================================================
\paragraph{Bridge to Linear Algebra}

\begin{remark}[Why This Chapter Exists]
Every proof in Axler's Chapter 1 draws on exactly two sources:
properties of the field $\mathbb{F}$, and properties of the
vector space $V$. This chapter is the permanent home of the
field side. The table below maps each type of linear algebra
proof step to the theorem it cites from this chapter.
\end{remark}

\begin{center}
\renewcommand{\arraystretch}{1.4}
\begin{tabular}{p{0.36\textwidth} p{0.26\textwidth} p{0.28\textwidth}}
\toprule
\textbf{Proof step in linear algebra} & \textbf{Structure used} & \textbf{Theorem} \\
\midrule
``since $a \neq 0$, $a^{-1}$ exists'' &
Field $\mathbb{F}$ &
Prop.~\ref{prop:field-inverse-exists} \\

``$ab = 0$ and $a \neq 0$ implies $b = 0$'' &
Field $\mathbb{F}$ &
Prop.~\ref{prop:field-zero-product} \\

``$(-1)v = -v$'' &
Ring theorem on $\mathbb{F}$ &
Prop.~\ref{prop:ring-mult-neg} \\

``$0 \cdot v = \mathbf{0}$'' &
Ring theorem on $\mathbb{F}$ acting on $V$ &
Prop.~\ref{prop:ring-mult-zero} \\

``the additive identity is unique'' &
Abelian group $(V, +)$ &
Prop.~\ref{prop:group-identity-unique} \\

``additive inverses are unique'' &
Abelian group $(V, +)$ &
Prop.~\ref{prop:group-inverse-unique} \\

``$-(-v) = v$'' &
Abelian group $(V, +)$ &
Prop.~\ref{prop:group-inverse-unique} \\
\bottomrule
\end{tabular}
\end{center}

\begin{remark}[The Interaction Layer]
The table above separates into two kinds of steps: those that use
the structure of the field $\mathbb{F}$ alone (rows 1--3), and those
that use the abelian group structure of $V$ alone (rows 5--7).
Row 4 is the interaction: it uses a field theorem applied across
scalar multiplication to a vector.

This separation is the heart of the vector space structure.
The field $\mathbb{F}$ and the abelian group $V$ are independent objects
connected by the scalar multiplication axioms. When a proof crosses
that connection, it is using the interaction layer. When it stays
on one side, it is using either field theory or group theory alone.
Knowing which side you are on at each step is what the DU/TA/AM
annotation system makes explicit.
\end{remark}


