% ======================================
% Johar — Chapter 5, Exercise 5.1 (a)–(h)
% Proof sheets (no proofs included)
% ======================================

% --------------------------------------
% Proof: RA-JOH-C05-S5-1-E01A
% --------------------------------------
\clearpage
\phantomsection
\hypertarget{proof-RA-JOH-C05-S5-1-E01A}{}

\section*{Proof — RA-JOH-C05-S5-1-E01A}

\noindent
\hyperlink{ws-RA-JOH-C05-S5-1-E01A}{\textbf{← Back to worksheet}}

\noindent
\textbf{Source.}
\srccite{JoharBigBookRealAnalysis}{Chapter~5, Exercise~5.1(a)}.

\vspace{0.75em}

\noindent
\textbf{Goal.}
Let $(a_n)$ be defined by
\[
a_n=\frac{1}{n^2+3}.
\]
Show that $a_n \to 0$. In $\varepsilon$--$N$ form: for a fixed $\varepsilon>0$, find
$N(\varepsilon)\in\mathbb{N}$ such that $|a_n|<\varepsilon$ for all $n\ge N(\varepsilon)$.

\vspace{0.75em}

\begin{proof} 

Let $\varepsilon > 0$ be arbitrary.

Let 
\[
N(\varepsilon) := \left\lceil \frac{1}{\sqrt{\varepsilon}} \right\rceil .
\] Then $n > N$ implies $n > \frac{1}{\sqrt{\varepsilon}}$ which implies $n^2 + 3 > \frac{1}{\varepsilon}$ and hence $\varepsilon > \frac{1}{n^2 + 3}.$ Thus $n > N$ implies $\left| {\frac{1}{n^2+3} - 0}  \right| < \varepsilon.$ This proves $\lim \frac{1}{n^2 + 3} = 0.$


\end{proof}

\vspace{1em}
\noindent
\hyperlink{ws-RA-JOH-C05-S5-1-E01A}{\textbf{← Back to worksheet}}


