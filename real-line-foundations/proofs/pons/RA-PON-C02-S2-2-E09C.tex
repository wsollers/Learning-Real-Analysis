% --------------------------------------
% Proof: RA-PON-C02-S2-2-E09C
% --------------------------------------
\clearpage
\phantomsection
\hypertarget{proof-RA-PON-C02-S2-2-E09C}{}

\section*{Proof — RA-PON-C02-S2-2-E09C}

\noindent
\hyperlink{ws-RA-PON-C02-S2-2-E09C}{\textbf{← Back to worksheet}}

\noindent
\textbf{Source.}
\srccite{PonsRealAnalysis}{Chapter~2, Exercise~2.2.9(c)}.

\vspace{0.75em}

\noindent
\textbf{Goal.}
If $(a_n)$ is bounded and $(b_n) \to 0$, show that $(a_n b_n) \to 0$.  
(Do not use the Algebraic Limit Theorem.)

\vspace{0.75em}

\begin{proof}

Let $(a_n)$ be bounded. Then there exists $M>0$ such that
$(\forall n\in\mathbb{N})(|a_n|\le M)$.

Let $(b_n)\to 0$. Let $\varepsilon>0$ be arbitrary.
Set $\delta := \varepsilon/M$. Since $b_n\to 0$, there exists $N\in\mathbb{N}$
such that for all $n\ge N$,
\[
|b_n-0|<\delta=\varepsilon/M.
\]

For $n\ge N$ we have
\[
|a_n b_n-0|
=|a_n||b_n|
\le M\cdot |b_n|
< M\cdot (\varepsilon/M)
=\varepsilon.
\]
Therefore $(a_n b_n)\to 0$.





\end{proof}

\vspace{1em}
\noindent
\hyperlink{ws-RA-PON-C02-S2-2-E09C}{\textbf{← Back to worksheet}}
