% =========================================================
% Intervals
% =========================================================

\subsection{Intervals in the Real Numbers}

Intervals are fundamental subsets of the real line defined using the order
relation on $\mathbb{R}$. Let $a,b \in \mathbb{R}$ with $a < b$.

% ---------------------------------------------------------
\subsubsection{Basic Definitions}
% ---------------------------------------------------------

\subsubsection*{Bounded Intervals}

\begin{definition}[Open Interval]
The \emph{open interval} from $a$ to $b$ is the set
\[
(a,b) := \{ x \in \mathbb{R} : a < x < b \}.
\]
\[
\forall x \, (x \in (a,b) \leftrightarrow a < x \wedge x < b)
\]
\end{definition}

\begin{definition}[Closed Interval]
The \emph{closed interval} from $a$ to $b$ is the set
\[
[a,b] := \{ x \in \mathbb{R} : a \le x \le b \}.
\]
\[
\forall x \, (x \in [a,b] \leftrightarrow a \le x \wedge x \le b)
\]
\end{definition}

\begin{definition}[Half-Open Intervals]
The \emph{left-closed, right-open interval} is
\[
[a,b) := \{ x \in \mathbb{R} : a \le x < b \}.
\]
The \emph{left-open, right-closed interval} is
\[
(a,b] := \{ x \in \mathbb{R} : a < x \le b \}.
\]
\end{definition}

\subsubsection*{Unbounded Intervals}

\begin{definition}[Open Rays]
The \emph{open rays} determined by $a \in \mathbb{R}$ are
\[
(a,\infty) := \{ x \in \mathbb{R} : x > a \},
\qquad
(-\infty,a) := \{ x \in \mathbb{R} : x < a \}.
\]
\end{definition}

\begin{definition}[Closed Rays]
The \emph{closed rays} determined by $a \in \mathbb{R}$ are
\[
[a,\infty) := \{ x \in \mathbb{R} : x \ge a \},
\qquad
(-\infty,a] := \{ x \in \mathbb{R} : x \le a \}.
\]
\end{definition}

\subsubsection*{Degenerate and Trivial Intervals}

\begin{definition}[Degenerate Interval]
If $a=b$, the closed interval
\[
[a,a] = \{a\}
\]
is called a \emph{degenerate interval}.
\end{definition}

\begin{definition}[Empty Interval]
If $a>b$, the set
\[
(a,b) = \varnothing
\]
is called an \emph{empty interval}.
\end{definition}

% ---------------------------------------------------------
\subsubsection{Main Theorems}
% ---------------------------------------------------------

\begin{theorem}[Characterization of Intervals]
A subset $I \subseteq \mathbb{R}$ is an interval if and only if
whenever $x,z \in I$ and $x < y < z$, then $y \in I$.
Equivalently,
\[
\forall x,z \in I,\; x<z \Rightarrow
\bigl(
\forall y \in \mathbb{R},\;
(x<y<z \Rightarrow y \in I)
\bigr).
\]
\end{theorem}

\begin{proof}
($\Rightarrow$) Each of the standard interval forms defined above
contains every real number between any two of its elements by definition.

($\Leftarrow$) Suppose $I$ satisfies the stated property.
Let $a = \inf I$ and $b = \sup I$ in the extended real line.
Using the order axioms and the interval property, one verifies that
$I$ must be one of the standard forms:
\[
(a,b),\ [a,b],\ (a,b],\ [a,b),
\]
or one of the unbounded variants.
\qed
\end{proof}

\begin{remark}
Intervals are precisely the subsets of $\mathbb{R}$ with the property that if
$x<y<z$ and $x,z$ belong to the set, then $y$ also belongs to the set.
\end{remark}

% ---------------------------------------------------------
\subsubsection{Consequences}
% ---------------------------------------------------------

The logical implication of this section is:

\[
\text{Order Structure of }\mathbb{R}
\Rightarrow
\text{Intervals}
\Rightarrow
\text{Convex subsets of }\mathbb{R}.
\]

Intervals encode the idea of “no gaps” between points of a set.
They are the basic building blocks for:

\begin{itemize}
\item neighborhoods,
\item open sets,
\item topology of $\mathbb{R}$,
\item and completeness arguments (e.g. nested intervals).
\end{itemize}

\begin{remark}[Logical Structure]
The major structural flow is:

\[
\text{Field Axioms}
\Rightarrow
\text{Order Axioms}
\Rightarrow
\text{Definition of Intervals}
\Rightarrow
\text{Convexity Property}
\Rightarrow
\text{Topological Structure of }\mathbb{R}.
\]
\end{remark}
