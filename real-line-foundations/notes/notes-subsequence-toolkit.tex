% =========================================================
% Subsequence Toolkit
% File: notes-subsequence-toolkit.tex
% =========================================================

\subsection{Subsequence Toolkit}

\begin{remark}[Relationship to Subsequences Notes]
The results in this section assume the foundational facts about subsequences
proved in the main subsequences section, including:
\begin{itemize}
\item Limits are inherited by subsequences.
\item Convergent sequences have unique limits.
\item Bolzano–Weierstrass Theorem.
\end{itemize}
This section abstracts and generalizes those structural principles.
\end{remark}

% =========================================================
% Partition Convergence Principles
% =========================================================

\subsubsection{Partition Convergence Principles}

\begin{theorem}[Finite Partition Convergence Principle]
Let $(a_n)$ be a sequence in $\mathbb{R}$.
Suppose the index set $\mathbb{N}$ admits a finite partition
\[
\mathbb{N} = E_1 \cup E_2 \cup \cdots \cup E_k,
\]
where:
\begin{enumerate}
\item $E_i \cap E_j = \varnothing$ whenever $i \neq j$,
\item each $E_i$ is infinite.
\end{enumerate}

For each $i \in \{1, \ldots, k\}$, let $(a_n)_{n \in E_i}$ denote the
subsequence indexed by $E_i$.

If each subsequence $(a_n)_{n \in E_i}$ converges to the same limit $L$,
then the full sequence $(a_n)$ converges to $L$.
\end{theorem}

\begin{proof}
Let $\varepsilon > 0$ be arbitrary.

For each $i \in \{1, \ldots, k\}$, since $(a_n)_{n \in E_i} \to L$,
there exists $N_i \in \mathbb{N}$ such that for all $n \in E_i$ with $n \geq N_i$,
\[
|a_n - L| < \varepsilon.
\]

Define
\[
N := \max\{N_1, \ldots, N_k\}.
\]

Now let $m \geq N$.
Since $\mathbb{N} = \bigcup_{i=1}^k E_i$, there exists $i$ such that $m \in E_i$.
Since the partition is disjoint, this $i$ is unique.

Because $m \geq N \geq N_i$ and $m \in E_i$, we have
\[
|a_m - L| < \varepsilon.
\]

Since $\varepsilon > 0$ was arbitrary, $(a_n) \to L$.
\end{proof}

\begin{remark}
The theorem generalizes to metric spaces: replace $|a_n - L|$ with $d(a_n, L)$.
\end{remark}

\begin{corollary}[Residue-Class Convergence]
Let $(a_n)$ be a sequence in $\mathbb{R}$ and let $k \geq 1$.
If for each $r \in \{0, 1, \ldots, k-1\}$ the subsequence $(a_{kn+r})$
converges to the same limit $L$, then $(a_n) \to L$.
\end{corollary}

\begin{proof}
For each $r \in \{0, 1, \ldots, k-1\}$, define
\[
E_r := \{kn + r : n \in \mathbb{N}_0\} = \{r, k+r, 2k+r, \ldots\}.
\]
By the Division Algorithm, every $m \in \mathbb{N}_0$ can be written uniquely as
$m = kn + r$ with $0 \leq r < k$, so the sets $E_0, E_1, \ldots, E_{k-1}$
partition $\mathbb{N}_0$. Each $E_r$ is infinite.

Apply the Finite Partition Convergence Principle.
\end{proof}

\begin{corollary}[Even-Odd Convergence]
Let $(a_n)$ be a sequence. If $(a_{2n}) \to L$ and $(a_{2n+1}) \to L$,
then $(a_n) \to L$.
\end{corollary}

\begin{proof}
Apply Residue-Class Convergence with $k = 2$.
\end{proof}

% =========================================================
% Inheritance of Sequence Properties
% =========================================================

\subsubsection{Inheritance of Sequence Properties}

\begin{definition}[Inherited Property]
A property $\mathcal{P}$ of sequences is \emph{inherited by subsequences}
if whenever $(a_n)$ satisfies $\mathcal{P}$, every subsequence $(a_{n_k})$
also satisfies $\mathcal{P}$.
\end{definition}

\begin{definition}[Reflected Property]
A property $\mathcal{P}$ of sequences is \emph{reflected by subsequences}
if whenever some subsequence $(a_{n_k})$ satisfies $\mathcal{P}$,
the original sequence $(a_n)$ also satisfies $\mathcal{P}$.
\end{definition}

\begin{proposition}[Convergence is Inherited]
If $(a_n) \to L$, then every subsequence $(a_{n_k}) \to L$.
\end{proposition}

\begin{proof}
See the main subsequences section (Theorem: Subsequences Inherit Limits).
\end{proof}

\begin{proposition}[Boundedness is Inherited]
If $(a_n)$ is bounded, then every subsequence $(a_{n_k})$ is bounded.
\end{proposition}

\begin{proof}
Suppose $|a_n| \leq M$ for all $n \in \mathbb{N}$.
For any subsequence $(a_{n_k})$, we have $n_k \in \mathbb{N}$ for each $k$,
so $|a_{n_k}| \leq M$ for all $k$.
Thus $(a_{n_k})$ is bounded by $M$.
\end{proof}

\begin{proposition}[Monotonicity is Inherited]
If $(a_n)$ is monotone increasing (resp.\ decreasing),
then every subsequence $(a_{n_k})$ is monotone increasing (resp.\ decreasing).
\end{proposition}

\begin{proof}
Suppose $(a_n)$ is increasing. Let $(a_{n_k})$ be a subsequence.
Since $(n_k)$ is strictly increasing, $n_k < n_{k+1}$, so
\[
a_{n_k} \leq a_{n_{k+1}}
\]
by monotonicity of $(a_n)$. Thus $(a_{n_k})$ is increasing.
The decreasing case is analogous.
\end{proof}

\begin{proposition}[Cauchy Property is Inherited]
If $(a_n)$ is Cauchy, then every subsequence $(a_{n_k})$ is Cauchy.
\end{proposition}

\begin{proof}
Let $\varepsilon > 0$. Since $(a_n)$ is Cauchy, there exists $N$ such that
for all $m, n \geq N$, $|a_m - a_n| < \varepsilon$.

Since $(n_k)$ is strictly increasing, $n_k \to \infty$.
Thus there exists $K$ such that $k \geq K$ implies $n_k \geq N$.

For $j, k \geq K$, we have $n_j, n_k \geq N$, so
\[
|a_{n_j} - a_{n_k}| < \varepsilon.
\]
Thus $(a_{n_k})$ is Cauchy.
\end{proof}

\begin{example}[Convergence is Not Reflected]
The sequence $a_n = (-1)^n$ has the convergent subsequence
\[
(a_{2n}) = (1, 1, 1, \ldots) \to 1,
\]
but $(a_n)$ itself does not converge.

Similarly, $(a_{2n+1}) = (-1, -1, -1, \ldots) \to -1$.

This shows that a divergent sequence can have convergent subsequences,
so convergence is inherited but not reflected.
\end{example}

\begin{example}[Boundedness is Not Reflected]
Let $a_n = n$ for $n$ odd and $a_n = 0$ for $n$ even.
The subsequence $(a_{2n}) = (0, 0, 0, \ldots)$ is bounded,
but $(a_n)$ is unbounded.
\end{example}

\begin{example}[Monotonicity is Not Reflected]
The sequence $a_n = (-1)^n$ has the monotone (constant) subsequence 
$(a_{2n}) = (1, 1, 1, \ldots)$, but $(a_n)$ is not monotone.
\end{example}

% =========================================================
% Additional Subsequence Tools
% =========================================================

\subsubsection{Additional Subsequence Tools}

\begin{proposition}[Subsequence of a Subsequence]
A subsequence of a subsequence is a subsequence of the original sequence.
\end{proposition}

\begin{proof}
Let $(a_{n_k})$ be a subsequence of $(a_n)$, where $(n_k)$ is strictly increasing.
Let $(a_{n_{k_j}})$ be a subsequence of $(a_{n_k})$, where $(k_j)$ is strictly increasing.

Define $m_j := n_{k_j}$. We claim $(m_j)$ is strictly increasing:
\[
m_{j+1} = n_{k_{j+1}} > n_{k_j} = m_j,
\]
since $k_{j+1} > k_j$ and $(n_k)$ is strictly increasing.

Thus $(a_{m_j}) = (a_{n_{k_j}})$ is a subsequence of $(a_n)$.
\end{proof}

\begin{theorem}[Dense Subsequence Criterion]
Let $(a_n)$ be a sequence and $L \in \mathbb{R}$.
If every neighborhood of $L$ contains infinitely many terms of $(a_n)$,
then $(a_n)$ has a subsequence converging to $L$.
\end{theorem}

\begin{proof}
We construct $(n_k)$ inductively.

For $k = 1$: The interval $(L - 1, L + 1)$ contains infinitely many terms.
Choose $n_1$ such that $a_{n_1} \in (L - 1, L + 1)$.

For $k \geq 1$: Suppose $n_1 < n_2 < \cdots < n_k$ have been chosen with
$|a_{n_j} - L| < 1/j$ for $j = 1, \ldots, k$.

The interval $(L - \frac{1}{k+1}, L + \frac{1}{k+1})$ contains infinitely many terms,
so there exists $n_{k+1} > n_k$ with $|a_{n_{k+1}} - L| < \frac{1}{k+1}$.

By construction, $|a_{n_k} - L| < \frac{1}{k} \to 0$, so $a_{n_k} \to L$.
\end{proof}

\begin{remark}[Connection to Cluster Points]
The Dense Subsequence Criterion characterizes subsequential limits:
$L$ is a subsequential limit of $(a_n)$ if and only if every neighborhood
of $L$ contains infinitely many terms of $(a_n)$. In topological language,
$L$ is a \emph{cluster point} (or \emph{accumulation point}) of the sequence.
\end{remark}

\begin{theorem}[Diagonal Subsequence Lemma]
Let $(a_n^{(m)})_{n=1}^{\infty}$ be a sequence of sequences indexed by $m \in \mathbb{N}$.
Suppose that for each $m$, the sequence $(a_n^{(m)})_{n=1}^{\infty}$ converges to a limit $L_m$,
and suppose $L_m \to L$ as $m \to \infty$.

Then there exists a strictly increasing sequence $(n_m)$ such that
the diagonal sequence $(a_{n_m}^{(m)})$ converges to $L$.
\end{theorem}

\begin{proof}
We construct $(n_m)$ inductively.

Since $L_m \to L$, for each $\varepsilon > 0$, there exists $M$ such that
$m \geq M$ implies $|L_m - L| < \varepsilon$.

For each $m$, since $a_n^{(m)} \to L_m$, there exists $N_m$ such that
$n \geq N_m$ implies $|a_n^{(m)} - L_m| < \frac{1}{m}$.

Define $n_1 := N_1$. Given $n_1 < n_2 < \cdots < n_{m-1}$, define
\[
n_m := \max\{n_{m-1} + 1, N_m\}.
\]
Then $(n_m)$ is strictly increasing and $n_m \geq N_m$.

For any $m$,
\begin{align*}
|a_{n_m}^{(m)} - L|
&\leq |a_{n_m}^{(m)} - L_m| + |L_m - L| \\
&< \frac{1}{m} + |L_m - L|.
\end{align*}

Given $\varepsilon > 0$, choose $M_1$ such that $m \geq M_1$ implies $|L_m - L| < \varepsilon/2$,
and choose $M_2$ such that $m \geq M_2$ implies $\frac{1}{m} < \varepsilon/2$.
For $m \geq \max\{M_1, M_2\}$,
\[
|a_{n_m}^{(m)} - L| < \varepsilon.
\]
Thus $a_{n_m}^{(m)} \to L$.
\end{proof}

\begin{remark}
The Diagonal Subsequence Lemma is used extensively in functional analysis
and measure theory, particularly in proofs involving weak convergence
and the construction of convergent subsequences from families of sequences.
\end{remark}

% =========================================================
% Structural Classification
% =========================================================

\subsubsection{Structural Classification of Sequence Properties}

\begin{definition}[Tail Property]
A property $\mathcal{P}$ of sequences is called a \emph{tail property}
if whenever two sequences $(a_n)$ and $(b_n)$ satisfy
\[
a_n = b_n \quad \text{for all } n \geq N
\]
for some $N \in \mathbb{N}$, then
\[
(a_n) \text{ satisfies } \mathcal{P}
\quad \Longleftrightarrow \quad
(b_n) \text{ satisfies } \mathcal{P}.
\]
In other words, altering finitely many initial terms does not affect
whether the sequence has property $\mathcal{P}$.
\end{definition}

\begin{definition}[Universal Subsequence Property]
A property $\mathcal{P}$ of sequences is called a
\emph{universal subsequence property}
if for every sequence $(a_n)$,
\[
(a_n) \text{ satisfies } \mathcal{P}
\quad \Longleftrightarrow \quad
\text{every subsequence of } (a_n) \text{ satisfies } \mathcal{P}.
\]
\end{definition}

\begin{remark}[Universal Properties and Inheritance]
Universal subsequence properties are necessarily inherited.
If $\mathcal{P}$ is universal and $(a_n)$ satisfies $\mathcal{P}$,
then by definition every subsequence satisfies $\mathcal{P}$.
\end{remark}

\begin{proposition}[Reflection via Universal Properties]
Let $\mathcal{P}$ be a universal subsequence property. Then:
\begin{enumerate}
\item $\mathcal{P}$ is inherited by subsequences.
\item $\mathcal{P}$ is reflected by the collection of \emph{all} subsequences
      (though not necessarily by any single subsequence).
\end{enumerate}
\end{proposition}

\begin{proof}
(1) If $(a_n)$ satisfies $\mathcal{P}$, then by definition of universal property,
every subsequence satisfies $\mathcal{P}$.

(2) If every subsequence of $(a_n)$ satisfies $\mathcal{P}$, then in particular
$(a_n)$ itself (viewed as the subsequence with $n_k = k$) satisfies $\mathcal{P}$.
\end{proof}

\begin{example}[Universal Property: Every Subsequence is Cauchy]
The property ``every subsequence is Cauchy'' is a universal subsequence property.
Moreover, this property is equivalent to the sequence being Cauchy:
\begin{itemize}
\item If $(a_n)$ is Cauchy, then every subsequence is Cauchy (inheritance).
\item If every subsequence is Cauchy, then $(a_n)$ is Cauchy (since $(a_n)$ is a subsequence of itself).
\end{itemize}
In $\mathbb{R}$, this is equivalent to convergence.
\end{example}

\begin{example}[Universal Property: Every Subsequence Has a Convergent Subsubsequence]
The property ``every subsequence has a convergent subsubsequence'' 
is equivalent to boundedness:
\begin{itemize}
\item If $(a_n)$ is bounded, then every subsequence is bounded (inheritance),
      and by Bolzano--Weierstrass, every bounded subsequence has a convergent subsubsequence.
\item Conversely, if $(a_n)$ is unbounded, then it has a subsequence $(a_{n_k})$
      with $|a_{n_k}| \to \infty$. This subsequence has no bounded subsubsequence,
      hence no convergent subsubsequence.
\end{itemize}
This characterization connects subsequence logic to compactness:
a set $K \subseteq \mathbb{R}$ is sequentially compact if and only if
every sequence in $K$ has a convergent subsequence with limit in $K$.
\end{example}

\begin{example}[Tail Properties]
The following are tail properties:
\begin{itemize}
\item Convergence (and convergence to a specific limit $L$)
\item Cauchy property
\item Boundedness
\item Eventually monotone
\item Eventually constant
\end{itemize}
\end{example}

\begin{example}[Non-Tail Properties]
The following are \emph{not} tail properties:
\begin{itemize}
\item Monotonicity (changing $a_1$ can destroy monotonicity)
\item Having $a_1 = 0$
\end{itemize}
\end{example}

% =========================================================
% Summary Table
% =========================================================

\subsubsection{Summary of Property Classification}

\begin{center}
\begin{tabular}{|l|c|c|l|}
\hline
\textbf{Property} 
& \makecell{\textbf{Inherited by} \\ \textbf{Subsequences?}} 
& \makecell{\textbf{Reflected by One} \\ \textbf{Subsequence?}}
& \textbf{Structural Type} \\
\hline
Convergent 
& Yes 
& No 
& Tail Property \\
\hline
Cauchy 
& Yes 
& No 
& Tail Property \\
\hline
Bounded 
& Yes 
& No 
& Tail Property\textsuperscript{*} \\
\hline
Monotone 
& Yes 
& No 
& Global Structural \\
\hline
Eventually monotone 
& Yes 
& No 
& Tail Property \\
\hline
Every subseq.\ convergent 
& Yes 
& Yes (all) 
& Universal Property \\
\hline
Every subseq.\ Cauchy 
& Yes 
& Yes (all) 
& Universal Property \\
\hline
\end{tabular}
\end{center}

\smallskip

\noindent\textsuperscript{*}\textit{Boundedness is a tail property 
(changing finitely many terms preserves boundedness) and also a global 
magnitude condition (it constrains all terms).}

\begin{remark}[Interpretation of the Table]
\emph{Inherited} means: if $(a_n)$ has property $P$, then every subsequence has $P$.

\emph{Reflected by one} means: if some single subsequence has property $P$, 
then $(a_n)$ has $P$. This is rare.

\emph{Reflected by all} means: if every subsequence has property $P$,
then $(a_n)$ has $P$. This holds for universal properties.

Most natural properties are inherited but not reflected by any single subsequence.
Universal properties are both inherited and reflected by the totality of subsequences.
\end{remark}

% =========================================================
% Structural Summary
% =========================================================

\begin{remark}[Hierarchy of Subsequence Properties]
The structural types form a hierarchy:
\begin{center}
\begin{tabular}{ll}
\textbf{Type} & \textbf{Characterization} \\
\hline
Tail Property & Determined by sufficiently late terms \\
Global Property & Depends on all terms (not just tail) \\
Universal Property & Equivalent to holding for all subsequences \\
\end{tabular}
\end{center}

Understanding this hierarchy clarifies which proof strategies apply:
\begin{itemize}
\item Tail properties allow discarding finitely many ``bad'' initial terms.
\item Global properties require controlling the entire sequence.
\item Universal properties can be verified by checking all subsequences,
      or refuted by finding one counterexample subsequence.
\end{itemize}
\end{remark}