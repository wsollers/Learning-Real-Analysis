% =========================================================
% Multiplication Toolkit (Tao §2.3)
% =========================================================

\begin{tcolorbox}[colback=gray!6, colframe=gray!40, arc=2pt,
  left=6pt, right=6pt, top=4pt, bottom=4pt,
  title={\small\textbf{Multiplication Toolkit — Quick Reference}},
  fonttitle=\small\bfseries]
\small
\begin{tabular}{l l l}
\toprule
\textbf{Label} & \textbf{Statement} & \textbf{Key tool} \\
\midrule
M1 & $0 \times m = 0$ & Definition \\
M2 & $(n\pp) \times m = (n \times m) + m$ & Definition \\
L2.3.2 & $n \times m = m \times n$ & Induction \\
L2.3.3 & $nm = 0 \iff n = 0$ or $m = 0$ & P2.2.8 (contrapositive) \\
P2.3.4 & $a(b+c) = ab + ac$ (distributive law) & Induction on $c$ \\
P2.3.5 & $(a \times b) \times c = a \times (b \times c)$ & Induction \\
P2.3.6 & $a < b,\ c > 0 \Rightarrow ac < bc$ & P2.2.12(f), P2.3.4 \\
C2.3.7 & $ac = bc,\ c \neq 0 \Rightarrow a = b$ & P2.2.13 + P2.3.6 \\
P2.3.9 & Euclidean algorithm: $\exists\, m, r,\ n = mq + r,\ 0 \leq r < q$
        & Induction \\
\bottomrule
\end{tabular}
\end{tcolorbox}

\vspace{1em}

% ---------------------------------------------------------
% L2.3.2
% ---------------------------------------------------------
\begin{lemma}[Tao 2.3.2 — Commutativity]
\label{lem:mult-comm}
For all $n, m \in \mathbb{N}$,\quad $n \times m = m \times n$.
\end{lemma}

% ---------------------------------------------------------
% L2.3.3
% ---------------------------------------------------------
\begin{lemma}[Tao 2.3.3 — No zero divisors]
\label{lem:no-zero-div}
For all $n, m \in \mathbb{N}$,\quad $n \times m = 0$ if and only if
$n = 0$ or $m = 0$.

In particular, if $n$ and $m$ are both positive, then $nm$ is positive.
\end{lemma}

% ---------------------------------------------------------
% P2.3.4
% ---------------------------------------------------------
\begin{proposition}[Tao 2.3.4 — Distributive law]
\label{prop:distrib}
For all $a, b, c \in \mathbb{N}$:
\[
a(b+c) = ab + ac \qquad \text{and} \qquad (b+c)a = ba + ca.
\]
\end{proposition}

\begin{remark}[Proof strategy]
The second identity follows from the first by commutativity (L2.3.2).
For the first, induct on $c$ (the right summand), keeping $a$ and $b$ fixed.
M2 fires on the left side; the inductive hypothesis closes the step.
\end{remark}

% ---------------------------------------------------------
% P2.3.5
% ---------------------------------------------------------
\begin{proposition}[Tao 2.3.5 — Associativity]
\label{prop:mult-assoc}
For all $a, b, c \in \mathbb{N}$,\quad $(a \times b) \times c = a \times (b \times c)$.
\end{proposition}

% ---------------------------------------------------------
% P2.3.6
% ---------------------------------------------------------
\begin{proposition}[Tao 2.3.6 — Multiplication preserves order]
\label{prop:mult-order}
If $a < b$ and $c$ is positive, then $ac < bc$.
\end{proposition}

% ---------------------------------------------------------
% C2.3.7
% ---------------------------------------------------------
\begin{corollary}[Tao 2.3.7 — Cancellation for multiplication]
\label{cor:mult-cancel}
Let $a, b, c \in \mathbb{N}$ with $ac = bc$ and $c \neq 0$. Then $a = b$.
\end{corollary}

\begin{remark}[Virtual division]
This is the multiplicative analogue of the additive cancellation law
(P2.2.6). It provides ``virtual division'' before division is defined,
and will be used to define $\mathbb{Q}$ later.
\end{remark}

% ---------------------------------------------------------
% P2.3.9
% ---------------------------------------------------------
\begin{proposition}[Tao 2.3.9 — Euclidean algorithm]
\label{prop:euclid}
Let $n \in \mathbb{N}$ and let $q$ be positive. Then there exist
$m, r \in \mathbb{N}$ such that:
\[
0 \leq r < q \qquad \text{and} \qquad n = mq + r.
\]
\end{proposition}

\begin{remark}
This is the seed of number theory: every natural number has a
unique quotient and remainder when divided by a positive number.
\end{remark}
