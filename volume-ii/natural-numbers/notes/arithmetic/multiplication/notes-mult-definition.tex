% =========================================================
% Definition of Multiplication (Tao Def. 2.3.1)
% =========================================================

\begin{tcolorbox}[colback=propbox, colframe=propborder, arc=2pt,
  left=6pt, right=6pt, top=4pt, bottom=4pt,
  title={\small\textbf{Definition 2.3.1 (Multiplication)}},
  fonttitle=\small\bfseries]
Let $m \in \mathbb{N}$ be fixed. Multiplication is defined recursively
on the \textbf{left} argument:
\[
\textbf{M1:}\quad 0 \times m \;:=\; 0
\]
\[
\textbf{M2:}\quad (n\pp) \times m \;:=\; (n \times m) + m
\]
\end{tcolorbox}

\begin{remark}[Multiplication as iterated addition]
M2 says: multiplying $n\pp$ by $m$ is the same as multiplying $n$ by $m$
and then adding one more copy of $m$.
Thus $n \times m$ adds $m$ to itself $n$ times:
\[
0 \times m = 0, \quad
1 \times m = 0 + m, \quad
2 \times m = 0 + m + m, \quad \ldots
\]
\end{remark}

\begin{remark}[Same asymmetry as addition]
M1 and M2 recurse on the left argument only.
Commutativity $n \times m = m \times n$ is again a theorem,
not a definition.
\end{remark}

\begin{definition}[Exponentiation — Tao 2.3.11]
\label{def:exp}
Let $m \in \mathbb{N}$. Exponentiation is defined recursively:
\[
m^0 := 1, \qquad m^{n\pp} := m^n \times m.
\]
In particular, $0^0 := 1$.
\end{definition}
