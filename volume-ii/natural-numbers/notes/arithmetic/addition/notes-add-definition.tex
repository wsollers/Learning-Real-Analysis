% =========================================================
% Definition of Addition (Tao Def. 2.2.1)
% =========================================================

\begin{tcolorbox}[colback=propbox, colframe=propborder, arc=2pt,
  left=6pt, right=6pt, top=4pt, bottom=4pt,
  title={\small\textbf{Definition 2.2.1 (Addition)}},
  fonttitle=\small\bfseries]
Let $m \in \mathbb{N}$ be fixed. Addition is defined recursively
on the \textbf{left} argument:
\[
\textbf{A1:}\quad 0 + m \;:=\; m
\]
\[
\textbf{A2:}\quad (n\pp) + m \;:=\; (n + m)\pp
\]
\end{tcolorbox}

\begin{remark}[The definition is asymmetric]
A1 and A2 recurse on the \emph{left} argument only.
$3 + 5$ means: increment $5$ three times.
$5 + 3$ means: increment $3$ five times.
That both give $8$ is a \emph{theorem} (commutativity), not a
definition. Never assume $a + b = b + a$ until it is proved.
\end{remark}

\begin{remark}[Consequence for proof strategy]
When proving an identity involving addition by induction,
the induction variable must be in the \emph{left} position of
the outermost addition on both sides, so that A2 fires cleanly.
\end{remark}
