% =========================================================
% Addition Toolkit (Tao §2.2, in order)
% =========================================================

% ---------------------------------------------------------
% Reference table
% ---------------------------------------------------------
\begin{tcolorbox}[colback=gray!6, colframe=gray!40, arc=2pt,
  left=6pt, right=6pt, top=4pt, bottom=4pt,
  title={\small\textbf{Addition Toolkit — Quick Reference}},
  fonttitle=\small\bfseries]
\small
\begin{tabular}{l l l}
\toprule
\textbf{Label} & \textbf{Statement} & \textbf{Proof method} \\
\midrule
A1 & $0 + m = m$ & Definition \\
A2 & $(n\pp) + m = (n+m)\pp$ & Definition \\
L2.2.2 & $n + 0 = n$ & Induction on $n$ \\
L2.2.3 & $n + (m\pp) = (n+m)\pp$ & Induction on $n$ \\
P2.2.4 & $n + m = m + n$ & Induction on $n$, uses L2.2.2, L2.2.3 \\
P2.2.5 & $(a+b)+c = a+(b+c)$ & Induction on $a$, uses A1, A2 \\
P2.2.6 & $a+b = a+c \Rightarrow b = c$ & Induction on $a$, uses P4 \\
Def 2.2.7 & $n$ positive $\iff$ $n \neq 0$ & Definition \\
P2.2.8 & $a$ positive, $b \in \mathbb{N}$ $\Rightarrow$ $a+b$ positive
        & Induction on $b$, uses P3 \\
C2.2.9 & $a+b=0 \Rightarrow a=0$ and $b=0$ & Contradiction + P2.2.8 \\
L2.2.10 & $a$ positive $\Rightarrow$ $\exists!\ b,\ b\pp = a$
         & Induction (existence); P4 (uniqueness) \\
\bottomrule
\end{tabular}
\end{tcolorbox}

\vspace{1em}

% ---------------------------------------------------------
% L2.2.2
% ---------------------------------------------------------
\begin{lemma}[Tao 2.2.2]
\label{lem:add-zero-right}
For all $n \in \mathbb{N}$,\quad $n + 0 = n$.
\end{lemma}

\begin{remark}[Proof]
See \hyperref[prf:add-zero-right]{Proof $\to$ NN-TAO-L2.2.2 (Zero on the Right)}.
\end{remark}

\begin{remark}
A1 gives $0 + m = m$ (zero on the left).
This lemma gives $n + 0 = n$ (zero on the right).
The asymmetry of the definition means these are different facts.
Commutativity cannot be invoked to pass from one to the other
because commutativity has not been proved yet.
\end{remark}

% ---------------------------------------------------------
% L2.2.3
% ---------------------------------------------------------
\begin{lemma}[Tao 2.2.3]
\label{lem:add-succ-right}
For all $n, m \in \mathbb{N}$,\quad $n + (m\pp) = (n+m)\pp$.
\end{lemma}

\begin{remark}[Proof]
See \hyperref[prf:add-succ-right]{Proof $\to$ NN-TAO-L2.2.3 (Successor on the Right)}.
\end{remark}

\begin{remark}
A2 gives $(n\pp) + m = (n+m)\pp$ (successor on the left).
This lemma gives $n + (m\pp) = (n+m)\pp$ (successor on the right).
Again, a different fact requiring its own proof.
\end{remark}

\begin{corollary}
\label{cor:succ-is-add-one}
For all $n \in \mathbb{N}$,\quad $n\pp = n + 1$.
\end{corollary}
\begin{proof}
$n + 1 = n + (0\pp) = (n + 0)\pp = n\pp$ by L2.2.3 then L2.2.2.
\end{proof}

% ---------------------------------------------------------
% P2.2.4
% ---------------------------------------------------------
\begin{proposition}[Tao 2.2.4 --- Commutativity]
\label{prop:add-comm}
For all $n, m \in \mathbb{N}$,\quad $n + m = m + n$.
\end{proposition}

\begin{remark}[Proof]
See \hyperref[prf:add-comm]{Proof $\to$ NN-TAO-P2.2.4 (Commutativity)}.
\end{remark}

% ---------------------------------------------------------
% P2.2.5
% ---------------------------------------------------------
\begin{proposition}[Tao 2.2.5 --- Associativity]
\label{prop:add-assoc}
For all $a, b, c \in \mathbb{N}$,\quad $(a+b)+c = a+(b+c)$.
\end{proposition}

\begin{remark}[Induction variable]
Induct on $a$ (the left argument of the outermost addition on both sides).
A2 fires on both sides simultaneously only when $a$ is the variable.
\end{remark}

% ---------------------------------------------------------
% P2.2.6
% ---------------------------------------------------------
\begin{proposition}[Tao 2.2.6 --- Cancellation]
\label{prop:add-cancel}
Let $a, b, c \in \mathbb{N}$ with $a + b = a + c$. Then $b = c$.
\end{proposition}

\begin{remark}[Proof]
See \hyperref[prf:add-cancel]{Proof $\to$ NN-TAO-P2.2.6 (Cancellation)}.
\end{remark}

\begin{remark}[Virtual subtraction]
Subtraction does not exist in $\mathbb{N}$ yet. This proposition
provides a ``virtual subtraction'': it lets us cancel $a$ from both
sides of an equation without ever invoking $-$.
It will later be used to \emph{define} subtraction on $\mathbb{Z}$.
\end{remark}

% ---------------------------------------------------------
% Def 2.2.7
% ---------------------------------------------------------
\begin{definition}[Tao 2.2.7 — Positive natural numbers]
\label{def:positive}
A natural number $n$ is \emph{positive} if and only if $n \neq 0$.
\end{definition}

% ---------------------------------------------------------
% P2.2.8
% ---------------------------------------------------------
\begin{proposition}[Tao 2.2.8]
\label{prop:pos-add-pos}
If $a$ is positive and $b \in \mathbb{N}$, then $a + b$ is positive
(and hence $b + a$ is positive by commutativity).
\end{proposition}

% ---------------------------------------------------------
% C2.2.9
% ---------------------------------------------------------
\begin{corollary}[Tao 2.2.9]
\label{cor:sum-zero}
If $a, b \in \mathbb{N}$ and $a + b = 0$, then $a = 0$ and $b = 0$.
\end{corollary}

% ---------------------------------------------------------
% L2.2.10
% ---------------------------------------------------------
\begin{lemma}[Tao 2.2.10 — Predecessor]
\label{lem:predecessor}
Let $a$ be a positive natural number. Then there exists exactly one
$b \in \mathbb{N}$ such that $b\pp = a$.
\end{lemma}

\begin{remark}[Two-part proof]
Existence: induction on $a$. Base case is vacuous ($a=0$ excluded by
positivity). Inductive step: witness $b = a$ satisfies $b\pp = a\pp$
directly; inductive hypothesis is not needed.

Uniqueness: if $b\pp = a$ and $c\pp = a$ then $b\pp = c\pp$,
so $b = c$ by P4. One axiom citation closes uniqueness entirely.
\end{remark}
