% --------------------------------------
% Proof: RA-MEN-C02-S2.4-E9a
% --------------------------------------
\clearpage
\phantomsection
\hypertarget{proof-RA-MEN-C02-S2.4-E9a}{}

\subsubsection[Uniqueness of Greatest Element]{Proof --- RA-MEN-C02-S2.4-E9a}

\bigskip

\noindent
\textbf{Source.}
\srccite{Mendelson2008NumberSystems}{Chapter~2, \S2.4, Exercise~9(a)}.

\vspace{0.75em}

\noindent
\textbf{Definition.}
By a \emph{greatest element} of a set $B \subseteq P$ we mean an element
$z \in B$ such that $(\forall u)(u \in B \Rightarrow u \leq z)$.

\vspace{0.75em}

\noindent
\textbf{Goal.}
A set has at most one greatest element.

\vspace{0.75em}

\noindent
\textbf{Logical form.}
$\forall B \subseteq P,\; \forall z, z' \in B$: if $z$ and $z'$ are both
greatest elements of $B$, then $z = z'$.

\vspace{0.75em}

\noindent
\textbf{Proof strategy.}
Direct proof. Assume two greatest elements exist; each bounds the other
by the definition of greatest element; antisymmetry closes the argument.

\vspace{0.75em}

\noindent
\textbf{Proof.}
\begin{proof}
Let $B \subseteq P$ and suppose $z$ and $z'$ are both greatest elements
of $B$. We show $z = z'$.

Since $z$ is a greatest element of $B$ and $z' \in B$:
\[
  z' \leq z.
\]
Since $z'$ is a greatest element of $B$ and $z \in B$:
\[
  z \leq z'.
\]
% Apply antisymmetry of $\leq$ to conclude $z = z'$.
\end{proof}

\vspace{0.75em}

\noindent
\textbf{Remarks.}
\begin{itemize}
  \item \textbf{Note:} This argument is purely order-theoretic and applies
    verbatim to any antisymmetric relation — partial orders and total orders
    alike.
  \item \textbf{Mistakes to avoid:} Do not assume $B$ is nonempty or that a
    greatest element exists. The statement is ``at most one,'' so the proof
    only handles the case where two exist and shows they coincide.
\end{itemize}
