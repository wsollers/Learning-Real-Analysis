% --------------------------------------
% Proof: RA-MEN-C02-S2.4-E9c
% --------------------------------------
\clearpage
\phantomsection
\hypertarget{proof-RA-MEN-C02-S2.4-E9c}{}

\subsubsection[Nonempty Set Bounded Above Has a Greatest Element]{Proof --- RA-MEN-C02-S2.4-E9c}

\bigskip

\noindent
\textbf{Source.}
\srccite{Mendelson2008NumberSystems}{Chapter~2, \S2.4, Exercise~9(c)}.

\vspace{0.75em}

\noindent
\textbf{Goal.}
If $\varnothing \neq A \subseteq P$ and $A$ is bounded above, that is,
$(\exists w)(\forall u)(u \in A \Rightarrow u \leq w)$, then $A$ has a
greatest element.

\vspace{0.75em}

\noindent
\textbf{Logical form.}
$\forall A \subseteq P$: $A \neq \varnothing \land A$ bounded above
$\Rightarrow$ $A$ has a greatest element.

\vspace{0.75em}

\noindent
\textbf{Hint} (Mendelson).
Let $B = \{w : (\forall u)(u \in A \Rightarrow u \leq w)\}$.
By hypothesis, $B \neq \varnothing$. Apply the Least Number Principle.

\vspace{0.75em}

\noindent
\textbf{Proof strategy.}
Construct the set $B$ of all upper bounds of $A$; apply the Least Number
Principle to obtain the least upper bound $w_0$; then verify (i) $w_0 \in A$
and (ii) $w_0$ is an upper bound for $A$. Part (ii) is immediate from
$w_0 \in B$; part (i) is the crux and requires a separate argument.

\vspace{0.75em}

\noindent
\textbf{Proof.}
\begin{proof}
Define the set of upper bounds of $A$:
\[
  B = \{ w \in P : (\forall u)(u \in A \Rightarrow u \leq w) \}.
\]
By the hypothesis that $A$ is bounded above, $B \neq \varnothing$.

Apply the Least Number Principle to $B$: since $B$ is a nonempty subset
of $P$, it has a least element. Call it $w_0$.

% Verify (i) w_0 in A and (ii) w_0 is an upper bound for A.
% (ii) is immediate: w_0 in B means w_0 is an upper bound for A.
% (i) requires showing w_0 must actually belong to A.
\end{proof}

\vspace{0.75em}

\noindent
\textbf{Remarks.}
\begin{itemize}
  \item \textbf{This is the maximum vs.\ supremum distinction:} In $\mathbb{N}$
    (or any well-ordered set), a nonempty set bounded above always has a
    greatest element --- the supremum is attained. In $\mathbb{Q}$ or
    $\mathbb{R}$ this fails (e.g., $\{x \in \mathbb{Q} : x < 1\}$ has
    supremum~1 but no greatest element). The well-ordering is doing
    essential work here.
  \item \textbf{Mistakes to avoid:} Do not assume $w_0 \in A$ without
    proof. This is the crux of the argument. Also confirm $B \subseteq P$
    so the Least Number Principle applies.
  \item \textbf{Forward reference:} In $\mathbb{R}$, every nonempty set
    bounded above has a \emph{supremum}, but not necessarily a
    \emph{maximum}. The distinction traces back to this exercise.
\end{itemize}
