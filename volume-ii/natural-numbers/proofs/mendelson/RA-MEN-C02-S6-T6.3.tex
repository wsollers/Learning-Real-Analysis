% --------------------------------------
% Proof: RA-MEN-C02-S6-T6.3
% --------------------------------------
\clearpage
\phantomsection
\hypertarget{proof-RA-MEN-C02-S6-T6.3}{}

\subsubsection[Basic Properties of Exponentiation (Thm 6.3)]{Proof --- RA-MEN-C02-S6-T6.3}

\bigskip

\noindent
\textbf{Source.}
\srccite{Mendelson2008NumberSystems}{Chapter~2, \S2.6, Theorem~6.3}.

\vspace{0.75em}

\noindent
\textbf{Goal.}
Prove the three basic laws of exponentiation for natural numbers.
For all $x, y, z \in \mathbb{N}$:

\medskip
\begin{center}
\begin{tabular}{ll}
\textbf{(a)} & $x^{y+z} = x^y \times x^z$ \quad (exponent addition law) \\[4pt]
\textbf{(b)} & $(x^y)^z = x^{y \times z}$ \quad (power of a power law) \\[4pt]
\textbf{(c)} & $(x \times y)^z = x^z \times y^z$ \quad (product power law)
\end{tabular}
\end{center}

\vspace{0.75em}

\noindent
\textbf{Logical form.}
$\forall x, y, z \in \mathbb{N}$: (a) $x^{y+z} = x^y \times x^z$;
\quad (b) $(x^y)^z = x^{y \times z}$;
\quad (c) $(xy)^z = x^z y^z$.

\vspace{0.75em}

\noindent
\textbf{Available toolkit.}
Def~2.3.11 (exponentiation: $x^0 := 1$, $x^{n\pp} := x^n \times x$),
L2.2.2 ($n+0=n$), L2.2.3 ($n+m\pp = (n+m)\pp$),
M1 ($0 \times m = 0$), M2 ($(n\pp)\times m = n \times m + m$),
P2.2.5 (associativity of $+$), P2.3.4 (distributive law),
P2.3.5 (associativity of $\times$), L2.3.2 (commutativity of $\times$),
P5 (induction).

\vspace{0.75em}

\noindent
\textbf{Strategy.}
Each part is proved by induction on $z$, with $x$ (and $y$) held fixed.
For each part: identify the induction variable, write out what the
definition of $x^{z\pp}$ gives you, apply the inductive hypothesis,
then use the toolkit to close.

\vspace{0.75em}

\noindent
\textbf{Dependency.}
Part (b) uses part (a) in its inductive step.
Prove (a) first, then (b), then (c).

\vspace{0.75em}

\noindent
\textbf{$Q(z)$ chosen as:} \quad (fill in for each part before starting)

\vspace{2em}

% ---------------------------------------------------------
\noindent\rule{\textwidth}{0.4pt}
\noindent\textbf{Part (a): $x^{y+z} = x^y \times x^z$. \quad Induct on $z$.}

\medskip
\noindent\textbf{Base case} ($z = 0$):

\noindent
\begin{tabular}{T S J}
\toprule
\textbf{Tag} & \multicolumn{1}{p{0.44\textwidth}}{\textbf{Step}} & \textbf{Justification} \\
\midrule
\addlinespace[4pt]
& & \\[50pt]
\bottomrule
\end{tabular}

\vspace{1em}

\noindent\textbf{Inductive step} (assume $x^{y+z} = x^y \times x^z$, prove for $z\pp$):

\noindent
\begin{tabular}{T S J}
\toprule
\textbf{Tag} & \multicolumn{1}{p{0.44\textwidth}}{\textbf{Step}} & \textbf{Justification} \\
\midrule
\addlinespace[4pt]
& & \\[100pt]
\bottomrule
\end{tabular}

\vspace{1.5em}

% ---------------------------------------------------------
\noindent\rule{\textwidth}{0.4pt}
\noindent\textbf{Part (b): $(x^y)^z = x^{y \times z}$. \quad Induct on $z$.}

\medskip
\noindent\textbf{Base case} ($z = 0$):

\noindent
\begin{tabular}{T S J}
\toprule
\textbf{Tag} & \multicolumn{1}{p{0.44\textwidth}}{\textbf{Step}} & \textbf{Justification} \\
\midrule
\addlinespace[4pt]
& & \\[50pt]
\bottomrule
\end{tabular}

\vspace{1em}

\noindent\textbf{Inductive step} (assume $(x^y)^z = x^{y \times z}$, prove for $z\pp$):

\noindent
\begin{tabular}{T S J}
\toprule
\textbf{Tag} & \multicolumn{1}{p{0.44\textwidth}}{\textbf{Step}} & \textbf{Justification} \\
\midrule
\addlinespace[4pt]
& & \\[100pt]
\bottomrule
\end{tabular}

\vspace{1.5em}

% ---------------------------------------------------------
\noindent\rule{\textwidth}{0.4pt}
\noindent\textbf{Part (c): $(x \times y)^z = x^z \times y^z$. \quad Induct on $z$.}

\medskip
\noindent\textbf{Base case} ($z = 0$):

\noindent
\begin{tabular}{T S J}
\toprule
\textbf{Tag} & \multicolumn{1}{p{0.44\textwidth}}{\textbf{Step}} & \textbf{Justification} \\
\midrule
\addlinespace[4pt]
& & \\[50pt]
\bottomrule
\end{tabular}

\vspace{1em}

\noindent\textbf{Inductive step} (assume $(x \times y)^z = x^z \times y^z$, prove for $z\pp$):

\noindent
\begin{tabular}{T S J}
\toprule
\textbf{Tag} & \multicolumn{1}{p{0.44\textwidth}}{\textbf{Step}} & \textbf{Justification} \\
\midrule
\addlinespace[4pt]
& & \\[100pt]
\bottomrule
\end{tabular}

\vspace{1.5em}

\noindent\textbf{Conclusion.}

\vspace{2em}
