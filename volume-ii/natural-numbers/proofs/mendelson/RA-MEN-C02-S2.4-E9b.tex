% --------------------------------------
% Proof: RA-MEN-C02-S2.4-E9b
% --------------------------------------
\clearpage
\phantomsection
\hypertarget{proof-RA-MEN-C02-S2.4-E9b}{}

\subsubsection[A Set Need Not Have a Greatest Element]{Proof --- RA-MEN-C02-S2.4-E9b}

\bigskip

\noindent
\textbf{Source.}
\srccite{Mendelson2008NumberSystems}{Chapter~2, \S2.4, Exercise~9(b)}.

\vspace{0.75em}

\noindent
\textbf{Goal.}
A set need not have a greatest element.

\vspace{0.75em}

\noindent
\textbf{Logical form.}
$\exists B \subseteq P$ such that $B$ has no greatest element.

\vspace{0.75em}

\noindent
\textbf{Proof strategy.}
Counterexample. Exhibit a concrete nonempty set $B \subseteq P$ and verify
that no element of $B$ satisfies the definition of greatest element.

\vspace{0.75em}

\noindent
\textbf{Proof.}
\begin{proof}
% Provide a specific nonempty set B and show it has no greatest element.
\end{proof}

\vspace{0.75em}

\noindent
\textbf{Remarks.}
\begin{itemize}
  \item \textbf{Connection to analysis:} This is precisely why the supremum
    (least upper bound) is more useful than the maximum in analysis --- a set
    can be bounded above without having a greatest element, and the supremum
    captures the bound even when no maximum exists.
  \item \textbf{Mistakes to avoid:} Do not confuse ``bounded above'' with
    ``has a greatest element.'' These are distinct conditions and this
    exercise illustrates the gap.
\end{itemize}
