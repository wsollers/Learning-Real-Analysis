% --------------------------------------
% Proof: RA-TAO-C02-S2.3-E03
% --------------------------------------
\clearpage
\phantomsection
\hypertarget{proof-RA-TAO-C02-S2.3-E03}{}

\subsubsection[Multiplication is Associative (P2.3.5)]{Proof --- RA-TAO-C02-S2.3-E03}

\noindent
\textbf{Source.}
\srccite{TaoAnalysis1}{Chapter~2, \S2.3, Exercise~2.3.3}.

\vspace{0.75em}

\noindent
\textbf{Goal.}
Prove Proposition~2.3.5 (Multiplication is associative):

\medskip
\noindent
\textbf{Proposition~2.3.5.}
For any natural numbers $a, b, c$, we have $(a \times b) \times c = a \times (b \times c)$.

\vspace{0.75em}

\noindent
% =========================================================
% Proof: Proposition 2.3.5 — Multiplication is Associative
% =========================================================

\begin{proposition}[Multiplication is associative]
\label{prop:mult-assoc}
For any natural numbers $a, b, c$:
\[
(a \times b) \times c = a \times (b \times c).
\]
\end{proposition}

\begin{proof}
The goal is a universal equality. We prove by induction on $a$
the statement:
\[
P(a) :\equiv (a \times b) \times c = a \times (b \times c),
\]
holding $b$ and $c$ fixed as arbitrary natural numbers.

\medskip
\noindent\textbf{Base case} ($a = 0$).

\noindent
\begin{tabular}{T S J}
\toprule
\textbf{Tag} &
\multicolumn{1}{p{0.44\textwidth}}{\textbf{Step}} &
\textbf{Justification} \\
\midrule
\addlinespace[4pt]

\tagDU
&
(0 \times b) \times c = 0 \times c
&
M1: $0 \times b := 0$. Left side becomes $0 \times c$.
\\[10pt]

\tagDU
&
0 \times c = 0
&
M1: $0 \times c := 0$.
\\[10pt]

\tagDU
&
0 \times (b \times c) = 0
&
M1: $0 \times (b \times c) := 0$.
\\[10pt]

\tagAM
&
(0 \times b) \times c = 0 \times (b \times c)
&
Both sides equal $0$. \hfill$\blacksquare_{P(0)}$
\\[6pt]

\bottomrule
\end{tabular}

\medskip
\noindent\textbf{Inductive step.}
Assume $P(a)$ holds, i.e., assume:
\[
(a \times b) \times c = a \times (b \times c). \tag{IH}
\]
We show $P(a\pp)$, i.e., that
$(a\pp \times b) \times c = a\pp \times (b \times c)$.

\noindent
\begin{tabular}{T S J}
\toprule
\textbf{Tag} &
\multicolumn{1}{p{0.44\textwidth}}{\textbf{Step}} &
\textbf{Justification} \\
\midrule
\addlinespace[4pt]

\tagDU
&
a\pp \times b = (a \times b) + b
&
M2: $(a\pp) \times b := (a \times b) + b$.
\\[10pt]

\tagAM
&
(a\pp \times b) \times c
= ((a \times b) + b) \times c
&
Substituting M2 into the left side of $P(a\pp)$.
\\[10pt]

\tagTA
&
((a \times b) + b) \times c
= (a \times b) \times c + b \times c
&
Distributive law, Proposition~2.3.4:
$(\alpha + \beta) \times \gamma
= \alpha \times \gamma + \beta \times \gamma$.
Apply with $\alpha := a \times b$, $\beta := b$,
$\gamma := c$.
\\[10pt]

\tagIH
&
(a \times b) \times c + b \times c
= a \times (b \times c) + b \times c
&
Inductive hypothesis: replace
$(a \times b) \times c$ with $a \times (b \times c)$.
\\[10pt]

\tagDU
&
a\pp \times (b \times c)
= a \times (b \times c) + b \times c
&
M2: $(a\pp) \times (b \times c)
:= a \times (b \times c) + (b \times c)$.
\\[10pt]

\tagAM
&
(a\pp \times b) \times c
= a\pp \times (b \times c)
&
Both sides equal
$a \times (b \times c) + b \times c$.
\hfill$\blacksquare_{P(a\pp)}$
\\[6pt]

\bottomrule
\end{tabular}

\medskip
\noindent
By induction (P5), $P(a)$ holds for all $a \in \mathbb{N}$.
Since $b$ and $c$ were arbitrary, associativity holds for all
$a, b, c \in \mathbb{N}$.

\end{proof}

\begin{remark}[The proof engine: M2 $\to$ distributivity $\to$ IH $\to$ M2]
The inductive step follows a four-move pattern that recurs
throughout Chapter~2 multiplication proofs:
(1) unpack the left side using M2;
(2) apply distributivity to expose a subexpression matching the IH;
(3) substitute the IH;
(4) recognize the result as M2 applied to the right side.
When a multiplication proof stalls, check whether distributivity
has been applied.
\end{remark}

\begin{remark}[Why induct on $a$]
M2 recurses on the left argument of multiplication:
$(a\pp) \times m := (a \times m) + m$.
Inducting on $a$ ensures M2 fires in the inductive step,
reducing $a\pp \times b$ to $(a \times b) + b$ and exposing
the inductive hypothesis.
\end{remark}

\begin{remark}[Distributivity as the bridge]
The inductive hypothesis gives a fact about $(a \times b) \times c$.
Distributivity is the bridge that exposes this subexpression
inside the expanded left side $((a \times b) + b) \times c$.
Without it, the IH has nothing to attach to.
\end{remark}

\vspace{1em}
