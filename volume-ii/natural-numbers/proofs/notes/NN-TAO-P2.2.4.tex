% --------------------------------------
% Proof: NN-TAO-P2.2.4
% Proposition 2.2.4 — Commutativity of addition
% Format: Deep study (3-column)
% --------------------------------------
\clearpage
\phantomsection
\label{prf:add-comm}
\hypertarget{proof-NN-TAO-P2.2.4}{}

\subsubsection[Commutativity of Addition (P2.2.4)]{Proof --- NN-TAO-P2.2.4}

\begin{remark}[Return]
\hyperref[prop:add-comm]{$\leftarrow$ Back to Proposition 2.2.4 in Notes}
\end{remark}

\bigskip

\noindent
\textbf{Source.}
\srccite{TaoAnalysis1}{Chapter~2, \S2.2, Proposition~2.2.4}.

\vspace{0.75em}

\noindent
\textbf{Goal.}
For all $n, m \in \mathbb{N}$, $n + m = m + n$.

\vspace{0.75em}

\noindent
\textbf{Logical form.}
$\forall n \in \mathbb{N},\ \forall m \in \mathbb{N},\ n + m = m + n$.

\vspace{0.75em}

\noindent
\textbf{Key background.}
L2.2.2: $n + 0 = n$. \quad
L2.2.3: $n + (m\pp) = (n+m)\pp$. \quad
Axiom A1: $0 + m = m$.

\vspace{0.75em}

\noindent
\textbf{Proof strategy.}
Fix $m$; induct on $n$.
Base case uses L2.2.2 (to handle $0 + m$) and A1 (to handle $m + 0$).
Inductive step uses L2.2.3 and the IH.

\vspace{0.75em}

\noindent
\textbf{Proof.}
\begin{proof}
Fix $m \in \mathbb{N}$ arbitrary.

\begin{longtable}{p{0.28\textwidth} p{0.35\textwidth} p{0.28\textwidth}}
\toprule
\textbf{Step / Claim} & \textbf{Justification} & \textbf{Annotation} \\
\midrule
\emph{Base case.} $0 + m = m + 0$.
&
LHS: $0 + m = m$ by A1. RHS: $m + 0 = m$ by L2.2.2.
&
Both sides equal $m$. Uses \emph{both} L2.2.2 and A1.
\\
\addlinespace
\emph{IH.} Assume $n + m = m + n$.
&
Induction assumption.
&
$P(n)$ for fixed $m$.
\\
\addlinespace
\emph{Inductive step.} Show $(n\pp) + m = m + (n\pp)$.
&
&
\\
$(n\pp) + m = (n + m)\pp$
&
A2.
&
Left argument is a successor; A2 fires.
\\
$(n + m)\pp = (m + n)\pp$
&
IH: $n + m = m + n$.
&
Substitute; successor respects equality.
\\
$(m + n)\pp = m + (n\pp)$
&
L2.2.3 (read right to left): $m + (n\pp) = (m+n)\pp$.
&
This is the key move: L2.2.3 lets us move the successor from left to right.
\\
\emph{Conclusion.} $(n\pp) + m = m + (n\pp)$.
&
Chain above.
&
$P(n\pp)$ established.
\\
\addlinespace
By induction on $n$, the result holds for all $n$ and all fixed $m$.
Since $m$ was arbitrary, it holds for all $m$.
&
P5.
&
\\
\bottomrule
\end{longtable}
\end{proof}

\vspace{0.75em}

\noindent
\textbf{Remarks.}
\begin{itemize}
  \item \textbf{Scaffolding dependency.} This proof is only possible \emph{after}
    L2.2.2 and L2.2.3. Those lemmas are purpose-built for exactly this use.
    The logical dependency graph is: A1, A2 $\to$ L2.2.2, L2.2.3 $\to$ P2.2.4.

  \item \textbf{Induction variable.} We induct on $n$ (the left argument),
    not $m$. The reason is the same as always: A2 fires on the left argument,
    and A2 is what the inductive step needs.

  \item \textbf{Consequence.} With commutativity established, the distinction
    between ``left'' and ``right'' properties collapses for addition.
    Every ``right'' induction result (L2.2.2, L2.2.3) can now be transferred
    to a ``left'' result by commutativity.
\end{itemize}
