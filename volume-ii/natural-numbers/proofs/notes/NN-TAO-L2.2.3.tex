% --------------------------------------
% Proof: NN-TAO-L2.2.3
% Lemma 2.2.3 — Successor on the right: n + (m++) = (n + m)++
% Format: Deep study (3-column)
% --------------------------------------
\clearpage
\phantomsection
\label{prf:add-succ-right}
\hypertarget{proof-NN-TAO-L2.2.3}{}

\subsubsection[Successor on the Right (L2.2.3)]{Proof --- NN-TAO-L2.2.3}

\begin{remark}[Return]
\hyperref[lem:add-succ-right]{$\leftarrow$ Back to Lemma 2.2.3 in Notes}
\end{remark}

\bigskip

\noindent
\textbf{Source.}
\srccite{TaoAnalysis1}{Chapter~2, \S2.2, Lemma~2.2.3}.

\vspace{0.75em}

\noindent
\textbf{Goal.}
For all $n, m \in \mathbb{N}$, $n + (m\pp) = (n + m)\pp$.
This is the \emph{right} successor identity; the left version $(n\pp) + m = (n+m)\pp$ is Axiom A2 (definitional).

\vspace{0.75em}

\noindent
\textbf{Logical form.}
$\forall n \in \mathbb{N},\ \forall m \in \mathbb{N},\ n + (m\pp) = (n+m)\pp$.

\vspace{0.75em}

\noindent
\textbf{Key background.}
A1, A2 as in L2.2.2. Lemma 2.2.2 ($n + 0 = n$) is used in the base case.

\vspace{0.75em}

\noindent
\textbf{Proof strategy.}
Fix $m$; induct on $n$.
Again, A2 fires on the left argument, so $n$ is the induction variable.
The key structural move is that A2 and the inductive hypothesis together
propagate the successor from right to left inside the expression.

\vspace{0.75em}

\noindent
\textbf{Proof.}
\begin{proof}
Fix $m \in \mathbb{N}$ arbitrary.

\begin{longtable}{p{0.28\textwidth} p{0.35\textwidth} p{0.28\textwidth}}
\toprule
\textbf{Step / Claim} & \textbf{Justification} & \textbf{Annotation} \\
\midrule
\emph{Base case.} $0 + (m\pp) = (0 + m)\pp$.
&
A1 applied twice: $0 + (m\pp) = m\pp$ and $0 + m = m$, so $(0+m)\pp = m\pp$.
&
A1 with $m := m\pp$ on left; A1 with $m := m$ on right.
Both sides reduce to $m\pp$.
\\
\addlinespace
\emph{IH.} Assume $n + (m\pp) = (n+m)\pp$ for some $n$.
&
Induction assumption, with $m$ fixed.
&
$P(n)$ states the equation holds for this $n$ and our fixed $m$.
\\
\addlinespace
\emph{Inductive step.} Show $(n\pp) + (m\pp) = ((n\pp) + m)\pp$.
&
&
\\
$(n\pp) + (m\pp) = (n + (m\pp))\pp$
&
A2 with left argument $n\pp$, right argument $m\pp$.
&
A2 fires because left argument is a successor.
\\
$(n + (m\pp))\pp = ((n+m)\pp)\pp$
&
IH: $n + (m\pp) = (n+m)\pp$.
&
Substitute IH under the outer $\pp$.
\\
$((n+m)\pp)\pp = ((n\pp)+m)\pp$
&
A2 (reversed): $(n\pp)+m = (n+m)\pp$.
&
A2 gives $(n\pp)+m=(n+m)\pp$; rewrite the inner term.
\\
\emph{Conclusion.} $(n\pp) + (m\pp) = ((n\pp)+m)\pp$.
&
Chain of equalities above.
&
$P(n\pp)$ established.
\\
\addlinespace
By induction on $n$, the result holds for all $n \in \mathbb{N}$ and all fixed $m$.
Since $m$ was arbitrary, the result holds for all $m$.
&
P5; $m$ was universally quantified from the start.
&
\\
\bottomrule
\end{longtable}
\end{proof}

\vspace{0.75em}

\noindent
\textbf{Remarks.}
\begin{itemize}
  \item \textbf{Pair with L2.2.2.} L2.2.2 moves zero from left to right; L2.2.3
    moves a successor from left to right. Together they provide the
    two base cases that commutativity (P2.2.4) needs.

  \item \textbf{Induction variable.} Fix $m$ first, then induct on $n$.
    If you tried to induct on $m$ with $n$ fixed, the recursive structure
    of addition (A1/A2) would not fire.

  \item \textbf{Consequence.} The pair (L2.2.2, L2.2.3) is all that P2.2.4
    (commutativity) requires. The proof of commutativity by induction on $n$
    invokes exactly these two lemmas at the base and inductive step.
\end{itemize}
