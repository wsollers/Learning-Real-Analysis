% --------------------------------------
% Proof: NN-TAO-P2.2.6
% Proposition 2.2.6 — Cancellation law for N
% Format: Sketch (compact)
% --------------------------------------
\clearpage
\phantomsection
\label{prf:add-cancel}
\hypertarget{proof-NN-TAO-P2.2.6}{}

\subsubsection[Cancellation Law for $\mathbb{N}$ (P2.2.6)]{Proof --- NN-TAO-P2.2.6}

\begin{remark}[Return]
\hyperref[prop:add-cancel]{$\leftarrow$ Back to Proposition 2.2.6 in Notes}
\end{remark}

\bigskip

\begin{tcolorbox}[colback=gray!6, colframe=gray!40, arc=2pt,
  left=6pt, right=6pt, top=4pt, bottom=4pt,
  title={\small\textbf{Proof Sketch — Cancellation in $\mathbb{N}$ (Tao P2.2.6)}},
  fonttitle=\small\bfseries]

\textbf{Statement.}
Let $a, b, c \in \mathbb{N}$ with $a + b = a + c$. Then $b = c$.

\textbf{Key move.}
Induct on $a$. Base case $a = 0$: $0 + b = 0 + c$ gives $b = c$ by A1.
Inductive step: if $(a\pp) + b = (a\pp) + c$, then A2 gives
$(a + b)\pp = (a + c)\pp$, so $a + b = a + c$ by Peano Axiom P4
(injectivity of successor), and the IH gives $b = c$.

\textbf{Depends on.}
Peano Axiom P4 (successor is injective). Axioms A1, A2.
This result is load-bearing: it is used to prove transitivity of integer equality
(Lemma 4.1.3) and is the \emph{reason} the equivalence class construction of $\mathbb{Z}$ works.
\end{tcolorbox}

\vspace{0.75em}

\noindent
\textbf{Remarks.}
\begin{itemize}
  \item \textbf{Virtual subtraction.} This proposition provides a substitute for
    subtraction in $\mathbb{N}$: it lets us ``cancel'' $a$ from both sides of
    $a + b = a + c$ without invoking $-$. Subtraction will be defined only
    after $\mathbb{Z}$ is constructed.

  \item \textbf{Connection to integers.} The proof of transitivity of equality
    in the integer construction (Tao Lemma 4.1.3) reduces to exactly this
    cancellation law. It is the algebraic load-bearer for the equivalence
    class argument.
\end{itemize}
