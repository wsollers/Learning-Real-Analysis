% --------------------------------------
% Proof: NN-TAO-L2.2.2
% Lemma 2.2.2 — Zero on the right: n + 0 = n
% Format: Deep study (3-column)
% --------------------------------------
\clearpage
\phantomsection
\label{prf:add-zero-right}
\hypertarget{proof-NN-TAO-L2.2.2}{}

\subsubsection[Zero on the Right (L2.2.2)]{Proof --- NN-TAO-L2.2.2}

\begin{remark}[Return]
\hyperref[lem:add-zero-right]{$\leftarrow$ Back to Lemma 2.2.2 in Notes}
\end{remark}

\bigskip

\noindent
\textbf{Source.}
\srccite{TaoAnalysis1}{Chapter~2, \S2.2, Lemma~2.2.2}.

\vspace{0.75em}

\noindent
\textbf{Goal.}
For all $n \in \mathbb{N}$, $n + 0 = n$.
This is the \emph{right} zero identity; the left identity $0 + m = m$ is Axiom A1 (definitional).

\vspace{0.75em}

\noindent
\textbf{Logical form.}
$\forall n \in \mathbb{N},\ n + 0 = n$.

\vspace{0.75em}

\noindent
\textbf{Key background.}
A1: $0 + m = m$ (zero on the left, definitional). \quad
A2: $(n\pp) + m = (n + m)\pp$ (successor on the left, definitional). \quad
Commutativity is \emph{not} yet available.

\vspace{0.75em}

\noindent
\textbf{Proof strategy.}
Induction on $n$.
The definition of addition only fires when the \emph{left} argument is $0$ or a successor.
Since $n$ is the left argument here, induct on $n$; that is what makes A1 and A2 applicable.

\vspace{0.75em}

\noindent
\textbf{Proof.}
\begin{proof}
\begin{longtable}{p{0.28\textwidth} p{0.35\textwidth} p{0.28\textwidth}}
\toprule
\textbf{Step / Claim} & \textbf{Justification} & \textbf{Annotation} \\
\midrule
\emph{Base case.} $0 + 0 = 0$.
&
A1 with $m := 0$.
&
A1 says $0 + m = m$; substitute $m = 0$.
\\
\addlinespace
\emph{Inductive hypothesis (IH).} Assume $n + 0 = n$ for some fixed $n \in \mathbb{N}$.
&
Induction assumption.
&
This is $P(n)$; we want to establish $P(n\pp)$.
\\
\addlinespace
\emph{Inductive step.} Show $(n\pp) + 0 = n\pp$.
&
&
\\
$(n\pp) + 0 = (n + 0)\pp$
&
A2 with $m := 0$.
&
A2 fires because the left argument is a successor $n\pp$.
\\
$(n + 0)\pp = n\pp$
&
IH: $n + 0 = n$.
&
Substitute $n + 0 = n$ under the successor operator.
\\
\emph{Conclusion.} $(n\pp) + 0 = n\pp$.
&
Transitivity of equality.
&
$P(n\pp)$ established.
\\
\addlinespace
By induction, $n + 0 = n$ for all $n \in \mathbb{N}$.
&
Peano axiom P5 (induction).
&
$P(0)$ and $P(n) \Rightarrow P(n\pp)$ established; induction closes.
\\
\bottomrule
\end{longtable}
\end{proof}

\vspace{0.75em}

\noindent
\textbf{Remarks.}
\begin{itemize}
  \item \textbf{Asymmetry.} The definition of addition treats the two arguments
    differently. $0 + m = m$ holds by definition (A1), but $n + 0 = n$ requires
    a proof. This asymmetry recurs in L2.2.3 (successor on right vs.\ left).

  \item \textbf{Why induct on $n$, not $m$?} A1 and A2 fire on the left argument.
    The right argument $0$ is fixed. Inducting on the varying left argument $n$
    is the only way to apply the recursive structure of addition.

  \item \textbf{Consequence.} Together with L2.2.3, this lemma enables the proof
    of commutativity (P2.2.4). Those two lemmas are the essential scaffolding.

  \item \textbf{Common error.} Assuming commutativity to pass from A1 to this
    lemma. Commutativity is not yet available; this lemma is part of \emph{building}
    the tools needed to prove commutativity.
\end{itemize}
