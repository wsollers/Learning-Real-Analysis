% =========================================================
% Rational Numbers — Planned
% =========================================================
\chapter{Rational Numbers ($\mathbb{Q}$)}

\begin{tcolorbox}[
  colback=gray!6,
  colframe=gray!40,
  arc=2pt,
  left=8pt, right=8pt, top=6pt, bottom=6pt,
  title={\small\textbf{Where You Are in the Journey}},
  fonttitle=\small\bfseries
]
\begin{center}
\small
$\mathbb{N}$
$\;\to\;$ Integers ($\mathbb{Z}$)
$\;\to\;$ \textbf{Rationals ($\mathbb{Q}$)}
$\;\to\;$ Real Numbers ($\mathbb{R}$)
$\;\to\;$ $\cdots$
\end{center}

\medskip
\noindent\textbf{How we got here.}
The integers gave us additive inverses, making subtraction always
possible. But division remains problematic: $1 \div 2$ has no integer
solution. The rationals adjoin multiplicative inverses for all
non-zero integers, giving us a field.

\medskip
\noindent\textbf{What this chapter will build.}
The construction of $\mathbb{Q}$ from $\mathbb{Z}$ via equivalence
classes of pairs, the field axioms, the dense order, and
the incompleteness of $\mathbb{Q}$ (existence of $\sqrt{2}$ gaps).

\medskip
\noindent\textbf{Where this leads.}
The incompleteness of $\mathbb{Q}$ is the motivation for constructing
$\mathbb{R}$. The Dedekind and Cauchy constructions both start from $\mathbb{Q}$.
\end{tcolorbox}

\vspace{1em}
\begin{tcolorbox}[
  colback=gray!6, colframe=gray!40, arc=2pt,
  left=8pt, right=8pt, top=6pt, bottom=6pt,
  title={\small\textbf{Status: Planned}},
  fonttitle=\small\bfseries
]
\begin{center}\Large\bfseries Coming Soon\end{center}
\vspace{6pt}
\noindent Notes, proofs, and exercises will appear here in a future revision.
\end{tcolorbox}

\section{Notes}
\textit{To be populated.}

\section{Proofs}
\textit{To be populated.}

\section{Capstone}
\textit{To be populated.}
