% --------------------------------------
% Proof: RA-TAO-C04-S4.1-E06
% --------------------------------------
\clearpage
\phantomsection
\hypertarget{proof-RA-TAO-C04-S4.1-E06}{}

\subsubsection[Cancellation Law for Z (Ex 4.1.6)]{Proof --- RA-TAO-C04-S4.1-E06}

\bigskip

\noindent\textbf{Source.}
\srccite{TaoAnalysis1}{Chapter~4, \S4.1, Exercise~4.1.6}.

\vspace{0.75em}

\noindent\textbf{Goal.}
Prove Corollary~4.1.9: if $a, b, c \in \mathbb{Z}$, $ac = bc$, and
$c \neq 0$, then $a = b$.

\vspace{0.75em}

\noindent\textbf{Logical form.}
$\forall a,b,c \in \mathbb{Z}:\; ac = bc \wedge c \neq 0 \Rightarrow a = b$.

\vspace{0.75em}

\noindent\textbf{Note.} Tao identifies two distinct proof approaches.
Choose one and justify why it works; if you attempt both, present them
separately.

\vspace{0.75em}

\noindent\textbf{Available toolkit.}
\textbf{Approach A:} Proposition~4.1.8 (no zero divisors in $\mathbb{Z}$),
laws of algebra P4.1.6.\\
\textbf{Approach B:} Corollary~2.3.7 (cancellation in $\mathbb{N}$),
Lemma~4.1.5 (trichotomy).

\vspace{0.75em}

\noindent\textbf{Strategy.}
(Fill in before beginning --- which approach, and why.)

\vspace{2em}

\noindent\rule{\textwidth}{0.4pt}
\noindent\textbf{Proof.}

\noindent
\begin{tabular}{T S J}
\toprule
\textbf{Tag} & \multicolumn{1}{p{0.44\textwidth}}{\textbf{Step}} & \textbf{Justification} \\
\midrule
\addlinespace[4pt]
& & \\[140pt]
\bottomrule
\end{tabular}

\vspace{1.5em}

\noindent\textbf{Conclusion.}

\vspace{2em}
