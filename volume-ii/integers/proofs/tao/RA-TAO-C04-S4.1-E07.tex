% --------------------------------------
% Proof: RA-TAO-C04-S4.1-E07
% --------------------------------------
\clearpage
\phantomsection
\hypertarget{proof-RA-TAO-C04-S4.1-E07}{}

\subsubsection[Properties of Order on Z (Ex 4.1.7)]{Proof --- RA-TAO-C04-S4.1-E07}

\bigskip

\noindent\textbf{Source.}
\srccite{TaoAnalysis1}{Chapter~4, \S4.1, Exercise~4.1.7}.

\vspace{0.75em}

\noindent\textbf{Goal.}
Prove Lemma~4.1.11: for all $a, b, c \in \mathbb{Z}$:
\begin{enumerate}
  \item[(a)] $a > b \iff a - b$ is a positive natural number.
  \item[(b)] If $a > b$, then $a + c > b + c$.
  \item[(c)] If $a > b$ and $c$ is positive, then $ac > bc$.
  \item[(d)] If $a > b$, then $-a < -b$.
  \item[(e)] If $a > b$ and $b > c$, then $a > c$.
  \item[(f)] Exactly one of $a > b$, $a < b$, $a = b$ holds.
\end{enumerate}

\vspace{0.75em}

\noindent\textbf{Logical form.} Each part is a separate universally
quantified statement over $\mathbb{Z}$.

\vspace{0.75em}

\noindent\textbf{Available toolkit.}
Definition~4.1.10 (order: $n \geq m \iff n = m+a$, $a \in \mathbb{N}$),
Proposition~4.1.6 (ring laws for $\mathbb{Z}$),
Lemma~4.1.5 (trichotomy of integers),
properties of $\mathbb{N}$: P2.2.8, P2.3.6, C2.2.9.

\vspace{0.75em}

\noindent\textbf{Strategy.}
(Fill in before beginning --- note that Tao suggests proving (a) first
and deriving (b)--(f) from it.)

\vspace{2em}

\noindent\rule{\textwidth}{0.4pt}
\noindent\textbf{Part (a): $a > b \iff a - b$ is a positive natural number.}

\noindent
\begin{tabular}{T S J}
\toprule
\textbf{Tag} & \multicolumn{1}{p{0.44\textwidth}}{\textbf{Step}} & \textbf{Justification} \\
\midrule
\addlinespace[4pt]
& & \\[80pt]
\bottomrule
\end{tabular}

\vspace{1.5em}

\noindent\rule{\textwidth}{0.4pt}
\noindent\textbf{Part (b): Addition preserves order.}

\noindent
\begin{tabular}{T S J}
\toprule
\textbf{Tag} & \multicolumn{1}{p{0.44\textwidth}}{\textbf{Step}} & \textbf{Justification} \\
\midrule
\addlinespace[4pt]
& & \\[60pt]
\bottomrule
\end{tabular}

\vspace{1.5em}

\noindent\rule{\textwidth}{0.4pt}
\noindent\textbf{Part (c): Positive multiplication preserves order.}

\noindent
\begin{tabular}{T S J}
\toprule
\textbf{Tag} & \multicolumn{1}{p{0.44\textwidth}}{\textbf{Step}} & \textbf{Justification} \\
\midrule
\addlinespace[4pt]
& & \\[60pt]
\bottomrule
\end{tabular}

\vspace{1.5em}

\noindent\rule{\textwidth}{0.4pt}
\noindent\textbf{Part (d): Negation reverses order.}

\noindent
\begin{tabular}{T S J}
\toprule
\textbf{Tag} & \multicolumn{1}{p{0.44\textwidth}}{\textbf{Step}} & \textbf{Justification} \\
\midrule
\addlinespace[4pt]
& & \\[60pt]
\bottomrule
\end{tabular}

\vspace{1.5em}

\noindent\rule{\textwidth}{0.4pt}
\noindent\textbf{Part (e): Transitivity.}

\noindent
\begin{tabular}{T S J}
\toprule
\textbf{Tag} & \multicolumn{1}{p{0.44\textwidth}}{\textbf{Step}} & \textbf{Justification} \\
\midrule
\addlinespace[4pt]
& & \\[60pt]
\bottomrule
\end{tabular}

\vspace{1.5em}

\noindent\rule{\textwidth}{0.4pt}
\noindent\textbf{Part (f): Trichotomy.}

\noindent
\begin{tabular}{T S J}
\toprule
\textbf{Tag} & \multicolumn{1}{p{0.44\textwidth}}{\textbf{Step}} & \textbf{Justification} \\
\midrule
\addlinespace[4pt]
& & \\[80pt]
\bottomrule
\end{tabular}

\vspace{1.5em}

\noindent\textbf{Conclusion.}

\vspace{2em}
