% --------------------------------------
% Proof: INT-TAO-L4.1.3
% Lemma 4.1.3 — Addition and multiplication on Z are well-defined
% Format: Deep study (3-column)
% --------------------------------------
\clearpage
\phantomsection
\label{prf:int-well-defined}
\hypertarget{proof-INT-TAO-L4.1.3}{}

\subsubsection[Well-Definedness of Integer Arithmetic (L4.1.3)]{Proof --- INT-TAO-L4.1.3}

\begin{remark}[Return]
\hyperref[lem:int-well-defined]{$\leftarrow$ Back to Lemma 4.1.3 in Notes}
\end{remark}

\bigskip

\noindent
\textbf{Source.}
\srccite{TaoAnalysis1}{Chapter~4, \S4.1, Lemma~4.1.3}.

\vspace{0.75em}

\noindent
\textbf{Goal.}
The operations $+$ and $\times$ on $\mathbb{Z}$, defined by
\[
(a \!-\!\!- b) + (c \!-\!\!- d) := (a+c) \!-\!\!- (b+d),
\qquad
(a \!-\!\!- b) \times (c \!-\!\!- d) := (ac+bd) \!-\!\!- (ad+bc),
\]
are well-defined: the output class does not depend on which representatives
$(a,b)$ and $(c,d)$ are chosen from each input class.

\vspace{0.75em}

\noindent
\textbf{Logical form.}
If $a \!-\!\!- b = a' \!-\!\!- b'$ (i.e.\ $a + b' = a' + b$) and
$c \!-\!\!- d = c' \!-\!\!- d'$ (i.e.\ $c + d' = c' + d$), then
$(a+c) \!-\!\!- (b+d) = (a'+c') \!-\!\!- (b'+d')$, i.e.\ $(a+c)+(b'+d') = (a'+c')+(b+d)$.

\vspace{0.75em}

\noindent
\textbf{Key background.}
P2.2.6 (Cancellation in $\mathbb{N}$) is the essential tool for the transitivity
verification that the integer equality relation is well-defined.
For this lemma, only commutativity and associativity of addition in $\mathbb{N}$ are needed.

\vspace{0.75em}

\noindent
\textbf{Proof strategy.}
Given the two hypothesis equations, expand the target equality condition and
show it follows by rearranging terms using commutativity and associativity of
addition in $\mathbb{N}$. This is pure algebraic manipulation; no induction is needed.

\vspace{0.75em}

\noindent
\textbf{Proof (addition well-defined).}
\begin{proof}
Assume $a + b' = a' + b$ and $c + d' = c' + d$.
We must show $(a+c) + (b'+d') = (a'+c') + (b+d)$.

\begin{longtable}{p{0.30\textwidth} p{0.33\textwidth} p{0.28\textwidth}}
\toprule
\textbf{Step / Claim} & \textbf{Justification} & \textbf{Annotation} \\
\midrule
$(a + b') + (c + d') = (a' + b) + (c' + d)$
&
Add the two hypothesis equations.
&
Addition of equations in $\mathbb{N}$; valid because $\mathbb{N}$ has the ring properties we need.
\\
\addlinespace
$(a + c) + (b' + d') = (a' + c') + (b + d)$
&
Commutativity and associativity of $+$ in $\mathbb{N}$.
&
Rearrange both sides: the LHS groups $a,c$ and $b',d'$; the RHS groups $a',c'$ and $b,d$.
\\
\addlinespace
This is exactly the target condition.
&
Definition of equality $\!-\!\!-$ on $\mathbb{Z}$.
&
$(a+c)\!-\!\!-(b+d) = (a'+c')\!-\!\!-(b'+d')$ as integers. $\square$
\\
\bottomrule
\end{longtable}
\end{proof}

\vspace{0.75em}

\noindent
\textbf{Remarks.}
\begin{itemize}
  \item \textbf{What ``well-defined'' means.} The definition of $+$ refers to
    representatives $(a,b)$; but a class has many representatives. Well-definedness
    asserts the result is the same whichever representative you pick. This must
    be verified for every operation defined on equivalence classes.

  \item \textbf{The load-bearing lemma.} If this fails, the entire $\mathbb{Z}$
    construction collapses: the ``operations'' would be set-valued rather than
    functions. This is why it is the \emph{first} thing to check after defining
    an operation on a quotient.

  \item \textbf{Technique.} The proof is a direct algebraic chase using only
    the arithmetic of $\mathbb{N}$. Multiplication well-definedness is more involved
    (requires expanding and regrouping four terms) but uses the same idea.

  \item \textbf{Pattern.} This is the canonical example of equivalence-class
    manipulation: (1) assume the hypothesis equations, (2) expand the target,
    (3) substitute and rearrange. This pattern recurs in rational number and
    real number constructions.
\end{itemize}
