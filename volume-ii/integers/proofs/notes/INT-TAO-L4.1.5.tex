% --------------------------------------
% Proof: INT-TAO-L4.1.5
% Lemma 4.1.5 — Trichotomy of integers
% Format: Sketch (compact)
% --------------------------------------
\clearpage
\phantomsection
\label{prf:int-trichotomy}
\hypertarget{proof-INT-TAO-L4.1.5}{}

\subsubsection[Trichotomy of Integers (L4.1.5)]{Proof --- INT-TAO-L4.1.5}

\begin{remark}[Return]
\hyperref[lem:int-trichotomy]{$\leftarrow$ Back to Lemma 4.1.5 in Notes}
\end{remark}

\bigskip

\begin{tcolorbox}[colback=gray!6, colframe=gray!40, arc=2pt,
  left=6pt, right=6pt, top=4pt, bottom=4pt,
  title={\small\textbf{Proof Sketch — Trichotomy of $\mathbb{Z}$ (Tao L4.1.5)}},
  fonttitle=\small\bfseries]

\textbf{Statement.}
For any integer $z = a \!-\!\!- b$ (with $a, b \in \mathbb{N}$), exactly one of the following holds:
$z$ is positive (equal to $n \!-\!\!- 0$ for some positive $n \in \mathbb{N}$),
$z$ is zero ($z = 0 \!-\!\!- 0$), or $z$ is negative (equal to $0 \!-\!\!- n$ for some positive $n$).

\textbf{Key move.}
By the natural number trichotomy, exactly one of $a > b$, $a = b$, or $a < b$ holds.
Each case translates directly: $a > b$ gives a positive representative $a - b \!-\!\!- 0$
(using P2.2.10 predecessor lemma); $a = b$ gives $0 \!-\!\!- 0$; $a < b$ gives a negative
representative $0 \!-\!\!- (b - a)$.
Mutual exclusion follows from the injectivity of the natural number ordering.

\textbf{Depends on.}
Trichotomy of $\mathbb{N}$ (order on $\mathbb{N}$). Predecessor Lemma (L2.2.10).
Definition of positive/negative integers (Tao Def 4.1.4).
\end{tcolorbox}

\vspace{0.75em}

\noindent
\textbf{Remarks.}
\begin{itemize}
  \item \textbf{Structural significance.} This lemma justifies the signed-number
    picture of $\mathbb{Z}$ as a \emph{theorem} rather than a definition.
    The equivalence-class construction gives integers as pairs; trichotomy
    proves every integer is one of three types.

  \item \textbf{Connection.} The same trichotomy argument recurs for $\mathbb{Q}$
    and (via sign of Dedekind cuts or Cauchy sequences) for $\mathbb{R}$.
    Recognising this pattern is preparation for those constructions.
\end{itemize}
