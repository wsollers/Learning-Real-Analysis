% =========================================================
% The Integers — Chapter Header
% (Drop this at the top of integers/notes/index.tex,
%  before \subsubsection{Tao Construction})
% =========================================================

% ---------------------------------------------------------
% "Where You Are in the Journey" breadcrumb box
% ---------------------------------------------------------
\begin{tcolorbox}[colback=gray!6, colframe=gray!50, arc=2pt,
  left=6pt, right=6pt, top=4pt, bottom=4pt,
  title={\small\textbf{Where You Are in the Journey}},
  fonttitle=\small\bfseries]
\small
Propositional Logic $\to$ Predicate Calculus $\to$ Sets \& Functions $\to$
Proof Techniques $\to$ Axiom Systems \& Natural Numbers $\to$
\textbf{The Integers} $\to$ Rationals \& Ordered Fields
$\to$ Real Line Foundations $\to$ \(\cdots\)

\medskip
\textbf{How we got here.}
The natural numbers gave us addition, multiplication, and a well-founded
order, but subtraction is not available: for $a < b$ in $\mathbb{N}$,
the expression $a - b$ is meaningless.
Every subsequent number system depends on fixing this, so we pause here
to do it rigorously.

\medskip
\textbf{What this chapter builds.}
We follow two complementary sources.
Tao's \textit{Analysis~I}, Chapter~4 constructs $\mathbb{Z}$ from
$\mathbb{N}\times\mathbb{N}$ via formal pairs and proves all arithmetic
properties directly.
Mendelson's \textit{Number Systems and the Foundations of Analysis},
Chapters~3--4 performs the same construction but frames the result in
the language of abstract algebra: rings, integral domains, and ordered
integral domains.
Together they give both the computational fluency and the structural
vocabulary that real analysis and abstract algebra require.

\medskip
\textbf{Where this leads.}
The ring and integral-domain structure established here is not special
to $\mathbb{Z}$: the same axioms, the same order properties, and the
same absolute-value toolkit reappear verbatim for $\mathbb{Q}$ and
$\mathbb{R}$.
Everything algebraic built later inherits the patterns established here.
\end{tcolorbox}

\bigskip

% ---------------------------------------------------------
% Structural Roadmap
% ---------------------------------------------------------
\noindent\textbf{Structural Roadmap}

\medskip
\noindent
Both sources follow the same logical spine:

\begin{center}
\textbf{Equivalence Relation}
$\longrightarrow$ \textbf{Definitions}
$\longrightarrow$ \textbf{Well-Definedness}
$\longrightarrow$ \textbf{Ring Laws}
$\longrightarrow$ \textbf{Order}
$\longrightarrow$ \textbf{Absolute Value}
\end{center}

\noindent
Nothing is assumed beyond what has been proved for $\mathbb{N}$.
The available toolkit at each step is exactly what precedes it ---
most critically, the cancellation law for natural-number addition, which
carries the entire construction forward.

\medskip
\noindent The global progression is:

\begin{enumerate}
  \item \textbf{The equivalence relation.}
        Integers are equivalence classes of pairs $(a,b)\in\mathbb{N}
        \times\mathbb{N}$ under $(a,b)\sim(c,d)\iff a+d=c+b$.
        Transitivity requires cancellation in $\mathbb{N}$.

  \item \textbf{Addition and multiplication.}
        Defined on representatives; well-definedness verified separately
        for each operation (Tao L4.1.3; Mendelson Lemmas 2.1, 2.3).

  \item \textbf{Ring laws.}
        The nine identities of Tao's P4.1.6 (= Mendelson Thms 2.2,~2.4)
        are exactly the axioms of a commutative ring with unit.
        All proofs expand pairs and reduce to arithmetic in $\mathbb{N}$.

  \item \textbf{Integral domain.}
        No zero divisors (P4.1.8 / Thm~2.4(v)) and the cancellation law
        (C4.1.9 / Thm~3.3) elevate $\mathbb{Z}$ from a ring to an
        integral domain.
        Mendelson's \S3.3 names and defines these structures abstractly.

  \item \textbf{Order.}
        Tao defines order directly via $\mathbb{N}$-differences
        (Def~4.1.10).
        Mendelson axiomatises an ordered integral domain (Def~4.1,
        axioms O1--O5) and instantiates it for $\mathbb{Z}$ via a
        positivity set (Thm~4.3, Cor~4.6).
        The same five axioms recur unchanged for $\mathbb{R}$.

  \item \textbf{Absolute value.}
        Mendelson's Theorem~4.8 (11~parts, for any ordered integral
        domain) provides a permanent toolkit ---
        triangle inequality, reverse triangle inequality, multiplicativity
        --- that requires no re-proof when we reach $\mathbb{Q}$ and
        $\mathbb{R}$.
        Tao defers absolute value to the rational chapter.

  \item \textbf{Induction boundary.}
        Induction fails for $\mathbb{Z}$ (Tao Ex~4.1.8): there is no
        smallest element to anchor a descent below $0$.
        Strong induction on $\mathbb{N}$ remains available for the
        positive integers inside $\mathbb{Z}$; Mendelson Thm~4.7 recovers
        a Peano system from $\mathcal{P}_{\mathbb{Z}}$ to make this
        precise.
\end{enumerate}