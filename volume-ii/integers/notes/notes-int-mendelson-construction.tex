% =========================================================
% The Integers — Mendelson Construction
% Source: Mendelson, \textit{Number Systems and the
%   Foundations of Analysis}, Chapters 3--4 (core);
%   Chapters 5--9 (extended theory).
% =========================================================

% ---------------------------------------------------------
% §3.1  The equivalence relation on P × P
% ---------------------------------------------------------
\subsubsection*{§3.1 \quad The Equivalence Relation on $P \times P$}

\begin{remark}[Starting point]
Mendelson begins from $P \times P$, the set of all ordered pairs of
positive integers (his notation for $\mathbb{N} \setminus \{0\}$).
Tao uses $\mathbb{N} \times \mathbb{N}$ (including zero).
The idea is identical: a pair $(n,j)$ represents the
``formal difference'' $n - j$.
\end{remark}

\begin{tcolorbox}[colback=propbox, colframe=propborder, arc=2pt,
  left=6pt, right=6pt, top=4pt, bottom=4pt,
  title={\small\textbf{Definition 3.1 (Equivalence relation on $P \times P$)}},
  fonttitle=\small\bfseries]
For natural numbers $n, k, j, i$, define a relation $\sim$ on
ordered pairs by
\[
(n,j) \sim (k,i)
\quad\Longleftrightarrow\quad
n + i = k + j.
\]
\end{tcolorbox}

\begin{theorem}[Mendelson 1.1 --- $\sim$ is an equivalence relation]
\label{thm:men-equiv}
For all natural numbers $h,i,j,k,m,n$:
\begin{enumerate}
  \item[\normalfont(R)] $(h,i) \sim (h,i)$ \hfill\textit{(Reflexivity)}
  \item[\normalfont(S)] $(h,i) \sim (j,k) \Rightarrow (j,k) \sim (h,i)$
    \hfill\textit{(Symmetry)}
  \item[\normalfont(T)] $\bigl[(h,i)\sim(j,k)\wedge(j,k)\sim(m,n)\bigr]
    \Rightarrow (h,i)\sim(m,n)$ \hfill\textit{(Transitivity)}
\end{enumerate}
\end{theorem}

\begin{remark}[Transitivity uses cancellation]
As in Tao, transitivity of $\sim$ requires the cancellation law for
addition in $\mathbb{N}$.
This is the same load-bearing step in both constructions.
\end{remark}

\begin{tcolorbox}[colback=propbox, colframe=propborder, arc=2pt,
  left=6pt, right=6pt, top=4pt, bottom=4pt,
  title={\small\textbf{Definition 3.2 (The integers $\mathbb{Z}$)}},
  fonttitle=\small\bfseries]
$\mathbb{Z}$ is the set of all equivalence classes of $P \times P$
under $\sim$.  Elements of $\mathbb{Z}$ are called \textbf{integers}.

Distinguished elements:
$0_{\mathbb{Z}} = [(1,1)]$, \quad $1_{\mathbb{Z}} = [(2,1)]$.
\end{tcolorbox}

\begin{remark}[Notation comparison]
Mendelson writes $[(n,j)]$ for an equivalence class;
Tao writes $a {-\!\!-} b$ for the same object.
Mendelson's notation makes the set-theoretic content explicit.
Tao's notation is cleaner for computation.
\end{remark}

% ---------------------------------------------------------
% §3.2  Addition and Multiplication
% ---------------------------------------------------------
\subsubsection*{§3.2 \quad Addition and Multiplication}

\begin{lemma}[Mendelson 2.1 --- Addition well-defined]
\label{lem:men-add-welld}
If $(n,j) \sim (n_1,j_1)$ and $(k,i) \sim (k_1,i_1)$, then
$(n+k,\; j+i) \sim (n_1+k_1,\; j_1+i_1)$.
\end{lemma}

\begin{tcolorbox}[colback=propbox, colframe=propborder, arc=2pt,
  left=6pt, right=6pt, top=4pt, bottom=4pt,
  title={\small\textbf{Definition 3.3 (Addition)}},
  fonttitle=\small\bfseries]
For $\alpha, \beta \in \mathbb{Z}$ with representatives
$(n,j) \in \alpha$ and $(k,i) \in \beta$:
\[
\alpha +_{\mathbb{Z}} \beta = [(n+k,\; j+i)].
\]
Lemma~2.1 guarantees independence of representatives.
\end{tcolorbox}

\begin{theorem}[Mendelson 2.2 --- Properties of addition]
\label{thm:men-add}
For all $\alpha, \beta, \gamma \in \mathbb{Z}$:
\begin{enumerate}
  \item[(i)] Commutativity: $\alpha +_{\mathbb{Z}} \beta
    = \beta +_{\mathbb{Z}} \alpha$.
  \item[(ii)] Associativity: $\alpha +_{\mathbb{Z}}
    (\beta +_{\mathbb{Z}} \gamma)
    = (\alpha +_{\mathbb{Z}} \beta) +_{\mathbb{Z}} \gamma$.
  \item[(iii)] Additive identity: $\alpha +_{\mathbb{Z}} 0_{\mathbb{Z}}
    = \alpha$.
  \item[(iv)] Unique additive inverse: $\exists!\,\delta \in \mathbb{Z}$
    such that $\alpha +_{\mathbb{Z}} \delta = 0_{\mathbb{Z}}$.
\end{enumerate}
\end{theorem}

\begin{remark}[Uniqueness of the inverse]
Part (iv) asserts both existence \emph{and} uniqueness of $-\alpha$.
Tao's P4.1.6 asserts existence only; uniqueness follows later from
C4.1.9.  Mendelson builds uniqueness into the statement.
\end{remark}

\begin{lemma}[Mendelson 2.3 --- Multiplication well-defined]
\label{lem:men-mul-welld}
If $(n,j) \sim (n_1,j_1)$ and $(k,i) \sim (k_1,i_1)$, then
$(nk+ji,\; jk+ni) \sim (n_1k_1+j_1i_1,\; j_1k_1+n_1i_1)$.
\end{lemma}

\begin{tcolorbox}[colback=propbox, colframe=propborder, arc=2pt,
  left=6pt, right=6pt, top=4pt, bottom=4pt,
  title={\small\textbf{Definition 3.4 (Multiplication)}},
  fonttitle=\small\bfseries]
For $\alpha, \beta \in \mathbb{Z}$ with representatives
$(n,j) \in \alpha$ and $(k,i) \in \beta$:
\[
\alpha \times_{\mathbb{Z}} \beta = [(nk+ji,\; jk+ni)].
\]
Lemma~2.3 guarantees well-definedness.
\end{tcolorbox}

\begin{theorem}[Mendelson 2.4 --- Properties of multiplication]
\label{thm:men-mul}
For all $\alpha, \beta, \gamma \in \mathbb{Z}$:
\begin{enumerate}
  \item[(i)] Commutativity: $\alpha \times_{\mathbb{Z}} \beta
    = \beta \times_{\mathbb{Z}} \alpha$.
  \item[(ii)] Associativity: $\alpha \times_{\mathbb{Z}}
    (\beta \times_{\mathbb{Z}} \gamma)
    = (\alpha \times_{\mathbb{Z}} \beta) \times_{\mathbb{Z}} \gamma$.
  \item[(iii)] Distributivity: $\alpha \times_{\mathbb{Z}}
    (\beta +_{\mathbb{Z}} \gamma)
    = (\alpha \times_{\mathbb{Z}} \beta)
    +_{\mathbb{Z}} (\alpha \times_{\mathbb{Z}} \gamma)$.
  \item[(iv)] Multiplicative identity: $\alpha \times_{\mathbb{Z}}
    1_{\mathbb{Z}} = \alpha$.
  \item[(v)] No zero divisors: $\alpha \neq 0_{\mathbb{Z}} \wedge
    \beta \neq 0_{\mathbb{Z}} \Rightarrow
    \alpha \times_{\mathbb{Z}} \beta \neq 0_{\mathbb{Z}}$.
\end{enumerate}
\end{theorem}

\begin{remark}[Structural consequence]
Theorems 2.2 and 2.4 establish that
$(\mathbb{Z}, +_\mathbb{Z}, \times_\mathbb{Z})$ is an
\textbf{integral domain}: a commutative ring with unit and no zero divisors.
Mendelson makes this explicit in \S3.3 via abstract definitions.
\end{remark}

% ---------------------------------------------------------
% §3.3  Rings and Integral Domains (abstract)
% ---------------------------------------------------------
\subsubsection*{§3.3 \quad Rings and Integral Domains (Abstract Theory)}

\begin{remark}[Abstract vs concrete]
Mendelson develops ring theory abstractly before applying it to $\mathbb{Z}$.
Tao works concretely throughout, naming the ring structure only in
Remark 4.1.7.
\end{remark}

\begin{tcolorbox}[colback=propbox, colframe=propborder, arc=2pt,
  left=6pt, right=6pt, top=4pt, bottom=4pt,
  title={\small\textbf{Definition 3.5 (Ring; integral domain)}},
  fonttitle=\small\bfseries]
A \textbf{ring} $(R,+,\times)$: $(R,+)$ is an abelian group;
$\times$ is associative; $\times$ distributes over $+$ on both sides.
A ring is \textbf{commutative} if $xy=yx$;
has a \textbf{unit} if $\exists\,1$ with $x\cdot 1=1\cdot x=x$.
A nonzero $x$ is a \textbf{zero divisor} if $\exists$ nonzero $y$
with $xy=0$.
An \textbf{integral domain} is a commutative ring with unit,
$0 \neq 1$, and no zero divisors.
\end{tcolorbox}

\begin{theorem}[Mendelson 3.3 --- Cancellation $\Leftrightarrow$ no zero divisors]
\label{thm:men-cancel-equiv}
In a commutative ring with unit:
\[
(\forall x,y,z:\; xy=xz \wedge x\neq 0 \Rightarrow y=z)
\;\Longleftrightarrow\;
\text{no zero divisors.}
\]
\end{theorem}

\begin{theorem}[Mendelson 3.4 --- Trivial ring]
\label{thm:men-trivial}
In a ring with unit: $0=1 \Longleftrightarrow R$ is a singleton.
\end{theorem}

% ---------------------------------------------------------
% §4  Order
% ---------------------------------------------------------
\subsubsection*{§4 \quad Ordered Integral Domains and Order on $\mathbb{Z}$}

\begin{remark}[Strategy]
Mendelson proves order theory for any integral domain first, then
instantiates it for $\mathbb{Z}$.
This gives order on $\mathbb{Q}$ and $\mathbb{R}$ for free later.
\end{remark}

\begin{tcolorbox}[colback=propbox, colframe=propborder, arc=2pt,
  left=6pt, right=6pt, top=4pt, bottom=4pt,
  title={\small\textbf{Definition 4.1 (Ordered integral domain)}},
  fonttitle=\small\bfseries]
$(R,+,\times,<)$ is an \textbf{ordered integral domain} if
$(R,+,\times)$ is an integral domain ($0\neq 1$) and $<$ satisfies:
\begin{enumerate}
  \item[(O1)] $x \not< x$ \hfill\textit{Irreflexivity}
  \item[(O2)] $x<y \wedge y<z \Rightarrow x<z$ \hfill\textit{Transitivity}
  \item[(O3)] $x<y \vee x=y \vee y<x$ \hfill\textit{Trichotomy}
  \item[(O4)] $x<y \Rightarrow x+z<y+z$ \hfill\textit{Addition-monotone}
  \item[(O5)] $x<y \wedge 0<z \Rightarrow xz<yz$ \hfill\textit{Positive-mult-monotone}
\end{enumerate}
$x$ is \textbf{positive} if $0<x$; \textbf{negative} if $x<0$.
\end{tcolorbox}

\begin{theorem}[Mendelson 4.1 --- Consequences of order axioms]
In any ordered integral domain:
(i)~exactly one of $x<y$, $x=y$, $y<x$ holds;
(ii)~$x<y \wedge u<v \Rightarrow x+u<y+v$;
(iii)~$0<z \wedge xz<yz \Rightarrow x<y$.
\end{theorem}

\begin{theorem}[Mendelson 4.2 --- Order and positivity]
In any ordered integral domain:
(i)~$x<y \iff y-x$ positive;
(ii)~$x<y \iff x-y$ negative;
(iii)~$x<y \iff -y<-x$;
(iv)~sum and product of positives are positive;
(v)~product of two negatives is positive.
\end{theorem}

\begin{theorem}[Mendelson 4.3 --- Positivity set construction]
\label{thm:men-pos-set}
Let $(R,+,\times)$ be an integral domain, $0\neq 1$.
If $\mathcal{P}\subseteq R$ satisfies:
$0\notin\mathcal{P}$;
$\forall x:\; x\in\mathcal{P} \vee x=0 \vee {-x}\in\mathcal{P}$;
$\mathcal{P}$ closed under $+$ and $\times$;
then $(R,+,\times,<)$ with $x<y \iff y-x\in\mathcal{P}$ is an
ordered integral domain and $\mathcal{P}$ is its positive set.
\end{theorem}

\begin{remark}[Why this matters]
Theorem 4.3 reduces verifying five order axioms to four positivity
conditions.  Mendelson uses it to order $\mathbb{Z}$ via
$\mathcal{P}_\mathbb{Z}$ without checking (O1)--(O5) directly.
\end{remark}

\begin{lemma}[Mendelson 4.4 --- Positivity is class-invariant]
For $\alpha\in\mathbb{Z}$ and any two representatives
$(n,j),(k,i)\in\alpha$: $j<n \iff i<k$.
\end{lemma}

\begin{tcolorbox}[colback=propbox, colframe=propborder, arc=2pt,
  left=6pt, right=6pt, top=4pt, bottom=4pt,
  title={\small\textbf{Definition 4.5 (Positivity set $\mathcal{P}_\mathbb{Z}$)}},
  fonttitle=\small\bfseries]
\[
\mathcal{P}_{\mathbb{Z}}
= \bigl\{\alpha\in\mathbb{Z}
: \forall(n,j)\in\alpha,\; j<n \bigr\}.
\]
Lemma 4.4 ensures this is well-defined.
Intuitively: $\alpha$ is positive iff its representative has
second coordinate strictly less than first, i.e.\ $n-j > 0$.
\end{tcolorbox}

\begin{lemma}[Mendelson 4.5]
$\mathcal{P}_\mathbb{Z}$ satisfies the four conditions of Theorem~4.3.
\end{lemma}

\begin{corollary}[Mendelson 4.6]
$(\mathbb{Z},+_\mathbb{Z},\times_\mathbb{Z},<_\mathbb{Z})$ is an
ordered integral domain with positive set $\mathcal{P}_\mathbb{Z}$.
\end{corollary}

\begin{theorem}[Mendelson 4.7 --- Recovery of Peano system]
Let $T(x)=x+_\mathbb{Z}1_\mathbb{Z}$ for $x\in\mathcal{P}_\mathbb{Z}$.
Then $(\mathcal{P}_\mathbb{Z},T,1_\mathbb{Z})$ is a Peano system.
\end{theorem}

\begin{remark}[Significance]
Starting from a Peano system for $\mathbb{N}$, we construct $\mathbb{Z}$,
and then prove the positive integers \emph{inside} $\mathbb{Z}$ form a
new Peano system consistent with the original.
Tao does not prove this explicitly.
\end{remark}

% ---------------------------------------------------------
% §4  Absolute Value
% ---------------------------------------------------------
\subsubsection*{§4 \quad Absolute Value}

\begin{tcolorbox}[colback=propbox, colframe=propborder, arc=2pt,
  left=6pt, right=6pt, top=4pt, bottom=4pt,
  title={\small\textbf{Definition 4.6 (Absolute value)}},
  fonttitle=\small\bfseries]
In any ordered integral domain:
$|x| = x$ if $0\leq x$; $|x|=-x$ if $x<0$.
\end{tcolorbox}

\begin{theorem}[Mendelson 4.8 --- Properties of absolute value]
In any ordered integral domain, $|{\cdot}|$ satisfies:
(1)~$|x|\geq 0$;
(2)~$|x|=0\iff x=0$;
(3)~$|-x|=|x|$;
(4)~$|x-y|=|y-x|$;
(5)~$|xy|=|x||y|$;
(6)~$-|x|\leq x\leq|x|$;
(7)~$|z|<u\iff -u<z<u$;
(8)~$|z|\leq u\iff -u\leq z\leq u$;
(9)~$u\geq v \wedge u\geq -v \Rightarrow u\geq|v|$;
(10)~$|x+y|\leq|x|+|y|$ \textit{(triangle inequality)};
(11)~$|x-y|\geq\bigl||x|-|y|\bigr|$ \textit{(reverse triangle inequality)}.
\end{theorem}

% ---------------------------------------------------------
% §5  Number theory
% ---------------------------------------------------------
\subsubsection*{§5 \quad Division, Divisibility, Primes}

\begin{remark}[Scope]
This material is not in Tao Ch~4 and is not a prerequisite for real
analysis.  It is valuable background for number theory and abstract algebra.
See the comparison table for coverage decisions.
\end{remark}

\begin{theorem}[Mendelson 5.1 --- Euclidean division]
For any integer $\alpha>1$ and any integer $\beta$, there exist unique
integers $q,r$ with $\beta=q\alpha+r$ and $0\leq r<\alpha$.
\end{theorem}

\begin{definition}[Divisibility]
$\alpha\mid\beta \iff \exists\,\gamma\in\mathbb{Z}:\;\beta=\alpha\gamma$.
\end{definition}

\begin{theorem}[Mendelson 5.2--5.9]
Standard divisibility properties, existence and Bézout form of gcd,
characterisation of relative primeness, infinitude of primes
(Euclid), Euclid's lemma ($\rho\mid\alpha\beta\Rightarrow\rho\mid\alpha$
or $\rho\mid\beta$), and the fundamental theorem of arithmetic
(unique prime factorisation up to order and sign).
\end{theorem}

% ---------------------------------------------------------
% §6  Modular arithmetic
% ---------------------------------------------------------
\subsubsection*{§6 \quad Integers Modulo $n$ (Optional)}

\begin{remark}[Scope]
Congruence mod $n$ as an equivalence relation, $\mathbb{Z}_n$ as a
commutative ring with unit; integral domain iff $n$ is prime.
Not required for the real analysis track.
\end{remark}

% ---------------------------------------------------------
% §7--9  Structural theory
% ---------------------------------------------------------
\subsubsection*{§7--9 \quad Integer Action, Embedding Theory,
  Uniqueness of $\mathbb{Z}$}

\begin{remark}[Scope]
Chapters 7--9 are graduate-level algebra.
Key results:
\textbf{Thm~8.8}: every characteristic-$0$ integral domain contains
a copy of $\mathbb{Z}$.
\textbf{Thm~9.5}: $\mathcal{D}\cong\mathbb{Z}$ iff characteristic~$0$
and no proper subdomains.
\textbf{Thm~9.8}: any two well-ordered integral domains are isomorphic
--- $\mathbb{Z}$ is unique up to isomorphism.
These results are beyond what is needed for real analysis but are
worth knowing as statements.
\end{remark}

\begin{definition}[Well-ordered integral domain]
An ordered integral domain is \textbf{well-ordered} if every nonempty
subset of its positive elements contains a least element.
\end{definition}

\begin{theorem}[Mendelson 9.8 --- Uniqueness of $\mathbb{Z}$]
Any two well-ordered integral domains are isomorphic.
In particular, $\mathbb{Z}$ is, up to isomorphism, the unique
well-ordered integral domain.
\end{theorem}
