% =========================================================
% The Integers — Tao Construction (Analysis I, §4.1)
% =========================================================

% ---------------------------------------------------------
% Motivation
% ---------------------------------------------------------
\begin{remark}[Why the integers are needed]
The natural number system $\mathbb{N}$ with addition and multiplication
has reached its limits: subtraction is not available.
Given $a, b \in \mathbb{N}$ with $a < b$, the expression $a - b$ has
no value in $\mathbb{N}$. To fix this, we pass to a larger system $\mathbb{Z}$
in which every natural number has an additive inverse.
\end{remark}

\begin{remark}[The circularity problem]
The naive approach — define an integer as a difference $a - b$ of two
natural numbers — is circular, because the symbol $-$ is exactly what we
are trying to construct.
Tao resolves this by introducing a \textbf{placeholder symbol} $--$
with no arithmetic meaning, using $a {-\!\!-} b$ purely as notation for a
formal pair $(a, b)$.
Once subtraction is defined at the end of the construction, we verify that
$a {-\!\!-} b$ coincides with $a - b$ and discard the scaffolding.
\end{remark}

\begin{remark}[Why not define integers as signed naturals?]
One might define an integer as either a positive natural number, zero,
or the negation of a positive natural number.
Lemma~4.1.5 (trichotomy) shows this classification is correct, but
\emph{using it as the definition} forces case-splits for every arithmetic
operation: negative $\times$ negative, positive $+$ negative of different
sizes, etc. The verification of the ring laws becomes enormously messy.
The equivalence-class construction pays one upfront cost and avoids
all subsequent case-explosion.
\end{remark}

% ---------------------------------------------------------
% Definition 4.1.1 — Integers
% ---------------------------------------------------------
\begin{tcolorbox}[colback=propbox, colframe=propborder, arc=2pt,
  left=6pt, right=6pt, top=4pt, bottom=4pt,
  title={\small\textbf{Definition 4.1.1 (Integers)}},
  fonttitle=\small\bfseries]
An \textbf{integer} is an expression of the form $a {-\!\!-} b$, where
$a$ and $b$ are natural numbers.

Two integers are \textbf{equal}:
\[
a {-\!\!-} b \;=\; c {-\!\!-} d
\quad\iff\quad
a + d \;=\; c + b.
\]
The set of all integers is denoted $\mathbb{Z}$.
\end{tcolorbox}

\begin{remark}[Reading the equality condition]
The condition $a + d = c + b$ is the cross-addition criterion that would
hold for genuine differences: $a - b = c - d \iff a + d = c + b$.
It uses only addition in $\mathbb{N}$, which is already available,
and avoids any reference to subtraction.
\end{remark}

\begin{remark}[Equality must be verified as a legitimate equivalence relation]
Four axioms must be checked: reflexivity, symmetry, transitivity,
and substitution.
Reflexivity and symmetry are immediate from the definition.
Transitivity requires the cancellation law for natural numbers
(Proposition~2.2.6): the key load-bearing result from Chapter~2.
Substitution cannot be verified until operations on $\mathbb{Z}$ are
defined, and must be re-checked for each operation.
\end{remark}

% ---------------------------------------------------------
% Definition 4.1.2 — Addition and Multiplication
% ---------------------------------------------------------
\begin{tcolorbox}[colback=propbox, colframe=propborder, arc=2pt,
  left=6pt, right=6pt, top=4pt, bottom=4pt,
  title={\small\textbf{Definition 4.1.2 (Addition and Multiplication)}},
  fonttitle=\small\bfseries]
\[
(a {-\!\!-} b) + (c {-\!\!-} d) \;:=\; (a+c) {-\!\!-} (b+d)
\]
\[
(a {-\!\!-} b) \times (c {-\!\!-} d) \;:=\; (ac+bd) {-\!\!-} (ad+bc)
\]
\end{tcolorbox}

\begin{remark}[Motivation for the multiplication formula]
Expanding the product of genuine differences:
$(a-b)(c-d) = ac - ad - bc + bd = (ac+bd) - (ad+bc)$.
The formula is forced by this foreknowledge.
\end{remark}

% ---------------------------------------------------------
% Lemma 4.1.3 — Well-definedness
% ---------------------------------------------------------
\begin{lemma}[Tao 4.1.3 --- Addition and multiplication are well-defined]
\label{lem:int-well-defined}
Let $a, b, a', b', c, d \in \mathbb{N}$.
If $(a {-\!\!-} b) = (a' {-\!\!-} b')$, then:
\begin{align*}
(a {-\!\!-} b) + (c {-\!\!-} d) &= (a' {-\!\!-} b') + (c {-\!\!-} d), \\
(a {-\!\!-} b) \times (c {-\!\!-} d) &= (a' {-\!\!-} b') \times (c {-\!\!-} d),
\end{align*}
and similarly when replacing $(c {-\!\!-} d)$ by an equal integer.
\end{lemma}

\begin{remark}[Proof]
See \hyperref[prf:int-well-defined]{Proof $\to$ INT-TAO-L4.1.3 (Well-Definedness of Integer Arithmetic)}.
\end{remark}

\begin{remark}[Why well-definedness must be checked]
The operations are defined in terms of representatives $(a, b)$ of an
equivalence class.
If equal inputs (i.e., equal integers with different representatives)
gave different outputs, the operation would not be a function on
$\mathbb{Z}$ — it would be a function on representations, which is
meaningless.
This check is the substitution axiom for each operation.
\end{remark}

% ---------------------------------------------------------
% Embedding N in Z
% ---------------------------------------------------------
\begin{remark}[Embedding $\mathbb{N}$ in $\mathbb{Z}$]
The integers of the form $n {-\!\!-} 0$ behave identically to the
natural numbers:
\[
(n {-\!\!-} 0) + (m {-\!\!-} 0) = (n+m) {-\!\!-} 0,
\qquad
(n {-\!\!-} 0) \times (m {-\!\!-} 0) = nm {-\!\!-} 0.
\]
Furthermore $(n {-\!\!-} 0) = (m {-\!\!-} 0)$ if and only if $n = m$.
We therefore identify $n \equiv n {-\!\!-} 0$, embedding $\mathbb{N}$
into $\mathbb{Z}$.
In particular $0 = 0 {-\!\!-} 0$ and $1 = 1 {-\!\!-} 0$.
\end{remark}

% ---------------------------------------------------------
% Definition 4.1.4 — Negation
% ---------------------------------------------------------
\begin{tcolorbox}[colback=propbox, colframe=propborder, arc=2pt,
  left=6pt, right=6pt, top=4pt, bottom=4pt,
  title={\small\textbf{Definition 4.1.4 (Negation)}},
  fonttitle=\small\bfseries]
$-(a {-\!\!-} b) \;:=\; b {-\!\!-} a$.

In particular, for a positive natural number $n = n {-\!\!-} 0$:
\[
-n \;:=\; 0 {-\!\!-} n.
\]
\end{tcolorbox}

\begin{remark}[Well-definedness of negation]
Negation must also be checked as well-defined: if
$(a {-\!\!-} b) = (a' {-\!\!-} b')$ then $-(a {-\!\!-} b) = -(a' {-\!\!-} b')$.
This is Exercise~4.1.2.
\end{remark}

% ---------------------------------------------------------
% Lemma 4.1.5 --- Trichotomy of integers
% ---------------------------------------------------------
\begin{lemma}[Tao 4.1.5 --- Trichotomy of integers]
\label{lem:int-trichotomy}
Let $x$ be an integer. Then exactly one of the following holds:
\begin{enumerate}
  \item[(a)] $x = 0$.
  \item[(b)] $x = n$ for some positive natural number $n$.
  \item[(c)] $x = -n$ for some positive natural number $n$.
\end{enumerate}
If (b) holds we call $x$ a \textbf{positive integer};
if (c) holds we call $x$ a \textbf{negative integer}.
\end{lemma}

\begin{remark}[Proof]
See \hyperref[prf:int-trichotomy]{Proof $\to$ INT-TAO-L4.1.5 (Trichotomy of Integers)}.
\end{remark}

\begin{remark}[Proof sketch]
Existence of one case: $x = a {-\!\!-} b$.
By trichotomy of $\mathbb{N}$ (P2.2.13), either $a > b$, $a = b$,
or $a < b$.
Each case yields (b), (a), or (c) respectively.
Mutual exclusion uses P2.2.8 and Proposition~2.2.6.
\end{remark}

% ---------------------------------------------------------
% Proposition 4.1.6 — Laws of algebra
% ---------------------------------------------------------
\begin{tcolorbox}[colback=propbox, colframe=propborder, arc=2pt,
  left=6pt, right=6pt, top=4pt, bottom=4pt,
  title={\small\textbf{Proposition 4.1.6 (Laws of Algebra for $\mathbb{Z}$)}},
  fonttitle=\small\bfseries]
Let $x, y, z \in \mathbb{Z}$. Then:
\begin{align*}
&x + y = y + x &\quad& \text{(commutativity of addition)}\\
&(x+y)+z = x+(y+z) &\quad& \text{(associativity of addition)}\\
&x + 0 = 0 + x = x &\quad& \text{(additive identity)}\\
&x + (-x) = (-x) + x = 0 &\quad& \text{(additive inverse)}\\
&xy = yx &\quad& \text{(commutativity of multiplication)}\\
&(xy)z = x(yz) &\quad& \text{(associativity of multiplication)}\\
&x \cdot 1 = 1 \cdot x = x &\quad& \text{(multiplicative identity)}\\
&x(y+z) = xy + xz &\quad& \text{(left distributive law)}\\
&(y+z)x = yx + zx &\quad& \text{(right distributive law)}
\end{align*}
\end{tcolorbox}

\begin{remark}[This is the commutative ring axioms]
These nine identities are exactly the definition of a
\textbf{commutative ring}.
Without the identity $xy = yx$ the remaining eight would give a ring.
Note: these properties were proved for $\mathbb{N}$, but
$\mathbb{Z} \supsetneq \mathbb{N}$, so the proofs must be redone.
\end{remark}

\begin{remark}[Proof strategy]
Write $x = (a {-\!\!-} b)$, $y = (c {-\!\!-} d)$, $z = (e {-\!\!-} f)$
and expand both sides in terms of natural number arithmetic.
This is far cleaner than case-splitting on sign via Lemma~4.1.5.
Tao demonstrates the method on associativity of multiplication.
\end{remark}

% ---------------------------------------------------------
% Subtraction
% ---------------------------------------------------------
\begin{definition}[Subtraction]
\label{def:int-subtraction}
For integers $x, y$:
\[
x - y \;:=\; x + (-y).
\]
\end{definition}

\begin{remark}[Recovering the $--$ notation]
Once subtraction is defined, one checks that for natural numbers
$a, b$:
\[
a - b = (a {-\!\!-} 0) + (0 {-\!\!-} b) = a {-\!\!-} b.
\]
The placeholder $--$ is now equal to genuine subtraction; the
scaffolding is removed and $--$ is discarded.
\end{remark}

% ---------------------------------------------------------
% Proposition 4.1.8 — No zero divisors
% ---------------------------------------------------------
\begin{proposition}[Tao 4.1.8 — No zero divisors]
\label{prop:int-no-zero-div}
Let $a, b \in \mathbb{Z}$ with $ab = 0$. Then $a = 0$ or $b = 0$ (or both).
\end{proposition}

% ---------------------------------------------------------
% Corollary 4.1.9 — Cancellation law
% ---------------------------------------------------------
\begin{corollary}[Tao 4.1.9 — Cancellation law for $\mathbb{Z}$]
\label{cor:int-cancel}
Let $a, b, c \in \mathbb{Z}$ with $ac = bc$ and $c \neq 0$. Then $a = b$.
\end{corollary}

% ---------------------------------------------------------
% Definition 4.1.10 — Order
% ---------------------------------------------------------
\begin{tcolorbox}[colback=propbox, colframe=propborder, arc=2pt,
  left=6pt, right=6pt, top=4pt, bottom=4pt,
  title={\small\textbf{Definition 4.1.10 (Ordering of $\mathbb{Z}$)}},
  fonttitle=\small\bfseries]
Let $n, m \in \mathbb{Z}$.
\[
n \geq m \quad\iff\quad n = m + a \text{ for some } a \in \mathbb{N}.
\]
\[
n > m \quad\iff\quad n \geq m \text{ and } n \neq m.
\]
\end{tcolorbox}

\begin{remark}[Consistency with $\mathbb{N}$]
This definition is verbatim the same as Definition~2.2.11 for $\mathbb{N}$.
Since the embedding $n \equiv n {-\!\!-} 0$ is consistent with addition,
the two orderings agree on natural numbers.
\end{remark}

% ---------------------------------------------------------
% Lemma 4.1.11 — Properties of order
% ---------------------------------------------------------
\begin{lemma}[Tao 4.1.11 — Properties of order on $\mathbb{Z}$]
\label{lem:int-order}
Let $a, b, c \in \mathbb{Z}$. Then:
\begin{enumerate}
  \item[(a)] $a > b$ if and only if $a - b$ is a positive natural number.
  \item[(b)] \textbf{(Addition preserves order.)} If $a > b$, then $a + c > b + c$.
  \item[(c)] \textbf{(Positive multiplication preserves order.)}
    If $a > b$ and $c$ is a positive integer, then $ac > bc$.
  \item[(d)] \textbf{(Negation reverses order.)} If $a > b$, then $-a < -b$.
  \item[(e)] \textbf{(Transitivity.)} If $a > b$ and $b > c$, then $a > c$.
  \item[(f)] \textbf{(Trichotomy.)} Exactly one of $a > b$,\ $a < b$,\ $a = b$ holds.
\end{enumerate}
\end{lemma}

\begin{remark}[Strategy: derive from part (a)]
Part (a) reformulates $>$ in terms of positivity.
Parts (b)--(f) all follow from (a) by translating into statements about
positive natural numbers, where the analogous results are already known.
\end{remark}

% ---------------------------------------------------------
% Exercise 4.1.8 — Induction fails for Z
% ---------------------------------------------------------
\begin{remark}[Exercise 4.1.8 — Induction does not apply to $\mathbb{Z}$]
Axiom P5 (induction) does not carry over to the integers.
Specifically: there exists a property $P(n)$ of integers such that
$P(0)$ is true and $P(n) \Rightarrow P(n\pp)$ for all $n \in \mathbb{Z}$,
yet $P(n)$ fails for some $n \in \mathbb{Z}$.
This is because $\mathbb{Z}$ has no smallest element — there is nothing
to anchor a descent below $0$.
The situation becomes worse for $\mathbb{Q}$ and $\mathbb{R}$.
\end{remark}

% ---------------------------------------------------------
% Toolkit reference table
% ---------------------------------------------------------
\begin{tcolorbox}[colback=gray!6, colframe=gray!40, arc=2pt,
  left=6pt, right=6pt, top=4pt, bottom=4pt,
  title={\small\textbf{Integer Toolkit — Tao §4.1 Quick Reference}},
  fonttitle=\small\bfseries]
\small
\begin{tabular}{l l l}
\toprule
\textbf{Label} & \textbf{Statement} & \textbf{Proof method} \\
\midrule
Def~4.1.1   & $a{-\!\!-}b = c{-\!\!-}d \iff a+d=c+b$ & Definition \\
Def~4.1.2   & Addition and multiplication formulas & Definition \\
L4.1.3      & Operations are well-defined & Expand; use $a+b'=a'+b$ \\
Def~4.1.4   & $-(a{-\!\!-}b) := b{-\!\!-}a$ & Definition \\
L4.1.5      & Trichotomy of $\mathbb{Z}$ & Cases $a>b$, $a=b$, $a<b$ \\
P4.1.6      & Nine ring laws & Expand via representatives \\
Def~4.1.7   & $x - y := x + (-y)$ & Definition \\
P4.1.8      & No zero divisors & Uses L2.3.3 \\
C4.1.9      & Cancellation: $ac=bc, c\neq 0 \Rightarrow a=b$ & Uses P4.1.8 or C2.3.7+L4.1.5 \\
Def~4.1.10  & Order: $n \geq m \iff n = m+a$, $a \in \mathbb{N}$ & Definition \\
L4.1.11     & Six order properties & Use part (a) as bridge \\
\bottomrule
\end{tabular}
\end{tcolorbox}
