% =========================================================
% Tao vs. Mendelson — Integer Theory Comparison Table
% For use in planning coverage decisions.
% =========================================================

\clearpage
\section*{Tao vs.\ Mendelson: Integer Theory — Statement-by-Statement Comparison}

\begin{remark*}[How to read this table]
Each row is one mathematical concept.
\textbf{Coverage} recommendations: \textbf{Core} = do both;
\textbf{Tao only} = Tao suffices, Mendelson adds abstraction only;
\textbf{Men. only} = Mendelson covers this, Tao does not;
\textbf{Skip} = beyond real analysis scope for now.
\end{remark*}

\bigskip

% ------------------------------------------------------------------
% Block 1: The construction
% ------------------------------------------------------------------
\noindent\textbf{Block 1: The Construction of $\mathbb{Z}$}

\smallskip
\begin{tabular}{p{0.22\textwidth} p{0.21\textwidth} p{0.21\textwidth} p{0.10\textwidth} p{0.14\textwidth}}
\toprule
\textbf{Concept} & \textbf{Tao} & \textbf{Mendelson} & \textbf{Same idea?} & \textbf{Coverage} \\
\midrule

Formal pairs as integers
& Def 4.1.1: $a{-\!\!-}b$, $a{-\!\!-}b=c{-\!\!-}d\iff a+d=c+b$
& Def 3.1--3.2: $(n,j)\sim(k,i)\iff n+i=k+j$; $\mathbb{Z}=$ equiv.\ classes
& Yes — identical relation, different notation
& \textbf{Core}
\\[6pt]

Equality is an equivalence relation
& Ex 4.1.1 (reflexivity, symmetry); transitivity proved in text
& Thm 1.1 (all three parts)
& Yes
& \textbf{Core}
\\[6pt]

Distinguished elements $0,1$
& $0=0{-\!\!-}0$, $1=1{-\!\!-}0$ (implicit in Def 4.1.1)
& $0_\mathbb{Z}=[(1,1)]$, $1_\mathbb{Z}=[(2,1)]$ (explicit definition)
& Yes — same objects
& \textbf{Core}
\\[6pt]

Embedding $\mathbb{N}\hookrightarrow\mathbb{Z}$
& Remark after Def 4.1.2: $n\equiv n{-\!\!-}0$
& Thm 4.7 + Cor 9.7 (recovered as Peano system)
& Tao informal; Mendelson proves formally
& \textbf{Core}
\\

\bottomrule
\end{tabular}

\bigskip

% ------------------------------------------------------------------
% Block 2: Addition
% ------------------------------------------------------------------
\noindent\textbf{Block 2: Addition}

\smallskip
\begin{tabular}{p{0.22\textwidth} p{0.21\textwidth} p{0.21\textwidth} p{0.10\textwidth} p{0.14\textwidth}}
\toprule
\textbf{Concept} & \textbf{Tao} & \textbf{Mendelson} & \textbf{Same?} & \textbf{Coverage} \\
\midrule

Addition formula
& Def 4.1.2: $(a{-\!\!-}b)+(c{-\!\!-}d):=(a+c){-\!\!-}(b+d)$
& Def 3.3 (after Lem 2.1): $\alpha+_\mathbb{Z}\beta=[(n+k,j+i)]$
& Yes
& \textbf{Core}
\\[6pt]

Well-definedness of addition
& L4.1.3 (combines add.\ and mult.)
& Lem 2.1 (addition separately)
& Yes — Mendelson separates the two
& \textbf{Core}
\\[6pt]

Commutativity
& P4.1.6 (i)
& Thm 2.2 (i)
& Yes
& \textbf{Core}
\\[6pt]

Associativity
& P4.1.6 (ii)
& Thm 2.2 (ii)
& Yes
& \textbf{Core}
\\[6pt]

Additive identity
& P4.1.6 (iii): $x+0=0+x=x$
& Thm 2.2 (iii): $\alpha+0_\mathbb{Z}=\alpha$
& Yes
& \textbf{Core}
\\[6pt]

Additive inverse (negation)
& Def 4.1.4: $-(a{-\!\!-}b):=b{-\!\!-}a$; P4.1.6 (iv) existence
& Thm 2.2 (iv): unique $\delta$ with $\alpha+\delta=0$
& Tao: existence; Men.: existence + uniqueness in one statement
& \textbf{Core}
\\[6pt]

Negation well-defined
& Ex 4.1.2
& (Implicit in Lem 2.1 / Thm 2.2)
& Yes — Tao makes it an explicit exercise
& \textbf{Core}
\\

\bottomrule
\end{tabular}

\bigskip

% ------------------------------------------------------------------
% Block 3: Multiplication
% ------------------------------------------------------------------
\noindent\textbf{Block 3: Multiplication}

\smallskip
\begin{tabular}{p{0.22\textwidth} p{0.21\textwidth} p{0.21\textwidth} p{0.10\textwidth} p{0.14\textwidth}}
\toprule
\textbf{Concept} & \textbf{Tao} & \textbf{Mendelson} & \textbf{Same?} & \textbf{Coverage} \\
\midrule

Multiplication formula
& Def 4.1.2: $(a{-\!\!-}b)(c{-\!\!-}d):=(ac+bd){-\!\!-}(ad+bc)$
& Def 3.4 (after Lem 2.3): $\alpha\times_\mathbb{Z}\beta=[(nk+ji,jk+ni)]$
& Yes
& \textbf{Core}
\\[6pt]

Well-definedness of multiplication
& L4.1.3
& Lem 2.3
& Yes
& \textbf{Core}
\\[6pt]

Commutativity
& P4.1.6 (v)
& Thm 2.4 (i)
& Yes
& \textbf{Core}
\\[6pt]

Associativity
& P4.1.6 (vi); proved in text as model calculation
& Thm 2.4 (ii)
& Yes
& \textbf{Core}
\\[6pt]

Multiplicative identity
& P4.1.6 (vii)
& Thm 2.4 (iv)
& Yes
& \textbf{Core}
\\[6pt]

Distributive law
& P4.1.6 (viii, ix)
& Thm 2.4 (iii)
& Yes
& \textbf{Core}
\\[6pt]

$(-1)\times a = -a$
& Ex 4.1.3
& Not stated separately (follows from ring axioms)
& Tao only as exercise
& \textbf{Core (Tao)}
\\[6pt]

No zero divisors
& P4.1.8
& Thm 2.4 (v)
& Yes
& \textbf{Core}
\\[6pt]

Cancellation law
& C4.1.9
& Thm 3.3 (as equivalence, in any comm.\ ring with unit)
& Mendelson more general
& \textbf{Core}
\\[6pt]

$\mathbb{Z}$ is a commutative ring / integral domain
& Rem 4.1.7 (named but not defined abstractly)
& Def 3.5 + Thms 2.2, 2.4 (built up formally)
& Mendelson makes it explicit
& \textbf{Core}
\\

\bottomrule
\end{tabular}

\bigskip

% ------------------------------------------------------------------
% Block 4: Order
% ------------------------------------------------------------------
\noindent\textbf{Block 4: Order}

\smallskip
\begin{tabular}{p{0.22\textwidth} p{0.21\textwidth} p{0.21\textwidth} p{0.10\textwidth} p{0.14\textwidth}}
\toprule
\textbf{Concept} & \textbf{Tao} & \textbf{Mendelson} & \textbf{Same?} & \textbf{Coverage} \\
\midrule

Order definition
& Def 4.1.10: $n\geq m\iff n=m+a$, $a\in\mathbb{N}$
& Def 4.1 (abstract OID axioms) + $<_\mathbb{Z}$ via $\mathcal{P}_\mathbb{Z}$ (Cor 4.6)
& Same result; Mendelson builds via positivity set
& \textbf{Core}
\\[6pt]

Order irreflexivity
& (Implicit in Def 4.1.10 and L4.1.11(f))
& Explicitly O1 in Def 4.1
& Men.\ more explicit
& \textbf{Core}
\\[6pt]

Trichotomy of integers
& L4.1.5: every integer is positive, zero, or negative (one only)
& Thm 4.1(i) + Def 4.1 O3
& Yes
& \textbf{Core}
\\[6pt]

$a>b \iff a-b$ positive
& L4.1.11(a)
& Thm 4.2(i)
& Yes
& \textbf{Core}
\\[6pt]

Addition preserves order
& L4.1.11(b)
& Def 4.1 O4 (axiom); Thm 4.1(ii) (addition of inequalities)
& Yes
& \textbf{Core}
\\[6pt]

Positive mult.\ preserves order
& L4.1.11(c)
& Def 4.1 O5 (axiom)
& Yes
& \textbf{Core}
\\[6pt]

Negation reverses order
& L4.1.11(d)
& Thm 4.2(iii)
& Yes
& \textbf{Core}
\\[6pt]

Transitivity
& L4.1.11(e)
& Def 4.1 O2 (axiom)
& Yes
& \textbf{Core}
\\[6pt]

Order trichotomy (strict)
& L4.1.11(f)
& Thm 4.1(i)
& Yes
& \textbf{Core}
\\[6pt]

Positivity set construction (Thm 4.3)
& ---
& Thm 4.3: $\mathcal{P}$ satisfying 4 conditions $\Rightarrow$ OID
& Men.\ only
& \textbf{Men.\ only} (worth knowing)
\\[6pt]

Product of negatives is positive
& (Derivable from P4.1.6 + L4.1.11)
& Thm 4.2(v) (stated explicitly)
& Tao implicit; Men.\ explicit
& \textbf{Core}
\\[6pt]

Recovery of Peano system in $\mathbb{Z}$
& ---
& Thm 4.7: $(\mathcal{P}_\mathbb{Z},T,1_\mathbb{Z})$ is a Peano system
& Men.\ only
& \textbf{Men.\ only} (important conceptually)
\\

\bottomrule
\end{tabular}

\bigskip

% ------------------------------------------------------------------
% Block 5: Absolute value
% ------------------------------------------------------------------
\noindent\textbf{Block 5: Absolute Value}

\smallskip
\begin{tabular}{p{0.22\textwidth} p{0.21\textwidth} p{0.21\textwidth} p{0.10\textwidth} p{0.14\textwidth}}
\toprule
\textbf{Concept} & \textbf{Tao} & \textbf{Mendelson} & \textbf{Same?} & \textbf{Coverage} \\
\midrule

Definition of $|x|$
& (Defined for rationals in Def 4.3.1, not integers separately)
& Def 4.6 (for any ordered integral domain)
& Men.\ gives it for $\mathbb{Z}$ directly; Tao defers to $\mathbb{Q}$
& \textbf{Men.\ only} (do it here)
\\[6pt]

Triangle inequality $|x+y|\leq|x|+|y|$
& P4.3.3(b) (for rationals)
& Thm 4.8(10)
& Yes — Mendelson proves it for $\mathbb{Z}$
& \textbf{Core}
\\[6pt]

Multiplicativity $|xy|=|x||y|$
& P4.3.3(d) (for rationals)
& Thm 4.8(5)
& Yes
& \textbf{Core}
\\[6pt]

Reverse triangle inequality
& (Implicit)
& Thm 4.8(11) explicitly
& Men.\ explicit
& \textbf{Core}
\\[6pt]

Full 11-part absolute value theorem
& Spread across P4.3.3 (for $\mathbb{Q}$)
& Thm 4.8 (complete, for any OID)
& Men.\ more systematic
& \textbf{Men.\ only} (do once, use forever)
\\

\bottomrule
\end{tabular}

\bigskip

% ------------------------------------------------------------------
% Block 6: Induction failure / structural notes
% ------------------------------------------------------------------
\noindent\textbf{Block 6: Induction and Structural Observations}

\smallskip
\begin{tabular}{p{0.22\textwidth} p{0.21\textwidth} p{0.21\textwidth} p{0.10\textwidth} p{0.14\textwidth}}
\toprule
\textbf{Concept} & \textbf{Tao} & \textbf{Mendelson} & \textbf{Same?} & \textbf{Coverage} \\
\midrule

Induction fails for $\mathbb{Z}$
& Ex 4.1.8: explicit counterexample
& Implied by lack of least element; Thm 9.6 addresses positives only
& Tao makes it an explicit exercise
& \textbf{Core (Tao)}
\\[6pt]

Subtraction definition
& $x-y:=x+(-y)$ (after Def 4.1.4)
& Standard (follows from additive inverse)
& Yes
& \textbf{Core}
\\[6pt]

$0\neq 1$ in $\mathbb{Z}$
& Implicit (P2.2.8 + L4.1.5)
& Thm 3.4 (explicit: $0=1\iff$ trivial ring)
& Men.\ explicit
& \textbf{Core}
\\

\bottomrule
\end{tabular}

\bigskip

% ------------------------------------------------------------------
% Block 7: Beyond Tao (Mendelson only)
% ------------------------------------------------------------------
\noindent\textbf{Block 7: Material in Mendelson Only --- Coverage Decision Required}

\smallskip
\begin{tabular}{p{0.22\textwidth} p{0.29\textwidth} p{0.10\textwidth} p{0.28\textwidth}}
\toprule
\textbf{Concept} & \textbf{Mendelson ref.} & \textbf{Prereq for RA?} & \textbf{Recommendation} \\
\midrule

Abstract ring and integral domain definitions
& §3.3, Def 3.5
& No (but clarifying)
& Include notes; no proof sheets needed
\\[6pt]

Cancellation $\Leftrightarrow$ no zero divisors (abstract)
& Thm 3.3
& No
& Notes only; proof optional
\\[6pt]

Trivial ring theorem ($0=1\Rightarrow$ singleton)
& Thm 3.4
& No
& Skip proof; note the statement
\\[6pt]

Abstract ordered integral domain axioms
& Def 4.1
& Indirectly (same axioms used for $\mathbb{R}$)
& \textbf{Include} — these axioms recur throughout real analysis
\\[6pt]

Positivity set construction (Thm 4.3)
& Thm 4.3
& No (used internally)
& Notes only — elegant but not exercised directly
\\[6pt]

Peano recovery (Thm 4.7)
& Thm 4.7
& No
& Notes only; important for logical completeness
\\[6pt]

Euclidean division (§5, Thm 5.1)
& Thm 5.1
& No
& \textbf{Proof sheet} — classical and useful
\\[6pt]

Divisibility properties (§5, Thm 5.2)
& Thm 5.2 (11 parts)
& No
& \textbf{Proof sheet} — good algebraic practice
\\[6pt]

GCD existence and Bézout (Thm 5.3)
& Thm 5.3
& No
& \textbf{Proof sheet} — important
\\[6pt]

Infinitude of primes (Thm 5.6)
& Thm 5.6
& No
& \textbf{Proof sheet} — Euclid's classic argument
\\[6pt]

Fundamental theorem of arithmetic (Thm 5.9)
& Thm 5.9
& No
& \textbf{Proof sheet} — core number theory
\\[6pt]

Modular arithmetic (§6)
& Thms 6.1--6.6
& No
& \textbf{Skip} for now; revisit in algebra track
\\[6pt]

Integer action on domains (§7, Thms 7.1--7.4)
& Thms 7.1--7.4
& No
& \textbf{Skip} — graduate algebra
\\[6pt]

$N_\mathcal{D}$, $Z_\mathcal{D}$ embedding theory (§8)
& Thms 8.1--8.9
& No
& \textbf{Skip} — graduate algebra
\\[6pt]

Subdomain theory (§9, Thms 9.1--9.4)
& Thms 9.1--9.4
& No
& \textbf{Skip} — graduate algebra
\\[6pt]

Characterisation of $\mathbb{Z}$ (Thm 9.5)
& Thm 9.5
& No
& Notes only — worth knowing as a statement
\\[6pt]

Well-ordered integral domain (Def, Cor 9.7)
& Def + Cor 9.7
& No
& Notes only
\\[6pt]

Uniqueness of $\mathbb{Z}$ up to isomorphism (Thm 9.8)
& Thm 9.8
& No
& \textbf{Note the statement} — philosophically important
\\

\bottomrule
\end{tabular}

\bigskip

% ------------------------------------------------------------------
% Summary
% ------------------------------------------------------------------
\begin{tcolorbox}[colback=gray!6, colframe=gray!40, arc=2pt,
  left=6pt, right=6pt, top=4pt, bottom=4pt,
  title={\small\textbf{Summary: Recommended Coverage}},
  fonttitle=\small\bfseries]
\small
\textbf{Core (both sources):}
Equality, addition, multiplication, well-definedness, ring laws,
no zero divisors, cancellation, order, trichotomy, six order properties.

\medskip
\textbf{Mendelson adds (do notes, consider proof sheets):}
Abstract ring/integral domain definitions (\S3.3);
abstract OID axioms (Def~4.1) --- these recur in real analysis;
absolute value on $\mathbb{Z}$ (Thm~4.8) --- Tao defers to $\mathbb{Q}$;
positivity set construction (Thm~4.3) --- elegant technique;
Peano recovery (Thm~4.7) --- logical closure;
number theory: Euclidean division, divisibility, GCD, primes (§5).

\medskip
\textbf{Skip for now (graduate algebra):}
Modular arithmetic (§6), integer action on domains (§7),
$N_\mathcal{D}$/$Z_\mathcal{D}$ embedding theory (§8--9).
Know the statements of Thms~9.5 and~9.8 (uniqueness of $\mathbb{Z}$).
\end{tcolorbox}
