% =========================================================
% Complex Numbers — Planned
% =========================================================
\chapter{Complex Numbers ($\mathbb{C}$)}

\begin{tcolorbox}[
  colback=gray!6,
  colframe=gray!40,
  arc=2pt,
  left=8pt, right=8pt, top=6pt, bottom=6pt,
  title={\small\textbf{Where You Are in the Journey}},
  fonttitle=\small\bfseries
]
\begin{center}
\small
$\mathbb{N}$, $\mathbb{Z}$, $\mathbb{Q}$
$\;\to\;$ Real Numbers ($\mathbb{R}$)
$\;\to\;$ \textbf{Complex Numbers ($\mathbb{C}$)}
$\;\to\;$ Complex Analysis
$\;\to\;$ $\cdots$
\end{center}

\medskip
\noindent\textbf{How we got here.}
The reals are algebraically incomplete: $x^2 = -1$ has no real solution.
The complex numbers adjoin $i = \sqrt{-1}$, yielding an algebraically
closed field --- every polynomial splits completely.

\medskip
\noindent\textbf{What this chapter will build.}
Construction of $\mathbb{C}$ as $\mathbb{R}^2$ with complex multiplication,
the field axioms, modulus and argument, polar form, De Moivre's theorem,
and the fundamental theorem of algebra.

\medskip
\noindent\textbf{Where this leads.}
Complex analysis studies differentiable functions on $\mathbb{C}$.
The algebraic closure of $\mathbb{C}$ is the starting point for
algebraic geometry over $\mathbb{C}$.
\end{tcolorbox}

\vspace{1em}
\begin{tcolorbox}[
  colback=gray!6, colframe=gray!40, arc=2pt,
  left=8pt, right=8pt, top=6pt, bottom=6pt,
  title={\small\textbf{Status: Planned}},
  fonttitle=\small\bfseries
]
\begin{center}\Large\bfseries Coming Soon\end{center}
\vspace{6pt}
\noindent Notes, proofs, and exercises will appear here in a future revision.
\end{tcolorbox}

\section{Notes}
\textit{To be populated.}

\section{Proofs}
\textit{To be populated.}

\section{Capstone}
\textit{To be populated.}
