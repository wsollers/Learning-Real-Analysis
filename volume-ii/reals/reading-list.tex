% reading-list.tex

% =========================================================
% REAL ANALYSIS — VOLUME I
% Real-Line Foundations & Series (Core)
% =========================================================

\noindent
This syllabus is designed for deep mastery of real analysis
on the real number line, culminating in convergence tests for
infinite series.

The focus is structural understanding: order, completeness,
limits, compactness on $\mathbb{R}$, and convergence behavior
of sequences and series.

Each topic is studied through multiple authors to expose
distinct motivations, proof styles, and problem sets.

The plan is aligned to \textit{Rudin, Principles of Mathematical Analysis}
(3rd ed.), but aims for understanding beyond procedural fluency.

Each topic is allocated approximately three days of focused study.

% ---------------------------------------------------------
\section*{Topic 1: Order Structure, Bounds, Suprema, and Infima}
\addcontentsline{toc}{section}{Topic 1: Order Structure, Bounds, Suprema, and Infima}

\textbf{Core ideas:}
ordered fields, upper/lower bounds, least upper bound property.

\begin{itemize}
  \item \textbf{Abbott} — Ch.~1 §§1.3--1.4
  \item \textbf{Ross} — Ch.~3
  \item \textbf{Bruckner--Thomson} — Ch.~1 §§1--3
  \item \textbf{Kolmogorov--Fomin} — Ch.~1 §§2--3
  \item \textbf{Lebl} — Ch.~1
\end{itemize}

\textbf{Goal:}
Master the structural role of bounds and least upper bounds.

% ---------------------------------------------------------
\section*{Topic 2: Completeness of $\mathbb{R}$}
\addcontentsline{toc}{section}{Topic 2: Completeness of $\mathbb{R}$}

\textbf{Core ideas:}
supremum property, nested interval principle, Archimedean property.

\begin{itemize}
  \item \textbf{Abbott} — Ch.~1 §1.5
  \item \textbf{Ross} — Ch.~4
  \item \textbf{Tao} — Ch.~1 §§1.6--1.7
  \item \textbf{Lebl} — Ch.~1
\end{itemize}

\textbf{Goal:}
Understand completeness as the defining analytic property of $\mathbb{R}$.

% ---------------------------------------------------------
\section*{Topic 3: Sequences and Convergence}
\addcontentsline{toc}{section}{Topic 3: Sequences and Convergence}

\textbf{Core ideas:}
$\varepsilon$--$N$ convergence, uniqueness of limits, algebra of limits.

\begin{itemize}
  \item \textbf{Abbott} — Ch.~2 §§2.1--2.3
  \item \textbf{Ross} — Ch.~5
  \item \textbf{Tao} — Ch.~2 §§2.1--2.2
  \item \textbf{Lebl} — Ch.~2
\end{itemize}

\textbf{Goal:}
Control quantifiers rigorously and fluently.

% ---------------------------------------------------------
\section*{Topic 4: Cauchy Sequences}
\addcontentsline{toc}{section}{Topic 4: Cauchy Sequences}

\textbf{Core ideas:}
Cauchy criterion, equivalence with convergence in $\mathbb{R}$.

\begin{itemize}
  \item \textbf{Abbott} — Ch.~2 §§2.4--2.6
  \item \textbf{Ross} — Chs.~6--7
  \item \textbf{Tao} — Ch.~2 §§2.3--2.4
  \item \textbf{Lebl} — Ch.~2
\end{itemize}

\textbf{Goal:}
Internalize completeness through sequence behavior.

% ---------------------------------------------------------
\section*{Topic 5: Subsequences and Bolzano--Weierstrass}
\addcontentsline{toc}{section}{Topic 5: Subsequences and Bolzano--Weierstrass}

\textbf{Core ideas:}
subsequence extraction, bounded sequences, compact intervals.

\begin{itemize}
  \item \textbf{Abbott} — Ch.~2 §2.7
  \item \textbf{Ross} — Ch.~8
  \item \textbf{Tao} — Ch.~3
  \item \textbf{Lebl} — Ch.~2
\end{itemize}

\textbf{Goal:}
Recognize compactness on $\mathbb{R}$ via sequential arguments.

% ---------------------------------------------------------
\section*{Topic 6: Limit Superior and Limit Inferior}
\addcontentsline{toc}{section}{Topic 6: Limit Superior and Limit Inferior}

\textbf{Core ideas:}
tail suprema/infima, oscillation structure.

\begin{itemize}
  \item \textbf{Ross} — Ch.~10
  \item \textbf{Tao} — Ch.~3
  \item \textbf{Lebl} — Ch.~3
\end{itemize}

\textbf{Goal:}
Understand eventual behavior of sequences rigorously.

% ---------------------------------------------------------
\section*{Topic 7: Infinite Series}
\addcontentsline{toc}{section}{Topic 7: Infinite Series}

\textbf{Core ideas:}
series as limits of partial sums.

\begin{itemize}
  \item \textbf{Abbott} — Ch.~7 §7.1
  \item \textbf{Ross} — Ch.~18
  \item \textbf{Tao} — Ch.~7 §7.1
  \item \textbf{Lebl} — Ch.~7
\end{itemize}

\textbf{Goal:}
Reduce series convergence to sequence convergence.

% ---------------------------------------------------------
\section*{Topic 8: Tests for Series}
\addcontentsline{toc}{section}{Topic 8: Tests for Series}

\textbf{Core ideas:}
comparison tests, ratio/root tests, alternating series test.

\begin{itemize}
  \item \textbf{Abbott} — Ch.~7
  \item \textbf{Ross} — Ch.~19
  \item \textbf{Tao} — Ch.~7
  \item \textbf{Lebl} — Ch.~7
\end{itemize}

\textbf{Goal:}
Apply convergence tests with structural understanding.

\bigskip
\noindent
\textbf{Outcome:}
Completion of this syllabus provides rigorous mastery of
real-line foundations, sequences, and series convergence.
The reader will possess strong control over $\varepsilon$--$N$ reasoning,
Cauchy arguments, compactness on $\mathbb{R}$, and structural convergence tests —
forming a solid foundation for further abstraction.
