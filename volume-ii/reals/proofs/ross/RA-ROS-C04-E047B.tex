% --------------------------------------
% Proof: RA-ROS-C04-E047B
% --------------------------------------
\clearpage
\phantomsection
\hypertarget{proof-RA-ROS-C04-E047B}{}

\section*{Proof — RA-ROS-C04-E047B}

\noindent
\hyperlink{ws-RA-ROS-C04-E047B}{\textbf{← Back to worksheet}}

\noindent
\textbf{Source.}
\srccite{RossElementaryAnalysis}{Chapter~4, Exercise~4.7(b)}.

\vspace{0.75em}

\noindent
\textbf{Goal.}
Let $S$ and $T$ be nonempty bounded subsets of $\mathbb{R}$.
Prove:
\[
\sup(S\cup T)=\max\{\sup S,\sup T\}.
\]
(\emph{Do not assume $S\subseteq T$.})

\vspace{0.75em}

\noindent
\textbf{Proof.}

Let $S$ and $T$ be nonempty bounded subsets of $\mathbb{R}$.

Since the union of two bounded sets is itself bounded, $S \cup T$ is bounded.

Let $s_0 := \sup(S\cup T)$. By the definition of the supremum, $s_0 \ge s$ for all $s \in S$. Similarly $s_0 \ge t$ for all $t \in T$. Therefore $s_0 \ge \sup S$ and $s_0 \ge \sup T$.

max is defined as 
\[
\max\{\sup S, \sup T\} :=
\begin{cases}
\sup S, & \text{if } \sup S \ge \sup T, \\
\sup T, & \text{if } \sup T > \sup S.
\end{cases}
\]

Let $M := \max\{\sup S,\sup T\}$. By definition of the maximum, $M \ge \sup S$ and $M \ge \sup T$. \\
Let $x \in S \cup T$. If $x \in S$, then $x \le \sup S \le M$.
If $x \in T$, then $x \le \sup T \le M$.
Hence $M$ is an upper bound of $S \cup T$.


Therefore,
\[
\sup(S\cup T)=\max\{\sup S,\sup T\}.
\]
\vspace{6cm}

\noindent
\hyperlink{ws-RA-ROS-C04-E047B}{\textbf{← Back to worksheet}}
