% --------------------------------------
% Proof: RA-ROS-C04-E046B
% --------------------------------------
\clearpage
\phantomsection
\hypertarget{proof-RA-ROS-C04-E046B}{}

\section*{Proof — RA-ROS-C04-E046B}

\noindent
\hyperlink{ws-RA-ROS-C04-E046B}{\textbf{← Back to worksheet}}

\noindent
\textbf{Source.}
\srccite{RossElementaryAnalysis}{Chapter~4, Exercise~4.6(b)}.

\vspace{0.75em}

\noindent
\textbf{Goal.}
Let $S$ be a nonempty bounded subset of $\mathbb{R}$.
What can you say about $S$ if $\inf S=\sup S$?

\vspace{0.75em}

\noindent
\textbf{Proof.}

Let $S$ be a nonempty bounded subset of $\mathbb{R}$.

Define
\[
\begin{aligned}
s_0 := \sup S. \\
i_0 := \inf S.
\end{aligned}
\]

Let $s$ be arbitrary element $\in S$.
By the definition of supremum, we have $s \leq s_0$.
By the definition of infimum, we have $s \geq i_0$. \\
Therefore $i_0 <= s <= s_0$. \\
Assume $\inf S=\sup S$. \\
Therefore, $\inf S = s = \sup S$.  \\

Hence, $S$ contains only one element.

\qed















\vspace{6cm}

\noindent
\hyperlink{ws-RA-ROS-C04-E046B}{\textbf{← Back to worksheet}}
