% =========================================================
% Worksheet: Pons — Sequences and Limits (Chapter 2)
% File: real-line-foundations/worksheets/pons.tex
% =========================================================

\subsection{Pons}

\noindent\textbf{Source.} Matthew A. Pons, \textit{(see bibliography entry for full title)}.

\vspace{0.75em}
\begin{center}
\begin{tabular}{|p{5.0cm}|p{9.0cm}|}
\hline
\textbf{Problem ID} & \textbf{Exercise (descriptor)} \\
\hline

% -------------------------
% Chapter 2, §2.1 — Basic limits + uniqueness + divergence to infinity
% -------------------------

\phantomsection
\hypertarget{ws-RA-PON-C02-S2-1-E01}{}
\hyperlink{proof-RA-PON-C02-S2-1-E01}{\texttt{RA-PON-C02-S2-1-E01}}
&
Pons, Chapter~2, Ex.~2.1.1 — Show that a convergent sequence has a unique limit.
\\ \hline

\phantomsection
\hypertarget{ws-RA-PON-C02-S2-1-E02A}{}
\hyperlink{proof-RA-PON-C02-S2-1-E02A}{\texttt{RA-PON-C02-S2-1-E02A}}
&
Pons, Chapter~2, Ex.~2.1.2(a) — Verify $\lim_{n\to\infty} a = a$ using the definition.
\\ \hline

\phantomsection
\hypertarget{ws-RA-PON-C02-S2-1-E02B}{}
\hyperlink{proof-RA-PON-C02-S2-1-E02B}{\texttt{RA-PON-C02-S2-1-E02B}}
&
Pons, Chapter~2, Ex.~2.1.2(b) — Verify $\lim_{n\to\infty}\dfrac{1}{6n^2+1}=0$ using the definition.
\\ \hline

\phantomsection
\hypertarget{ws-RA-PON-C02-S2-1-E02C}{}
\hyperlink{proof-RA-PON-C02-S2-1-E02C}{\texttt{RA-PON-C02-S2-1-E02C}}
&
Pons, Chapter~2, Ex.~2.1.2(c) — Verify $\lim_{n\to\infty}\dfrac{3n+1}{2n+5}=\dfrac{3}{2}$ using the definition.
\\ \hline

\phantomsection
\hypertarget{ws-RA-PON-C02-S2-1-E02D}{}
\hyperlink{proof-RA-PON-C02-S2-1-E02D}{\texttt{RA-PON-C02-S2-1-E02D}}
&
Pons, Chapter~2, Ex.~2.1.2(d) — Verify $\lim_{n\to\infty}\dfrac{2}{\sqrt{n+3}}=0$ using the definition.
\\ \hline

\phantomsection
\hypertarget{ws-RA-PON-C02-S2-1-E02E}{}
\hyperlink{proof-RA-PON-C02-S2-1-E02E}{\texttt{RA-PON-C02-S2-1-E02E}}
&
Pons, Chapter~2, Ex.~2.1.2(e) — Verify $\lim_{n\to\infty}\dfrac{n^2+6}{n^2}=1$ using the definition.
\\ \hline

\phantomsection
\hypertarget{ws-RA-PON-C02-S2-1-E02F}{}
\hyperlink{proof-RA-PON-C02-S2-1-E02F}{\texttt{RA-PON-C02-S2-1-E02F}}
&
Pons, Chapter~2, Ex.~2.1.2(f) — Verify $\lim_{n\to\infty}\dfrac{2n+3}{3n+1}=\dfrac{2}{3}$ using the definition.
\\ \hline

\phantomsection
\hypertarget{ws-RA-PON-C02-S2-1-E03A}{}
\hyperlink{proof-RA-PON-C02-S2-1-E03A}{\texttt{RA-PON-C02-S2-1-E03A}}
&
Pons, Chapter~2, Ex.~2.1.3(a) — Let $(a_n)$ be positive and $a_n\to 0$. Show that $\lim_{n\to\infty}\dfrac{1}{a_n}=\infty$.
\\ \hline

\phantomsection
\hypertarget{ws-RA-PON-C02-S2-1-E03B}{}
\hyperlink{proof-RA-PON-C02-S2-1-E03B}{\texttt{RA-PON-C02-S2-1-E03B}}
&
Pons, Chapter~2, Ex.~2.1.3(b) — Let $(a_n)$ be positive and $a_n\to\infty$. Show that $\lim_{n\to\infty}\dfrac{1}{a_n}=0$.
\\ \hline


% -------------------------
% Chapter 2, §2.2 — Boundedness, subsequences, more limit verifications, abs value
% -------------------------

\phantomsection
\hypertarget{ws-RA-PON-C02-S2-2-E02A}{}
\hyperlink{proof-RA-PON-C02-S2-2-E02A}{\texttt{RA-PON-C02-S2-2-E02A}}
&
Pons, Chapter~2, Ex.~2.2.2(a) — Show a sequence is bounded iff it is bounded above and bounded below.
\\ \hline

\phantomsection
\hypertarget{ws-RA-PON-C02-S2-2-E02B}{}
\hyperlink{proof-RA-PON-C02-S2-2-E02B}{\texttt{RA-PON-C02-S2-2-E02B}}
&
Pons, Chapter~2, Ex.~2.2.2(b) — Show a sequence not bounded above has a subsequence that diverges to $\infty$.
\\ \hline

\phantomsection
\hypertarget{ws-RA-PON-C02-S2-2-E02C}{}
\hyperlink{proof-RA-PON-C02-S2-2-E02C}{\texttt{RA-PON-C02-S2-2-E02C}}
&
Pons, Chapter~2, Ex.~2.2.2(c) — Show a sequence not bounded below has a subsequence that diverges to $-\infty$.
\\ \hline

\phantomsection
\hypertarget{ws-RA-PON-C02-S2-2-E03A}{}
\hyperlink{proof-RA-PON-C02-S2-2-E03A}{\texttt{RA-PON-C02-S2-2-E03A}}
&
Pons, Chapter~2, Ex.~2.2.3(a) — Verify $\lim_{n\to\infty}\dfrac{4n+7}{2n-9}=2$.
\\ \hline

\phantomsection
\hypertarget{ws-RA-PON-C02-S2-2-E03B}{}
\hyperlink{proof-RA-PON-C02-S2-2-E03B}{\texttt{RA-PON-C02-S2-2-E03B}}
&
Pons, Chapter~2, Ex.~2.2.3(b) — Verify $\lim_{n\to\infty}\sqrt{\dfrac{1}{6n^2+1}}=0$.
\\ \hline

\phantomsection
\hypertarget{ws-RA-PON-C02-S2-2-E03C}{}
\hyperlink{proof-RA-PON-C02-S2-2-E03C}{\texttt{RA-PON-C02-S2-2-E03C}}
&
Pons, Chapter~2, Ex.~2.2.3(c) — Verify $\lim_{n\to\infty}\dfrac{\sqrt{9n^2+n}}{n+1}=3$.
\\ \hline

\phantomsection
\hypertarget{ws-RA-PON-C02-S2-2-E03D}{}
\hyperlink{proof-RA-PON-C02-S2-2-E03D}{\texttt{RA-PON-C02-S2-2-E03D}}
&
Pons, Chapter~2, Ex.~2.2.3(d) — Verify $\lim_{n\to\infty}\dfrac{n^3+6n-1}{n^4+n+2}=0$.
\\ \hline

\phantomsection
\hypertarget{ws-RA-PON-C02-S2-2-E03E}{}
\hyperlink{proof-RA-PON-C02-S2-2-E03E}{\texttt{RA-PON-C02-S2-2-E03E}}
&
Pons, Chapter~2, Ex.~2.2.3(e) — Verify $\lim_{n\to\infty}\dfrac{n^2-3n+6}{n^2}=1$.
\\ \hline

\phantomsection
\hypertarget{ws-RA-PON-C02-S2-2-E03F}{}
\hyperlink{proof-RA-PON-C02-S2-2-E03F}{\texttt{RA-PON-C02-S2-2-E03F}}
&
Pons, Chapter~2, Ex.~2.2.3(f) — Verify $\lim_{n\to\infty}\bigl(\sqrt{9n^2+n}-3n\bigr)=\dfrac{1}{6}$.
\\ \hline

\phantomsection
\hypertarget{ws-RA-PON-C02-S2-2-E05}{}
\hyperlink{proof-RA-PON-C02-S2-2-E05}{\texttt{RA-PON-C02-S2-2-E05}}
&
Pons, Chapter~2, Ex.~2.2.5 — If $(a_n)\to a$, show $(a_n^k)\to a^k$ for every $k\in\mathbb{N}$.
\\ \hline

\phantomsection
\hypertarget{ws-RA-PON-C02-S2-2-E06}{}
\hyperlink{proof-RA-PON-C02-S2-2-E06}{\texttt{RA-PON-C02-S2-2-E06}}
&
Pons, Chapter~2, Ex.~2.2.6 — If $(a_n)\to a$ and $a_n\ge 0$ for all $n$, show $\sqrt{a_n}\to \sqrt{a}$ (cases $a=0$ and $a>0$).
\\ \hline

\phantomsection
\hypertarget{ws-RA-PON-C02-S2-2-E07A}{}
\hyperlink{proof-RA-PON-C02-S2-2-E07A}{\texttt{RA-PON-C02-S2-2-E07A}}
&
Pons, Chapter~2, Ex.~2.2.7(a) — If $(a_n)\to a$, show $(|a_n|)\to |a|$.
\\ \hline

\phantomsection
\hypertarget{ws-RA-PON-C02-S2-2-E07B}{}
\hyperlink{proof-RA-PON-C02-S2-2-E07B}{\texttt{RA-PON-C02-S2-2-E07B}}
&
Pons, Chapter~2, Ex.~2.2.7(b) — Give an example of $(a_n)$ with $(|a_n|)\to 1$ but $(a_n)$ does not converge to $1$ or $-1$.
\\ \hline

\phantomsection
\hypertarget{ws-RA-PON-C02-S2-2-E07C}{}
\hyperlink{proof-RA-PON-C02-S2-2-E07C}{\texttt{RA-PON-C02-S2-2-E07C}}
&
Pons, Chapter~2, Ex.~2.2.7(c) — If $(|a_n|)\to 0$, show $(a_n)\to 0$.
\\ \hline

\phantomsection
\hypertarget{ws-RA-PON-C02-S2-2-E08A}{}
\hyperlink{proof-RA-PON-C02-S2-2-E08A}{\texttt{RA-PON-C02-S2-2-E08A}}
&
Pons, Chapter~2, Ex.~2.2.8(a) — If $(a_n)$ converges but $(b_n)$ diverges, show $(a_n+b_n)$ must diverge.
\\ \hline

\phantomsection
\hypertarget{ws-RA-PON-C02-S2-2-E08B}{}
\hyperlink{proof-RA-PON-C02-S2-2-E08B}{\texttt{RA-PON-C02-S2-2-E08B}}
&
Pons, Chapter~2, Ex.~2.2.8(b) — If $(a_n)$ and $(b_n)$ both diverge, must $(a_n+b_n)$ diverge? Explain.
\\ \hline

\phantomsection
\hypertarget{ws-RA-PON-C02-S2-2-E08C}{}
\hyperlink{proof-RA-PON-C02-S2-2-E08C}{\texttt{RA-PON-C02-S2-2-E08C}}
&
Pons, Chapter~2, Ex.~2.2.8(c) — If $(a_n)$ diverges and $c\in\mathbb{R}$ with $c\ne 0$, show $(c a_n)$ diverges.
\\ \hline

\phantomsection
\hypertarget{ws-RA-PON-C02-S2-2-E09A}{}
\hyperlink{proof-RA-PON-C02-S2-2-E09A}{\texttt{RA-PON-C02-S2-2-E09A}}
&
Pons, Chapter~2, Ex.~2.2.9(a) — If $(a_n)$ is bounded and $(b_n)\to 0$, show $(a_n b_n)\to 0$ (and explain why ALT cannot be used).
\\ \hline

\phantomsection
\hypertarget{ws-RA-PON-C02-S2-2-E09B}{}
\hyperlink{proof-RA-PON-C02-S2-2-E09B}{\texttt{RA-PON-C02-S2-2-E09B}}
&
Pons, Chapter~2, Ex.~2.2.9(b) — Does a similar result hold if $(b_n)$ converges to a nonzero limit?
\\ \hline

\phantomsection
\hypertarget{ws-RA-PON-C02-S2-2-E09C}{}
\hyperlink{proof-RA-PON-C02-S2-2-E09C}{\texttt{RA-PON-C02-S2-2-E09C}}
&
Pons, Chapter~2, Ex.~2.2.9(c) — Does the result hold if $(a_n)$ is not bounded and $(b_n)\to 0$?
\\ \hline

\phantomsection
\hypertarget{ws-RA-PON-C02-S2-2-E10A}{}
\hyperlink{proof-RA-PON-C02-S2-2-E10A}{\texttt{RA-PON-C02-S2-2-E10A}}
&
Pons, Chapter~2, Ex.~2.2.10(a) — If $c>0$ and $(a_n)\to\infty$, show $(c a_n)\to\infty$.
\\ \hline

\phantomsection
\hypertarget{ws-RA-PON-C02-S2-2-E10B}{}
\hyperlink{proof-RA-PON-C02-S2-2-E10B}{\texttt{RA-PON-C02-S2-2-E10B}}
&
Pons, Chapter~2, Ex.~2.2.10(b) — If $(a_n)\to\infty$ and $(b_n)\to\infty$, show $(a_n+b_n)\to\infty$.
\\ \hline

\phantomsection
\hypertarget{ws-RA-PON-C02-S2-2-E10C}{}
\hyperlink{proof-RA-PON-C02-S2-2-E10C}{\texttt{RA-PON-C02-S2-2-E10C}}
&
Pons, Chapter~2, Ex.~2.2.10(c) — If $(a_n)\to\infty$ and $(b_n)\to\infty$, show $(a_n b_n)\to\infty$.
\\ \hline

\phantomsection
\hypertarget{ws-RA-PON-C02-S2-2-E10D}{}
\hyperlink{proof-RA-PON-C02-S2-2-E10D}{\texttt{RA-PON-C02-S2-2-E10D}}
&
Pons, Chapter~2, Ex.~2.2.10(d) — What conclusions can be drawn about the behavior of $(a_n/b_n)$?
\\ \hline

\phantomsection
\hypertarget{ws-RA-PON-C02-S2-2-E14}{}
\hyperlink{proof-RA-PON-C02-S2-2-E14}{\texttt{RA-PON-C02-S2-2-E14}}
&
Pons, Chapter~2, Ex.~2.2.14 — Let $A\subseteq\mathbb{R}$ be nonempty and bounded. Show there exist sequences $(x_n)$ and $(y_n)$ in $A$ with $x_n\to \sup(A)$ and $y_n\to \inf(A)$.
\\ \hline

\phantomsection
\hypertarget{ws-RA-PON-C02-S2-2-E15}{}
\hyperlink{proof-RA-PON-C02-S2-2-E15}{\texttt{RA-PON-C02-S2-2-E15}}
&
Pons, Chapter~2, Ex.~2.2.15 — Suppose $(a_n)$ has positive terms. If $\lim (n a_n)$ exists, show that $(a_n)\to 0$.
\\ \hline

\phantomsection
\hypertarget{ws-RA-PON-C02-S2-2-E16}{}
\hyperlink{proof-RA-PON-C02-S2-2-E16}{\texttt{RA-PON-C02-S2-2-E16}}
&
Pons, Chapter~2, Ex.~2.2.16 — For each of the following, give an example of a sequence with the specified property.
\\ \hline


% -------------------------
% Chapter 2, §2.3 — Completeness in $\mathbb{R}$ Revisited
% -------------------------

\phantomsection
\hypertarget{ws-RA-PON-C02-S2-3-E01}{}
\hyperlink{proof-RA-PON-C02-S2-3-E01}{\texttt{RA-PON-C02-S2-3-E01}}
&
Pons, Chapter~2, Ex.~2.3.1 — $y_1=1$, $y_{n+1}=(3y_n+4)/4$: show $(y_n)$ converges and find its limit.
\\ \hline

\phantomsection
\hypertarget{ws-RA-PON-C02-S2-3-E02}{}
\hyperlink{proof-RA-PON-C02-S2-3-E02}{\texttt{RA-PON-C02-S2-3-E02}}
&
Pons, Chapter~2, Ex.~2.3.2 — $y_1=8$, $y_{n+1}=(3y_n+4)/4$: show $(y_n)$ converges and find its limit.
\\ \hline

\phantomsection
\hypertarget{ws-RA-PON-C02-S2-3-E03}{}
\hyperlink{proof-RA-PON-C02-S2-3-E03}{\texttt{RA-PON-C02-S2-3-E03}}
&
Pons, Chapter~2, Ex.~2.3.3 — $y_1=1$, $y_{n+1}=4-\dfrac{1}{y_n}$: show $(y_n)$ converges and find its limit.
\\ \hline

\phantomsection
\hypertarget{ws-RA-PON-C02-S2-3-E04}{}
\hyperlink{proof-RA-PON-C02-S2-3-E04}{\texttt{RA-PON-C02-S2-3-E04}}
&
Pons, Chapter~2, Ex.~2.3.4 — $x_k=\max\{\sin 1,\sin 2,\dots,\sin k\}$: show $(x_k)$ converges (no need to find the limit).
\\ \hline

\phantomsection
\hypertarget{ws-RA-PON-C02-S2-3-E07A}{}
\hyperlink{proof-RA-PON-C02-S2-3-E07A}{\texttt{RA-PON-C02-S2-3-E07A}}
&
Pons, Chapter~2, Ex.~2.3.7(a) — If $0<x<1$, show $(x^n)\to 0$.
\\ \hline

\phantomsection
\hypertarget{ws-RA-PON-C02-S2-3-E07B}{}
\hyperlink{proof-RA-PON-C02-S2-3-E07B}{\texttt{RA-PON-C02-S2-3-E07B}}
&
Pons, Chapter~2, Ex.~2.3.7(b) — If $-1<x<0$, show $(x^n)\to 0$.
\\ \hline

\phantomsection
\hypertarget{ws-RA-PON-C02-S2-3-E08A}{}
\hyperlink{proof-RA-PON-C02-S2-3-E08A}{\texttt{RA-PON-C02-S2-3-E08A}}
&
Pons, Chapter~2, Ex.~2.3.8(a) — Additive sequence: show $(2n)_{n=1}^\infty$ is additive.
\\ \hline

\phantomsection
\hypertarget{ws-RA-PON-C02-S2-3-E08B}{}
\hyperlink{proof-RA-PON-C02-S2-3-E08B}{\texttt{RA-PON-C02-S2-3-E08B}}
&
Pons, Chapter~2, Ex.~2.3.8(b) — If $(a_n)$ is additive, show $\left(\dfrac{a_n}{n}\right)$ converges and find the limit.
\\ \hline

\phantomsection
\hypertarget{ws-RA-PON-C02-S2-3-E09A}{}
\hyperlink{proof-RA-PON-C02-S2-3-E09A}{\texttt{RA-PON-C02-S2-3-E09A}}
&
Pons, Chapter~2, Ex.~2.3.9(a) — Subadditive sequence: show $(2n+1)_{n=1}^\infty$ is subadditive.
\\ \hline

\phantomsection
\hypertarget{ws-RA-PON-C02-S2-3-E09B}{}
\hyperlink{proof-RA-PON-C02-S2-3-E09B}{\texttt{RA-PON-C02-S2-3-E09B}}
&
Pons, Chapter~2, Ex.~2.3.9(b) — If $(a_n)$ is subadditive of positive reals, show $\left(\dfrac{a_n}{n}\right)$ has a convergent subsequence.
\\ \hline

\phantomsection
\hypertarget{ws-RA-PON-C02-S2-3-E09C}{}
\hyperlink{proof-RA-PON-C02-S2-3-E09C}{\texttt{RA-PON-C02-S2-3-E09C}}
&
Pons, Chapter~2, Ex.~2.3.9(c) — If $(a_n)$ is subadditive of positive reals, show $\left(\dfrac{a_n}{n}\right)$ is convergent.
\\ \hline

\phantomsection
\hypertarget{ws-RA-PON-C02-S2-3-E11A}{}
\hyperlink{proof-RA-PON-C02-S2-3-E11A}{\texttt{RA-PON-C02-S2-3-E11A}}
&
Pons, Chapter~2, Ex.~2.3.11(a) — If $(a_n)$ is bounded, show there is a subsequence converging to $\limsup a_n$.
\\ \hline

\phantomsection
\hypertarget{ws-RA-PON-C02-S2-3-E11B}{}
\hyperlink{proof-RA-PON-C02-S2-3-E11B}{\texttt{RA-PON-C02-S2-3-E11B}}
&
Pons, Chapter~2, Ex.~2.3.11(b) — Give an analogous statement for $\liminf a_n$ and supply details.
\\ \hline

\phantomsection
\hypertarget{ws-RA-PON-C02-S2-3-E11C}{}
\hyperlink{proof-RA-PON-C02-S2-3-E11C}{\texttt{RA-PON-C02-S2-3-E11C}}
&
Pons, Chapter~2, Ex.~2.3.11(c) — Use (a) or (b) to give a short proof of the Bolzano--Weierstrass Theorem.
\\ \hline

\phantomsection
\hypertarget{ws-RA-PON-C02-S2-3-E11D}{}
\hyperlink{proof-RA-PON-C02-S2-3-E11D}{\texttt{RA-PON-C02-S2-3-E11D}}
&
Pons, Chapter~2, Ex.~2.3.11(d) — If $(a_n)$ is bounded and non-convergent, show it has at least two subsequences converging to different values.
\\ \hline

\phantomsection
\hypertarget{ws-RA-PON-C02-S2-3-E11E}{}
\hyperlink{proof-RA-PON-C02-S2-3-E11E}{\texttt{RA-PON-C02-S2-3-E11E}}
&
Pons, Chapter~2, Ex.~2.3.11(e) — If $(a_n)$ is bounded and every convergent subsequence converges to the same $a$, show $(a_n)\to a$.
\\ \hline

\end{tabular}
\end{center}
