% --------------------------------------
% Proof: RE-ABB-T-DENSE-Q
% Theorem — Density of Q in R
% Format: Deep study (3-column)
% --------------------------------------
\clearpage
\phantomsection
\label{prf:density-Q}
\hypertarget{proof-RE-ABB-T-DENSE-Q}{}

\subsubsection[Density of $\mathbb{Q}$ in $\mathbb{R}$]{Proof --- RE-ABB-T-DENSE-Q}

\begin{remark}[Return]
\hyperref[thm:density-Q]{$\leftarrow$ Back to Theorem (Density of $\mathbb{Q}$) in Notes}
\end{remark}

\bigskip

\noindent
\textbf{Source.}
\srccite{AbbottUnderstandingAnalysis}{Chapter~1, \S1.4, Theorem~1.4.3};
\srccite{TaoAnalysis1}{Chapter~5}.

\vspace{0.75em}

\noindent
\textbf{Goal.}
For every $a, b \in \mathbb{R}$ with $a < b$, there exists $q \in \mathbb{Q}$
with $a < q < b$.

\vspace{0.75em}

\noindent
\textbf{Logical form.}
$\forall a, b \in \mathbb{R},\ (a < b) \Rightarrow \exists q \in \mathbb{Q},\ (a < q < b)$.

\vspace{0.75em}

\noindent
\textbf{Key background.}
Archimedean Property (Theorem \ref{thm:archimedean}): $\forall x > 0,\ \exists n \in \mathbb{N},\ 1/n < x$.
Floor Lemma: $\forall x \in \mathbb{R},\ \exists! m \in \mathbb{Z},\ m \le x < m+1$.

\vspace{0.75em}

\noindent
\textbf{Proof strategy.}
Find $n \in \mathbb{N}$ large enough that the grid $\{\ldots, -1/n, 0, 1/n, 2/n, \ldots\}$
is fine enough to fit a grid point between $a$ and $b$.
Use the Floor Lemma to locate the right grid point $m/n$.

\vspace{0.75em}

\noindent
\textbf{Proof.}
\begin{proof}
\begin{longtable}{p{0.28\textwidth} p{0.35\textwidth} p{0.28\textwidth}}
\toprule
\textbf{Step / Claim} & \textbf{Justification} & \textbf{Annotation} \\
\midrule
Choose $n \in \mathbb{N}$ with $1/n < b - a$.
& Archimedean Property applied to $b - a > 0$.
& The grid of spacing $1/n$ is fine enough to fit between $a$ and $b$.
\\
\addlinespace
Let $m := \lfloor na \rfloor$, so $m \le na < m+1$.
& Floor Lemma applied to $na$.
& $m$ is the largest integer not exceeding $na$.
\\
\addlinespace
Set $q := (m+1)/n$.
& Explicit construction.
& $q$ is a rational number: integer numerator, natural denominator.
\\
\addlinespace
$q > a$: $q = (m+1)/n > na/n = a$.
& $m+1 > na$ (from $m = \lfloor na \rfloor$).
& $q$ exceeds $a$.
\\
\addlinespace
$q < b$: $q = (m+1)/n \le (na+1)/n = a + 1/n < a + (b-a) = b$.
& $m+1 \le na + 1$ (from floor) and $1/n < b-a$.
& $q$ is below $b$. Both bounds established.
\\
\addlinespace
$q \in \mathbb{Q}$.
& $q = (m+1)/n$ with $m+1 \in \mathbb{Z}$, $n \in \mathbb{N}$.
& By construction.
\\
\addlinespace
Hence $a < q < b$ and $q \in \mathbb{Q}$.
& Summary of above.
& Done.
\\
\bottomrule
\end{longtable}
\end{proof}

\vspace{0.75em}

\noindent
\textbf{Remarks.}
\begin{itemize}
  \item \textbf{Proof shape.} Choose $n$ (via Archimedean), choose $m$ (via Floor),
    construct $q = (m+1)/n$, verify two inequalities. This is the canonical shape
    of a density argument: gridding + floor.

  \item \textbf{What the proof uses.} Archimedean (to make the grid fine) and
    Floor (to locate the right grid point). Both depend on Completeness.
    Density of $\mathbb{Q}$ in $\mathbb{R}$ is thus a consequence of completeness.

  \item \textbf{Corollary: irrationals are dense.} Between any $a < b$ there
    is a rational $q$, hence also an irrational $q + \sqrt{2}/n$ for $n$ large.
    The irrationals are also dense.

  \item \textbf{Common error.} Not verifying both $q > a$ and $q < b$.
    Both inequalities need explicit calculation; neither is automatic from
    the floor construction alone.
\end{itemize}
