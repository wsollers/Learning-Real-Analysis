% --------------------------------------
% Proof: RE-TAO-T-ARCH
% Theorem — Archimedean Property of R
% Format: Deep study (3-column)
% --------------------------------------
\clearpage
\phantomsection
\label{prf:archimedean}
\hypertarget{proof-RE-TAO-T-ARCH}{}

\subsubsection[Archimedean Property of $\mathbb{R}$]{Proof --- RE-TAO-T-ARCH}

\begin{remark}[Return]
\hyperref[thm:archimedean]{$\leftarrow$ Back to Theorem (Archimedean Property) in Notes}
\end{remark}

\bigskip

\noindent
\textbf{Source.}
\srccite{AbbottUnderstandingAnalysis}{Chapter~1, \S1.4, Theorem~1.4.2};
\srccite{TaoAnalysis1}{Chapter~5}.

\vspace{0.75em}

\noindent
\textbf{Goal.}
For every $x \in \mathbb{R}$, there exists $n \in \mathbb{N}$ with $n > x$.
Equivalently: $\mathbb{N}$ is not bounded above in $\mathbb{R}$.

\vspace{0.75em}

\noindent
\textbf{Logical form.}
$\forall x \in \mathbb{R},\ \exists n \in \mathbb{N},\ n > x$.

\vspace{0.75em}

\noindent
\textbf{Key background.}
Completeness Axiom: every nonempty subset of $\mathbb{R}$ bounded above has a supremum.
$\varepsilon$-characterization of supremum (Prop.~\ref{prop:eps-char}).

\vspace{0.75em}

\noindent
\textbf{Proof strategy.}
Suppose $\mathbb{N}$ is bounded above. Completeness gives $s = \sup \mathbb{N}$.
By the $\varepsilon$-characterization with $\varepsilon = 1$, some $m \in \mathbb{N}$
satisfies $m > s - 1$, so $m + 1 > s$. But $m + 1 \in \mathbb{N}$, contradicting
$s$ being an upper bound. Hence $\mathbb{N}$ is not bounded above.

\vspace{0.75em}

\noindent
\textbf{Proof.}
\begin{proof}
\begin{longtable}{p{0.28\textwidth} p{0.35\textwidth} p{0.28\textwidth}}
\toprule
\textbf{Step / Claim} & \textbf{Justification} & \textbf{Annotation} \\
\midrule
Suppose $\mathbb{N}$ is bounded above in $\mathbb{R}$.
& Assumption for contradiction.
& We derive a contradiction from this.
\\
\addlinespace
$s := \sup \mathbb{N}$ exists.
& Completeness Axiom: $\mathbb{N}$ is nonempty and bounded above.
& The supremum lives in $\mathbb{R}$.
\\
\addlinespace
$\exists m \in \mathbb{N}$ with $m > s - 1$.
& $\varepsilon$-characterization of $\sup$ with $\varepsilon := 1$.
& The element $m$ is in $\mathbb{N}$; it comes within $1$ of $s$.
\\
\addlinespace
$m + 1 > s$.
& Add $1$ to $m > s - 1$.
& This is the key: $m + 1$ overshoots the claimed supremum.
\\
\addlinespace
$m + 1 \in \mathbb{N}$.
& $\mathbb{N}$ is closed under successor.
& $m + 1$ is a natural number (as $m$ is and $\mathbb{N}$ contains all successors).
\\
\addlinespace
Contradiction: $m + 1 \in \mathbb{N}$ yet $m + 1 > s = \sup \mathbb{N}$.
& $s$ is an upper bound for $\mathbb{N}$; no element of $\mathbb{N}$ exceeds $s$.
& The element $m+1$ violates the upper bound.
\\
\addlinespace
Hence $\mathbb{N}$ is not bounded above.
& Contradiction closed.
& Given $x \in \mathbb{R}$, there exists $n \in \mathbb{N}$ with $n > x$.
\\
\bottomrule
\end{longtable}
\end{proof}

\vspace{0.75em}

\noindent
\textbf{Remarks.}
\begin{itemize}
  \item \textbf{Completeness is essential.} The Archimedean property does not
    hold in all ordered fields. It fails, for example, in the field of rational
    functions with formal infinitely large elements. It holds in $\mathbb{R}$
    \emph{because} of completeness. This proof makes that dependency explicit.

  \item \textbf{Proof structure.} The proof is contradiction + completeness +
    $\varepsilon$-characterization. The triple is a template that recurs
    throughout analysis whenever one wants to derive arithmetic consequences
    of the least upper bound property.

  \item \textbf{Consequence.} The Archimedean property unlocks the Floor Lemma,
    which in turn enables density of $\mathbb{Q}$ in $\mathbb{R}$.
    Every subsequent limit argument that requires choosing $N$ large enough
    invokes the Archimedean property (often silently).

  \item \textbf{Alternative form.} The property is equivalent to: for all $\varepsilon > 0$,
    there exists $n \in \mathbb{N}$ with $1/n < \varepsilon$. This is the form
    most used in $\varepsilon$-$N$ arguments.
\end{itemize}
