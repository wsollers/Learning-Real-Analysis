% --------------------------------------
% Proof: RE-ABB-T-NIP
% Theorem — Nested Interval Property
% Format: Sketch (compact)
% --------------------------------------
\clearpage
\phantomsection
\label{prf:nested-interval}
\hypertarget{proof-RE-ABB-T-NIP}{}

\subsubsection[Nested Interval Property]{Proof --- RE-ABB-T-NIP}

\begin{remark}[Return]
\hyperref[thm:nested-interval]{$\leftarrow$ Back to Theorem (Nested Interval Property) in Notes}
\end{remark}

\bigskip

\begin{tcolorbox}[colback=gray!6, colframe=gray!40, arc=2pt,
  left=6pt, right=6pt, top=4pt, bottom=4pt,
  title={\small\textbf{Proof Sketch --- Nested Interval Property}},
  fonttitle=\small\bfseries]

\textbf{Statement.}
Let $[a_n, b_n]$ be closed intervals with $[a_{n+1}, b_{n+1}] \subseteq [a_n, b_n]$
for all $n$. Then $\bigcap_{n=1}^\infty [a_n, b_n] \neq \varnothing$.
If additionally $b_n - a_n \to 0$, the intersection is a single point.

\textbf{Key move.}
The left endpoints $(a_n)$ form an increasing sequence bounded above by $b_1$.
By the Completeness Axiom, $a := \sup\{a_n\}$ exists.
Since $a_n \le a$ and $a \le b_n$ for all $n$ (from the nesting),
$a$ lies in every interval $[a_n, b_n]$.
If $b_n - a_n \to 0$, any two intersection points must satisfy $|x - y| \le b_n - a_n \to 0$,
forcing uniqueness.

\textbf{Depends on.}
Completeness Axiom (for existence of $a = \sup\{a_n\}$).
The nesting condition translates to: $(a_n)$ increasing and bounded above by every $b_n$.
\end{tcolorbox}

\vspace{0.75em}

\noindent
\textbf{Remarks.}
\begin{itemize}
  \item \textbf{Topological significance.} This is a special case of the fact
    that $\mathbb{R}$ (with the standard metric) is \emph{complete as a metric space}:
    every Cauchy sequence converges. The Nested Interval Property is
    equivalent to completeness over the ordered field axioms.

  \item \textbf{Open intervals.} The theorem fails for open intervals: the
    intersection of $(0, 1/n)$ is empty. Closedness is essential.

  \item \textbf{Use in later proofs.} The Nested Interval Property is the
    basis for the bisection argument used to prove the Bolzano--Weierstrass
    theorem. Recognise it as the structural backbone of those arguments.
\end{itemize}
