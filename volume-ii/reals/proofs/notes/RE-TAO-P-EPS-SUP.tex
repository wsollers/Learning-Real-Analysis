% --------------------------------------
% Proof: RE-TAO-P-EPS-SUP
% Proposition — epsilon-Characterization of Supremum and Infimum
% Format: Deep study (3-column)
% --------------------------------------
\clearpage
\phantomsection
\label{prf:eps-char}
\hypertarget{proof-RE-TAO-P-EPS-SUP}{}

\subsubsection[$\varepsilon$-Characterization of Supremum]{Proof --- RE-TAO-P-EPS-SUP}

\begin{remark}[Return]
\hyperref[prop:eps-char]{$\leftarrow$ Back to Proposition ($\varepsilon$-Characterization) in Notes}
\end{remark}

\bigskip

\noindent
\textbf{Source.}
\srccite{AbbottUnderstandingAnalysis}{Chapter~1, \S1.3 (Lemma~1.3.8)};
also \srccite{TaoAnalysis1}{Chapter~5}.

\vspace{0.75em}

\noindent
\textbf{Goal.}
If $A \subseteq \mathbb{R}$ is nonempty and bounded above, then $s = \sup A$
if and only if:
(a) $s$ is an upper bound for $A$, and
(b) for every $\varepsilon > 0$, there exists $a \in A$ with $s - \varepsilon < a$.

\vspace{0.75em}

\noindent
\textbf{Logical form.}
$s = \sup A \iff \bigl[(\forall a \in A,\ a \le s) \land
(\forall \varepsilon > 0,\ \exists a \in A,\ a > s - \varepsilon)\bigr]$.

\vspace{0.75em}

\noindent
\textbf{Key background.}
Definition of supremum (least upper bound): $s = \sup A$ means $s$ is an upper
bound and $s \le u$ for every upper bound $u$.

\vspace{0.75em}

\noindent
\textbf{Proof strategy.}
Biconditional: prove both directions.
$(\Rightarrow)$: The ``least'' in ``least upper bound'' immediately implies the $\varepsilon$-condition,
by contrapositive (if $s - \varepsilon$ is not below any element of $A$, it would be a smaller upper bound).
$(\Leftarrow)$: Show any upper bound $u < s$ leads to contradiction via the $\varepsilon$-condition
with $\varepsilon := s - u$.

\vspace{0.75em}

\noindent
\textbf{Proof.}
\begin{proof}
\begin{longtable}{p{0.30\textwidth} p{0.33\textwidth} p{0.28\textwidth}}
\toprule
\textbf{Step / Claim} & \textbf{Justification} & \textbf{Annotation} \\
\midrule
\multicolumn{3}{l}{\textbf{($\Rightarrow$)} Assume $s = \sup A$.} \\
\addlinespace
$s$ is an upper bound for $A$.
& Definition of supremum.
& Condition (a) holds directly.
\\
\addlinespace
Let $\varepsilon > 0$. Then $s - \varepsilon < s$.
& Arithmetic.
& We need to find $a \in A$ above $s - \varepsilon$.
\\
$s - \varepsilon$ is not an upper bound for $A$.
& $s$ is the \emph{least} upper bound; any smaller value fails.
& If $s - \varepsilon$ were an upper bound, $s$ would not be least.
\\
$\exists a \in A$ with $a > s - \varepsilon$.
& Definition of ``not an upper bound.''
& Condition (b) holds. $(\Rightarrow)$ complete.
\\
\midrule
\multicolumn{3}{l}{\textbf{($\Leftarrow$)} Assume (a) and (b) hold.} \\
\addlinespace
Let $u$ be any upper bound for $A$.
& Hypothesis: must show $s \le u$.
& We want $s$ to be the \emph{least} upper bound.
\\
Suppose for contradiction $u < s$.
& Proof by contradiction.
& If no such $u$ exists, done; so assume one does.
\\
Set $\varepsilon := s - u > 0$.
& $u < s$ implies $s - u > 0$.
& Choose $\varepsilon$ to be exactly the gap.
\\
By (b), $\exists a \in A$ with $a > s - \varepsilon = u$.
& Condition (b) with this $\varepsilon$.
& An element of $A$ exceeds $u$.
\\
Contradiction: $u$ is an upper bound for $A$ yet $a > u$ with $a \in A$.
& Definition of upper bound.
& $u$ fails to be an upper bound. Contradiction.
\\
Hence $s \le u$.
& Contradiction closed.
& $s$ is the least upper bound: $s = \sup A$. $(\Leftarrow)$ complete.
\\
\bottomrule
\end{longtable}
\end{proof}

\vspace{0.75em}

\noindent
\textbf{Remarks.}
\begin{itemize}
  \item \textbf{Why this matters.} The definition of $\sup A$ as the least upper
    bound is hard to use directly: checking that $s \le u$ for \emph{every} upper
    bound $u$ requires knowing all upper bounds.
    The $\varepsilon$-characterization replaces this with a local condition:
    for each $\varepsilon$, just exhibit one element of $A$ above $s - \varepsilon$.
    This is the tool actually used in proofs.

  \item \textbf{Pattern.} The $(\Leftarrow)$ direction uses a classic structure:
    assume $u < s$, set $\varepsilon := s - u$, apply the $\varepsilon$-condition,
    derive a contradiction. Memorise this as the standard proof template for
    ``$s$ is the infimum/supremum.''

  \item \textbf{Consequence.} Corollary ($\varepsilon$-Approximation):
    if $s = \sup A$ then for every $\varepsilon > 0$ there exists $a \in A$
    with $s - \varepsilon < a \le s$. This follows immediately from the
    $\varepsilon$-characterization plus the upper bound condition.

  \item \textbf{Common error.} Confusing direction: in the $(\Leftarrow)$ proof,
    the contradiction comes from assuming $u < s$ and deriving that $u$ is
    \emph{not} an upper bound. Do not accidentally assume $u > s$.
\end{itemize}
