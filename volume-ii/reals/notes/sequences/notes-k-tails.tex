% =========================================================
% K-Tails of Sequences
% File: notes-k-tails.tex
% =========================================================

\subsection{K-Tails of Sequences}

% ---------------------------------------------------------
% Toolkit
% ---------------------------------------------------------
\begin{tcolorbox}[colback=gray!6, colframe=gray!40, arc=2pt,
  left=6pt, right=6pt, top=4pt, bottom=4pt,
  title={\small\textbf{K Tails — Quick Reference}},
  fonttitle=\small\bfseries]
\begin{tabular}{@{}p{0.28\textwidth}p{0.68\textwidth}@{}}
\textbf{Core items} & Key definitions/results introduced in this file.\\
\textbf{How to use} & Read the boxed items first; proofs and consequences follow.\\
\textbf{Dependencies} & Refer back to earlier sections as needed.\\
\end{tabular}
\end{tcolorbox}


% (Original local heading preserved)
\subsubsection{K-Tails of Sequences}

The long-term behavior of a sequence is completely determined by its
\emph{tails}.  Finite initial segments play no role in convergence,
Cauchy behavior, limit superior/inferior, or subsequence extraction.

This section formalizes that principle.

% ---------------------------------------------------------
\subsubsection{Basic Definitions}

\begin{tcolorbox}[colback=propbox, colframe=propborder, arc=2pt,
  left=6pt, right=6pt, top=4pt, bottom=4pt,
  title={\small\textbf{Definition (k-tail of a sequence)}},
  fonttitle=\small\bfseries]
Let $(x_n)$ be a sequence in $\mathbb{R}$ and let $k\in\mathbb{N}$.
The \emph{$k$-tail} of $(x_n)$ is the sequence
\[
(x_{k}, x_{k+1}, x_{k+2}, \dots).
\]
\end{tcolorbox}

\begin{tcolorbox}[colback=propbox, colframe=propborder, arc=2pt,
  left=6pt, right=6pt, top=4pt, bottom=4pt,
  title={\small\textbf{Definition (Tail set)}},
  fonttitle=\small\bfseries]
Let $(x_n)$ be a sequence and let $k\in\mathbb{N}$.
The \emph{tail set} beginning at $k$ is
\[
T_k := \{x_n : n \ge k\}.
\]
\end{tcolorbox}

\begin{remark}
A $k$-tail removes the first $k-1$ terms of the sequence.
All analytic behavior of a sequence occurs in its tails.
\end{remark}

% ---------------------------------------------------------
\subsubsection{Main Theorems}

\begin{theorem}[Convergence is a Tail Property]
A sequence $(x_n)$ converges to $L$ if and only if
every $k$-tail converges to $L$.
\end{theorem}

\begin{proof}
($\Rightarrow$)
Assume $x_n \to L$.  
Let $\varepsilon > 0$.  
There exists $N$ such that for all $n \ge N$,
\[
|x_n - L| < \varepsilon.
\]
Fix $k$.  If $k \ge N$, then the entire $k$-tail lies
inside the $\varepsilon$-neighborhood of $L$.
If $k < N$, then all terms of the $k$-tail beyond index $N$
lie inside the neighborhood.  Hence the $k$-tail converges to $L$.

($\Leftarrow$)
If the first tail (the original sequence) converges to $L$,
then $(x_n)$ converges to $L$.
\end{proof}

\begin{theorem}[Finite Modification Theorem]
If two sequences differ in only finitely many terms,
then either both converge to the same limit or neither converges.
\end{theorem}

\begin{proof}
If $(x_n)$ and $(y_n)$ differ only in finitely many terms,
then there exists $N$ such that
\[
x_n = y_n \quad \text{for all } n \ge N.
\]
Hence their $N$-tails are identical.
Since convergence depends only on tails,
the result follows.
\end{proof}

\begin{theorem}[Cauchy is a Tail Property]
A sequence $(x_n)$ is Cauchy if and only if
every $k$-tail is Cauchy.
\end{theorem}

\begin{proof}
The Cauchy definition requires:
\[
\forall \varepsilon > 0 \; \exists N \;
\text{such that } m,n \ge N
\Rightarrow |x_m - x_n| < \varepsilon.
\]
This condition concerns only sufficiently large indices.
Hence it depends solely on the tail behavior.
\end{proof}

\begin{theorem}[Tail Suprema are Monotone]
Let $(x_n)$ be bounded and define
\[
s_k := \sup T_k.
\]
Then $(s_k)$ is a decreasing sequence.
\end{theorem}

\begin{proof}
Since $T_{k+1} \subseteq T_k$,
every upper bound of $T_k$ is also an upper bound of $T_{k+1}$.
Hence
\[
s_{k+1} \le s_k.
\]
\end{proof}

\begin{theorem}[Limit Superior via Tails]
If $(x_n)$ is bounded and
\[
s_k := \sup T_k,
\]
then
\[
\limsup x_n = \lim_{k\to\infty} s_k.
\]
\end{theorem}

\begin{remark}
Limit superior captures the eventual upper behavior of
shrinking tails.
\end{remark}

\begin{theorem}[Subsequences are Iterated Tails]
Every subsequence is obtained by repeatedly passing to
later tails.
\end{theorem}

\begin{proof}
A subsequence $(x_{n_k})$ satisfies
\[
n_1 < n_2 < n_3 < \dots.
\]
Thus $x_{n_1}$ lies in some tail,
$x_{n_2}$ lies in a later tail,
and so on.
Each selection is taken from a further tail.
\end{proof}

% ---------------------------------------------------------
\subsubsection{Consequences}

The logical implications of this section are:

\begin{itemize}
\item Convergence ignores finite initial segments.
\item Cauchy behavior ignores finite initial segments.
\item $\limsup$ and $\liminf$ are defined entirely through tail sets.
\item Subsequences are constructed by iterating tails.
\item All asymptotic behavior is tail behavior.
\end{itemize}

\begin{corollary}
Removing or altering finitely many terms of a convergent sequence
does not change its limit.
\end{corollary}

\begin{corollary}
If $(x_n)$ converges, then any tail of $(x_n)$ converges to the same limit.
\end{corollary}

\begin{remark}
The phrase ``for all sufficiently large $n$''
is precisely tail language.
\end{remark}

% ---------------------------------------------------------
\subsubsection{Structural Principle}

\begin{remark}[Tail Principle]
In real analysis, properties defined by
``eventually'' or ``for sufficiently large indices''
depend only on the tails of sequences.

Finite behavior is analytically irrelevant.
\end{remark}

\begin{remark}[Logical Structure]
\[
\text{Tail containment}
\Rightarrow
\text{Convergence}
\Rightarrow
\text{Finite modification invariance}.
\]

\[
\text{Tail nesting}
\Rightarrow
\text{Monotonic tail suprema}
\Rightarrow
\limsup \text{ theory}.
\]
\end{remark}
