% =========================================================
% Power Series and Radius of Convergence
% File: notes-power-series.tex
% =========================================================

\subsection{Power Series and Radius of Convergence}

% ---------------------------------------------------------
\subsubsection{Basic Definitions}

\begin{definition}[Power series]
Let $(a_n)$ be a sequence of real numbers and let $c \in \mathbb{R}$.
A \emph{power series centered at $c$} is a series of the form
\[
\sum_{n=0}^{\infty} a_n (x-c)^n.
\]
\end{definition}

\begin{definition}[Radius of convergence]
The \emph{radius of convergence} of a power series
\[
\sum_{n=0}^{\infty} a_n (x-c)^n
\]
is the number $R \in [0,\infty]$ such that:

\begin{itemize}
\item the series converges absolutely whenever $|x-c| < R$,
\item the series diverges whenever $|x-c| > R$.
\end{itemize}
\end{definition}

\begin{definition}[Interval of convergence]
The \emph{interval of convergence} is the set of $x$ for which the series converges.
It is of the form
\[
(c-R,c+R)
\]
possibly including one or both endpoints.
\end{definition}

% ---------------------------------------------------------
\subsubsection{Main Theorems}

\begin{theorem}[Radius of Convergence Theorem]
For every power series
\[
\sum_{n=0}^{\infty} a_n (x-c)^n,
\]
there exists $R \in [0,\infty]$ such that:

\begin{enumerate}
\item The series converges absolutely for all $|x-c|<R$.
\item The series diverges for all $|x-c|>R$.
\end{enumerate}
\end{theorem}

\begin{proof}
Consider the root test applied to the general term:
\[
\left|a_n (x-c)^n\right|^{1/n}
=
|a_n|^{1/n} |x-c|.
\]

Let
\[
L := \limsup_{n\to\infty} |a_n|^{1/n}.
\]

By the root test, the series converges absolutely when
\[
\limsup_{n\to\infty}
|a_n|^{1/n} |x-c|
< 1,
\]
i.e.,
\[
|x-c| < \frac{1}{L}.
\]

Similarly, it diverges when
\[
|x-c| > \frac{1}{L}.
\]

Define
\[
R := \frac{1}{L},
\]
with the conventions:
\[
L=0 \Rightarrow R=\infty,
\qquad
L=\infty \Rightarrow R=0.
\]

This establishes the existence of a radius $R$ with the stated properties.
\end{proof}

\begin{theorem}[Cauchy--Hadamard Formula]
Let
\[
\sum_{n=0}^{\infty} a_n (x-c)^n
\]
be a power series. Then the radius of convergence is
\[
R
=
\frac{1}{\limsup_{n\to\infty} |a_n|^{1/n}}.
\]
\end{theorem}

\begin{proof}
This follows directly from the root test applied in the previous theorem.
The convergence boundary is determined precisely by
\[
\limsup |a_n|^{1/n} |x-c| = 1.
\]
Solving for $|x-c|$ gives the formula for $R$.
\end{proof}

\begin{theorem}[Term-by-term differentiation]
Let
\[
f(x) = \sum_{n=0}^{\infty} a_n (x-c)^n
\]
have radius of convergence $R>0$.
Then for all $|x-c|<R$:

\begin{enumerate}
\item The series converges uniformly on every closed interval
\[
[c-r,c+r] \subset (c-R,c+R).
\]
\item The function $f$ is differentiable on $(c-R,c+R)$.
\item The derivative is obtained by term-by-term differentiation:
\[
f'(x)
=
\sum_{n=1}^{\infty} n a_n (x-c)^{n-1}.
\]
\item The differentiated series has the same radius of convergence $R$.
\end{enumerate}
\end{theorem}

\begin{proof}
Fix $r<R$. For $|x-c|\le r$, we have
\[
|a_n (x-c)^n|
\le
|a_n| r^n.
\]
Since $r<R$, the series
\[
\sum |a_n| r^n
\]
converges by the definition of $R$.
By the Weierstrass $M$-test, the power series converges uniformly on
$[c-r,c+r]$.

The same argument applies to
\[
\sum n a_n (x-c)^{n-1}.
\]

Uniform convergence of the derivative series implies that differentiation
may be performed term-by-term, and the radius remains unchanged because
\[
\limsup (n|a_n|)^{1/n}
=
\limsup |a_n|^{1/n}.
\]
\end{proof}

% ---------------------------------------------------------
\subsubsection{Consequences and Logical Structure}

\begin{remark}[Structural Position]
Power series sit at the intersection of:

\[
\text{Sequences}
\rightarrow
\text{Series}
\rightarrow
\text{Absolute convergence}
\rightarrow
\text{Root test}
\rightarrow
\text{limsup}.
\]

The Cauchy--Hadamard formula is the culmination of the entire
limsup theory.
\end{remark}

\begin{remark}[Uniform convergence inside the radius]
On every compact subinterval of $(c-R,c+R)$, power series converge uniformly.
This makes them exceptionally well-behaved:
\[
\text{Inside } R:
\quad
\text{Uniform convergence}
\Rightarrow
\text{Continuous}
\Rightarrow
\text{Differentiable}
\Rightarrow
\text{Smooth}.
\]
\end{remark}

\begin{remark}[Completeness connection]
The existence of $R$ ultimately depends on:
\begin{itemize}
\item completeness of $\mathbb{R}$,
\item limsup existence,
\item root test,
\item absolute convergence theory.
\end{itemize}

Thus power series are a structural synthesis of the entire sequence
and series development.
\end{remark}


