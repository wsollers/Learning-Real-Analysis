% =========================================================
% Completeness and Its Consequences
% =========================================================

\subsection{Completeness of the Real Numbers}

% ---------------------------------------------------------
% Toolkit
% ---------------------------------------------------------
\begin{tcolorbox}[colback=gray!6, colframe=gray!40, arc=2pt,
  left=6pt, right=6pt, top=4pt, bottom=4pt,
  title={\small\textbf{Completeness — Quick Reference}},
  fonttitle=\small\bfseries]
\begin{tabular}{@{}p{0.28\textwidth}p{0.68\textwidth}@{}}
\textbf{Core items} & Key definitions/results introduced in this file.\\
\textbf{How to use} & Read the boxed items first; proofs and consequences follow.\\
\textbf{Dependencies} & Refer back to earlier sections as needed.\\
\end{tabular}
\end{tcolorbox}


% ---------------------------------------------------------
\subsubsection{Completeness Axiom}
% ---------------------------------------------------------

\begin{remark}[Why completeness is needed]
The ordered field axioms alone do not prevent ``holes'' in the number line.
The rationals $\mathbb{Q}$ satisfy all field and order axioms, yet fail
completeness. Consider the set
\[
S = \{x \in \mathbb{Q} : x^2 < 2\}.
\]
This set is nonempty (e.g.\ $1 \in S$) and bounded above in $\mathbb{Q}$
(e.g.\ $2$ is an upper bound). Yet $\sup S$ does not exist as a rational
number: the candidate $\sqrt{2}$ is irrational. The set $S$ has no least
upper bound in $\mathbb{Q}$ --- a hole sits exactly where $\sqrt{2}$ should be.

The Completeness Axiom asserts that $\mathbb{R}$ has no such holes:
every nonempty bounded-above set has a supremum \emph{inside} $\mathbb{R}$.
This single axiom is what distinguishes $\mathbb{R}$ from $\mathbb{Q}$,
and it underlies every major theorem in analysis.
\end{remark}

\begin{tcolorbox}[colback=axiombox, colframe=axiomborder, arc=2pt,
  left=6pt, right=6pt, top=4pt, bottom=4pt,
  title={\small\textbf{Axiom (Completeness Axiom (Least Upper Bound Property))}},
  fonttitle=\small\bfseries]
Every nonempty subset of $\mathbb{R}$ that is bounded above has a supremum
in $\mathbb{R}$.

Equivalently: if $S \subseteq \mathbb{R}$ is nonempty and bounded above,
then $\sup S$ exists as a real number.
\end{tcolorbox}

\begin{remark}[Logical form]
\[
\forall S\;\Bigl(
(S \subseteq \mathbb{R} \wedge S \neq \varnothing \wedge
\exists M \in \mathbb{R}\ \forall x \in S\ (x \le M))
\rightarrow
\exists s \in \mathbb{R}\ (s = \sup S)
\Bigr).
\]
\end{remark}

\begin{remark}[Equivalent formulations]
Over the ordered-field axioms, completeness is equivalent to each of the
following:
\begin{itemize}
  \item Every nonempty set bounded below has an infimum.
  \item Every Cauchy sequence in $\mathbb{R}$ converges.
  \item \emph{(Nested Interval Property)} Every nested sequence of nonempty
        closed bounded intervals has nonempty intersection.
\end{itemize}
\end{remark}

% =========================================================
% Nested Interval Property
% =========================================================

\subsubsection{Nested Interval Property}

\begin{theorem}[Nested Interval Property]
\label{thm:nested-interval}
Let $\{[a_n, b_n]\}_{n \in \mathbb{N}}$ be nonempty closed intervals such that
\[
[a_{n+1}, b_{n+1}] \subseteq [a_n, b_n]
\quad \text{for all } n.
\]
Then
\[
\bigcap_{n=1}^\infty [a_n, b_n] \neq \varnothing.
\]
If additionally $b_n - a_n \to 0$, the intersection consists of exactly one
point.
\end{theorem}

\begin{remark}[Proof]
See \hyperref[prf:nested-interval]{Proof $\to$ RE-ABB-T-NIP (Nested Interval Property)}.
\end{remark}

\begin{proof}
From the nesting we have
\[
a_n \le a_{n+1}, \qquad b_{n+1} \le b_n.
\]
Thus $(a_n)$ is increasing and bounded above by $b_1$. By completeness,
\[
a := \sup\{a_n\}
\]
exists. Since $a_n \le a$ for all $n$, and for each fixed $n$ every $a_k$
with $k \ge n$ lies in $[a_n, b_n]$, we have $a \le b_n$. Thus
$a_n \le a \le b_n$, so
\[
a \in [a_n, b_n] \quad \text{for all } n,
\]
and the intersection is nonempty.

If $b_n - a_n \to 0$ and $x, y$ both lie in the intersection, then
\[
|x - y| \le b_n - a_n \to 0,
\]
so $x = y$.
\end{proof}

% =========================================================
% Archimedean Property
% =========================================================

\subsubsection{Archimedean Property}

\begin{tcolorbox}[colback=propbox, colframe=propborder, arc=2pt,
  left=6pt, right=6pt, top=4pt, bottom=4pt,
  title={\small\textbf{Definition}},
  fonttitle=\small\bfseries]
$\mathbb{R}$ satisfies the \emph{Archimedean property} if
\[
\forall x \in \mathbb{R}\; \exists n \in \mathbb{N}\; (n > x).
\]
Equivalently:
\[
\forall x > 0\; \forall y \in \mathbb{R}\; \exists n \in \mathbb{N}\; (nx > y).
\]
\end{tcolorbox}

\begin{remark}[Logical form]
\[
\forall x\; \exists n\; (n > x).
\qquad
\forall x > 0\; \forall y\; \exists n\; (nx > y).
\]
\end{remark}

\begin{theorem}[Archimedean Property]
\label{thm:archimedean}
$\mathbb{R}$ satisfies the Archimedean property.
\end{theorem}

\begin{remark}[Proof]
See \hyperref[prf:archimedean]{Proof $\to$ RE-TAO-T-ARCH (Archimedean Property)}.
\end{remark}

\begin{proof}
If not, then $\mathbb{N}$ is bounded above. By completeness,
$s := \sup \mathbb{N}$ exists. Then $s - 1$ is not an upper bound, so there
exists $m \in \mathbb{N}$ with $m > s - 1$. Thus $m + 1 > s$, contradicting
that $s$ is an upper bound for $\mathbb{N}$.
\end{proof}

\begin{corollary}
If $x > 0$ and $y \in \mathbb{R}$, then $\exists n \in \mathbb{N}$ such that
$nx > y$.
\end{corollary}

\begin{proof}
Apply the Archimedean property to $y/x$.
\end{proof}

% =========================================================
% Integer Part
% =========================================================

\subsubsection{Integer Part}

\begin{lemma}[Floor lemma]
For every $x \in \mathbb{R}$ there exists a unique $m \in \mathbb{Z}$ such that
\[
m \le x < m + 1.
\]
\end{lemma}

\begin{proof}
\textbf{Existence.} By the Archimedean property, the set
\[
L := \{k \in \mathbb{Z} : k \le x\}
\]
is nonempty (since $\mathbb{N}$ is unbounded below, some integer lies below
$x$) and bounded above by $x$. Since $\mathbb{Z}$ is discrete, $L$ attains
its maximum; let $m := \max L$. Then $m \le x$ by definition of $L$, and
$m + 1 > x$ by maximality of $m$. Hence $m \le x < m + 1$.

\textbf{Uniqueness.} If $m \le x < m + 1$ and $m' \le x < m' + 1$, then
$|m - m'| < 1$. Since $m - m' \in \mathbb{Z}$, we conclude $m = m'$.
\end{proof}

% =========================================================
% Density
% =========================================================

\subsubsection{Density}

\begin{tcolorbox}[colback=propbox, colframe=propborder, arc=2pt,
  left=6pt, right=6pt, top=4pt, bottom=4pt,
  title={\small\textbf{Definition}},
  fonttitle=\small\bfseries]
A subset $A$ of a linearly ordered set $X$ is \emph{dense in $X$} if
\[
\forall a < b\; \exists c \in A\; (a < c < b).
\]
\end{tcolorbox}

\begin{remark}[Logical form for $\mathbb{Q}$ dense in $\mathbb{R}$]
\[
\forall a < b\; \exists q \in \mathbb{Q}\; (a < q < b).
\]
\end{remark}

\begin{theorem}[Density of $\mathbb{Q}$]
\label{thm:density-Q}
$\mathbb{Q}$ is dense in $\mathbb{R}$.
\end{theorem}

\begin{remark}[Proof]
See \hyperref[prf:density-Q]{Proof $\to$ RE-ABB-T-DENSE-Q (Density of $\mathbb{Q}$ in $\mathbb{R}$)}.
\end{remark}

\begin{proof}
Let $a < b$. By the Archimedean property, choose $n \in \mathbb{N}$ with
$1/n < b - a$. By the Floor Lemma, let $m := \lfloor na \rfloor$, so
$m \le na < m + 1$. Set $q := (m + 1)/n$. Then
\[
q = \frac{m+1}{n} > \frac{na}{n} = a,
\qquad
q = \frac{m+1}{n} \le \frac{na + 1}{n} = a + \frac{1}{n} < a + (b-a) = b.
\]
Hence $a < q < b$ and $q \in \mathbb{Q}$.
\end{proof}

\begin{corollary}
Between any two distinct real numbers lies an irrational.
\end{corollary}

\begin{corollary}
The irrationals are dense in $\mathbb{R}$.
\end{corollary}

% =========================================================
% Existence of Square Roots
% =========================================================

\subsubsection{Existence of Square Roots}

\begin{theorem}
For every $a \ge 0$ there exists a unique $x \ge 0$ such that $x^2 = a$.
\end{theorem}

\begin{remark}[Logical form]
\[
\forall a \ge 0\; \exists!\, x \ge 0\; (x^2 = a).
\]
\end{remark}

\begin{proof}[Existence]
Let
\[
S = \{t \ge 0 : t^2 \le a\}.
\]
$S$ is nonempty ($0 \in S$) and bounded above (by $a + 1$), so $s := \sup S$
exists by completeness. If $s^2 > a$, then $s - \varepsilon$ is an upper
bound for $S$ for sufficiently small $\varepsilon > 0$, contradicting
minimality of $s$. If $s^2 < a$, then $s + \delta \in S$ for sufficiently
small $\delta > 0$, contradicting that $s$ is an upper bound. Hence $s^2 = a$.
\end{proof}

\begin{proof}[Uniqueness]
If $x^2 = y^2$ and $x, y \ge 0$, then $(x - y)(x + y) = 0$. Since
$x + y \ge 0$, either $x + y = 0$ (forcing $x = y = 0$) or $x - y = 0$.
In either case $x = y$.
\end{proof}

% =========================================================
% Structural Summary
% =========================================================

\subsubsection*{Structural Summary}

\[
\text{Field Axioms}
\;\Rightarrow\;
\text{Order Axioms}
\;\Rightarrow\;
\text{Completeness Axiom}
\]

Completeness yields, in logical order:

\[
\text{Nested Interval Property}
\;\Rightarrow\;
\text{Archimedean Property}
\;\Rightarrow\;
\text{Floor Lemma}
\;\Rightarrow\;
\text{Density of } \mathbb{Q}
\;\Rightarrow\;
\text{Existence of } \sqrt{a}
\]

Completeness is the property that prevents holes in $\mathbb{R}$. It is
equivalent to the Cauchy Criterion and underlies every major limit theorem
in real analysis.