% =========================================================
% Axioms of the Real Numbers
% File: notes-axioms-of-the-reals.tex
% =========================================================

\subsection{Axioms of the Real Numbers}

% ---------------------------------------------------------
% Toolkit
% ---------------------------------------------------------
\begin{tcolorbox}[colback=gray!6, colframe=gray!40, arc=2pt,
  left=6pt, right=6pt, top=4pt, bottom=4pt,
  title={\small\textbf{Axioms Of The Reals — Quick Reference}},
  fonttitle=\small\bfseries]
\begin{tabular}{@{}p{0.28\textwidth}p{0.68\textwidth}@{}}
\textbf{Core items} & Key definitions/results introduced in this file.\\
\textbf{How to use} & Read the boxed items first; proofs and consequences follow.\\
\textbf{Dependencies} & Refer back to earlier sections as needed.\\
\end{tabular}
\end{tcolorbox}


The real numbers $\mathbb{R}$ form a \emph{totally ordered field}.
This structure consists of:

\begin{itemize}
\item Field axioms (algebraic structure),
\item Order axioms (order structure),
\item Completeness axiom (analytic structure).
\end{itemize}

% ---------------------------------------------------------
\subsubsection{Basic Definitions}

\begin{remark}
A \emph{field} is a set equipped with two binary operations,
addition $(+)$ and multiplication $(\cdot)$,
satisfying closure, associativity, commutativity,
identity, inverse, and distributive laws.

An \emph{ordered field} is a field equipped with a total order
compatible with the algebraic operations.
\end{remark}

% ---------------------------------------------------------
\subsubsection{Main Theorems (Axioms)}

% =========================================================
% Field Axioms
% =========================================================

\paragraph{Additive Axioms}

\begin{description}

\item[\textbf{Axiom A1 (Additive Closure).}]
\[
\forall x \forall y \, (x,y \in \mathbb{R} \rightarrow x+y \in \mathbb{R})
\]

\item[\textbf{Axiom A2 (Additive Commutativity).}]
\[
\forall x \forall y \, (x+y = y+x)
\]

\item[\textbf{Axiom A3 (Additive Associativity).}]
\[
\forall x \forall y \forall z \, ((x+y)+z = x+(y+z))
\]

\item[\textbf{Axiom A4 (Additive Identity).}]
\[
\exists 0 \, \forall x \, (x+0 = x)
\]

\item[\textbf{Axiom A5 (Additive Inverse).}]
\[
\forall x \, \exists y \, (x+y = 0)
\]

\end{description}

\paragraph{Multiplicative Axioms}

\begin{description}

\item[\textbf{Axiom M1 (Multiplicative Closure).}]
\[
\forall x \forall y \, (x,y \in \mathbb{R} \rightarrow x\cdot y \in \mathbb{R})
\]

\item[\textbf{Axiom M2 (Multiplicative Commutativity).}]
\[
\forall x \forall y \, (x\cdot y = y\cdot x)
\]

\item[\textbf{Axiom M3 (Multiplicative Associativity).}]
\[
\forall x \forall y \forall z \, ((x\cdot y)\cdot z = x\cdot (y\cdot z))
\]

\item[\textbf{Axiom M4 (Multiplicative Identity).}]
\[
\exists 1 \, (1 \neq 0 \wedge \forall x \, (x\cdot 1 = x))
\]

\item[\textbf{Axiom M5 (Multiplicative Inverse).}]
\[
\forall x \, (x \neq 0 \rightarrow \exists y \, (x\cdot y = 1))
\]

\end{description}

\paragraph{Distributive Axiom}

\begin{description}

\item[\textbf{Axiom D (Distributivity).}]
\[
\forall x \forall y \forall z \, (x\cdot (y+z) = x\cdot y + x\cdot z)
\]

\end{description}

% =========================================================
% Order Axioms
% =========================================================

\paragraph{Linear Order Axioms}

\begin{description}

\item[\textbf{O1 (Reflexivity).}]
\[
\forall x\in\mathbb{R},\quad x \le x
\]

\item[\textbf{O2 (Antisymmetry).}]
\[
\forall x,y\in\mathbb{R},\quad
(x \le y \wedge y \le x) \rightarrow x = y
\]

\item[\textbf{O3 (Transitivity).}]
\[
\forall x,y,z\in\mathbb{R},\quad
(x \le y \wedge y \le z) \rightarrow x \le z
\]

\item[\textbf{O4 (Totality / Comparability).}]
\[
\forall x,y\in\mathbb{R},\quad
x \le y \ \vee\ y \le x
\]

\end{description}

The strict order is defined by:
\[
x < y
\quad\text{iff}\quad
x \le y \text{ and } x \neq y.
\]

\paragraph{Compatibility with Field Operations}

\begin{description}

\item[\textbf{O5 (Additive Monotonicity).}]
\[
\forall x,y,z\in\mathbb{R},\quad
x \le y \rightarrow x+z \le y+z
\]

\item[\textbf{O6 (Multiplicative Monotonicity for Nonnegative Factors).}]
\[
\forall x,y,z\in\mathbb{R},\quad
(x \le y \wedge 0 \le z) \rightarrow xz \le yz
\]

\item[\textbf{O7 (Positivity of the Unit).}]
\[
0 < 1
\]

\end{description}

\begin{remark}
Axiom O6 implies that multiplication by a negative number reverses
inequalities; this will later be proved as a theorem.
\end{remark}

% ---------------------------------------------------------
\subsubsection{Consequences}

The logical implication of this entire section is:

\[
\text{Field Axioms}
\Rightarrow
\text{Algebraic Structure}
\]

\[
\text{Field + Order Axioms}
\Rightarrow
\text{Ordered Field}
\]

To uniquely characterize $\mathbb{R}$ among ordered fields,
one must additionally assume:

\[
\textbf{Completeness}.
\]

\begin{remark}[Logical Structure]
\[
\text{Field}
\Rightarrow
\text{Ordered Field}
\Rightarrow
\text{Complete Ordered Field}.
\]

The real numbers $\mathbb{R}$ are the unique (up to isomorphism)
complete ordered field.
\end{remark}
