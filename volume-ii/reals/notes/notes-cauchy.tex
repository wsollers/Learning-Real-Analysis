% =========================================================
% Cauchy Sequences
% File: notes-cauchy-sequences.tex
% =========================================================

\subsection{Cauchy Sequences}

% (Original local heading preserved)
\subsubsection{Cauchy Sequences}

% ---------------------------------------------------------
\subsubsection{Basic Definitions}

\begin{definition}[Cauchy sequence]
A sequence $(a_n)$ in $\mathbb{R}$ is Cauchy if
\[
\forall \varepsilon > 0 \;\exists N \in \mathbb{N}
\;\forall m,n \in \mathbb{N},
\quad
(m \ge N \land n \ge N)
\Rightarrow
|a_n - a_m| < \varepsilon.
\]
\end{definition}

\begin{remark}
    A Cauchy sequence is one whose terms become arbitrarily close to each other —
without mentioning a limit.
\end{remark}

% ---------------------------------------------------------
\subsubsection{Main Theorems}

% NOTE: The following theorem + proof appears twice in the source extract.
% Both instances are preserved verbatim below to ensure no content is lost.

\begin{theorem}[Every convergent sequence is Cauchy]
If $(a_n)$ converges in $\mathbb{R}$, then $(a_n)$ is a Cauchy sequence.
\end{theorem}

\begin{proof}
Assume $a_n \to L$. Let $\varepsilon>0$ be arbitrary.
By convergence, there exists $N\in\mathbb{N}$ such that
\[
n\ge N \implies |a_n-L|<\frac{\varepsilon}{2}.
\]
Now let $m,n\ge N$. Then, by the triangle inequality,
\[
|a_n-a_m|
= |(a_n-L)-(a_m-L)|
\le |a_n-L|+|a_m-L|
< \frac{\varepsilon}{2}+\frac{\varepsilon}{2}
= \varepsilon.
\]
Hence $(a_n)$ is Cauchy.
\end{proof}

\begin{corollary}[Every convergent series has Cauchy partial sums]
If $\sum_{n=1}^\infty a_n$ converges and $s_n:=\sum_{k=1}^n a_k$ denotes the sequence
of partial sums, then $(s_n)$ is a Cauchy sequence. Equivalently,
\[
\forall \varepsilon>0\ \exists N\ \text{s.t.}\ m>n\ge N \implies
\left|\sum_{k=n+1}^{m} a_k\right| < \varepsilon.
\]
\end{corollary}

\begin{proof}
If $\sum_{n=1}^\infty a_n$ converges, then by definition its partial sums satisfy
$s_n \to S$ for some $S\in\mathbb{R}$. By the theorem above, $(s_n)$ is Cauchy.
For $m>n$ we have
\[
s_m-s_n=\sum_{k=n+1}^{m} a_k,
\]
so the Cauchy condition for $(s_n)$ is exactly the displayed tail estimate.
\end{proof}

\begin{theorem}[Every convergent sequence is Cauchy]
If $(a_n)$ converges in $\mathbb{R}$, then $(a_n)$ is a Cauchy sequence.
\end{theorem}

\begin{proof}
Assume $a_n \to L$. Let $\varepsilon>0$ be arbitrary.
By convergence, there exists $N\in\mathbb{N}$ such that
\[
n\ge N \implies |a_n-L|<\frac{\varepsilon}{2}.
\]
Now let $m,n\ge N$. Then, by the triangle inequality,
\[
|a_n-a_m|
= |(a_n-L)-(a_m-L)|
\le |a_n-L|+|a_m-L|
< \frac{\varepsilon}{2}+\frac{\varepsilon}{2}
= \varepsilon.
\]
Hence $(a_n)$ is Cauchy.
\end{proof}

\begin{corollary}[Every convergent series has Cauchy partial sums]
If $\sum_{n=1}^\infty a_n$ converges and $s_n:=\sum_{k=1}^n a_k$ denotes the sequence
of partial sums, then $(s_n)$ is a Cauchy sequence. Equivalently,
\[
\forall \varepsilon>0\ \exists N\ \text{s.t.}\ m>n\ge N \implies
\left|\sum_{k=n+1}^{m} a_k\right| < \varepsilon.
\]
\end{corollary}

\begin{proof}
If $\sum_{n=1}^\infty a_n$ converges, then by definition its partial sums satisfy
$s_n \to S$ for some $S\in\mathbb{R}$. By the theorem above, $(s_n)$ is Cauchy.
For $m>n$ we have
\[
s_m-s_n=\sum_{k=n+1}^{m} a_k,
\]
so the Cauchy condition for $(s_n)$ is exactly the displayed tail estimate.
\end{proof}

\begin{theorem}[Cauchy sequences are bounded]
If $(a_n)$ is a Cauchy sequence in $\mathbb{R}$, then $(a_n)$ is bounded.
\end{theorem}

\begin{proof}
Assume $(a_n)$ is Cauchy. Take $\varepsilon = 1$. Then there exists $N\in\mathbb{N}$
such that
\[
m,n \ge N \implies |a_n-a_m|<1.
\]
In particular, fixing $m=N$ gives: for all $n\ge N$,
\[
|a_n-a_N|<1 \quad\Longrightarrow\quad |a_n|\le |a_N|+1
\]
by the triangle inequality.

Now define
\[
M := \max\bigl\{|a_1|,|a_2|,\dots,|a_{N-1}|,\ |a_N|+1\bigr\}.
\]
Then for $n<N$ we have $|a_n|\le M$ by construction, and for $n\ge N$ we have
$|a_n|\le |a_N|+1 \le M$. Hence $|a_n|\le M$ for all $n\in\mathbb{N}$, so $(a_n)$ is bounded.
\end{proof}

% ---------------------------------------------------------
\begin{theorem}[Bolzano--Weierstrass Theorem]
Every bounded sequence in $\mathbb{R}$ has a convergent subsequence.
\end{theorem}

\begin{proof}
Let $(a_n)$ be bounded. Then there exists $M>0$ such that
\[
|a_n| \le M \quad \text{for all } n\in\mathbb{N}.
\]
Thus $(a_n)$ is contained in the closed interval $[-M,M]$.

We first show that every sequence has a monotone subsequence.

\medskip

\emph{Step 1: Monotone Subsequence Lemma.}
Every sequence of real numbers has a monotone subsequence.

Indeed, define
\[
A := \{ m\in\mathbb{N} : a_m \ge a_n \text{ for all } n>m \}.
\]

There are two cases:

\textbf{Case 1:} $A$ is infinite.  
Then we may enumerate its elements increasingly,
$m_1 < m_2 < \cdots$.  
By definition of $A$, the subsequence $(a_{m_k})$
is decreasing.

\textbf{Case 2:} $A$ is finite.  
Then beyond some index $N$, every term fails to be a
terminal maximum.  
Hence for each $n\ge N$ there exists $m>n$ with
$a_m > a_n$.  
By induction we construct indices
$n_1<n_2<n_3<\cdots$ such that
\[
a_{n_1} < a_{n_2} < a_{n_3} < \cdots.
\]
Thus we obtain a strictly increasing subsequence.

\medskip

In either case, $(a_n)$ admits a monotone subsequence.

\medskip

\emph{Step 2: Apply Monotone Convergence.}

Since $(a_n)$ is bounded, any monotone subsequence is
bounded and monotone.  
By the Monotone Convergence Theorem,
every bounded monotone sequence converges.

Therefore the monotone subsequence constructed above
converges.

Hence $(a_n)$ has a convergent subsequence.
\end{proof}

\begin{remark}[Logical Position]
Bolzano--Weierstrass is the bridge between boundedness
and convergence.  
A bounded sequence need not converge, but it cannot
avoid convergence entirely: some subsequence must
converge.
\end{remark}

\begin{remark}[Dependence on Completeness]
The proof relies on the Monotone Convergence Theorem,
which in turn depends on the Least Upper Bound Property
of $\mathbb{R}$.  
Thus Bolzano--Weierstrass is a manifestation of
completeness.
\end{remark}

\begin{corollary}[Sequential Compactness of Closed Intervals]
Every sequence in a closed bounded interval
$[a,b]\subset\mathbb{R}$ has a convergent subsequence
whose limit lies in $[a,b]$.
\end{corollary}

\begin{proof}
If $(x_n)\subset [a,b]$, then it is bounded.
By Bolzano--Weierstrass it has a convergent subsequence.
Since $[a,b]$ is closed, its limit lies in $[a,b]$.
\end{proof}

\begin{remark}[Structural Consequence]
Bolzano--Weierstrass supplies the critical step in the
Cauchy Criterion:
\[
\text{Cauchy}
\Rightarrow
\text{bounded}
\Rightarrow
\text{convergent subsequence}
\Rightarrow
\text{full convergence}.
\]
It is therefore one of the equivalent formulations of
completeness.
\end{remark}





\begin{theorem}[Cauchy Criterion in $\mathbb{R}$]
A sequence $(a_n)$ of real numbers converges if and only if it is Cauchy.
\end{theorem}

\begin{proof}
($\Rightarrow$) Assume $a_n \to L$. Let $\varepsilon>0$ be arbitrary.
Choose $N\in\mathbb{N}$ such that
\[
n\ge N \implies |a_n-L|<\frac{\varepsilon}{2}.
\]
Then for any $m,n\ge N$, the triangle inequality gives
\[
|a_n-a_m|
\le |a_n-L|+|a_m-L|
<\frac{\varepsilon}{2}+\frac{\varepsilon}{2}
=\varepsilon.
\]
Hence $(a_n)$ is Cauchy.

($\Leftarrow$) Assume $(a_n)$ is Cauchy. By the theorem \emph{Cauchy sequences are bounded},
$(a_n)$ is bounded. By the Bolzano--Weierstrass theorem, there exists a convergent
subsequence $(a_{n_k})$ with
\[
a_{n_k}\to L
\]
for some $L\in\mathbb{R}$.

We claim $a_n\to L$. Let $\varepsilon>0$ be arbitrary. Since $(a_n)$ is Cauchy,
there exists $N\in\mathbb{N}$ such that
\[
m,n\ge N \implies |a_n-a_m|<\frac{\varepsilon}{2}.
\]
Because $a_{n_k}\to L$, there exists $K\in\mathbb{N}$ such that
\[
k\ge K \implies |a_{n_k}-L|<\frac{\varepsilon}{2}.
\]
Pick $k\ge K$ so that also $n_k\ge N$ (possible since $n_k\to\infty$).
Then for any $n\ge N$ we have, using the triangle inequality and the Cauchy property,
\[
|a_n-L|
\le |a_n-a_{n_k}| + |a_{n_k}-L|
< \frac{\varepsilon}{2}+\frac{\varepsilon}{2}
=\varepsilon.
\]
Thus $n\ge N \implies |a_n-L|<\varepsilon$, so $a_n\to L$.
\end{proof}

% ---------------------------------------------------------
\subsubsection{Consequences}

The logical implication of this entire section is:

\begin{remark}[Interdependence of the Major Theorems on Sequences]

At first glance, the principal results of this section may appear to be
independent facts:

\begin{itemize}
    \item Convergent $\Rightarrow$ Cauchy
    \item Cauchy $\Rightarrow$ bounded
    \item Bolzano--Weierstrass (bounded $\Rightarrow$ convergent subsequence)
    \item Cauchy Criterion (Cauchy $\Leftrightarrow$ convergent)
\end{itemize}

In reality, these theorems form a tightly interlocking structure whose
foundation is the completeness of $\mathbb{R}$.

\medskip

\textbf{Structural Dependencies.}

\begin{enumerate}
    \item \emph{Convergence $\Rightarrow$ Cauchy.}  
    This direction uses only the triangle inequality.
    It does not rely on completeness.

    \item \emph{Cauchy $\Rightarrow$ bounded.}  
    Once the terms eventually cluster tightly,
    the entire sequence must lie inside some finite interval.

    \item \emph{Bounded $\Rightarrow$ convergent subsequence} (Bolzano--Weierstrass).  
    Boundedness alone does not guarantee convergence,
    but it guarantees partial convergence.

    \item \emph{Cauchy $\Rightarrow$ Convergent} (Cauchy Criterion).  
    This direction synthesizes the previous results:
    \[
    \text{Cauchy}
    \Rightarrow
    \text{bounded}
    \Rightarrow
    \text{convergent subsequence}
    \Rightarrow
    \text{full convergence}.
    \]
\end{enumerate}

\medskip

\textbf{Logical Structure.}

\[
\text{Convergent}
\Rightarrow
\text{Cauchy}
\Rightarrow
\text{Bounded}
\Rightarrow
\text{Convergent subsequence}.
\]

Completeness of $\mathbb{R}$ upgrades the final step:
\[
\text{Cauchy}
\Longleftrightarrow
\text{Convergent}.
\]

Without completeness (for example in $\mathbb{Q}$),
the implication $\text{Cauchy} \Rightarrow \text{Convergent}$ fails.

\medskip
\medskip

\begin{remark}[Direct Consequences of Bolzano--Weierstrass]

The Bolzano--Weierstrass Theorem yields several immediate structural facts:

\begin{enumerate}
    \item \textbf{Bounded nonconvergent sequences oscillate.}  
    If a bounded sequence does not converge, then it must admit
    at least two subsequences converging to different limits.

    \item \textbf{Extremal subsequences exist.}  
    For every bounded sequence $(a_n)$,
    there exist subsequences converging to
    $\limsup a_n$ and to $\liminf a_n$.

    \item \textbf{Sequential compactness of closed intervals.}  
    Every sequence contained in a closed bounded interval
    $[a,b]$ admits a convergent subsequence
    whose limit lies in $[a,b]$.
\end{enumerate}

These facts show that Bolzano--Weierstrass
controls the \emph{long-term structure} of bounded sequences.
\end{remark}

\medskip

\textbf{Extended Logical Chain.}

Combining all major results of this section, we obtain:

\[
\text{Convergent}
\Rightarrow
\text{Cauchy}
\Rightarrow
\text{Bounded}
\Rightarrow
\text{Convergent subsequence}
\]

and, using completeness,

\[
\text{Cauchy}
\Longleftrightarrow
\text{Convergent}.
\]

Thus boundedness alone does not ensure convergence,
but it prevents total divergence.
Some limiting behavior must emerge.

\textbf{Conceptual Summary.}

Convergence is an external statement about approaching a number.  
Cauchy is an internal statement about self-consistency of the sequence.  

Completeness asserts that internal consistency is sufficient:
there are no ``holes'' in the real line.

Thus the Cauchy Criterion is not merely a technical tool —
it is an equivalent formulation of completeness.
\end{remark}

