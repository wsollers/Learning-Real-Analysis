% =========================================================
% The Dedekind Cut Construction of ℝ
% =========================================================

\subsection{Dedekind Cut Construction of $\mathbb{R}$}

\begin{tcolorbox}[colback=gray!6, colframe=gray!40, arc=2pt,
  left=6pt, right=6pt, top=4pt, bottom=4pt,
  title={\small\textbf{Dedekind Cuts — Quick Reference}},
  fonttitle=\small\bfseries]
\small
\begin{tabular}{l l l}
\toprule
\textbf{Concept} & \textbf{Meaning} & \textbf{Detail} \\
\midrule
Dedekind cut  & Partition $(A \mid B)$ of $\mathbb{Q}$ & \hyperref[def:dedekind-cut]{↓ Def} \\
Real number   & An equivalence class of Dedekind cuts & \hyperref[def:real-as-cut]{↓ Def} \\
Completeness  & Every non-empty bounded cut has a sup & \hyperref[thm:dedekind-complete]{↓ Thm} \\
\bottomrule
\end{tabular}
\end{tcolorbox}

\vspace{1em}

\begin{tcolorbox}[colback=propbox, colframe=propborder, arc=2pt,
  left=6pt, right=6pt, top=4pt, bottom=4pt,
  title={\small\textbf{Definition (Dedekind Cut)}},
  fonttitle=\small\bfseries]
\label{def:dedekind-cut}
A \emph{Dedekind cut} is a pair $(A, B)$ with $A \cup B = \mathbb{Q}$,
$A \cap B = \emptyset$, $A \neq \emptyset$, $B \neq \emptyset$, satisfying:
\begin{enumerate}
  \item If $p \in A$ and $q < p$ then $q \in A$ \quad (downward closed),
  \item $A$ has no greatest element.
\end{enumerate}
\end{tcolorbox}

\begin{remark}[English reading]
A Dedekind cut slices the rationals into a lower set $A$ and upper set $B$
so that $A$ contains all rationals ``below'' the cut point and has no maximum.
Each cut corresponds to exactly one real number.
\end{remark}

\begin{tcolorbox}[colback=propbox, colframe=propborder, arc=2pt,
  left=6pt, right=6pt, top=4pt, bottom=4pt,
  title={\small\textbf{Definition (Real Number via Dedekind Cut)}},
  fonttitle=\small\bfseries]
\label{def:real-as-cut}
A \emph{real number} is a Dedekind cut $(A, B)$ of $\mathbb{Q}$.
The set $\mathbb{R}$ is defined as the collection of all Dedekind cuts,
equipped with order: $(A_1, B_1) \leq (A_2, B_2)$ iff $A_1 \subseteq A_2$.
\end{tcolorbox}

\begin{remark}[Status]
\textit{This section is a stub. Full content — arithmetic on cuts, the proof
that $\mathbb{R}$ is a complete ordered field, and the uniqueness theorem —
will be developed in a future revision. Primary source: Rudin,
\textit{Principles of Mathematical Analysis}, Appendix to Chapter~1.}
\end{remark}

\begin{tcolorbox}[colback=thmbox, colframe=thmborder, arc=2pt,
  left=6pt, right=6pt, top=4pt, bottom=4pt,
  title={\small\textbf{Theorem (Dedekind Completeness)}},
  fonttitle=\small\bfseries]
\label{thm:dedekind-complete}
The ordered field $(\mathbb{R}, +, \cdot, \leq)$ constructed from Dedekind cuts
is complete: every non-empty subset of $\mathbb{R}$ that is bounded above
has a least upper bound in $\mathbb{R}$.
\end{tcolorbox}

\begin{remark}[Significance]
This is the point of the entire construction: $\mathbb{Q}$ has ``gaps''
(e.g.\ no rational satisfies $x^2 = 2$), and Dedekind cuts fill exactly
those gaps, yielding a complete ordered field.
\end{remark}
