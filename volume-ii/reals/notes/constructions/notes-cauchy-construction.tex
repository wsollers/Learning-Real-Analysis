% =========================================================
% The Cauchy Sequence Construction of ℝ
% =========================================================

\subsection{Cauchy Sequence Construction of $\mathbb{R}$}

\begin{tcolorbox}[colback=gray!6, colframe=gray!40, arc=2pt,
  left=6pt, right=6pt, top=4pt, bottom=4pt,
  title={\small\textbf{Cauchy Completion — Quick Reference}},
  fonttitle=\small\bfseries]
\small
\begin{tabular}{l l l}
\toprule
\textbf{Concept} & \textbf{Meaning} & \textbf{Detail} \\
\midrule
Cauchy sequence over $\mathbb{Q}$ & Sequence with terms arbitrarily close & \hyperref[def:cauchy-q]{↓ Def} \\
Equivalence of Cauchy seqs        & Same limit behaviour & \hyperref[def:cauchy-equiv]{↓ Def} \\
Real number as equivalence class  & $\mathbb{R} = \mathbb{Q}^{\text{Cauchy}} / {\sim}$ & \hyperref[def:real-as-cauchy]{↓ Def} \\
\bottomrule
\end{tabular}
\end{tcolorbox}

\vspace{1em}

\begin{tcolorbox}[colback=propbox, colframe=propborder, arc=2pt,
  left=6pt, right=6pt, top=4pt, bottom=4pt,
  title={\small\textbf{Definition (Cauchy Sequence over $\mathbb{Q}$)}},
  fonttitle=\small\bfseries]
\label{def:cauchy-q}
A sequence $(a_n)$ of rationals is \emph{Cauchy} if
\[
  \forall \varepsilon \in \mathbb{Q}_{>0},\;
  \exists N \in \mathbb{N},\;
  \forall m, n \geq N,\quad |a_m - a_n| < \varepsilon.
\]
\end{tcolorbox}

\begin{tcolorbox}[colback=propbox, colframe=propborder, arc=2pt,
  left=6pt, right=6pt, top=4pt, bottom=4pt,
  title={\small\textbf{Definition (Equivalence of Cauchy Sequences)}},
  fonttitle=\small\bfseries]
\label{def:cauchy-equiv}
Two Cauchy sequences $(a_n)$ and $(b_n)$ over $\mathbb{Q}$ are
\emph{equivalent}, written $(a_n) \sim (b_n)$, if
\[
  \lim_{n\to\infty}(a_n - b_n) = 0 \quad \text{(in } \mathbb{Q}\text{)}.
\]
\end{tcolorbox}

\begin{tcolorbox}[colback=propbox, colframe=propborder, arc=2pt,
  left=6pt, right=6pt, top=4pt, bottom=4pt,
  title={\small\textbf{Definition (Real Number via Cauchy Sequences)}},
  fonttitle=\small\bfseries]
\label{def:real-as-cauchy}
A \emph{real number} is an equivalence class $[(a_n)]$ of Cauchy sequences
of rationals under $\sim$.
The set $\mathbb{R}$ is defined as the quotient
$\mathbb{R} = \mathcal{C}(\mathbb{Q}) / {\sim}$,
where $\mathcal{C}(\mathbb{Q})$ denotes the set of all Cauchy sequences
of rationals.
\end{tcolorbox}

\begin{remark}[Status]
\textit{This section is a stub. Full content — arithmetic on equivalence classes,
verification of field axioms, proof of completeness, and comparison with the
Dedekind construction — will be developed in a future revision.
Primary source: Tao, \textit{Analysis~I}, Chapters~5--6.}
\end{remark}

\begin{remark}[Comparison with Dedekind cuts]
Both constructions yield the same object up to isomorphism.
The Dedekind approach is more geometric (cutting the line);
the Cauchy approach is more analytic (completing via limits).
Tao develops the Cauchy approach; Rudin the Dedekind approach.
\end{remark}
