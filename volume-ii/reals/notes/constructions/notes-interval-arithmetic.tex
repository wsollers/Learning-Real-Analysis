% =========================================================
% Interval Arithmetic
% =========================================================

\subsection{Interval Arithmetic}

\begin{tcolorbox}[colback=gray!6, colframe=gray!40, arc=2pt,
  left=6pt, right=6pt, top=4pt, bottom=4pt,
  title={\small\textbf{Interval Arithmetic — Quick Reference}},
  fonttitle=\small\bfseries]
\small
\begin{tabular}{l l l}
\toprule
\textbf{Operation} & \textbf{Rule} & \textbf{Detail} \\
\midrule
Addition       & $[a,b] + [c,d] = [a+c,\, b+d]$           & \hyperref[def:interval-add]{↓ Def} \\
Subtraction    & $[a,b] - [c,d] = [a-d,\, b-c]$           & \hyperref[def:interval-sub]{↓ Def} \\
Multiplication & $[a,b] \cdot [c,d]$ (case analysis)      & \hyperref[def:interval-mult]{↓ Def} \\
Containment    & $[a,b] \subseteq [c,d] \iff c\leq a, b\leq d$ & \hyperref[def:interval-contain]{↓ Def} \\
\bottomrule
\end{tabular}
\end{tcolorbox}

\vspace{1em}

\begin{tcolorbox}[colback=propbox, colframe=propborder, arc=2pt,
  left=6pt, right=6pt, top=4pt, bottom=4pt,
  title={\small\textbf{Definition (Interval Addition)}},
  fonttitle=\small\bfseries]
\label{def:interval-add}
For closed intervals $[a,b]$ and $[c,d]$, their \emph{sum} is
\[
  [a,b] + [c,d] \;=\; [a+c,\; b+d].
\]
\end{tcolorbox}

\begin{tcolorbox}[colback=propbox, colframe=propborder, arc=2pt,
  left=6pt, right=6pt, top=4pt, bottom=4pt,
  title={\small\textbf{Definition (Interval Subtraction)}},
  fonttitle=\small\bfseries]
\label{def:interval-sub}
For closed intervals $[a,b]$ and $[c,d]$, their \emph{difference} is
\[
  [a,b] - [c,d] \;=\; [a-d,\; b-c].
\]
\end{tcolorbox}

\begin{tcolorbox}[colback=propbox, colframe=propborder, arc=2pt,
  left=6pt, right=6pt, top=4pt, bottom=4pt,
  title={\small\textbf{Definition (Interval Multiplication)}},
  fonttitle=\small\bfseries]
\label{def:interval-mult}
For closed intervals $[a,b]$ and $[c,d]$, their \emph{product} is
\[
  [a,b] \cdot [c,d] \;=\;
  \bigl[\min(ac, ad, bc, bd),\; \max(ac, ad, bc, bd)\bigr].
\]
\end{tcolorbox}

\begin{tcolorbox}[colback=propbox, colframe=propborder, arc=2pt,
  left=6pt, right=6pt, top=4pt, bottom=4pt,
  title={\small\textbf{Definition (Interval Containment)}},
  fonttitle=\small\bfseries]
\label{def:interval-contain}
$[a,b] \subseteq [c,d]$ if and only if $c \leq a$ and $b \leq d$.
\end{tcolorbox}

\begin{remark}[Motivation]
Interval arithmetic provides a framework for rigorous numerical computation
with guaranteed error bounds. A real number $x \in [a,b]$ means the computed
value may lie anywhere in the interval; operations propagate these bounds.
\end{remark}

\begin{remark}[Status]
\textit{This section is a stub. A full treatment — the sub-distributivity law,
dependency problem, applications to root-finding, and connections to
compactness — will be developed in a future revision.}
\end{remark}
