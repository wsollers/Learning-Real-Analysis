% ======================================
% Capstone Assessment — Propositional Logic
% ======================================

\section{Capstone Assessment: Propositional Logic}

\noindent
\textbf{Purpose.}
This capstone assesses mastery of propositional logic as a formal system.
All problems must be solved using only propositional reasoning:
logical connectives, equivalence, implication, and formal proof techniques.
No quantifiers or set-theoretic arguments are permitted.

\vspace{1em}

\noindent
\textbf{Instructions.}
For each problem, write a complete and rigorous proof.
You may use truth tables, equivalence transformations, or natural deduction,
but your method must be explicit and logically justified.

\vspace{1.5em}

% --------------------------------------------------
\subsection*{Problem 1 — Nontrivial Implication Structure}

Prove that the following implication is logically valid:
\[
\bigl[(p \rightarrow q) \land (r \rightarrow s) \land (p \lor r)\bigr]
\;\rightarrow\;
(q \lor s).
\]

Your proof must make explicit where each assumption is used.

\vspace{1.5em}

% --------------------------------------------------
\subsection*{Problem 2 — Hidden Equivalence}

Prove that the following two formulas are logically equivalent:
\[
(p \rightarrow q) \lor (q \rightarrow p)
\quad\text{and}\quad
(p \leftrightarrow q) \lor (\neg p \land \neg q).
\]

You may not appeal to intuition; your argument must rely on formal equivalence
rules or semantic reasoning.

\vspace{1.5em}

% --------------------------------------------------
\subsection*{Problem 3 — Disguised Contradiction}

Show that the formula
\[
(p \land (p \rightarrow q)) \land \neg q
\]
is a contradiction.

Your proof must explicitly identify the source of inconsistency.

\vspace{1.5em}

% --------------------------------------------------
\subsection*{Problem 4 — Implication Elimination Challenge}

Prove that the formula
\[
(p \rightarrow (q \rightarrow r))
\]
is logically equivalent to
\[
(p \land q) \rightarrow r.
\]

Your proof must use only propositional equivalences
and must not rely on informal reasoning.

\vspace{1.5em}

% --------------------------------------------------
\subsection*{Problem 5 — Semantic Consequence}

Prove that the set of formulas
\[
\{\, p \rightarrow q,\; q \rightarrow r,\; \neg r \,\}
\]
logically implies \(\neg p\).

That is, show that every valuation that makes all three premises true
must also make \(\neg p\) true.

\vspace{1em}

\noindent
\textbf{Completion Criterion.}
You pass the Propositional Logic capstone if all five proofs are:
\begin{itemize}
  \item logically correct,
  \item explicitly justified at each step, and
  \item written without appeal to informal semantic intuition.
\end{itemize}

\noindent
Successful completion certifies readiness to proceed to Predicate Calculus.
