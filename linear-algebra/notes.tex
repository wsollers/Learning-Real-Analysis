% =========================================================
% Linear Algebra — Structural Roadmap
% Driver Text: Sheldon Axler, Linear Algebra Done Right
% File: 100-linear-algebra/linear-algebra.tex
% =========================================================

\subsection{Linear Algebra}

% =========================================================
% Structural Roadmap
% =========================================================
\subsubsection*{Structural Roadmap}

The development of linear algebra in this project follows
the definition--theorem--structure architecture used
throughout the analysis volumes.

The primary driver is \textit{Linear Algebra Done Right}
by Sheldon Axler. The emphasis is on linear maps,
invariant structure, and conceptual clarity
rather than computational techniques.

Each major topic is organized as:
\begin{center}
\textbf{Definitions $\longrightarrow$ Main Theorems $\longrightarrow$ Consequences and Structural Insight}
\end{center}

The global progression is:

\begin{enumerate}
  \item Vector spaces and subspaces
  \item Linear combinations, span, and linear independence
  \item Bases and dimension
  \item Linear maps
  \item Null space and range
  \item Matrix representations of linear maps
  \item Eigenvalues and eigenvectors
  \item Invariant subspaces
  \item Triangularization and diagonalization
  \item Inner product spaces
  \item Orthogonality and projections
  \item Self-adjoint and normal operators
  \item Spectral theorem (finite-dimensional case)
\end{enumerate}

\vspace{1em}

\begin{remark}
This treatment prioritizes structural understanding over
algorithmic manipulation. Determinants are introduced only
after linear maps and eigenvalues have been conceptually
understood.
\end{remark}

\begin{remark}
The central object of study is not matrices,
but linear maps between vector spaces.
Matrices serve only as representations relative to a basis.
\end{remark}

\begin{remark}[Structural Position]
Linear algebra is developed independently of analysis.
Its structural viewpoint will later support operator theory,
Hilbert spaces, and functional analysis.
\end{remark}

% =========================================================
% I. Core Linear Structure
% =========================================================

% =========================================================
% Vector Spaces
% =========================================================

% ---------------------------------------------------------
\subsubsection{Preliminary Definitions}
% ---------------------------------------------------------

\begin{definition}[Abelian group]
A set $G$ with a binary operation $\ast : G \times G \to G$
is an \emph{abelian group} if:

\begin{enumerate}
\item \textbf{Associativity:}
\[
(a \ast b) \ast c = a \ast (b \ast c)
\quad\text{for all } a,b,c \in G.
\]

\item \textbf{Identity element:}
There exists $e \in G$ such that
\[
a \ast e = e \ast a = a
\quad\text{for all } a \in G.
\]

\item \textbf{Inverse element:}
For each $a \in G$ there exists $b \in G$ such that
\[
a \ast b = b \ast a = e.
\]

\item \textbf{Commutativity:}
\[
a \ast b = b \ast a
\quad\text{for all } a,b \in G.
\]
\end{enumerate}
\end{definition}

\begin{remark}
An abelian group is simply a group whose operation is commutative.
\end{remark}

\begin{definition}[Field — structural form]
A \emph{field} is a set $\mathbb{F}$ equipped with two operations
$+$ and $\cdot$ such that:

\begin{itemize}
\item $(\mathbb{F}, +)$ is an abelian group,
\item $(\mathbb{F}\setminus\{0\}, \cdot)$ is an abelian group,
\item Multiplication distributes over addition.
\end{itemize}
\end{definition}

\begin{remark}[Expanded Axioms]
Expanding the abelian group axioms yields the familiar list:

\begin{itemize}
\item Associativity, commutativity, identity, and inverse for addition.
\item Associativity, commutativity, identity, and inverse (for nonzero elements) for multiplication.
\item The distributive law.
\end{itemize}

Thus the usual field axioms are not independent assumptions,
but consequences of the two abelian group structures together with distributivity.
\end{remark}

\begin{remark}[Structural Summary]
A field consists of two compatible abelian group structures:

\begin{center}
\begin{tabular}{l|c|c}
Structure & Additive Group & Multiplicative Structure \\
\hline
Field & Abelian & Abelian (on nonzero elements)
\end{tabular}
\end{center}

Distributivity links the two operations.
\end{remark}

% ---------------------------------------------------------
\subsubsection{The Complex Numbers}

\begin{definition}[Complex Numbers]
The set of \emph{complex numbers}, denoted $\mathbb{C}$, is defined as
\[
\mathbb{C} := \{(a,b) : a,b \in \mathbb{R}\},
\]
with operations defined by
\[
(a,b) + (c,d) := (a+c,\; b+d),
\]
\[
(a,b)\cdot(c,d) := (ac - bd,\; ad + bc).
\]
\end{definition}

\begin{remark}
We identify $(a,b)$ with the symbol
\[
a + bi,
\]
where $i := (0,1)$ satisfies
\[
i^2 = -1.
\]
\end{remark}

\begin{remark}
With these operations, $\mathbb{C}$ is a field containing $\mathbb{R}$
as the subset $\{(a,0): a \in \mathbb{R}\}$.
\end{remark}

\begin{remark}[Field structure of $\mathbb{C}$]
With the operations defined above, the set $\mathbb{C}$
satisfies the field axioms.

That is:
\begin{itemize}
\item $(\mathbb{C}, +)$ is an abelian group,
\item $(\mathbb{C}\setminus\{0\}, \cdot)$ is an abelian group,
\item multiplication distributes over addition.
\end{itemize}

Hence $\mathbb{C}$ is a field.
\end{remark}

% ---------------------------------------------------------
\subsubsection{The Space $\mathbb{C}^n$}

\begin{definition}[$\mathbb{C}^n$]
Let $n \in \mathbb{N}$.
The set
\[
\mathbb{C}^n := \{(z_1, \dots, z_n) : z_i \in \mathbb{C}\}
\]
is the set of ordered $n$-tuples of complex numbers.

Addition and scalar multiplication are defined componentwise:
\[
(z_1,\dots,z_n) + (w_1,\dots,w_n)
:=
(z_1+w_1,\dots,z_n+w_n),
\]
\[
\lambda (z_1,\dots,z_n)
:=
(\lambda z_1,\dots,\lambda z_n),
\quad
\lambda \in \mathbb{C}.
\]
\end{definition}

\begin{remark}
With these operations, $\mathbb{C}^n$ is a vector space over the field $\mathbb{C}$.
\end{remark}

% ---------------------------------------------------------
\subsubsection{Basic Definitions}

\begin{definition}[Vector Space — informal description]
A vector space is a set equipped with addition and scalar multiplication
that behave like familiar vector arithmetic.
\end{definition}

\begin{remark}
The informal description suggests structure, but precision requires axioms.
\end{remark}

\begin{definition}[Vector Space]
Let $\mathbb{F}$ be a field.
A \textbf{vector space} over $\mathbb{F}$ is a triple $(V, +, \cdot)$ where $V$ is a set,
$+: V \times V \to V$ is a binary operation called \textbf{vector addition}, and
$\cdot : \mathbb{F} \times V \to V$ is a binary operation called \textbf{scalar multiplication}, such that:
\begin{itemize}
    \item $(V, +)$ is an abelian group,
    \item $\alpha (u + v) = \alpha u + \alpha v$ for all $\alpha \in \mathbb{F}$ and $u, v \in V$,
    \item $(\alpha + \beta) v = \alpha v + \beta v$ for all $\alpha, \beta \in \mathbb{F}$ and $v \in V$,
    \item $(\alpha \beta) v = \alpha (\beta v)$ for all $\alpha, \beta \in \mathbb{F}$ and $v \in V$,
    \item $1_{\mathbb{F}} v = v$ for all $v \in V$.
\end{itemize}
The elements of $V$ are called \textbf{vectors} and the elements of $\mathbb{F}$ are called \textbf{scalars}.
\end{definition}

\begin{remark}[Expanded Axioms]
Expanding the abelian group structure of $(V,+)$ yields the full list of vector space axioms:

\begin{itemize}
\item \textbf{Associativity of addition:} $(u + v) + w = u + (v + w)$ for all $u,v,w \in V$.
\item \textbf{Commutativity of addition:} $u + v = v + u$ for all $u,v \in V$.
\item \textbf{Additive identity:} There exists $\mathbf{0} \in V$ such that $v + \mathbf{0} = v$ for all $v \in V$.
\item \textbf{Additive inverse:} For each $v \in V$ there exists $-v \in V$ such that $v + (-v) = \mathbf{0}$.
\item \textbf{Distributivity over vector addition:} $\alpha(u + v) = \alpha u + \alpha v$ for all $\alpha \in \mathbb{F}$, $u,v \in V$.
\item \textbf{Distributivity over scalar addition:} $(\alpha + \beta)v = \alpha v + \beta v$ for all $\alpha,\beta \in \mathbb{F}$, $v \in V$.
\item \textbf{Compatibility:} $(\alpha\beta)v = \alpha(\beta v)$ for all $\alpha,\beta \in \mathbb{F}$, $v \in V$.
\item \textbf{Unit law:} $1_{\mathbb{F}} v = v$ for all $v \in V$.
\end{itemize}

Thus the structural definition compresses eight axioms into one abelian group condition
plus four scalar multiplication laws.
\end{remark}

\begin{remark}[Structural Summary]
A vector space combines an abelian group with a compatible scalar action:

\begin{center}
\begin{tabular}{l|c|c}
Structure & Additive Group & Scalar Action \\
\hline
Vector Space & Abelian & $\mathbb{F}$-linear
\end{tabular}
\end{center}

The scalar multiplication axioms express compatibility between the field $\mathbb{F}$
and the group $(V,+)$.
\end{remark}

\begin{remark}
With the componentwise operations defined above, $\mathbb{F}^n$ is a vector space
over $\mathbb{F}$ for any field $\mathbb{F}$ and any $n \in \mathbb{N}$.
\end{remark}

% ---------------------------------------------------------
\subsubsection{Tuples and Lists}

\begin{definition}[$n$-tuple]
Let $n \in \mathbb{N}$ and let $X$ be a set.
An \emph{$n$-tuple} of elements of $X$ is a function
\[
f : \{1,2,\dots,n\} \to X.
\]
We write such a function as
\[
(x_1,\dots,x_n),
\]
where $x_i := f(i)$ for each $i$.
\end{definition}

\begin{definition}[Finite list]
Let $X$ be a set.
A \emph{finite list} of elements of $X$ is a function
\[
f : \{1,2,\dots,n\} \to X
\]
for some $n \in \mathbb{N}$.
\end{definition}

\begin{definition}[Set]
A \emph{set} is a collection of distinct elements
with no inherent ordering.
\end{definition}

\begin{example}
Consider the elements $1$ and $2$.

The set
\[
\{1,2,2\}
\]
is equal to
\[
\{1,2\},
\]
because sets do not record duplicates.

However, the list
\[
(1,2,2)
\]
is different from
\[
(1,2),
\]
because lists record both order and repetition.
\end{example}

\begin{definition}[Coordinate]
Let $\mathbb{F}$ be a field and let
\[
v = (v_1,\dots,v_n) \in \mathbb{F}^n.
\]
The element $v_i$ is called the \emph{$i$th coordinate} of $v$.
\end{definition}

% ---------------------------------------------------------
\subsubsection{Structural Interpretation of $\mathbb{F}^n$}

\begin{definition}[The space $\mathbb{F}^n$ as a function space]
Let $\mathbb{F}$ be a field and let $n \in \mathbb{N}$.
Define
\[
\mathbb{F}^n := \{ f : \{1,\dots,n\} \to \mathbb{F} \}.
\]
\end{definition}

\begin{remark}
An element $v \in \mathbb{F}^n$ is therefore a function
\[
v : \{1,\dots,n\} \to \mathbb{F}.
\]
We write
\[
v = (v_1,\dots,v_n),
\quad \text{where } v_i := v(i).
\]
Thus an ordered $n$-tuple is simply a function from an index set.
\end{remark}

\begin{definition}[Operations in $\mathbb{F}^n$]
For $v,w \in \mathbb{F}^n$ and $\alpha \in \mathbb{F}$, define:
\[
(v+w)(i) := v(i) + w(i),
\]
\[
(\alpha v)(i) := \alpha \, v(i),
\]
for each $i \in \{1,\dots,n\}$.
\end{definition}

\begin{remark}[Pointwise structure]
Vector addition and scalar multiplication in $\mathbb{F}^n$
are defined \emph{pointwise}.
All vector space axioms follow from the corresponding field axioms
applied coordinatewise.
\end{remark}

\begin{remark}[Structural Insight]
When Axler writes
\[
v = (v_1,\dots,v_n),
\]
he is implicitly identifying $v$ with a function
\[
v : \{1,\dots,n\} \to \mathbb{F}.
\]
Thus
\[
\mathbb{F}^n = \{ f : \{1,\dots,n\} \to \mathbb{F} \}.
\]
This function viewpoint makes the structure of $\mathbb{F}^n$
completely rigorous and conceptually clean.
\end{remark}

% ---------------------------------------------------------
\subsubsection{The Function-Space Viewpoint}

\begin{remark}
Our next example of a vector space involves a set of functions.
Just as $\mathbb{F}^n$ consists of functions from a finite index set into $\mathbb{F}$,
we may replace that finite index set with any set $S$ whatsoever.
The resulting space $\mathbb{F}^S$ is again a vector space,
and this construction subsumes $\mathbb{F}^n$ as a special case.
\end{remark}

\begin{definition}[Set Exponentiation]
Let $X$ and $Y$ be sets.
Define
\[
Y^X := \{ f : X \to Y \},
\]
the set of all functions from $X$ to $Y$.
\end{definition}

\begin{remark}
If $|X| = n$ and $|Y| = m$ are finite, then
\[
|Y^X| = m^n.
\]
Thus the notation $Y^X$ agrees with the rules of cardinal arithmetic.
\end{remark}

\begin{definition}[$\mathbb{F}^S$ — the function space over a set]
Let $\mathbb{F}$ be a field and let $S$ be a set.
Define
\[
\mathbb{F}^S := \{ f : S \to \mathbb{F} \},
\]
the set of all functions from $S$ to $\mathbb{F}$.

For $f, g \in \mathbb{F}^S$, define \textbf{addition} by
\[
(f + g)(x) := f(x) + g(x)
\quad \text{for all } x \in S.
\]

For $\lambda \in \mathbb{F}$ and $f \in \mathbb{F}^S$, define \textbf{scalar multiplication} by
\[
(\lambda f)(x) := \lambda f(x)
\quad \text{for all } x \in S.
\]
\end{definition}

\begin{example}[$\mathbb{F}^S$ is a vector space]
Let $\mathbb{F}$ be a field and let $S$ be a set.
With the addition and scalar multiplication defined above,
$\mathbb{F}^S$ is a vector space over $\mathbb{F}$.

The zero vector is the function $\mathbf{0} : S \to \mathbb{F}$ defined by
$\mathbf{0}(x) = 0$ for all $x \in S$,
and the additive inverse of $f$ is the function $-f$ defined by
$(-f)(x) = -f(x)$ for all $x \in S$.
All vector space axioms follow from the corresponding field axioms applied pointwise.
\end{example}

\begin{definition}[Coordinate space as a function space]
Let $\mathbb{F}$ be a field and let
\[
[n] := \{1,\dots,n\}.
\]
Then
\[
\mathbb{F}^n \;=\; \mathbb{F}^{[n]}
\;=\;
\{ f : [n] \to \mathbb{F} \}.
\]
\end{definition}

\begin{remark}
An element $v \in \mathbb{F}^n$ is therefore a function
\[
v : [n] \to \mathbb{F}.
\]
The ordered tuple notation
\[
v = (v_1,\dots,v_n)
\]
means precisely that
\[
v(i) = v_i.
\]
Thus an $n$-tuple is simply a function from a finite index set.
\end{remark}

\begin{definition}[Pointwise operations]
Let $v,w \in \mathbb{F}^n$ and $\alpha \in \mathbb{F}$.
Define
\[
(v+w)(i) := v(i) + w(i),
\qquad
(\alpha v)(i) := \alpha\, v(i),
\]
for each $i \in [n]$.
\end{definition}

\begin{remark}
Vector addition and scalar multiplication in $\mathbb{F}^n$
are defined \emph{pointwise}.
The vector space axioms follow from the field axioms
applied at each index $i$.
\end{remark}

\begin{remark}[Structural Insight]
The notation
\[
\mathbb{F}^n = \mathbb{F}^{[n]}
\]
reveals that finite-dimensional coordinate spaces
are function spaces on finite index sets.

More generally, for any set $S$,
\[
\mathbb{F}^S := \{ f : S \to \mathbb{F} \}
\]
is a vector space under pointwise operations.

Thus finite-dimensional linear algebra is the special case
of function spaces where the index set is finite.
\end{remark}

\begin{remark}[Conceptual Unification]
The passage
\[
\mathbb{F}^n
\quad\longrightarrow\quad
\mathbb{F}^{\mathbb{N}}
\quad\longrightarrow\quad
\mathbb{F}^S
\]
shows that:
\begin{itemize}
\item Finite tuples are functions on finite sets.
\item Sequences are functions on $\mathbb{N}$.
\item General function spaces are vector spaces indexed by arbitrary sets.
\end{itemize}

This viewpoint unifies finite-dimensional linear algebra
with infinite-dimensional vector spaces and
functional analysis.
\end{remark}

\begin{definition}[Zero vector]
Let $V$ be a vector space over a field $\mathbb{F}$.
The \emph{zero vector} of $V$, denoted $\mathbf{0}$, is the additive identity
of the abelian group $(V,+)$; that is, the unique element $\mathbf{0} \in V$
such that
\[
v + \mathbf{0} = v
\quad \text{for all } v \in V.
\]
\end{definition}

\begin{remark}[Zero vector in $\mathbb{F}^S$]
In the function space $\mathbb{F}^S$, the zero vector is the function
\[
\mathbf{0} : S \to \mathbb{F}
\quad \text{defined by} \quad
\mathbf{0}(x) = 0
\text{ for all } x \in S.
\]
\end{remark}

% ---------------------------------------------------------
\subsubsection{The Space $\mathbb{F}^{\infty}$}

\begin{definition}[$\mathbb{F}^{\infty}$ — the space of sequences]
Let $\mathbb{F}$ be a field.
Define
\[
\mathbb{F}^{\infty}
:=
\{ f : \mathbb{N} \to \mathbb{F} \}.
\]
Equivalently,
\[
\mathbb{F}^{\infty}
=
\{ (x_1,x_2,\dots) : x_k \in \mathbb{F} \text{ for all } k \in \mathbb{N} \}.
\]
An element of $\mathbb{F}^{\infty}$ is called a \emph{sequence} of elements of $\mathbb{F}$.
\end{definition}

\begin{definition}[Pointwise operations on $\mathbb{F}^{\infty}$]
Let $x=(x_1,x_2,\dots)$ and $y=(y_1,y_2,\dots)$ be elements of $\mathbb{F}^{\infty}$,
and let $\lambda \in \mathbb{F}$.

Define addition by
\[
x+y := (x_1+y_1,\; x_2+y_2,\; \dots),
\]
and define scalar multiplication by
\[
\lambda x := (\lambda x_1,\; \lambda x_2,\; \dots).
\]
Equivalently, in function notation:
\[
(x+y)(k) := x(k)+y(k),
\qquad
(\lambda x)(k) := \lambda x(k),
\quad k \in \mathbb{N}.
\]
\end{definition}

\begin{remark}
With the pointwise operations defined above,
$\mathbb{F}^{\infty}$ is a vector space over $\mathbb{F}$.
All vector space axioms follow from the corresponding field axioms,
applied coordinatewise.
\end{remark}

\begin{remark}[Structural Perspective]
The space $\mathbb{F}^{\infty}$ is simply the function space
\[
\mathbb{F}^{\mathbb{N}},
\]
the set of all $\mathbb{F}$-valued functions on $\mathbb{N}$.
Thus:
\[
\mathbb{F}^n = \mathbb{F}^{\{1,\dots,n\}}
\quad\text{and}\quad
\mathbb{F}^{\infty} = \mathbb{F}^{\mathbb{N}}
\]
are instances of the same construction:
vector spaces of functions indexed by a set.
\end{remark}

% ---------------------------------------------------------
\subsubsection{Example: $\mathbb{F}^S$ is a Vector Space}

\begin{example}[$\mathbb{F}^S$ is a vector space]
Let $\mathbb{F}$ be a field and let $S$ be a nonempty set.
With addition and scalar multiplication defined pointwise by
\[
(f+g)(x) := f(x) + g(x),
\qquad
(\lambda f)(x) := \lambda f(x),
\]
for all $f,g \in \mathbb{F}^S$, $\lambda \in \mathbb{F}$, and $x \in S$,
the set $\mathbb{F}^S$ is a vector space over $\mathbb{F}$.
\end{example}

\begin{remark}[Additive identity]
The additive identity of $\mathbb{F}^S$ is the function
\[
\mathbf{0} : S \to \mathbb{F}
\quad \text{defined by} \quad
\mathbf{0}(x) = 0
\text{ for all } x \in S.
\]
\end{remark}

\begin{remark}[Additive inverse]
For $f \in \mathbb{F}^S$, the additive inverse of $f$ is the function
\[
- f : S \to \mathbb{F}
\quad \text{defined by} \quad
(-f)(x) = -\,f(x)
\text{ for all } x \in S.
\]
\end{remark}

\begin{remark}[Why the axioms hold]
All vector space axioms follow from the field axioms,
applied pointwise. For example,
\[
((f+g)+h)(x)
=
f(x)+g(x)+h(x)
=
(f+(g+h))(x),
\]
so addition is associative because addition in $\mathbb{F}$ is associative.
Every other axiom follows similarly.
\end{remark}

























% =========================================================
% II. Linear Maps
% =========================================================

% \input{linear-algebra/notes/notes-linear-independence}
% \input{linear-algebra/notes/notes-basis-dimension}
% \input{linear-algebra/notes/notes-linear-maps}
% \input{linear-algebra/notes/notes-null-range}
% \input{linear-algebra/notes/notes-matrix-representations}

% =========================================================
% III. Eigenstructure
% =========================================================

% \input{linear-algebra/notes/notes-eigenvalues}
% \input{linear-algebra/notes/notes-invariant-subspaces}
% \input{linear-algebra/notes/notes-diagonalization}

% =========================================================
% IV. Inner Product Structure
% =========================================================

% \input{linear-algebra/notes/notes-inner-products}
% \input{linear-algebra/notes/notes-orthogonality}
% \input{linear-algebra/notes/notes-spectral-theorem}

% =========================================================
% Future Sections (To Be Developed)
% =========================================================

% \input{linear-algebra/notes/notes-determinants}
% \input{linear-algebra/notes/notes-jordan-form}
% \input{linear-algebra/notes/notes-advanced-operators}

% =========================================================
% Logical Positioning
% =========================================================

