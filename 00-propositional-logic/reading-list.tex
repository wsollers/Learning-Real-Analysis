% ======================================
% Chapter 1: Propositional Logic
% Reading List (Mastery Plan)
% ======================================



\noindent
\textbf{Goal.} Master propositional logic as a formal system: languages and formation rules,
truth valuations, truth tables, semantic validity and consequence, logical equivalence,
and propositional proof systems (natural deduction). This chapter excludes quantifiers
and predicate logic.

\vspace{1em}

\begin{longtable}{|p{4.2cm}|p{10.8cm}|}
\hline
\textbf{Source} & \textbf{Required Reading for Mastery} \\
\hline
\endfirsthead
\hline
\textbf{Source} & \textbf{Required Reading for Mastery} \\
\hline
\endhead
\hline
\endfoot
\hline
\endlastfoot

% ============================================================
Open Logic Project — \textit{forall x: Calgary} &
\begin{itemize}
  \item Chapter 1 — \textit{Arguments}
  \item Chapter 2 — \textit{The Scope of Logic}
  \item Chapter 3 — \textit{Other Logical Notions}
  \item Chapter 4 — \textit{First Steps to Symbolization}
  \item Chapter 5 — \textit{Connectives}
  \item Chapter 6 — \textit{Sentences of TFL}
  \item Chapter 7 — \textit{Ambiguity}
  \item Chapter 8 — \textit{Use and Mention}
  \item Chapter 9 — \textit{Characteristic Truth Tables}
  \item Chapter 10 — \textit{Truth-Functional Connectives}
  \item Chapter 11 — \textit{Complete Truth Tables}
  \item Chapter 12 — \textit{Semantic Concepts}
  \item Chapter 14 — \textit{Truth Table Shortcuts}
  \item Chapter 15 — \textit{Partial Truth Tables}
  \item Chapter 16 — \textit{The Very Idea of Natural Deduction}
  \item Chapter 17 — \textit{Basic Rules for TFL}
  \item Chapter 18 — \textit{Constructing Proofs}
  \item Chapter 19 — \textit{Additional Rules for TFL}
  \item Chapter 21 — \textit{Derived Rules}
  \item Chapter 22 — \textit{Soundness and Completeness}
\end{itemize}
\\ \hline

% ============================================================
Hurley — \textit{A Concise Introduction to Logic} &
\begin{itemize}
  \item Chapter 6 — \textit{Propositional Logic}
  \begin{itemize}
    \item 6.1 Symbols and Translation
    \item 6.2 Truth Functions
    \item 6.3 Truth Tables for Propositions
    \item 6.4 Truth Tables for Arguments
    \item 6.5 Indirect Truth Tables
    \item 6.6 Argument Forms and Fallacies
  \end{itemize}
  \item Chapter 7 — \textit{Natural Deduction in Propositional Logic}
  \begin{itemize}
    \item 7.1 Rules of Implication I
    \item 7.2 Rules of Implication II
    \item 7.3 Rules of Replacement I
    \item 7.4 Rules of Replacement II
    \item 7.5 Conditional Proof
    \item 7.6 Indirect Proof
    \item 7.7 Proving Logical Truths
  \end{itemize}
\end{itemize}
\\ \hline

% ============================================================
Gerstein — \textit{Introduction to Mathematical Structures and Proofs} &
\begin{itemize}
  \item Section 1.1 — Statements, Propositions, and Theorems
  \item Section 1.2 — Logical Connectives and Truth Tables
  \item Section 1.3 — Conditional Statements
  \item Section 1.5 — Logical Equivalence
  \item Section 1.4 — Proofs: Structures and Strategies (style only)
\end{itemize}
\\ \hline

% ============================================================
Tao — \textit{Analysis I} &
\begin{itemize}
  \item Appendix A.1 — Mathematical Statements
  \item Appendix A.2 — Implication
  \item Appendix A.3 — The Structure of Proofs
\end{itemize}
\\ \hline


% ============================================================
Goldrei — \textit{Propositional and Predicate Calculus} &
\begin{itemize}
  \item Chapter 1 — \textit{Introduction}
  \begin{itemize}
    \item 1.1 Outline of the book
    \item 1.2 Assumed knowledge
  \end{itemize}

  \item Chapter 2 — \textit{Propositions and Truth Assignments}
  \begin{itemize}
    \item 2.1 Introduction
    \item 2.2 The construction of propositional formulas
    \item 2.3 The interpretation of propositional formulas
    \item 2.4 Logical equivalence
    \item 2.5 The expressive power of connectives
    \item 2.6 Logical consequence
  \end{itemize}

  \item Chapter 3 — \textit{Formal Propositional Calculus}
  \begin{itemize}
    \item 3.1 Introduction
    \item 3.2 A formal system for propositional calculus
    \item 3.3 Soundness and completeness
    \item 3.4 Independence of axioms and alternative systems
  \end{itemize}

  \item \textbf{Stop here.} Chapters 4–6 (Predicates, Models, Compactness)
  are deferred to later chapters.
\end{itemize}
\\ \hline


% ============================================================
Stoll — \textit{Set Theory and Logic} &
\begin{itemize}
  \item Propositional calculus
  \item Deduction patterns for propositional logic
\end{itemize}
\\ \hline

% ============================================================
Suppes — \textit{Axiomatic Set Theory} &
\begin{itemize}
  \item Logical preliminaries: propositional schemata and inference
\end{itemize}
\\ \hline

% ============================================================
Halmos — \textit{Naive Set Theory} &
\begin{itemize}
  \item Informal propositional logic embedded in mathematical prose
\end{itemize}
\\ \hline

% ============================================================
Johar — \textit{The Big Book of Real Analysis} &
\begin{itemize}
  \item Chapter 1 — \textit{Logic and Proof Foundations}
  \begin{itemize}
    \item Statements and truth values
    \item Logical connectives ($\neg,\land,\lor,\rightarrow,\leftrightarrow$)
    \item Conditional statements and logical equivalence
    \item Propositional reasoning in mathematical proofs
  \end{itemize}
  \item \textbf{Stop here.} Defer quantifiers, sets, and analysis content to later chapters.
\end{itemize}
\\ \hline


% ============================================================
Chiossi — \textit{Essential Mathematics for Undergraduates} &
\begin{itemize}
  \item Part I — \textit{Round-Up of Elementary Logic}
  \begin{itemize}
    \item 1.1 First-Order Languages (setup only)
    \item 1.2 Propositional Calculus
    \item 1.4 Deduction
    \item 1.5 Soundness and Completeness
  \end{itemize}
  \item \textbf{Stop here.} Skip 1.3 Predicative Calculus (quantifiers) and all later parts for this chapter.
\end{itemize}
\\ \hline

% ============================================================
Rosen — \textit{Discrete Mathematics and Its Applications} &
\begin{itemize}
  \item Chapter 1 — \textit{The Foundations}
  \begin{itemize}
    \item 1.1 Propositions and Logical Connectives
    \item 1.2 Applications of Propositional Logic
    \item 1.3 Propositional Equivalences
    \item 1.4 Predicates and Quantifiers (overview only)
    \item 1.5 Nested Quantifiers (skip formal semantics)
    \item 1.6 Rules of Inference
  \end{itemize}
  \item \textbf{Emphasis:}
  \begin{itemize}
    \item Truth tables and equivalence manipulation
    \item Translation between English and symbolic logic
    \item Concrete inference patterns
  \end{itemize}
  \item \textbf{Stop here.} Predicate logic formalism deferred to Chapter 2.
\end{itemize}
\\ \hline

% ============================================================
Bjørndahl — \textit{An Introduction to Classical and Modal Logics} &
\begin{itemize}
  \item Chapter 1 — \textit{Classical Logic}
  \begin{itemize}
    \item 1.1 Propositions and truth values
    \item 1.2 Truth-functional connectives
    \item 1.3 Truth tables and semantic consequence
    \item 1.4 Logical equivalence
  \end{itemize}

  \item Chapter 2 — \textit{Proof Systems for Propositional Logic}
  \begin{itemize}
    \item Natural deduction for propositional logic
    \item Soundness and completeness (conceptual level)
  \end{itemize}

  \item \textbf{Stop here.}  
  Defer modal operators ($\Box,\Diamond$), possible worlds semantics,
  and all non-classical logics to later chapters.
\end{itemize}
\\ \hline


\end{longtable}

\vspace{1em}

\noindent
\textbf{Mastery Criteria.}
You have mastered propositional logic when you can:
\begin{itemize}
  \item Define the syntax of a propositional language.
  \item Define semantics using truth valuations.
  \item Distinguish truth, validity, and semantic consequence.
  \item Prove logical equivalences formally.
  \item Construct propositional proofs using natural deduction.
  \item Translate complex English arguments into symbolic logic.
\end{itemize}

\noindent
\textbf{Boundary.} Predicate logic and quantifiers begin in Chapter 2.
