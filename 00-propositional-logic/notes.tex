% =========================================================
% Propositional Logic — Reorganized Notes
% =========================================================



% ---------------------------------------------------------
\subsection{Syntax of Propositional Logic}
% ---------------------------------------------------------

\subsubsection{Propositional Variables}

\begin{definition}[Propositional Variable]
A \emph{propositional variable} (or \emph{atomic proposition}) is a symbol that
represents a statement that can be either true or false.

The set of propositional variables is denoted
\[
\mathsf{Prop} = \{P_1, P_2, P_3, \dots\}
\]
or, informally, by letters such as $P, Q, R, S, \dots$.
\end{definition}

\begin{remark}
Propositional variables are the atomic building blocks of propositional logic.
They have no internal structure and are assigned truth values directly by a
truth assignment.
\end{remark}

\subsubsection{Logical Connectives}

\begin{definition}[Logical Connectives]
The \emph{logical connectives} of propositional logic are symbols used to build
compound formulas from simpler ones. The standard connectives are:

\begin{center}
\renewcommand{\arraystretch}{1.3}
\begin{tabular}{|c|c|c|c|}
\hline
\textbf{Symbol} & \textbf{Name} & \textbf{Arity} & \textbf{Reading} \\
\hline
$\neg$ & Negation & Unary & ``not'' \\
$\wedge$ & Conjunction & Binary & ``and'' \\
$\vee$ & Disjunction & Binary & ``or'' \\
$\rightarrow$ & Conditional (Implication) & Binary & ``if \dots\ then'' \\
$\leftrightarrow$ & Biconditional & Binary & ``if and only if'' \\
\hline
\end{tabular}
\end{center}
\end{definition}

\begin{remark}
Different authors use different symbols. Common alternatives include:
$\sim$ or $!$ for negation, $\&$ or $\cdot$ for conjunction, $|$ or $+$ for
disjunction, and $\supset$ or $\Rightarrow$ for the conditional.
\end{remark}

\subsubsection{Well-Formed Formulas}

\begin{definition}[Well-Formed Formula]
Let $\mathcal{L}$ be a propositional language with propositional variables
$\mathsf{Prop}$ and logical connectives $\neg, \wedge, \vee, \rightarrow,
\leftrightarrow$.

The set $\mathsf{WFF}_{\mathcal{L}}$ of \emph{well-formed formulas} (wffs) of
$\mathcal{L}$ is defined recursively as follows:

\begin{enumerate}
  \item \textbf{Atomic formulas.}
  Every propositional variable $P \in \mathsf{Prop}$ is a well-formed formula.

  \item \textbf{Negation.}
  If $\varphi$ is a well-formed formula, then $\neg \varphi$ is a well-formed
  formula.

  \item \textbf{Binary connectives.}
  If $\varphi$ and $\psi$ are well-formed formulas, then each of the following
  is a well-formed formula:
  \[
  (\varphi \wedge \psi), \quad
  (\varphi \vee \psi), \quad
  (\varphi \rightarrow \psi), \quad
  (\varphi \leftrightarrow \psi).
  \]

  \item \textbf{Closure.}
  No expression is a well-formed formula unless it can be obtained by finitely
  many applications of rules (1)--(3).
\end{enumerate}
\end{definition}

\begin{definition}[Atomic and Molecular Formulas]
A formula is \emph{atomic} if it is a propositional variable.

A formula is \emph{molecular} (or \emph{compound}) if it is not atomic; that is,
if it is constructed using at least one logical connective.
\end{definition}

\begin{remark}
The definition of well-formed formulas is recursive (or inductive). As a
consequence, proofs about well-formed formulas typically proceed by
\emph{structural induction} on the formation of the formula.
\end{remark}

\subsubsection{Subformulas}

\begin{definition}[Subformula]
The set of \emph{subformulas} of a formula $\varphi$, denoted
$\mathrm{Sub}(\varphi)$, is defined recursively as follows:

\begin{enumerate}
  \item If $\varphi$ is atomic, then $\mathrm{Sub}(\varphi) = \{\varphi\}$.

  \item If $\varphi = \neg\psi$, then
  $\mathrm{Sub}(\varphi) = \{\varphi\} \cup \mathrm{Sub}(\psi)$.

  \item If $\varphi = (\psi \circ \chi)$ for a binary connective $\circ$, then
  $\mathrm{Sub}(\varphi) = \{\varphi\} \cup \mathrm{Sub}(\psi) \cup \mathrm{Sub}(\chi)$.
\end{enumerate}

A subformula of $\varphi$ other than $\varphi$ itself is called a \emph{proper
subformula}.
\end{definition}

\subsubsection{Formula Complexity}

\begin{definition}[Formula Depth]
The \emph{depth} (or \emph{complexity}) of a formula $\varphi$, denoted
$\mathrm{depth}(\varphi)$, is defined recursively as follows:

\begin{enumerate}
  \item If $\varphi$ is atomic, then $\mathrm{depth}(\varphi) = 0$.

  \item If $\varphi = \neg\psi$, then
  $\mathrm{depth}(\varphi) = \mathrm{depth}(\psi) + 1$.

  \item If $\varphi = (\psi \circ \chi)$ for a binary connective $\circ$, then
  \[
  \mathrm{depth}(\varphi) = \max\{\mathrm{depth}(\psi), \mathrm{depth}(\chi)\} + 1.
  \]
\end{enumerate}
\end{definition}

\begin{remark}
The depth of a formula corresponds to the height of its parse tree. It measures
the maximum nesting of connectives.
\end{remark}

\subsubsection{Precedence and Parentheses}

\begin{definition}[Operator Precedence]
The standard precedence of logical connectives (from highest to lowest) is:
\begin{enumerate}
  \item Parentheses (override all precedence)
  \item Negation ($\neg$)
  \item Conjunction ($\wedge$)
  \item Disjunction ($\vee$)
  \item Conditional ($\rightarrow$)
  \item Biconditional ($\leftrightarrow$)
\end{enumerate}
\end{definition}

\begin{remark}
When connectives have the same precedence, parentheses must be used to avoid
ambiguity. Some authors treat $\wedge$ and $\vee$ as having equal precedence;
others assign $\wedge$ higher precedence than $\vee$.

The conditional $\rightarrow$ is typically right-associative:
$P \rightarrow Q \rightarrow R$ means $P \rightarrow (Q \rightarrow R)$.
\end{remark}

\begin{example}
Using standard precedence:
\begin{itemize}
  \item $\neg P \wedge Q$ means $(\neg P) \wedge Q$, not $\neg(P \wedge Q)$.
  \item $P \vee Q \wedge R$ means $P \vee (Q \wedge R)$ if $\wedge$ has higher
        precedence than $\vee$.
  \item $P \rightarrow Q \vee R$ means $P \rightarrow (Q \vee R)$.
\end{itemize}
\end{example}

% ---------------------------------------------------------
\subsection{Semantics of Propositional Logic}
% ---------------------------------------------------------

\subsubsection{Truth Assignments}

\begin{definition}[Truth Assignment]
A \emph{truth assignment} (or \emph{valuation}) is a function
\[
v : \mathsf{Prop} \to \{\mathsf{T}, \mathsf{F}\}
\]
that assigns a truth value to each propositional variable.
\end{definition}

\begin{definition}[Extension to All Formulas]
Every truth assignment $v$ extends uniquely to a function
$\hat{v} : \mathsf{WFF} \to \{\mathsf{T}, \mathsf{F}\}$ defined recursively:

\begin{enumerate}
  \item If $\varphi \in \mathsf{Prop}$, then $\hat{v}(\varphi) = v(\varphi)$.

  \item $\hat{v}(\neg\varphi) = \mathsf{T}$ iff $\hat{v}(\varphi) = \mathsf{F}$.

  \item $\hat{v}(\varphi \wedge \psi) = \mathsf{T}$ iff
        $\hat{v}(\varphi) = \mathsf{T}$ and $\hat{v}(\psi) = \mathsf{T}$.

  \item $\hat{v}(\varphi \vee \psi) = \mathsf{T}$ iff
        $\hat{v}(\varphi) = \mathsf{T}$ or $\hat{v}(\psi) = \mathsf{T}$.

  \item $\hat{v}(\varphi \rightarrow \psi) = \mathsf{T}$ iff
        $\hat{v}(\varphi) = \mathsf{F}$ or $\hat{v}(\psi) = \mathsf{T}$.

  \item $\hat{v}(\varphi \leftrightarrow \psi) = \mathsf{T}$ iff
        $\hat{v}(\varphi) = \hat{v}(\psi)$.
\end{enumerate}

We typically write $v(\varphi)$ instead of $\hat{v}(\varphi)$.
\end{definition}

\subsubsection{Truth Tables}

\begin{definition}[Truth Table]
A \emph{truth table} is a tabular representation of all possible truth values
of a formula under all possible truth assignments to its propositional variables.
\end{definition}

\begin{center}
\textbf{Truth Tables for the Basic Logical Connectives}

\vspace{0.5em}
\renewcommand{\arraystretch}{1.2}
\begin{tabular}{|c|c||c|c|c|c|c|}
\hline
$P$ & $Q$ & $\neg P$ & $P \wedge Q$ & $P \vee Q$ & $P \rightarrow Q$ & $P \leftrightarrow Q$ \\
\hline
T & T & F & T & T & T & T \\
T & F & F & F & T & F & F \\
F & T & T & F & T & T & F \\
F & F & T & F & F & T & T \\
\hline
\end{tabular}
\end{center}

\begin{remark}
A formula with $n$ distinct propositional variables has $2^n$ rows in its truth
table, corresponding to the $2^n$ possible truth assignments.
\end{remark}

\subsubsection{Satisfaction and Truth}

\begin{definition}[Satisfaction]
A truth assignment $v$ \emph{satisfies} a formula $\varphi$, written
$v \models \varphi$, if $v(\varphi) = \mathsf{T}$.

A truth assignment $v$ satisfies a set of formulas $\Gamma$ if $v$ satisfies
every formula in $\Gamma$.
\end{definition}

\begin{definition}[Truth under an Assignment]
A formula $\varphi$ is \emph{true under} $v$ if $v \models \varphi$.
A formula $\varphi$ is \emph{false under} $v$ if $v \not\models \varphi$.
\end{definition}

\subsubsection{Validity, Satisfiability, and Contradiction}

\begin{definition}[Tautology]
A formula $\varphi$ is a \emph{tautology} (or is \emph{valid}) if it is true
under every truth assignment.

We write $\models \varphi$ to indicate that $\varphi$ is a tautology.

Tautologies are sometimes denoted by the symbol $\top$.
\end{definition}

\begin{definition}[Contradiction]
A formula $\varphi$ is a \emph{contradiction} (or is \emph{unsatisfiable}) if
it is false under every truth assignment.

Contradictions are sometimes denoted by the symbol $\bot$.
\end{definition}

\begin{definition}[Satisfiable Formula]
A formula $\varphi$ is \emph{satisfiable} if there exists at least one truth
assignment under which $\varphi$ is true.

A formula is satisfiable if and only if it is not a contradiction.
\end{definition}

\begin{definition}[Contingency]
A formula $\varphi$ is a \emph{contingency} if it is neither a tautology nor a
contradiction; that is, it is true under some assignments and false under others.
\end{definition}

\begin{remark}
Every formula falls into exactly one of these three categories:
\begin{itemize}
  \item Tautology: true under all assignments.
  \item Contradiction: false under all assignments.
  \item Contingency: true under some, false under others.
\end{itemize}
\end{remark}

\begin{example}
\begin{itemize}
  \item $P \vee \neg P$ is a tautology (law of excluded middle).
  \item $P \wedge \neg P$ is a contradiction.
  \item $P \rightarrow Q$ is a contingency.
\end{itemize}
\end{example}

\subsubsection{Logical Consequence}

\begin{definition}[Logical Consequence]
Let $\Gamma$ be a set of formulas and $\varphi$ a formula.
We say that $\varphi$ is a \emph{logical consequence} of $\Gamma$, written
\[
\Gamma \models \varphi,
\]
if every truth assignment that satisfies all formulas in $\Gamma$ also satisfies
$\varphi$.

Equivalently, there is no truth assignment under which all formulas in $\Gamma$
are true and $\varphi$ is false.
\end{definition}

\begin{definition}[Tautological Implication]
Let $P$ and $Q$ be formulas.
We say that $P$ \emph{tautologically implies} $Q$, written
\[
P \models_{\mathsf{taut}} Q,
\]
if and only if the formula $P \rightarrow Q$ is a tautology.

Equivalently, $\{P\} \models Q$.
\end{definition}

\begin{remark}
Tautological implication is a special case of logical consequence where the
premise set is a singleton.
\end{remark}

\subsubsection{Logical Equivalence}

\begin{definition}[Logical Equivalence]
Two formulas $\varphi$ and $\psi$ are \emph{logically equivalent}, written
\[
\varphi \equiv \psi,
\]
if they have the same truth value under every truth assignment.

Equivalently:
\begin{itemize}
  \item $\varphi \models \psi$ and $\psi \models \varphi$.
  \item $\varphi \leftrightarrow \psi$ is a tautology.
  \item $\varphi$ and $\psi$ have identical truth tables.
\end{itemize}
\end{definition}

\begin{remark}
Logically equivalent formulas may be substituted for one another in any context
without changing the truth value of the containing formula.
\end{remark}

% ---------------------------------------------------------
\subsection{Logical Equivalences}
% ---------------------------------------------------------

\subsubsection{Double Negation}

\begin{theorem}[Double Negation]
\[
\neg\neg P \;\equiv\; P
\]
\end{theorem}

\subsubsection{De Morgan's Laws}

\begin{theorem}[De Morgan's Laws]
\begin{align*}
\neg(P \wedge Q) &\;\equiv\; \neg P \vee \neg Q \\
\neg(P \vee Q) &\;\equiv\; \neg P \wedge \neg Q
\end{align*}
\end{theorem}

\subsubsection{Commutativity}

\begin{theorem}[Commutativity]
\begin{align*}
P \wedge Q &\;\equiv\; Q \wedge P \\
P \vee Q &\;\equiv\; Q \vee P \\
P \leftrightarrow Q &\;\equiv\; Q \leftrightarrow P
\end{align*}
\end{theorem}

\subsubsection{Associativity}

\begin{theorem}[Associativity]
\begin{align*}
(P \wedge Q) \wedge R &\;\equiv\; P \wedge (Q \wedge R) \\
(P \vee Q) \vee R &\;\equiv\; P \vee (Q \vee R) \\
(P \leftrightarrow Q) \leftrightarrow R &\;\equiv\; P \leftrightarrow (Q \leftrightarrow R)
\end{align*}
\end{theorem}

\subsubsection{Distributivity}

\begin{theorem}[Distributivity]
\begin{align*}
P \wedge (Q \vee R) &\;\equiv\; (P \wedge Q) \vee (P \wedge R) \\
P \vee (Q \wedge R) &\;\equiv\; (P \vee Q) \wedge (P \vee R)
\end{align*}
\end{theorem}

\subsubsection{Idempotence}

\begin{theorem}[Idempotence]
\begin{align*}
P \wedge P &\;\equiv\; P \\
P \vee P &\;\equiv\; P
\end{align*}
\end{theorem}

\subsubsection{Absorption}

\begin{theorem}[Absorption]
\begin{align*}
P \wedge (P \vee Q) &\;\equiv\; P \\
P \vee (P \wedge Q) &\;\equiv\; P
\end{align*}
\end{theorem}

\subsubsection{Identity Laws}

\begin{theorem}[Identity Laws]
\begin{align*}
P \wedge \top &\;\equiv\; P \\
P \vee \bot &\;\equiv\; P
\end{align*}
\end{theorem}

\subsubsection{Domination Laws}

\begin{theorem}[Domination Laws]
\begin{align*}
P \vee \top &\;\equiv\; \top \\
P \wedge \bot &\;\equiv\; \bot
\end{align*}
\end{theorem}

\subsubsection{Negation Laws}

\begin{theorem}[Negation Laws]
\begin{align*}
P \vee \neg P &\;\equiv\; \top \quad \text{(Law of Excluded Middle)} \\
P \wedge \neg P &\;\equiv\; \bot \quad \text{(Law of Non-Contradiction)}
\end{align*}
\end{theorem}

\subsubsection{Conditional Equivalences}

\begin{theorem}[Material Implication]
\[
P \rightarrow Q \;\equiv\; \neg P \vee Q
\]
\end{theorem}

\begin{theorem}[Contraposition]
\[
P \rightarrow Q \;\equiv\; \neg Q \rightarrow \neg P
\]
\end{theorem}

\begin{theorem}[Exportation / Importation]
\[
(P \wedge Q) \rightarrow R \;\equiv\; P \rightarrow (Q \rightarrow R)
\]
\end{theorem}

\begin{theorem}[Negation of Conditional]
\[
\neg(P \rightarrow Q) \;\equiv\; P \wedge \neg Q
\]
\end{theorem}

\subsubsection{Biconditional Equivalences}

\begin{theorem}[Biconditional Expansion]
\[
P \leftrightarrow Q \;\equiv\; (P \rightarrow Q) \wedge (Q \rightarrow P)
\]
\end{theorem}

\begin{theorem}[Biconditional as Disjunction]
\[
P \leftrightarrow Q \;\equiv\; (P \wedge Q) \vee (\neg P \wedge \neg Q)
\]
\end{theorem}

\begin{theorem}[Negation of Biconditional]
\[
\neg(P \leftrightarrow Q) \;\equiv\; P \leftrightarrow \neg Q \;\equiv\; \neg P \leftrightarrow Q
\]
\end{theorem}

\subsubsection{Summary of Logical Equivalences}

\begin{center}
\renewcommand{\arraystretch}{1.3}
\begin{tabular}{|l|l|}
\hline
\textbf{Name} & \textbf{Equivalence} \\
\hline
Double Negation & $\neg\neg P \equiv P$ \\
\hline
De Morgan (1) & $\neg(P \wedge Q) \equiv \neg P \vee \neg Q$ \\
De Morgan (2) & $\neg(P \vee Q) \equiv \neg P \wedge \neg Q$ \\
\hline
Commutativity & $P \wedge Q \equiv Q \wedge P$; \; $P \vee Q \equiv Q \vee P$ \\
\hline
Associativity & $(P \wedge Q) \wedge R \equiv P \wedge (Q \wedge R)$ \\
              & $(P \vee Q) \vee R \equiv P \vee (Q \vee R)$ \\
\hline
Distributivity & $P \wedge (Q \vee R) \equiv (P \wedge Q) \vee (P \wedge R)$ \\
               & $P \vee (Q \wedge R) \equiv (P \vee Q) \wedge (P \vee R)$ \\
\hline
Idempotence & $P \wedge P \equiv P$; \; $P \vee P \equiv P$ \\
\hline
Absorption & $P \wedge (P \vee Q) \equiv P$; \; $P \vee (P \wedge Q) \equiv P$ \\
\hline
Identity & $P \wedge \top \equiv P$; \; $P \vee \bot \equiv P$ \\
\hline
Domination & $P \vee \top \equiv \top$; \; $P \wedge \bot \equiv \bot$ \\
\hline
Negation & $P \vee \neg P \equiv \top$; \; $P \wedge \neg P \equiv \bot$ \\
\hline
Material Implication & $P \rightarrow Q \equiv \neg P \vee Q$ \\
\hline
Contraposition & $P \rightarrow Q \equiv \neg Q \rightarrow \neg P$ \\
\hline
Exportation & $(P \wedge Q) \rightarrow R \equiv P \rightarrow (Q \rightarrow R)$ \\
\hline
Biconditional & $P \leftrightarrow Q \equiv (P \rightarrow Q) \wedge (Q \rightarrow P)$ \\
\hline
\end{tabular}
\end{center}

% ---------------------------------------------------------
\subsection{Normal Forms}
% ---------------------------------------------------------

\subsubsection{Literals and Clauses}

\begin{definition}[Literal]
A \emph{literal} is either a propositional variable or the negation of a
propositional variable.

A \emph{positive literal} is a propositional variable $P$.
A \emph{negative literal} is the negation of a propositional variable $\neg P$.
\end{definition}

\begin{definition}[Clause]
A \emph{clause} is a disjunction of literals.

A \emph{conjunctive clause} (or \emph{term}) is a conjunction of literals.
\end{definition}

\subsubsection{Negation Normal Form}

\begin{definition}[Negation Normal Form]
A formula is in \emph{negation normal form} (NNF) if:
\begin{enumerate}
  \item Negation ($\neg$) is applied only to propositional variables.
  \item The only connectives used are $\neg$, $\wedge$, and $\vee$.
\end{enumerate}
\end{definition}

\begin{theorem}[NNF Conversion]
Every propositional formula is logically equivalent to a formula in negation
normal form.
\end{theorem}

\begin{remark}[Procedure for NNF Conversion]
\begin{enumerate}
  \item Eliminate $\rightarrow$ and $\leftrightarrow$ using:
  \begin{align*}
    P \rightarrow Q &\equiv \neg P \vee Q \\
    P \leftrightarrow Q &\equiv (P \wedge Q) \vee (\neg P \wedge \neg Q)
  \end{align*}
  \item Push negations inward using De Morgan's laws and double negation:
  \begin{align*}
    \neg(P \wedge Q) &\equiv \neg P \vee \neg Q \\
    \neg(P \vee Q) &\equiv \neg P \wedge \neg Q \\
    \neg\neg P &\equiv P
  \end{align*}
\end{enumerate}
\end{remark}

\subsubsection{Conjunctive Normal Form}

\begin{definition}[Conjunctive Normal Form]
A formula is in \emph{conjunctive normal form} (CNF) if it is a conjunction of
clauses; that is, a conjunction of disjunctions of literals:
\[
\bigwedge_{i} \bigvee_{j} L_{ij}
\]
where each $L_{ij}$ is a literal.
\end{definition}

\begin{theorem}[CNF Existence]
Every propositional formula is logically equivalent to a formula in conjunctive
normal form.
\end{theorem}

\begin{example}
The formula $(P \vee Q) \wedge (\neg P \vee R) \wedge (Q \vee \neg R)$ is in CNF.
\end{example}

\subsubsection{Disjunctive Normal Form}

\begin{definition}[Disjunctive Normal Form]
A formula is in \emph{disjunctive normal form} (DNF) if it is a disjunction of
conjunctive clauses; that is, a disjunction of conjunctions of literals:
\[
\bigvee_{i} \bigwedge_{j} L_{ij}
\]
where each $L_{ij}$ is a literal.
\end{definition}

\begin{theorem}[DNF Existence]
Every propositional formula is logically equivalent to a formula in disjunctive
normal form.
\end{theorem}

\begin{example}
The formula $(P \wedge Q) \vee (\neg P \wedge R) \vee (Q \wedge \neg R)$ is in DNF.
\end{example}

\subsubsection{Conversion Procedures}

\begin{remark}[CNF Conversion Procedure]
To convert a formula to CNF:
\begin{enumerate}
  \item Convert to negation normal form.
  \item Apply distribution of $\vee$ over $\wedge$:
  \[
  P \vee (Q \wedge R) \equiv (P \vee Q) \wedge (P \vee R)
  \]
  \item Repeat until the formula is a conjunction of disjunctions.
\end{enumerate}
\end{remark}

\begin{remark}[DNF Conversion Procedure]
To convert a formula to DNF:
\begin{enumerate}
  \item Convert to negation normal form.
  \item Apply distribution of $\wedge$ over $\vee$:
  \[
  P \wedge (Q \vee R) \equiv (P \wedge Q) \vee (P \wedge R)
  \]
  \item Repeat until the formula is a disjunction of conjunctions.
\end{enumerate}
\end{remark}

\begin{remark}[Truth Table Method]
CNF and DNF can also be constructed directly from a truth table:
\begin{itemize}
  \item \textbf{DNF:} For each row where the formula is true, form a conjunctive
        clause using positive literals for variables assigned T and negative
        literals for variables assigned F. Take the disjunction of all such clauses.
  \item \textbf{CNF:} For each row where the formula is false, form a disjunctive
        clause using positive literals for variables assigned F and negative
        literals for variables assigned T. Take the conjunction of all such clauses.
\end{itemize}
\end{remark}

% ---------------------------------------------------------
\subsection{Inference Rules}
% ---------------------------------------------------------

\subsubsection{Basic Inference Rules}

\begin{definition}[Modus Ponens]
\textbf{Modus Ponens} (Law of Detachment): From a conditional and its antecedent,
infer the consequent.
\[
\begin{array}{l}
P \rightarrow Q \\
P \\ \hline
Q
\end{array}
\]
\end{definition}

\begin{definition}[Modus Tollens]
\textbf{Modus Tollens} (Denying the Consequent): From a conditional and the
negation of its consequent, infer the negation of the antecedent.
\[
\begin{array}{l}
P \rightarrow Q \\
\neg Q \\ \hline
\neg P
\end{array}
\]
\end{definition}

\begin{definition}[Hypothetical Syllogism]
\textbf{Hypothetical Syllogism} (Chain Rule): From two conditionals where the
consequent of the first is the antecedent of the second, infer a conditional
linking the first antecedent to the second consequent.
\[
\begin{array}{l}
P \rightarrow Q \\
Q \rightarrow R \\ \hline
P \rightarrow R
\end{array}
\]
\end{definition}

\begin{definition}[Disjunctive Syllogism]
\textbf{Disjunctive Syllogism} (Modus Tollendo Ponens): From a disjunction and the
negation of one disjunct, infer the other disjunct.
\[
\begin{array}{l}
P \vee Q \\
\neg P \\ \hline
Q
\end{array}
\qquad\text{or}\qquad
\begin{array}{l}
P \vee Q \\
\neg Q \\ \hline
P
\end{array}
\]
\end{definition}

\subsubsection{Conjunction Rules}

\begin{definition}[Conjunction Introduction]
\textbf{Conjunction Introduction} (Adjunction): From two propositions, infer
their conjunction.
\[
\begin{array}{l}
P \\
Q \\ \hline
P \wedge Q
\end{array}
\]
\end{definition}

\begin{definition}[Conjunction Elimination]
\textbf{Conjunction Elimination} (Simplification): From a conjunction, infer
either conjunct.
\[
\begin{array}{l}
P \wedge Q \\ \hline
P
\end{array}
\qquad\text{or}\qquad
\begin{array}{l}
P \wedge Q \\ \hline
Q
\end{array}
\]
\end{definition}

\subsubsection{Disjunction Rules}

\begin{definition}[Disjunction Introduction]
\textbf{Disjunction Introduction} (Addition): From a proposition, infer any
disjunction containing it.
\[
\begin{array}{l}
P \\ \hline
P \vee Q
\end{array}
\]
\end{definition}

\begin{definition}[Disjunction Elimination]
\textbf{Disjunction Elimination} (Proof by Cases): If a disjunction is true and
the same conclusion follows from each disjunct, then that conclusion may be
inferred.
\[
\begin{array}{l}
P \vee Q \\
\begin{array}{l}
\text{Assume } P \\ \vdots \\ R
\end{array}
\quad
\begin{array}{l}
\text{Assume } Q \\ \vdots \\ R
\end{array}
\\ \hline
R
\end{array}
\]
\end{definition}

\subsubsection{Conditional Rules}

\begin{definition}[Conditional Introduction]
\textbf{Conditional Introduction} (Conditional Proof): If assuming $P$ leads to
$Q$, then infer $P \rightarrow Q$.
\[
\begin{array}{l}
\text{Assume } P \\ \vdots \\ Q \\ \hline
P \rightarrow Q
\end{array}
\]
\end{definition}

\subsubsection{Biconditional Rules}

\begin{definition}[Biconditional Introduction]
\textbf{Biconditional Introduction}: From two conditionals in opposite directions,
infer their biconditional.
\[
\begin{array}{l}
P \rightarrow Q \\
Q \rightarrow P \\ \hline
P \leftrightarrow Q
\end{array}
\]
\end{definition}

\begin{definition}[Biconditional Elimination]
\textbf{Biconditional Elimination}: From a biconditional, infer either component
conditional.
\[
\begin{array}{l}
P \leftrightarrow Q \\ \hline
P \rightarrow Q
\end{array}
\qquad
\begin{array}{l}
P \leftrightarrow Q \\ \hline
Q \rightarrow P
\end{array}
\]
\end{definition}

\subsubsection{Negation Rules}

\begin{definition}[Negation Introduction]
\textbf{Negation Introduction} (Reductio ad Absurdum): If assuming $P$ leads to
a contradiction, then infer $\neg P$.
\[
\begin{array}{l}
\text{Assume } P \\ \vdots \\ \bot \\ \hline
\neg P
\end{array}
\]
\end{definition}

\begin{definition}[Negation Elimination]
\textbf{Negation Elimination} (Indirect Proof): If assuming $\neg P$ leads to a
contradiction, then infer $P$.
\[
\begin{array}{l}
\text{Assume } \neg P \\ \vdots \\ \bot \\ \hline
P
\end{array}
\]
\end{definition}

\begin{definition}[Double Negation Elimination]
\textbf{Double Negation Elimination}: From a double negation, infer the original
proposition.
\[
\begin{array}{l}
\neg\neg P \\ \hline
P
\end{array}
\]
\end{definition}

\subsubsection{Dilemma Rules}

\begin{definition}[Constructive Dilemma]
\textbf{Constructive Dilemma}: From a disjunction and two conditionals whose
antecedents are the disjuncts, infer the disjunction of the consequents.
\[
\begin{array}{l}
P \vee Q \\
P \rightarrow R \\
Q \rightarrow S \\ \hline
R \vee S
\end{array}
\]
\end{definition}

\begin{definition}[Destructive Dilemma]
\textbf{Destructive Dilemma}: From two conditionals and the disjunction of their
negated consequents, infer the disjunction of their negated antecedents.
\[
\begin{array}{l}
P \rightarrow R \\
Q \rightarrow S \\
\neg R \vee \neg S \\ \hline
\neg P \vee \neg Q
\end{array}
\]
\end{definition}

\subsubsection{Structural Rules}

\begin{definition}[Reiteration]
\textbf{Reiteration}: Any previously derived statement may be restated.
\[
\begin{array}{l}
P \\ \hline
P
\end{array}
\]
\end{definition}

\begin{definition}[Assumption]
\textbf{Assumption Rule}: A proposition may be temporarily assumed for the
purpose of a subproof. Assumptions must be discharged before the proof is
complete.
\end{definition}

\subsubsection{Summary of Inference Rules}

\begin{center}
\renewcommand{\arraystretch}{1.3}
\begin{tabular}{|l|l|l|}
\hline
\textbf{Rule} & \textbf{Premises} & \textbf{Conclusion} \\
\hline
Modus Ponens & $P \rightarrow Q$, $P$ & $Q$ \\
Modus Tollens & $P \rightarrow Q$, $\neg Q$ & $\neg P$ \\
Hypothetical Syllogism & $P \rightarrow Q$, $Q \rightarrow R$ & $P \rightarrow R$ \\
Disjunctive Syllogism & $P \vee Q$, $\neg P$ & $Q$ \\
Conjunction Introduction & $P$, $Q$ & $P \wedge Q$ \\
Conjunction Elimination & $P \wedge Q$ & $P$ (or $Q$) \\
Disjunction Introduction & $P$ & $P \vee Q$ \\
Constructive Dilemma & $P \vee Q$, $P \rightarrow R$, $Q \rightarrow S$ & $R \vee S$ \\
Destructive Dilemma & $P \rightarrow R$, $Q \rightarrow S$, $\neg R \vee \neg S$ & $\neg P \vee \neg Q$ \\
Biconditional Introduction & $P \rightarrow Q$, $Q \rightarrow P$ & $P \leftrightarrow Q$ \\
Biconditional Elimination & $P \leftrightarrow Q$ & $P \rightarrow Q$ \\
Double Negation Elimination & $\neg\neg P$ & $P$ \\
\hline
\end{tabular}
\end{center}

% ---------------------------------------------------------
\subsection{Proof Systems}
% ---------------------------------------------------------

\subsubsection{Derivability}

\begin{definition}[Derivability]
Let $\Gamma$ be a set of formulas and $\varphi$ a formula.
We say that $\varphi$ is \emph{derivable} from $\Gamma$, written
\[
\Gamma \vdash \varphi,
\]
if there exists a formal proof of $\varphi$ from the assumptions in $\Gamma$
using the rules of a specified deductive system.
\end{definition}

\begin{remark}
Derivability is a \emph{syntactic} notion. It depends on a chosen proof system
and concerns what can be obtained by applying inference rules step by step.
Unlike logical consequence, derivability makes no direct reference to truth or
interpretations.
\end{remark}

\subsubsection{Soundness and Completeness}

\begin{definition}[Soundness]
A proof system is \emph{sound} if every derivable formula is valid.

Formally, if $\Gamma \vdash \varphi$, then $\Gamma \models \varphi$.
\end{definition}

\begin{definition}[Completeness]
A proof system is \emph{complete} if every valid formula is derivable.

Formally, if $\Gamma \models \varphi$, then $\Gamma \vdash \varphi$.
\end{definition}

\begin{theorem}[Soundness and Completeness of Propositional Logic]
The standard proof systems for propositional logic (natural deduction, sequent
calculus, Hilbert systems) are both sound and complete.

Thus, for propositional logic:
\[
\Gamma \vdash \varphi \quad\Longleftrightarrow\quad \Gamma \models \varphi.
\]
\end{theorem}

\begin{remark}
Soundness ensures that proofs do not lead us astray: we cannot prove false
statements from true premises. Completeness ensures that proofs are powerful
enough: every valid argument can be formally verified.
\end{remark}

\subsubsection{Comparison: Semantic vs.\ Syntactic Notions}

\begin{center}
\renewcommand{\arraystretch}{1.3}
\begin{tabular}{|l|l|l|}
\hline
\textbf{Notion} & \textbf{Symbol} & \textbf{Nature} \\
\hline
Tautological implication & $P \models_{\mathsf{taut}} Q$ & Semantic, truth-table based \\
Logical consequence & $\Gamma \models \varphi$ & Semantic, model-theoretic \\
Derivability & $\Gamma \vdash \varphi$ & Syntactic, proof-theoretic \\
\hline
\end{tabular}
\end{center}

\begin{remark}
The distinction between $\models$ (semantic entailment) and $\vdash$ (syntactic
derivability) is central to logic. Soundness and completeness theorems establish
the correspondence between these two perspectives.
\end{remark}

% ---------------------------------------------------------
\subsection{Common Errors and Fallacies}
% ---------------------------------------------------------

\subsubsection{Formal Fallacies}

\begin{definition}[Affirming the Consequent]
\textbf{Affirming the Consequent} is the invalid inference:
\[
\begin{array}{l}
P \rightarrow Q \\
Q \\ \hline
P \quad \text{(INVALID)}
\end{array}
\]

This is fallacious because $Q$ being true does not guarantee that $P$ is true;
$Q$ might be true for other reasons.
\end{definition}

\begin{example}
``If it rains, the ground is wet. The ground is wet. Therefore, it rained.''

This is invalid: the ground might be wet because of a sprinkler.
\end{example}

\begin{definition}[Denying the Antecedent]
\textbf{Denying the Antecedent} is the invalid inference:
\[
\begin{array}{l}
P \rightarrow Q \\
\neg P \\ \hline
\neg Q \quad \text{(INVALID)}
\end{array}
\]

This is fallacious because the conditional $P \rightarrow Q$ says nothing about
what happens when $P$ is false.
\end{definition}

\begin{example}
``If it rains, the ground is wet. It did not rain. Therefore, the ground is not wet.''

This is invalid: the ground might be wet for other reasons.
\end{example}

\subsubsection{Comparison: Valid vs.\ Invalid Inference Patterns}

\begin{center}
\renewcommand{\arraystretch}{1.3}
\begin{tabular}{|l|c|l|}
\hline
\textbf{Pattern} & \textbf{Valid?} & \textbf{Form} \\
\hline
Modus Ponens & Yes & $P \rightarrow Q$, $P$ $\therefore$ $Q$ \\
Modus Tollens & Yes & $P \rightarrow Q$, $\neg Q$ $\therefore$ $\neg P$ \\
Affirming the Consequent & No & $P \rightarrow Q$, $Q$ $\therefore$ $P$ \\
Denying the Antecedent & No & $P \rightarrow Q$, $\neg P$ $\therefore$ $\neg Q$ \\
\hline
\end{tabular}
\end{center}

\subsubsection{Common Truth Table Errors}

\begin{remark}[Common Errors]
\begin{itemize}
  \item \textbf{Misunderstanding the conditional:} The conditional
        $P \rightarrow Q$ is false only when $P$ is true and $Q$ is false. It is
        true in all other cases, including when $P$ is false.

  \item \textbf{Confusing $\rightarrow$ with $\leftrightarrow$:} The conditional
        is not symmetric; $P \rightarrow Q$ is not equivalent to $Q \rightarrow P$.

  \item \textbf{Confusing ``or'' meanings:} In logic, $\vee$ is inclusive
        (true when at least one disjunct is true). Exclusive or (true when
        exactly one is true) is written $P \oplus Q$ or $(P \vee Q) \wedge \neg(P \wedge Q)$.

  \item \textbf{Forgetting rows:} A truth table for $n$ variables must have
        $2^n$ rows.

  \item \textbf{Operator precedence:} Failing to account for precedence leads
        to incorrect parsing of formulas.
\end{itemize}
\end{remark}

\subsubsection{Fallacy Checklist}

\begin{center}
\renewcommand{\arraystretch}{1.4}
\begin{tabular}{|p{5cm}|p{8cm}|}
\hline
\textbf{Fallacy} & \textbf{Diagnostic Question} \\
\hline
Affirming the Consequent & Was $Q$ used to infer $P$ from $P \rightarrow Q$? \\
\hline
Denying the Antecedent & Was $\neg P$ used to infer $\neg Q$ from $P \rightarrow Q$? \\
\hline
Misreading Conditional & Was the conditional treated as false when the antecedent is false? \\
\hline
Confusing Implication Direction & Was $P \rightarrow Q$ confused with $Q \rightarrow P$? \\
\hline
Missing Cases & Were all $2^n$ rows of the truth table considered? \\
\hline
\end{tabular}
\end{center}

% ---------------------------------------------------------
\subsection{Summary Tables}
% ---------------------------------------------------------

\subsubsection{Connective Truth Tables}

\begin{center}
\renewcommand{\arraystretch}{1.2}
\begin{tabular}{|c|c||c|c|c|c|c|}
\hline
$P$ & $Q$ & $\neg P$ & $P \wedge Q$ & $P \vee Q$ & $P \rightarrow Q$ & $P \leftrightarrow Q$ \\
\hline
T & T & F & T & T & T & T \\
T & F & F & F & T & F & F \\
F & T & T & F & T & T & F \\
F & F & T & F & F & T & T \\
\hline
\end{tabular}
\end{center}

\subsubsection{Classification of Formulas}

\begin{center}
\renewcommand{\arraystretch}{1.3}
\begin{tabular}{|l|l|l|}
\hline
\textbf{Classification} & \textbf{Definition} & \textbf{Example} \\
\hline
Tautology & True under all assignments & $P \vee \neg P$ \\
Contradiction & False under all assignments & $P \wedge \neg P$ \\
Contingency & True under some, false under others & $P \rightarrow Q$ \\
Satisfiable & True under at least one assignment & $P$, $P \vee Q$ \\
\hline
\end{tabular}
\end{center}

\subsubsection{Comparison: Propositional vs.\ Predicate Logic}

\begin{center}
\renewcommand{\arraystretch}{1.3}
\begin{tabular}{|p{4cm}|p{5cm}|p{5cm}|}
\hline
\textbf{Aspect} & \textbf{Propositional Logic} & \textbf{Predicate Logic} \\
\hline
Atomic formulas & Propositional variables & Predicate symbols applied to terms \\
\hline
Internal structure & None & Contains terms, variables, functions \\
\hline
Quantifiers & Not available & $\forall$, $\exists$ \\
\hline
Semantic evaluation & Truth assignments & Structures and variable assignments \\
\hline
Decision problem & Decidable (truth tables) & Undecidable in general \\
\hline
\end{tabular}
\end{center}