% notes.tex









\noindent\textbf{Source.} \srccite{Gerstein}{Chapter~1, \S1.2}





\noindent
The basic logical connectives in propositional logic are:
\emph{and}, \emph{or}, \emph{not}, and \emph{implication}.
They are denoted symbolically by
\[
\land \quad \lor \quad \neg \quad \rightarrow
\]
respectively.

\vspace{1em}

\noindent
Their meanings are defined by the following truth tables.

\vspace{1em}

\subsubsection*{Truth Tables for the Basic Logical Connectives}

\begin{center}
\renewcommand{\arraystretch}{1.2}
\begin{tabular}{|c|c||c|c|c|c|}
\hline
$p$ & $q$ & $\neg p$ & $p \land q$ & $p \lor q$ & $p \rightarrow q$ \\
\hline
T & T & F & T & T & T \\
T & F & F & F & T & F \\
F & T & T & F & T & T \\
F & F & T & F & F & T \\
\hline
\end{tabular}
\end{center}

\noindent\textbf{Order of Operations in Propositional Logic.}
The standard precedence of logical connectives, from highest to lowest, is as follows:
\begin{enumerate}
  \item Parentheses
  \item Negation ($\neg$)
  \item Conjunction ($\wedge$) and Disjunction ($\vee$)
  \item Conditional ($\rightarrow$) and Biconditional ($\leftrightarrow$)
\end{enumerate}

\noindent
When connectives have the same precedence level (for example, $\wedge$ and $\vee$),
parentheses should be used to avoid ambiguity.

The \textbf{converse} of a conditional proposition $P \rightarrow Q$ is the conditional
proposition obtained by interchanging the antecedent and the consequent, namely
$Q \rightarrow P$. In general, a conditional proposition and its converse are not
logically equivalent.

The \textbf{contrapositive} of a conditional proposition $P \rightarrow Q$ is the
conditional proposition obtained by negating both the antecedent and the consequent
and reversing their order, namely $\neg Q \rightarrow \neg P$. A conditional
proposition is logically equivalent to its contrapositive.

The \textbf{inverse} of a conditional proposition $P \rightarrow Q$ is the conditional
proposition obtained by negating both the antecedent and the consequent without
reversing their order, namely $\neg P \rightarrow \neg Q$. In general, a conditional
proposition and its inverse are not logically equivalent.

\begin{center}
\begin{tabular}{|c|c|c|c|}
\hline
\textbf{Form} 
& \textbf{Expression} 
& \textbf{Equivalent to $P \rightarrow Q$?} 
& \textbf{Equivalent to $Q \rightarrow P$?} \\
\hline
Conditional 
& $P \rightarrow Q$ 
& Yes 
& No \\
\hline
Converse 
& $Q \rightarrow P$ 
& No 
& Yes \\
\hline
Inverse 
& $\neg P \rightarrow \neg Q$ 
& No 
& Yes \\
\hline
Contrapositive 
& $\neg Q \rightarrow \neg P$ 
& Yes 
& No \\
\hline
\end{tabular}
\end{center}


\noindent\textbf{Source.}
\srccite{LibreTextsLogicForAll}{Chapter~3}.

\subsubsection*{Rules of Inference}

\noindent\textbf{Source.} \srccite{SuppesHillFirstCourseLogic}{Chapter~2, \S2}

$\textbf{The Law of Detachment} $
(Modus Ponens) is a form of syllogism. It states that if a conditional statement and its hypothesis are accepted as true, its conclusion will also be true. Three basic elements are at the core of the Law (or Rule) of Detachment. These can be expressed in a simple formulation:

\textbf{Modus Ponens} is a rule of inference stating that from a conditional proposition
$P \rightarrow Q$ and its antecedent $P$, one may validly infer the consequent $Q$.
Symbolically:
\[
\begin{array}{ll}
\text{Premise 1.} & P \rightarrow Q \\
\text{Premise 2.} & P \\
\hline
\text{Conclusion.} & Q
\end{array}
\]


\[
\text{If } P \rightarrow Q \text{ is true and } P \text{ is true, then } Q \text{ must be true.}
\]

\noindent\textbf{Remark (Contrapositive Form).}
The conditional proposition $P \rightarrow Q$ is logically equivalent to its
contrapositive $\neg Q \rightarrow \neg P$. Consequently, modus ponens may be
applied to the contrapositive form: from $\neg Q \rightarrow \neg P$ and $\neg Q$,
one may validly infer $\neg P$. This use is justified by logical equivalence and
does not introduce a new rule of inference.


\textbf{Modus Tollens} is a rule of inference stating that from a conditional proposition
$P \rightarrow Q$ and the negation of its consequent $\neg Q$, one may validly infer
the negation of the antecedent $\neg P$.
Symbolically:
\[
\begin{array}{c}
P \rightarrow Q \\
\neg Q \\
\hline
\neg P
\end{array}
\]

\[
\text{If } P \rightarrow Q \text{ is true and } \neg Q \text{ is true, then } \neg P \text{ must be true.}
\]


$\textbf{Law of Detachment}$
The Law of Detachment is a rule of inference stating that if a conditional
proposition $P \rightarrow Q$ is true and its antecedent $P$ is true, then the
consequent $Q$ may be validly inferred. The law of detachment is equivalent to
modus ponens.

$\textbf{Law of Premises}$
The Law of Premises is the principle that any statement explicitly assumed
as a premise in a proof may be asserted as true at any subsequent step of that proof,
provided it has not been discharged. Premises serve as the foundational assumptions
from which conclusions are derived.

\noindent\textbf{Double Negation.}
The \textbf{Rule of Double Negation} permits the elimination or introduction of
two successive negations without changing the truth value of a proposition.

\vspace{0.5em}

\noindent
\textbf{(Elimination Form).}
From the negation of a negation of a proposition, one may validly infer the
proposition itself:
\[
\begin{array}{ll}
\text{Premise.} & \neg\neg P \\
\hline
\text{Conclusion.} & P
\end{array}
\]

\vspace{0.75em}

\noindent
\textbf{(Introduction Form).}
From a proposition, one may validly infer its double negation:
\[
\begin{array}{ll}
\text{Premise.} & P \\
\hline
\text{Conclusion.} & \neg\neg P
\end{array}
\]

\vspace{0.75em}

\textbf{Rule of Adjunction} is a rule of inference stating that if two propositions
$P$ and $Q$ have each been established as true, then their conjunction $P \wedge Q$
may be validly inferred. The rule permits the introduction of a conjunction from
its conjuncts.

Symbolically:
\[
\begin{array}{ll}
\text{Premise 1.} & P \\
\text{Premise 2.} & Q \\
\hline
\text{Conclusion.} & P \wedge Q
\end{array}
\]

\textbf{Rule of Simplification} is a rule of inference stating that from a conjunction
$P \wedge Q$, one may validly infer either of its conjuncts. The rule permits the
elimination of a conjunction into its component propositions.

Symbolically:
\[
\begin{array}{ll}
\text{Premise.} & P \wedge Q \\
\hline
\text{Conclusion.} & P
\end{array}
\qquad
\text{or}
\qquad
\begin{array}{ll}
\text{Premise.} & P \wedge Q \\
\hline
\text{Conclusion.} & Q
\end{array}
\]

\textbf{Disjunction Introduction / Law of Addition}
 is a rule of inference stating that from a proposition
$P$, one may validly infer a disjunction in which $P$ is a disjunct. That is, if $P$ has
been established as true, then $P \vee Q$ may be inferred for any proposition $Q$.
Likewise, from $Q$ one may infer $P \vee Q$.

Symbolically:
\[
\begin{array}{ll}
\text{Premise.} & P \\
\hline
\text{Conclusion.} & P \vee Q
\end{array}
\qquad
\text{or}
\qquad
\begin{array}{ll}
\text{Premise.} & Q \\
\hline
\text{Conclusion.} & P \vee Q
\end{array}
\]
\end{definition}

\textbf{Disjunction Elimination}
is a rule of inference stating that if a disjunction
$P \vee Q$ has been established, and if a proposition $R$ can be derived from the
assumption that $P$ is true and also from the assumption that $Q$ is true, then $R$
may be validly inferred without either assumption. The rule justifies reasoning by
cases.

Symbolically:
\[
\begin{array}{l}
P \vee Q \\
\begin{array}{ll}
\text{Assume } P & \vdots \\
& R
\end{array}
\qquad
\begin{array}{ll}
\text{Assume } Q & \vdots \\
& R
\end{array}
\\ \hline
R
\end{array}
\]
\end{definition}







The rule states that if a disjunction
$P \vee Q$ is given as a premise, and if a conclusion $R$ can be derived both from the
assumption that $P$ is true and from the assumption that $Q$ is true, then $R$ may be
validly inferred without either assumption.

Symbolically:
\[
\begin{array}{l}
P \vee Q \\
\begin{array}{ll}
\text{Assume } P & \vdots \\
& R
\end{array}
\qquad
\begin{array}{ll}
\text{Assume } Q & \vdots \\
& R
\end{array}
\\ \hline
R
\end{array}
\]


\textbf{Modus Tollendo Ponens} is a rule of inference stating that if a disjunction
$P \vee Q$ has been established and one of its disjuncts is denied, then the other
disjunct may be validly inferred. The rule licenses the elimination of a disjunction
when one alternative is shown to be false.

Symbolically:
\[
\begin{array}{ll}
\text{Premise 1.} & P \vee Q \\
\text{Premise 2.} & \neg P \\
\hline
\text{Conclusion.} & Q
\end{array}
\qquad
\text{or}
\qquad
\begin{array}{ll}
\text{Premise 1.} & P \vee Q \\
\text{Premise 2.} & \neg Q \\
\hline
\text{Conclusion.} & P
\end{array}
\]


Remove below


\noindent\textbf{Symbolically:}

\[
\begin{array}{ll}
\begin{array}{l}
\text{Premise 1.}\; P \vee Q \\
\text{Premise 2.}\; \neg P \\ \hline
\text{Conclusion.}\; Q
\end{array}
&
\qquad\text{or}\qquad
\begin{array}{l}
\text{Premise 1.}\; P \vee Q \\
\text{Premise 2.}\; \neg Q \\ \hline
\text{Conclusion.}\; P
\end{array}
\end{array}
\]


remove above







\textbf{Reiteration Rule} is a rule of inference that permits any previously
established statement in a proof to be restated at a later step. Reiteration does not
introduce new information, but allows earlier results to be reused when needed.

\noindent\textbf{Source.}
\srccite{OpenLogicForallxCalgary}{Chapter~17 — Basic Rules for TFL}.

The \textbf{Law of Hypothetical Syllogism} is a rule of inference stating that if one
conditional proposition asserts that $P$ implies $Q$, and a second conditional
proposition asserts that $Q$ implies $R$, then one may validly infer that $P$ implies
$R$. The rule captures the transitivity of implication.

Symbolically:
\[
\begin{array}{ll}
\text{Premise 1.} & P \rightarrow Q \\
\text{Premise 2.} & Q \rightarrow R \\
\hline
\text{Conclusion.} & P \rightarrow R
\end{array}
\]












