% notes.tex









\noindent\textbf{Source.} \srccite{Gerstein}{Chapter~1, \S1.2}





\noindent
The basic logical connectives in propositional logic are:
\emph{and}, \emph{or}, \emph{not}, and \emph{implication}.
They are denoted symbolically by
\[
\land \quad \lor \quad \neg \quad \rightarrow
\]
respectively.

\vspace{1em}

\noindent
Their meanings are defined by the following truth tables.

\vspace{1em}

\subsubsection*{Truth Tables for the Basic Logical Connectives}

\begin{center}
\renewcommand{\arraystretch}{1.2}
\begin{tabular}{|c|c||c|c|c|c|}
\hline
$p$ & $q$ & $\neg p$ & $p \land q$ & $p \lor q$ & $p \rightarrow q$ \\
\hline
T & T & F & T & T & T \\
T & F & F & F & T & F \\
F & T & T & F & T & T \\
F & F & T & F & F & T \\
\hline
\end{tabular}
\end{center}





\subsubsection*{Rules of Inference}

\noindent\textbf{Source.} \srccite{SuppesHillFirstCourseLogic}{Chapter~2, \S2}

$\textbf{The Law of Detachment} $
(Modus Ponens) is a form of syllogism. It states that if a conditional statement and its hypothesis are accepted as true, its conclusion will also be true. Three basic elements are at the core of the Law (or Rule) of Detachment. These can be expressed in a simple formulation:

\textbf{Modus Ponens} is a rule of inference stating that from a conditional proposition
$P \rightarrow Q$ and its antecedent $P$, one may validly infer the consequent $Q$.
Symbolically:
\[
\begin{array}{ll}
\text{Premise 1.} & P \rightarrow Q \\
\text{Premise 2.} & P \\
\hline
\text{Conclusion.} & Q
\end{array}
\]


\[
\text{If } P \rightarrow Q \text{ is true and } P \text{ is true, then } Q \text{ must be true.}
\]

\noindent\textbf{Remark (Contrapositive Form).}
The conditional proposition $P \rightarrow Q$ is logically equivalent to its
contrapositive $\neg Q \rightarrow \neg P$. Consequently, modus ponens may be
applied to the contrapositive form: from $\neg Q \rightarrow \neg P$ and $\neg Q$,
one may validly infer $\neg P$. This use is justified by logical equivalence and
does not introduce a new rule of inference.


\textbf{Modus Tollens} is a rule of inference stating that from a conditional proposition
$P \rightarrow Q$ and the negation of its consequent $\neg Q$, one may validly infer
the negation of the antecedent $\neg P$.
Symbolically:
\[
\begin{array}{c}
P \rightarrow Q \\
\neg Q \\
\hline
\neg P
\end{array}
\]

\[
\text{If } P \rightarrow Q \text{ is true and } \neg Q \text{ is true, then } \neg P \text{ must be true.}
\]


$\textbf{Law of Detachment}$
The Law of Detachment is a rule of inference stating that if a conditional
proposition $P \rightarrow Q$ is true and its antecedent $P$ is true, then the
consequent $Q$ may be validly inferred. The law of detachment is equivalent to
modus ponens.

$\textbf{Law of Premises}$
The Law of Premises is the principle that any statement explicitly assumed
as a premise in a proof may be asserted as true at any subsequent step of that proof,
provided it has not been discharged. Premises serve as the foundational assumptions
from which conclusions are derived.

\noindent\textbf{Double Negation.}
The \textbf{Rule of Double Negation} permits the elimination or introduction of
two successive negations without changing the truth value of a proposition.

\vspace{0.5em}

\noindent
\textbf{(Elimination Form).}
From the negation of a negation of a proposition, one may validly infer the
proposition itself:
\[
\begin{array}{ll}
\text{Premise.} & \neg\neg P \\
\hline
\text{Conclusion.} & P
\end{array}
\]

\vspace{0.75em}

\noindent
\textbf{(Introduction Form).}
From a proposition, one may validly infer its double negation:
\[
\begin{array}{ll}
\text{Premise.} & P \\
\hline
\text{Conclusion.} & \neg\neg P
\end{array}
\]

\vspace{0.75em}
























