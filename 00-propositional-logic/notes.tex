% =========================================================
% Propositional Logic — Rules, Connectives, and Equivalences
% =========================================================

\chapter*{Propositional Logic Foundations}
\addcontentsline{toc}{chapter}{Propositional Logic Foundations}

% =========================================================
% Logical Connectives
% =========================================================

\noindent\textbf{Logical Connectives (Semantic)}
The basic logical connectives in propositional logic are:
\emph{negation}, \emph{conjunction}, \emph{disjunction}, and \emph{conditional}.
They are denoted symbolically by
\[
\neg \quad \wedge \quad \vee \quad \rightarrow
\]
Their meanings are defined semantically by truth tables.

\vspace{1em}

\subsection*{Truth Tables for the Basic Logical Connectives}

\begin{center}
\renewcommand{\arraystretch}{1.2}
\begin{tabular}{|c|c||c|c|c|c|}
\hline
$p$ & $q$ & $\neg p$ & $p \wedge q$ & $p \vee q$ & $p \rightarrow q$ \\
\hline
T & T & F & T & T & T \\
T & F & F & F & T & F \\
F & T & T & F & T & T \\
F & F & T & F & F & T \\
\hline
\end{tabular}
\end{center}

% =========================================================
% Order of Operations
% =========================================================

\subsection*{Order of Operations in Propositional Logic}

The standard precedence of logical connectives (highest to lowest) is:
\begin{enumerate}
  \item Parentheses
  \item Negation ($\neg$)
  \item Conjunction ($\wedge$) and Disjunction ($\vee$)
  \item Conditional ($\rightarrow$) and Biconditional ($\leftrightarrow$)
\end{enumerate}

When connectives have the same precedence, parentheses must be used to avoid ambiguity.
\subsection*{Tautology, Contradiction, and Logical Equivalence}

In propositional logic, formulas may be classified according to their truth
values under all possible truth assignments.

A \textbf{tautology} is a propositional formula that is true under every possible
truth valuation. Tautologies represent statements that are logically true
independently of the truth values of their component propositions. Tautologies
are denoted symbolically by
\[
\top
\]
or, in semantic notation, by writing $\vDash P$ to indicate that $P$ is true in
all interpretations.

A \textbf{contradiction} is a propositional formula that is false under every
possible truth valuation. Contradictions represent statements that cannot be
true under any interpretation. Contradictions are denoted symbolically by
\[
\bot
\]
or, equivalently, by writing $\vDash \neg P$ for a formula that is false in all
interpretations.

Two propositional formulas $P$ and $Q$ are said to be \textbf{logically
equivalent} if they have the same truth value under every possible truth
valuation. Logical equivalence is written symbolically as
\[
P \equiv Q
\]
and indicates that $P$ and $Q$ may be substituted for one another in any logical
context without changing truth value.

% =========================================================
% Inference Rules (Logical Rules)
% =========================================================

\section*{Rules of Inference (Logical / Truth-Preserving)}
\addcontentsline{toc}{section}{Rules of Inference}

\begin{definition}[Modus Ponens — Logical Inference Rule]
\textbf{Modus Ponens}, also called the \emph{Law of Detachment}, states that from a
conditional proposition and its antecedent, the consequent may be inferred.

\[
\begin{array}{ll}
P \rightarrow Q \\
P \\ \hline
Q
\end{array}
\]

This rule is truth-preserving and fundamental to all deductive systems.
\end{definition}

\begin{definition}[Modus Tollens — Logical Inference Rule]
\textbf{Modus Tollens}, also called \emph{denying the consequent}, states that from a
conditional proposition and the negation of its consequent, the negation of the
antecedent may be inferred.

\[
\begin{array}{ll}
P \rightarrow Q \\
\neg Q \\ \hline
\neg P
\end{array}
\]
\end{definition}

\begin{definition}[Hypothetical Syllogism — Logical Inference Rule]
\textbf{Hypothetical Syllogism}, also called the \emph{chain rule}, states that if one
conditional implies a second and the second implies a third, then the first implies the
third.

\[
\begin{array}{ll}
P \rightarrow Q \\
Q \rightarrow R \\ \hline
P \rightarrow R
\end{array}
\]
\end{definition}

\begin{definition}[Disjunctive Syllogism — Logical Inference Rule]
\textbf{Disjunctive Syllogism}, also called \emph{modus tollendo ponens}, states that from
a disjunction and the negation of one disjunct, the other disjunct may be inferred.

\[
\begin{array}{ll}
P \vee Q \\
\neg P \\ \hline
Q
\end{array}
\qquad\text{or}\qquad
\begin{array}{ll}
P \vee Q \\
\neg Q \\ \hline
P
\end{array}
\]
\end{definition}

\begin{definition}[Adjunction — Logical Inference Rule]
\textbf{Adjunction}, also called \emph{Conjunction Introduction}, states that if two
propositions have been established, their conjunction may be inferred.

\[
\begin{array}{ll}
P \\
Q \\ \hline
P \wedge Q
\end{array}
\]
\end{definition}

\begin{definition}[Simplification — Logical Inference Rule]
\textbf{Simplification}, also called \emph{Conjunction Elimination}, states that from a
conjunction, either conjunct may be inferred.

\[
\begin{array}{ll}
P \wedge Q \\ \hline
P
\end{array}
\qquad\text{or}\qquad
\begin{array}{ll}
P \wedge Q \\ \hline
Q
\end{array}
\]
\end{definition}

\begin{definition}[Disjunction Elimination — Logical Inference Rule]
\textbf{Disjunction Elimination}, also called \emph{Proof by Cases}, states that if a
disjunction is true and the same conclusion can be derived from each disjunct, then that
conclusion may be inferred.

\[
\begin{array}{l}
P \vee Q \\
\begin{array}{l}
\text{Assume } P \\ \vdots \\ R
\end{array}
\quad
\begin{array}{l}
\text{Assume } Q \\ \vdots \\ R
\end{array}
\\ \hline
R
\end{array}
\]
\end{definition}

\begin{definition}[Conditional Proof — Structural Rule]
\textbf{Conditional Proof} states that if assuming $P$ leads to $Q$, then the conditional
$P \rightarrow Q$ may be inferred.

\[
\begin{array}{l}
\text{Assume } P \\ \vdots \\ Q \\ \hline
P \rightarrow Q
\end{array}
\]
\end{definition}

\begin{definition}[Indirect Proof — Structural Rule]
\textbf{Indirect Proof}, also called \emph{Reductio ad Absurdum}, states that if assuming
$\neg P$ leads to a contradiction, then $P$ may be inferred.

\[
\begin{array}{l}
\text{Assume } \neg P \\ \vdots \\ \bot \\ \hline
P
\end{array}
\]
\end{definition}

\begin{definition}[Constructive Dilemma — Logical Inference Rule]
The \textbf{Constructive Dilemma} propagates cases forward through conditionals.

\[
\begin{array}{l}
P \vee Q \\
P \rightarrow R \\
Q \rightarrow S \\ \hline
R \vee S
\end{array}
\]
\end{definition}

\begin{definition}[Destructive Dilemma — Logical Inference Rule]
The \textbf{Destructive Dilemma} propagates negated consequents backward.

\[
\begin{array}{l}
P \rightarrow R \\
Q \rightarrow S \\
\neg R \vee \neg S \\ \hline
\neg P \vee \neg Q
\end{array}
\]
\end{definition}

\begin{definition}[Reiteration — Structural Rule]
\textbf{Reiteration} allows any previously derived statement to be restated later in a
proof.

\[
\begin{array}{l}
P \\ \hline
P
\end{array}
\]
\end{definition}

% =========================================================
% Structural / Syntactic Rules
% =========================================================


\begin{definition}[Rule of Premises — Structural Rule]
The \textbf{Rule of Premises} states that any proposition listed as a premise in a
proof may be asserted at any later step of that proof. This rule introduces no
new information and serves only to make assumptions explicit.

\[
\begin{array}{l}
P \\ \hline
P
\end{array}
\]
\end{definition}

\begin{definition}[Reiteration — Structural Rule]
\textbf{Reiteration} permits a statement that has already been derived earlier in
a proof to be restated at a later point. It is used to make prior results
available in new contexts, such as subproofs.

\[
\begin{array}{l}
P \\ \hline
P
\end{array}
\]
\end{definition}

\begin{definition}[Assumption Rule — Structural Rule]
The \textbf{Assumption Rule} allows a proposition to be temporarily introduced
for the purpose of reasoning within a subproof. Assumptions must be discharged
before the proof is completed.

\[
\begin{array}{l}
\text{Assume } P
\end{array}
\]
\end{definition}

\begin{definition}[Subproof Discharge — Structural Rule]
\textbf{Subproof Discharge} marks the closing of a subproof and permits conclusions
derived under an assumption to be used to justify conditionals, negations, or
case reasoning. The assumption is no longer available after discharge.

\[
\begin{array}{l}
\text{Assume } P \\ \vdots \\ Q \\ \hline
P \rightarrow Q
\end{array}
\]
\end{definition}

% =========================================================
% Structural / Syntactic Rules
% =========================================================

\begin{definition}[De Morgan's Laws — Replacement Rule]
\textbf{De Morgan's Laws} describe how negation distributes over conjunction and
disjunction. They allow logically equivalent formulas to replace one another.

\[
\neg (P \wedge Q) \;\equiv\; \neg P \vee \neg Q
\]
\[
\neg (P \vee Q) \;\equiv\; \neg P \wedge \neg Q
\]
\end{definition}

\begin{definition}[Commutation (Commutativity) — Replacement Rule]
\textbf{Commutation} states that the order of propositions joined by conjunction
or disjunction does not affect truth value.

\[
P \wedge Q \;\equiv\; Q \wedge P
\]
\[
P \vee Q \;\equiv\; Q \vee P
\]
\end{definition}

\begin{definition}[Association (Associativity) — Replacement Rule]
\textbf{Association} states that when the same connective is repeated, the placement
of parentheses does not affect logical meaning.

\[
(P \wedge Q) \wedge R \;\equiv\; P \wedge (Q \wedge R)
\]
\[
(P \vee Q) \vee R \;\equiv\; P \vee (Q \vee R)
\]
\end{definition}

\begin{definition}[Distribution — Replacement Rule]
The \textbf{Distributive Laws} describe how conjunction and disjunction distribute
over one another.

\[
P \wedge (Q \vee R) \;\equiv\; (P \wedge Q) \vee (P \wedge R)
\]
\[
P \vee (Q \wedge R) \;\equiv\; (P \vee Q) \wedge (P \vee R)
\]
\end{definition}

\begin{definition}[Double Negation — Replacement Rule]
The \textbf{Double Negation Law} states that a proposition is logically equivalent
to the negation of its negation.

\[
\neg\neg P \;\equiv\; P
\]
\end{definition}

\begin{definition}[Contraposition — Replacement Rule]
\textbf{Contraposition} states that a conditional proposition is logically equivalent
to its contrapositive.

\[
P \rightarrow Q \;\equiv\; \neg Q \rightarrow \neg P
\]
\end{definition}

\begin{definition}[Material Implication — Replacement Rule]
\textbf{Material Implication} rewrites a conditional using negation and disjunction.

\[
P \rightarrow Q \;\equiv\; \neg P \vee Q
\]
\end{definition}

\begin{definition}[Biconditional Expansion — Replacement Rule]
\textbf{Biconditional Expansion} states that a biconditional is equivalent to a
conjunction of two conditionals.

\[
P \leftrightarrow Q \;\equiv\; (P \rightarrow Q) \wedge (Q \rightarrow P)
\]
\end{definition}

\begin{definition}[Exportation / Importation — Replacement Rule]
\textbf{Exportation} and \textbf{Importation} relate conditionals with conjunctions
in the antecedent.

\[
(P \wedge Q) \rightarrow R \;\equiv\; P \rightarrow (Q \rightarrow R)
\]
\end{definition}

\begin{definition}[Idempotent Laws — Replacement Rule]
The \textbf{Idempotent Laws} state that repeating a proposition under conjunction
or disjunction does not change its value.

\[
P \wedge P \;\equiv\; P
\]
\[
P \vee P \;\equiv\; P
\]
\end{definition}

\begin{definition}[Absorption Laws — Replacement Rule]
The \textbf{Absorption Laws} describe how a proposition absorbs a compound containing
itself.

\[
P \wedge (P \vee Q) \;\equiv\; P
\]
\[
P \vee (P \wedge Q) \;\equiv\; P
\]
\end{definition}

\begin{definition}[Identity, Domination, and Negation Laws — Replacement Rules]
These laws describe interactions with truth and falsity.

\[
P \wedge \top \;\equiv\; P \qquad P \vee \bot \;\equiv\; P
\]
\[
P \vee \top \;\equiv\; \top \qquad P \wedge \bot \;\equiv\; \bot
\]
\[
P \vee \neg P \;\equiv\; \top \qquad P \wedge \neg P \;\equiv\; \bot
\]
\end{definition}























