% =========================================================
% Algebra of Sequences
% File: notes-algebra-of-sequences.tex
% =========================================================

\subsection{Algebra of Sequences}

% (Original local heading preserved)
\subsubsection{Algebra of Sequences}

The set of all real sequences is closed under the usual algebraic operations.
Moreover, convergence is preserved under these operations, provided mild
conditions are satisfied.

% ---------------------------------------------------------
\subsubsection{Basic Definitions}

\begin{definition}[Algebraic operations]
Let $(x_n)$ and $(y_n)$ be sequences in $\mathbb{R}$, and let $c\in\mathbb{R}$.
Define new sequences by:
\begin{enumerate}
\item \textbf{Sum:}
\[
(x_n)+(y_n) := (x_n+y_n)
\]

\item \textbf{Difference:}
\[
(x_n)-(y_n) := (x_n-y_n)
\]

\item \textbf{Scalar multiple:}
\[
c(x_n) := (c x_n)
\]

\item \textbf{Product:}
\[
(x_n)(y_n) := (x_n y_n)
\]

\item \textbf{Quotient:}
\[
\frac{(x_n)}{(y_n)} := \left(\frac{x_n}{y_n}\right),
\quad \text{provided } y_n \neq 0 \text{ for all } n.
\]
\end{enumerate}
\end{definition}

% ---------------------------------------------------------
\subsubsection{Main Theorems}

\begin{theorem}[Uniqueness of Limits]
If a sequence $(x_n)$ converges to a real number $L$ and also converges to a real
number $M$, then $L = M$.
\end{theorem}

\begin{proof}
Suppose $(x_n)$ converges to both $L$ and $M$.
Let $\varepsilon > 0$ be arbitrary.

Since $x_n \to L$, there exists $N_1 \in \mathbb{N}$ such that
\[
n \ge N_1 \implies |x_n - L| < \varepsilon/2.
\]

Since $x_n \to M$, there exists $N_2 \in \mathbb{N}$ such that
\[
n \ge N_2 \implies |x_n - M| < \varepsilon/2.
\]

Let $N = \max\{N_1, N_2\}$. Then for all $n \ge N$,
\[
|L - M| \le |L - x_n| + |x_n - M| < \varepsilon.
\]

Since $\varepsilon > 0$ was arbitrary, it follows that $|L - M| = 0$, and hence
$L = M$.
\end{proof}

\begin{theorem}[Order Limit Theorem]
Let $(x_n)$ and $(y_n)$ be sequences of real numbers such that $x_n \le y_n$ for
all $n \in \mathbb{N}$. If $x_n \to x$ and $y_n \to y$, then $x \le y$.
\end{theorem}

\begin{proof}
Assume for contradiction that $x > y$. Set
\[
\varepsilon := \frac{x-y}{2} > 0.
\]
Since $x_n \to x$, there exists $N_1 \in \mathbb{N}$ such that for all $n \ge N_1$,
\[
|x_n - x| < \varepsilon \quad \Longrightarrow \quad x_n > x - \varepsilon
= \frac{x+y}{2}.
\]
Since $y_n \to y$, there exists $N_2 \in \mathbb{N}$ such that for all $n \ge N_2$,
\[
|y_n - y| < \varepsilon \quad \Longrightarrow \quad y_n < y + \varepsilon
= \frac{x+y}{2}.
\]
Let $N := \max\{N_1,N_2\}$. Then for all $n \ge N$ we have
\[
x_n > \frac{x+y}{2} \quad \text{and} \quad y_n < \frac{x+y}{2},
\]
which implies $x_n > y_n$. This contradicts the hypothesis that $x_n \le y_n$
for all $n$. Therefore $x \le y$.
\end{proof}

\begin{theorem}[Squeeze Theorem]
Let $(x_n)$, $(y_n)$, and $(z_n)$ be sequences in $\mathbb{R}$.
Assume that
\[
x_n \le y_n \le z_n
\quad\text{for all } n\in\mathbb{N}.
\]
If $x_n \to L$ and $z_n \to L$, then $y_n \to L$.
\end{theorem}

\begin{proof}
Let $\varepsilon>0$ be arbitrary. Since $x_n\to L$, there exists $N_1\in\mathbb{N}$ such that
for all $n\ge N_1$,
\[
|x_n-L|<\varepsilon \quad\Longrightarrow\quad L-\varepsilon < x_n < L+\varepsilon.
\]
Since $z_n\to L$, there exists $N_2\in\mathbb{N}$ such that for all $n\ge N_2$,
\[
|z_n-L|<\varepsilon \quad\Longrightarrow\quad L-\varepsilon < z_n < L+\varepsilon.
\]
Let $N=\max\{N_1,N_2\}$. Then for all $n\ge N$ we have
\[
L-\varepsilon < x_n \le y_n \le z_n < L+\varepsilon,
\]
which implies $|y_n-L|<\varepsilon$. Since $\varepsilon>0$ was arbitrary, it follows that
$y_n\to L$.
\end{proof}

\begin{theorem}[Algebra of limits]
Let $(x_n)$ and $(y_n)$ be sequences in $\mathbb{R}$ such that
\[
x_n \to x,
\qquad
y_n \to y.
\]
Then:
\begin{enumerate}
\item \textbf{Sum rule}
\[
x_n + y_n \to x + y
\]

\item \textbf{Difference rule}
\[
x_n - y_n \to x - y
\]

\item \textbf{Scalar multiple rule}
\[
c x_n \to c x
\quad \text{for any } c\in\mathbb{R}
\]

\item \textbf{Product rule}
\[
x_n y_n \to xy
\]

\item \textbf{Quotient rule}
If $y\neq 0$ and $y_n\neq 0$ for all sufficiently large $n$, then
\[
\frac{x_n}{y_n} \to \frac{x}{y}.
\]
\end{enumerate}
\end{theorem}

\begin{remark}
Each part of the algebra of limits theorem is proved directly from the
$\varepsilon$–definition of convergence using:
\begin{itemize}
\item the triangle inequality,
\item boundedness of convergent sequences,
\item basic inequalities in $\mathbb{R}$.
\end{itemize}
No appeal to continuity is required.
\end{remark}

\begin{remark}
The quotient rule requires the additional assumption $y\neq 0$.
This ensures that the sequence $(1/y_n)$ is eventually well-defined
and bounded.
\end{remark}

% ---------------------------------------------------------
\subsubsection{Consequences}

The logical implication of this entire section is:

\begin{itemize}
\item Algebraic operations on sequences produce sequences (closure).
\item When sequences converge, their limits interact predictably with
addition, subtraction, scalar multiplication, multiplication, and (with mild
extra hypotheses) division.
\item Order hypotheses are inherited by limits (Order Limit Theorem), and sandwiching
forces convergence (Squeeze Theorem).
\end{itemize}

\begin{corollary}
If $x_n \le b$ for all $n$ and $x_n \to x$, then $x \le b$.
\end{corollary}

\begin{corollary}
If $(x_n)$ converges and $(y_n)$ is bounded, then the product sequence
$(x_n y_n)$ is bounded.
\end{corollary}

\begin{corollary}
If $(x_n)$ converges, then $(|x_n|)$ converges and
\[
|x_n| \to |x|.
\]
\end{corollary}

\begin{remark}
These algebraic rules justify the informal manipulation of limits familiar
from calculus, but their validity rests entirely on the $\varepsilon$–definition
of convergence.
\end{remark}

\begin{remark}[Logical Structure]
The major sequence theorems interlock as follows:
\[
\text{Uniqueness of Limits}
\Rightarrow
\text{Well-definedness of ``the'' limit}
\Rightarrow
\text{Algebra of limits}.
\]
and
\[
\text{Order Limit Theorem}
\Rightarrow
\text{Squeeze Theorem}
\Rightarrow
\text{Many convergence proofs in practice}.
\]
\end{remark}
