% --------------------------------------
% Proof: RA-PON-C02-S2-2-E07C
% --------------------------------------
\clearpage
\phantomsection
\hypertarget{proof-RA-PON-C02-S2-2-E07C}{}

\section*{Proof — RA-PON-C02-S2-2-E07C}

\noindent
\hyperlink{ws-RA-PON-C02-S2-2-E07C}{\textbf{← Back to worksheet}}

\noindent
\textbf{Source.}
\srccite{PonsRealAnalysis}{Chapter~2, Exercise~2.2.7(c)}.

\vspace{0.75em}

\noindent
\textbf{Goal.}
Show that if $(|a_n|)\to 0$, then $(a_n)\to 0$.

\vspace{0.75em}

\begin{proof}
Let $(a_n)$ be a sequence such that $|a_n| \to 0$.

Let $\varepsilon > 0$ be arbitrary.
Then there exists $N \in \mathbb{N}$ such that for all $n \ge N$,
\[
\bigl||a_n| - 0\bigr| < \varepsilon.
\]

But $\bigl||a_n| - 0\bigr| = |a_n|$, so for all $n \ge N$,
\[
|a_n| < \varepsilon.
\]

Since $|a_n - 0| = |a_n|$, we conclude that
\[
|a_n - 0| < \varepsilon
\]
for all $n \ge N$.

Therefore, $a_n \to 0$.
\end{proof}



\vspace{1em}
\noindent
\hyperlink{ws-RA-PON-C02-S2-2-E07C}{\textbf{← Back to worksheet}}

