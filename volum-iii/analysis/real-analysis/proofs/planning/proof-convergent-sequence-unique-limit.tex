\newpage

% =========================================================
% Proof Planning Worksheet (Quantifier-Aware Version)
% Theorem: A convergent sequence has a unique limit
% =========================================================

\section*{Proof Planning Worksheet}

\textbf{Theorem (as written).}  
A convergent sequence has a unique limit.

\bigskip

% ---------------------------------------------------------
\subsection*{1. Quantifier Detection (Natural Language Analysis)}

Rewrite the statement in explicit English,
making all hidden quantifiers visible.

\begin{itemize}
    \item Is the phrase generic or existential?
    \item What quantifier does it encode?
    \item Is the statement conditional?
    \item Is the statement conjunctive/disjunctive or compound?
\end{itemize}

Write the expanded English sentence below: \\
 \\
For every sequence, if the sequence converges, then there exists a real number L such that the sequence converges to L,
and if the sequence also converges to some real number M, then M=L.
\vspace{3cm}
\hrule
\vspace{1cm}

% ---------------------------------------------------------
\subsection*{2. Logical Skeleton (Propositional Form)}

Abstract the theorem into propositional structure.

Let:

\[
C(x) := \text{“$x$ is a convergent sequence”}
\]
\[
C(x) := \exists \ell \; L(x,\ell).
\]

\[
Q(x) := \text{“$x$ has at most one limit.”}
\]
\[
Q(x) :=
\forall \ell_1 \forall \ell_2
\bigl(
L(x,\ell_1) \land L(x,\ell_2)
\rightarrow
\ell_1 = \ell_2
\bigr).
\]
Rewrite the theorem schematically:

\vspace{2cm}
\hrule
\vspace{1cm}

% ---------------------------------------------------------
\subsection*{3. Predicate-Level Formalization}

Now rewrite the theorem with explicit quantifiers:

\vspace{3cm}
\hrule
\vspace{1cm}

% ---------------------------------------------------------
\subsection*{4. Expand the Meaning of “Unique”}

Write the full logical expansion of:

\[
\exists! L \text{ such that } P(L)
\]

\vspace{3cm}
\hrule
\vspace{1cm}

% ---------------------------------------------------------
\subsection*{5. Separate Existence from Uniqueness}

Is existence already contained in the hypothesis?
Or must it be proved?

Explain clearly which part of uniqueness remains to be shown:

\vspace{3cm}
\hrule
\vspace{1cm}

% ---------------------------------------------------------
\subsection*{6. Structural Decomposition of the Proof}

Before writing any proof, answer:

\begin{itemize}
    \item What object is fixed?
    \item What hypothesis is assumed?
    \item What must ultimately be shown?
    \item What equality (if any) must be derived?
\end{itemize}

Write these in clean logical phrases (not narrative prose):

\vspace{4cm}
\hrule
\vspace{1cm}

% ---------------------------------------------------------
\subsection*{7. Quantifier Expansion of Definitions}

Write the full quantified definition of:

\begin{itemize}
    \item Convergence
    \item Limit
\end{itemize}

Be explicit about:

\[
\forall \varepsilon \;
\exists N \;
\forall n \ge N
\]

\vspace{5cm}
\hrule
\vspace{1cm}

% ---------------------------------------------------------
\subsection*{8. Scope and Dependency Tracking}

For each quantifier appearing in the theorem or definitions:

\begin{itemize}
    \item Identify what it ranges over.
    \item Identify what it may depend on.
    \item Identify which step of the proof will eliminate it.
\end{itemize}

\vspace{4cm}
\hrule

\newpage
\begin{theorem}
A convergent sequence has a unique limit.
\end{theorem}

\begin{proof}
\noindent
\begin{minipage}[t]{0.48\textwidth}
\textbf{Symmetric Form}

\medskip

Let $(x_n)$ be a convergent sequence.

Assume $x_n \to \ell_1$ and $x_n \to \ell_2$.

Let $\varepsilon > 0$ be arbitrary.

From $x_n \to \ell_1$, there exists $N_1$ such that
\[
n \ge N_1 \Rightarrow |x_n - \ell_1| < \varepsilon/2.
\]

From $x_n \to \ell_2$, there exists $N_2$ such that
\[
n \ge N_2 \Rightarrow |x_n - \ell_2| < \varepsilon/2.
\]

Let $N = \max\{N_1, N_2\}$.

Then for $n \ge N$,
\[
|\ell_1 - \ell_2|
\le |\ell_1 - x_n| + |x_n - \ell_2|
< \varepsilon/2 + \varepsilon/2
= \varepsilon.
\]

Since $\varepsilon > 0$ was arbitrary,
$|\ell_1 - \ell_2| = 0$, hence
\[
\ell_1 = \ell_2.
\]

\medskip

Thus the limit is unique.
\end{minipage}
\hfill
\begin{minipage}[t]{0.48\textwidth}
\textbf{Witness-Based Form}

\medskip

Let $(x_n)$ be a convergent sequence.

Then there exists $L_c$ such that
\[
x_n \to L_c.
\]

Let $m$ be any real number and assume
\[
x_n \to m.
\]

Let $\varepsilon > 0$ be arbitrary.

From $x_n \to L_c$, there exists $N_1$ such that
\[
n \ge N_1 \Rightarrow |x_n - L_c| < \varepsilon/2.
\]

From $x_n \to m$, there exists $N_2$ such that
\[
n \ge N_2 \Rightarrow |x_n - m| < \varepsilon/2.
\]

Let $N = \max\{N_1, N_2\}$.

Then for $n \ge N$,
\[
|m - L_c|
\le |m - x_n| + |x_n - L_c|
< \varepsilon/2 + \varepsilon/2
= \varepsilon.
\]

Since $\varepsilon > 0$ was arbitrary,
$|m - L_c| = 0$, hence
\[
m = L_c.
\]

\medskip

Thus every limit equals $L_c$, so the limit is unique.
\end{minipage}
\end{proof}
\newpage
