% Proof: Scalar Multiplication by Zero and Negation
% Three-column format: Tag / Step / Justification
%
% Column widths (must sum to <= \textwidth, leaving room for inter-column padding):
%   Tag:           0.07\textwidth  — short label
%   Step:          0.36\textwidth  — math, centered via >{\centering\arraybackslash}
%   Justification: 0.50\textwidth  — wrapped prose

\clearpage
\begin{center}
  {\large\bfseries Proof: Scalar Multiplication by Zero and Negation}\\[4pt]
  {\small\itshape (Three-column format: tag / step / justification)}\\[6pt]
  {\small
    \tagDU\ = Definition Unpacked \quad
    \tagTA\ = Theorem Applied \quad
    \tagAM\ = Algebraic Manipulation
  }
\end{center}

\bigskip

% ------------------------------------------------------------------
\noindent\textit{Part 1: $0_{\mathbb{F}} v = \mathbf{0}$.}
\medskip

\noindent
\begin{tabular}{p{0.07\textwidth}
                >{\centering\arraybackslash}p{0.36\textwidth}
                p{0.50\textwidth}}
\toprule
\textbf{Tag} & \textbf{Step} & \textbf{Justification} \\
\midrule
\addlinespace[4pt]

\tagAM
&
$0_{\mathbb{F}} v = (0_{\mathbb{F}} + 0_{\mathbb{F}}) v
= 0_{\mathbb{F}} v + 0_{\mathbb{F}} v$
&
$0_{\mathbb{F}} + 0_{\mathbb{F}} = 0_{\mathbb{F}}$ in $\mathbb{F}$;
distributivity over scalar addition.
\\[10pt]

\tagTA
&
$\mathbf{0} = 0_{\mathbb{F}} v$
&
Add $-(0_{\mathbb{F}} v)$ to both sides; uniqueness of the
additive inverse collapses the right side.
\hfill$\square$
\\[6pt]

\bottomrule
\end{tabular}

\bigskip

% ------------------------------------------------------------------
\noindent\textit{Part 2: $\alpha \mathbf{0} = \mathbf{0}$.}
\medskip

\noindent
\begin{tabular}{p{0.07\textwidth}
                >{\centering\arraybackslash}p{0.36\textwidth}
                p{0.50\textwidth}}
\toprule
\textbf{Tag} & \textbf{Step} & \textbf{Justification} \\
\midrule
\addlinespace[4pt]

\tagAM
&
$\alpha \mathbf{0} = \alpha(\mathbf{0} + \mathbf{0})
= \alpha \mathbf{0} + \alpha \mathbf{0}$
&
$\mathbf{0} + \mathbf{0} = \mathbf{0}$ in $V$;
distributivity over vector addition.
\\[10pt]

\tagTA
&
$\mathbf{0} = \alpha \mathbf{0}$
&
Add $-(\alpha \mathbf{0})$ to both sides; uniqueness of the
additive inverse collapses the right side.
\hfill$\square$
\\[6pt]

\bottomrule
\end{tabular}

\bigskip

% ------------------------------------------------------------------
\noindent\textit{Part 3: $(-1_{\mathbb{F}}) v = -v$.}
\medskip

\noindent
\begin{tabular}{p{0.07\textwidth}
                >{\centering\arraybackslash}p{0.36\textwidth}
                p{0.50\textwidth}}
\toprule
\textbf{Tag} & \textbf{Step} & \textbf{Justification} \\
\midrule
\addlinespace[4pt]

\tagDU
&
$\text{Show } v + (-1_{\mathbb{F}})v = \mathbf{0}$
&
By uniqueness of the additive inverse, it suffices to
verify this characterizing property.
\\[10pt]

\tagAM
&
$v + (-1_{\mathbb{F}})v
= 1_{\mathbb{F}} v + (-1_{\mathbb{F}}) v
= (1_{\mathbb{F}} + (-1_{\mathbb{F}})) v$
&
Unit law; distributivity over scalar addition.
\\[10pt]

\tagTA
&
$(1_{\mathbb{F}} + (-1_{\mathbb{F}})) v = 0_{\mathbb{F}} v = \mathbf{0}$
&
$1_{\mathbb{F}} + (-1_{\mathbb{F}}) = 0_{\mathbb{F}}$ in $\mathbb{F}$;
Part~1.
\\[10pt]

\tagDU
&
$\therefore\; (-1_{\mathbb{F}}) v = -v$
&
Uniqueness of the additive inverse (Proposition above).
\hfill$\blacksquare$
\\[6pt]

\bottomrule
\end{tabular}

\bigskip\bigskip
\noindent\hrule
\medskip
\noindent\textbf{Study Notes.}
\medskip

\noindent
\begin{tabular}{p{0.22\textwidth} p{0.72\textwidth}}
\toprule
\textbf{Tool} & \textbf{Role in this proof} \\
\midrule
\addlinespace[4pt]
Self-cancellation trick
&
Parts 1 and 2 use the same strategy: write zero as a sum of two zeros,
distribute, then cancel. The argument runs in two directions ---
once through scalar addition, once through vector addition.
\\[8pt]
Characterization by property
&
Part 3 avoids direct manipulation by instead verifying the
defining property of $-v$, then invoking uniqueness.
This is cleaner and more structurally transparent.
\\[8pt]
Uniqueness results
&
Both the uniqueness of the additive identity and the uniqueness of the
additive inverse are essential; without them the cancellation steps
in Parts 1--2 and the conclusion of Part 3 would not be justified.
\\[6pt]
\bottomrule
\end{tabular}