% =========================================================
% Abelian Groups and Fields
% =========================================================

% ---------------------------------------------------------
\subsubsection{Preliminary Definitions}
% ---------------------------------------------------------

\begin{definition}[Abelian group]
A set $G$ with a binary operation $\ast : G \times G \to G$
is an \emph{abelian group} if:

\begin{enumerate}
\item \textbf{Associativity:}
\[
(a \ast b) \ast c = a \ast (b \ast c)
\quad\text{for all } a,b,c \in G.
\]

\item \textbf{Identity element:}
There exists $e \in G$ such that
\[
a \ast e = e \ast a = a
\quad\text{for all } a \in G.
\]

\item \textbf{Inverse element:}
For each $a \in G$ there exists $b \in G$ such that
\[
a \ast b = b \ast a = e.
\]

\item \textbf{Commutativity:}
\[
a \ast b = b \ast a
\quad\text{for all } a,b \in G.
\]
\end{enumerate}
\end{definition}

\begin{remark}
An abelian group is simply a group whose operation is commutative.
\end{remark}

\begin{definition}[Field --- structural form]
A \emph{field} is a set $\mathbb{F}$ equipped with two operations
$+$ and $\cdot$ such that:

\begin{itemize}
\item $(\mathbb{F}, +)$ is an abelian group,
\item $(\mathbb{F}\setminus\{0\}, \cdot)$ is an abelian group,
\item Multiplication distributes over addition.
\end{itemize}
\end{definition}

\begin{remark}[Expanded Axioms]
Expanding the abelian group axioms yields the familiar list:

\begin{itemize}
\item Associativity, commutativity, identity, and inverse for addition.
\item Associativity, commutativity, identity, and inverse (for nonzero elements)
      for multiplication.
\item The distributive law.
\end{itemize}

Thus the usual field axioms are not independent assumptions,
but consequences of the two abelian group structures together with distributivity.
\end{remark}

\begin{remark}[Structural Summary]
A field consists of two compatible abelian group structures:

\begin{center}
\begin{tabular}{l|c|c}
Structure & Additive Group & Multiplicative Structure \\
\hline
Field & Abelian & Abelian (on nonzero elements)
\end{tabular}
\end{center}

Distributivity links the two operations.
\end{remark}