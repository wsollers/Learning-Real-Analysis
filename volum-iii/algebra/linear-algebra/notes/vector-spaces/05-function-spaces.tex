% =========================================================
% Function Spaces
% =========================================================

\subsubsection{The Function-Space Viewpoint}

\begin{definition}[Set Exponentiation]
Let $X$ and $Y$ be sets. Define
\[
Y^X := \{ f : X \to Y \},
\]
the set of all functions from $X$ to $Y$.
\end{definition}

\begin{remark}
If $|X| = n$ and $|Y| = m$ are finite, then $|Y^X| = m^n$.
Thus the notation $Y^X$ agrees with the rules of cardinal arithmetic.
\end{remark}

\begin{definition}[$\mathbb{F}^S$ --- the function space over a set]
Let $\mathbb{F}$ be a field and let $S$ be a set. Define
\[
\mathbb{F}^S := \{ f : S \to \mathbb{F} \},
\]
the set of all functions from $S$ to $\mathbb{F}$, with \textbf{pointwise} operations:
\[
(f + g)(x) := f(x) + g(x),
\qquad
(\lambda f)(x) := \lambda f(x),
\]
for all $f,g \in \mathbb{F}^S$, $\lambda \in \mathbb{F}$, and $x \in S$.
\end{definition}

\begin{example}[$\mathbb{F}^S$ is a vector space]
With the pointwise operations above, $\mathbb{F}^S$ is a vector space over $\mathbb{F}$.
The zero vector is $\mathbf{0}(x) = 0$ for all $x \in S$,
and the additive inverse of $f$ is $(-f)(x) = -f(x)$ for all $x \in S$.
All vector space axioms follow from the field axioms applied pointwise; for example,
\[
((f+g)+h)(x) = f(x)+g(x)+h(x) = (f+(g+h))(x),
\]
so associativity holds because addition in $\mathbb{F}$ is associative.
Every other axiom follows similarly.
\end{example}

\begin{remark}[Zero vector in $\mathbb{F}^S$]
In the function space $\mathbb{F}^S$, the zero vector is the function
\[
\mathbf{0} : S \to \mathbb{F}
\quad \text{defined by} \quad
\mathbf{0}(x) = 0 \text{ for all } x \in S.
\]
\end{remark}

% ---------------------------------------------------------
\subsubsection{Coordinate Spaces as Function Spaces}

\begin{definition}[Coordinate space as a function space]
Let $\mathbb{F}$ be a field and let $[n] := \{1,\dots,n\}$. Then
\[
\mathbb{F}^n \;=\; \mathbb{F}^{[n]} \;=\; \{ f : [n] \to \mathbb{F} \}.
\]
An element $v \in \mathbb{F}^n$ is a function $v : [n] \to \mathbb{F}$,
written $v = (v_1,\dots,v_n)$ with $v_i := v(i)$.
Thus an $n$-tuple is simply a function from a finite index set,
and $\mathbb{F}^n$ inherits its vector space structure as a special case of $\mathbb{F}^S$.
\end{definition}

\begin{remark}
With the componentwise operations defined above, $\mathbb{F}^n$ is a vector space
over $\mathbb{F}$ for any field $\mathbb{F}$ and any $n \in \mathbb{N}$.
\end{remark}

% ---------------------------------------------------------
\subsubsection{The Space $\mathbb{F}^{\infty}$}

\begin{definition}[$\mathbb{F}^{\infty}$ --- the space of sequences]
Let $\mathbb{F}$ be a field. Define
\[
\mathbb{F}^{\infty}
:= \mathbb{F}^{\mathbb{N}}
= \{ f : \mathbb{N} \to \mathbb{F} \}
= \{ (x_1,x_2,\dots) : x_k \in \mathbb{F} \text{ for all } k \in \mathbb{N} \}.
\]
An element of $\mathbb{F}^{\infty}$ is called a \emph{sequence} of elements of $\mathbb{F}$.
With pointwise operations inherited from $\mathbb{F}^S$,
$\mathbb{F}^{\infty}$ is a vector space over $\mathbb{F}$.
\end{definition}

\begin{remark}[Conceptual Unification]
The passage
\[
\mathbb{F}^n
\quad\longrightarrow\quad
\mathbb{F}^{\mathbb{N}}
\quad\longrightarrow\quad
\mathbb{F}^S
\]
shows that finite tuples, sequences, and general function spaces are all instances
of the same construction. Finite-dimensional linear algebra is the special case
where the index set is finite, and this viewpoint connects directly to
infinite-dimensional spaces and functional analysis.
\end{remark}