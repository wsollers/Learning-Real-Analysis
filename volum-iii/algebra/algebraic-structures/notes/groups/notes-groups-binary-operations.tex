% =========================================================
% Binary Operations
% =========================================================
\paragraph{Binary Operations}

\begin{definition}[Binary Operation]
Let $G$ be a set.
A \emph{binary operation} on $G$ is a function
\[
\star : G \times G \to G.
\]
For $a, b \in G$, we write $a \star b$ for the image of $(a,b)$ under $\star$.
\end{definition}

\begin{remark}
The codomain of $\star$ is $G$ itself. This means that for any $a, b \in G$,
the result $a \star b$ is again an element of $G$.
This property is called \emph{closure} and is built into the definition of a
binary operation.

Intuitively: a binary operation takes two elements of a set and produces
a third element of the \emph{same} set.
\end{remark}

\begin{definition}[Associativity]
A binary operation $\star$ on $G$ is \emph{associative} if
\[
(a \star b) \star c = a \star (b \star c)
\quad \text{for all } a, b, c \in G.
\]
\end{definition}

\begin{definition}[Commutativity]
A binary operation $\star$ on $G$ is \emph{commutative} if
\[
a \star b = b \star a
\quad \text{for all } a, b \in G.
\]
\end{definition}

\begin{remark}
Associativity and commutativity are independent properties.
Addition on $\mathbb{Z}$ is both; matrix multiplication is associative
but not commutative; subtraction on $\mathbb{Z}$ is neither.
\end{remark}