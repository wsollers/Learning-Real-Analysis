% =========================================================
% Induction on the integers
% =========================================================

\subsection{Induction}

% ---------------------------------------------------------
\subsubsection{Definitions and Theorems}
% ---------------------------------------------------------

\begin{definition}[Principle of Mathematical Induction]
Let $a \in \mathbb{Z}$ and let $S \subseteq \mathbb{Z}$ be a set of integers
such that $a \in S$.

Suppose $S$ has the property that whenever an integer $n \ge a$
belongs to $S$, then $n+1$ also belongs to $S$.
That is,
\[
n \in S \;\Rightarrow\; n+1 \in S
\quad \text{for all } n \ge a.
\]

Then $S$ contains every integer greater than or equal to $a$.
In other words,
\[
\{ n \in \mathbb{Z} : n \ge a \} \subseteq S.
\]
\end{definition}

\begin{definition}[Mathematical Induction — Logical Form]
Let $a \in \mathbb{Z}$ and let $P(n)$ be a predicate defined
for integers $n \ge a$.

Suppose:

\begin{enumerate}
\item \textbf{Base case:}
\[
P(a) \text{ is true.}
\]

\item \textbf{Inductive step:}
\[
\forall n \ge a \;
\bigl(P(n) \Rightarrow P(n+1)\bigr).
\]
\end{enumerate}

Then
\[
\forall n \ge a,\; P(n) \text{ is true.}
\]
\end{definition}

\begin{remark}[Equivalence with the Well-Ordering Principle]
The Principle of Mathematical Induction is logically equivalent
to the Well-Ordering Principle.

\medskip

\textbf{Equivalence idea:}

\begin{itemize}
\item \emph{Well-Ordering $\Rightarrow$ Induction:}
If a statement $P(n)$ were false for some $n \ge a$,
then the set of counterexamples would be nonempty.
By the Well-Ordering Principle, it would contain a smallest element.
This smallest counterexample contradicts the inductive hypothesis.

\item \emph{Induction $\Rightarrow$ Well-Ordering:}
If a nonempty set of integers had no smallest element,
one can use induction to show that no integer belongs to it,
yielding a contradiction.
\end{itemize}

Thus induction and well-ordering are two formulations
of the same fundamental structural property of the integers.
\end{remark}

\begin{remark}[Logical Structure of Induction]
The Principle of Mathematical Induction has the quantifier pattern
\[
\text{(initial truth)}
\;\wedge\;
\text{(truth propagates)}
\;\Rightarrow\;
\text{(universal truth)}.
\]

More precisely,
\[
\bigl(P(a)
\;\wedge\;
\forall n \ge a\, (P(n) \Rightarrow P(n+1))\bigr)
\;\Rightarrow\;
\forall n \ge a\, P(n).
\]

Thus induction is a mechanism for converting a \emph{local implication}
($P(n) \Rightarrow P(n+1)$) into a \emph{global conclusion}
($P(n)$ holds for all $n \ge a$).

It upgrades step-by-step propagation into a universal statement.
\end{remark}

\begin{theorem}[Second Principle of Mathematical Induction (Strong Induction)]
Let $a \in \mathbb{Z}$ and let $S \subseteq \mathbb{Z}$ be a set of integers
such that $a \in S$.

Suppose $S$ has the following property:
for every integer $n \ge a$,
\[
\bigl(\forall k \in \mathbb{Z}\; (a \le k < n \Rightarrow k \in S)\bigr)
\;\Rightarrow\;
n \in S.
\]
Then
\[
\{\, n \in \mathbb{Z} : n \ge a \,\} \subseteq S,
\]
i.e., $S$ contains every integer greater than or equal to $a$.
\end{theorem}

\begin{remark}[Weak vs.\ Strong Induction]
There are two common formulations of induction:

\medskip
\textbf{Weak (ordinary) induction:}
\[
P(a)
\;\wedge\;
\forall n \ge a\, (P(n) \Rightarrow P(n+1))
\;\Rightarrow\;
\forall n \ge a\, P(n).
\]

\textbf{Strong induction:}
\[
P(a)
\;\wedge\;
\forall n > a\,
\Bigl(\bigl(\forall k \ (a \le k < n \Rightarrow P(k))\bigr)
\Rightarrow P(n)\Bigr)
\;\Rightarrow\;
\forall n \ge a\, P(n).
\]

\medskip
\textbf{Difference:}
Weak induction assumes only the immediately preceding case $P(n)$
to prove $P(n+1)$.
Strong induction assumes all earlier cases $P(a),\dots,P(n-1)$
to prove $P(n)$.

\medskip
\textbf{Use cases:}
\begin{itemize}
\item Weak induction is natural when the proof of $P(n+1)$
depends only on $P(n)$.
\item Strong induction is appropriate when the proof of $P(n)$
requires information about several earlier values,
such as in factorization arguments (e.g., the Fundamental
Theorem of Arithmetic) or recursive definitions.
\end{itemize}

\medskip
Although they appear different, weak and strong induction
are logically equivalent; each can be derived from the other.
They are two formulations of the same structural property
of the integers.
\end{remark}




% ---------------------------------------------------------
\subsubsection{Consequences}
% ---------------------------------------------------------

\begin{remark}[Logical Equivalence]
The following principles are logically equivalent:

\begin{itemize}
\item The Well-Ordering Principle.
\item The Principle of Mathematical Induction.
\item The Second (Strong) Principle of Mathematical Induction.
\item The Least Element Principle.
\end{itemize}

Each encodes the same structural property of the integers:
every nonempty subset of $\mathbb{N}$ has a minimal element,
and there are no infinite strictly descending chains.
\end{remark}

\begin{remark}[Induction as Minimal Counterexample Argument]
Induction may be reformulated as a \emph{minimal counterexample} principle.

To prove a statement $P(n)$ for all $n \ge a$,
it suffices to assume that a counterexample exists,
choose the smallest such counterexample using well-ordering,
and derive a contradiction.

Thus induction and minimal-counterexample arguments
are two views of the same logical engine.
\end{remark}

\begin{remark}[Recursive Definitions]
Induction justifies recursive constructions.

If a function or object is defined by:
\begin{itemize}
\item specifying its value at $a$, and
\item specifying how to construct its value at $n+1$
from its value at $n$,
\end{itemize}
then induction guarantees that the definition
extends uniquely to all $n \ge a$.

Thus induction underlies the construction of:
\begin{itemize}
\item exponentiation,
\item factorials,
\item recursively defined sequences,
\item algorithms on the integers.
\end{itemize}
\end{remark}

\begin{remark}[Structural Position]
Induction is not merely a proof technique;
it characterizes the order structure of the integers.

It upgrades a local propagation rule
\[
P(n) \Rightarrow P(n+1)
\]
into a global universal conclusion
\[
\forall n \ge a\, P(n).
\]

This mechanism is the prototype for:
\begin{itemize}
\item Euclidean-domain arguments (minimal element methods),
\item termination proofs in algorithms,
\item structural recursion in algebra and logic.
\end{itemize}
\end{remark}

\begin{remark}[Failure Outside Well-Ordered Sets]
Induction depends essentially on the well-ordering of $\mathbb{N}$.
In sets that admit infinite descending chains
(such as $\mathbb{Z}$ under the usual order),
induction in this form fails.

Thus induction is a structural consequence
of well-ordering, not a purely algebraic property.
\end{remark}

\begin{remark}[Logical Flow]
\[
\text{Well-Ordering}
\;\Longleftrightarrow\;
\text{Induction}
\;\Longleftrightarrow\;
\text{Strong Induction}
\]

These are different formulations of the same foundational
order-theoretic principle governing the integers.
\end{remark}




































% ---------------------------------------------------------
\subsubsection{Consequences}
% ---------------------------------------------------------

\begin{corollary}
<Immediate consequence.>
\end{corollary}

\begin{proof}
<Short proof if needed.>
\end{proof}

\begin{remark}[Structural Insight]
<Explain what this section reveals about the structure.>
\end{remark}

\begin{remark}[Logical Structure]
\[
\text{Local Definitions + Theorems}
\;\Rightarrow\;
\text{Structural Consequences}.
\]
\end{remark}
