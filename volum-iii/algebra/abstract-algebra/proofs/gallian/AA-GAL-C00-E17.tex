% --------------------------------------
% Proof: AA-GAL-C00-E17
% --------------------------------------
\clearpage
\phantomsection
\hypertarget{proof-AA-GAL-C00-E17}{}

\section*{Proof — AA-GAL-C00-E17}

\noindent
\hyperlink{ws-AA-GAL-C00-E17}{\textbf{← Back to worksheet}}

\noindent
\textbf{Source.}
\srccite{GallianCAA7}{Chapter~0, Exercise~17}.

\vspace{0.75em}

\noindent
\textbf{Goal.}
Let $a,b,s,t$ be integers.
If
\[
a \equiv b \pmod{st},
\]
show that
\[
a \equiv b \pmod{s}
\quad \text{and} \quad
a \equiv b \pmod{t}.
\]
Determine the condition on $s$ and $t$ that makes the converse true.

\vspace{0.75em}

\noindent
\textbf{Proof.}

\begin{proof}
Assume $a \equiv b \pmod{st}$, so $st \mid (a - b)$.
Then $a - b = k(st)$ for some integer $k$. We can rewrite this as
\[
a - b = (kt)s = (ks)t.
\]
Since $kt$ and $ks$ are integers, we have $s \mid (a - b)$ and $t \mid (a - b)$,
hence $a \equiv b \pmod{s}$ and $a \equiv b \pmod{t}$.
\end{proof}

\begin{proof}
Assume $a \equiv b \pmod{st}$. Then there exists a common remainder $r$ such that
\begin{align*}
a &= q_1(st) + r \\
b &= q_2(st) + r.
\end{align*}
Since $q_1(st) = (q_1 t)s$ and $q_2(st) = (q_2 t)s$ are multiples of $s$,
\begin{align*}
(q_1(st)) \bmod s &= 0 \\
(q_2(st)) \bmod s &= 0,
\end{align*}
and therefore $a \bmod s = r \bmod s = b \bmod s$, which gives $a \equiv b \pmod{s}$.
The same argument with $t$ in place of $s$ gives $a \equiv b \pmod{t}$.
\end{proof}

\vspace{2cm}
\noindent
\hyperlink{ws-AA-GAL-C00-E17}{\textbf{← Back to worksheet}}
