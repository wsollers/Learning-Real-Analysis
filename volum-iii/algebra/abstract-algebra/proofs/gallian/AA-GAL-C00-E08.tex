% --------------------------------------
% Proof: AA-GAL-C00-E08
% --------------------------------------
\clearpage
\phantomsection
\hypertarget{proof-AA-GAL-C00-E08}{}

\section*{Proof — AA-GAL-C00-E08}

\noindent
\hyperlink{ws-AA-GAL-C00-E08}{\textbf{← Back to worksheet}}

\noindent
\textbf{Source.}
\srccite{GallianCAA7}{Chapter~0, Exercise~8}.

\vspace{0.75em}

\noindent
\textbf{Goal.}
Suppose $a$ and $b$ are integers that divide the integer $c$.
If $a$ and $b$ are relatively prime, show that $ab$ divides $c$.
Give an example showing that if $a$ and $b$ are not relatively prime,
then $ab$ need not divide $c$.

\vspace{0.75em}

\noindent

\begin{theorem*}
Let $a$, $b$, and $c$ be integers. Suppose $a \mid c$ and $b \mid c$.
If $\gcd(a,b) = 1$, then $ab \mid c$.
\end{theorem*}

\begin{proof}
Since $\gcd(a,b) = 1$, Bézout's Identity gives integers $s$ and $t$ such that
\[
  as + bt = 1.
\]
Multiplying both sides by $c$ gives
\[
  cas + cbt = c.
\]
Since $b \mid c$, we may write $c = bk$ for some integer $k$, so
\[
  cas = a(bk)s = ab(ks),
\]
and hence $ab \mid cas$. Since $a \mid c$, we may write $c = am$ for some integer $m$, so
\[
  cbt = (am)bt = ab(mt),
\]
and hence $ab \mid cbt$. Since $ab$ divides both $cas$ and $cbt$, it divides their sum:
\[
  ab \mid cas + cbt = c.
\]
Therefore $ab \mid c$.
\end{proof}


\vspace{2cm}
\noindent
\hyperlink{ws-AA-GAL-C00-E08}{\textbf{← Back to worksheet}}
