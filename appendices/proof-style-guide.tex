
% ============================================================
% Proof Style Guide (Modern Mathematical Practice)
% ============================================================

\chapter*{Proof Style Guide}
\addcontentsline{toc}{chapter}{Proof Style Guide}

\section*{Purpose and Scope}

This guide specifies the standards for writing mathematical proofs used in
modern mathematics, as practiced in universities, professional organizations,
and peer-reviewed publications.  
It applies to all proofs in this document unless explicitly stated otherwise.

A mathematical proof is not a record of exploration or computation.
It is a logically complete, human-readable argument whose purpose is to
convince a mathematically trained reader that a statement follows
inevitably from accepted definitions, axioms, and previously established results.

\section*{1. General Principles}

\begin{itemize}
  \item Proofs must be written in complete, grammatical sentences.
  \item Mathematical symbols support prose; they do not replace it.
  \item Every assertion must be justified by a definition, assumption,
        or previously established result.
  \item Clarity and correctness take priority over brevity.
\end{itemize}

\section*{2. Structure of a Proof}

A well-written proof typically has the following structure:

\begin{enumerate}
  \item \textbf{Context and setup}: identification of hypotheses and objects.
  \item \textbf{Logical development}: the argument itself.
  \item \textbf{Conclusion}: an explicit statement that the goal has been achieved.
\end{enumerate}

Longer or more complex proofs should additionally include a brief statement
of strategy.

\section*{3. Use of Language and Symbols}

\subsection*{3.1 Sentences and Grammar}

Proofs must be written as coherent mathematical prose.
Lists of equations or implications without explanation are unacceptable.

\medskip
\noindent
\textbf{Acceptable:}
\begin{quote}
Let $\varepsilon > 0$ be arbitrary. Choose $\delta > 0$ such that
$|x-a|<\delta$ implies $|f(x)-f(a)|<\varepsilon$.
\end{quote}

\noindent
\textbf{Unacceptable:}
\begin{quote}
$\forall \varepsilon > 0\ \exists \delta > 0:\ |x-a|<\delta \Rightarrow |f(x)-f(a)|<\varepsilon$.
\end{quote}

\subsection*{3.2 Symbol Discipline}

\begin{itemize}
  \item Every symbol must be defined before it is used.
  \item Each symbol must have a single, consistent meaning within a proof.
  \item Excessive symbolic shorthand should be avoided.
\end{itemize}

\section*{4. Definitions and Assumptions}

Definitions are the primary tools of proof writing.

\begin{itemize}
  \item When a definition is used in a nontrivial way, it should be explicitly invoked.
  \item Standard properties (e.g.\ reflexivity of an equivalence relation) may be cited
        without restating the full definition, provided they are used correctly.
\end{itemize}

\section*{5. Quantifiers and Logical Control}

Proper handling of quantifiers is essential.

\begin{itemize}
  \item Arbitrary objects must be explicitly declared as such.
  \item Objects whose existence is asserted must not be used before their existence
        has been established.
  \item Dependencies between variables must be logically correct and clearly stated.
\end{itemize}

\medskip
\noindent
\textbf{Example:}
\begin{quote}
Let $\varepsilon > 0$ be arbitrary. Choose $\delta > 0$ such that \dots
\end{quote}

\section*{6. Common Proof Techniques}

\subsection*{6.1 Direct Proof}

This is the default method unless another approach is more natural.
Most proofs in analysis, algebra, and topology are written directly.

\subsection*{6.2 Proof by Contrapositive}

When proving an implication, the contrapositive may be proved instead.
This must be stated explicitly.

\subsection*{6.3 Proof by Contradiction}

When using contradiction, the assumed negation and the resulting contradiction
must be clearly identified.

\subsection*{6.4 Mathematical Induction}

Proofs by induction must include:
\begin{itemize}
  \item a base case,
  \item a clearly stated inductive hypothesis,
  \item a valid inductive step,
  \item an explicit conclusion.
\end{itemize}

\section*{7. Referencing Prior Results}

Previously established theorems, lemmas, and propositions may be cited by name
or label. Results may not be used before they have been proved.

Circular reasoning is strictly prohibited.

\section*{8. Clarity and Professional Tone}

\begin{itemize}
  \item Avoid phrases such as ``it is obvious'' or ``clearly'' unless the claim is truly trivial.
  \item If a step would require thought from a trained reader, it requires justification.
  \item Proofs should be readable without external explanation.
\end{itemize}

\section*{9. Ending a Proof}

A proof must end with an explicit indication that the desired conclusion
has been reached.

Common closing phrases include:
\begin{quote}
Therefore, \dots \\
Hence, \dots \ as required. \\
This completes the proof.
\end{quote}

The symbol $\square$ may be used to mark the end of a proof.

\section*{10. Final Checklist}

Before a proof is considered complete, verify that:

\begin{itemize}
  \item All symbols are defined.
  \item All existence claims are justified.
  \item All logical steps are valid.
  \item The conclusion exactly matches the statement being proved.
  \item The argument is readable as continuous mathematical prose.
\end{itemize}

