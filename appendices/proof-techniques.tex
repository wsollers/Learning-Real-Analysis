% ======================================================
% Appendix: Proof Techniques and Discipline
% ======================================================

\chapter*{Appendix: Proof Techniques and Discipline}
\addcontentsline{toc}{chapter}{Appendix: Proof Techniques and Discipline}

\noindent
This appendix records the formal proof conventions, allowable logical moves,
and common proof patterns used throughout this document.
Its purpose is to make implicit proof discipline explicit,
to prevent logical errors, and to standardize reasoning across topics and authors.

\vspace{1.5em}

% ======================================================
\section*{A. Allowed Moves in Proofs}
\addcontentsline{toc}{section}{A. Allowed Moves in Proofs}

\noindent
The following checklist specifies which logical moves are permitted in rigorous
mathematical proofs and under what conditions they may be used.

\vspace{1em}

\subsection*{A.1 Introducing Arbitrary Elements}

\begin{itemize}
  \item To prove a universally quantified statement, e.g.\ $\forall x \in A\, P(x)$:
  \begin{itemize}
    \item You may begin with: \emph{“Let $x$ be an arbitrary element of $A$.”}
    \item The element $x$ must not satisfy any special properties beyond membership in $A$.
    \item The conclusion must hold for this arbitrary $x$.
  \end{itemize}

  \item You may not assume existence of special elements unless explicitly justified.
\end{itemize}

\vspace{0.75em}

\subsection*{A.2 Using Existential Statements}

\begin{itemize}
  \item From a statement of the form $\exists x\, P(x)$:
  \begin{itemize}
    \item You may introduce a new symbol $a$ such that $P(a)$ holds.
    \item This symbol must be treated as fixed but otherwise arbitrary.
    \item You may not assume $a$ has properties beyond those guaranteed by $P(a)$.
  \end{itemize}

  \item You may not choose an existential witness before the existential statement has been established.
\end{itemize}

\vspace{0.75em}

\subsection*{A.3 Conditional Proofs}

\begin{itemize}
  \item To prove an implication $P \rightarrow Q$:
  \begin{itemize}
    \item You may assume $P$ temporarily.
    \item Under this assumption, you must derive $Q$.
    \item Once $Q$ is established, the assumption $P$ is discharged.
  \end{itemize}

  \item The assumption $P$ may not be used outside the scope of the conditional proof.
\end{itemize}

\vspace{0.75em}

\subsection*{A.4 Proof by Contradiction}

\begin{itemize}
  \item To prove a statement $P$ by contradiction:
  \begin{itemize}
    \item Assume $\neg P$.
    \item Deduce a contradiction (e.g.\ $Q \land \neg Q$).
    \item Conclude that $P$ must be true.
  \end{itemize}

  \item The contradiction must be explicit and logically unavoidable.
\end{itemize}

\vspace{0.75em}

\subsection*{A.5 Using Definitions and Previously Proven Results}

\begin{itemize}
  \item You may invoke a definition once the object being defined has been introduced.
  \item You may invoke a theorem or lemma only after it has been proved or cited.
  \item Phrases such as “by definition” or “by theorem” must correspond to explicit prior material.
\end{itemize}

\vspace{1.5em}

% ======================================================
\section*{B. Proof Patterns Index}
\addcontentsline{toc}{section}{B. Proof Patterns Index}

\noindent
The following are common proof patterns that recur throughout mathematics.
Recognizing these patterns reduces cognitive load and improves proof clarity.

\vspace{1em}

\subsection*{B.1 Showing Set Inclusion}

To prove $A \subseteq B$:
\begin{itemize}
  \item Let $x$ be an arbitrary element of $A$.
  \item Show that $x \in B$.
\end{itemize}

\vspace{0.75em}

\subsection*{B.2 Proving Equality of Sets}

To prove $A = B$:
\begin{itemize}
  \item Prove $A \subseteq B$.
  \item Prove $B \subseteq A$.
\end{itemize}

\vspace{0.75em}

\subsection*{B.3 Proving Logical Equivalence}

To prove $P \leftrightarrow Q$:
\begin{itemize}
  \item Prove $P \rightarrow Q$.
  \item Prove $Q \rightarrow P$.
\end{itemize}

\vspace{0.75em}

\subsection*{B.4 Proving Existence}

To prove $\exists x\, P(x)$:
\begin{itemize}
  \item Explicitly define or construct an object $a$.
  \item Verify that $P(a)$ holds.
\end{itemize}

\vspace{0.75em}

\subsection*{B.5 Proving Uniqueness}

To prove existence and uniqueness:
\begin{itemize}
  \item First prove existence.
  \item Then assume two objects satisfy the required property.
  \item Show they must be equal.
\end{itemize}

\vspace{0.75em}

\subsection*{B.6 Proving a Statement is Not True}

To disprove a universal statement:
\begin{itemize}
  \item Provide a counterexample.
  \item Verify explicitly that it violates the statement.
\end{itemize}

\vspace{0.75em}

\subsection*{B.7 Working with Relations}

To prove a relation $R$ is:
\begin{itemize}
  \item Reflexive: show $(x,x) \in R$ for all $x$.
  \item Symmetric: assume $(x,y) \in R$ and show $(y,x) \in R$.
  \item Transitive: assume $(x,y) \in R$ and $(y,z) \in R$, then show $(x,z) \in R$.
\end{itemize}

\vspace{0.75em}

\subsection*{B.8 Function Equality}

To prove two functions $f$ and $g$ are equal:
\begin{itemize}
  \item Let $x$ be an arbitrary element of the domain.
  \item Show that $f(x) = g(x)$.
\end{itemize}

\vspace{1em}

\noindent
This index is not exhaustive and may be expanded as new proof patterns arise.
