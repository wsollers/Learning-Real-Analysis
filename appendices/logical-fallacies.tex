% ============================================================
% Logical Fallacies — Reference Page (Propositional + Quantifiers)
% Includes: name, symbolic schema, and why invalid
% ============================================================

\clearpage
\section*{Common Logical Fallacies (Invalid Inference Patterns)}
\addcontentsline{toc}{section}{Common Logical Fallacies (Invalid Inference Patterns)}

\noindent
\textbf{Note.} This page cannot list \emph{all} logical fallacies (there is no
finite, universally agreed-upon exhaustive list), but it does collect the most
common and practically important \emph{formal} fallacies encountered in
propositional and predicate calculus proofs.

\vspace{0.75em}

\renewcommand{\arraystretch}{1.25}

% ------------------------------------------------------------
% A. Propositional (Formal) Fallacies
% ------------------------------------------------------------
\subsection*{A. Propositional Fallacies}

\begin{center}
\begin{tabular}{|p{3.7cm}|p{5.3cm}|p{6.0cm}|}
\hline
\textbf{Fallacy} & \textbf{Invalid Schema} & \textbf{Why Incorrect} \\
\hline

Affirming the Consequent
&
$\begin{array}{l}
P \rightarrow Q\\
Q\\ \hline
P
\end{array}$
&
$Q$ can be true for reasons other than $P$; the conditional does not assert that
$P$ is the only way to obtain $Q$. \\
\hline

Denying the Antecedent
&
$\begin{array}{l}
P \rightarrow Q\\
\neg P\\ \hline
\neg Q
\end{array}$
&
A conditional does not say $Q$ depends only on $P$; $Q$ may still hold even if
$P$ fails. \\
\hline

Illicit Inference from Disjunction
(``Affirming a Disjunct'')
&
$\begin{array}{l}
P \vee Q\\
P\\ \hline
\neg Q
\end{array}$
&
Inclusive $\vee$ allows both disjuncts to be true simultaneously; $P$ does not
exclude $Q$. \\
\hline

Converse Error
&
From $P \rightarrow Q$ infer $Q \rightarrow P$
&
A conditional is not symmetric; reversing direction changes its truth
conditions. \\
\hline

Inverse Error
&
From $P \rightarrow Q$ infer $\neg P \rightarrow \neg Q$
&
Negating both parts does not preserve implication. The valid related form is the
contrapositive $\neg Q \rightarrow \neg P$. \\
\hline

Improper Distribution over $\vee$
&
From $P \rightarrow (Q \vee R)$ infer $(P \rightarrow Q)\vee(P \rightarrow R)$
&
$(P\rightarrow Q)\vee(P\rightarrow R)$ is strictly weaker/stronger depending on
valuation; the transformation is not valid in general. \\
\hline

Fallacy of the Undistributed Middle
(standard syllogistic form)
&
$\begin{array}{l}
A \rightarrow B\\
C \rightarrow B\\ \hline
A \rightarrow C
\end{array}$
&
Sharing a common consequent $B$ does not connect $A$ to $C$. \\
\hline

\end{tabular}
\end{center}

\vspace{0.5em}

% ------------------------------------------------------------
% B. Quantifier / Predicate Calculus Fallacies
% ------------------------------------------------------------
\subsection*{B. Quantifier and Predicate Calculus Fallacies}

\begin{center}
\begin{tabular}{|p{3.7cm}|p{5.3cm}|p{6.0cm}|}
\hline
\textbf{Fallacy} & \textbf{Invalid Schema} & \textbf{Why Incorrect} \\
\hline

Quantifier Negation Error I
&
$\neg \forall x\,\varphi(x)\;\Rightarrow\; \forall x\,\neg\varphi(x)$
&
Negating $\forall$ produces a counterexample, not universal failure.
Correct: $\neg\forall x\,\varphi \equiv \exists x\,\neg\varphi$. \\
\hline

Quantifier Negation Error II
&
$\neg \exists x\,\varphi(x)\;\Rightarrow\; \exists x\,\neg\varphi(x)$
&
Negating existence yields universal failure.
Correct: $\neg\exists x\,\varphi \equiv \forall x\,\neg\varphi$. \\
\hline

Illegal Quantifier Swap
&
$\forall x\,\exists y\,R(x,y)\;\Rightarrow\;\exists y\,\forall x\,R(x,y)$
&
The right side requires a \emph{single} $y$ working for all $x$; the left side
allows $y$ to depend on $x$. \\
\hline

Illicit Universal Generalization (UG)
&
$\varphi(x)\;\Rightarrow\;\forall x\,\varphi(x)$
(with $x$ depending on an assumption)
&
UG is valid only when $x$ is not free in any undischarged assumption; otherwise
the statement may hold only for a special $x$. \\
\hline

Illegal Existential Instantiation (EI)
&
$\exists x\,\varphi(x)$, choose a witness term $t$ \emph{already constrained}
&
Existential instantiation requires a \emph{fresh} witness; choosing a term with
prior properties can smuggle in extra assumptions. \\
\hline

Existential Fallacy (Overgeneralization)
&
$\exists x\,\varphi(x)\;\Rightarrow\;\forall x\,\varphi(x)$
&
A single witness does not justify a universal claim. \\
\hline

Universal-to-Existential Drop (Wrong Direction)
&
$\exists x\,\varphi(x)\;\Rightarrow\;\forall x\,\varphi(x)$
&
Same error pattern: existence does not imply universality. \\
\hline

Scope Error
&
$(\forall x\,P(x))\wedge Q(x)\;\Rightarrow\;\forall x\,(P(x)\wedge Q(x))$
&
The occurrence of $x$ in $Q(x)$ may be free; extending scope changes meaning. \\
\hline

Variable Capture (Substitution Fallacy)
&
Substitute a term $t$ for $x$ where variables in $t$ become bound
&
Capture changes the formula’s meaning; substitution must be capture-avoiding. \\
\hline

\end{tabular}
\end{center}

\vspace{0.75em}

% ------------------------------------------------------------
% C. Quick Counterexample Reminder
% ------------------------------------------------------------
\subsection*{C. Counterexample Reminder (How to Refute a Fallacious Form)}

\begin{remark}
A reliable way to diagnose a suspected fallacy is to exhibit a truth assignment
(or structure/assignment in predicate logic) under which the premises are true
but the conclusion is false. This demonstrates invalidity immediately.
\end{remark}
