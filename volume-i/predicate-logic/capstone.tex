% ======================================
% Capstone Assessment — Predicate Calculus
% ======================================

\section{Capstone Assessment: Predicate Calculus}

\noindent
\textbf{Purpose.}
This capstone assesses mastery of first-order predicate logic, including
quantifiers, identity, scope, semantic consequence, and formal proof structure.
All arguments must be expressed using first-order logical reasoning.
No set-theoretic arguments may be used unless explicitly stated.

\vspace{1em}

\noindent
\textbf{Instructions.}
Each problem requires a complete and rigorous proof.
You must explicitly justify quantifier introduction and elimination steps.
Appeals to intuition or informal paraphrase are not sufficient.

\vspace{1.5em}

% --------------------------------------------------
\subsection*{Problem 1 — Quantifier Order Sensitivity}

Prove that the following implication is logically valid:
\[
\forall x (P(x) \rightarrow Q(x)) \;\land\; \exists x\, P(x)
\;\rightarrow\;
\exists x\, Q(x).
\]

Your proof must explicitly identify where existential instantiation
and universal instantiation are applied.

\vspace{1.5em}

% --------------------------------------------------
\subsection*{Problem 2 — Failure of Converse}

Show that the converse of the statement in Problem 1 is not logically valid.
That is, show that
\[
\exists x\, Q(x) \;\rightarrow\; \exists x\, P(x)
\]
does not follow from
\[
\forall x (P(x) \rightarrow Q(x)).
\]

Your argument must be semantic (valuation/model-based), not syntactic.

\vspace{1.5em}

% --------------------------------------------------
\subsection*{Problem 3 — Scope and Negation}

Prove that the following two formulas are logically equivalent:
\[
\neg \forall x\, P(x)
\quad\text{and}\quad
\exists x\, \neg P(x).
\]

Your proof must make explicit use of the semantics of quantifiers.

\vspace{1.5em}

% --------------------------------------------------
\subsection*{Problem 4 — Identity Reasoning}

Assume identity is part of the language.
Prove that the following implication is logically valid:
\[
x = y \;\rightarrow\; (P(x) \leftrightarrow P(y)).
\]

Your proof must not assume substitutivity without justification.

\vspace{1.5em}

% --------------------------------------------------
\subsection*{Problem 5 — Semantic Consequence}

Show that the set of formulas
\[
\{\, \forall x (P(x) \rightarrow Q(x)),\; \forall x (Q(x) \rightarrow R(x)),\;
\exists x\, P(x) \,\}
\]
logically implies
\[
\exists x\, R(x).
\]

Your proof must argue that every structure satisfying the premises
also satisfies the conclusion.

\vspace{1em}

\noindent
\textbf{Completion Criterion.}
You have mastered predicate calculus if all five proofs:
\begin{itemize}
  \item use quantifier rules correctly,
  \item respect variable scope and dependency,
  \item distinguish syntax from semantics, and
  \item are written without hidden assumptions.
\end{itemize}

\noindent
Successful completion certifies readiness to proceed to
set-theoretic foundations.
