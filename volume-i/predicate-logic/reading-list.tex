% reading-list.tex

\noindent
\textbf{Goal.} Master first-order predicate logic: quantifiers, identity, semantics,
validity/logical consequence, and formal proof systems.

\vspace{1em}

\begin{longtable}{|p{4.2cm}|p{10.8cm}|}
\hline
\textbf{Source} & \textbf{Required Reading for Mastery} \\
\hline
Open Logic Project — \textit{forall x: Calgary} &
\begin{itemize}
  \item First-order logic chapters (quantifiers, identity, semantics)
  \item Natural deduction rules for $\forall,\exists$ and identity
  \item Soundness/completeness overview for FOL
\end{itemize}
\\ \hline

Goldrei — \textit{Propositional and Predicate Calculus} &
\begin{itemize}
  \item Chapter 4 — Predicates and models (4.1–4.5)
  \item Chapter 5 — Formal predicate calculus (5.1–5.5)
\end{itemize}
\\ \hline

Hurley — \textit{A Concise Introduction to Logic} &
\begin{itemize}
  \item Chapter 8 — Predicate Logic (8.1–8.7)
\end{itemize}
\\ \hline

Chiossi — \textit{Essential Mathematics for Undergraduates} &
\begin{itemize}
  \item Part I §1.1 First-Order Languages (as needed)
  \item Part I §1.3 Predicative Calculus
\end{itemize}
\\ \hline

Tao — \textit{Analysis I} &
\begin{itemize}
  \item Appendix A.4–A.6 (variables/quantifiers; nested quantifiers; examples)
\end{itemize}
\\ \hline

Stoll — \textit{Set Theory and Logic} &
\begin{itemize}
  \item Predicate calculus portions (stop before set theory if desired)
\end{itemize}
\\ \hline

Suppes — \textit{Axiomatic Set Theory} &
\begin{itemize}
  \item First-order preliminaries (quantifiers/equality) as reference
\end{itemize}
\\ \hline


% ============================================================
Suppes \& Hill — \textit{First Course in Mathematical Logic} &
\begin{itemize}
  \item \textbf{Chapter Five — Terms, Predicates, and Universal Quantifiers}
  \begin{itemize}
    \item 5.2 Terms
    \item 5.3 Predicates
    \item 5.4 Common nouns as predicates
    \item 5.5 Atomic formulas and variables
    \item 5.6 Universal quantifiers
    \item 5.7 Two standard forms
  \end{itemize}

  \item \textbf{Chapter Six — Universal Specification and Logical Inference}
  \begin{itemize}
    \item 6.1 One quantifier
    \item 6.2 Two or more quantifiers
    \item 6.3 Logic of identity
    \item 6.4 Truths of logic
  \end{itemize}

  \item \textbf{Chapter Eight — Universal Generalization}
  \begin{itemize}
    \item 8.1 Theorems with variables
    \item 8.2 Theorems with universal quantifiers
  \end{itemize}

  \item \textbf{Emphasis:}
  \begin{itemize}
    \item Variable binding and scope discipline
    \item Universal instantiation and generalization
    \item Restrictions on variable dependence
    \item Formal line-by-line proofs with quantifiers
    \item Identity and substitution rules
  \end{itemize}

  \item \textbf{Stop here.}  
  Defer Chapter Seven (axiomatic arithmetic) and all metalogical topics
  to later chapters in the curriculum.
\end{itemize}
\\ \hline



























\end{longtable}
