% =========================================================
% Proof: Renaming Bound Variables
% Source: volume-i/predicate-logic/notes/quantifiers/notes-quantifier-laws.tex
% =========================================================

\subsection*{Renaming Bound Variables}
\label{prf:rename}

\begin{remark}[Return]
\hyperref[thm:rename]{$\leftarrow$ Back to Proposition (Renaming Bound Variables) in Notes}
\end{remark}

\begin{proposition}[Renaming Bound Variables]
If $y$ does not occur in $\varphi$ (neither free nor bound), then:
\begin{align*}
\forall x\,\varphi &\;\equiv\; \forall y\,\varphi[y/x], \\
\exists x\,\varphi &\;\equiv\; \exists y\,\varphi[y/x].
\end{align*}
\end{proposition}

\begin{proof}
\Given The semantics of $\forall$: $\mathfrak{M} \models \forall x\,\varphi$
iff $\varphi[d/x]$ holds for all $d \in D$. The semantics of substitution: if
$y$ does not occur in $\varphi$, then $\varphi[y/x][d/y] = \varphi[d/x]$ for
any $d \in D$ (substituting $y$ for $x$ and then $d$ for $y$ is the same as
substituting $d$ for $x$ directly, since $y$ is fresh and causes no capture).

\Goal To show $\forall x\,\varphi \equiv \forall y\,\varphi[y/x]$ and
$\exists x\,\varphi \equiv \exists y\,\varphi[y/x]$.

\medskip
\noindent\textbf{Universal case.} Fix a model $\mathfrak{M}$ with domain $D$.

$\mathfrak{M} \models \forall x\,\varphi$
\quad iff \quad
for all $d \in D$, $\mathfrak{M} \models \varphi[d/x]$
\quad iff \quad
for all $d \in D$, $\mathfrak{M} \models (\varphi[y/x])[d/y]$
\quad iff \quad
$\mathfrak{M} \models \forall y\,\varphi[y/x]$.

The middle step uses the substitution identity $\varphi[d/x] = (\varphi[y/x])[d/y]$,
which holds because $y$ is fresh in $\varphi$ so no variable capture occurs when
replacing $x$ by $y$.

\medskip
\noindent\textbf{Existential case.} The same chain of equivalences with
$\exists$ in place of $\forall$ and ``some $d \in D$'' in place of ``all $d
\in D$.'' \AsReq
\end{proof}

\begin{remark}[Proof shape]
The proof is a one-step semantic chain in each direction, mediated by the
substitution identity $\varphi[d/x] = (\varphi[y/x])[d/y]$. This identity is
the core of the argument and holds precisely because $y$ is fresh (does not
appear in $\varphi$), ensuring no capture occurs.
\end{remark}

\begin{remark}[The freshness condition is essential]
If $y$ already occurs free in $\varphi$, the substitution $\varphi[y/x]$ may
capture $y$. For example, $\forall x\,(x < y)$ and $\forall y\,(y < y)$ are
not equivalent: the former asserts all elements are less than $y$ (a free
variable), while the latter (after careless renaming) asserts all elements are
less than themselves, which is false.
\end{remark}

\begin{remark}[Connection to alpha-equivalence]
In type theory and lambda calculus, renaming bound variables is called
\emph{alpha-equivalence}. This proposition is the first-order logic analogue:
bound variables are placeholders, and their names carry no semantic content.
\end{remark}

\begin{remark}[Dependencies]
The proof depends on: the semantics of $\forall$ and $\exists$, and the
substitution identity $\varphi[d/x] = (\varphi[y/x])[d/y]$ when $y$ is fresh
in $\varphi$.
\end{remark}
