% =========================================================
% Proof: Quantifier Negation Laws
% Source: volume-i/predicate-logic/notes/quantifiers/notes-quantifier-laws.tex
% =========================================================

\subsection*{Quantifier Negation Laws}
\label{prf:qneg}

\begin{remark}[Return]
\hyperref[thm:qneg]{$\leftarrow$ Back to Proposition (Quantifier Negation Laws) in Notes}
\end{remark}

\begin{proof}
\Claim For any formula $\varphi$ and model $\mathfrak{M}$ with domain $D$: $\neg \forall x\,\varphi \;\equiv\; \exists x\,\neg \varphi$ and $\neg \exists x\,\varphi \;\equiv\; \forall x\,\neg \varphi$.

\Given The semantics of the quantifiers: $\mathfrak{M} \models \forall x\,\varphi$
iff $\varphi$ holds for all $d \in D$; $\mathfrak{M} \models \exists x\,\varphi$
iff $\varphi$ holds for some $d \in D$. The semantics of negation:
$\mathfrak{M} \models \neg\psi$ iff $\mathfrak{M} \not\models \psi$.

\Goal To show that the two equivalences hold in every model.

\Strategy We prove both equivalences by unfolding the semantic definitions and
applying classical logic (specifically, the equivalence $\neg\forall \equiv
\exists\neg$ and $\neg\exists \equiv \forall\neg$ for sets).

\medskip
\noindent\textbf{First law: $\neg\forall x\,\varphi \equiv \exists x\,\neg\varphi$.}

Fix a model $\mathfrak{M}$ with domain $D$.

($\Rightarrow$) Suppose $\mathfrak{M} \models \neg\forall x\,\varphi$. Then
$\mathfrak{M} \not\models \forall x\,\varphi$, so it is not the case that
$\varphi$ holds for all $d \in D$. Hence there exists some $d \in D$ for which
$\varphi[d/x]$ fails, i.e.\ $\neg\varphi[d/x]$ holds. Therefore $\mathfrak{M}
\models \exists x\,\neg\varphi$.

($\Leftarrow$) Suppose $\mathfrak{M} \models \exists x\,\neg\varphi$. Then
there exists $d \in D$ with $\mathfrak{M} \models \neg\varphi[d/x]$, so
$\varphi[d/x]$ fails. Therefore $\varphi$ does not hold for all $d \in D$,
meaning $\mathfrak{M} \not\models \forall x\,\varphi$, i.e.\ $\mathfrak{M}
\models \neg\forall x\,\varphi$.

\medskip
\noindent\textbf{Second law: $\neg\exists x\,\varphi \equiv \forall x\,\neg\varphi$.}

($\Rightarrow$) Suppose $\mathfrak{M} \models \neg\exists x\,\varphi$. Then
$\mathfrak{M} \not\models \exists x\,\varphi$, so no $d \in D$ satisfies
$\varphi[d/x]$. Hence for every $d \in D$, $\neg\varphi[d/x]$ holds.
Therefore $\mathfrak{M} \models \forall x\,\neg\varphi$.

($\Leftarrow$) Suppose $\mathfrak{M} \models \forall x\,\neg\varphi$. Then for
every $d \in D$, $\varphi[d/x]$ fails. So no $d \in D$ satisfies $\varphi$,
meaning $\mathfrak{M} \not\models \exists x\,\varphi$, i.e.\ $\mathfrak{M}
\models \neg\exists x\,\varphi$. \AsReq
\end{proof}

\begin{remark}[Proof shape]
Each law is a direct biconditional proof: the two directions are proved by
unfolding the semantic definitions of $\neg$, $\forall$, and $\exists$ and
applying the classical logical equivalence between ``not for all'' and ``there
exists a counterexample.'' No structural induction is required.
\end{remark}

\begin{remark}[Classical logic dependence]
The first law uses the classical fact that $\neg(\forall d \in D, P(d))
\Leftrightarrow \exists d \in D, \neg P(d)$. This is a standard set-theoretic
equivalence that holds for any domain $D$ and property $P$. In
intuitionistic logic, $\neg\forall \Rightarrow \exists\neg$ may fail when $D$
is infinite: one may know the universal quantification fails without being able
to produce a witness.
\end{remark}

\begin{remark}[Dependencies]
The proof depends on: the semantics of $\forall$, $\exists$, and $\neg$ in
first-order logic, and the classical set-theoretic equivalence between ``not
all'' and ``some not.''
\end{remark}
