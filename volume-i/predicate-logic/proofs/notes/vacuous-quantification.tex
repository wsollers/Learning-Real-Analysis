% =========================================================
% Proof: Vacuous Quantification
% Source: volume-i/predicate-logic/notes/quantifiers/notes-quantifier-laws.tex
% =========================================================

\subsection*{Vacuous Quantification}
\label{prf:vacuous}

\begin{remark}[Return]
\hyperref[thm:vacuous]{$\leftarrow$ Back to Proposition (Vacuous Quantification) in Notes}
\end{remark}

\begin{proposition}[Vacuous Quantification]
If $x$ does not occur free in $\varphi$, then:
\begin{align*}
\forall x\,\varphi &\;\equiv\; \varphi, \\
\exists x\,\varphi &\;\equiv\; \varphi.
\end{align*}
\end{proposition}

\begin{proof}
\Given The semantics of $\forall$ and $\exists$. The fact that if $x$ does not
occur free in $\varphi$, then the truth value of $\varphi$ is independent of
the value assigned to $x$: for any model $\mathfrak{M}$, domain $D$, and
assignment $\sigma$, $\mathfrak{M}, \sigma \models \varphi$ iff $\mathfrak{M},
\sigma[x \mapsto d] \models \varphi$ for any (hence every) $d \in D$.

\Goal To show both equivalences hold in every nonempty model.

\medskip
\noindent\textbf{Universal case: $\forall x\,\varphi \equiv \varphi$.}

Fix a model $\mathfrak{M}$ with nonempty domain $D$.

($\Rightarrow$) Suppose $\mathfrak{M} \models \forall x\,\varphi$. Then
$\varphi$ holds for every $d \in D$. Since $D$ is nonempty, pick any $d_0 \in
D$. Then $\varphi[d_0/x]$ holds. Since $x \notin \mathrm{FV}(\varphi)$, the
truth of $\varphi$ is independent of $x$'s value, so $\varphi$ holds.

($\Leftarrow$) Suppose $\mathfrak{M} \models \varphi$. Since $x \notin
\mathrm{FV}(\varphi)$, the truth value of $\varphi$ is unchanged when $x$ is
assigned any value $d \in D$. Hence $\varphi$ holds under any assignment to
$x$, so $\mathfrak{M} \models \forall x\,\varphi$.

\medskip
\noindent\textbf{Existential case: $\exists x\,\varphi \equiv \varphi$.}

($\Rightarrow$) Suppose $\mathfrak{M} \models \exists x\,\varphi$. Then there
is some $d \in D$ with $\varphi[d/x]$. Since $x \notin \mathrm{FV}(\varphi)$,
this is the same as $\varphi$ holding, so $\mathfrak{M} \models \varphi$.

($\Leftarrow$) Suppose $\mathfrak{M} \models \varphi$. Since $D$ is nonempty,
pick any $d_0 \in D$. Since $x \notin \mathrm{FV}(\varphi)$, $\varphi$ holds
under the assignment $x \mapsto d_0$. Hence $\mathfrak{M} \models \exists
x\,\varphi$. \AsReq
\end{proof}

\begin{remark}[Proof shape]
Both directions of both equivalences follow from one key semantic fact: if $x$
is not free in $\varphi$, the truth of $\varphi$ is independent of what $x$ is
assigned. The universal case uses nonemptiness of $D$ only in the
$(\Rightarrow)$ direction (to pick a witness). The existential case uses it in
the $(\Leftarrow)$ direction.
\end{remark}

\begin{remark}[Why nonemptiness matters]
In first-order logic over an empty domain, $\forall x\,\varphi$ is vacuously
true but $\exists x\,\varphi$ is false. Standard first-order logic requires
nonempty domains, so both equivalences hold. In empty-domain logic, the
universal equivalence would fail: $\forall x\,\varphi$ would be true but
$\varphi$ might be false.
\end{remark}

\begin{remark}[Dependencies]
The proof depends on: the semantics of $\forall$ and $\exists$, the definition
of free variables ($\mathrm{FV}(\varphi)$), and the assumption that models have
nonempty domains.
\end{remark}
