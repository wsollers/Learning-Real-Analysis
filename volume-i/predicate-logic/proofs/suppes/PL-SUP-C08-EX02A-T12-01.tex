% ======================================
% Proof: PL-SUP-C08-EX02A-T12-01
% ======================================
\clearpage
\phantomsection
\hypertarget{proof-PL-SUP-C08-EX02A-T12-01}{}

\section*{Proof — PL-SUP-C08-EX02A-T12-01}

\noindent
\hyperlink{ws-PL-SUP-C08-EX02A-T12-01}{\textbf{← Back to worksheet}}

\noindent
\textbf{Source.}
\srccite{SuppesHillFirstCourseLogic}{Chapter~8, Exercise~2(A)}.

\vspace{0.75em}

\noindent
\textbf{Goal.} Prove $(\forall x)(\forall y)(-(x+y)=(-x)+(-y))$.

\vspace{0.75em}

\noindent
\textbf{Proof.}
\[
\begin{array}{r l l}
(1) & \text{Let $x,y$ be arbitrary.} &  \\
(2) & (x+y)+(-(x+y))=0 & \text{Definition of additive inverse} \\
(3) & (x+y)+((-x)+(-y))=0 & \text{From previous theorems} \\
(4) & -(x+y)=(-x)+(-y) & \text{Uniqueness of additive inverse} \\
(5) & (\forall x)(\forall y)(-(x+y)=(-x)+(-y)) & \forall\text{-Gen}
\end{array}
\]

\begin{remark}[Uniqueness of the additive inverse]
If $a+b=0$ and $a+c=0$, then $b=c$.

In particular, if $A=0$ and $B=0$ are both obtained as sums with the same left operand,
then $A=B$ follows by uniqueness of the additive inverse.
\end{remark}

\vspace{1em}
\noindent
\hyperlink{ws-PL-SUP-C08-EX02A-T12-01}{\textbf{← Back to worksheet}}
