% =========================================================
% Logical Strength and Quantifier Order
% =========================================================

\subsection{Quantifiers: Logical Strength and Quantifier Order}

% ---------------------------------------------------------
% TOOLKIT
% ---------------------------------------------------------
\begin{tcolorbox}[colback=gray!6, colframe=gray!40, arc=2pt,
  left=6pt, right=6pt, top=4pt, bottom=4pt,
  title={\small\textbf{Logical Strength — Quick Reference}},
  fonttitle=\small\bfseries]
\small
\begin{tabular}{l l l}
\toprule
\textbf{Concept} & \textbf{Meaning} & \textbf{Detail} \\
\midrule
Entailment $A \models B$ & $A$ is true implies $B$ is true & \hyperref[def:entailment]{↓ Def} \\
Logical strength & $A$ stronger than $B$ if $A \models B$ but $B \not\models A$ & \hyperref[def:strength]{↓ Def} \\
Model-set viewpoint & $A \models B \iff \mathrm{Mod}(A) \subseteq \mathrm{Mod}(B)$ & \hyperref[def:model-set]{↓ Def} \\
Quantifier alternation & Alternating $\forall$/$\exists$ in prefix & \hyperref[def:q-alt]{↓ Def} \\
\bottomrule
\end{tabular}
\end{tcolorbox}

\vspace{1em}

% ---------------------------------------------------------
% Entailment
% ---------------------------------------------------------
\begin{tcolorbox}[colback=propbox, colframe=propborder, arc=2pt,
  left=6pt, right=6pt, top=4pt, bottom=4pt,
  title={\small\textbf{Definition (Logical Entailment)}},
  fonttitle=\small\bfseries]
\label{def:entailment}
Let $A$ and $B$ be sentences. We write $A \models B$ and say that \emph{$A$
entails $B$} if in every structure in which $A$ is true, $B$ is also true.
Equivalently, $A \models B$ means $A \rightarrow B$ is valid.
\end{tcolorbox}

% ---------------------------------------------------------
% Logical Strength
% ---------------------------------------------------------
\begin{tcolorbox}[colback=propbox, colframe=propborder, arc=2pt,
  left=6pt, right=6pt, top=4pt, bottom=4pt,
  title={\small\textbf{Definition (Logical Strength)}},
  fonttitle=\small\bfseries]
\label{def:strength}
We say that \emph{$A$ is (logically) stronger than $B$} if $A \models B$ but
$B \not\models A$.

$A$ and $B$ are \emph{logically equivalent} if $A \models B$ and $B \models A$.
\end{tcolorbox}

\begin{remark}[English reading]
``Stronger'' means ``harder to satisfy'' (fewer models make it true). If $A$ is
stronger than $B$, then $A$ rules out more possibilities: every world where $A$
holds is a world where $B$ holds, but not conversely.
\end{remark}

% ---------------------------------------------------------
% Model-Set Viewpoint
% ---------------------------------------------------------
\begin{tcolorbox}[colback=propbox, colframe=propborder, arc=2pt,
  left=6pt, right=6pt, top=4pt, bottom=4pt,
  title={\small\textbf{Definition (Model-Set Viewpoint)}},
  fonttitle=\small\bfseries]
\label{def:model-set}
For a statement $A$, let $\mathrm{Mod}(A)$ denote the class of all structures
in which $A$ is true. Then
\[
A \models B \quad\Longleftrightarrow\quad \mathrm{Mod}(A) \subseteq \mathrm{Mod}(B).
\]
\end{tcolorbox}

\begin{remark}[English reading]
Strength is set inclusion of model classes. ``$A$ is stronger than $B$'' is
literally $\mathrm{Mod}(A) \subsetneq \mathrm{Mod}(B)$: $A$ satisfies fewer
structures than $B$.
\end{remark}

% ---------------------------------------------------------
% Quantifier Alternation
% ---------------------------------------------------------
\begin{tcolorbox}[colback=propbox, colframe=propborder, arc=2pt,
  left=6pt, right=6pt, top=4pt, bottom=4pt,
  title={\small\textbf{Definition (Quantifier Alternation)}},
  fonttitle=\small\bfseries]
\label{def:q-alt}
A \emph{quantifier alternation} occurs when $Q_i \neq Q_{i+1}$ for some $i$ in
the prefix $Q_1 x_1 \cdots Q_k x_k$ of a formula. For example,
$\forall x\,\exists y\,\Phi(x,y)$ has one alternation, while
$\forall x\,\forall y\,\Phi(x,y)$ has none.
\end{tcolorbox}

\begin{remark}[Why alternation matters for strength]
$\forall x\,\exists y\,\Phi(x,y)$ requires a rule assigning a (possibly
$x$-dependent) witness $y$ to each $x$. In contrast,
$\exists y\,\forall x\,\Phi(x,y)$ requires a single uniform witness $y$ that
works for all $x$. Uniformity is strictly stronger than dependence.
\end{remark}

\begin{remark}[Strength hierarchy]
The following entailments hold:
\begin{enumerate}
  \item $\forall x\,\Phi(x) \models \exists x\,\Phi(x)$ (universal is stronger).
  \item $\exists y\,\forall x\,\Phi(x,y) \models \forall x\,\exists y\,\Phi(x,y)$
  (uniform witness is stronger than dependent witness).
\end{enumerate}
Neither converse holds in general.
\end{remark}

\begin{example}[Swap failure on $\mathbb{R}$]
Let the domain be $\mathbb{R}$ and $\Phi(x,y) = (y > x)$.

$\forall x\,\exists y\,(y > x)$ is true: given any $x$, take $y = x+1$.

$\exists y\,\forall x\,(y > x)$ is false: no single real exceeds all reals.

Hence $\exists y\,\forall x$ is strictly stronger than $\forall x\,\exists y$.
\end{example}

\begin{example}[Universal vs.\ existential]
Let the domain be $\mathbb{R}$ and $\Phi(x) = (x^2 \geq 0)$.

$\forall x\,(x^2 \geq 0)$ entails $\exists x\,(x^2 \geq 0)$, but not conversely.
Thus the universal statement is strictly stronger.
\end{example}

\begin{remark}[Logical strength increases when you:]
\begin{itemize}
  \item add conjuncts ($A \wedge B$ is stronger than $A$),
  \item replace $\exists$ with $\forall$,
  \item move existential quantifiers outward (requiring uniform witnesses),
  \item introduce quantifier alternation.
\end{itemize}
\end{remark}
