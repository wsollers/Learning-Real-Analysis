% =========================================================
% Universal, Existential, Unique Existential, and Bounded Quantifiers
% =========================================================

\subsection{Quantifiers: Universal, Existential, and Bounded}

% ---------------------------------------------------------
% TOOLKIT
% ---------------------------------------------------------
\begin{tcolorbox}[colback=gray!6, colframe=gray!40, arc=2pt,
  left=6pt, right=6pt, top=4pt, bottom=4pt,
  title={\small\textbf{Basic Quantifiers — Quick Reference}},
  fonttitle=\small\bfseries]
\small
\begin{tabular}{l l l l}
\toprule
\textbf{Symbol} & \textbf{Reading} & \textbf{Meaning} & \textbf{Detail} \\
\midrule
$\forall x\,\varphi$ & For all $x$, $\varphi$ & True for every domain element & \hyperref[def:universal-q]{↓ Def} \\
$\exists x\,\varphi$ & There exists $x$ such that $\varphi$ & True for some domain element & \hyperref[def:existential-q]{↓ Def} \\
$\exists! x\,\varphi$ & There exists exactly one $x$ such that $\varphi$ & Existence + uniqueness & \hyperref[def:unique-exist]{↓ Def} \\
$\forall x \in A\,\varphi$ & For all $x$ in $A$, $\varphi$ & Restricted universal & \hyperref[def:bounded-q]{↓ Def} \\
$\exists x \in A\,\varphi$ & There exists $x$ in $A$ such that $\varphi$ & Restricted existential & \hyperref[def:bounded-q]{↓ Def} \\
\bottomrule
\end{tabular}
\end{tcolorbox}

\vspace{1em}

% ---------------------------------------------------------
% Universal Quantifier
% ---------------------------------------------------------
\begin{tcolorbox}[colback=propbox, colframe=propborder, arc=2pt,
  left=6pt, right=6pt, top=4pt, bottom=4pt,
  title={\small\textbf{Definition (Universal Quantifier)}},
  fonttitle=\small\bfseries]
\label{def:universal-q}
The \emph{universal quantifier}, denoted $\forall$, is a logical operator that
binds a variable and asserts that a formula holds for all elements of the
domain of discourse.

If $\varphi$ is a formula and $x$ is a variable, then $\forall x\,\varphi$ is
a formula, read ``for all $x$, $\varphi$.''
\end{tcolorbox}

\begin{remark}[English reading]
The universal quantifier expresses a global condition over the entire domain.
It does not claim the domain is nonempty on its own, but by the nonempty domain
convention, there is always at least one element to instantiate.
\end{remark}

% ---------------------------------------------------------
% Existential Quantifier
% ---------------------------------------------------------
\begin{tcolorbox}[colback=propbox, colframe=propborder, arc=2pt,
  left=6pt, right=6pt, top=4pt, bottom=4pt,
  title={\small\textbf{Definition (Existential Quantifier)}},
  fonttitle=\small\bfseries]
\label{def:existential-q}
The \emph{existential quantifier}, denoted $\exists$, is a logical operator that
binds a variable and asserts that a formula holds for at least one element of
the domain of discourse.

If $\varphi$ is a formula and $x$ is a variable, then $\exists x\,\varphi$ is
a formula, read ``there exists an $x$ such that $\varphi$.''
\end{tcolorbox}

\begin{remark}[English reading]
The existential quantifier expresses a local condition. It does not specify
which element witnesses the claim; it merely asserts that such a witness exists
somewhere in the domain.
\end{remark}

% ---------------------------------------------------------
% Unique Existential Quantifier
% ---------------------------------------------------------
\begin{tcolorbox}[colback=propbox, colframe=propborder, arc=2pt,
  left=6pt, right=6pt, top=4pt, bottom=4pt,
  title={\small\textbf{Definition (Unique Existential Quantifier)}},
  fonttitle=\small\bfseries]
\label{def:unique-exist}
The \emph{unique existential quantifier} $\exists!$ asserts that there exists
exactly one element satisfying a given property. It is an abbreviation:
\[
\exists! x\,\varphi
\;:=\;
\exists x\,\bigl(\varphi \wedge \forall y\,(\varphi[y/x] \rightarrow y = x)\bigr),
\]
where $y$ is a variable not occurring in $\varphi$.

Equivalently,
\[
\exists! x\,\varphi
\;\equiv\;
\exists x\,\varphi \;\wedge\; \forall x\,\forall y\,
\bigl((\varphi \wedge \varphi[y/x]) \rightarrow x = y\bigr).
\]
\end{tcolorbox}

\begin{remark}[English reading]
$\exists! x\,\varphi$ combines existence (at least one $x$ satisfies $\varphi$)
and uniqueness (at most one such $x$ exists). It is not a primitive quantifier
but a convenient abbreviation.
\end{remark}

\begin{example}
``There is exactly one even prime number'' formalizes as
$\exists! x\,(P(x) \wedge E(x))$, where $P(x)$ means ``$x$ is prime'' and
$E(x)$ means ``$x$ is even.''
\end{example}

% ---------------------------------------------------------
% Bounded Quantifiers
% ---------------------------------------------------------
\begin{tcolorbox}[colback=propbox, colframe=propborder, arc=2pt,
  left=6pt, right=6pt, top=4pt, bottom=4pt,
  title={\small\textbf{Definition (Bounded Quantifiers)}},
  fonttitle=\small\bfseries]
\label{def:bounded-q}
Let $A$ be a set (or a predicate defining a set). The \emph{bounded
quantifiers} are abbreviations:
\begin{align*}
\forall x \in A\,\varphi &\;:=\; \forall x\,(x \in A \rightarrow \varphi), \\
\exists x \in A\,\varphi &\;:=\; \exists x\,(x \in A \wedge \varphi).
\end{align*}
\end{tcolorbox}

\begin{remark}[Asymmetry]
Bounded universal uses implication ($\rightarrow$), while bounded existential
uses conjunction ($\wedge$). This is not arbitrary: if the set $A$ is empty,
$\forall x \in A\,\varphi$ is vacuously true (there is no $x$ to violate the
implication), while $\exists x \in A\,\varphi$ is false (no witness exists).
\end{remark}

\begin{remark}[Common error]
Writing $\forall x \in A\,\varphi$ as $\forall x\,(x \in A \wedge \varphi)$
is wrong: this would assert that every element is in $A$, not just that every
element of $A$ satisfies $\varphi$.
\end{remark}

\begin{theorem}[Negation of Bounded Quantifiers]\label{thm:neg-bounded}
Let $A$ be a set and $\varphi$ a formula. Then:
\begin{align*}
\neg(\forall x \in A\,\varphi) &\;\equiv\; \exists x \in A\,\neg\varphi, \\
\neg(\exists x \in A\,\varphi) &\;\equiv\; \forall x \in A\,\neg\varphi.
\end{align*}
\end{theorem}

\begin{proof}
We derive each case by expanding and applying standard quantifier negation laws.

\textbf{Universal case.}
\begin{align*}
\neg(\forall x \in A\,\varphi)
&\equiv \neg\forall x\,(x \in A \rightarrow \varphi) \\
&\equiv \exists x\,\neg(x \in A \rightarrow \varphi) \\
&\equiv \exists x\,(x \in A \wedge \neg\varphi) \\
&\equiv \exists x \in A\,\neg\varphi.
\end{align*}

\textbf{Existential case.}
\begin{align*}
\neg(\exists x \in A\,\varphi)
&\equiv \neg\exists x\,(x \in A \wedge \varphi) \\
&\equiv \forall x\,\neg(x \in A \wedge \varphi) \\
&\equiv \forall x\,(x \in A \rightarrow \neg\varphi) \\
&\equiv \forall x \in A\,\neg\varphi. \qedhere
\end{align*}
\end{proof}

\begin{remark}
The domain restriction $x \in A$ is preserved under negation because it
functions as a constraint on which elements are quantified over, not as part
of the claim being negated.
\end{remark}
