% =========================================================
% Prenex Normal Form
% =========================================================

\subsection{Quantifiers: Prenex Normal Form}

% ---------------------------------------------------------
% TOOLKIT
% ---------------------------------------------------------
\begin{tcolorbox}[colback=gray!6, colframe=gray!40, arc=2pt,
  left=6pt, right=6pt, top=4pt, bottom=4pt,
  title={\small\textbf{Prenex Normal Form — Quick Reference}},
  fonttitle=\small\bfseries]
\small
\begin{tabular}{l l l}
\toprule
\textbf{Concept} & \textbf{Meaning} & \textbf{Detail} \\
\midrule
Prenex normal form (PNF) & All quantifiers at front, quantifier-free matrix & \hyperref[def:pnf]{↓ Def} \\
PNF Theorem & Every formula is equivalent to a PNF formula & \hyperref[thm:pnf]{↓ Thm} \\
Quantifier-pulling rules & Equivalences for extracting quantifiers & \hyperref[def:pnf]{↓ Def} \\
\bottomrule
\end{tabular}
\end{tcolorbox}

\vspace{1em}

% ---------------------------------------------------------
% Prenex Normal Form
% ---------------------------------------------------------
\begin{tcolorbox}[colback=propbox, colframe=propborder, arc=2pt,
  left=6pt, right=6pt, top=4pt, bottom=4pt,
  title={\small\textbf{Definition (Prenex Normal Form)}},
  fonttitle=\small\bfseries]
\label{def:pnf}
A first-order formula is in \emph{prenex normal form} (PNF) if it has the shape
\[
Q_1 x_1\, Q_2 x_2 \cdots Q_n x_n \;\; \psi,
\]
where each $Q_i \in \{\forall,\exists\}$ and $\psi$ is quantifier-free. The
string $Q_1 x_1 \cdots Q_n x_n$ is the \emph{quantifier prefix}; $\psi$ is
the \emph{matrix}.
\end{tcolorbox}

% ---------------------------------------------------------
% PNF Theorem
% ---------------------------------------------------------
\begin{tcolorbox}[colback=thmbox, colframe=thmborder, arc=2pt,
  left=6pt, right=6pt, top=4pt, bottom=4pt,
  title={\small\textbf{Theorem (Prenex Normal Form Theorem)}},
  fonttitle=\small\bfseries]
\label{thm:pnf}
Every first-order formula is logically equivalent to a formula in prenex normal
form.
\end{tcolorbox}

\begin{remark}[Proof strategy — conversion procedure]
Given a formula $\varphi$, the following four steps produce an equivalent PNF:
\begin{enumerate}
  \item \textbf{Eliminate $\leftrightarrow$ and $\rightarrow$.}
  Rewrite using only $\neg,\wedge,\vee$:
  $(A \rightarrow B) \equiv (\neg A \vee B)$.
  \item \textbf{Push negations inward (Negation Normal Form).}
  Use De Morgan's laws and quantifier-negation laws until $\neg$ applies only
  to atomic formulas.
  \item \textbf{Standardize bound variables apart.}
  Rename bound variables so that no variable is quantified twice and no bound
  variable coincides with any free variable.
  \item \textbf{Pull quantifiers outward.}
  Apply the quantifier-pulling equivalences (below) to move all quantifiers
  to the front.
\end{enumerate}
\end{remark}

\begin{remark}[Quantifier-pulling equivalences]
The following hold when $x \notin \mathrm{FV}(\psi)$:
\begin{center}
\renewcommand{\arraystretch}{1.25}
\begin{tabular}{ll}
\toprule
\textbf{Equivalence} & \textbf{Side condition} \\
\midrule
$(\forall x\,\phi)\wedge \psi \equiv \forall x\,(\phi\wedge \psi)$ & $x \notin \mathrm{FV}(\psi)$ \\
$(\exists x\,\phi)\vee \psi \equiv \exists x\,(\phi\vee \psi)$ & $x \notin \mathrm{FV}(\psi)$ \\
$(\forall x\,\phi)\vee \psi \equiv \forall x\,(\phi\vee \psi)$ & $x \notin \mathrm{FV}(\psi)$ \\
$(\exists x\,\phi)\wedge \psi \equiv \exists x\,(\phi\wedge \psi)$ & $x \notin \mathrm{FV}(\psi)$ \\
\bottomrule
\end{tabular}
\end{center}
\end{remark}

\begin{example}[PNF conversion]
Normalize $\varphi := \neg\bigl(\forall x\,P(x) \rightarrow \exists y\,Q(y)\bigr)$.

\textbf{Step 1 (eliminate $\rightarrow$):}
$\varphi \equiv \neg\bigl(\neg\forall x\,P(x) \vee \exists y\,Q(y)\bigr)$.

\textbf{Step 2 (push $\neg$ inward):}
$\varphi \equiv (\forall x\,P(x)) \wedge (\forall y\,\neg Q(y))$.

\textbf{Step 3 (standardize apart):} Variables are already distinct.

\textbf{Step 4 (pull quantifiers outward):}
$\varphi \equiv \forall x\,\forall y\,(P(x) \wedge \neg Q(y))$.
\end{example}
