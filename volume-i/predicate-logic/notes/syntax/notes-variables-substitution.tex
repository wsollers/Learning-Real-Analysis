% =========================================================
% Free and Bound Variables; Substitution
% =========================================================

\subsection{Syntax: Free Variables, Bound Variables, and Substitution}

% ---------------------------------------------------------
% TOOLKIT
% ---------------------------------------------------------
\begin{tcolorbox}[colback=gray!6, colframe=gray!40, arc=2pt,
  left=6pt, right=6pt, top=4pt, bottom=4pt,
  title={\small\textbf{Variables and Substitution — Quick Reference}},
  fonttitle=\small\bfseries]
\small
\begin{tabular}{l l l}
\toprule
\textbf{Concept} & \textbf{Meaning} & \textbf{Detail} \\
\midrule
Scope of a quantifier  & Formula governed by the quantifier  & \hyperref[def:scope]{↓ Def} \\
Bound/free occurrence  & Whether $x$ is captured by a quantifier & \hyperref[def:bound-free]{↓ Def} \\
Free variable $\mathrm{FV}(\varphi)$ & Variables whose values affect truth & \hyperref[def:fv]{↓ Def} \\
Sentence               & Formula with $\mathrm{FV}(\varphi)=\varnothing$ & \hyperref[def:sentence]{↓ Def} \\
Substitution $\varphi[t/x]$ & Replace free $x$ by term $t$      & \hyperref[def:subst-notation]{↓ Def} \\
Free for substitution  & No variable in $t$ becomes bound    & \hyperref[def:free-for]{↓ Def} \\
Alpha-equivalence      & Renaming bound variables             & \hyperref[def:alpha-equiv]{↓ Def} \\
\bottomrule
\end{tabular}
\end{tcolorbox}

\vspace{1em}

% ---------------------------------------------------------
% Scope
% ---------------------------------------------------------
\begin{tcolorbox}[colback=propbox, colframe=propborder, arc=2pt,
  left=6pt, right=6pt, top=4pt, bottom=4pt,
  title={\small\textbf{Definition (Scope of a Quantifier)}},
  fonttitle=\small\bfseries]
\label{def:scope}
Let $\varphi$ be a formula of a first-order language.

If $\varphi$ is of the form $\forall x\,\psi$ or $\exists x\,\psi$, then the
formula $\psi$ is called the \emph{scope} of the quantifier.

The quantifier is said to \emph{bind} all occurrences of the variable $x$ that
appear within its scope.
\end{tcolorbox}

\begin{remark}[English reading]
The scope is the reach of a quantifier — the subformula it governs. Scope is
syntactically determined by the parenthesisation of the formula, not by
proximity to the quantifier symbol.
\end{remark}

\begin{example}
In $(\forall x\,P(x)) \wedge Q(x)$, the scope of $\forall x$ is $P(x)$ only.
The occurrence of $x$ in $Q(x)$ falls outside the scope and is free.
\end{example}

% ---------------------------------------------------------
% Bound and Free Occurrences
% ---------------------------------------------------------
\begin{tcolorbox}[colback=propbox, colframe=propborder, arc=2pt,
  left=6pt, right=6pt, top=4pt, bottom=4pt,
  title={\small\textbf{Definition (Bound and Free Occurrences)}},
  fonttitle=\small\bfseries]
\label{def:bound-free}
An occurrence of a variable $x$ in a formula $\varphi$ is \emph{bound} if it
lies within the scope of a quantifier $\forall x$ or $\exists x$.

An occurrence of $x$ is \emph{free} if it is not bound by any quantifier in
$\varphi$.
\end{tcolorbox}

\begin{remark}[English reading]
The same variable can have both bound and free occurrences in a single formula.
Each occurrence is classified independently by checking whether it falls within
the scope of a binding quantifier.
\end{remark}

% ---------------------------------------------------------
% Free Variables FV(φ)
% ---------------------------------------------------------
\begin{tcolorbox}[colback=propbox, colframe=propborder, arc=2pt,
  left=6pt, right=6pt, top=4pt, bottom=4pt,
  title={\small\textbf{Definition (Free Variables of a Formula)}},
  fonttitle=\small\bfseries]
\label{def:fv}
The set $\mathrm{FV}(\varphi)$ of \emph{free variables} of a formula $\varphi$
is defined recursively:
\begin{enumerate}
  \item If $\varphi = P(t_1,\dots,t_n)$ is atomic, then
  $\mathrm{FV}(\varphi) = \bigcup_{i=1}^n \mathrm{Var}(t_i)$,
  where $\mathrm{Var}(t_i)$ is the set of variables in term $t_i$.

  \item $\mathrm{FV}(\neg \varphi) = \mathrm{FV}(\varphi)$.

  \item $\mathrm{FV}(\varphi \circ \psi) = \mathrm{FV}(\varphi) \cup \mathrm{FV}(\psi)$
  for any binary connective $\circ$.

  \item $\mathrm{FV}(\forall x\,\varphi) = \mathrm{FV}(\varphi) \setminus \{x\}$.

  \item $\mathrm{FV}(\exists x\,\varphi) = \mathrm{FV}(\varphi) \setminus \{x\}$.
\end{enumerate}
\end{tcolorbox}

\begin{remark}[English reading]
$\mathrm{FV}(\varphi)$ collects exactly those variables whose values can
influence the truth of $\varphi$ under a structure. Quantifying over $x$
removes $x$ from the free set because the quantifier takes responsibility for
ranging over all (or some) values of $x$.
\end{remark}

\begin{remark}[Common error]
Terms themselves contain only free variables. Variables become bound only
through quantification in formulas. The interpretation of a formula depends
exactly on the values assigned to its free variables.
\end{remark}

% ---------------------------------------------------------
% Sentence
% ---------------------------------------------------------
\begin{tcolorbox}[colback=propbox, colframe=propborder, arc=2pt,
  left=6pt, right=6pt, top=4pt, bottom=4pt,
  title={\small\textbf{Definition (Sentence)}},
  fonttitle=\small\bfseries]
\label{def:sentence}
A \emph{sentence} (or \emph{closed formula}) is a formula with no free variables.

Formally, $\varphi$ is a sentence if and only if $\mathrm{FV}(\varphi) = \varnothing$.
\end{tcolorbox}

\begin{remark}[Consequence]
For sentences, truth depends only on the structure, not on the variable
assignment. This makes sentences the natural objects to be called true or false
in a model, without qualification.
\end{remark}

% ---------------------------------------------------------
% Substitution Notation
% ---------------------------------------------------------
\begin{tcolorbox}[colback=propbox, colframe=propborder, arc=2pt,
  left=6pt, right=6pt, top=4pt, bottom=4pt,
  title={\small\textbf{Definition (Substitution Notation)}},
  fonttitle=\small\bfseries]
\label{def:subst-notation}
Let $\varphi$ be a formula, $x$ a variable, and $t$ a term.

The expression $\varphi[t/x]$ denotes the formula obtained from $\varphi$ by
replacing every \emph{free} occurrence of $x$ with the term $t$, leaving bound
occurrences of $x$ unchanged.
\end{tcolorbox}

% ---------------------------------------------------------
% Free for Substitution
% ---------------------------------------------------------
\begin{tcolorbox}[colback=propbox, colframe=propborder, arc=2pt,
  left=6pt, right=6pt, top=4pt, bottom=4pt,
  title={\small\textbf{Definition (Free for Substitution)}},
  fonttitle=\small\bfseries]
\label{def:free-for}
A term $t$ is \emph{free for} $x$ in $\varphi$ if no free occurrence of $x$ in
$\varphi$ lies within the scope of a quantifier $\forall y$ or $\exists y$
where $y$ is a variable occurring in $t$.

Equivalently, $t$ is free for $x$ in $\varphi$ if the substitution $\varphi[t/x]$
does not result in any variable in $t$ becoming bound.
\end{tcolorbox}

\begin{remark}[Capture-avoiding substitution]
A substitution $\varphi[t/x]$ is admissible only if $t$ is free for $x$ in
$\varphi$. If this condition is violated, a variable occurring in $t$ may
become bound after substitution, silently changing the meaning of the formula.
This is called \emph{variable capture}.
\end{remark}

\begin{remark}[Common error]
Variable capture is one of the most common syntactic mistakes in predicate
logic. Always check that the term being substituted introduces no variables that
fall inside a binding quantifier in the target formula.
\end{remark}

% ---------------------------------------------------------
% Alpha-Equivalence
% ---------------------------------------------------------
\begin{tcolorbox}[colback=propbox, colframe=propborder, arc=2pt,
  left=6pt, right=6pt, top=4pt, bottom=4pt,
  title={\small\textbf{Definition (Alpha-Equivalence)}},
  fonttitle=\small\bfseries]
\label{def:alpha-equiv}
Formulas that differ only by the names of bound variables are logically
equivalent. This is called \emph{alpha-equivalence} (or \emph{alphabetic
variance}):
\[
\forall x\,\varphi \;\equiv\; \forall y\,\varphi[y/x]
\quad\text{(provided $y$ is not free in $\varphi$).}
\]
\end{tcolorbox}

\begin{remark}[Consequence]
Alpha-equivalence is the formal basis for the bound variable renaming used
in prenex normal form conversion and in any proof where variable capture must
be avoided. Two alpha-equivalent formulas are interchangeable in all contexts.
\end{remark}
