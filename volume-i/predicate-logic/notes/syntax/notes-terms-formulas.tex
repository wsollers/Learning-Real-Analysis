% =========================================================
% Terms and Formulas
% =========================================================

\subsection{Syntax of First-Order Logic: Terms and Formulas}

% ---------------------------------------------------------
% TOOLKIT
% ---------------------------------------------------------
\begin{tcolorbox}[colback=gray!6, colframe=gray!40, arc=2pt,
  left=6pt, right=6pt, top=4pt, bottom=4pt,
  title={\small\textbf{Terms and Formulas — Quick Reference}},
  fonttitle=\small\bfseries]
\small
\begin{tabular}{l l l}
\toprule
\textbf{Concept} & \textbf{Meaning} & \textbf{Detail} \\
\midrule
Variable          & Symbol ranging over domain elements    & \hyperref[def:variable]{↓ Def} \\
Term              & Syntactic object denoting a domain element & \hyperref[def:term]{↓ Def} \\
Atomic formula    & Predicate applied to terms             & \hyperref[def:atomic-formula]{↓ Def} \\
Well-formed formula & Recursively constructed formula      & \hyperref[def:wff-pred]{↓ Def} \\
Molecular formula & Non-atomic wff                         & \hyperref[def:molecular]{↓ Def} \\
\bottomrule
\end{tabular}
\end{tcolorbox}

\vspace{1em}

% ---------------------------------------------------------
% Variable
% ---------------------------------------------------------
\begin{tcolorbox}[colback=propbox, colframe=propborder, arc=2pt,
  left=6pt, right=6pt, top=4pt, bottom=4pt,
  title={\small\textbf{Definition (Variable)}},
  fonttitle=\small\bfseries]
\label{def:variable}
A \emph{variable} is a syntactic symbol that ranges over elements of a fixed
domain of discourse.

Variables serve as placeholders in formulas and do not refer to specific objects
until they are assigned values or bound by quantifiers.
\end{tcolorbox}

\begin{remark}[English reading]
Variables are the unknowns of predicate logic. They pick out no particular
object on their own; their value is supplied either by a variable assignment
(semantics) or by a quantifier (syntax).
\end{remark}

\begin{remark}[Consequence]
Because variables have no fixed denotation, the truth of a formula containing
free variables depends on what values are assigned to those variables. This
dependence is tracked formally by the variable assignment function $s$.
\end{remark}

% ---------------------------------------------------------
% Term
% ---------------------------------------------------------
\begin{tcolorbox}[colback=propbox, colframe=propborder, arc=2pt,
  left=6pt, right=6pt, top=4pt, bottom=4pt,
  title={\small\textbf{Definition (Term)}},
  fonttitle=\small\bfseries]
\label{def:term}
A \emph{term} is a syntactic expression intended to denote an object in the
domain of discourse.

The set of terms of a formal language is defined recursively:
\begin{enumerate}
  \item \textbf{Variables.} Every variable is a term.
  \item \textbf{Constants.} Every constant symbol is a term.
  \item \textbf{Function application.} If $f$ is an $n$-ary function symbol and
  $t_1, \dots, t_n$ are terms, then $f(t_1, \dots, t_n)$ is a term.
  \item \textbf{Closure.} No expression is a term unless it can be obtained by
  finitely many applications of rules (1)--(3).
\end{enumerate}
\end{tcolorbox}

\begin{remark}[English reading]
Terms are the noun phrases of predicate logic: they name (or describe) objects
in the domain. Constants name fixed objects; variables name arbitrary ones;
function symbols build complex names from simpler ones.
\end{remark}

\begin{remark}[Fully quantified form]
Terms are purely syntactic objects. Under an interpretation, each term denotes
an element of the domain of discourse, but the term itself is not an object of
the domain.
\end{remark}

% ---------------------------------------------------------
% Atomic Formula
% ---------------------------------------------------------
\begin{tcolorbox}[colback=propbox, colframe=propborder, arc=2pt,
  left=6pt, right=6pt, top=4pt, bottom=4pt,
  title={\small\textbf{Definition (Atomic Formula)}},
  fonttitle=\small\bfseries]
\label{def:atomic-formula}
An \emph{atomic formula} is a well-formed formula obtained by applying an
$n$-ary predicate symbol to $n$ terms.

If $P$ is an $n$-ary predicate symbol and $t_1, \dots, t_n$ are terms, then
\[
P(t_1, \dots, t_n)
\]
is an atomic formula.

Atomic formulas contain no logical connectives or quantifiers and serve as the
base case for the recursive definition of well-formed formulas.
\end{tcolorbox}

\begin{remark}[English reading]
Atomic formulas are the simplest complete statements: they assert that a
predicate (property or relation) holds of specific objects. They are the
predicate-logic counterpart of propositional variables.
\end{remark}

% ---------------------------------------------------------
% Well-Formed Formula
% ---------------------------------------------------------
\begin{tcolorbox}[colback=propbox, colframe=propborder, arc=2pt,
  left=6pt, right=6pt, top=4pt, bottom=4pt,
  title={\small\textbf{Definition (Well-Formed Formula)}},
  fonttitle=\small\bfseries]
\label{def:wff-pred}
The set of \emph{well-formed formulas} (wffs) of a first-order language is
defined recursively:
\begin{enumerate}
  \item \textbf{Atomic formulas.} Every atomic formula is a well-formed formula.
  \item \textbf{Negation.} If $\varphi$ is a formula, then $\neg\varphi$ is a formula.
  \item \textbf{Binary connectives.} If $\varphi$ and $\psi$ are formulas and
  $\circ \in \{\wedge, \vee, \rightarrow, \leftrightarrow\}$, then
  $(\varphi \circ \psi)$ is a formula.
  \item \textbf{Quantification.} If $\varphi$ is a formula and $x$ is a variable,
  then $\forall x\,\varphi$ and $\exists x\,\varphi$ are formulas.
  \item \textbf{Closure.} No expression is a formula unless it can be obtained
  by finitely many applications of rules (1)--(4).
\end{enumerate}
\end{tcolorbox}

\begin{remark}[English reading]
The wff definition says exactly which strings of symbols count as grammatical
sentences of first-order logic. Everything in the language is built bottom-up
from atomic formulas by the four formation rules.
\end{remark}

\begin{remark}[Distinction from propositional logic]
Rule (4) is new: quantifiers bind variables and create formulas from formulas.
This extra layer is what makes predicate logic more expressive than
propositional logic.
\end{remark}

% ---------------------------------------------------------
% Molecular Formula
% ---------------------------------------------------------
\begin{tcolorbox}[colback=propbox, colframe=propborder, arc=2pt,
  left=6pt, right=6pt, top=4pt, bottom=4pt,
  title={\small\textbf{Definition (Molecular Formula)}},
  fonttitle=\small\bfseries]
\label{def:molecular}
A \emph{molecular formula} is a well-formed formula that is not atomic.

Equivalently, a formula is molecular if it is formed from one or more atomic
formulas by the application of logical connectives or quantifiers.
\end{tcolorbox}

\begin{remark}[English reading]
Atomic formulas express basic properties or relations. Molecular formulas
express compound statements built from atomic formulas using the logical
apparatus of the language.
\end{remark}
