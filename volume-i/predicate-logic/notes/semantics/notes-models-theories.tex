% =========================================================
% Models, Theories, Logical Consequence, Logical Equivalence
% =========================================================

\subsection{Semantics: Models, Theories, and Logical Consequence}

% ---------------------------------------------------------
% TOOLKIT
% ---------------------------------------------------------
\begin{tcolorbox}[colback=gray!6, colframe=gray!40, arc=2pt,
  left=6pt, right=6pt, top=4pt, bottom=4pt,
  title={\small\textbf{Models and Theories — Quick Reference}},
  fonttitle=\small\bfseries]
\small
\begin{tabular}{l l l}
\toprule
\textbf{Concept} & \textbf{Meaning} & \textbf{Detail} \\
\midrule
Model $\mathcal{M} \models \Gamma$ & Structure satisfying all sentences of $\Gamma$ & \hyperref[def:model]{↓ Def} \\
Theory & Set of sentences; consistent if it has a model & \hyperref[def:theory]{↓ Def} \\
Logical consequence $\Gamma \models \varphi$ & True in every model of $\Gamma$ & \hyperref[def:log-consequence]{↓ Def} \\
Logical equivalence $\varphi \equiv \psi$ & Mutually entailing formulas & \hyperref[def:log-equiv]{↓ Def} \\
\bottomrule
\end{tabular}
\end{tcolorbox}

\vspace{1em}

% ---------------------------------------------------------
% Model
% ---------------------------------------------------------
\begin{tcolorbox}[colback=propbox, colframe=propborder, arc=2pt,
  left=6pt, right=6pt, top=4pt, bottom=4pt,
  title={\small\textbf{Definition (Model)}},
  fonttitle=\small\bfseries]
\label{def:model}
Let $\varphi$ be a sentence and $\mathcal{M}$ a structure. We say that
$\mathcal{M}$ is a \emph{model} of $\varphi$ if $\mathcal{M} \models \varphi$.

More generally, if $\Gamma$ is a set of sentences, then $\mathcal{M}$ is a
model of $\Gamma$ if $\mathcal{M} \models \gamma$ for every $\gamma \in \Gamma$.
In this case, we write $\mathcal{M} \models \Gamma$.
\end{tcolorbox}

% ---------------------------------------------------------
% Theory
% ---------------------------------------------------------
\begin{tcolorbox}[colback=propbox, colframe=propborder, arc=2pt,
  left=6pt, right=6pt, top=4pt, bottom=4pt,
  title={\small\textbf{Definition (Theory)}},
  fonttitle=\small\bfseries]
\label{def:theory}
A \emph{theory} is a set of sentences.

A theory $T$ is \emph{satisfiable} (or \emph{consistent}) if it has at least
one model.

A theory $T$ is \emph{complete} if for every sentence $\varphi$ in the
language, either $T \models \varphi$ or $T \models \neg\varphi$.
\end{tcolorbox}

\begin{remark}[Source note]
In some texts, a theory is required to be closed under logical consequence: if
$T \models \varphi$ then $\varphi \in T$. In other texts, a theory is simply
any set of axioms from which consequences are derived. Both usages appear in
the literature.
\end{remark}

% ---------------------------------------------------------
% Logical Consequence
% ---------------------------------------------------------
\begin{tcolorbox}[colback=propbox, colframe=propborder, arc=2pt,
  left=6pt, right=6pt, top=4pt, bottom=4pt,
  title={\small\textbf{Definition (Logical Consequence)}},
  fonttitle=\small\bfseries]
\label{def:log-consequence}
Let $\Gamma$ be a set of formulas and $\varphi$ a formula. We say that
$\varphi$ is a \emph{logical consequence} of $\Gamma$, written
$\Gamma \models \varphi$, if for every structure $\mathcal{M}$ and every
variable assignment $s$:
\[
\text{if } \mathcal{M}, s \models \gamma \text{ for all } \gamma \in \Gamma,
\text{ then } \mathcal{M}, s \models \varphi.
\]
Equivalently, every model of $\Gamma$ is a model of $\varphi$.
\end{tcolorbox}

\begin{remark}[English reading]
Logical consequence says: if all the hypotheses in $\Gamma$ are true in some
structure, then $\varphi$ must be true there too. This is the semantic notion
of entailment.
\end{remark}

\begin{remark}[Notation]
The notation $\models \varphi$ (empty left-hand side) means $\varphi$ is
valid: it is a logical consequence of the empty set of premises.
\end{remark}

% ---------------------------------------------------------
% Logical Equivalence
% ---------------------------------------------------------
\begin{tcolorbox}[colback=propbox, colframe=propborder, arc=2pt,
  left=6pt, right=6pt, top=4pt, bottom=4pt,
  title={\small\textbf{Definition (Logical Equivalence)}},
  fonttitle=\small\bfseries]
\label{def:fol-log-equiv}
Two formulas $\varphi$ and $\psi$ are \emph{logically equivalent}, written
$\varphi \equiv \psi$, if each is a logical consequence of the other:
\[
\varphi \equiv \psi
\quad\Longleftrightarrow\quad
(\varphi \models \psi \text{ and } \psi \models \varphi).
\]
Equivalently, $\varphi$ and $\psi$ have the same truth value in every structure
under every variable assignment.
\end{tcolorbox}

\begin{remark}[Consequence]
Logical equivalence is the semantic counterpart of provable bi-implication.
Two logically equivalent formulas are interchangeable in any context without
changing truth values.
\end{remark}
