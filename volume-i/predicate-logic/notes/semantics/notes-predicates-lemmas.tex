% =========================================================
% Predicates, Relations, and Substitution Lemmas
% =========================================================

\subsection{Semantics: Predicates, Relations, and Substitution Lemmas}

% ---------------------------------------------------------
% TOOLKIT
% ---------------------------------------------------------
\begin{tcolorbox}[colback=gray!6, colframe=gray!40, arc=2pt,
  left=6pt, right=6pt, top=4pt, bottom=4pt,
  title={\small\textbf{Predicates and Substitution Lemmas — Quick Reference}},
  fonttitle=\small\bfseries]
\small
\begin{tabular}{l l l}
\toprule
\textbf{Concept} & \textbf{Meaning} & \textbf{Detail} \\
\midrule
Predicate & Open formula or truth-valued function on domain & \hyperref[def:predicate]{↓ Def} \\
Substitution Lemma (terms) & Assign vs.\ substitute, same result & \hyperref[lem:subst-terms]{↓ Lem} \\
Substitution Lemma (formulas) & Syntactic substitution $=$ semantic update & \hyperref[lem:subst-formulas]{↓ Lem} \\
\bottomrule
\end{tabular}
\end{tcolorbox}

\vspace{1em}

% ---------------------------------------------------------
% Predicate
% ---------------------------------------------------------
\begin{tcolorbox}[colback=propbox, colframe=propborder, arc=2pt,
  left=6pt, right=6pt, top=4pt, bottom=4pt,
  title={\small\textbf{Definition (Predicate)}},
  fonttitle=\small\bfseries]
\label{def:predicate}
A \emph{predicate} is an expression containing one or more variables that
represents a property or relation and becomes a proposition when all its
variables are instantiated.

Formally, a predicate has two equivalent views:
\begin{enumerate}
  \item \textbf{Syntactic view.} A predicate is an open formula
  $\varphi(x_1,\dots,x_n)$ that has no truth value until specific objects are
  substituted for its variables.
  \item \textbf{Semantic view.} Given a domain $D$, an $n$-ary predicate
  determines a function $P : D^n \to \{\mathsf{T}, \mathsf{F}\}$.
\end{enumerate}
\end{tcolorbox}

\begin{remark}[Predicates vs.\ relations]
A \emph{predicate} is a syntactic or semantic device: syntactically, an open
formula; semantically, a truth-valued function. A \emph{relation} is a purely
set-theoretic object: an $n$-ary relation on $D$ is a subset $R \subseteq D^n$.
Under an interpretation, predicates and relations correspond via
$P(a_1,\dots,a_n) \text{ is true} \iff (a_1,\dots,a_n) \in R$. Thus predicates
belong to the language of logic, while relations belong to the structures
interpreting that language.
\end{remark}

% ---------------------------------------------------------
% Substitution Lemmas
% ---------------------------------------------------------

\begin{lemma}[Substitution Lemma for Terms]\label{lem:subst-terms}
Let $\mathcal{M}$ be a structure, $s$ a variable assignment, $x$ a variable,
and $d \in D$. For any term $t$,
\[
\llbracket t \rrbracket_{\mathcal{M},\,s[x\mapsto d]}
=
\llbracket t[d/x] \rrbracket_{\mathcal{M},\,s}.
\]
\end{lemma}

\begin{remark}[English reading]
Evaluating a term after modifying an assignment is the same as substituting
the value directly into the term and then evaluating. This shows that syntactic
substitution and semantic update commute for terms.
\end{remark}

\begin{lemma}[Substitution Lemma for Formulas]\label{lem:subst-formulas}
Let $\varphi$ be a formula, $t$ a term free for $x$ in $\varphi$. Then for
any structure $\mathcal{M}$ and assignment $s$,
\[
\mathcal{M},s \models \varphi[t/x]
\quad\Longleftrightarrow\quad
\mathcal{M},\,s\bigl[x\mapsto \llbracket t \rrbracket_{\mathcal{M},s}\bigr] \models \varphi.
\]
\end{lemma}

\begin{remark}[English reading]
Satisfying $\varphi$ with $x$ replaced by $t$ is the same as satisfying
$\varphi$ under the assignment that sends $x$ to the value of $t$. This lemma
formally connects syntactic substitution with semantic evaluation and is
essential for proving the soundness of Universal Instantiation.
\end{remark}

\begin{remark}[Proof strategy]
Both lemmas are proved by structural induction on the term (Lemma for terms)
or formula (Lemma for formulas). The key cases are variables and quantifiers;
the remaining cases follow directly from the inductive hypothesis.
\end{remark}
