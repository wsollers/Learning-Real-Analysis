% =========================================================
% Satisfaction, Truth, Validity, Satisfiability
% =========================================================

\subsection{Semantics: Satisfaction and Truth}

% ---------------------------------------------------------
% TOOLKIT
% ---------------------------------------------------------
\begin{tcolorbox}[colback=gray!6, colframe=gray!40, arc=2pt,
  left=6pt, right=6pt, top=4pt, bottom=4pt,
  title={\small\textbf{Satisfaction and Truth — Quick Reference}},
  fonttitle=\small\bfseries]
\small
\begin{tabular}{l l l}
\toprule
\textbf{Concept} & \textbf{Meaning} & \textbf{Detail} \\
\midrule
Satisfaction $\mathcal{M}, s \models \varphi$ & Formula true in $\mathcal{M}$ under $s$ & \hyperref[def:satisfaction]{↓ Def} \\
Truth in a structure $\mathcal{M} \models \varphi$ & True for every assignment  & \hyperref[def:truth-structure]{↓ Def} \\
Validity / logical truth & True in every structure    & \hyperref[def:validity]{↓ Def} \\
Satisfiability & True in some structure & \hyperref[def:validity]{↓ Def} \\
\bottomrule
\end{tabular}
\end{tcolorbox}

\vspace{1em}

% ---------------------------------------------------------
% Satisfaction
% ---------------------------------------------------------
\begin{tcolorbox}[colback=propbox, colframe=propborder, arc=2pt,
  left=6pt, right=6pt, top=4pt, bottom=4pt,
  title={\small\textbf{Definition (Satisfaction)}},
  fonttitle=\small\bfseries]
\label{def:fol-satisfaction}
Let $\mathcal{M} = \langle D, I \rangle$ be a structure, $s$ a variable
assignment, and $\varphi$ a formula. The relation $\mathcal{M}, s \models \varphi$
(read ``$\mathcal{M}$ satisfies $\varphi$ under $s$'') is defined recursively:
\begin{enumerate}
  \item \textbf{Atomic.}
  $\mathcal{M}, s \models P(t_1, \dots, t_n) \iff
  (\llbracket t_1 \rrbracket_{\mathcal{M},s}, \dots,
   \llbracket t_n \rrbracket_{\mathcal{M},s}) \in I(P)$.

  \item \textbf{Equality.}
  $\mathcal{M}, s \models (t_1 = t_2) \iff
  \llbracket t_1 \rrbracket_{\mathcal{M},s} = \llbracket t_2 \rrbracket_{\mathcal{M},s}$.

  \item \textbf{Negation.}
  $\mathcal{M}, s \models \neg\varphi \iff \mathcal{M}, s \not\models \varphi$.

  \item \textbf{Conjunction.}
  $\mathcal{M}, s \models (\varphi \wedge \psi) \iff
  \mathcal{M}, s \models \varphi \text{ and } \mathcal{M}, s \models \psi$.

  \item \textbf{Disjunction.}
  $\mathcal{M}, s \models (\varphi \vee \psi) \iff
  \mathcal{M}, s \models \varphi \text{ or } \mathcal{M}, s \models \psi$.

  \item \textbf{Implication.}
  $\mathcal{M}, s \models (\varphi \rightarrow \psi) \iff
  \mathcal{M}, s \not\models \varphi \text{ or } \mathcal{M}, s \models \psi$.

  \item \textbf{Biconditional.}
  $\mathcal{M}, s \models (\varphi \leftrightarrow \psi) \iff
  (\mathcal{M}, s \models \varphi \Leftrightarrow \mathcal{M}, s \models \psi)$.

  \item \textbf{Universal quantification.}
  $\mathcal{M}, s \models \forall x\,\varphi \iff
  \text{for all } d \in D,\; \mathcal{M}, s[x \mapsto d] \models \varphi$.

  \item \textbf{Existential quantification.}
  $\mathcal{M}, s \models \exists x\,\varphi \iff
  \text{there exists } d \in D \text{ such that }
  \mathcal{M}, s[x \mapsto d] \models \varphi$.
\end{enumerate}
\end{tcolorbox}

\begin{remark}[English reading]
Satisfaction is the bridge between syntax and semantics. It tells us, for each
formula formation rule, what it means for the formula to hold. Clauses~(1)--(7)
mirror the propositional connectives; clauses~(8)--(9) are the new quantifier
clauses unique to first-order logic.
\end{remark}

\begin{remark}[Fully quantified form for universal quantification]
The clause $\mathcal{M}, s \models \forall x\,\varphi$ holds if and only if
for every element $d$ of the domain $D$, the formula $\varphi$ is satisfied
under the assignment that agrees with $s$ everywhere except that it sends $x$
to $d$. This is what it means for a property to hold universally.
\end{remark}

% ---------------------------------------------------------
% Truth in a Structure
% ---------------------------------------------------------
\begin{tcolorbox}[colback=propbox, colframe=propborder, arc=2pt,
  left=6pt, right=6pt, top=4pt, bottom=4pt,
  title={\small\textbf{Definition (Truth in a Structure)}},
  fonttitle=\small\bfseries]
\label{def:truth-structure}
A sentence $\varphi$ is \emph{true} in a structure $\mathcal{M}$, written
$\mathcal{M} \models \varphi$, if $\mathcal{M}, s \models \varphi$ for every
variable assignment $s$.

A sentence $\varphi$ is \emph{false} in $\mathcal{M}$ if $\mathcal{M}
\not\models \varphi$.
\end{tcolorbox}

\begin{remark}[English reading]
For sentences (formulas with no free variables), the choice of assignment is
irrelevant — all assignments agree. So $\mathcal{M} \models \varphi$ is
well-defined without reference to any particular $s$.
\end{remark}

% ---------------------------------------------------------
% Validity and Satisfiability
% ---------------------------------------------------------
\begin{tcolorbox}[colback=propbox, colframe=propborder, arc=2pt,
  left=6pt, right=6pt, top=4pt, bottom=4pt,
  title={\small\textbf{Definition (Validity and Satisfiability)}},
  fonttitle=\small\bfseries]
\label{def:validity}
A formula $\varphi$ is:
\begin{itemize}
  \item \emph{valid} (or a \emph{logical truth}) if $\mathcal{M}, s \models \varphi$
  for every structure $\mathcal{M}$ and every assignment $s$;
  \item \emph{satisfiable} if $\mathcal{M}, s \models \varphi$ for some $\mathcal{M}$
  and some $s$;
  \item \emph{unsatisfiable} (or a \emph{contradiction}) if it is not satisfiable.
\end{itemize}
\end{tcolorbox}

\begin{remark}[Consequence]
A formula is valid if no structure can make it false. These are the logical
truths of predicate logic — the formulas provable by logic alone, without any
special assumptions about the domain.
\end{remark}
