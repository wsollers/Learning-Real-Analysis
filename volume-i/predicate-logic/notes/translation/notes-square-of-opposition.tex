% =========================================================
% Square of Opposition
% =========================================================

\subsection{Translation: Square of Opposition}

% ---------------------------------------------------------
% TOOLKIT
% ---------------------------------------------------------
\begin{tcolorbox}[colback=gray!6, colframe=gray!40, arc=2pt,
  left=6pt, right=6pt, top=4pt, bottom=4pt,
  title={\small\textbf{Square of Opposition — Quick Reference}},
  fonttitle=\small\bfseries]
\small
\begin{tabular}{l l l l}
\toprule
\textbf{Form} & \textbf{Name} & \textbf{Formula} & \textbf{Detail} \\
\midrule
A & Universal Affirmative  & $\forall x\,P(x)$       & \hyperref[def:opp-forms]{↓ Def} \\
E & Universal Negative     & $\forall x\,\neg P(x)$  & \hyperref[def:opp-forms]{↓ Def} \\
I & Existential Affirmative & $\exists x\,P(x)$      & \hyperref[def:opp-forms]{↓ Def} \\
O & Existential Negative   & $\exists x\,\neg P(x)$  & \hyperref[def:opp-forms]{↓ Def} \\
\bottomrule
\end{tabular}
\end{tcolorbox}

\vspace{1em}

\begin{tcolorbox}[colback=propbox, colframe=propborder, arc=2pt,
  left=6pt, right=6pt, top=4pt, bottom=4pt,
  title={\small\textbf{Definition (Quantified Opposition Forms)}},
  fonttitle=\small\bfseries]
\label{def:opp-forms}
Let $P(x)$ be a formula with one free variable. The four standard quantified
forms are:
\begin{align*}
\textbf{Universal Affirmative (A):} &\quad \forall x\,P(x) \\
\textbf{Universal Negative (E):} &\quad \forall x\,\neg P(x) \\
\textbf{Existential Affirmative (I):} &\quad \exists x\,P(x) \\
\textbf{Existential Negative (O):} &\quad \exists x\,\neg P(x)
\end{align*}
\end{tcolorbox}

\vspace{1em}

\begin{center}
\begin{tikzpicture}[
  every node/.style={draw, rectangle, align=center, minimum width=4cm, minimum height=1.2cm},
  node distance=3.8cm
]

\node (A) {$\forall x\,P(x)$\\\small Universal Affirmative (A)};
\node (E) [right=of A] {$\forall x\,\neg P(x)$\\\small Universal Negative (E)};
\node (I) [below=of A] {$\exists x\,P(x)$\\\small Existential Affirmative (I)};
\node (O) [below=of E] {$\exists x\,\neg P(x)$\\\small Existential Negative (O)};

\draw[<->, thick] (A) -- (O);
\draw[<->, thick] (E) -- (I);
\draw[->, thick]  (A) -- (I);
\draw[->, thick]  (E) -- (O);
\draw[<->, dashed] (A) -- (E);
\draw[<->, dashed] (I) -- (O);

\end{tikzpicture}
\end{center}

\begin{remark}[Logical relations in first-order logic with nonempty domains]
\begin{itemize}
  \item \textbf{Contradictories:} A and O ($\forall x\,P(x)$ vs $\exists x\,\neg P(x)$);
  E and I ($\forall x\,\neg P(x)$ vs $\exists x\,P(x)$). These pairs cannot both
  be true and cannot both be false.
  \item \textbf{Subalternation:} A entails I ($\forall x\,P(x) \Rightarrow \exists x\,P(x)$);
  E entails O ($\forall x\,\neg P(x) \Rightarrow \exists x\,\neg P(x)$). These
  hold in classical first-order logic with nonempty domains.
  \item \textbf{Contraries and subcontraries} do not generally hold in first-order
  logic without existential presuppositions.
\end{itemize}
\end{remark}

\begin{remark}[Modern status of the square]
In classical first-order logic, only contradiction and subalternation are
logically valid relations. The traditional notions of contrariety and
subcontrariety rely on existential assumptions and are not preserved in general
model-theoretic semantics.
\end{remark}
