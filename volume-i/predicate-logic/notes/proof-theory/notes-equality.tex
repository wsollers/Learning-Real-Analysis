% =========================================================
% Equality Rules
% =========================================================

\subsection{Proof Theory: Equality Rules}

% ---------------------------------------------------------
% TOOLKIT
% ---------------------------------------------------------
\begin{tcolorbox}[colback=gray!6, colframe=gray!40, arc=2pt,
  left=6pt, right=6pt, top=4pt, bottom=4pt,
  title={\small\textbf{Equality Rules — Quick Reference}},
  fonttitle=\small\bfseries]
\small
\begin{tabular}{l l l l}
\toprule
\textbf{Rule} & \textbf{Schema} & \textbf{Status} & \textbf{Detail} \\
\midrule
Reflexivity (EqI)   & $t = t$ & Primitive rule & \hyperref[def:eq-refl]{↓ Def} \\
Equality Elim (EqE) & $t_1=t_2,\;\varphi[t_1/x] \Rightarrow \varphi[t_2/x]$ & Primitive rule & \hyperref[def:eq-elim]{↓ Def} \\
Symmetry            & $t_1=t_2 \Rightarrow t_2=t_1$ & Derived & \hyperref[thm:eq-sym]{↓ Thm} \\
Transitivity        & $t_1=t_2,\;t_2=t_3 \Rightarrow t_1=t_3$ & Derived & \hyperref[thm:eq-trans]{↓ Thm} \\
Term substitution   & $f(\dots,t_1,\dots)=f(\dots,t_2,\dots)$ & Derived & \hyperref[thm:eq-term]{↓ Thm} \\
Predicate substitution & $P(\dots,t_1,\dots)\Leftrightarrow P(\dots,t_2,\dots)$ & Derived & \hyperref[thm:eq-pred]{↓ Thm} \\
\bottomrule
\end{tabular}
\end{tcolorbox}

\vspace{1em}

% ---------------------------------------------------------
% Equality Introduction (Reflexivity)
% ---------------------------------------------------------
\begin{tcolorbox}[colback=propbox, colframe=propborder, arc=2pt,
  left=6pt, right=6pt, top=4pt, bottom=4pt,
  title={\small\textbf{Definition (Equality Introduction — Reflexivity)}},
  fonttitle=\small\bfseries]
\label{def:eq-refl}
For any term $t$, one may infer $t = t$.
\end{tcolorbox}

\begin{remark}[English reading]
Reflexivity requires no premises and holds for all terms under all
interpretations. It is the foundational rule that allows equality reasoning to
get off the ground.
\end{remark}

% ---------------------------------------------------------
% Equality Elimination
% ---------------------------------------------------------
\begin{tcolorbox}[colback=propbox, colframe=propborder, arc=2pt,
  left=6pt, right=6pt, top=4pt, bottom=4pt,
  title={\small\textbf{Definition (Equality Elimination — Substitution of Equals)}},
  fonttitle=\small\bfseries]
\label{def:eq-elim}
From $t_1 = t_2$ and $\varphi[t_1/x]$, one may infer $\varphi[t_2/x]$.
\end{tcolorbox}

\begin{remark}[English reading]
Equality elimination permits replacing a term by an equal term in any formula
position, provided the substitution is capture-avoiding. This formalizes the
intuition that equal things can be substituted for each other.
\end{remark}

% ---------------------------------------------------------
% Derived Rules
% ---------------------------------------------------------

\begin{theorem}[Symmetry of Equality]\label{thm:eq-sym}
From $t_1 = t_2$, one may infer $t_2 = t_1$.
\end{theorem}

\begin{theorem}[Transitivity of Equality]\label{thm:eq-trans}
From $t_1 = t_2$ and $t_2 = t_3$, one may infer $t_1 = t_3$.
\end{theorem}

\begin{theorem}[Term Substitution under Equality]\label{thm:eq-term}
If $t_1 = t_2$, then for any function symbol $f$,
$f(\dots,t_1,\dots) = f(\dots,t_2,\dots)$.
\end{theorem}

\begin{theorem}[Predicate Substitution under Equality]\label{thm:eq-pred}
If $t_1 = t_2$ and $P$ is an $n$-ary predicate symbol, then
$P(\dots,t_1,\dots) \;\Leftrightarrow\; P(\dots,t_2,\dots)$.
\end{theorem}

\begin{remark}[Summary]
These derived rules collectively express that functions and predicates
\emph{respect equality}: equal inputs produce equal outputs (for functions) or
equivalent propositions (for predicates). Together they amount to the principle
of Leibniz substitutivity.
\end{remark}

\begin{remark}[Common error]
When substituting equals for equals, the substitution must be made consistently.
Replacing a term by an equal in one occurrence but not another in the same
formula can produce incorrect conclusions.
\end{remark}
