% =========================================================
% Inference Rules for Quantifiers
% =========================================================

\subsection{Proof Theory: Quantifier Inference Rules}

% ---------------------------------------------------------
% TOOLKIT
% ---------------------------------------------------------
\begin{tcolorbox}[colback=gray!6, colframe=gray!40, arc=2pt,
  left=6pt, right=6pt, top=4pt, bottom=4pt,
  title={\small\textbf{Quantifier Inference Rules — Quick Reference}},
  fonttitle=\small\bfseries]
\small
\begin{tabular}{l l l l}
\toprule
\textbf{Rule} & \textbf{Schema} & \textbf{Key restriction} & \textbf{Detail} \\
\midrule
UI  & $\forall x\,\varphi \Rightarrow \varphi[t/x]$ & $t$ free for $x$ & \hyperref[def:ui]{↓ Def} \\
UG  & $\varphi \Rightarrow \forall x\,\varphi$ & $x$ not free in assumptions & \hyperref[def:ug]{↓ Def} \\
EI  & $\exists x\,\varphi \Rightarrow \varphi[c/x]$ & $c$ fresh witness & \hyperref[def:ei]{↓ Def} \\
EG  & $\varphi[t/x] \Rightarrow \exists x\,\varphi$ & none & \hyperref[def:eg]{↓ Def} \\
\bottomrule
\end{tabular}
\end{tcolorbox}

\vspace{1em}

% ---------------------------------------------------------
% Universal Instantiation
% ---------------------------------------------------------
\begin{tcolorbox}[colback=propbox, colframe=propborder, arc=2pt,
  left=6pt, right=6pt, top=4pt, bottom=4pt,
  title={\small\textbf{Definition (Universal Instantiation — UI)}},
  fonttitle=\small\bfseries]
\label{def:ui}
From a universally quantified statement, infer any instance obtained by
substituting a term for the bound variable:
\[
\forall x\,\varphi \;\Rightarrow\; \varphi[t/x].
\]
\end{tcolorbox}

\begin{remark}[Side condition]
The term $t$ must be free for $x$ in $\varphi$. If $t$ contains a variable
that would become bound after substitution, the inference is invalid.
\end{remark}

% ---------------------------------------------------------
% Universal Generalization
% ---------------------------------------------------------
\begin{tcolorbox}[colback=propbox, colframe=propborder, arc=2pt,
  left=6pt, right=6pt, top=4pt, bottom=4pt,
  title={\small\textbf{Definition (Universal Generalization — UG)}},
  fonttitle=\small\bfseries]
\label{def:ug}
From a formula, infer a universally quantified statement:
\[
\varphi \;\Rightarrow\; \forall x\,\varphi.
\]
\end{tcolorbox}

\begin{remark}[Restriction on UG]
The variable $x$ must not occur free in any undischarged assumption on which
$\varphi$ depends. This restriction ensures that $x$ is truly arbitrary —
not tied to a specific hypothesis about $x$.
\end{remark}

\begin{remark}[Arbitrary element argument]
To prove $\forall x\,\varphi(x)$, fix an arbitrary element $x$ (introducing no
special assumptions about $x$), prove $\varphi(x)$, and then apply UG. The
variable $x$ must remain unconstrained throughout.
\end{remark}

% ---------------------------------------------------------
% Existential Instantiation
% ---------------------------------------------------------
\begin{tcolorbox}[colback=propbox, colframe=propborder, arc=2pt,
  left=6pt, right=6pt, top=4pt, bottom=4pt,
  title={\small\textbf{Definition (Existential Instantiation — EI)}},
  fonttitle=\small\bfseries]
\label{def:ei}
From an existentially quantified statement, infer an instance with a fresh
constant (witness):
\[
\exists x\,\varphi \;\Rightarrow\; \varphi[c/x].
\]
\end{tcolorbox}

\begin{remark}[Witness discipline for EI]
The constant $c$ must be fresh: it must not appear in $\varphi$, in any
undischarged assumption, or in the final conclusion of the proof. The constant
represents an arbitrary but fixed witness to the existential claim, not a
specific known object.
\end{remark}

\begin{remark}[Common error]
Using a witness constant introduced by EI outside its allowed scope, or
choosing $c$ before establishing the existential premise, invalidates the proof.
\end{remark}

% ---------------------------------------------------------
% Existential Generalization
% ---------------------------------------------------------
\begin{tcolorbox}[colback=propbox, colframe=propborder, arc=2pt,
  left=6pt, right=6pt, top=4pt, bottom=4pt,
  title={\small\textbf{Definition (Existential Generalization — EG)}},
  fonttitle=\small\bfseries]
\label{def:eg}
From a formula containing a term, infer an existential statement:
\[
\varphi[t/x] \;\Rightarrow\; \exists x\,\varphi.
\]
\end{tcolorbox}

\begin{remark}[No side conditions]
EG has no side conditions: any term $t$ may be replaced by an existentially
quantified variable. To prove $\exists x\,\varphi(x)$, explicitly exhibit a
term $t$ such that $\varphi(t)$ holds, then apply EG.
\end{remark}

\begin{remark}[Counterexample argument]
To refute $\forall x\,\varphi(x)$: produce a single term $t$ such that
$\neg\varphi(t)$ holds. To refute $\exists x\,\varphi(x)$: show that $\varphi(t)$
fails for every term $t$ in the domain.
\end{remark}
