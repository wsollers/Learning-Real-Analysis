% =========================================================
% Soundness and Completeness of First-Order Logic
% =========================================================

\subsection{Proof Theory: Soundness and Completeness}

% ---------------------------------------------------------
% TOOLKIT
% ---------------------------------------------------------
\begin{tcolorbox}[colback=gray!6, colframe=gray!40, arc=2pt,
  left=6pt, right=6pt, top=4pt, bottom=4pt,
  title={\small\textbf{Soundness and Completeness — Quick Reference}},
  fonttitle=\small\bfseries]
\small
\begin{tabular}{l l l}
\toprule
\textbf{Property} & \textbf{Direction} & \textbf{Detail} \\
\midrule
Soundness    & $\Gamma \vdash \varphi \Rightarrow \Gamma \models \varphi$ & \hyperref[def:soundness]{↓ Def} \\
Completeness & $\Gamma \models \varphi \Rightarrow \Gamma \vdash \varphi$ & \hyperref[def:completeness]{↓ Def} \\
Soundness Theorem (FOL) & All standard FOL rules are sound & \hyperref[thm:fol-sound]{↓ Thm} \\
Gödel's Completeness Theorem & FOL is complete & \hyperref[thm:godel-complete]{↓ Thm} \\
\bottomrule
\end{tabular}
\end{tcolorbox}

\vspace{1em}

% ---------------------------------------------------------
% Soundness and Completeness (Definitions)
% ---------------------------------------------------------
\begin{tcolorbox}[colback=propbox, colframe=propborder, arc=2pt,
  left=6pt, right=6pt, top=4pt, bottom=4pt,
  title={\small\textbf{Definition (Soundness)}},
  fonttitle=\small\bfseries]
\label{def:soundness}
A proof system is \emph{sound} if every provable formula is valid:
\[
\Gamma \vdash \varphi \implies \Gamma \models \varphi.
\]
\end{tcolorbox}

\begin{tcolorbox}[colback=propbox, colframe=propborder, arc=2pt,
  left=6pt, right=6pt, top=4pt, bottom=4pt,
  title={\small\textbf{Definition (Completeness)}},
  fonttitle=\small\bfseries]
\label{def:completeness}
A proof system is \emph{complete} if every valid formula is provable:
\[
\Gamma \models \varphi \implies \Gamma \vdash \varphi.
\]
\end{tcolorbox}

% ---------------------------------------------------------
% Soundness Theorem for FOL
% ---------------------------------------------------------
\begin{theorem}[Soundness of First-Order Logic]\label{thm:fol-sound}
The standard inference rules for first-order logic (UI, UG, EI, EG, and the
propositional rules) are sound: they preserve truth in all structures.

If $\Gamma \vdash \varphi$, then $\Gamma \models \varphi$.
\end{theorem}

% ---------------------------------------------------------
% Gödel's Completeness Theorem
% ---------------------------------------------------------
\begin{tcolorbox}[colback=thmbox, colframe=thmborder, arc=2pt,
  left=6pt, right=6pt, top=4pt, bottom=4pt,
  title={\small\textbf{Theorem (Gödel's Completeness Theorem)}},
  fonttitle=\small\bfseries]
\label{thm:godel-complete}
First-order logic is complete: every logically valid formula is provable.
\[
\Gamma \models \varphi \implies \Gamma \vdash \varphi.
\]
Equivalently: if a set of sentences $\Gamma$ is consistent (has no proof of
contradiction), then $\Gamma$ has a model.
\end{tcolorbox}

\begin{remark}[English reading]
Soundness ensures proofs do not lead us astray: we cannot prove false statements
from true premises. Completeness ensures proofs are powerful enough: every
statement that is true in all models can be established by a formal proof.

Together, soundness and completeness show that syntactic provability ($\vdash$)
and semantic entailment ($\models$) coincide for first-order logic.
\end{remark}

\begin{remark}[Completeness vs.\ incompleteness]
Gödel's completeness theorem (1930) should not be confused with his
incompleteness theorems (1931). The completeness theorem concerns first-order
logic as a proof system. The incompleteness theorems concern the limitations of
formal theories capable of expressing arithmetic, and are a separate result.
\end{remark}
