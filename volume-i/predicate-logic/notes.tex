% =========================================================
% Breadcrumb
% =========================================================
\begin{tcolorbox}[
  colback=gray!6,
  colframe=gray!40,
  arc=2pt,
  left=8pt, right=8pt, top=6pt, bottom=6pt,
  title={\small\textbf{Where You Are in the Journey}},
  fonttitle=\small\bfseries
]
\begin{center}
\small
Propositional Logic
$\;\to\;$ \textbf{Predicate Calculus}
$\;\to\;$ Sets \& Functions
$\;\to\;$ Proof Techniques
$\;\to\;$ Axiom Systems
$\;\to\;$ $\mathbb{N}$, $\mathbb{Z}$, $\mathbb{Q}$, $\mathbb{R}$
$\;\to\;$ Real Analysis
$\;\to\;$ $\cdots$
\end{center}

\medskip
\noindent\textbf{How we got here.}
Propositional logic gave us a language for combining truth values with
connectives. But it cannot express ``every integer has a successor'' or
``there exists a prime greater than 100'' --- statements that quantify
over objects. Predicate logic provides the missing machinery.

\medskip
\noindent\textbf{What this chapter builds.}
We extend propositional syntax with variables, constants, function
symbols, predicate symbols, and quantifiers.
The semantics are now structures: a domain of objects plus
interpretations of every symbol.
The proof theory adds quantifier introduction and elimination rules.
Soundness and completeness (Gödel's theorem) are the major metatheorems.

\medskip
\noindent\textbf{Where this leads.}
Every subsequent chapter uses predicate logic as its underlying language.
The axiomatic construction of $\mathbb{N}$ is a predicate-logic theory.
Set theory, ring theory, and topology are all first-order theories.
\end{tcolorbox}
\vspace{1em}

% =========================================================
% Structural Roadmap
% =========================================================
\subsection*{Structural Roadmap}

The same four-layer architecture as propositional logic applies:
\begin{center}
\textbf{Syntax $\longrightarrow$ Semantics
$\longrightarrow$ Proof Theory $\longrightarrow$ Metatheory}
\end{center}

The global progression is:
\begin{enumerate}
  \item Syntax: terms, atomic formulas, well-formed formulas,
        free and bound variables, substitution
  \item Semantics: structures, variable assignments, satisfaction,
        validity, logical equivalence, substitution lemmas
  \item Quantifiers: universal, existential, unique existential,
        bounded quantifiers, negation, commutation, prenex normal form
  \item Proof theory: quantifier rules (UI, UG, EI, EG),
        equality, soundness and completeness
  \item Translation: English to predicate logic, scope, square of opposition
  \item Reference: quantifier fallacies, comparison tables
\end{enumerate}

\vspace{1em}

\begin{remark}[Primary source]
The primary driver is Bjørndahl's \textit{Logic and Proof},
supplemented by Gerstein's \textit{Introduction to Mathematical Structures
and Proofs}.
\end{remark}

% =========================================================
% Content
% =========================================================
% =========================================================
% Predicate Logic Notes — Master Orchestrator
% =========================================================
% Subject subfolder structure:
%
%   syntax/        ← variables, constants, terms, atomic formulas,
%                     well-formed formulas, free/bound variables,
%                     substitution, alpha-equivalence, formula depth
%
%   semantics/     ← first-order languages, structures/interpretations,
%                     variable assignments, term interpretation,
%                     satisfaction, truth, validity, models, theories,
%                     logical consequence, logical equivalence,
%                     predicates vs. relations, substitution lemmas
%
%   quantifiers/   ← universal and existential quantifiers, unique
%                     existential, bounded quantifiers, quantifier
%                     negation, commutation, distribution, vacuous
%                     quantification, renaming, prenex normal form,
%                     logical strength and quantifier order
%
%   proof-theory/  ← inference rules (UI, UG, EI, EG), derived
%                     argument patterns, equality rules,
%                     soundness and completeness
%
%   translation/   ← English-to-logic, quantifier scope ambiguity,
%                     square of opposition
%
%   reference/     ← quantifier fallacies, common negation errors,
%                     inference errors, summary comparison tables
%
% Dependency order: syntax → semantics → quantifiers → proof-theory
% Translation and reference are self-contained.
% =========================================================

% =========================================================
% Syntax — Orchestrator
% =========================================================
% Dependency order: terms precede formulas; free/bound variables
% follow from the recursive formula definition; substitution and
% formula depth are downstream tools.

% =========================================================
% Terms and Formulas
% =========================================================

\subsection{Syntax of First-Order Logic: Terms and Formulas}

% ---------------------------------------------------------
% TOOLKIT
% ---------------------------------------------------------
\begin{tcolorbox}[colback=gray!6, colframe=gray!40, arc=2pt,
  left=6pt, right=6pt, top=4pt, bottom=4pt,
  title={\small\textbf{Terms and Formulas — Quick Reference}},
  fonttitle=\small\bfseries]
\small
\begin{tabular}{l l l}
\toprule
\textbf{Concept} & \textbf{Meaning} & \textbf{Detail} \\
\midrule
Variable          & Symbol ranging over domain elements    & \hyperref[def:variable]{↓ Def} \\
Term              & Syntactic object denoting a domain element & \hyperref[def:term]{↓ Def} \\
Atomic formula    & Predicate applied to terms             & \hyperref[def:atomic-formula]{↓ Def} \\
Well-formed formula & Recursively constructed formula      & \hyperref[def:wff-pred]{↓ Def} \\
Molecular formula & Non-atomic wff                         & \hyperref[def:molecular]{↓ Def} \\
\bottomrule
\end{tabular}
\end{tcolorbox}

\vspace{1em}

% ---------------------------------------------------------
% Variable
% ---------------------------------------------------------
\begin{tcolorbox}[colback=propbox, colframe=propborder, arc=2pt,
  left=6pt, right=6pt, top=4pt, bottom=4pt,
  title={\small\textbf{Definition (Variable)}},
  fonttitle=\small\bfseries]
\label{def:variable}
A \emph{variable} is a syntactic symbol that ranges over elements of a fixed
domain of discourse.

Variables serve as placeholders in formulas and do not refer to specific objects
until they are assigned values or bound by quantifiers.
\end{tcolorbox}

\begin{remark}[English reading]
Variables are the unknowns of predicate logic. They pick out no particular
object on their own; their value is supplied either by a variable assignment
(semantics) or by a quantifier (syntax).
\end{remark}

\begin{remark}[Consequence]
Because variables have no fixed denotation, the truth of a formula containing
free variables depends on what values are assigned to those variables. This
dependence is tracked formally by the variable assignment function $s$.
\end{remark}

% ---------------------------------------------------------
% Term
% ---------------------------------------------------------
\begin{tcolorbox}[colback=propbox, colframe=propborder, arc=2pt,
  left=6pt, right=6pt, top=4pt, bottom=4pt,
  title={\small\textbf{Definition (Term)}},
  fonttitle=\small\bfseries]
\label{def:term}
A \emph{term} is a syntactic expression intended to denote an object in the
domain of discourse.

The set of terms of a formal language is defined recursively:
\begin{enumerate}
  \item \textbf{Variables.} Every variable is a term.
  \item \textbf{Constants.} Every constant symbol is a term.
  \item \textbf{Function application.} If $f$ is an $n$-ary function symbol and
  $t_1, \dots, t_n$ are terms, then $f(t_1, \dots, t_n)$ is a term.
  \item \textbf{Closure.} No expression is a term unless it can be obtained by
  finitely many applications of rules (1)--(3).
\end{enumerate}
\end{tcolorbox}

\begin{remark}[English reading]
Terms are the noun phrases of predicate logic: they name (or describe) objects
in the domain. Constants name fixed objects; variables name arbitrary ones;
function symbols build complex names from simpler ones.
\end{remark}

\begin{remark}[Fully quantified form]
Terms are purely syntactic objects. Under an interpretation, each term denotes
an element of the domain of discourse, but the term itself is not an object of
the domain.
\end{remark}

% ---------------------------------------------------------
% Atomic Formula
% ---------------------------------------------------------
\begin{tcolorbox}[colback=propbox, colframe=propborder, arc=2pt,
  left=6pt, right=6pt, top=4pt, bottom=4pt,
  title={\small\textbf{Definition (Atomic Formula)}},
  fonttitle=\small\bfseries]
\label{def:atomic-formula}
An \emph{atomic formula} is a well-formed formula obtained by applying an
$n$-ary predicate symbol to $n$ terms.

If $P$ is an $n$-ary predicate symbol and $t_1, \dots, t_n$ are terms, then
\[
P(t_1, \dots, t_n)
\]
is an atomic formula.

Atomic formulas contain no logical connectives or quantifiers and serve as the
base case for the recursive definition of well-formed formulas.
\end{tcolorbox}

\begin{remark}[English reading]
Atomic formulas are the simplest complete statements: they assert that a
predicate (property or relation) holds of specific objects. They are the
predicate-logic counterpart of propositional variables.
\end{remark}

% ---------------------------------------------------------
% Well-Formed Formula
% ---------------------------------------------------------
\begin{tcolorbox}[colback=propbox, colframe=propborder, arc=2pt,
  left=6pt, right=6pt, top=4pt, bottom=4pt,
  title={\small\textbf{Definition (Well-Formed Formula)}},
  fonttitle=\small\bfseries]
\label{def:wff-pred}
The set of \emph{well-formed formulas} (wffs) of a first-order language is
defined recursively:
\begin{enumerate}
  \item \textbf{Atomic formulas.} Every atomic formula is a well-formed formula.
  \item \textbf{Negation.} If $\varphi$ is a formula, then $\neg\varphi$ is a formula.
  \item \textbf{Binary connectives.} If $\varphi$ and $\psi$ are formulas and
  $\circ \in \{\wedge, \vee, \rightarrow, \leftrightarrow\}$, then
  $(\varphi \circ \psi)$ is a formula.
  \item \textbf{Quantification.} If $\varphi$ is a formula and $x$ is a variable,
  then $\forall x\,\varphi$ and $\exists x\,\varphi$ are formulas.
  \item \textbf{Closure.} No expression is a formula unless it can be obtained
  by finitely many applications of rules (1)--(4).
\end{enumerate}
\end{tcolorbox}

\begin{remark}[English reading]
The wff definition says exactly which strings of symbols count as grammatical
sentences of first-order logic. Everything in the language is built bottom-up
from atomic formulas by the four formation rules.
\end{remark}

\begin{remark}[Distinction from propositional logic]
Rule (4) is new: quantifiers bind variables and create formulas from formulas.
This extra layer is what makes predicate logic more expressive than
propositional logic.
\end{remark}

% ---------------------------------------------------------
% Molecular Formula
% ---------------------------------------------------------
\begin{tcolorbox}[colback=propbox, colframe=propborder, arc=2pt,
  left=6pt, right=6pt, top=4pt, bottom=4pt,
  title={\small\textbf{Definition (Molecular Formula)}},
  fonttitle=\small\bfseries]
\label{def:molecular}
A \emph{molecular formula} is a well-formed formula that is not atomic.

Equivalently, a formula is molecular if it is formed from one or more atomic
formulas by the application of logical connectives or quantifiers.
\end{tcolorbox}

\begin{remark}[English reading]
Atomic formulas express basic properties or relations. Molecular formulas
express compound statements built from atomic formulas using the logical
apparatus of the language.
\end{remark}

% =========================================================
% Free and Bound Variables; Substitution
% =========================================================

\subsection{Syntax: Free Variables, Bound Variables, and Substitution}

% ---------------------------------------------------------
% TOOLKIT
% ---------------------------------------------------------
\begin{tcolorbox}[colback=gray!6, colframe=gray!40, arc=2pt,
  left=6pt, right=6pt, top=4pt, bottom=4pt,
  title={\small\textbf{Variables and Substitution — Quick Reference}},
  fonttitle=\small\bfseries]
\small
\begin{tabular}{l l l}
\toprule
\textbf{Concept} & \textbf{Meaning} & \textbf{Detail} \\
\midrule
Scope of a quantifier  & Formula governed by the quantifier  & \hyperref[def:scope]{↓ Def} \\
Bound/free occurrence  & Whether $x$ is captured by a quantifier & \hyperref[def:bound-free]{↓ Def} \\
Free variable $\mathrm{FV}(\varphi)$ & Variables whose values affect truth & \hyperref[def:fv]{↓ Def} \\
Sentence               & Formula with $\mathrm{FV}(\varphi)=\varnothing$ & \hyperref[def:sentence]{↓ Def} \\
Substitution $\varphi[t/x]$ & Replace free $x$ by term $t$      & \hyperref[def:subst-notation]{↓ Def} \\
Free for substitution  & No variable in $t$ becomes bound    & \hyperref[def:free-for]{↓ Def} \\
Alpha-equivalence      & Renaming bound variables             & \hyperref[def:alpha-equiv]{↓ Def} \\
\bottomrule
\end{tabular}
\end{tcolorbox}

\vspace{1em}

% ---------------------------------------------------------
% Scope
% ---------------------------------------------------------
\begin{tcolorbox}[colback=propbox, colframe=propborder, arc=2pt,
  left=6pt, right=6pt, top=4pt, bottom=4pt,
  title={\small\textbf{Definition (Scope of a Quantifier)}},
  fonttitle=\small\bfseries]
\label{def:scope}
Let $\varphi$ be a formula of a first-order language.

If $\varphi$ is of the form $\forall x\,\psi$ or $\exists x\,\psi$, then the
formula $\psi$ is called the \emph{scope} of the quantifier.

The quantifier is said to \emph{bind} all occurrences of the variable $x$ that
appear within its scope.
\end{tcolorbox}

\begin{remark}[English reading]
The scope is the reach of a quantifier — the subformula it governs. Scope is
syntactically determined by the parenthesisation of the formula, not by
proximity to the quantifier symbol.
\end{remark}

\begin{example}
In $(\forall x\,P(x)) \wedge Q(x)$, the scope of $\forall x$ is $P(x)$ only.
The occurrence of $x$ in $Q(x)$ falls outside the scope and is free.
\end{example}

% ---------------------------------------------------------
% Bound and Free Occurrences
% ---------------------------------------------------------
\begin{tcolorbox}[colback=propbox, colframe=propborder, arc=2pt,
  left=6pt, right=6pt, top=4pt, bottom=4pt,
  title={\small\textbf{Definition (Bound and Free Occurrences)}},
  fonttitle=\small\bfseries]
\label{def:bound-free}
An occurrence of a variable $x$ in a formula $\varphi$ is \emph{bound} if it
lies within the scope of a quantifier $\forall x$ or $\exists x$.

An occurrence of $x$ is \emph{free} if it is not bound by any quantifier in
$\varphi$.
\end{tcolorbox}

\begin{remark}[English reading]
The same variable can have both bound and free occurrences in a single formula.
Each occurrence is classified independently by checking whether it falls within
the scope of a binding quantifier.
\end{remark}

% ---------------------------------------------------------
% Free Variables FV(φ)
% ---------------------------------------------------------
\begin{tcolorbox}[colback=propbox, colframe=propborder, arc=2pt,
  left=6pt, right=6pt, top=4pt, bottom=4pt,
  title={\small\textbf{Definition (Free Variables of a Formula)}},
  fonttitle=\small\bfseries]
\label{def:fv}
The set $\mathrm{FV}(\varphi)$ of \emph{free variables} of a formula $\varphi$
is defined recursively:
\begin{enumerate}
  \item If $\varphi = P(t_1,\dots,t_n)$ is atomic, then
  $\mathrm{FV}(\varphi) = \bigcup_{i=1}^n \mathrm{Var}(t_i)$,
  where $\mathrm{Var}(t_i)$ is the set of variables in term $t_i$.

  \item $\mathrm{FV}(\neg \varphi) = \mathrm{FV}(\varphi)$.

  \item $\mathrm{FV}(\varphi \circ \psi) = \mathrm{FV}(\varphi) \cup \mathrm{FV}(\psi)$
  for any binary connective $\circ$.

  \item $\mathrm{FV}(\forall x\,\varphi) = \mathrm{FV}(\varphi) \setminus \{x\}$.

  \item $\mathrm{FV}(\exists x\,\varphi) = \mathrm{FV}(\varphi) \setminus \{x\}$.
\end{enumerate}
\end{tcolorbox}

\begin{remark}[English reading]
$\mathrm{FV}(\varphi)$ collects exactly those variables whose values can
influence the truth of $\varphi$ under a structure. Quantifying over $x$
removes $x$ from the free set because the quantifier takes responsibility for
ranging over all (or some) values of $x$.
\end{remark}

\begin{remark}[Common error]
Terms themselves contain only free variables. Variables become bound only
through quantification in formulas. The interpretation of a formula depends
exactly on the values assigned to its free variables.
\end{remark}

% ---------------------------------------------------------
% Sentence
% ---------------------------------------------------------
\begin{tcolorbox}[colback=propbox, colframe=propborder, arc=2pt,
  left=6pt, right=6pt, top=4pt, bottom=4pt,
  title={\small\textbf{Definition (Sentence)}},
  fonttitle=\small\bfseries]
\label{def:sentence}
A \emph{sentence} (or \emph{closed formula}) is a formula with no free variables.

Formally, $\varphi$ is a sentence if and only if $\mathrm{FV}(\varphi) = \varnothing$.
\end{tcolorbox}

\begin{remark}[Consequence]
For sentences, truth depends only on the structure, not on the variable
assignment. This makes sentences the natural objects to be called true or false
in a model, without qualification.
\end{remark}

% ---------------------------------------------------------
% Substitution Notation
% ---------------------------------------------------------
\begin{tcolorbox}[colback=propbox, colframe=propborder, arc=2pt,
  left=6pt, right=6pt, top=4pt, bottom=4pt,
  title={\small\textbf{Definition (Substitution Notation)}},
  fonttitle=\small\bfseries]
\label{def:subst-notation}
Let $\varphi$ be a formula, $x$ a variable, and $t$ a term.

The expression $\varphi[t/x]$ denotes the formula obtained from $\varphi$ by
replacing every \emph{free} occurrence of $x$ with the term $t$, leaving bound
occurrences of $x$ unchanged.
\end{tcolorbox}

% ---------------------------------------------------------
% Free for Substitution
% ---------------------------------------------------------
\begin{tcolorbox}[colback=propbox, colframe=propborder, arc=2pt,
  left=6pt, right=6pt, top=4pt, bottom=4pt,
  title={\small\textbf{Definition (Free for Substitution)}},
  fonttitle=\small\bfseries]
\label{def:free-for}
A term $t$ is \emph{free for} $x$ in $\varphi$ if no free occurrence of $x$ in
$\varphi$ lies within the scope of a quantifier $\forall y$ or $\exists y$
where $y$ is a variable occurring in $t$.

Equivalently, $t$ is free for $x$ in $\varphi$ if the substitution $\varphi[t/x]$
does not result in any variable in $t$ becoming bound.
\end{tcolorbox}

\begin{remark}[Capture-avoiding substitution]
A substitution $\varphi[t/x]$ is admissible only if $t$ is free for $x$ in
$\varphi$. If this condition is violated, a variable occurring in $t$ may
become bound after substitution, silently changing the meaning of the formula.
This is called \emph{variable capture}.
\end{remark}

\begin{remark}[Common error]
Variable capture is one of the most common syntactic mistakes in predicate
logic. Always check that the term being substituted introduces no variables that
fall inside a binding quantifier in the target formula.
\end{remark}

% ---------------------------------------------------------
% Alpha-Equivalence
% ---------------------------------------------------------
\begin{tcolorbox}[colback=propbox, colframe=propborder, arc=2pt,
  left=6pt, right=6pt, top=4pt, bottom=4pt,
  title={\small\textbf{Definition (Alpha-Equivalence)}},
  fonttitle=\small\bfseries]
\label{def:alpha-equiv}
Formulas that differ only by the names of bound variables are logically
equivalent. This is called \emph{alpha-equivalence} (or \emph{alphabetic
variance}):
\[
\forall x\,\varphi \;\equiv\; \forall y\,\varphi[y/x]
\quad\text{(provided $y$ is not free in $\varphi$).}
\]
\end{tcolorbox}

\begin{remark}[Consequence]
Alpha-equivalence is the formal basis for the bound variable renaming used
in prenex normal form conversion and in any proof where variable capture must
be avoided. Two alpha-equivalent formulas are interchangeable in all contexts.
\end{remark}

% =========================================================
% Formula Depth (Complexity)
% =========================================================

\subsection{Syntax: Formula Depth}

% ---------------------------------------------------------
% TOOLKIT
% ---------------------------------------------------------
\begin{tcolorbox}[colback=gray!6, colframe=gray!40, arc=2pt,
  left=6pt, right=6pt, top=4pt, bottom=4pt,
  title={\small\textbf{Formula Depth — Quick Reference}},
  fonttitle=\small\bfseries]
\small
\begin{tabular}{l l l}
\toprule
\textbf{Concept} & \textbf{Meaning} & \textbf{Detail} \\
\midrule
Formula depth & Max formation steps from atomic formulas & \hyperref[def:formula-depth]{↓ Def} \\
\bottomrule
\end{tabular}
\end{tcolorbox}

\vspace{1em}

% ---------------------------------------------------------
% Formula Depth
% ---------------------------------------------------------
\begin{tcolorbox}[colback=propbox, colframe=propborder, arc=2pt,
  left=6pt, right=6pt, top=4pt, bottom=4pt,
  title={\small\textbf{Definition (Formula Depth)}},
  fonttitle=\small\bfseries]
\label{def:formula-depth}
The \emph{depth} (or \emph{complexity}) of a formula $\varphi$, denoted
$\mathrm{depth}(\varphi)$, is a natural number defined recursively:
\begin{enumerate}
  \item \textbf{Atomic case.} If $\varphi$ is atomic, then $\mathrm{depth}(\varphi) = 0$.

  \item \textbf{Negation.} If $\varphi = \neg \psi$, then
  $\mathrm{depth}(\varphi) = \mathrm{depth}(\psi) + 1$.

  \item \textbf{Binary connectives.} If $\varphi = (\psi \circ \chi)$
  for $\circ \in \{\wedge,\vee,\rightarrow,\leftrightarrow\}$, then
  \[
  \mathrm{depth}(\varphi) = \max\{\mathrm{depth}(\psi),\,\mathrm{depth}(\chi)\} + 1.
  \]

  \item \textbf{Quantifiers.} If $\varphi = \forall x\,\psi$ or $\varphi = \exists x\,\psi$,
  then $\mathrm{depth}(\varphi) = \mathrm{depth}(\psi) + 1$.
\end{enumerate}
\end{tcolorbox}

\begin{remark}[English reading]
The depth of a formula measures the maximum number of logical formation steps
required to build the formula from atomic formulas. It corresponds to the height
of the formula's syntactic parse tree.
\end{remark}

\begin{remark}[Proof strategy]
Formula depth is the standard measure used in structural induction proofs over
first-order formulas. The base case is depth~$0$ (atomic formulas); the
inductive step handles each formation rule and assumes the result holds for
sub-formulas of strictly smaller depth.
\end{remark}


% =========================================================
% Semantics — Orchestrator
% =========================================================
% Dependency order: languages and structures first; variable
% assignments enable term interpretation; satisfaction builds on
% both; models and theories require satisfaction; predicates and
% substitution lemmas are further semantic tools.

% =========================================================
% Structures, Interpretations, and Variable Assignments
% =========================================================

\subsection{Semantics: Structures and Variable Assignments}

% ---------------------------------------------------------
% TOOLKIT
% ---------------------------------------------------------
\begin{tcolorbox}[colback=gray!6, colframe=gray!40, arc=2pt,
  left=6pt, right=6pt, top=4pt, bottom=4pt,
  title={\small\textbf{Structures — Quick Reference}},
  fonttitle=\small\bfseries]
\small
\begin{tabular}{l l l}
\toprule
\textbf{Concept} & \textbf{Meaning} & \textbf{Detail} \\
\midrule
First-order language $\mathcal{L}$ & Signature specifying symbols & \hyperref[def:fo-language]{↓ Def} \\
Structure $\mathcal{M} = \langle D, I \rangle$ & Interpretation of $\mathcal{L}$ & \hyperref[def:structure]{↓ Def} \\
Variable assignment $s$ & Function assigning domain elements to variables & \hyperref[def:var-assign]{↓ Def} \\
Modified assignment $s[x \mapsto d]$ & Change $s$ at $x$ only & \hyperref[def:mod-assign]{↓ Def} \\
Term interpretation $\llbracket t \rrbracket_{\mathcal{M},s}$ & Semantic value of a term & \hyperref[def:term-interp]{↓ Def} \\
\bottomrule
\end{tabular}
\end{tcolorbox}

\vspace{1em}

% ---------------------------------------------------------
% First-Order Language
% ---------------------------------------------------------
\begin{tcolorbox}[colback=propbox, colframe=propborder, arc=2pt,
  left=6pt, right=6pt, top=4pt, bottom=4pt,
  title={\small\textbf{Definition (First-Order Language)}},
  fonttitle=\small\bfseries]
\label{def:fo-language}
A \emph{first-order language} $\mathcal{L}$ consists of:
\begin{itemize}
  \item a set of constant symbols,
  \item a set of function symbols, each with a specified arity, and
  \item a set of predicate symbols, each with a specified arity.
\end{itemize}
Every first-order language also includes a countable set of variables and the
logical symbols $\neg, \wedge, \vee, \rightarrow, \leftrightarrow, \forall,
\exists$, and (optionally) $=$.
\end{tcolorbox}

\begin{remark}[English reading]
A first-order language is the alphabet of a formal system. It specifies which
non-logical names are available (constants, functions, predicates) but says
nothing yet about what those names mean. Meaning is supplied by a structure.
\end{remark}

% ---------------------------------------------------------
% Structure
% ---------------------------------------------------------
\begin{tcolorbox}[colback=propbox, colframe=propborder, arc=2pt,
  left=6pt, right=6pt, top=4pt, bottom=4pt,
  title={\small\textbf{Definition (Structure)}},
  fonttitle=\small\bfseries]
\label{def:structure}
Let $\mathcal{L}$ be a first-order language. A \emph{structure} (or
\emph{interpretation}) for $\mathcal{L}$ is a pair
\[
\mathcal{M} = \langle D, I \rangle,
\]
where $D$ is a nonempty set called the \emph{domain of discourse} (or
\emph{universe}), and $I$ is an interpretation function assigning:
\begin{itemize}
  \item to each constant symbol $c$: an element $I(c) \in D$,
  \item to each $n$-ary function symbol $f$: a function $I(f) : D^n \to D$,
  \item to each $n$-ary predicate symbol $P$: a relation $I(P) \subseteq D^n$.
\end{itemize}
\end{tcolorbox}

\begin{remark}[English reading]
A structure gives the language its meaning. The domain specifies the universe
of objects being reasoned about, and the interpretation function assigns each
non-logical symbol a concrete mathematical object.
\end{remark}

\begin{remark}[Nonempty domain convention]
Throughout classical first-order logic, the domain $D$ is required to be
nonempty. Without this, the inference $\forall x\,\varphi \Rightarrow
\exists x\,\varphi$ (subalternation) would fail: a universal statement is
vacuously true in an empty domain while the existential statement is false.
\end{remark}

\begin{remark}[Source note]
Some texts use the term \emph{interpretation} and others use \emph{structure};
the two are synonymous. The notation $\mathcal{M}$, $\mathfrak{A}$, or
$\mathbf{A}$ all appear in the literature.
\end{remark}

% ---------------------------------------------------------
% Variable Assignment
% ---------------------------------------------------------
\begin{tcolorbox}[colback=propbox, colframe=propborder, arc=2pt,
  left=6pt, right=6pt, top=4pt, bottom=4pt,
  title={\small\textbf{Definition (Variable Assignment)}},
  fonttitle=\small\bfseries]
\label{def:var-assign}
Let $\mathcal{M} = \langle D, I \rangle$ be a structure for $\mathcal{L}$.
A \emph{variable assignment} is a function
\[
s : \mathsf{Var} \to D
\]
that assigns to each variable an element of the domain.
\end{tcolorbox}

\begin{remark}[English reading]
A structure assigns meaning to non-logical symbols; a variable assignment
assigns meaning to variables. Together they determine the truth value of every
formula in the language.
\end{remark}

% ---------------------------------------------------------
% Modified Assignment
% ---------------------------------------------------------
\begin{tcolorbox}[colback=propbox, colframe=propborder, arc=2pt,
  left=6pt, right=6pt, top=4pt, bottom=4pt,
  title={\small\textbf{Definition (Modified Assignment)}},
  fonttitle=\small\bfseries]
\label{def:mod-assign}
Let $s$ be a variable assignment, $x$ a variable, and $d \in D$. The
\emph{modified assignment} $s[x \mapsto d]$ is defined by
\[
s[x \mapsto d](y) =
\begin{cases}
d & \text{if } y = x, \\
s(y) & \text{if } y \neq x.
\end{cases}
\]
\end{tcolorbox}

\begin{remark}[Consequence]
Modified assignments are the formal device for evaluating quantified formulas.
The clause $\mathcal{M}, s \models \forall x\,\varphi$ checks $\mathcal{M},
s[x \mapsto d] \models \varphi$ for every $d \in D$, testing the formula
as $x$ ranges over all domain elements.
\end{remark}

% ---------------------------------------------------------
% Term Interpretation
% ---------------------------------------------------------
\begin{tcolorbox}[colback=propbox, colframe=propborder, arc=2pt,
  left=6pt, right=6pt, top=4pt, bottom=4pt,
  title={\small\textbf{Definition (Interpretation of a Term)}},
  fonttitle=\small\bfseries]
\label{def:term-interp}
The \emph{interpretation} of a term $t$ in $\mathcal{M}$ under $s$, denoted
$\llbracket t \rrbracket_{\mathcal{M},s}$, is defined recursively:
\begin{enumerate}
  \item If $t$ is a variable $x$: $\llbracket x \rrbracket_{\mathcal{M},s} = s(x)$.
  \item If $t$ is a constant $c$: $\llbracket c \rrbracket_{\mathcal{M},s} = I(c)$.
  \item If $t = f(t_1,\dots,t_n)$:
  $\llbracket f(t_1,\dots,t_n) \rrbracket_{\mathcal{M},s}
  = I(f)\bigl(\llbracket t_1 \rrbracket_{\mathcal{M},s},\dots,
  \llbracket t_n \rrbracket_{\mathcal{M},s}\bigr)$.
\end{enumerate}
\end{tcolorbox}

\begin{remark}[English reading]
This is the semantic counterpart of the recursive term definition. Each
formation rule for terms has a corresponding clause that computes a domain
element from the interpretations of its components.
\end{remark}

% =========================================================
% Satisfaction, Truth, Validity, Satisfiability
% =========================================================

\subsection{Semantics: Satisfaction and Truth}

% ---------------------------------------------------------
% TOOLKIT
% ---------------------------------------------------------
\begin{tcolorbox}[colback=gray!6, colframe=gray!40, arc=2pt,
  left=6pt, right=6pt, top=4pt, bottom=4pt,
  title={\small\textbf{Satisfaction and Truth — Quick Reference}},
  fonttitle=\small\bfseries]
\small
\begin{tabular}{l l l}
\toprule
\textbf{Concept} & \textbf{Meaning} & \textbf{Detail} \\
\midrule
Satisfaction $\mathcal{M}, s \models \varphi$ & Formula true in $\mathcal{M}$ under $s$ & \hyperref[def:satisfaction]{↓ Def} \\
Truth in a structure $\mathcal{M} \models \varphi$ & True for every assignment  & \hyperref[def:truth-structure]{↓ Def} \\
Validity / logical truth & True in every structure    & \hyperref[def:validity]{↓ Def} \\
Satisfiability & True in some structure & \hyperref[def:validity]{↓ Def} \\
\bottomrule
\end{tabular}
\end{tcolorbox}

\vspace{1em}

% ---------------------------------------------------------
% Satisfaction
% ---------------------------------------------------------
\begin{tcolorbox}[colback=propbox, colframe=propborder, arc=2pt,
  left=6pt, right=6pt, top=4pt, bottom=4pt,
  title={\small\textbf{Definition (Satisfaction)}},
  fonttitle=\small\bfseries]
\label{def:fol-satisfaction}
Let $\mathcal{M} = \langle D, I \rangle$ be a structure, $s$ a variable
assignment, and $\varphi$ a formula. The relation $\mathcal{M}, s \models \varphi$
(read ``$\mathcal{M}$ satisfies $\varphi$ under $s$'') is defined recursively:
\begin{enumerate}
  \item \textbf{Atomic.}
  $\mathcal{M}, s \models P(t_1, \dots, t_n) \iff
  (\llbracket t_1 \rrbracket_{\mathcal{M},s}, \dots,
   \llbracket t_n \rrbracket_{\mathcal{M},s}) \in I(P)$.

  \item \textbf{Equality.}
  $\mathcal{M}, s \models (t_1 = t_2) \iff
  \llbracket t_1 \rrbracket_{\mathcal{M},s} = \llbracket t_2 \rrbracket_{\mathcal{M},s}$.

  \item \textbf{Negation.}
  $\mathcal{M}, s \models \neg\varphi \iff \mathcal{M}, s \not\models \varphi$.

  \item \textbf{Conjunction.}
  $\mathcal{M}, s \models (\varphi \wedge \psi) \iff
  \mathcal{M}, s \models \varphi \text{ and } \mathcal{M}, s \models \psi$.

  \item \textbf{Disjunction.}
  $\mathcal{M}, s \models (\varphi \vee \psi) \iff
  \mathcal{M}, s \models \varphi \text{ or } \mathcal{M}, s \models \psi$.

  \item \textbf{Implication.}
  $\mathcal{M}, s \models (\varphi \rightarrow \psi) \iff
  \mathcal{M}, s \not\models \varphi \text{ or } \mathcal{M}, s \models \psi$.

  \item \textbf{Biconditional.}
  $\mathcal{M}, s \models (\varphi \leftrightarrow \psi) \iff
  (\mathcal{M}, s \models \varphi \Leftrightarrow \mathcal{M}, s \models \psi)$.

  \item \textbf{Universal quantification.}
  $\mathcal{M}, s \models \forall x\,\varphi \iff
  \text{for all } d \in D,\; \mathcal{M}, s[x \mapsto d] \models \varphi$.

  \item \textbf{Existential quantification.}
  $\mathcal{M}, s \models \exists x\,\varphi \iff
  \text{there exists } d \in D \text{ such that }
  \mathcal{M}, s[x \mapsto d] \models \varphi$.
\end{enumerate}
\end{tcolorbox}

\begin{remark}[English reading]
Satisfaction is the bridge between syntax and semantics. It tells us, for each
formula formation rule, what it means for the formula to hold. Clauses~(1)--(7)
mirror the propositional connectives; clauses~(8)--(9) are the new quantifier
clauses unique to first-order logic.
\end{remark}

\begin{remark}[Fully quantified form for universal quantification]
The clause $\mathcal{M}, s \models \forall x\,\varphi$ holds if and only if
for every element $d$ of the domain $D$, the formula $\varphi$ is satisfied
under the assignment that agrees with $s$ everywhere except that it sends $x$
to $d$. This is what it means for a property to hold universally.
\end{remark}

% ---------------------------------------------------------
% Truth in a Structure
% ---------------------------------------------------------
\begin{tcolorbox}[colback=propbox, colframe=propborder, arc=2pt,
  left=6pt, right=6pt, top=4pt, bottom=4pt,
  title={\small\textbf{Definition (Truth in a Structure)}},
  fonttitle=\small\bfseries]
\label{def:truth-structure}
A sentence $\varphi$ is \emph{true} in a structure $\mathcal{M}$, written
$\mathcal{M} \models \varphi$, if $\mathcal{M}, s \models \varphi$ for every
variable assignment $s$.

A sentence $\varphi$ is \emph{false} in $\mathcal{M}$ if $\mathcal{M}
\not\models \varphi$.
\end{tcolorbox}

\begin{remark}[English reading]
For sentences (formulas with no free variables), the choice of assignment is
irrelevant — all assignments agree. So $\mathcal{M} \models \varphi$ is
well-defined without reference to any particular $s$.
\end{remark}

% ---------------------------------------------------------
% Validity and Satisfiability
% ---------------------------------------------------------
\begin{tcolorbox}[colback=propbox, colframe=propborder, arc=2pt,
  left=6pt, right=6pt, top=4pt, bottom=4pt,
  title={\small\textbf{Definition (Validity and Satisfiability)}},
  fonttitle=\small\bfseries]
\label{def:validity}
A formula $\varphi$ is:
\begin{itemize}
  \item \emph{valid} (or a \emph{logical truth}) if $\mathcal{M}, s \models \varphi$
  for every structure $\mathcal{M}$ and every assignment $s$;
  \item \emph{satisfiable} if $\mathcal{M}, s \models \varphi$ for some $\mathcal{M}$
  and some $s$;
  \item \emph{unsatisfiable} (or a \emph{contradiction}) if it is not satisfiable.
\end{itemize}
\end{tcolorbox}

\begin{remark}[Consequence]
A formula is valid if no structure can make it false. These are the logical
truths of predicate logic — the formulas provable by logic alone, without any
special assumptions about the domain.
\end{remark}

% =========================================================
% Models, Theories, Logical Consequence, Logical Equivalence
% =========================================================

\subsection{Semantics: Models, Theories, and Logical Consequence}

% ---------------------------------------------------------
% TOOLKIT
% ---------------------------------------------------------
\begin{tcolorbox}[colback=gray!6, colframe=gray!40, arc=2pt,
  left=6pt, right=6pt, top=4pt, bottom=4pt,
  title={\small\textbf{Models and Theories — Quick Reference}},
  fonttitle=\small\bfseries]
\small
\begin{tabular}{l l l}
\toprule
\textbf{Concept} & \textbf{Meaning} & \textbf{Detail} \\
\midrule
Model $\mathcal{M} \models \Gamma$ & Structure satisfying all sentences of $\Gamma$ & \hyperref[def:model]{↓ Def} \\
Theory & Set of sentences; consistent if it has a model & \hyperref[def:theory]{↓ Def} \\
Logical consequence $\Gamma \models \varphi$ & True in every model of $\Gamma$ & \hyperref[def:log-consequence]{↓ Def} \\
Logical equivalence $\varphi \equiv \psi$ & Mutually entailing formulas & \hyperref[def:log-equiv]{↓ Def} \\
\bottomrule
\end{tabular}
\end{tcolorbox}

\vspace{1em}

% ---------------------------------------------------------
% Model
% ---------------------------------------------------------
\begin{tcolorbox}[colback=propbox, colframe=propborder, arc=2pt,
  left=6pt, right=6pt, top=4pt, bottom=4pt,
  title={\small\textbf{Definition (Model)}},
  fonttitle=\small\bfseries]
\label{def:model}
Let $\varphi$ be a sentence and $\mathcal{M}$ a structure. We say that
$\mathcal{M}$ is a \emph{model} of $\varphi$ if $\mathcal{M} \models \varphi$.

More generally, if $\Gamma$ is a set of sentences, then $\mathcal{M}$ is a
model of $\Gamma$ if $\mathcal{M} \models \gamma$ for every $\gamma \in \Gamma$.
In this case, we write $\mathcal{M} \models \Gamma$.
\end{tcolorbox}

% ---------------------------------------------------------
% Theory
% ---------------------------------------------------------
\begin{tcolorbox}[colback=propbox, colframe=propborder, arc=2pt,
  left=6pt, right=6pt, top=4pt, bottom=4pt,
  title={\small\textbf{Definition (Theory)}},
  fonttitle=\small\bfseries]
\label{def:theory}
A \emph{theory} is a set of sentences.

A theory $T$ is \emph{satisfiable} (or \emph{consistent}) if it has at least
one model.

A theory $T$ is \emph{complete} if for every sentence $\varphi$ in the
language, either $T \models \varphi$ or $T \models \neg\varphi$.
\end{tcolorbox}

\begin{remark}[Source note]
In some texts, a theory is required to be closed under logical consequence: if
$T \models \varphi$ then $\varphi \in T$. In other texts, a theory is simply
any set of axioms from which consequences are derived. Both usages appear in
the literature.
\end{remark}

% ---------------------------------------------------------
% Logical Consequence
% ---------------------------------------------------------
\begin{tcolorbox}[colback=propbox, colframe=propborder, arc=2pt,
  left=6pt, right=6pt, top=4pt, bottom=4pt,
  title={\small\textbf{Definition (Logical Consequence)}},
  fonttitle=\small\bfseries]
\label{def:log-consequence}
Let $\Gamma$ be a set of formulas and $\varphi$ a formula. We say that
$\varphi$ is a \emph{logical consequence} of $\Gamma$, written
$\Gamma \models \varphi$, if for every structure $\mathcal{M}$ and every
variable assignment $s$:
\[
\text{if } \mathcal{M}, s \models \gamma \text{ for all } \gamma \in \Gamma,
\text{ then } \mathcal{M}, s \models \varphi.
\]
Equivalently, every model of $\Gamma$ is a model of $\varphi$.
\end{tcolorbox}

\begin{remark}[English reading]
Logical consequence says: if all the hypotheses in $\Gamma$ are true in some
structure, then $\varphi$ must be true there too. This is the semantic notion
of entailment.
\end{remark}

\begin{remark}[Notation]
The notation $\models \varphi$ (empty left-hand side) means $\varphi$ is
valid: it is a logical consequence of the empty set of premises.
\end{remark}

% ---------------------------------------------------------
% Logical Equivalence
% ---------------------------------------------------------
\begin{tcolorbox}[colback=propbox, colframe=propborder, arc=2pt,
  left=6pt, right=6pt, top=4pt, bottom=4pt,
  title={\small\textbf{Definition (Logical Equivalence)}},
  fonttitle=\small\bfseries]
\label{def:fol-log-equiv}
Two formulas $\varphi$ and $\psi$ are \emph{logically equivalent}, written
$\varphi \equiv \psi$, if each is a logical consequence of the other:
\[
\varphi \equiv \psi
\quad\Longleftrightarrow\quad
(\varphi \models \psi \text{ and } \psi \models \varphi).
\]
Equivalently, $\varphi$ and $\psi$ have the same truth value in every structure
under every variable assignment.
\end{tcolorbox}

\begin{remark}[Consequence]
Logical equivalence is the semantic counterpart of provable bi-implication.
Two logically equivalent formulas are interchangeable in any context without
changing truth values.
\end{remark}

% =========================================================
% Predicates, Relations, and Substitution Lemmas
% =========================================================

\subsection{Semantics: Predicates, Relations, and Substitution Lemmas}

% ---------------------------------------------------------
% TOOLKIT
% ---------------------------------------------------------
\begin{tcolorbox}[colback=gray!6, colframe=gray!40, arc=2pt,
  left=6pt, right=6pt, top=4pt, bottom=4pt,
  title={\small\textbf{Predicates and Substitution Lemmas — Quick Reference}},
  fonttitle=\small\bfseries]
\small
\begin{tabular}{l l l}
\toprule
\textbf{Concept} & \textbf{Meaning} & \textbf{Detail} \\
\midrule
Predicate & Open formula or truth-valued function on domain & \hyperref[def:predicate]{↓ Def} \\
Substitution Lemma (terms) & Assign vs.\ substitute, same result & \hyperref[lem:subst-terms]{↓ Lem} \\
Substitution Lemma (formulas) & Syntactic substitution $=$ semantic update & \hyperref[lem:subst-formulas]{↓ Lem} \\
\bottomrule
\end{tabular}
\end{tcolorbox}

\vspace{1em}

% ---------------------------------------------------------
% Predicate
% ---------------------------------------------------------
\begin{tcolorbox}[colback=propbox, colframe=propborder, arc=2pt,
  left=6pt, right=6pt, top=4pt, bottom=4pt,
  title={\small\textbf{Definition (Predicate)}},
  fonttitle=\small\bfseries]
\label{def:predicate}
A \emph{predicate} is an expression containing one or more variables that
represents a property or relation and becomes a proposition when all its
variables are instantiated.

Formally, a predicate has two equivalent views:
\begin{enumerate}
  \item \textbf{Syntactic view.} A predicate is an open formula
  $\varphi(x_1,\dots,x_n)$ that has no truth value until specific objects are
  substituted for its variables.
  \item \textbf{Semantic view.} Given a domain $D$, an $n$-ary predicate
  determines a function $P : D^n \to \{\mathsf{T}, \mathsf{F}\}$.
\end{enumerate}
\end{tcolorbox}

\begin{remark}[Predicates vs.\ relations]
A \emph{predicate} is a syntactic or semantic device: syntactically, an open
formula; semantically, a truth-valued function. A \emph{relation} is a purely
set-theoretic object: an $n$-ary relation on $D$ is a subset $R \subseteq D^n$.
Under an interpretation, predicates and relations correspond via
$P(a_1,\dots,a_n) \text{ is true} \iff (a_1,\dots,a_n) \in R$. Thus predicates
belong to the language of logic, while relations belong to the structures
interpreting that language.
\end{remark}

% ---------------------------------------------------------
% Substitution Lemmas
% ---------------------------------------------------------

\begin{lemma}[Substitution Lemma for Terms]\label{lem:subst-terms}
Let $\mathcal{M}$ be a structure, $s$ a variable assignment, $x$ a variable,
and $d \in D$. For any term $t$,
\[
\llbracket t \rrbracket_{\mathcal{M},\,s[x\mapsto d]}
=
\llbracket t[d/x] \rrbracket_{\mathcal{M},\,s}.
\]
\end{lemma}

\begin{remark}[English reading]
Evaluating a term after modifying an assignment is the same as substituting
the value directly into the term and then evaluating. This shows that syntactic
substitution and semantic update commute for terms.
\end{remark}

\begin{lemma}[Substitution Lemma for Formulas]\label{lem:subst-formulas}
Let $\varphi$ be a formula, $t$ a term free for $x$ in $\varphi$. Then for
any structure $\mathcal{M}$ and assignment $s$,
\[
\mathcal{M},s \models \varphi[t/x]
\quad\Longleftrightarrow\quad
\mathcal{M},\,s\bigl[x\mapsto \llbracket t \rrbracket_{\mathcal{M},s}\bigr] \models \varphi.
\]
\end{lemma}

\begin{remark}[English reading]
Satisfying $\varphi$ with $x$ replaced by $t$ is the same as satisfying
$\varphi$ under the assignment that sends $x$ to the value of $t$. This lemma
formally connects syntactic substitution with semantic evaluation and is
essential for proving the soundness of Universal Instantiation.
\end{remark}

\begin{remark}[Proof strategy]
Both lemmas are proved by structural induction on the term (Lemma for terms)
or formula (Lemma for formulas). The key cases are variables and quantifiers;
the remaining cases follow directly from the inductive hypothesis.
\end{remark}


% =========================================================
% Quantifiers — Orchestrator
% =========================================================
% Dependency order: basic quantifiers first; bounded and unique
% existential extend them; negation and commutation are laws;
% prenex normal form uses all the preceding material;
% logical strength is a meta-level comparison tool.

% =========================================================
% Universal, Existential, Unique Existential, and Bounded Quantifiers
% =========================================================

\subsection{Quantifiers: Universal, Existential, and Bounded}

% ---------------------------------------------------------
% TOOLKIT
% ---------------------------------------------------------
\begin{tcolorbox}[colback=gray!6, colframe=gray!40, arc=2pt,
  left=6pt, right=6pt, top=4pt, bottom=4pt,
  title={\small\textbf{Basic Quantifiers — Quick Reference}},
  fonttitle=\small\bfseries]
\small
\begin{tabular}{l l l l}
\toprule
\textbf{Symbol} & \textbf{Reading} & \textbf{Meaning} & \textbf{Detail} \\
\midrule
$\forall x\,\varphi$ & For all $x$, $\varphi$ & True for every domain element & \hyperref[def:universal-q]{↓ Def} \\
$\exists x\,\varphi$ & There exists $x$ such that $\varphi$ & True for some domain element & \hyperref[def:existential-q]{↓ Def} \\
$\exists! x\,\varphi$ & There exists exactly one $x$ such that $\varphi$ & Existence + uniqueness & \hyperref[def:unique-exist]{↓ Def} \\
$\forall x \in A\,\varphi$ & For all $x$ in $A$, $\varphi$ & Restricted universal & \hyperref[def:bounded-q]{↓ Def} \\
$\exists x \in A\,\varphi$ & There exists $x$ in $A$ such that $\varphi$ & Restricted existential & \hyperref[def:bounded-q]{↓ Def} \\
\bottomrule
\end{tabular}
\end{tcolorbox}

\vspace{1em}

% ---------------------------------------------------------
% Universal Quantifier
% ---------------------------------------------------------
\begin{tcolorbox}[colback=propbox, colframe=propborder, arc=2pt,
  left=6pt, right=6pt, top=4pt, bottom=4pt,
  title={\small\textbf{Definition (Universal Quantifier)}},
  fonttitle=\small\bfseries]
\label{def:universal-q}
The \emph{universal quantifier}, denoted $\forall$, is a logical operator that
binds a variable and asserts that a formula holds for all elements of the
domain of discourse.

If $\varphi$ is a formula and $x$ is a variable, then $\forall x\,\varphi$ is
a formula, read ``for all $x$, $\varphi$.''
\end{tcolorbox}

\begin{remark}[English reading]
The universal quantifier expresses a global condition over the entire domain.
It does not claim the domain is nonempty on its own, but by the nonempty domain
convention, there is always at least one element to instantiate.
\end{remark}

% ---------------------------------------------------------
% Existential Quantifier
% ---------------------------------------------------------
\begin{tcolorbox}[colback=propbox, colframe=propborder, arc=2pt,
  left=6pt, right=6pt, top=4pt, bottom=4pt,
  title={\small\textbf{Definition (Existential Quantifier)}},
  fonttitle=\small\bfseries]
\label{def:existential-q}
The \emph{existential quantifier}, denoted $\exists$, is a logical operator that
binds a variable and asserts that a formula holds for at least one element of
the domain of discourse.

If $\varphi$ is a formula and $x$ is a variable, then $\exists x\,\varphi$ is
a formula, read ``there exists an $x$ such that $\varphi$.''
\end{tcolorbox}

\begin{remark}[English reading]
The existential quantifier expresses a local condition. It does not specify
which element witnesses the claim; it merely asserts that such a witness exists
somewhere in the domain.
\end{remark}

% ---------------------------------------------------------
% Unique Existential Quantifier
% ---------------------------------------------------------
\begin{tcolorbox}[colback=propbox, colframe=propborder, arc=2pt,
  left=6pt, right=6pt, top=4pt, bottom=4pt,
  title={\small\textbf{Definition (Unique Existential Quantifier)}},
  fonttitle=\small\bfseries]
\label{def:unique-exist}
The \emph{unique existential quantifier} $\exists!$ asserts that there exists
exactly one element satisfying a given property. It is an abbreviation:
\[
\exists! x\,\varphi
\;:=\;
\exists x\,\bigl(\varphi \wedge \forall y\,(\varphi[y/x] \rightarrow y = x)\bigr),
\]
where $y$ is a variable not occurring in $\varphi$.

Equivalently,
\[
\exists! x\,\varphi
\;\equiv\;
\exists x\,\varphi \;\wedge\; \forall x\,\forall y\,
\bigl((\varphi \wedge \varphi[y/x]) \rightarrow x = y\bigr).
\]
\end{tcolorbox}

\begin{remark}[English reading]
$\exists! x\,\varphi$ combines existence (at least one $x$ satisfies $\varphi$)
and uniqueness (at most one such $x$ exists). It is not a primitive quantifier
but a convenient abbreviation.
\end{remark}

\begin{example}
``There is exactly one even prime number'' formalizes as
$\exists! x\,(P(x) \wedge E(x))$, where $P(x)$ means ``$x$ is prime'' and
$E(x)$ means ``$x$ is even.''
\end{example}

% ---------------------------------------------------------
% Bounded Quantifiers
% ---------------------------------------------------------
\begin{tcolorbox}[colback=propbox, colframe=propborder, arc=2pt,
  left=6pt, right=6pt, top=4pt, bottom=4pt,
  title={\small\textbf{Definition (Bounded Quantifiers)}},
  fonttitle=\small\bfseries]
\label{def:bounded-q}
Let $A$ be a set (or a predicate defining a set). The \emph{bounded
quantifiers} are abbreviations:
\begin{align*}
\forall x \in A\,\varphi &\;:=\; \forall x\,(x \in A \rightarrow \varphi), \\
\exists x \in A\,\varphi &\;:=\; \exists x\,(x \in A \wedge \varphi).
\end{align*}
\end{tcolorbox}

\begin{remark}[Asymmetry]
Bounded universal uses implication ($\rightarrow$), while bounded existential
uses conjunction ($\wedge$). This is not arbitrary: if the set $A$ is empty,
$\forall x \in A\,\varphi$ is vacuously true (there is no $x$ to violate the
implication), while $\exists x \in A\,\varphi$ is false (no witness exists).
\end{remark}

\begin{remark}[Common error]
Writing $\forall x \in A\,\varphi$ as $\forall x\,(x \in A \wedge \varphi)$
is wrong: this would assert that every element is in $A$, not just that every
element of $A$ satisfies $\varphi$.
\end{remark}

\begin{theorem}[Negation of Bounded Quantifiers]\label{thm:neg-bounded}
Let $A$ be a set and $\varphi$ a formula. Then:
\begin{align*}
\neg(\forall x \in A\,\varphi) &\;\equiv\; \exists x \in A\,\neg\varphi, \\
\neg(\exists x \in A\,\varphi) &\;\equiv\; \forall x \in A\,\neg\varphi.
\end{align*}
\end{theorem}

\begin{proof}
We derive each case by expanding and applying standard quantifier negation laws.

\textbf{Universal case.}
\begin{align*}
\neg(\forall x \in A\,\varphi)
&\equiv \neg\forall x\,(x \in A \rightarrow \varphi) \\
&\equiv \exists x\,\neg(x \in A \rightarrow \varphi) \\
&\equiv \exists x\,(x \in A \wedge \neg\varphi) \\
&\equiv \exists x \in A\,\neg\varphi.
\end{align*}

\textbf{Existential case.}
\begin{align*}
\neg(\exists x \in A\,\varphi)
&\equiv \neg\exists x\,(x \in A \wedge \varphi) \\
&\equiv \forall x\,\neg(x \in A \wedge \varphi) \\
&\equiv \forall x\,(x \in A \rightarrow \neg\varphi) \\
&\equiv \forall x \in A\,\neg\varphi. \qedhere
\end{align*}
\end{proof}

\begin{remark}
The domain restriction $x \in A$ is preserved under negation because it
functions as a constraint on which elements are quantified over, not as part
of the claim being negated.
\end{remark}

% =========================================================
% Quantifier Laws: Negation, Commutation, Distribution,
% Vacuous Quantification, and Renaming
% =========================================================

\subsection{Quantifiers: Quantifier Laws}

% ---------------------------------------------------------
% TOOLKIT
% ---------------------------------------------------------
\begin{tcolorbox}[colback=gray!6, colframe=gray!40, arc=2pt,
  left=6pt, right=6pt, top=4pt, bottom=4pt,
  title={\small\textbf{Quantifier Laws — Quick Reference}},
  fonttitle=\small\bfseries]
\small
\begin{tabular}{l l l l}
\toprule
\textbf{Law} & \textbf{Equivalence} & \textbf{Condition} & \textbf{Detail} \\
\midrule
Negation (1) & $\neg\forall x\,\varphi \equiv \exists x\,\neg\varphi$ & — & \hyperref[thm:qneg]{↓ Thm} \\
Negation (2) & $\neg\exists x\,\varphi \equiv \forall x\,\neg\varphi$ & — & \hyperref[thm:qneg]{↓ Thm} \\
Same-type commutation & $\forall x\,\forall y\,\varphi \equiv \forall y\,\forall x\,\varphi$ & same type & \hyperref[thm:qcomm]{↓ Thm} \\
Distribution $\forall$/$\wedge$ & $\forall x\,(\varphi\wedge\psi)\equiv(\forall x\,\varphi)\wedge(\forall x\,\psi)$ & — & \hyperref[thm:qdist]{↓ Thm} \\
Distribution $\exists$/$\vee$ & $\exists x\,(\varphi\vee\psi)\equiv(\exists x\,\varphi)\vee(\exists x\,\psi)$ & — & \hyperref[thm:qdist]{↓ Thm} \\
Vacuous $\forall$ & $\forall x\,\varphi \equiv \varphi$ & $x \notin \mathrm{FV}(\varphi)$ & \hyperref[thm:vacuous]{↓ Thm} \\
Renaming & $\forall x\,\varphi \equiv \forall y\,\varphi[y/x]$ & $y$ not free in $\varphi$ & \hyperref[thm:rename]{↓ Thm} \\
\bottomrule
\end{tabular}
\end{tcolorbox}

\vspace{1em}

% ---------------------------------------------------------
% Quantifier Negation
% ---------------------------------------------------------

\begin{proposition}[\texorpdfstring{\hyperref[prf:qneg]{Quantifier Negation Laws}}{Quantifier Negation Laws}]\label{thm:qneg}
For any formula $\varphi$:
\begin{align*}
\neg \forall x\,\varphi &\;\equiv\; \exists x\,\neg \varphi, \\
\neg \exists x\,\varphi &\;\equiv\; \forall x\,\neg \varphi.
\end{align*}
\end{proposition}

\begin{remark}[English reading]
To negate a universally quantified statement, swap $\forall$ for $\exists$ and
negate the inner formula. To negate an existential, swap $\exists$ for $\forall$
and negate. The negation pushes through the quantifier, flipping it.
\end{remark}

\begin{remark}[Procedure for negating nested quantifiers]
When negating a statement with multiple quantifiers, push the negation inward
past every quantifier, flipping each one, until it reaches the atomic predicate.
For example:
\[
\neg(\forall x\,\exists y\,\varphi)
\equiv \exists x\,\neg(\exists y\,\varphi)
\equiv \exists x\,\forall y\,\neg\varphi.
\]
\end{remark}

\begin{remark}[General schema for negating quantified implications]
For $Q_1 x_1 \cdots Q_n x_n (\varphi \rightarrow \psi)$ where each $Q_i \in \{\forall,\exists\}$:
\[
\neg\bigl(Q_1 x_1 \cdots Q_n x_n (\varphi \rightarrow \psi)\bigr)
\;\equiv\;
Q_1' x_1 \cdots Q_n' x_n (\varphi \wedge \neg \psi),
\]
where $Q_i' = \exists$ when $Q_i = \forall$, and $Q_i' = \forall$ when $Q_i = \exists$.
\end{remark}

\begin{remark}[Common error]
Negating only the predicate without flipping the quantifier is incorrect:
$\neg(\forall x\,P(x)) \not\equiv \forall x\,\neg P(x)$.
The right side asserts that \emph{every} element fails $P$, which is much
stronger than merely asserting that not every element satisfies $P$.
\end{remark}

% ---------------------------------------------------------
% Commutation
% ---------------------------------------------------------

\begin{proposition}[Quantifier Commutation]\label{thm:qcomm}
Quantifiers of the same type commute:
\begin{align*}
\forall x\,\forall y\,\varphi &\;\equiv\; \forall y\,\forall x\,\varphi, \\
\exists x\,\exists y\,\varphi &\;\equiv\; \exists y\,\exists x\,\varphi.
\end{align*}
However, quantifiers of different types do \emph{not} commute in general:
\[
\forall x\,\exists y\,\varphi \;\not\equiv\; \exists y\,\forall x\,\varphi.
\]
\end{proposition}

\begin{remark}[Common error]
Swapping $\forall$ and $\exists$ changes the meaning. The statement
$\exists y\,\forall x\,\varphi$ requires a single uniform witness $y$ that
works for all $x$, whereas $\forall x\,\exists y\,\varphi$ allows a different
$y$ for each $x$.
\end{remark}

% ---------------------------------------------------------
% Distribution
% ---------------------------------------------------------

\begin{proposition}[Quantifier Distribution]\label{thm:qdist}
The following distribution equivalences hold:
\begin{align*}
\forall x\,(\varphi \wedge \psi) &\;\equiv\; (\forall x\,\varphi) \wedge (\forall x\,\psi), \\
\exists x\,(\varphi \vee \psi) &\;\equiv\; (\exists x\,\varphi) \vee (\exists x\,\psi).
\end{align*}
The following do \emph{not} hold in general:
\begin{align*}
\forall x\,(\varphi \vee \psi) &\;\not\equiv\; (\forall x\,\varphi) \vee (\forall x\,\psi), \\
\exists x\,(\varphi \wedge \psi) &\;\not\equiv\; (\exists x\,\varphi) \wedge (\exists x\,\psi).
\end{align*}
Additionally, when $x \notin \mathrm{FV}(\psi)$:
\begin{align*}
\forall x\,(\varphi \wedge \psi) &\;\equiv\; (\forall x\,\varphi) \wedge \psi, \\
\exists x\,(\varphi \vee \psi) &\;\equiv\; (\exists x\,\varphi) \vee \psi.
\end{align*}
\end{proposition}

\begin{remark}[Mnemonic]
Universal distributes over conjunction ($\forall$/$\wedge$ match because both
are ``all-or-nothing''); existential distributes over disjunction
($\exists$/$\vee$ match because both assert ``at least one'').
\end{remark}

% ---------------------------------------------------------
% Vacuous Quantification
% ---------------------------------------------------------

\begin{proposition}[\texorpdfstring{\hyperref[prf:vacuous]{Vacuous Quantification}}{Vacuous Quantification}]\label{thm:vacuous}
If $x$ does not occur free in $\varphi$, then:
\begin{align*}
\forall x\,\varphi &\;\equiv\; \varphi, \\
\exists x\,\varphi &\;\equiv\; \varphi.
\end{align*}
\end{proposition}

\begin{remark}[English reading]
A quantifier is vacuous if its bound variable does not appear free in the scope.
Such quantifiers may be freely added or removed without changing meaning.
\end{remark}

% ---------------------------------------------------------
% Renaming Bound Variables
% ---------------------------------------------------------

\begin{proposition}[\texorpdfstring{\hyperref[prf:rename]{Renaming Bound Variables}}{Renaming Bound Variables}]\label{thm:rename}
If $y$ does not occur in $\varphi$, then:
\begin{align*}
\forall x\,\varphi &\;\equiv\; \forall y\,\varphi[y/x], \\
\exists x\,\varphi &\;\equiv\; \exists y\,\varphi[y/x].
\end{align*}
\end{proposition}

\begin{remark}[Consequence]
This is a direct consequence of alpha-equivalence (Definition~\ref{def:alpha-equiv}).
Renaming bound variables is the standard technique for avoiding variable
capture when applying substitution rules.
\end{remark}

% =========================================================
% Prenex Normal Form
% =========================================================

\subsection{Quantifiers: Prenex Normal Form}

% ---------------------------------------------------------
% TOOLKIT
% ---------------------------------------------------------
\begin{tcolorbox}[colback=gray!6, colframe=gray!40, arc=2pt,
  left=6pt, right=6pt, top=4pt, bottom=4pt,
  title={\small\textbf{Prenex Normal Form — Quick Reference}},
  fonttitle=\small\bfseries]
\small
\begin{tabular}{l l l}
\toprule
\textbf{Concept} & \textbf{Meaning} & \textbf{Detail} \\
\midrule
Prenex normal form (PNF) & All quantifiers at front, quantifier-free matrix & \hyperref[def:pnf]{↓ Def} \\
PNF Theorem & Every formula is equivalent to a PNF formula & \hyperref[thm:pnf]{↓ Thm} \\
Quantifier-pulling rules & Equivalences for extracting quantifiers & \hyperref[def:pnf]{↓ Def} \\
\bottomrule
\end{tabular}
\end{tcolorbox}

\vspace{1em}

% ---------------------------------------------------------
% Prenex Normal Form
% ---------------------------------------------------------
\begin{tcolorbox}[colback=propbox, colframe=propborder, arc=2pt,
  left=6pt, right=6pt, top=4pt, bottom=4pt,
  title={\small\textbf{Definition (Prenex Normal Form)}},
  fonttitle=\small\bfseries]
\label{def:pnf}
A first-order formula is in \emph{prenex normal form} (PNF) if it has the shape
\[
Q_1 x_1\, Q_2 x_2 \cdots Q_n x_n \;\; \psi,
\]
where each $Q_i \in \{\forall,\exists\}$ and $\psi$ is quantifier-free. The
string $Q_1 x_1 \cdots Q_n x_n$ is the \emph{quantifier prefix}; $\psi$ is
the \emph{matrix}.
\end{tcolorbox}

% ---------------------------------------------------------
% PNF Theorem
% ---------------------------------------------------------
\begin{tcolorbox}[colback=thmbox, colframe=thmborder, arc=2pt,
  left=6pt, right=6pt, top=4pt, bottom=4pt,
  title={\small\textbf{Theorem (Prenex Normal Form Theorem)}},
  fonttitle=\small\bfseries]
\label{thm:pnf}
Every first-order formula is logically equivalent to a formula in prenex normal
form.
\end{tcolorbox}

\begin{remark}[Proof strategy — conversion procedure]
Given a formula $\varphi$, the following four steps produce an equivalent PNF:
\begin{enumerate}
  \item \textbf{Eliminate $\leftrightarrow$ and $\rightarrow$.}
  Rewrite using only $\neg,\wedge,\vee$:
  $(A \rightarrow B) \equiv (\neg A \vee B)$.
  \item \textbf{Push negations inward (Negation Normal Form).}
  Use De Morgan's laws and quantifier-negation laws until $\neg$ applies only
  to atomic formulas.
  \item \textbf{Standardize bound variables apart.}
  Rename bound variables so that no variable is quantified twice and no bound
  variable coincides with any free variable.
  \item \textbf{Pull quantifiers outward.}
  Apply the quantifier-pulling equivalences (below) to move all quantifiers
  to the front.
\end{enumerate}
\end{remark}

\begin{remark}[Quantifier-pulling equivalences]
The following hold when $x \notin \mathrm{FV}(\psi)$:
\begin{center}
\renewcommand{\arraystretch}{1.25}
\begin{tabular}{ll}
\toprule
\textbf{Equivalence} & \textbf{Side condition} \\
\midrule
$(\forall x\,\phi)\wedge \psi \equiv \forall x\,(\phi\wedge \psi)$ & $x \notin \mathrm{FV}(\psi)$ \\
$(\exists x\,\phi)\vee \psi \equiv \exists x\,(\phi\vee \psi)$ & $x \notin \mathrm{FV}(\psi)$ \\
$(\forall x\,\phi)\vee \psi \equiv \forall x\,(\phi\vee \psi)$ & $x \notin \mathrm{FV}(\psi)$ \\
$(\exists x\,\phi)\wedge \psi \equiv \exists x\,(\phi\wedge \psi)$ & $x \notin \mathrm{FV}(\psi)$ \\
\bottomrule
\end{tabular}
\end{center}
\end{remark}

\begin{example}[PNF conversion]
Normalize $\varphi := \neg\bigl(\forall x\,P(x) \rightarrow \exists y\,Q(y)\bigr)$.

\textbf{Step 1 (eliminate $\rightarrow$):}
$\varphi \equiv \neg\bigl(\neg\forall x\,P(x) \vee \exists y\,Q(y)\bigr)$.

\textbf{Step 2 (push $\neg$ inward):}
$\varphi \equiv (\forall x\,P(x)) \wedge (\forall y\,\neg Q(y))$.

\textbf{Step 3 (standardize apart):} Variables are already distinct.

\textbf{Step 4 (pull quantifiers outward):}
$\varphi \equiv \forall x\,\forall y\,(P(x) \wedge \neg Q(y))$.
\end{example}

% =========================================================
% Logical Strength and Quantifier Order
% =========================================================

\subsection{Quantifiers: Logical Strength and Quantifier Order}

% ---------------------------------------------------------
% TOOLKIT
% ---------------------------------------------------------
\begin{tcolorbox}[colback=gray!6, colframe=gray!40, arc=2pt,
  left=6pt, right=6pt, top=4pt, bottom=4pt,
  title={\small\textbf{Logical Strength — Quick Reference}},
  fonttitle=\small\bfseries]
\small
\begin{tabular}{l l l}
\toprule
\textbf{Concept} & \textbf{Meaning} & \textbf{Detail} \\
\midrule
Entailment $A \models B$ & $A$ is true implies $B$ is true & \hyperref[def:entailment]{↓ Def} \\
Logical strength & $A$ stronger than $B$ if $A \models B$ but $B \not\models A$ & \hyperref[def:strength]{↓ Def} \\
Model-set viewpoint & $A \models B \iff \mathrm{Mod}(A) \subseteq \mathrm{Mod}(B)$ & \hyperref[def:model-set]{↓ Def} \\
Quantifier alternation & Alternating $\forall$/$\exists$ in prefix & \hyperref[def:q-alt]{↓ Def} \\
\bottomrule
\end{tabular}
\end{tcolorbox}

\vspace{1em}

% ---------------------------------------------------------
% Entailment
% ---------------------------------------------------------
\begin{tcolorbox}[colback=propbox, colframe=propborder, arc=2pt,
  left=6pt, right=6pt, top=4pt, bottom=4pt,
  title={\small\textbf{Definition (Logical Entailment)}},
  fonttitle=\small\bfseries]
\label{def:entailment}
Let $A$ and $B$ be sentences. We write $A \models B$ and say that \emph{$A$
entails $B$} if in every structure in which $A$ is true, $B$ is also true.
Equivalently, $A \models B$ means $A \rightarrow B$ is valid.
\end{tcolorbox}

% ---------------------------------------------------------
% Logical Strength
% ---------------------------------------------------------
\begin{tcolorbox}[colback=propbox, colframe=propborder, arc=2pt,
  left=6pt, right=6pt, top=4pt, bottom=4pt,
  title={\small\textbf{Definition (Logical Strength)}},
  fonttitle=\small\bfseries]
\label{def:strength}
We say that \emph{$A$ is (logically) stronger than $B$} if $A \models B$ but
$B \not\models A$.

$A$ and $B$ are \emph{logically equivalent} if $A \models B$ and $B \models A$.
\end{tcolorbox}

\begin{remark}[English reading]
``Stronger'' means ``harder to satisfy'' (fewer models make it true). If $A$ is
stronger than $B$, then $A$ rules out more possibilities: every world where $A$
holds is a world where $B$ holds, but not conversely.
\end{remark}

% ---------------------------------------------------------
% Model-Set Viewpoint
% ---------------------------------------------------------
\begin{tcolorbox}[colback=propbox, colframe=propborder, arc=2pt,
  left=6pt, right=6pt, top=4pt, bottom=4pt,
  title={\small\textbf{Definition (Model-Set Viewpoint)}},
  fonttitle=\small\bfseries]
\label{def:model-set}
For a statement $A$, let $\mathrm{Mod}(A)$ denote the class of all structures
in which $A$ is true. Then
\[
A \models B \quad\Longleftrightarrow\quad \mathrm{Mod}(A) \subseteq \mathrm{Mod}(B).
\]
\end{tcolorbox}

\begin{remark}[English reading]
Strength is set inclusion of model classes. ``$A$ is stronger than $B$'' is
literally $\mathrm{Mod}(A) \subsetneq \mathrm{Mod}(B)$: $A$ satisfies fewer
structures than $B$.
\end{remark}

% ---------------------------------------------------------
% Quantifier Alternation
% ---------------------------------------------------------
\begin{tcolorbox}[colback=propbox, colframe=propborder, arc=2pt,
  left=6pt, right=6pt, top=4pt, bottom=4pt,
  title={\small\textbf{Definition (Quantifier Alternation)}},
  fonttitle=\small\bfseries]
\label{def:q-alt}
A \emph{quantifier alternation} occurs when $Q_i \neq Q_{i+1}$ for some $i$ in
the prefix $Q_1 x_1 \cdots Q_k x_k$ of a formula. For example,
$\forall x\,\exists y\,\Phi(x,y)$ has one alternation, while
$\forall x\,\forall y\,\Phi(x,y)$ has none.
\end{tcolorbox}

\begin{remark}[Why alternation matters for strength]
$\forall x\,\exists y\,\Phi(x,y)$ requires a rule assigning a (possibly
$x$-dependent) witness $y$ to each $x$. In contrast,
$\exists y\,\forall x\,\Phi(x,y)$ requires a single uniform witness $y$ that
works for all $x$. Uniformity is strictly stronger than dependence.
\end{remark}

\begin{remark}[Strength hierarchy]
The following entailments hold:
\begin{enumerate}
  \item $\forall x\,\Phi(x) \models \exists x\,\Phi(x)$ (universal is stronger).
  \item $\exists y\,\forall x\,\Phi(x,y) \models \forall x\,\exists y\,\Phi(x,y)$
  (uniform witness is stronger than dependent witness).
\end{enumerate}
Neither converse holds in general.
\end{remark}

\begin{example}[Swap failure on $\mathbb{R}$]
Let the domain be $\mathbb{R}$ and $\Phi(x,y) = (y > x)$.

$\forall x\,\exists y\,(y > x)$ is true: given any $x$, take $y = x+1$.

$\exists y\,\forall x\,(y > x)$ is false: no single real exceeds all reals.

Hence $\exists y\,\forall x$ is strictly stronger than $\forall x\,\exists y$.
\end{example}

\begin{example}[Universal vs.\ existential]
Let the domain be $\mathbb{R}$ and $\Phi(x) = (x^2 \geq 0)$.

$\forall x\,(x^2 \geq 0)$ entails $\exists x\,(x^2 \geq 0)$, but not conversely.
Thus the universal statement is strictly stronger.
\end{example}

\begin{remark}[Logical strength increases when you:]
\begin{itemize}
  \item add conjuncts ($A \wedge B$ is stronger than $A$),
  \item replace $\exists$ with $\forall$,
  \item move existential quantifiers outward (requiring uniform witnesses),
  \item introduce quantifier alternation.
\end{itemize}
\end{remark}


% =========================================================
% Proof Theory — Orchestrator
% =========================================================

% =========================================================
% Inference Rules for Quantifiers
% =========================================================

\subsection{Proof Theory: Quantifier Inference Rules}

% ---------------------------------------------------------
% TOOLKIT
% ---------------------------------------------------------
\begin{tcolorbox}[colback=gray!6, colframe=gray!40, arc=2pt,
  left=6pt, right=6pt, top=4pt, bottom=4pt,
  title={\small\textbf{Quantifier Inference Rules — Quick Reference}},
  fonttitle=\small\bfseries]
\small
\begin{tabular}{l l l l}
\toprule
\textbf{Rule} & \textbf{Schema} & \textbf{Key restriction} & \textbf{Detail} \\
\midrule
UI  & $\forall x\,\varphi \Rightarrow \varphi[t/x]$ & $t$ free for $x$ & \hyperref[def:ui]{↓ Def} \\
UG  & $\varphi \Rightarrow \forall x\,\varphi$ & $x$ not free in assumptions & \hyperref[def:ug]{↓ Def} \\
EI  & $\exists x\,\varphi \Rightarrow \varphi[c/x]$ & $c$ fresh witness & \hyperref[def:ei]{↓ Def} \\
EG  & $\varphi[t/x] \Rightarrow \exists x\,\varphi$ & none & \hyperref[def:eg]{↓ Def} \\
\bottomrule
\end{tabular}
\end{tcolorbox}

\vspace{1em}

% ---------------------------------------------------------
% Universal Instantiation
% ---------------------------------------------------------
\begin{tcolorbox}[colback=propbox, colframe=propborder, arc=2pt,
  left=6pt, right=6pt, top=4pt, bottom=4pt,
  title={\small\textbf{Definition (Universal Instantiation — UI)}},
  fonttitle=\small\bfseries]
\label{def:ui}
From a universally quantified statement, infer any instance obtained by
substituting a term for the bound variable:
\[
\forall x\,\varphi \;\Rightarrow\; \varphi[t/x].
\]
\end{tcolorbox}

\begin{remark}[Side condition]
The term $t$ must be free for $x$ in $\varphi$. If $t$ contains a variable
that would become bound after substitution, the inference is invalid.
\end{remark}

% ---------------------------------------------------------
% Universal Generalization
% ---------------------------------------------------------
\begin{tcolorbox}[colback=propbox, colframe=propborder, arc=2pt,
  left=6pt, right=6pt, top=4pt, bottom=4pt,
  title={\small\textbf{Definition (Universal Generalization — UG)}},
  fonttitle=\small\bfseries]
\label{def:ug}
From a formula, infer a universally quantified statement:
\[
\varphi \;\Rightarrow\; \forall x\,\varphi.
\]
\end{tcolorbox}

\begin{remark}[Restriction on UG]
The variable $x$ must not occur free in any undischarged assumption on which
$\varphi$ depends. This restriction ensures that $x$ is truly arbitrary —
not tied to a specific hypothesis about $x$.
\end{remark}

\begin{remark}[Arbitrary element argument]
To prove $\forall x\,\varphi(x)$, fix an arbitrary element $x$ (introducing no
special assumptions about $x$), prove $\varphi(x)$, and then apply UG. The
variable $x$ must remain unconstrained throughout.
\end{remark}

% ---------------------------------------------------------
% Existential Instantiation
% ---------------------------------------------------------
\begin{tcolorbox}[colback=propbox, colframe=propborder, arc=2pt,
  left=6pt, right=6pt, top=4pt, bottom=4pt,
  title={\small\textbf{Definition (Existential Instantiation — EI)}},
  fonttitle=\small\bfseries]
\label{def:ei}
From an existentially quantified statement, infer an instance with a fresh
constant (witness):
\[
\exists x\,\varphi \;\Rightarrow\; \varphi[c/x].
\]
\end{tcolorbox}

\begin{remark}[Witness discipline for EI]
The constant $c$ must be fresh: it must not appear in $\varphi$, in any
undischarged assumption, or in the final conclusion of the proof. The constant
represents an arbitrary but fixed witness to the existential claim, not a
specific known object.
\end{remark}

\begin{remark}[Common error]
Using a witness constant introduced by EI outside its allowed scope, or
choosing $c$ before establishing the existential premise, invalidates the proof.
\end{remark}

% ---------------------------------------------------------
% Existential Generalization
% ---------------------------------------------------------
\begin{tcolorbox}[colback=propbox, colframe=propborder, arc=2pt,
  left=6pt, right=6pt, top=4pt, bottom=4pt,
  title={\small\textbf{Definition (Existential Generalization — EG)}},
  fonttitle=\small\bfseries]
\label{def:eg}
From a formula containing a term, infer an existential statement:
\[
\varphi[t/x] \;\Rightarrow\; \exists x\,\varphi.
\]
\end{tcolorbox}

\begin{remark}[No side conditions]
EG has no side conditions: any term $t$ may be replaced by an existentially
quantified variable. To prove $\exists x\,\varphi(x)$, explicitly exhibit a
term $t$ such that $\varphi(t)$ holds, then apply EG.
\end{remark}

\begin{remark}[Counterexample argument]
To refute $\forall x\,\varphi(x)$: produce a single term $t$ such that
$\neg\varphi(t)$ holds. To refute $\exists x\,\varphi(x)$: show that $\varphi(t)$
fails for every term $t$ in the domain.
\end{remark}

% =========================================================
% Equality Rules
% =========================================================

\subsection{Proof Theory: Equality Rules}

% ---------------------------------------------------------
% TOOLKIT
% ---------------------------------------------------------
\begin{tcolorbox}[colback=gray!6, colframe=gray!40, arc=2pt,
  left=6pt, right=6pt, top=4pt, bottom=4pt,
  title={\small\textbf{Equality Rules — Quick Reference}},
  fonttitle=\small\bfseries]
\small
\begin{tabular}{l l l l}
\toprule
\textbf{Rule} & \textbf{Schema} & \textbf{Status} & \textbf{Detail} \\
\midrule
Reflexivity (EqI)   & $t = t$ & Primitive rule & \hyperref[def:eq-refl]{↓ Def} \\
Equality Elim (EqE) & $t_1=t_2,\;\varphi[t_1/x] \Rightarrow \varphi[t_2/x]$ & Primitive rule & \hyperref[def:eq-elim]{↓ Def} \\
Symmetry            & $t_1=t_2 \Rightarrow t_2=t_1$ & Derived & \hyperref[thm:eq-sym]{↓ Thm} \\
Transitivity        & $t_1=t_2,\;t_2=t_3 \Rightarrow t_1=t_3$ & Derived & \hyperref[thm:eq-trans]{↓ Thm} \\
Term substitution   & $f(\dots,t_1,\dots)=f(\dots,t_2,\dots)$ & Derived & \hyperref[thm:eq-term]{↓ Thm} \\
Predicate substitution & $P(\dots,t_1,\dots)\Leftrightarrow P(\dots,t_2,\dots)$ & Derived & \hyperref[thm:eq-pred]{↓ Thm} \\
\bottomrule
\end{tabular}
\end{tcolorbox}

\vspace{1em}

% ---------------------------------------------------------
% Equality Introduction (Reflexivity)
% ---------------------------------------------------------
\begin{tcolorbox}[colback=propbox, colframe=propborder, arc=2pt,
  left=6pt, right=6pt, top=4pt, bottom=4pt,
  title={\small\textbf{Definition (Equality Introduction — Reflexivity)}},
  fonttitle=\small\bfseries]
\label{def:eq-refl}
For any term $t$, one may infer $t = t$.
\end{tcolorbox}

\begin{remark}[English reading]
Reflexivity requires no premises and holds for all terms under all
interpretations. It is the foundational rule that allows equality reasoning to
get off the ground.
\end{remark}

% ---------------------------------------------------------
% Equality Elimination
% ---------------------------------------------------------
\begin{tcolorbox}[colback=propbox, colframe=propborder, arc=2pt,
  left=6pt, right=6pt, top=4pt, bottom=4pt,
  title={\small\textbf{Definition (Equality Elimination — Substitution of Equals)}},
  fonttitle=\small\bfseries]
\label{def:eq-elim}
From $t_1 = t_2$ and $\varphi[t_1/x]$, one may infer $\varphi[t_2/x]$.
\end{tcolorbox}

\begin{remark}[English reading]
Equality elimination permits replacing a term by an equal term in any formula
position, provided the substitution is capture-avoiding. This formalizes the
intuition that equal things can be substituted for each other.
\end{remark}

% ---------------------------------------------------------
% Derived Rules
% ---------------------------------------------------------

\begin{theorem}[Symmetry of Equality]\label{thm:eq-sym}
From $t_1 = t_2$, one may infer $t_2 = t_1$.
\end{theorem}

\begin{theorem}[Transitivity of Equality]\label{thm:eq-trans}
From $t_1 = t_2$ and $t_2 = t_3$, one may infer $t_1 = t_3$.
\end{theorem}

\begin{theorem}[Term Substitution under Equality]\label{thm:eq-term}
If $t_1 = t_2$, then for any function symbol $f$,
$f(\dots,t_1,\dots) = f(\dots,t_2,\dots)$.
\end{theorem}

\begin{theorem}[Predicate Substitution under Equality]\label{thm:eq-pred}
If $t_1 = t_2$ and $P$ is an $n$-ary predicate symbol, then
$P(\dots,t_1,\dots) \;\Leftrightarrow\; P(\dots,t_2,\dots)$.
\end{theorem}

\begin{remark}[Summary]
These derived rules collectively express that functions and predicates
\emph{respect equality}: equal inputs produce equal outputs (for functions) or
equivalent propositions (for predicates). Together they amount to the principle
of Leibniz substitutivity.
\end{remark}

\begin{remark}[Common error]
When substituting equals for equals, the substitution must be made consistently.
Replacing a term by an equal in one occurrence but not another in the same
formula can produce incorrect conclusions.
\end{remark}

% =========================================================
% Soundness and Completeness of First-Order Logic
% =========================================================

\subsection{Proof Theory: Soundness and Completeness}

% ---------------------------------------------------------
% TOOLKIT
% ---------------------------------------------------------
\begin{tcolorbox}[colback=gray!6, colframe=gray!40, arc=2pt,
  left=6pt, right=6pt, top=4pt, bottom=4pt,
  title={\small\textbf{Soundness and Completeness — Quick Reference}},
  fonttitle=\small\bfseries]
\small
\begin{tabular}{l l l}
\toprule
\textbf{Property} & \textbf{Direction} & \textbf{Detail} \\
\midrule
Soundness    & $\Gamma \vdash \varphi \Rightarrow \Gamma \models \varphi$ & \hyperref[def:soundness]{↓ Def} \\
Completeness & $\Gamma \models \varphi \Rightarrow \Gamma \vdash \varphi$ & \hyperref[def:completeness]{↓ Def} \\
Soundness Theorem (FOL) & All standard FOL rules are sound & \hyperref[thm:fol-sound]{↓ Thm} \\
Gödel's Completeness Theorem & FOL is complete & \hyperref[thm:godel-complete]{↓ Thm} \\
\bottomrule
\end{tabular}
\end{tcolorbox}

\vspace{1em}

% ---------------------------------------------------------
% Soundness and Completeness (Definitions)
% ---------------------------------------------------------
\begin{tcolorbox}[colback=propbox, colframe=propborder, arc=2pt,
  left=6pt, right=6pt, top=4pt, bottom=4pt,
  title={\small\textbf{Definition (Soundness)}},
  fonttitle=\small\bfseries]
\label{def:soundness}
A proof system is \emph{sound} if every provable formula is valid:
\[
\Gamma \vdash \varphi \implies \Gamma \models \varphi.
\]
\end{tcolorbox}

\begin{tcolorbox}[colback=propbox, colframe=propborder, arc=2pt,
  left=6pt, right=6pt, top=4pt, bottom=4pt,
  title={\small\textbf{Definition (Completeness)}},
  fonttitle=\small\bfseries]
\label{def:completeness}
A proof system is \emph{complete} if every valid formula is provable:
\[
\Gamma \models \varphi \implies \Gamma \vdash \varphi.
\]
\end{tcolorbox}

% ---------------------------------------------------------
% Soundness Theorem for FOL
% ---------------------------------------------------------
\begin{theorem}[Soundness of First-Order Logic]\label{thm:fol-sound}
The standard inference rules for first-order logic (UI, UG, EI, EG, and the
propositional rules) are sound: they preserve truth in all structures.

If $\Gamma \vdash \varphi$, then $\Gamma \models \varphi$.
\end{theorem}

% ---------------------------------------------------------
% Gödel's Completeness Theorem
% ---------------------------------------------------------
\begin{tcolorbox}[colback=thmbox, colframe=thmborder, arc=2pt,
  left=6pt, right=6pt, top=4pt, bottom=4pt,
  title={\small\textbf{Theorem (Gödel's Completeness Theorem)}},
  fonttitle=\small\bfseries]
\label{thm:godel-complete}
First-order logic is complete: every logically valid formula is provable.
\[
\Gamma \models \varphi \implies \Gamma \vdash \varphi.
\]
Equivalently: if a set of sentences $\Gamma$ is consistent (has no proof of
contradiction), then $\Gamma$ has a model.
\end{tcolorbox}

\begin{remark}[English reading]
Soundness ensures proofs do not lead us astray: we cannot prove false statements
from true premises. Completeness ensures proofs are powerful enough: every
statement that is true in all models can be established by a formal proof.

Together, soundness and completeness show that syntactic provability ($\vdash$)
and semantic entailment ($\models$) coincide for first-order logic.
\end{remark}

\begin{remark}[Completeness vs.\ incompleteness]
Gödel's completeness theorem (1930) should not be confused with his
incompleteness theorems (1931). The completeness theorem concerns first-order
logic as a proof system. The incompleteness theorems concern the limitations of
formal theories capable of expressing arithmetic, and are a separate result.
\end{remark}


% =========================================================
% Translation — Orchestrator
% =========================================================

% =========================================================
% English-to-Logic Translation and Scope Ambiguity
% =========================================================

\subsection{Translation: English to Logic and Scope Ambiguity}

% ---------------------------------------------------------
% TOOLKIT
% ---------------------------------------------------------
\begin{tcolorbox}[colback=gray!6, colframe=gray!40, arc=2pt,
  left=6pt, right=6pt, top=4pt, bottom=4pt,
  title={\small\textbf{Translation — Quick Reference}},
  fonttitle=\small\bfseries]
\small
\begin{tabular}{l l l l}
\toprule
\textbf{English pattern} & \textbf{Logical form} & \textbf{Negation} & \textbf{Detail} \\
\midrule
Everyone has $P$ & $\forall x\,P(x)$ & $\exists x\,\neg P(x)$ & \hyperref[tab:translations]{↓ Table} \\
Someone has $P$ & $\exists x\,P(x)$ & $\forall x\,\neg P(x)$ & \hyperref[tab:translations]{↓ Table} \\
Everyone likes something & $\forall x\,\exists y\,L(x,y)$ & $\exists x\,\forall y\,\neg L(x,y)$ & \hyperref[tab:translations]{↓ Table} \\
Someone likes everything & $\exists x\,\forall y\,L(x,y)$ & $\forall x\,\exists y\,\neg L(x,y)$ & \hyperref[tab:translations]{↓ Table} \\
Scope ambiguity & Two readings differ by quantifier order & — & \hyperref[tab:scope-ambiguity]{↓ Table} \\
\bottomrule
\end{tabular}
\end{tcolorbox}

\vspace{1em}

\begin{remark}[Translation discipline]
When translating a symbolic formula into natural language, respect the original
quantifier nesting of the formula. Do not reorder quantifiers unless explicitly
asked to normalize or rewrite the expression.
\end{remark}

\begin{center}\label{tab:translations}
\renewcommand{\arraystretch}{1.35}
\begin{tabular}{|p{5cm}|p{4.5cm}|p{4.5cm}|}
\hline
\textbf{English} & \textbf{Logical Form} & \textbf{Negation} \\ \hline
Everyone has property $P$ & $\forall x\,P(x)$ & $\exists x\,\neg P(x)$ \\ \hline
Someone has property $P$ & $\exists x\,P(x)$ & $\forall x\,\neg P(x)$ \\ \hline
Everyone likes something & $\forall x\,\exists y\,L(x,y)$ & $\exists x\,\forall y\,\neg L(x,y)$ \\ \hline
Someone likes everything & $\exists x\,\forall y\,L(x,y)$ & $\forall x\,\exists y\,\neg L(x,y)$ \\ \hline
Every student passed every exam & $\forall x\,\forall y\,P(x,y)$ & $\exists x\,\exists y\,\neg P(x,y)$ \\ \hline
Some student passed every exam & $\exists x\,\forall y\,P(x,y)$ & $\forall x\,\exists y\,\neg P(x,y)$ \\ \hline
\end{tabular}
\end{center}

\vspace{1em}

\begin{remark}[Scope determines meaning]\label{tab:scope-ambiguity}
Natural language often leaves quantifier scope ambiguous. Formal logic resolves
this by fixing a precise quantifier order. Changing the order of quantifiers
generally changes the meaning of the statement.
\end{remark}

\begin{center}
\renewcommand{\arraystretch}{1.35}
\begin{tabular}{|p{4.5cm}|p{4.5cm}|p{6cm}|}
\hline
\textbf{English} & \textbf{Logical Form} & \textbf{Meaning} \\ \hline
\multirow{2}{*}{Everyone loves someone}
  & $\forall x\,\exists y\,L(x,y)$ & Each person may love a different person. \\
  & $\exists y\,\forall x\,L(x,y)$ & There is one person whom everyone loves. \\ \hline
\multirow{2}{*}{Every student passed an exam}
  & $\forall x\,\exists y\,P(x,y)$ & Each student passed at least one (possibly different) exam. \\
  & $\exists y\,\forall x\,P(x,y)$ & There is a single exam that all students passed. \\ \hline
\multirow{2}{*}{A teacher knows every student}
  & $\exists x\,\forall y\,K(x,y)$ & There is one teacher who knows all students. \\
  & $\forall y\,\exists x\,K(x,y)$ & Every student is known by at least one teacher. \\ \hline
\multirow{2}{*}{Every function has a zero}
  & $\forall f\,\exists x\,Z(f,x)$ & Each function has at least one (possibly different) zero. \\
  & $\exists x\,\forall f\,Z(f,x)$ & There is a single point that is a zero of every function. \\ \hline
\end{tabular}
\end{center}

% =========================================================
% Square of Opposition
% =========================================================

\subsection{Translation: Square of Opposition}

% ---------------------------------------------------------
% TOOLKIT
% ---------------------------------------------------------
\begin{tcolorbox}[colback=gray!6, colframe=gray!40, arc=2pt,
  left=6pt, right=6pt, top=4pt, bottom=4pt,
  title={\small\textbf{Square of Opposition — Quick Reference}},
  fonttitle=\small\bfseries]
\small
\begin{tabular}{l l l l}
\toprule
\textbf{Form} & \textbf{Name} & \textbf{Formula} & \textbf{Detail} \\
\midrule
A & Universal Affirmative  & $\forall x\,P(x)$       & \hyperref[def:opp-forms]{↓ Def} \\
E & Universal Negative     & $\forall x\,\neg P(x)$  & \hyperref[def:opp-forms]{↓ Def} \\
I & Existential Affirmative & $\exists x\,P(x)$      & \hyperref[def:opp-forms]{↓ Def} \\
O & Existential Negative   & $\exists x\,\neg P(x)$  & \hyperref[def:opp-forms]{↓ Def} \\
\bottomrule
\end{tabular}
\end{tcolorbox}

\vspace{1em}

\begin{tcolorbox}[colback=propbox, colframe=propborder, arc=2pt,
  left=6pt, right=6pt, top=4pt, bottom=4pt,
  title={\small\textbf{Definition (Quantified Opposition Forms)}},
  fonttitle=\small\bfseries]
\label{def:opp-forms}
Let $P(x)$ be a formula with one free variable. The four standard quantified
forms are:
\begin{align*}
\textbf{Universal Affirmative (A):} &\quad \forall x\,P(x) \\
\textbf{Universal Negative (E):} &\quad \forall x\,\neg P(x) \\
\textbf{Existential Affirmative (I):} &\quad \exists x\,P(x) \\
\textbf{Existential Negative (O):} &\quad \exists x\,\neg P(x)
\end{align*}
\end{tcolorbox}

\vspace{1em}

\begin{center}
\begin{tikzpicture}[
  every node/.style={draw, rectangle, align=center, minimum width=4cm, minimum height=1.2cm},
  node distance=3.8cm
]

\node (A) {$\forall x\,P(x)$\\\small Universal Affirmative (A)};
\node (E) [right=of A] {$\forall x\,\neg P(x)$\\\small Universal Negative (E)};
\node (I) [below=of A] {$\exists x\,P(x)$\\\small Existential Affirmative (I)};
\node (O) [below=of E] {$\exists x\,\neg P(x)$\\\small Existential Negative (O)};

\draw[<->, thick] (A) -- (O);
\draw[<->, thick] (E) -- (I);
\draw[->, thick]  (A) -- (I);
\draw[->, thick]  (E) -- (O);
\draw[<->, dashed] (A) -- (E);
\draw[<->, dashed] (I) -- (O);

\end{tikzpicture}
\end{center}

\begin{remark}[Logical relations in first-order logic with nonempty domains]
\begin{itemize}
  \item \textbf{Contradictories:} A and O ($\forall x\,P(x)$ vs $\exists x\,\neg P(x)$);
  E and I ($\forall x\,\neg P(x)$ vs $\exists x\,P(x)$). These pairs cannot both
  be true and cannot both be false.
  \item \textbf{Subalternation:} A entails I ($\forall x\,P(x) \Rightarrow \exists x\,P(x)$);
  E entails O ($\forall x\,\neg P(x) \Rightarrow \exists x\,\neg P(x)$). These
  hold in classical first-order logic with nonempty domains.
  \item \textbf{Contraries and subcontraries} do not generally hold in first-order
  logic without existential presuppositions.
\end{itemize}
\end{remark}

\begin{remark}[Modern status of the square]
In classical first-order logic, only contradiction and subalternation are
logically valid relations. The traditional notions of contrariety and
subcontrariety rely on existential assumptions and are not preserved in general
model-theoretic semantics.
\end{remark}


% =========================================================
% Reference — Orchestrator
% =========================================================

% =========================================================
% Common Errors and Fallacies
% =========================================================

\subsection{Reference: Common Errors and Fallacies}

% ---------------------------------------------------------
% TOOLKIT
% ---------------------------------------------------------
\begin{tcolorbox}[colback=gray!6, colframe=gray!40, arc=2pt,
  left=6pt, right=6pt, top=4pt, bottom=4pt,
  title={\small\textbf{Errors and Fallacies — Quick Reference}},
  fonttitle=\small\bfseries]
\small
\begin{tabular}{l l l}
\toprule
\textbf{Error category} & \textbf{Where it arises} & \textbf{Detail} \\
\midrule
Quantifier fallacies & Negation, scope, and swap errors & \hyperref[tab:qfallacies]{↓ Table} \\
Common incorrect negations & Forgetting to flip quantifiers & \hyperref[tab:neg-errors]{↓ Table} \\
Inference rule errors & Misuse of UI, UG, EI, EG & \hyperref[rem:inf-errors]{↓ Rem} \\
\bottomrule
\end{tabular}
\end{tcolorbox}

\vspace{1em}

\begin{center}\label{tab:qfallacies}
\renewcommand{\arraystretch}{1.4}
\begin{tabular}{|p{5.5cm}|p{7.5cm}|}
\hline
\textbf{Fallacy} & \textbf{Diagnostic Question} \\ \hline
Failure to flip quantifier under negation & Was $\forall$ changed to $\exists$ (or vice versa) when negating? \\ \hline
Negating predicate only & Was only the predicate negated while the quantifier was left unchanged? \\ \hline
Partial quantifier negation & When negating nested quantifiers, were \emph{all} quantifiers flipped? \\ \hline
Quantifier order confusion & Was the order of quantifiers changed without justification? \\ \hline
Illicit quantifier swap & Were $\forall$ and $\exists$ commuted when they are not of the same type? \\ \hline
Variable capture & Did substitution introduce a variable that became bound unintentionally? \\ \hline
Illegal existential instantiation & Was a witness chosen before an existential statement was established? \\ \hline
Illicit universal generalization & Was a universally quantified conclusion drawn from a statement depending on a specific object? \\ \hline
Scope ambiguity & Was the scope of a quantifier unclear or implicitly extended beyond its syntactic bounds? \\ \hline
Vacuous quantification misuse & Was a quantifier added or removed even though the variable does not occur free in the formula? \\ \hline
\end{tabular}
\end{center}

\begin{remark}
Quantifier errors almost always stem from ignoring scope or treating quantifiers
as informal linguistic modifiers. Every quantifier introduces a binding context
that must be respected syntactically before semantic reasoning is applied.
\end{remark}

\vspace{1em}

\begin{center}\label{tab:neg-errors}
\renewcommand{\arraystretch}{1.35}
\begin{tabular}{|p{4.5cm}|p{4.5cm}|p{4.5cm}|}
\hline
\textbf{Original} & \textbf{Incorrect Negation} & \textbf{Why It Is Wrong} \\ \hline
$\forall x\,P(x)$ & $\forall x\,\neg P(x)$ & Fails to flip the quantifier; asserts everyone fails $P$. \\ \hline
$\exists x\,P(x)$ & $\exists x\,\neg P(x)$ & Negates the predicate but not the existential claim. \\ \hline
$\forall x\,\exists y\,P(x,y)$ & $\exists x\,\exists y\,\neg P(x,y)$ & Only the outer quantifier was flipped. \\ \hline
$\exists x\,\forall y\,P(x,y)$ & $\forall x\,\forall y\,\neg P(x,y)$ & Overstrengthened the negation. \\ \hline
$\forall x\,(P(x)\rightarrow Q(x))$ & $\forall x\,(P(x)\wedge \neg Q(x))$ & Correct negation is $\exists x\,(P(x)\wedge \neg Q(x))$. \\ \hline
\end{tabular}
\end{center}

\begin{remark}[Negation discipline]
When negating a quantified statement, push the negation inward across
\emph{every} quantifier, flipping each one, until it reaches the atomic
predicate. Negating only the predicate or only one quantifier changes the
logical claim in an incorrect direction.
\end{remark}

\begin{remark}[Common inference rule errors]\label{rem:inf-errors}
\begin{itemize}
  \item Choosing a witness before establishing an existential premise.
  \item Using a witness constant introduced by EI outside its allowed scope.
  \item Generalizing universally over a variable that depends on a special
  assumption.
  \item Substituting inside the scope of a quantifier without checking for
  variable capture.
  \item Replacing a term by an equal term in one occurrence but not another.
\end{itemize}
\end{remark}

\begin{remark}
A useful informal check: if a proof step would still make sense after replacing
$\forall$ with ``for most'' or $\exists$ with ``maybe'', the step is almost
certainly invalid.
\end{remark}

% =========================================================
% Summary Tables
% =========================================================

\subsection{Reference: Summary Tables}

\subsubsection*{Comparison: Propositional vs.\ Predicate Logic}

\begin{center}
\renewcommand{\arraystretch}{1.3}
\begin{tabular}{|p{4cm}|p{5cm}|p{5cm}|}
\hline
\textbf{Aspect} & \textbf{Propositional Logic} & \textbf{Predicate Logic} \\ \hline
Atomic formulas & Propositional variables (e.g.\ $P$, $Q$) & Predicate symbols applied to terms (e.g.\ $P(x)$, $R(x,y)$) \\ \hline
Internal structure & No internal structure & Terms, variables, constants, functions \\ \hline
Dependence on variables & None & May contain free variables \\ \hline
Semantic interpretation & Assigned a truth value directly & True or false relative to a structure and assignment \\ \hline
Use of quantifiers & Not available & Essential component \\ \hline
Semantic evaluation & Truth tables & Satisfaction in a structure \\ \hline
\end{tabular}
\end{center}

\vspace{1em}

\subsubsection*{Quantifier Rules Summary}

\begin{center}
\renewcommand{\arraystretch}{1.35}
\begin{tabular}{|p{4cm}|p{6cm}|p{5cm}|}
\hline
\textbf{Rule Name} & \textbf{Schema} & \textbf{Conditions / Notes} \\ \hline
Universal Instantiation & $\forall x\,\varphi \Rightarrow \varphi[t/x]$ & $t$ must be free for $x$ in $\varphi$ \\ \hline
Existential Generalization & $\varphi[t/x] \Rightarrow \exists x\,\varphi$ & $t$ may be any term \\ \hline
Existential Instantiation & $\exists x\,\varphi \Rightarrow \varphi[c/x]$ & $c$ is a new constant symbol \\ \hline
Universal Generalization & $\varphi \Rightarrow \forall x\,\varphi$ & $x$ not free in any undischarged assumption \\ \hline
Quantifier Negation & $\neg\forall x\,\varphi \equiv \exists x\,\neg\varphi$ & — \\ \cline{2-3}
                    & $\neg\exists x\,\varphi \equiv \forall x\,\neg\varphi$ & — \\ \hline
Quantifier Commutation & $\forall x\,\forall y\,\varphi \equiv \forall y\,\forall x\,\varphi$ & Same quantifier type only \\ \cline{2-3}
                       & $\exists x\,\exists y\,\varphi \equiv \exists y\,\exists x\,\varphi$ & — \\ \hline
Vacuous Quantification & $\forall x\,\varphi \equiv \varphi$ & $x$ not free in $\varphi$ \\ \cline{2-3}
                       & $\exists x\,\varphi \equiv \varphi$ & — \\ \hline
Renaming Bound Variables & $\forall x\,\varphi \equiv \forall y\,\varphi[y/x]$ & $y$ not free in $\varphi$ \\ \hline
\end{tabular}
\end{center}

\vspace{1em}

\subsubsection*{Syntax to Semantics Overview}

\begin{center}
\resizebox{\linewidth}{!}{%
\begin{tikzpicture}[
  node distance=3.2cm,
  every node/.style={draw, rectangle, rounded corners, align=center, inner sep=8pt},
  arrow/.style={->, thick}
]

\node (var) {\textbf{Variables}\\ $x_1,\dots,x_n$};
\node (term) [right=of var] {\textbf{Terms}\\ $t_1,\dots,t_n$};
\node (atom) [right=of term] {\textbf{Atomic Formula}\\ $P(t_1,\dots,t_n)$};
\node (form) [right=of atom] {\textbf{Formula}\\ $\forall x\,\varphi,\;\exists x\,\varphi$};
\node (rel)  [right=of form] {\textbf{Truth in Structure}\\ $\mathcal{M}, s \models \varphi$};

\draw[arrow] (var)  -- node[above]{\small substitution} (term);
\draw[arrow] (term) -- node[above]{\small formation} (atom);
\draw[arrow] (atom) -- node[above]{\small logical construction} (form);
\draw[arrow] (form) -- node[above]{\small satisfaction} (rel);

\end{tikzpicture}
}
\end{center}

\begin{remark}
Variables and terms belong purely to syntax. Predicates form atomic formulas.
Logical connectives and quantifiers build complex formulas. A structure and
variable assignment then determine whether a formula is satisfied, with
quantifiers ranging over the domain of discourse.
\end{remark}



