% =========================================================
% Type Theory — Planned
% =========================================================
\chapter{Type Theory}

\begin{tcolorbox}[
  colback=gray!6,
  colframe=gray!40,
  arc=2pt,
  left=8pt, right=8pt, top=6pt, bottom=6pt,
  title={\small\textbf{Where You Are in the Journey}},
  fonttitle=\small\bfseries
]
\begin{center}
\small
Propositional Logic
$\;\to\;$ Predicate Calculus
$\;\to\;$ Sets \& Functions
$\;\to\;$ Proof Techniques
$\;\to\;$ Model Theory
$\;\to\;$ \textbf{Type Theory}
$\;\to\;$ $\cdots$
\end{center}

\medskip
\noindent\textbf{How we got here.}
Set theory provides one foundation for mathematics.
Type theory provides an alternative, assigning every expression a
\emph{type} to avoid paradoxes and to make proof-checking mechanical.

\medskip
\noindent\textbf{What this chapter will build.}
Simply typed lambda calculus, the Curry--Howard correspondence
(proofs as programs), and dependent type theory as a foundation
for mechanised proof assistants (Lean, Coq).

\medskip
\noindent\textbf{Where this leads.}
Modern proof assistants (Lean, Coq, Agda) are built on type theory.
Understanding type theory illuminates what it means to formalise
mathematics completely.
\end{tcolorbox}

\vspace{1em}
\begin{tcolorbox}[
  colback=gray!6, colframe=gray!40, arc=2pt,
  left=8pt, right=8pt, top=6pt, bottom=6pt,
  title={\small\textbf{Status: Planned}},
  fonttitle=\small\bfseries
]
\begin{center}\Large\bfseries Coming Soon\end{center}
\vspace{6pt}
\noindent Notes, proofs, and exercises will appear here in a future revision.
\end{tcolorbox}

\section{Notes}
\textit{To be populated.}

\section{Proofs}
\textit{To be populated.}

\section{Capstone}
\textit{To be populated.}
