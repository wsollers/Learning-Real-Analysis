% =========================================================
% von Neumann Numerals (Set-Theoretic Construction of \mathbb{N})
% File: volume-i/axiomsystems/notes/von-neuman/von_neuman-numerals.tex
% =========================================================

\subsection{von Neumann Numerals}
\label{sec:von-neumann-numerals}

% ---------------------------------------------------------
\subsubsection{Definitions and Theorems}
% ---------------------------------------------------------

\begin{remark}[Purpose]
This section isolates the set-theoretic toolkit sufficient to define the
\emph{von Neumann numerals}
\[
0 := \varnothing,\qquad n^{+} := n\cup\{n\},
\]
and to prove the key structural facts:
\[
m<n\ \Longleftrightarrow\ m\in n,
\]
and each numeral is transitive and well-ordered by $\in$.
\end{remark}

% ---------------------------------------------------------
% Core Set-Theoretic Axioms Used
% ---------------------------------------------------------

\begin{axiom}[Extensionality]
\label{ax:vn-extensionality}
For all sets $A,B$,
\[
\bigl(\forall x\,(x\in A \leftrightarrow x\in B)\bigr)\ \Rightarrow\ A=B.
\]
\end{axiom}

\begin{axiom}[Empty Set]
\label{ax:vn-empty-set}
There exists a set with no elements, denoted $\varnothing$.
\end{axiom}

\begin{axiom}[Pairing]
\label{ax:vn-pairing}
For any sets $a,b$ there exists $\{a,b\}$.
\end{axiom}

\begin{definition}[Singleton]
\label{def:vn-singleton}
For any set $a$, define $\{a\}:=\{a,a\}$ (exists by Pairing).
\end{definition}

\begin{axiom}[Union]
\label{ax:union}
For any set $A$ there exists a set $\bigcup A$ such that
\[
x\in \bigcup A\ \leftrightarrow\ \exists y\in A\ (x\in y).
\]
\end{axiom}

\begin{definition}[Binary union]
\label{def:binary-union}
For sets $a,b$, define
\[
a\cup b := \bigcup\{a,b\}.
\]
\end{definition}

\begin{proposition}[Basic properties of union]
\label{prop:union-basics}
For all sets $a,b$ and all $x$,
\[
x\in a\cup b\ \leftrightarrow\ (x\in a \lor x\in b).
\]
\end{proposition}



% ---------------------------------------------------------
% von Neumann Numerals
% ---------------------------------------------------------

\begin{definition}[von Neumann successor]
\label{def:vn-successor}
For any set $n$, define its \emph{von Neumann successor} by
\[
S_V(n) := n \cup \{n\}.
\]
\end{definition}

\begin{definition}[von Neumann numerals]
\label{def:vn-numerals}
Define recursively:
\[
0_V := \varnothing,\qquad (n+1)_V := S_V(n_V)=n_V\cup\{n_V\}.
\]
The sets $n_V$ are called the \emph{von Neumann numerals}.
\end{definition}

\begin{proposition}[First few von Neumann numerals]
\label{prop:vn-first-few}
\[
0_V=\varnothing,\quad
1_V=\{0_V\}=\{\varnothing\},\quad
2_V=\{0_V,1_V\},\quad
3_V=\{0_V,1_V,2_V\},\ \dots
\]
More generally, for $n\ge 1$,
\[
n_V=\{0_V,1_V,\dots,(n-1)_V\}.
\]
\end{proposition}



\begin{definition}[Transitive set]
\label{def:transitive}
A set $A$ is \emph{transitive} if
\[
\forall x\,(x\in A \Rightarrow x\subseteq A).
\]
Equivalently, $\forall x\,\forall y\,((y\in x\in A)\Rightarrow y\in A)$.
\end{definition}

\begin{proposition}[Each von Neumann numeral is transitive]
\label{prop:vn-transitive}
For every $n$, the set $n_V$ is transitive.
\end{proposition}



\begin{proposition}[Successor is injective]
\label{prop:vn-successor-injective}
For all sets $a,b$,
\[
S_V(a)=S_V(b)\ \Rightarrow\ a=b.
\]
\end{proposition}



\begin{remark}[About Foundation]
Proposition~\ref{prop:vn-successor-injective} is usually proved inside ZFC using the
Axiom of Foundation (Regularity) to exclude membership cycles.
If you are working in a foundation-free setting, injectivity of successor may require an extra assumption.
\end{remark}

\begin{proposition}[Order by membership]
\label{prop:vn-order-by-membership}
For natural numbers $m,n$,
\[
m<n \iff m_V \in n_V.
\]
\end{proposition}



\begin{proposition}[Well-ordering by $\in$]
\label{prop:vn-well-ordered}
The relation $\in$ well-orders the set $\mathbb{N}_V := \{n_V : n\in\mathbb{N}\}$,
and this well-order is isomorphic to $(\mathbb{N},<)$ via $n\mapsto n_V$.
\end{proposition}



% ---------------------------------------------------------
\subsubsection{Consequences}
% ---------------------------------------------------------

\begin{corollary}[Each numeral is the set of its predecessors]
\label{cor:vn-predecessors}
For every $n$,
\[
n_V = \{m_V : m<n\}.
\]
\end{corollary}



\begin{corollary}[Monotonicity]
\label{cor:vn-monotone}
If $m<n$, then $m_V \subsetneq n_V$.
\end{corollary}


