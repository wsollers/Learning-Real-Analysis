% =========================================================
% Strong Induction and Backwards Induction (Tao §2.2)
% =========================================================

% ---------------------------------------------------------
% P2.2.14 — Strong Induction
% ---------------------------------------------------------
\begin{tcolorbox}[colback=propbox, colframe=propborder, arc=2pt,
  left=6pt, right=6pt, top=4pt, bottom=4pt,
  title={\small\textbf{Proposition 2.2.14 (Strong Principle of Induction)}},
  fonttitle=\small\bfseries]
Let $m_0 \in \mathbb{N}$, and let $P(m)$ be a property of natural numbers.
Suppose that for each $m \geq m_0$:
\[
\bigl[\forall m_0 \leq m' < m,\; P(m') \text{ is true}\bigr]
\;\Longrightarrow\; P(m) \text{ is true.}
\]
Then $P(m)$ is true for all $m \geq m_0$.
\end{tcolorbox}

\begin{remark}[What is different from ordinary induction]
In ordinary induction (P5), the inductive step assumes $P(n)$ and proves
$P(n\pp)$ --- you get to use exactly one prior case.
In strong induction, the inductive step assumes $P(m')$ for
\emph{all} $m' < m$ and proves $P(m)$ --- you get to use
\emph{every} prior case at once. This is essential when the proof of
$P(m)$ requires not just $P(m-1)$ but some earlier case $P(k)$ for
$k \ll m$, as frequently occurs in number theory, recursion, and
later real analysis arguments.
\end{remark}

\begin{remark}[The $Q$-trick: how the proof works]
The standard proof reduces strong induction to ordinary induction
by a single definitional move. Define:
\[
Q(n) \;:\Longleftrightarrow\; \forall m,\; m_0 \leq m < n \Rightarrow P(m).
\]
Then $Q(n)$ says: \emph{``$P$ holds for everything below $n$ (from $m_0$
onward).''} Crucially:
\begin{itemize}
  \item $Q(n)$ is \textbf{vacuously true} when $n < m_0$ or $n = m_0$,
    because the condition $m_0 \leq m < n$ has no solutions.
  \item The hypothesis of P2.2.14 says exactly: if $Q(m)$ holds then
    $P(m)$ holds (for $m \geq m_0$).
  \item This means $Q(n) \Rightarrow Q(n\pp)$ by ordinary induction,
    since $Q(n\pp)$ extends $Q(n)$ by one more case ($m = n$).
  \item By P5, $Q(n)$ holds for all $n \in \mathbb{N}$.
  \item Therefore $P(m)$ holds for all $m \geq m_0$.
\end{itemize}
The $Q$-trick is worth memorizing: \textbf{wrap the target property
in a universal quantifier over all prior values.}
\end{remark}

\begin{remark}[Base case is free]
Notice $P(m_0)$ follows immediately: the hypothesis of the implication
($P(m')$ true for all $m_0 \leq m' < m_0$) is vacuously satisfied,
so $P(m_0)$ holds. You do not need a separate base case argument ---
it is absorbed into the inductive step at $m = m_0$.
\end{remark}

\begin{remark}[Common values of $m_0$]
In practice $m_0 = 0$ or $m_0 = 1$. The general $m_0$ matters
for statements like ``for all $n \geq 2$, $n$ has a prime factorization''
where the base case $n = 0$ or $n = 1$ would be vacuous or trivial.
\end{remark}

% ---------------------------------------------------------
% Backwards induction — Ex 2.2.6
% ---------------------------------------------------------

\begin{tcolorbox}[colback=propbox, colframe=propborder, arc=2pt,
  left=6pt, right=6pt, top=4pt, bottom=4pt,
  title={\small\textbf{Exercise 2.2.6 (Backwards Induction)}},
  fonttitle=\small\bfseries]
Let $n \in \mathbb{N}$, and let $P(m)$ be a property of natural numbers
such that:
\begin{enumerate}
  \item[(i)] $P(n)$ is true.
  \item[(ii)] Whenever $P(m\pp)$ is true, $P(m)$ is also true.
\end{enumerate}
Then $P(m)$ is true for all $m \leq n$.
\end{tcolorbox}

\begin{remark}[Direction of travel]
Ordinary induction climbs \emph{up}: $P(0) \to P(1) \to P(2) \to \cdots$
Backwards induction descends \emph{down}: $P(n) \to P(n-1) \to \cdots \to P(0)$.
The descent is finite (stops at $0$), which is why it works.
An infinite descent in the other direction would not terminate.
\end{remark}

\begin{remark}[Proof strategy: induct on $n$, not on $m$]
The clever move in the proof is to induct on the \emph{ceiling} $n$,
not the property variable $m$.
Fix $P$ satisfying condition (ii). Define:
\[
R(n) \;:\Longleftrightarrow\;
\bigl[\,P(n) \Rightarrow \forall m \leq n,\; P(m)\,\bigr].
\]
Show $R(n)$ holds for all $n$ by ordinary induction. Then whenever
$P(n)$ is given, $R(n)$ delivers $P(m)$ for all $m \leq n$.
\end{remark}

\begin{remark}[When backwards induction appears]
Backwards induction arises in finite combinatorics and optimization:
to show a property holds at every stage of a finite process, establish
it at the end and propagate backward. It also appears in game theory
(backward induction in finite games) and in proofs that finite decreasing
sequences must terminate.
\end{remark}
