% =========================================================
% Zermelo Numerals (Set-Theoretic Construction of \mathbb{N})
% File: volume-i/axiomsystems/notes/zermelo/zermelo-numerals.tex
% =========================================================

\subsection{Zermelo Numerals}
\label{sec:zermelo-numerals}

% ---------------------------------------------------------
\subsubsection{Definitions and Theorems}
% ---------------------------------------------------------

\begin{remark}[Purpose]
This section records a minimal set-theoretic toolkit (axioms, derived constructions,
and lemmas) sufficient to define the \emph{Zermelo numerals}
\[
0 := \varnothing,\qquad n^{+} := \{n\}\quad (n \in \mathbb{N}).
\]
The goal is not to redevelop all of ZFC, but to isolate the ingredients used in this construction.
\end{remark}

% ---------------------------------------------------------
% Core Set-Theoretic Axioms Used
% ---------------------------------------------------------

\begin{axiom}[Extensionality]
\label{ax:zf-extensionality}
For all sets $A,B$,
\[
\bigl(\forall x\,(x\in A \leftrightarrow x\in B)\bigr)\ \Rightarrow\ A=B.
\]
\end{axiom}

\begin{axiom}[Empty Set]
\label{ax:zf-empty-set}
There exists a set with no elements. We denote one such set by $\varnothing$:
\[
\exists E\ \forall x\ (x\notin E).
\]
\end{axiom}

\begin{axiom}[Pairing]
\label{ax:zf-pairing}
For any sets $a,b$ there exists a set $\{a,b\}$ whose elements are exactly $a$ and $b$:
\[
\forall a\,\forall b\,\exists P\ \forall x\ (x\in P \leftrightarrow (x=a \lor x=b)).
\]
\end{axiom}

\begin{definition}[Singleton]
\label{def:singleton}
For any set $a$, define the \emph{singleton} of $a$ by
\[
\{a\} := \{a,a\},
\]
whose existence follows from the Pairing Axiom (Axiom~\ref{ax:zf-pairing}).
\end{definition}

\begin{proposition}[\texorpdfstring{\hyperref[prf:empty-unique]{Uniqueness of the empty set}}{Uniqueness of the empty set}]
\label{prop:empty-unique}
If $E$ and $E'$ have no elements, then $E=E'$.
\end{proposition}



\begin{proposition}[\texorpdfstring{\hyperref[prf:singleton-well-defined]{Singleton is well-defined}}{Singleton is well-defined}]
\label{prop:singleton-well-defined}
For each set $a$ there exists a unique set whose only element is $a$, denoted $\{a\}$.
\end{proposition}


% ---------------------------------------------------------
% Zermelo Numerals
% ---------------------------------------------------------

\begin{definition}[Zermelo successor]
\label{def:zermelo-successor}
For any set $n$, define its \emph{Zermelo successor} by
\[
S_Z(n) := \{n\}.
\]
\end{definition}

\begin{definition}[Zermelo numerals]
\label{def:zermelo-numerals}
Define recursively:
\[
0_Z := \varnothing,\qquad (n+1)_Z := S_Z(n_Z)=\{n_Z\}.
\]
The sets $n_Z$ are called the \emph{Zermelo numerals}.
\end{definition}

\begin{proposition}[\texorpdfstring{\hyperref[prf:zermelo-first-few]{First few Zermelo numerals}}{First few Zermelo numerals}]
\label{prop:zermelo-first-few}
\[
0_Z=\varnothing,\quad
1_Z=\{\varnothing\},\quad
2_Z=\{\{\varnothing\}\},\quad
3_Z=\{\{\{\varnothing\}\}\},\ \dots
\]
\end{proposition}



\begin{proposition}[\texorpdfstring{\hyperref[prf:zermelo-successor-injective]{Injectivity of Zermelo successor}}{Injectivity of Zermelo successor}]
\label{prop:zermelo-successor-injective}
For all sets $a,b$,
\[
S_Z(a)=S_Z(b)\ \Rightarrow\ a=b.
\]
\end{proposition}



\begin{proposition}[\texorpdfstring{\hyperref[prf:zermelo-distinct]{Distinctness of Zermelo numerals}}{Distinctness of Zermelo numerals}]
\label{prop:zermelo-distinct}
If $m\neq n$ as natural numbers, then $m_Z \neq n_Z$ as sets.
Equivalently, the map $n\mapsto n_Z$ is injective.
\end{proposition}



\begin{proposition}[Membership pattern]
\label{prop:zermelo-membership}
For Zermelo numerals, we have:
\[
m_Z \in n_Z \iff n = m+1.
\]
In particular, each $n_Z$ has exactly one element when $n\ge 1$.
\end{proposition}



\begin{remark}[Why Zermelo numerals are useful]
Zermelo numerals are the \emph{iterated singleton tower}.
They are convenient when you want a very rigid notion of ``$n$ steps'':
each numeral (except $0$) has exactly one element.
They are less convenient for encoding order by membership (that role is played better by von Neumann numerals).
\end{remark}

% ---------------------------------------------------------
\subsubsection{Consequences}
% ---------------------------------------------------------

\begin{corollary}[\texorpdfstring{\hyperref[prf:zermelo-no-self-membership]{No numeral contains itself}}{No numeral contains itself}]
\label{cor:zermelo-no-self-membership}
For every $n$, $n_Z \notin n_Z$.
\end{corollary}



\begin{corollary}[\texorpdfstring{\hyperref[prf:zermelo-chain]{Strict chain under membership}}{Strict chain under membership}]
\label{cor:zermelo-chain}
For $n\ge 1$,
\[
(n-1)_Z \in n_Z,\quad (n-2)_Z \in (n-1)_Z,\ \dots,\ 0_Z \in 1_Z.
\]
\end{corollary}


