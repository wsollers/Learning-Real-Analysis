% =========================================================
% Numerals and Basic Definitions (Tao §2.1)
% =========================================================

\begin{definition}[Standard numerals]
\label{def:numerals}
We define:
\[
1 := 0\pp, \quad
2 := 1\pp = (0\pp)\pp, \quad
3 := 2\pp, \quad \text{etc.}
\]
In general, each standard numeral is the successor of the previous one.
\end{definition}

\begin{proposition}[Tao 2.1.4]
$3$ is a natural number.
\end{proposition}
\begin{proof}
$0 \in \mathbb{N}$ by P1. Then $1 = 0\pp \in \mathbb{N}$ by P2.
Then $2 = 1\pp \in \mathbb{N}$ by P2. Then $3 = 2\pp \in \mathbb{N}$ by P2.
\end{proof}

\begin{proposition}[Tao 2.1.6 — $4 \neq 0$]
$4$ is not equal to $0$.
\end{proposition}
\begin{proof}
$4 = 3\pp$. Since $3 \in \mathbb{N}$, by P3 we have $3\pp \neq 0$,
i.e., $4 \neq 0$.
\end{proof}

\begin{remark}[Why prove $4 \neq 0$?]
Without P3, a wrap-around system where $3\pp = 0$ (like a 4-clock)
satisfies P1 and P2. P3 rules this out, and the proof above shows
P3 doing its work explicitly.
\end{remark}

\begin{proposition}[Tao 2.1.8 — $6 \neq 2$]
$6$ is not equal to $2$.
\end{proposition}
\begin{proof}
Suppose $6 = 2$. Then $5\pp = 1\pp$, so by P4, $5 = 1$.
Then $4\pp = 0\pp$, so by P4, $4 = 0$, contradicting $4 \neq 0$.
\end{proof}

\begin{remark}[Pattern]
Distinctness proofs always proceed by descending through P4 until
reaching $4 \neq 0$ (established via P3). P3 provides the
``ground'' that stops the descent.
\end{remark}
