% =========================================================
% Numerals and Basic Definitions (Tao §2.1)
% =========================================================

\begin{definition}[Standard numerals]
\label{def:numerals}
We define:
\[
1 := 0\pp, \quad
2 := 1\pp = (0\pp)\pp, \quad
3 := 2\pp, \quad \text{etc.}
\]
In general, each standard numeral is the successor of the previous one.
\end{definition}

\begin{proposition}[Tao 2.1.4]
\label{prop:tao-2-1-4}
$3$ is a natural number.
\end{proposition}

\begin{remark}[Proof]
See \hyperref[prf:tao-2-1-4]{Proof $\to$ Tao 2.1.4 --- $3$ is a natural number}.
\end{remark}


\begin{proposition}[Tao 2.1.6 — $4 \neq 0$]
\label{prop:tao-2-1-6}
$4$ is not equal to $0$.
\end{proposition}

\begin{remark}[Proof]
See \hyperref[prf:tao-2-1-6]{Proof $\to$ Tao 2.1.6 --- $4 \neq 0$}.
\end{remark}

\begin{remark}[Why prove $4 \neq 0$?]
Without P3, a wrap-around system where $3\pp = 0$ (like a 4-clock)
satisfies P1 and P2. P3 rules this out, and the proof above shows
P3 doing its work explicitly.
\end{remark}

\begin{proposition}[Tao 2.1.8 — $6 \neq 2$]
\label{prop:tao-2-1-8}
$6$ is not equal to $2$.
\end{proposition}

\begin{remark}[Proof]
See \hyperref[prf:tao-2-1-8]{Proof $\to$ Tao 2.1.8 --- $6 \neq 2$}.
\end{remark}

\begin{remark}[Pattern]
Distinctness proofs always proceed by descending through P4 until
reaching $4 \neq 0$ (established via P3). P3 provides the
``ground'' that stops the descent.
\end{remark}
