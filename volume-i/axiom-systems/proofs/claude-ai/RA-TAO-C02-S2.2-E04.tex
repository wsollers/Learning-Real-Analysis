% --------------------------------------
% Proof: RA-TAO-C02-S2.2-E04
% --------------------------------------
\clearpage
\phantomsection
\hypertarget{proof-RA-TAO-C02-S2.2-E04}{}

\subsubsection[Trichotomy Gaps (P2.2.13)]{Proof --- RA-TAO-C02-S2.2-E04}

\bigskip

\noindent
\textbf{Source.}
\srccite{TaoAnalysis1}{Chapter~2, \S2.2, Exercise~2.2.4}.

\vspace{0.75em}

\noindent
\textbf{Goal.}
Justify each of the three statements marked \emph{(why?)} in Tao's
sketch proof of Proposition~2.2.13 (Trichotomy of order for natural
numbers). The three gaps are:

\medskip
\begin{enumerate}
  \item[\textbf{(i)}] \textbf{Base case} ($a = 0$, $b$ fixed).
    Tao claims: $0 \leq b$ for all $b \in \mathbb{N}$. \emph{(why?)}

  \item[\textbf{(ii)}] \textbf{Inductive step, case $a > b$.}
    Tao claims: if $a > b$, then $a\pp > b$. \emph{(why?)}

  \item[\textbf{(iii)}] \textbf{Inductive step, case $a = b$.}
    Tao claims: if $a = b$, then $a\pp > b$. \emph{(why?)}
\end{enumerate}

\medskip
\noindent
For each item give a complete justification using only
Definition~2.2.11, Proposition~2.2.12, and the Peano axioms.
Do not appeal to trichotomy itself.

\vspace{0.75em}

\noindent\textbf{Available toolkit.}
P1--P5, Definition~2.2.11 (order), Lemma~2.2.2, Lemma~2.2.3,
Proposition~2.2.4 (commutativity), Proposition~2.2.5 (associativity),
Proposition~2.2.6 (cancellation), Proposition~2.2.12 (order properties).

\vspace{1em}

% =========================================================
% Part (i)
% =========================================================

\vspace{1em}
\noindent\rule{\textwidth}{0.4pt}
\noindent\textbf{Part (i).} Show $0 \leq b$ for all $b \in \mathbb{N}$.

\begin{proof}[Proof of Part (i)]
Let $b \in \mathbb{N}$ be arbitrary.
We must show $0 \leq b$, i.e.\ by Definition~2.2.11, that there exists
$a \in \mathbb{N}$ such that $b = 0 + a$.
Set $a := b$.

\noindent
\begin{tabular}{T S J}
\toprule
\textbf{Tag} & \multicolumn{1}{p{0.44\textwidth}}{\textbf{Step}} & \textbf{Justification} \\
\midrule
\addlinespace[4pt]

\tagUW
&
\text{Let } b \in \mathbb{N} \text{ be arbitrary.}
&
Introduce universal witness for the claim $\forall b \in \mathbb{N},\; 0 \leq b$.
\\[10pt]

\tagEX
&
\text{Set } a := b.
&
Candidate witness for the existential in Def~2.2.11:
we need $b = 0 + a$ for some $a \in \mathbb{N}$.
\\[10pt]

\tagDU
&
0 + b = b
&
Definition of addition, base clause (A1): $0 + m := m$.
\\[10pt]

\tagDU
&
b = 0 + a
&
Substituting $a := b$ gives $0 + a = 0 + b = b$.
Hence the witness satisfies the definition.
\\[10pt]

\tagDU
&
\therefore\; 0 \leq b
&
By Definition~2.2.11 ($n \geq m$ iff $\exists a \in \mathbb{N},\; n = m + a$),
with $n = b$, $m = 0$, $a = b$.
\hfill$\blacksquare$
\\[6pt]

\bottomrule
\end{tabular}
\end{proof}

\begin{remark}[Proof shape: universal-existential]
The claim $\forall b \in \mathbb{N},\; 0 \leq b$ has a universal-existential
structure. The outer universal quantifier is discharged by introducing an
arbitrary witness (UW); the inner existential in Definition~2.2.11
is satisfied by constructing the witness $a := b$ (EX) and verifying it
via the base clause of addition (A1). No induction is needed.
\end{remark}

% =========================================================
% Part (ii)
% =========================================================

\vspace{1em}
\noindent\rule{\textwidth}{0.4pt}
\noindent\textbf{Part (ii).} Assume $a > b$. Show $a\pp > b$.

\begin{proof}[Proof of Part (ii)]
Let $a, b \in \mathbb{N}$ be arbitrary. Assume $a > b$.
By Definition~2.2.11, $a > b$ means $a \geq b$ and $a \neq b$,
i.e.\ there exists $c \in \mathbb{N}$ such that $a = b + c$ and $a \neq b$.
We must show $a\pp > b$, i.e.\ $\exists c' \in \mathbb{N},\; a\pp = b + c'$
and $a\pp \neq b$.
Set $c' := c + 1$.

\noindent
\begin{tabular}{T S J}
\toprule
\textbf{Tag} & \multicolumn{1}{p{0.44\textwidth}}{\textbf{Step}} & \textbf{Justification} \\
\midrule
\addlinespace[4pt]

\tagUW
&
\text{Let } a, b \in \mathbb{N} \text{ be arbitrary.}
&
Introduce universal witnesses.
\\[10pt]

\tagDU
&
\text{Assume } a > b \text{, i.e. }
\exists\, c \in \mathbb{N},\; a = b + c \;\wedge\; a \neq b.
&
Definition~2.2.11: $n > m \Leftrightarrow n \geq m \wedge n \neq m$,
and $n \geq m \Leftrightarrow \exists a \in \mathbb{N},\; n = m + a$.
\\[10pt]

\tagEX
&
\text{Set } c' := c + 1.
&
Candidate witness for $a\pp = b + c'$.
\\[10pt]

\tagAM
&
a\pp = a + 1 = (b + c) + 1 = b + (c + 1) = b + c'
&
$a\pp = a + 1$ by the corollary $n\pp = n+1$;
substitute $a = b + c$; associativity (P~2.2.5).
\\[10pt]

\tagTA
&
a\pp \neq b
&
Suppose $a\pp = b$. Then $b + c + 1 = b$, so by cancellation (P~2.2.6),
$c + 1 = 0$. But $c + 1 = c\pp$ and $c\pp \neq 0$ by P3. Contradiction.
\\[10pt]

\tagDU
&
\therefore\; a\pp > b
&
We have $a\pp = b + c'$ with $c' \in \mathbb{N}$ and $a\pp \neq b$,
so $a\pp > b$ by Definition~2.2.11.
\hfill$\blacksquare$
\\[6pt]

\bottomrule
\end{tabular}
\end{proof}

\begin{remark}[Proof shape: universal conditional, direct]
The claim $\forall a,b \in \mathbb{N},\; a > b \Rightarrow a\pp > b$
is a universal conditional. Introduce arbitrary witnesses (UW),
assume the hypothesis (P), unpack it via Definition~2.2.11 (DU),
construct a witness for the conclusion (EX), verify both conjuncts,
and close by Definition~2.2.11 (DU). The key move is $c' := c + 1$:
incrementing the existing gap witness to account for the successor.
\end{remark}

% =========================================================
% Part (iii)
% =========================================================

\vspace{1em}
\noindent\rule{\textwidth}{0.4pt}
\noindent\textbf{Part (iii).} Assume $a = b$. Show $a\pp > b$.

\begin{proof}[Proof of Part (iii)]
Let $a, b \in \mathbb{N}$ be arbitrary. Assume $a = b$.
We must show $a\pp > b$, i.e.\ $\exists c \in \mathbb{N},\; a\pp = b + c$
and $a\pp \neq b$.
Set $c := 1$.

\noindent
\begin{tabular}{T S J}
\toprule
\textbf{Tag} & \multicolumn{1}{p{0.44\textwidth}}{\textbf{Step}} & \textbf{Justification} \\
\midrule
\addlinespace[4pt]

\tagUW
&
\text{Let } a, b \in \mathbb{N} \text{ be arbitrary.}
&
Introduce universal witnesses.
\\[10pt]

\tagDU
&
\text{Assume } a = b.
&
Hypothesis (P of the conditional).
\\[10pt]

\tagDU
&
\text{Goal: } \exists c \in \mathbb{N},\; a\pp = b + c \;\wedge\; a\pp \neq b.
&
Unpack $a\pp > b$ via Definition~2.2.11.
\\[10pt]

\tagEX
&
\text{Set } c := 1.
&
Candidate witness.
\\[10pt]

\tagAM
&
a\pp = a + 1 = b + 1 = b + c
&
$a\pp = a + 1$ by the corollary $n\pp = n + 1$;
substitute $a = b$ from hypothesis.
\\[10pt]

\tagTA
&
a\pp \neq b
&
Suppose $a\pp = b$. Then $b + 1 = b$, so by cancellation (P~2.2.6),
$1 = 0$. But $1 = 0\pp$ and $0\pp \neq 0$ by P3. Contradiction.
\\[10pt]

\tagDU
&
\therefore\; a\pp > b
&
We have $a\pp = b + c$ with $c = 1 \in \mathbb{N}$ and $a\pp \neq b$,
so $a\pp > b$ by Definition~2.2.11.
\hfill$\blacksquare$
\\[6pt]

\bottomrule
\end{tabular}
\end{proof}

\begin{remark}[Proof shape: universal conditional, direct]
The claim $\forall a,b \in \mathbb{N},\; a = b \Rightarrow a\pp > b$
is a universal conditional. The witness $c := 1$ is forced by the
structure of the goal: $a\pp = a + 1$ by the successor corollary,
and the hypothesis $a = b$ immediately gives $a + 1 = b + 1 = b + c$.
The non-equality $a\pp \neq b$ follows from P3 via cancellation ---
the same closing argument as Part~(ii).
\end{remark}

\begin{remark}[Common structure of Parts (ii) and (iii)]
Both parts share the same proof architecture: construct a witness
for the existential in Definition~2.2.11, verify the equality conjunct
by algebraic manipulation, and close the inequality conjunct by
deriving a contradiction with P3 via cancellation. The only difference
is the witness: $c' = c + 1$ in Part~(ii) (inherited from the hypothesis),
and $c = 1$ in Part~(iii) (forced directly by the successor corollary).
\end{remark}

\vspace{1em}