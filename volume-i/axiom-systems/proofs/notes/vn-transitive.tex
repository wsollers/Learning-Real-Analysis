% =========================================================
% Proof: Each von Neumann numeral is transitive
% Source: volume-i/axiom-systems/notes/von-neuman/von_neuman-numerals.tex
% =========================================================

\subsection*{Each von Neumann Numeral is Transitive}
\label{prf:vn-transitive}

\begin{remark}[Return]
\hyperref[prop:vn-transitive]{$\leftarrow$ Back to Proposition (Each von Neumann numeral is transitive) in Notes}
\end{remark}

\begin{proof}
\Claim For every $n$, the set $n_V$ is transitive: if $x \in n_V$ then $x \subseteq n_V$.

\Given The von Neumann numerals: $0_V = \varnothing$ and $(n+1)_V = n_V \cup
\{n_V\}$ (\hyperref[def:vn-numerals]{Definition~(von Neumann numerals)}).
The definition of transitivity (\hyperref[def:transitive-set]{Definition~(Transitive set)}):
$A$ is transitive iff $x \in A \Rightarrow x \subseteq A$.
Basic properties of union (\hyperref[prop:union-basics]{Basic properties of union}).

\Goal To show $n_V$ is transitive for every $n$.

\Strategy Induction on $n$.

\medskip
\noindent\textbf{Base case} ($n = 0$). $0_V = \varnothing$. There is no $x
\in \varnothing$, so the transitivity condition holds vacuously. \checkbox

\medskip
\noindent\textbf{Inductive step.} Suppose $n_V$ is transitive. We show
$(n+1)_V = n_V \cup \{n_V\}$ is transitive.

Let $x \in n_V \cup \{n_V\}$. By the union property, either $x \in n_V$ or $x
= n_V$.

\textit{Case 1: $x \in n_V$.} By the induction hypothesis, $x \subseteq n_V$.
Since $n_V \subseteq n_V \cup \{n_V\} = (n+1)_V$, we have $x \subseteq
(n+1)_V$.

\textit{Case 2: $x = n_V$.} We need $n_V \subseteq (n+1)_V = n_V \cup \{n_V\}$.
For any $y \in n_V$, $y \in n_V \cup \{n_V\}$ by the union property. So $n_V
\subseteq (n+1)_V$. \checkbox

In both cases, $x \subseteq (n+1)_V$. By induction, $n_V$ is transitive for
all $n$. \AsReq
\end{proof}

\begin{remark}[Proof shape]
Standard induction with a two-case split in the inductive step: elements of
$(n+1)_V = n_V \cup \{n_V\}$ are either elements of $n_V$ (handled by the IH)
or equal to $n_V$ itself (handled by the subset relation $n_V \subseteq
(n+1)_V$). Both cases reduce immediately to the definitions.
\end{remark}

\begin{remark}[Why transitivity matters]
Transitivity of von Neumann numerals is what makes the ordering $m_V \in n_V
\Leftrightarrow m < n$ work cleanly: if $k \in m_V$ and $m_V \in n_V$, then
by transitivity of $n_V$, $k \in n_V$. This transitivity at the set level
mirrors the transitivity of the usual ordering $<$ on natural numbers.
\end{remark}

\begin{remark}[Dependencies]
The proof depends on: the Definition of von Neumann numerals
(\hyperref[def:vn-numerals]{Definition~(von Neumann numerals)}), the Definition
of transitive set (\hyperref[def:transitive-set]{Definition~(Transitive set)}),
and \hyperref[prop:union-basics]{Basic properties of union}.
\end{remark}
