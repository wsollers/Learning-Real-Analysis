% =========================================================
% Proof: Basic properties of union
% Source: volume-i/axiom-systems/notes/von-neuman/von_neuman-numerals.tex
% =========================================================

\subsection*{Basic Properties of Union}
\label{prf:union-basics}

\begin{remark}[Return]
\hyperref[prop:union-basics]{$\leftarrow$ Back to Proposition (Basic properties of union) in Notes}
\end{remark}

\begin{proof}
\Claim For all sets $a, b$ and all $x$, $x \in a \cup b \;\leftrightarrow\; (x \in a \lor x \in b)$.

\Given The Axiom of Union (\hyperref[ax:zf-union]{Axiom~(Union)}): for any set
$A$ there exists $\bigcup A$ with $x \in \bigcup A \leftrightarrow \exists y \in
A\,(x \in y)$. The Definition of binary union
(\hyperref[def:binary-union]{Definition~(Binary union)}): $a \cup b := \bigcup
\{a, b\}$. The Axiom of Pairing (\hyperref[ax:vn-pairing]{Axiom~(Pairing)}):
$\{a, b\}$ exists with $z \in \{a, b\} \leftrightarrow (z = a \lor z = b)$.

\Goal To show $x \in a \cup b \leftrightarrow (x \in a \lor x \in b)$.

\Strategy Unfold the definition $a \cup b = \bigcup \{a, b\}$ and apply the
Union Axiom, then the characterisation of $\{a, b\}$.

\medskip
By the Union Axiom applied to $A = \{a, b\}$:
\[
x \in \bigcup\{a, b\}
\;\leftrightarrow\;
\exists y \in \{a, b\}\,(x \in y).
\]
By the Pairing Axiom, $y \in \{a, b\} \leftrightarrow (y = a \lor y = b)$.
Therefore:
\[
\exists y \in \{a, b\}\,(x \in y)
\;\leftrightarrow\;
(x \in a \lor x \in b).
\]
(The forward direction: if such $y$ exists and $y = a$, then $x \in a$; if $y =
b$, then $x \in b$. The backward direction: if $x \in a$, take $y = a \in
\{a,b\}$; if $x \in b$, take $y = b \in \{a,b\}$.)

Since $a \cup b = \bigcup\{a, b\}$ by definition, the chain of equivalences
gives $x \in a \cup b \leftrightarrow (x \in a \lor x \in b)$. \AsReq
\end{proof}

\begin{remark}[Proof shape]
The proof is a chain of equivalences: unfold the definition of $a \cup b$,
apply the Union Axiom, then unpack the membership condition for $\{a, b\}$ from
the Pairing Axiom. No case analysis is needed; both directions of the
biconditional follow from the same chain.
\end{remark}

\begin{remark}[Why this requires two axioms]
The binary union $a \cup b$ is defined in two steps: Pairing gives $\{a, b\}$,
then Union gives $\bigcup \{a, b\}$. Both axioms are genuinely needed: Pairing
constructs the ``indexing set'' $\{a, b\}$ over which Union ranges.
\end{remark}

\begin{remark}[Dependencies]
The proof depends on: the Axiom of Union (\hyperref[ax:zf-union]{Axiom~(Union)}),
the Axiom of Pairing (\hyperref[ax:vn-pairing]{Axiom~(Pairing)}), and the
Definition of binary union (\hyperref[def:binary-union]{Definition~(Binary union)}).
\end{remark}
