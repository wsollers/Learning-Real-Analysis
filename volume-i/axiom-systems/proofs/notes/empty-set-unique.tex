% =========================================================
% Proof: Uniqueness of the empty set
% Source: volume-i/axiom-systems/notes/zermelo/zermelo-numerals.tex
% =========================================================

\subsection*{Uniqueness of the Empty Set}
\label{prf:empty-unique}

\begin{remark}[Return]
\hyperref[prop:empty-unique]{$\leftarrow$ Back to Proposition (Uniqueness of the empty set) in Notes}
\end{remark}

\begin{proposition}[Uniqueness of the empty set]
If $E$ and $E'$ have no elements, then $E = E'$.
\end{proposition}

\begin{proof}
\Given The Axiom of Extensionality (\hyperref[ax:zf-extensionality]{Axiom~(Extensionality)}):
two sets are equal if and only if they have the same elements. The hypothesis
that $E$ and $E'$ are both empty: $\forall x\,(x \notin E)$ and $\forall x\,(x
\notin E')$.

\Goal To show $E = E'$.

\Strategy By Extensionality it suffices to show $\forall x\,(x \in E
\leftrightarrow x \in E')$. Since neither set has any elements, both sides of
the biconditional are false for every $x$, so the biconditional holds vacuously.

Let $x$ be arbitrary.

($\Rightarrow$) Suppose $x \in E$. This contradicts the hypothesis that $E$ has
no elements. The conclusion $x \in E'$ follows vacuously.

($\Leftarrow$) Suppose $x \in E'$. This contradicts the hypothesis that $E'$
has no elements. The conclusion $x \in E$ follows vacuously.

Since $x$ was arbitrary, $\forall x\,(x \in E \leftrightarrow x \in E')$.
By Extensionality, $E = E'$. \AsReq
\end{proof}

\begin{remark}[Proof shape]
This is a \emph{vacuous biconditional} argument. To prove $A = B$ via
Extensionality one must show membership is equivalent; when both sets are empty
both sides of each implication have false antecedents, so both conditionals hold
trivially. The logical form is: $(x \in E \rightarrow x \in E')$ holds because
$x \in E$ is false; $(x \in E' \rightarrow x \in E)$ holds because $x \in E'$
is false.
\end{remark}

\begin{remark}[Why Extensionality is the key axiom]
Without Extensionality, two empty sets might be considered distinct objects.
Extensionality is precisely the axiom that collapses all empty sets to one: sets
are individuated by their elements, and two sets with no elements have identical
element-profiles.
\end{remark}

\begin{remark}[Dependencies]
The proof depends on: the Axiom of Extensionality
(\hyperref[ax:zf-extensionality]{Axiom~(Extensionality)}) and the definition of
the empty set (\hyperref[ax:zf-empty-set]{Axiom~(Empty Set)}).
\end{remark}
