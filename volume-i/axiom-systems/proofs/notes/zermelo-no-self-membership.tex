% =========================================================
% Proof: No numeral contains itself
% Source: volume-i/axiom-systems/notes/zermelo/zermelo-numerals.tex
% =========================================================

\subsection*{No Numeral Contains Itself}
\label{prf:zermelo-no-self-membership}

\begin{remark}[Return]
\hyperref[cor:zermelo-no-self-membership]{$\leftarrow$ Back to Corollary (No numeral contains itself) in Notes}
\end{remark}

\begin{proof}
\Claim For every $n$, $n_Z \notin n_Z$.

\Given The Zermelo numerals: $0_Z = \varnothing$ and $(n+1)_Z = \{n_Z\}$
(\hyperref[def:zermelo-numerals]{Definition~(Zermelo numerals)}). Distinctness
of Zermelo numerals (\hyperref[prop:zermelo-distinct]{Distinctness of Zermelo
numerals}): $m \neq n \Rightarrow m_Z \neq n_Z$.

\Goal To show $n_Z \notin n_Z$ for all $n$.

\Strategy Induction on $n$.

\medskip
\noindent\textbf{Base case} ($n = 0$). $0_Z = \varnothing$ has no elements, so
$0_Z \notin 0_Z = \varnothing$ trivially. \checkbox

\medskip
\noindent\textbf{Inductive step.} Suppose $n_Z \notin n_Z$. We show $(n+1)_Z
\notin (n+1)_Z$.

We have $(n+1)_Z = \{n_Z\}$. The only element of $\{n_Z\}$ is $n_Z$. So
$(n+1)_Z \in (n+1)_Z$ would require $(n+1)_Z = n_Z$, i.e.\ $n_Z \in (n+1)_Z$
and thus $\{n_Z\} = n_Z$. But by distinctness, $(n+1)_Z \neq n_Z$ (since $n+1
\neq n$). Therefore $(n+1)_Z$ is not the unique element $n_Z$ of $(n+1)_Z$,
so $(n+1)_Z \notin (n+1)_Z$. \checkbox

By induction, $n_Z \notin n_Z$ for all $n$. \AsReq
\end{proof}

\begin{remark}[Proof shape]
The proof uses induction but the inductive hypothesis is not actually needed in
the step: the conclusion $(n+1)_Z \notin (n+1)_Z$ follows directly from the
distinctness of consecutive numerals ($(n+1)_Z \neq n_Z$) and the fact that
$(n+1)_Z = \{n_Z\}$ has $n_Z$ as its only element. The base case uses the
emptiness of $\varnothing$.
\end{remark}

\begin{remark}[Relation to Foundation]
In ZFC, the Axiom of Foundation (Regularity) directly prohibits $x \in x$ for
any set $x$. This proof establishes the same conclusion for Zermelo numerals
specifically without invoking Foundation, relying instead on the explicit
recursive structure and the distinctness result.
\end{remark}

\begin{remark}[Dependencies]
The proof depends on: the Definition of Zermelo numerals
(\hyperref[def:zermelo-numerals]{Definition~(Zermelo numerals)}) and
\hyperref[prop:zermelo-distinct]{Distinctness of Zermelo numerals}.
\end{remark}
