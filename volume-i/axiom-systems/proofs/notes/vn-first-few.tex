% =========================================================
% Proof: First few von Neumann numerals
% Source: volume-i/axiom-systems/notes/von-neuman/von_neuman-numerals.tex
% =========================================================

\subsection*{First Few von Neumann Numerals}
\label{prf:vn-first-few}

\begin{remark}[Return]
\hyperref[prop:vn-first-few]{$\leftarrow$ Back to Proposition (First few von Neumann numerals) in Notes}
\end{remark}

\begin{proof}
\Claim $0_V = \varnothing$, $1_V = \{\varnothing\}$, $2_V = \{\varnothing, \{\varnothing\}\}$, $3_V = \{\varnothing, \{\varnothing\}, \{\varnothing,\{\varnothing\}\}\}$, and in general $n_V = \{0_V, 1_V, \ldots, (n-1)_V\}$ for $n \geq 1$.

\Given The von Neumann numerals: $0_V := \varnothing$ and $(n+1)_V := n_V \cup
\{n_V\}$ (\hyperref[def:vn-numerals]{Definition~(von Neumann numerals)}).
Basic properties of union
(\hyperref[prop:union-basics]{Basic properties of union}): $x \in a \cup b
\leftrightarrow (x \in a \lor x \in b)$.

\Goal Compute $0_V, 1_V, 2_V, 3_V$ explicitly.

\Strategy Iterate the successor definition and apply the union characterisation.

\medskip
$0_V = \varnothing$ by definition.

$1_V = 0_V \cup \{0_V\} = \varnothing \cup \{\varnothing\}$.
An element $x \in \varnothing \cup \{\varnothing\}$ iff $x \in \varnothing$ or
$x \in \{\varnothing\}$, i.e.\ iff $x = \varnothing$. So $1_V = \{\varnothing\}
= \{0_V\}$.

$2_V = 1_V \cup \{1_V\} = \{\varnothing\} \cup \{\{\varnothing\}\}$.
An element $x$ belongs iff $x = \varnothing$ or $x = \{\varnothing\}$. So
$2_V = \{\varnothing, \{\varnothing\}\} = \{0_V, 1_V\}$.

$3_V = 2_V \cup \{2_V\} = \{0_V, 1_V\} \cup \{2_V\} = \{0_V, 1_V, 2_V\}$.

The pattern $n_V = \{0_V, 1_V, \ldots, (n-1)_V\}$ holds at each step: the
successor appends the current numeral to the existing set. \AsReq
\end{proof}

\begin{remark}[Proof shape]
Direct computation by iterated unfolding, exactly as for Zermelo numerals.
The key difference is that the von Neumann successor $n_V \cup \{n_V\}$
\emph{accumulates} all predecessors, whereas the Zermelo successor $\{n_Z\}$
discards all but the immediate predecessor.
\end{remark}

\begin{remark}[Cardinality encodes the number]
Each von Neumann numeral $n_V$ has exactly $n$ elements: $n_V =
\{0_V,\ldots,(n-1)_V\}$. This means $|n_V| = n$, so the cardinality of the
set directly encodes the number. This is one reason von Neumann numerals are
preferred in the standard development of ZFC.
\end{remark}

\begin{remark}[Dependencies]
The proof depends on: the Definition of von Neumann numerals
(\hyperref[def:vn-numerals]{Definition~(von Neumann numerals)}) and
\hyperref[prop:union-basics]{Basic properties of union}.
\end{remark}
