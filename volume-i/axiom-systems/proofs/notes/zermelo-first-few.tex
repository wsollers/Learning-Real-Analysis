% =========================================================
% Proof: First few Zermelo numerals
% Source: volume-i/axiom-systems/notes/zermelo/zermelo-numerals.tex
% =========================================================

\subsection*{First Few Zermelo Numerals}
\label{prf:zermelo-first-few}

\begin{remark}[Return]
\hyperref[prop:zermelo-first-few]{$\leftarrow$ Back to Proposition (First few Zermelo numerals) in Notes}
\end{remark}

\begin{proof}
\Claim $0_Z = \varnothing$, $1_Z = \{\varnothing\}$, $2_Z = \{\{\varnothing\}\}$, $3_Z = \{\{\{\varnothing\}\}\}$.

\Given The Definition of Zermelo numerals
(\hyperref[def:zermelo-numerals]{Definition~(Zermelo numerals)}):
$0_Z := \varnothing$ and $(n+1)_Z := S_Z(n_Z) = \{n_Z\}$. The Definition of
Zermelo successor (\hyperref[def:zermelo-successor]{Definition~(Zermelo
successor)}): $S_Z(n) := \{n\}$.

\Goal To compute $0_Z$, $1_Z$, $2_Z$, $3_Z$ explicitly.

\Strategy Iterate the successor definition starting from $0_Z = \varnothing$.

\medskip
$0_Z = \varnothing$ by definition.

$1_Z = S_Z(0_Z) = \{0_Z\} = \{\varnothing\}$.

$2_Z = S_Z(1_Z) = \{1_Z\} = \{\{\varnothing\}\}$.

$3_Z = S_Z(2_Z) = \{2_Z\} = \{\{\{\varnothing\}\}\}$.

\medskip
\noindent In general, $n_Z$ is a tower of $n$ nested singleton braces around
$\varnothing$. Each numeral is a singleton set containing the previous numeral.
\AsReq
\end{proof}

\begin{remark}[Proof shape]
This is a direct computation by iterated unfolding of the recursive definition.
No induction is needed: we simply apply the successor map three times and
substitute at each step.
\end{remark}

\begin{remark}[The tower structure]
Each Zermelo numeral wraps the previous in one more layer of braces:
$0_Z = \varnothing$, $1_Z = \{\varnothing\}$, $2_Z = \{\{\varnothing\}\}$, \ldots
The numeral $n_Z$ encodes $n$ entirely through nesting depth. This contrasts
with von Neumann numerals, where $n_V$ encodes $n$ through set size (cardinality).
\end{remark}

\begin{remark}[Dependencies]
The proof depends on: the Definition of Zermelo numerals
(\hyperref[def:zermelo-numerals]{Definition~(Zermelo numerals)}) and the
Definition of Zermelo successor
(\hyperref[def:zermelo-successor]{Definition~(Zermelo successor)}).
\end{remark}
