% =========================================================
% Proof: Tao 2.1.8 — 6 ≠ 2
% Source: volume-i/axiom-systems/notes/peano/notes-peano-numerals.tex
% =========================================================

\subsection*{Tao 2.1.8 --- $6 \neq 2$}
\label{prf:tao-2-1-8}

\begin{remark}[Return]
\hyperref[prop:tao-2-1-8]{$\leftarrow$ Back to Proposition (Tao 2.1.8) in Notes}
\end{remark}

\begin{proof}
\Claim $6$ is not equal to $2$.

\Given The numerals $6 := 5\pp$ and $2 := 1\pp$, the axiom that zero is not a
successor ($n\pp \neq 0$ for all $n \in \mathbb{N}$), and the axiom that
successor is injective ($n\pp = m\pp \Rightarrow n = m$).
\Goal To show $6 \neq 2$.

\Strategy We apply the axiom that successor is injective repeatedly to reduce
the inequality to the known result $4 \neq 0$
(Proposition~\ref{prop:tao-2-1-6}).

\textbf{Step 1.}
Suppose, toward a contradiction, that $6 = 2$.
By definition, $6 = 5\pp$ and $2 = 1\pp$, so
\[
    5\pp = 1\pp.
\]

\textbf{Step 2.}
\ByThm{Successor is injective} applied to $5\pp = 1\pp$: $5 = 1$.
Since $5 = 4\pp$ and $1 = 0\pp$, this gives $4\pp = 0\pp$.

\textbf{Step 3.}
\ByThm{Successor is injective} applied to $4\pp = 0\pp$: $4 = 0$.

\textbf{Step 4.}
But by Proposition~\ref{prop:tao-2-1-6}, $4 \neq 0$.
This is a contradiction.

\Hence the assumption $6 = 2$ is false, so $6 \neq 2$. \AsReq
\end{proof}

\begin{remark}[Proof shape]
This is a proof by contradiction. The key move is applying the axiom that
successor is injective twice to reduce $6 = 2$ to $4 = 0$, which has already
been ruled out by Proposition~\ref{prop:tao-2-1-6}.
\end{remark}

\begin{remark}[The descent pattern]
Every inequality $n \neq m$ between standard numerals with $n > m$ is proved by the
same descending argument: strip one successor from each side using the axiom that
successor is injective, repeat until one side reaches $0$, then derive a
contradiction using the axiom that zero is not a successor (via
Proposition~\ref{prop:tao-2-1-6} or directly). The number of steps equals
$\min(n, m)$; here we need exactly two.
\end{remark}

\begin{remark}[Dependencies]
This proof uses Proposition~\ref{prop:tao-2-1-6} ($4 \neq 0$) as its
contradiction target, the axiom that zero is not a successor, and the axiom
that successor is injective.
\end{remark}
