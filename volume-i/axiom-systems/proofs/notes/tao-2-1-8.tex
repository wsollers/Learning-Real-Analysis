% =========================================================
% Proof: Tao 2.1.8 — 6 ≠ 2
% Source: volume-i/axiom-systems/notes/peano/notes-peano-numerals.tex
% =========================================================

\subsection*{Tao 2.1.8 --- $6 \neq 2$}
\label{prf:tao-2-1-8}

\begin{remark}[Return]
\hyperref[prop:tao-2-1-8]{$\leftarrow$ Back to Proposition (Tao 2.1.8) in Notes}
\end{remark}

\begin{proposition}[Tao 2.1.8 --- $6 \neq 2$]
$6$ is not equal to $2$.
\end{proposition}

\begin{proof}
\Given The numerals $6 := 5\pp$ and $2 := 1\pp$, and the Peano axioms P3 and P4.
\Goal To show $6 \neq 2$.

\Strategy We apply P4 repeatedly to reduce the inequality to the known result
$4 \neq 0$ (Proposition~\ref{prop:tao-2-1-6}).

\textbf{Step 1.}
Suppose, toward a contradiction, that $6 = 2$.
By definition, $6 = 5\pp$ and $2 = 1\pp$, so
\[
    5\pp = 1\pp.
\]

\textbf{Step 2.}
\ByThm{P4} applied to the equation $5\pp = 1\pp$: $5 = 1$.
Since $5 = 4\pp$ and $1 = 0\pp$, this gives $4\pp = 0\pp$.

\textbf{Step 3.}
\ByThm{P4} applied to $4\pp = 0\pp$: $4 = 0$.

\textbf{Step 4.}
But by Proposition~\ref{prop:tao-2-1-6}, $4 \neq 0$.
This is a contradiction.

\Hence the assumption $6 = 2$ is false, so $6 \neq 2$. \AsReq
\end{proof}

\begin{remark}[Proof shape]
This is a proof by contradiction. The key move is applying P4 (injectivity of successor)
twice to reduce $6 = 2$ to $4 = 0$, which has already been ruled out.
\end{remark}

\begin{remark}[The descent pattern]
Every inequality $n \neq m$ between standard numerals with $n > m$ is proved by the
same descending argument: strip one successor from each side via P4, repeat until
one side reaches $0$, then apply P3 (via Proposition~\ref{prop:tao-2-1-6} or P3
directly) to obtain the contradiction. The number of P4-applications equals
$\min(n, m)$ when both are positive. Here $\min(6, 2) - \min(6,2) = 0$ after
adjusting, and we need exactly two applications.
\end{remark}

\begin{remark}[Dependencies]
This proof uses Proposition~\ref{prop:tao-2-1-6} ($4 \neq 0$) as its
contradiction target, and Axioms P3 and P4.
\end{remark}
