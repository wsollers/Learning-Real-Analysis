% =========================================================
% Proof: Tao 2.1.4 — 3 is a natural number
% Source: volume-i/axiom-systems/notes/peano/notes-peano-numerals.tex
% =========================================================

\subsection*{Tao 2.1.4 --- $3$ is a natural number}
\label{prf:tao-2-1-4}

\begin{remark}[Return]
\hyperref[prop:tao-2-1-4]{$\leftarrow$ Back to Proposition (Tao 2.1.4) in Notes}
\end{remark}

\begin{proposition}[Tao 2.1.4]
$3$ is a natural number.
\end{proposition}

\begin{proof}
\Given The standard numerals defined by $1 := 0\pp$, $2 := 1\pp$, $3 := 2\pp$
(Definition~\ref{def:numerals}).
\Goal To show $3 \in \mathbb{N}$.

\ByThm{P1} $0 \in \mathbb{N}$.
\ByThm{P2} applied to $0$: $0\pp \in \mathbb{N}$, i.e.\ $1 \in \mathbb{N}$.
\ByThm{P2} applied to $1$: $1\pp \in \mathbb{N}$, i.e.\ $2 \in \mathbb{N}$.
\ByThm{P2} applied to $2$: $2\pp \in \mathbb{N}$, i.e.\ $3 \in \mathbb{N}$.

\Hence $3 \in \mathbb{N}$. \AsReq
\end{proof}

\begin{remark}[Proof shape]
This is a chain of three applications of P2. Each step takes a known natural number
and produces its successor. The chain is anchored by P1, which supplies the base
element $0 \in \mathbb{N}$.
\end{remark}

\begin{remark}[Generalisation]
The same pattern shows every standard numeral $n$ is a natural number:
by induction, $0 \in \mathbb{N}$ (P1) and if $n \in \mathbb{N}$ then
$n\pp \in \mathbb{N}$ (P2), so all numerals are natural numbers.
Proposition 2.1.4 is just this argument made explicit for $n = 3$.
\end{remark}
