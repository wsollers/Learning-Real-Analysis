% =========================================================
% Axiom Systems — Notes Proofs Index
% =========================================================
% This file inputs canonical reference proofs tied to
% theorems and lemmas in the chapter notes.
%
% Add entries as:
%   \input{volume-i/axiom-systems/proofs/notes/<ID>}
%
% Classification criteria (from DESIGN.md §6.2):
%   - Proof introduces a technique that recurs in analysis
%   - Result is load-bearing for a subsequent construction
%   - Proof structure is non-obvious and worth studying again
%   - Proof is a canonical example of a general pattern
% =========================================================

% ---- Peano Numerals (notes-peano-numerals.tex) ----
% Dependency order: 2.1.4 → 2.1.6 → 2.1.8
% =========================================================
% Proof: Tao 2.1.4 — 3 is a natural number
% Source: volume-i/axiom-systems/notes/peano/notes-peano-numerals.tex
% =========================================================

\subsection*{Tao 2.1.4 --- $3$ is a natural number}
\label{prf:tao-2-1-4}

\begin{remark}[Return]
\hyperref[prop:tao-2-1-4]{$\leftarrow$ Back to Proposition (Tao 2.1.4) in Notes}
\end{remark}

\begin{proposition}[Tao 2.1.4]
$3$ is a natural number.
\end{proposition}

\begin{proof}
\Given The standard numerals defined by $1 := 0\pp$, $2 := 1\pp$, $3 := 2\pp$
(Definition~\ref{def:numerals}).
\Goal To show $3 \in \mathbb{N}$.

\ByThm{P1} $0 \in \mathbb{N}$.
\ByThm{P2} applied to $0$: $0\pp \in \mathbb{N}$, i.e.\ $1 \in \mathbb{N}$.
\ByThm{P2} applied to $1$: $1\pp \in \mathbb{N}$, i.e.\ $2 \in \mathbb{N}$.
\ByThm{P2} applied to $2$: $2\pp \in \mathbb{N}$, i.e.\ $3 \in \mathbb{N}$.

\Hence $3 \in \mathbb{N}$. \AsReq
\end{proof}

\begin{remark}[Proof shape]
This is a chain of three applications of P2. Each step takes a known natural number
and produces its successor. The chain is anchored by P1, which supplies the base
element $0 \in \mathbb{N}$.
\end{remark}

\begin{remark}[Generalisation]
The same pattern shows every standard numeral $n$ is a natural number:
by induction, $0 \in \mathbb{N}$ (P1) and if $n \in \mathbb{N}$ then
$n\pp \in \mathbb{N}$ (P2), so all numerals are natural numbers.
Proposition 2.1.4 is just this argument made explicit for $n = 3$.
\end{remark}

% =========================================================
% Proof: Tao 2.1.6 — 4 ≠ 0
% Source: volume-i/axiom-systems/notes/peano/notes-peano-numerals.tex
% =========================================================

\subsection*{Tao 2.1.6 --- $4 \neq 0$}
\label{prf:tao-2-1-6}

\begin{remark}[Return]
\hyperref[prop:tao-2-1-6]{$\leftarrow$ Back to Proposition (Tao 2.1.6) in Notes}
\end{remark}

\begin{proof}
\Claim $4$ is not equal to $0$.

\Given The numeral $4 := 3\pp$ (Definition~\ref{def:numerals})
and the axiom that zero is not a successor: $n\pp \neq 0$ for all $n \in \mathbb{N}$.
\Goal To show $4 \neq 0$.

\ByThm{3 is a natural number} (Proposition~\ref{prop:tao-2-1-4}): $3 \in \mathbb{N}$.
\ByThm{Zero is not a successor} applied to $n := 3$: $3\pp \neq 0$.
\ByDef $4 = 3\pp$.

\Hence $4 \neq 0$. \AsReq
\end{proof}

\begin{remark}[The role of the axiom that zero is not a successor]
This axiom is precisely the statement that no natural number's successor is $0$.
Without it, a cyclic system such as $\{0, 1, 2, 3\}$ with $3\pp = 0$ would satisfy
the axioms that zero is a natural number and that closure under successor holds,
but would fail this proposition. The axiom that zero is not a successor is what
rules out wrap-around.
\end{remark}

\begin{remark}[Dependencies]
This proof depends on Proposition~\ref{prop:tao-2-1-4} (so that $3 \in \mathbb{N}$
and the axiom that zero is not a successor may be applied), and on the
axiom that zero is not a successor itself.
It is used in turn by Proposition~\ref{prop:tao-2-1-8}: the descent argument for
$6 \neq 2$ eventually bottoms out here.
\end{remark}

% =========================================================
% Proof: Tao 2.1.8 — 6 ≠ 2
% Source: volume-i/axiom-systems/notes/peano/notes-peano-numerals.tex
% =========================================================

\subsection*{Tao 2.1.8 --- $6 \neq 2$}
\label{prf:tao-2-1-8}

\begin{remark}[Return]
\hyperref[prop:tao-2-1-8]{$\leftarrow$ Back to Proposition (Tao 2.1.8) in Notes}
\end{remark}

\begin{proof}
\Claim $6$ is not equal to $2$.

\Given The numerals $6 := 5\pp$ and $2 := 1\pp$, the axiom that zero is not a
successor ($n\pp \neq 0$ for all $n \in \mathbb{N}$), and the axiom that
successor is injective ($n\pp = m\pp \Rightarrow n = m$).
\Goal To show $6 \neq 2$.

\Strategy We apply the axiom that successor is injective repeatedly to reduce
the inequality to the known result $4 \neq 0$
(Proposition~\ref{prop:tao-2-1-6}).

\textbf{Step 1.}
Suppose, toward a contradiction, that $6 = 2$.
By definition, $6 = 5\pp$ and $2 = 1\pp$, so
\[
    5\pp = 1\pp.
\]

\textbf{Step 2.}
\ByThm{Successor is injective} applied to $5\pp = 1\pp$: $5 = 1$.
Since $5 = 4\pp$ and $1 = 0\pp$, this gives $4\pp = 0\pp$.

\textbf{Step 3.}
\ByThm{Successor is injective} applied to $4\pp = 0\pp$: $4 = 0$.

\textbf{Step 4.}
But by Proposition~\ref{prop:tao-2-1-6}, $4 \neq 0$.
This is a contradiction.

\Hence the assumption $6 = 2$ is false, so $6 \neq 2$. \AsReq
\end{proof}

\begin{remark}[Proof shape]
This is a proof by contradiction. The key move is applying the axiom that
successor is injective twice to reduce $6 = 2$ to $4 = 0$, which has already
been ruled out by Proposition~\ref{prop:tao-2-1-6}.
\end{remark}

\begin{remark}[The descent pattern]
Every inequality $n \neq m$ between standard numerals with $n > m$ is proved by the
same descending argument: strip one successor from each side using the axiom that
successor is injective, repeat until one side reaches $0$, then derive a
contradiction using the axiom that zero is not a successor (via
Proposition~\ref{prop:tao-2-1-6} or directly). The number of steps equals
$\min(n, m)$; here we need exactly two.
\end{remark}

\begin{remark}[Dependencies]
This proof uses Proposition~\ref{prop:tao-2-1-6} ($4 \neq 0$) as its
contradiction target, the axiom that zero is not a successor, and the axiom
that successor is injective.
\end{remark}


% ---- Zermelo Numerals (zermelo-numerals.tex) ----
% Dependency order: empty → singleton → first-few → injective → distinct → no-self → chain
% =========================================================
% Proof: Uniqueness of the empty set
% Source: volume-i/axiom-systems/notes/zermelo/zermelo-numerals.tex
% =========================================================

\subsection*{Uniqueness of the Empty Set}
\label{prf:empty-unique}

\begin{remark}[Return]
\hyperref[prop:empty-unique]{$\leftarrow$ Back to Proposition (Uniqueness of the empty set) in Notes}
\end{remark}

\begin{proposition}[Uniqueness of the empty set]
If $E$ and $E'$ have no elements, then $E = E'$.
\end{proposition}

\begin{proof}
\Given The Axiom of Extensionality (\hyperref[ax:zf-extensionality]{Axiom~(Extensionality)}):
two sets are equal if and only if they have the same elements. The hypothesis
that $E$ and $E'$ are both empty: $\forall x\,(x \notin E)$ and $\forall x\,(x
\notin E')$.

\Goal To show $E = E'$.

\Strategy By Extensionality it suffices to show $\forall x\,(x \in E
\leftrightarrow x \in E')$. Since neither set has any elements, both sides of
the biconditional are false for every $x$, so the biconditional holds vacuously.

Let $x$ be arbitrary.

($\Rightarrow$) Suppose $x \in E$. This contradicts the hypothesis that $E$ has
no elements. The conclusion $x \in E'$ follows vacuously.

($\Leftarrow$) Suppose $x \in E'$. This contradicts the hypothesis that $E'$
has no elements. The conclusion $x \in E$ follows vacuously.

Since $x$ was arbitrary, $\forall x\,(x \in E \leftrightarrow x \in E')$.
By Extensionality, $E = E'$. \AsReq
\end{proof}

\begin{remark}[Proof shape]
This is a \emph{vacuous biconditional} argument. To prove $A = B$ via
Extensionality one must show membership is equivalent; when both sets are empty
both sides of each implication have false antecedents, so both conditionals hold
trivially. The logical form is: $(x \in E \rightarrow x \in E')$ holds because
$x \in E$ is false; $(x \in E' \rightarrow x \in E)$ holds because $x \in E'$
is false.
\end{remark}

\begin{remark}[Why Extensionality is the key axiom]
Without Extensionality, two empty sets might be considered distinct objects.
Extensionality is precisely the axiom that collapses all empty sets to one: sets
are individuated by their elements, and two sets with no elements have identical
element-profiles.
\end{remark}

\begin{remark}[Dependencies]
The proof depends on: the Axiom of Extensionality
(\hyperref[ax:zf-extensionality]{Axiom~(Extensionality)}) and the definition of
the empty set (\hyperref[ax:zf-empty-set]{Axiom~(Empty Set)}).
\end{remark}

% =========================================================
% Proof: Singleton is well-defined
% Source: volume-i/axiom-systems/notes/zermelo/zermelo-numerals.tex
% =========================================================

\subsection*{Singleton is Well-Defined}
\label{prf:singleton-well-defined}

\begin{remark}[Return]
\hyperref[prop:singleton-well-defined]{$\leftarrow$ Back to Proposition (Singleton is well-defined) in Notes}
\end{remark}

\begin{proposition}[Singleton is well-defined]
For each set $a$ there exists a unique set whose only element is $a$, denoted
$\{a\}$.
\end{proposition}

\begin{proof}
\Given The Axiom of Pairing (\hyperref[ax:zf-pairing]{Axiom~(Pairing)}): for
any sets $a, b$ there exists a set $P$ with $\forall x\,(x \in P \leftrightarrow
(x = a \lor x = b))$. The Definition of Singleton
(\hyperref[def:singleton]{Definition~(Singleton)}): $\{a\} := \{a, a\}$.

\Goal To show that for each $a$, there exists a unique set $S$ such that
$\forall x\,(x \in S \leftrightarrow x = a)$.

\Strategy We prove existence and uniqueness separately.

\medskip
\noindent\textbf{Existence.} By the Axiom of Pairing applied to $a$ and $a$,
there exists a set $P$ such that
\[
    \forall x\,(x \in P \leftrightarrow (x = a \lor x = a)).
\]
Since $x = a \lor x = a$ is logically equivalent to $x = a$, we have
$\forall x\,(x \in P \leftrightarrow x = a)$. Take $\{a\} := P$.

\medskip
\noindent\textbf{Uniqueness.} Suppose $S$ and $S'$ both satisfy $\forall
x\,(x \in S \leftrightarrow x = a)$ and $\forall x\,(x \in S' \leftrightarrow
x = a)$. Then for any $x$:
\[
    x \in S \;\leftrightarrow\; x = a \;\leftrightarrow\; x \in S'.
\]
By Extensionality (\hyperref[ax:zf-extensionality]{Axiom~(Extensionality)}),
$S = S'$. \AsReq
\end{proof}

\begin{remark}[Proof shape]
This is a standard \emph{existence-and-uniqueness} argument. Existence comes
from the Pairing Axiom with both arguments identical ($\{a,a\}$); the
disjunction $x = a \lor x = a$ collapses to $x = a$. Uniqueness follows from
Extensionality: any two sets with the same membership condition are equal.
\end{remark}

\begin{remark}[Why $\{a,a\}$ works]
The Pairing Axiom provides a set containing $a$ and $b$ (both possibly equal).
Setting $b = a$ yields a set containing $a$ and $a$, which means only $a$. This
is the canonical trick for constructing singletons from pairing.
\end{remark}

\begin{remark}[Dependencies]
The proof depends on: the Axiom of Extensionality
(\hyperref[ax:zf-extensionality]{Axiom~(Extensionality)}), the Axiom of Pairing
(\hyperref[ax:zf-pairing]{Axiom~(Pairing)}), and the Definition of Singleton
(\hyperref[def:singleton]{Definition~(Singleton)}).
\end{remark}

% =========================================================
% Proof: First few Zermelo numerals
% Source: volume-i/axiom-systems/notes/zermelo/zermelo-numerals.tex
% =========================================================

\subsection*{First Few Zermelo Numerals}
\label{prf:zermelo-first-few}

\begin{remark}[Return]
\hyperref[prop:zermelo-first-few]{$\leftarrow$ Back to Proposition (First few Zermelo numerals) in Notes}
\end{remark}

\begin{proof}
\Claim $0_Z = \varnothing$, $1_Z = \{\varnothing\}$, $2_Z = \{\{\varnothing\}\}$, $3_Z = \{\{\{\varnothing\}\}\}$.

\Given The Definition of Zermelo numerals
(\hyperref[def:zermelo-numerals]{Definition~(Zermelo numerals)}):
$0_Z := \varnothing$ and $(n+1)_Z := S_Z(n_Z) = \{n_Z\}$. The Definition of
Zermelo successor (\hyperref[def:zermelo-successor]{Definition~(Zermelo
successor)}): $S_Z(n) := \{n\}$.

\Goal To compute $0_Z$, $1_Z$, $2_Z$, $3_Z$ explicitly.

\Strategy Iterate the successor definition starting from $0_Z = \varnothing$.

\medskip
$0_Z = \varnothing$ by definition.

$1_Z = S_Z(0_Z) = \{0_Z\} = \{\varnothing\}$.

$2_Z = S_Z(1_Z) = \{1_Z\} = \{\{\varnothing\}\}$.

$3_Z = S_Z(2_Z) = \{2_Z\} = \{\{\{\varnothing\}\}\}$.

\medskip
\noindent In general, $n_Z$ is a tower of $n$ nested singleton braces around
$\varnothing$. Each numeral is a singleton set containing the previous numeral.
\AsReq
\end{proof}

\begin{remark}[Proof shape]
This is a direct computation by iterated unfolding of the recursive definition.
No induction is needed: we simply apply the successor map three times and
substitute at each step.
\end{remark}

\begin{remark}[The tower structure]
Each Zermelo numeral wraps the previous in one more layer of braces:
$0_Z = \varnothing$, $1_Z = \{\varnothing\}$, $2_Z = \{\{\varnothing\}\}$, \ldots
The numeral $n_Z$ encodes $n$ entirely through nesting depth. This contrasts
with von Neumann numerals, where $n_V$ encodes $n$ through set size (cardinality).
\end{remark}

\begin{remark}[Dependencies]
The proof depends on: the Definition of Zermelo numerals
(\hyperref[def:zermelo-numerals]{Definition~(Zermelo numerals)}) and the
Definition of Zermelo successor
(\hyperref[def:zermelo-successor]{Definition~(Zermelo successor)}).
\end{remark}

% =========================================================
% Proof: Injectivity of Zermelo successor
% Source: volume-i/axiom-systems/notes/zermelo/zermelo-numerals.tex
% =========================================================

\subsection*{Injectivity of Zermelo Successor}
\label{prf:zermelo-successor-injective}

\begin{remark}[Return]
\hyperref[prop:zermelo-successor-injective]{$\leftarrow$ Back to Proposition (Injectivity of Zermelo successor) in Notes}
\end{remark}

\begin{proposition}[Injectivity of Zermelo successor]
For all sets $a, b$, $S_Z(a) = S_Z(b) \Rightarrow a = b$.
\end{proposition}

\begin{proof}
\Given The Zermelo successor $S_Z(n) = \{n\}$
(\hyperref[def:zermelo-successor]{Definition~(Zermelo successor)}). The
Proposition that singletons are well-defined
(\hyperref[prop:singleton-well-defined]{Singleton is well-defined}): the unique
element of $\{a\}$ is $a$.

\Goal To show: if $\{a\} = \{b\}$ then $a = b$.

\Strategy Suppose $\{a\} = \{b\}$. We extract the unique element of each side
and use the fact that equal sets have the same elements.

Since $\{a\} = \{b\}$, the sets have the same elements. We know $a \in \{a\}$
(by definition of singleton). Therefore $a \in \{b\}$. By definition of $\{b\}$,
the only element of $\{b\}$ is $b$. Hence $a = b$. \AsReq
\end{proof}

\begin{remark}[Proof shape]
The argument is a one-step membership extraction. If two singletons are equal as
sets, then any element of one is an element of the other. Since $a$ is the
unique element of $\{a\}$ and also an element of $\{b\}$, and $b$ is the unique
element of $\{b\}$, we conclude $a = b$. The proof does not require the Axiom
of Foundation; it uses only the definition of singleton and set equality.
\end{remark}

\begin{remark}[Contrast with von Neumann successor]
The von Neumann successor $S_V(n) = n \cup \{n\}$ also satisfies injectivity,
but the proof there is more involved because $S_V(a)$ has more than one element
when $a \neq \varnothing$. For Zermelo numerals, injectivity is immediate
because each successor is a singleton.
\end{remark}

\begin{remark}[Dependencies]
The proof depends on: the Definition of Zermelo successor
(\hyperref[def:zermelo-successor]{Definition~(Zermelo successor)}) and
\hyperref[prop:singleton-well-defined]{Singleton is well-defined}.
\end{remark}

% =========================================================
% Proof: Distinctness of Zermelo numerals
% Source: volume-i/axiom-systems/notes/zermelo/zermelo-numerals.tex
% =========================================================

\subsection*{Distinctness of Zermelo Numerals}
\label{prf:zermelo-distinct}

\begin{remark}[Return]
\hyperref[prop:zermelo-distinct]{$\leftarrow$ Back to Proposition (Distinctness of Zermelo numerals) in Notes}
\end{remark}

\begin{proof}
\Claim If $m \neq n$ as natural numbers, then $m_Z \neq n_Z$ as sets.

\Given The Zermelo numerals $0_Z = \varnothing$ and $(n+1)_Z = \{n_Z\}$
(\hyperref[def:zermelo-numerals]{Definition~(Zermelo numerals)}). Injectivity of
Zermelo successor
(\hyperref[prop:zermelo-successor-injective]{Injectivity of Zermelo successor}):
$\{a\} = \{b\} \Rightarrow a = b$.

\Goal To show: $m_Z = n_Z \Rightarrow m = n$ (equivalently, $m \neq n
\Rightarrow m_Z \neq n_Z$).

\Strategy We prove the contrapositive by strong induction on $m + n$: if $m_Z =
n_Z$ then $m = n$.

\medskip
\noindent\textbf{Base cases.} Suppose one of $m, n$ is $0$.

If $m = 0$: then $m_Z = \varnothing$. Suppose $0_Z = n_Z$, i.e.\ $\varnothing =
n_Z$. If $n \geq 1$, then $n_Z = \{(n-1)_Z\}$ is a singleton and hence
nonempty, contradicting $\varnothing = n_Z$. So $n = 0 = m$. The case $n = 0$
is symmetric.

\medskip
\noindent\textbf{Inductive step.} Suppose $m, n \geq 1$ and assume (induction
hypothesis) that for all $m', n'$ with $m' + n' < m + n$, $m'_Z = n'_Z$
implies $m' = n'$.

If $m_Z = n_Z$, then $\{(m-1)_Z\} = \{(n-1)_Z\}$. By injectivity of Zermelo
successor (\hyperref[prop:zermelo-successor-injective]{Injectivity of Zermelo
successor}), $(m-1)_Z = (n-1)_Z$. Since $(m-1) + (n-1) = m + n - 2 < m + n$,
the induction hypothesis applies: $m - 1 = n - 1$, so $m = n$. \AsReq
\end{proof}

\begin{remark}[Proof shape]
This is a proof by \emph{strong induction} on $m + n$ with a base case
covering the boundary where one index is $0$. The key step is that equal
successors (singletons) must wrap equal sets, which reduces the problem to
smaller indices where the induction hypothesis applies.
\end{remark}

\begin{remark}[Dependencies]
The proof depends on: the Definition of Zermelo numerals
(\hyperref[def:zermelo-numerals]{Definition~(Zermelo numerals)}),
\hyperref[prop:zermelo-successor-injective]{Injectivity of Zermelo successor},
and \hyperref[prop:empty-unique]{Uniqueness of the empty set} (implicitly: $\varnothing$ is
a unique set with no elements, so $\varnothing \neq \{x\}$ for any $x$).
\end{remark}

% =========================================================
% Proof: No numeral contains itself
% Source: volume-i/axiom-systems/notes/zermelo/zermelo-numerals.tex
% =========================================================

\subsection*{No Numeral Contains Itself}
\label{prf:zermelo-no-self-membership}

\begin{remark}[Return]
\hyperref[cor:zermelo-no-self-membership]{$\leftarrow$ Back to Corollary (No numeral contains itself) in Notes}
\end{remark}

\begin{proof}
\Claim For every $n$, $n_Z \notin n_Z$.

\Given The Zermelo numerals: $0_Z = \varnothing$ and $(n+1)_Z = \{n_Z\}$
(\hyperref[def:zermelo-numerals]{Definition~(Zermelo numerals)}). Distinctness
of Zermelo numerals (\hyperref[prop:zermelo-distinct]{Distinctness of Zermelo
numerals}): $m \neq n \Rightarrow m_Z \neq n_Z$.

\Goal To show $n_Z \notin n_Z$ for all $n$.

\Strategy Induction on $n$.

\medskip
\noindent\textbf{Base case} ($n = 0$). $0_Z = \varnothing$ has no elements, so
$0_Z \notin 0_Z = \varnothing$ trivially. \checkbox

\medskip
\noindent\textbf{Inductive step.} Suppose $n_Z \notin n_Z$. We show $(n+1)_Z
\notin (n+1)_Z$.

We have $(n+1)_Z = \{n_Z\}$. The only element of $\{n_Z\}$ is $n_Z$. So
$(n+1)_Z \in (n+1)_Z$ would require $(n+1)_Z = n_Z$, i.e.\ $n_Z \in (n+1)_Z$
and thus $\{n_Z\} = n_Z$. But by distinctness, $(n+1)_Z \neq n_Z$ (since $n+1
\neq n$). Therefore $(n+1)_Z$ is not the unique element $n_Z$ of $(n+1)_Z$,
so $(n+1)_Z \notin (n+1)_Z$. \checkbox

By induction, $n_Z \notin n_Z$ for all $n$. \AsReq
\end{proof}

\begin{remark}[Proof shape]
The proof uses induction but the inductive hypothesis is not actually needed in
the step: the conclusion $(n+1)_Z \notin (n+1)_Z$ follows directly from the
distinctness of consecutive numerals ($(n+1)_Z \neq n_Z$) and the fact that
$(n+1)_Z = \{n_Z\}$ has $n_Z$ as its only element. The base case uses the
emptiness of $\varnothing$.
\end{remark}

\begin{remark}[Relation to Foundation]
In ZFC, the Axiom of Foundation (Regularity) directly prohibits $x \in x$ for
any set $x$. This proof establishes the same conclusion for Zermelo numerals
specifically without invoking Foundation, relying instead on the explicit
recursive structure and the distinctness result.
\end{remark}

\begin{remark}[Dependencies]
The proof depends on: the Definition of Zermelo numerals
(\hyperref[def:zermelo-numerals]{Definition~(Zermelo numerals)}) and
\hyperref[prop:zermelo-distinct]{Distinctness of Zermelo numerals}.
\end{remark}

% =========================================================
% Proof: Strict chain under membership
% Source: volume-i/axiom-systems/notes/zermelo/zermelo-numerals.tex
% =========================================================

\subsection*{Strict Chain Under Membership}
\label{prf:zermelo-chain}

\begin{remark}[Return]
\hyperref[cor:zermelo-chain]{$\leftarrow$ Back to Corollary (Strict chain under membership) in Notes}
\end{remark}

\begin{corollary}[Strict chain under membership]
For $n \geq 1$,
\[
(n-1)_Z \in n_Z,\quad (n-2)_Z \in (n-1)_Z,\;\ldots,\; 0_Z \in 1_Z.
\]
\end{corollary}

\begin{proof}
\Given The Zermelo numerals: $0_Z = \varnothing$ and $k_Z = \{(k-1)_Z\}$ for
$k \geq 1$ (\hyperref[def:zermelo-numerals]{Definition~(Zermelo numerals)}).

\Goal To show: for each $k$ with $1 \leq k \leq n$, $(k-1)_Z \in k_Z$.

\Strategy For any $k \geq 1$, the definition gives $k_Z = \{(k-1)_Z\}$. By the
definition of singleton membership, $(k-1)_Z \in \{(k-1)_Z\} = k_Z$.

Applying this to $k = 1, 2, \ldots, n$ successively:
\[
0_Z \in 1_Z,\quad 1_Z \in 2_Z,\quad \ldots,\quad (n-1)_Z \in n_Z.
\]
This is precisely the stated chain. \AsReq
\end{proof}

\begin{remark}[Proof shape]
Each step in the chain is a one-line membership fact: $k_Z = \{(k-1)_Z\}$
immediately gives $(k-1)_Z \in k_Z$. The corollary is really just $n$ instances
of this single observation assembled together.
\end{remark}

\begin{remark}[The chain is strict]
Each membership relation in the chain is strict in the sense that $(k-1)_Z \neq
k_Z$ (by distinctness of numerals) and $(k-1)_Z \notin (k-1)_Z$ (no numeral
contains itself). The chain does not fold back on itself.
\end{remark}

\begin{remark}[Contrast with von Neumann]
In von Neumann numerals, $m_V \in n_V$ for \emph{all} $m < n$, not just
consecutive ones. In Zermelo numerals, membership is only between consecutive
numerals; $0_Z \notin 2_Z$ because $2_Z = \{1_Z\}$ contains only $1_Z$.
\end{remark}

\begin{remark}[Dependencies]
The proof depends on: the Definition of Zermelo numerals
(\hyperref[def:zermelo-numerals]{Definition~(Zermelo numerals)}) and
\hyperref[prop:singleton-well-defined]{Singleton is well-defined}.
\end{remark}


% ---- von Neumann Numerals (von_neuman-numerals.tex) ----
% Dependency order: union → first-few → transitive → injective → predecessors → monotone
% =========================================================
% Proof: Basic properties of union
% Source: volume-i/axiom-systems/notes/von-neuman/von_neuman-numerals.tex
% =========================================================

\subsection*{Basic Properties of Union}
\label{prf:union-basics}

\begin{remark}[Return]
\hyperref[prop:union-basics]{$\leftarrow$ Back to Proposition (Basic properties of union) in Notes}
\end{remark}

\begin{proof}
\Claim For all sets $a, b$ and all $x$, $x \in a \cup b \;\leftrightarrow\; (x \in a \lor x \in b)$.

\Given The Axiom of Union (\hyperref[ax:zf-union]{Axiom~(Union)}): for any set
$A$ there exists $\bigcup A$ with $x \in \bigcup A \leftrightarrow \exists y \in
A\,(x \in y)$. The Definition of binary union
(\hyperref[def:binary-union]{Definition~(Binary union)}): $a \cup b := \bigcup
\{a, b\}$. The Axiom of Pairing (\hyperref[ax:vn-pairing]{Axiom~(Pairing)}):
$\{a, b\}$ exists with $z \in \{a, b\} \leftrightarrow (z = a \lor z = b)$.

\Goal To show $x \in a \cup b \leftrightarrow (x \in a \lor x \in b)$.

\Strategy Unfold the definition $a \cup b = \bigcup \{a, b\}$ and apply the
Union Axiom, then the characterisation of $\{a, b\}$.

\medskip
By the Union Axiom applied to $A = \{a, b\}$:
\[
x \in \bigcup\{a, b\}
\;\leftrightarrow\;
\exists y \in \{a, b\}\,(x \in y).
\]
By the Pairing Axiom, $y \in \{a, b\} \leftrightarrow (y = a \lor y = b)$.
Therefore:
\[
\exists y \in \{a, b\}\,(x \in y)
\;\leftrightarrow\;
(x \in a \lor x \in b).
\]
(The forward direction: if such $y$ exists and $y = a$, then $x \in a$; if $y =
b$, then $x \in b$. The backward direction: if $x \in a$, take $y = a \in
\{a,b\}$; if $x \in b$, take $y = b \in \{a,b\}$.)

Since $a \cup b = \bigcup\{a, b\}$ by definition, the chain of equivalences
gives $x \in a \cup b \leftrightarrow (x \in a \lor x \in b)$. \AsReq
\end{proof}

\begin{remark}[Proof shape]
The proof is a chain of equivalences: unfold the definition of $a \cup b$,
apply the Union Axiom, then unpack the membership condition for $\{a, b\}$ from
the Pairing Axiom. No case analysis is needed; both directions of the
biconditional follow from the same chain.
\end{remark}

\begin{remark}[Why this requires two axioms]
The binary union $a \cup b$ is defined in two steps: Pairing gives $\{a, b\}$,
then Union gives $\bigcup \{a, b\}$. Both axioms are genuinely needed: Pairing
constructs the ``indexing set'' $\{a, b\}$ over which Union ranges.
\end{remark}

\begin{remark}[Dependencies]
The proof depends on: the Axiom of Union (\hyperref[ax:zf-union]{Axiom~(Union)}),
the Axiom of Pairing (\hyperref[ax:vn-pairing]{Axiom~(Pairing)}), and the
Definition of binary union (\hyperref[def:binary-union]{Definition~(Binary union)}).
\end{remark}

% =========================================================
% Proof: First few von Neumann numerals
% Source: volume-i/axiom-systems/notes/von-neuman/von_neuman-numerals.tex
% =========================================================

\subsection*{First Few von Neumann Numerals}
\label{prf:vn-first-few}

\begin{remark}[Return]
\hyperref[prop:vn-first-few]{$\leftarrow$ Back to Proposition (First few von Neumann numerals) in Notes}
\end{remark}

\begin{proof}
\Claim $0_V = \varnothing$, $1_V = \{\varnothing\}$, $2_V = \{\varnothing, \{\varnothing\}\}$, $3_V = \{\varnothing, \{\varnothing\}, \{\varnothing,\{\varnothing\}\}\}$, and in general $n_V = \{0_V, 1_V, \ldots, (n-1)_V\}$ for $n \geq 1$.

\Given The von Neumann numerals: $0_V := \varnothing$ and $(n+1)_V := n_V \cup
\{n_V\}$ (\hyperref[def:vn-numerals]{Definition~(von Neumann numerals)}).
Basic properties of union
(\hyperref[prop:union-basics]{Basic properties of union}): $x \in a \cup b
\leftrightarrow (x \in a \lor x \in b)$.

\Goal Compute $0_V, 1_V, 2_V, 3_V$ explicitly.

\Strategy Iterate the successor definition and apply the union characterisation.

\medskip
$0_V = \varnothing$ by definition.

$1_V = 0_V \cup \{0_V\} = \varnothing \cup \{\varnothing\}$.
An element $x \in \varnothing \cup \{\varnothing\}$ iff $x \in \varnothing$ or
$x \in \{\varnothing\}$, i.e.\ iff $x = \varnothing$. So $1_V = \{\varnothing\}
= \{0_V\}$.

$2_V = 1_V \cup \{1_V\} = \{\varnothing\} \cup \{\{\varnothing\}\}$.
An element $x$ belongs iff $x = \varnothing$ or $x = \{\varnothing\}$. So
$2_V = \{\varnothing, \{\varnothing\}\} = \{0_V, 1_V\}$.

$3_V = 2_V \cup \{2_V\} = \{0_V, 1_V\} \cup \{2_V\} = \{0_V, 1_V, 2_V\}$.

The pattern $n_V = \{0_V, 1_V, \ldots, (n-1)_V\}$ holds at each step: the
successor appends the current numeral to the existing set. \AsReq
\end{proof}

\begin{remark}[Proof shape]
Direct computation by iterated unfolding, exactly as for Zermelo numerals.
The key difference is that the von Neumann successor $n_V \cup \{n_V\}$
\emph{accumulates} all predecessors, whereas the Zermelo successor $\{n_Z\}$
discards all but the immediate predecessor.
\end{remark}

\begin{remark}[Cardinality encodes the number]
Each von Neumann numeral $n_V$ has exactly $n$ elements: $n_V =
\{0_V,\ldots,(n-1)_V\}$. This means $|n_V| = n$, so the cardinality of the
set directly encodes the number. This is one reason von Neumann numerals are
preferred in the standard development of ZFC.
\end{remark}

\begin{remark}[Dependencies]
The proof depends on: the Definition of von Neumann numerals
(\hyperref[def:vn-numerals]{Definition~(von Neumann numerals)}) and
\hyperref[prop:union-basics]{Basic properties of union}.
\end{remark}

% =========================================================
% Proof: Each von Neumann numeral is transitive
% Source: volume-i/axiom-systems/notes/von-neuman/von_neuman-numerals.tex
% =========================================================

\subsection*{Each von Neumann Numeral is Transitive}
\label{prf:vn-transitive}

\begin{remark}[Return]
\hyperref[prop:vn-transitive]{$\leftarrow$ Back to Proposition (Each von Neumann numeral is transitive) in Notes}
\end{remark}

\begin{proof}
\Claim For every $n$, the set $n_V$ is transitive: if $x \in n_V$ then $x \subseteq n_V$.

\Given The von Neumann numerals: $0_V = \varnothing$ and $(n+1)_V = n_V \cup
\{n_V\}$ (\hyperref[def:vn-numerals]{Definition~(von Neumann numerals)}).
The definition of transitivity (\hyperref[def:transitive-set]{Definition~(Transitive set)}):
$A$ is transitive iff $x \in A \Rightarrow x \subseteq A$.
Basic properties of union (\hyperref[prop:union-basics]{Basic properties of union}).

\Goal To show $n_V$ is transitive for every $n$.

\Strategy Induction on $n$.

\medskip
\noindent\textbf{Base case} ($n = 0$). $0_V = \varnothing$. There is no $x
\in \varnothing$, so the transitivity condition holds vacuously. \checkbox

\medskip
\noindent\textbf{Inductive step.} Suppose $n_V$ is transitive. We show
$(n+1)_V = n_V \cup \{n_V\}$ is transitive.

Let $x \in n_V \cup \{n_V\}$. By the union property, either $x \in n_V$ or $x
= n_V$.

\textit{Case 1: $x \in n_V$.} By the induction hypothesis, $x \subseteq n_V$.
Since $n_V \subseteq n_V \cup \{n_V\} = (n+1)_V$, we have $x \subseteq
(n+1)_V$.

\textit{Case 2: $x = n_V$.} We need $n_V \subseteq (n+1)_V = n_V \cup \{n_V\}$.
For any $y \in n_V$, $y \in n_V \cup \{n_V\}$ by the union property. So $n_V
\subseteq (n+1)_V$. \checkbox

In both cases, $x \subseteq (n+1)_V$. By induction, $n_V$ is transitive for
all $n$. \AsReq
\end{proof}

\begin{remark}[Proof shape]
Standard induction with a two-case split in the inductive step: elements of
$(n+1)_V = n_V \cup \{n_V\}$ are either elements of $n_V$ (handled by the IH)
or equal to $n_V$ itself (handled by the subset relation $n_V \subseteq
(n+1)_V$). Both cases reduce immediately to the definitions.
\end{remark}

\begin{remark}[Why transitivity matters]
Transitivity of von Neumann numerals is what makes the ordering $m_V \in n_V
\Leftrightarrow m < n$ work cleanly: if $k \in m_V$ and $m_V \in n_V$, then
by transitivity of $n_V$, $k \in n_V$. This transitivity at the set level
mirrors the transitivity of the usual ordering $<$ on natural numbers.
\end{remark}

\begin{remark}[Dependencies]
The proof depends on: the Definition of von Neumann numerals
(\hyperref[def:vn-numerals]{Definition~(von Neumann numerals)}), the Definition
of transitive set (\hyperref[def:transitive-set]{Definition~(Transitive set)}),
and \hyperref[prop:union-basics]{Basic properties of union}.
\end{remark}

% =========================================================
% Proof: Successor is injective (von Neumann)
% Source: volume-i/axiom-systems/notes/von-neuman/von_neuman-numerals.tex
% =========================================================

\subsection*{Successor is Injective}
\label{prf:vn-successor-injective}

\begin{remark}[Return]
\hyperref[prop:vn-successor-injective]{$\leftarrow$ Back to Proposition (Successor is injective) in Notes}
\end{remark}

\begin{proposition}[Successor is injective]
For all sets $a, b$, $S_V(a) = S_V(b) \Rightarrow a = b$.
\end{proposition}

\begin{proof}
\Given The von Neumann successor $S_V(n) = n \cup \{n\}$
(\hyperref[def:vn-successor]{Definition~(von Neumann successor)}). Basic
properties of union (\hyperref[prop:union-basics]{Basic properties of union}).
The Axiom of Extensionality (\hyperref[ax:vn-extensionality]{Axiom~(Extensionality)}).

\Goal To show: if $a \cup \{a\} = b \cup \{b\}$, then $a = b$.

\Strategy Suppose $a \cup \{a\} = b \cup \{b\}$. We show $a \in b \cup \{b\}$
and $b \in a \cup \{a\}$, and in each case extract $a = b$.

Since $a \in \{a\} \subseteq a \cup \{a\} = b \cup \{b\}$, we have $a \in b
\cup \{b\}$. By the union property, $a \in b$ or $a = b$.

Similarly, $b \in \{b\} \subseteq b \cup \{b\} = a \cup \{a\}$, so $b \in a$
or $b = a$.

\medskip
\textit{Case 1: $a = b$.} Done.

\textit{Case 2: $a \in b$ and $b \in a$.} This means $a \in b$ and $b \in a$
simultaneously. In the context of von Neumann numerals, the Axiom of Foundation
(Regularity) prohibits membership cycles: there cannot be sets $a$ and $b$ with
$a \in b \in a$. Therefore this case is ruled out by Foundation, and we
conclude $a = b$.

\textit{Case 3: $a \in b$ and $b = a$.} Then $a \in a$, contradicting
Foundation. So this case is also impossible.

\textit{Case 4: $a = b$ and $b \in a$ (i.e., $a = b$ and $b \in b$).}
Contradicts Foundation ($b \in b$ is excluded). So this is impossible.

In all reachable cases, $a = b$. \AsReq
\end{proof}

\begin{remark}[Proof shape]
The argument uses the union characterisation to extract that both $a$ and $b$
belong to the respective successors, then cases on the resulting membership
conditions. Cases involving membership cycles ($a \in b \in a$ or $a \in a$)
are excluded by the Axiom of Foundation. As noted in the source, in a
Foundation-free setting an extra assumption may be required.
\end{remark}

\begin{remark}[Contrast with Zermelo]
For Zermelo successors $S_Z(n) = \{n\}$, injectivity was immediate because
$\{a\} = \{b\}$ directly gives $a = b$ via singleton membership. For von
Neumann successors $S_V(n) = n \cup \{n\}$, the set has multiple elements in
general, making the argument more involved.
\end{remark}

\begin{remark}[Dependencies]
The proof depends on: the Definition of von Neumann successor
(\hyperref[def:vn-successor]{Definition~(von Neumann successor)}),
\hyperref[prop:union-basics]{Basic properties of union}, and the Axiom of
Foundation (Regularity) from ZFC.
\end{remark}

% =========================================================
% Proof: Each numeral is the set of its predecessors
% Source: volume-i/axiom-systems/notes/von-neuman/von_neuman-numerals.tex
% =========================================================

\subsection*{Each Numeral is the Set of Its Predecessors}
\label{prf:vn-predecessors}

\begin{remark}[Return]
\hyperref[cor:vn-predecessors]{$\leftarrow$ Back to Corollary (Each numeral is the set of its predecessors) in Notes}
\end{remark}

\begin{proof}
\Claim For every $n$, $n_V = \{m_V : m < n\}$.

\Given The von Neumann numerals: $0_V = \varnothing$ and $(n+1)_V = n_V \cup
\{n_V\}$ (\hyperref[def:vn-numerals]{Definition~(von Neumann numerals)}). Basic
properties of union (\hyperref[prop:union-basics]{Basic properties of union}).

\Goal To show $n_V = \{m_V : m < n\}$ for all $n$.

\Strategy Induction on $n$.

\medskip
\noindent\textbf{Base case} ($n = 0$). $0_V = \varnothing$ and $\{m_V : m < 0\}
= \varnothing$ (there are no natural numbers less than $0$). So $0_V = \{m_V :
m < 0\}$. \checkbox

\medskip
\noindent\textbf{Inductive step.} Suppose $n_V = \{m_V : m < n\}$. We show
$(n+1)_V = \{m_V : m < n+1\}$.

We have $\{m_V : m < n+1\} = \{m_V : m < n\} \cup \{n_V\}$ (the set of
predecessors of $n+1$ is the set of predecessors of $n$ together with $n$
itself).

By the induction hypothesis, $\{m_V : m < n\} = n_V$. Therefore:
\[
\{m_V : m < n+1\} = n_V \cup \{n_V\} = (n+1)_V.
\]
\checkbox

By induction, $n_V = \{m_V : m < n\}$ for all $n$. \AsReq
\end{proof}

\begin{remark}[Proof shape]
Straightforward induction. The key step is recognising that $\{m : m < n+1\} =
\{m : m < n\} \cup \{n\}$, which makes the successor step align perfectly with
the recursive definition $(n+1)_V = n_V \cup \{n_V\}$.
\end{remark}

\begin{remark}[Significance]
This corollary gives a clean slogan for von Neumann numerals: ``$n$ is the set
of all smaller numbers.'' This is the property that makes $m < n \Leftrightarrow
m_V \in n_V$ immediate: membership in $n_V$ is exactly being a predecessor.
\end{remark}

\begin{remark}[Dependencies]
The proof depends on: the Definition of von Neumann numerals
(\hyperref[def:vn-numerals]{Definition~(von Neumann numerals)}) and
\hyperref[prop:union-basics]{Basic properties of union}.
\end{remark}

% =========================================================
% Proof: Monotonicity
% Source: volume-i/axiom-systems/notes/von-neuman/von_neuman-numerals.tex
% =========================================================

\subsection*{Monotonicity}
\label{prf:vn-monotone}

\begin{remark}[Return]
\hyperref[cor:vn-monotone]{$\leftarrow$ Back to Corollary (Monotonicity) in Notes}
\end{remark}

\begin{corollary}[Monotonicity]
If $m < n$, then $m_V \subsetneq n_V$.
\end{corollary}

\begin{proof}
\Given Each numeral is the set of its predecessors
(\hyperref[cor:vn-predecessors]{Each numeral is the set of its predecessors}):
$n_V = \{k_V : k < n\}$.

\Goal To show $m_V \subseteq n_V$ and $m_V \neq n_V$ whenever $m < n$.

\medskip
\noindent\textbf{$m_V \subseteq n_V$.} Let $x \in m_V$. Since $m_V = \{k_V :
k < m\}$, we have $x = k_V$ for some $k < m$. Since $k < m < n$, $k < n$, so
$k_V \in n_V = \{k_V : k < n\}$. Hence $x \in n_V$.

\medskip
\noindent\textbf{$m_V \neq n_V$.} Since $m < n$, we have $m_V \in n_V = \{k_V
: k < n\}$. If $m_V = n_V$, then $n_V \in n_V$, i.e.\ $n_V$ contains itself.
But by the Axiom of Foundation this is impossible. Hence $m_V \neq n_V$.

Therefore $m_V \subsetneq n_V$. \AsReq
\end{proof}

\begin{remark}[Proof shape]
The subset direction is a direct membership-chase using the ``predecessors''
characterisation. Strictness ($m_V \neq n_V$) requires ruling out $n_V \in
n_V$, which is done by Foundation (or alternatively by distinctness of
numerals: $m \neq n$ implies $m_V \neq n_V$).
\end{remark}

\begin{remark}[Alternative for strictness without Foundation]
One can avoid invoking Foundation for strictness by instead using
\hyperref[prop:vn-successor-injective]{Successor is injective} and the
distinctness of von Neumann numerals (which follows from the injectivity result
by induction, analogously to the Zermelo case). The Foundation route is shorter
but depends on a stronger axiom.
\end{remark}

\begin{remark}[Dependencies]
The proof depends on: \hyperref[cor:vn-predecessors]{Each numeral is the set of
its predecessors} and the Axiom of Foundation for the strictness half.
\end{remark}

