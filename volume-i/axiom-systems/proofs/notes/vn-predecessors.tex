% =========================================================
% Proof: Each numeral is the set of its predecessors
% Source: volume-i/axiom-systems/notes/von-neuman/von_neuman-numerals.tex
% =========================================================

\subsection*{Each Numeral is the Set of Its Predecessors}
\label{prf:vn-predecessors}

\begin{remark}[Return]
\hyperref[cor:vn-predecessors]{$\leftarrow$ Back to Corollary (Each numeral is the set of its predecessors) in Notes}
\end{remark}

\begin{proof}
\Claim For every $n$, $n_V = \{m_V : m < n\}$.

\Given The von Neumann numerals: $0_V = \varnothing$ and $(n+1)_V = n_V \cup
\{n_V\}$ (\hyperref[def:vn-numerals]{Definition~(von Neumann numerals)}). Basic
properties of union (\hyperref[prop:union-basics]{Basic properties of union}).

\Goal To show $n_V = \{m_V : m < n\}$ for all $n$.

\Strategy Induction on $n$.

\medskip
\noindent\textbf{Base case} ($n = 0$). $0_V = \varnothing$ and $\{m_V : m < 0\}
= \varnothing$ (there are no natural numbers less than $0$). So $0_V = \{m_V :
m < 0\}$. \checkbox

\medskip
\noindent\textbf{Inductive step.} Suppose $n_V = \{m_V : m < n\}$. We show
$(n+1)_V = \{m_V : m < n+1\}$.

We have $\{m_V : m < n+1\} = \{m_V : m < n\} \cup \{n_V\}$ (the set of
predecessors of $n+1$ is the set of predecessors of $n$ together with $n$
itself).

By the induction hypothesis, $\{m_V : m < n\} = n_V$. Therefore:
\[
\{m_V : m < n+1\} = n_V \cup \{n_V\} = (n+1)_V.
\]
\checkbox

By induction, $n_V = \{m_V : m < n\}$ for all $n$. \AsReq
\end{proof}

\begin{remark}[Proof shape]
Straightforward induction. The key step is recognising that $\{m : m < n+1\} =
\{m : m < n\} \cup \{n\}$, which makes the successor step align perfectly with
the recursive definition $(n+1)_V = n_V \cup \{n_V\}$.
\end{remark}

\begin{remark}[Significance]
This corollary gives a clean slogan for von Neumann numerals: ``$n$ is the set
of all smaller numbers.'' This is the property that makes $m < n \Leftrightarrow
m_V \in n_V$ immediate: membership in $n_V$ is exactly being a predecessor.
\end{remark}

\begin{remark}[Dependencies]
The proof depends on: the Definition of von Neumann numerals
(\hyperref[def:vn-numerals]{Definition~(von Neumann numerals)}) and
\hyperref[prop:union-basics]{Basic properties of union}.
\end{remark}
