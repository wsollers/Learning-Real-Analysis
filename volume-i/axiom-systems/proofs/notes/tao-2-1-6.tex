% =========================================================
% Proof: Tao 2.1.6 — 4 ≠ 0
% Source: volume-i/axiom-systems/notes/peano/notes-peano-numerals.tex
% =========================================================

\subsection*{Tao 2.1.6 --- $4 \neq 0$}
\label{prf:tao-2-1-6}

\begin{remark}[Return]
\hyperref[prop:tao-2-1-6]{$\leftarrow$ Back to Proposition (Tao 2.1.6) in Notes}
\end{remark}

\begin{proposition}[Tao 2.1.6 --- $4 \neq 0$]
$4$ is not equal to $0$.
\end{proposition}

\begin{proof}
\Given The numeral $4 := 3\pp$ (Definition~\ref{def:numerals})
and the Peano axiom P3: $n\pp \neq 0$ for all $n \in \mathbb{N}$.
\Goal To show $4 \neq 0$.

\ByThm{Proposition~\ref{prop:tao-2-1-4}} $3 \in \mathbb{N}$.
\ByThm{P3} applied to $n := 3$: $3\pp \neq 0$.
\ByDef $4 = 3\pp$.

\Hence $4 \neq 0$. \AsReq
\end{proof}

\begin{remark}[The role of P3]
Axiom P3 is precisely the statement that no natural number's successor is $0$.
Without it, a cyclic system such as $\{0, 1, 2, 3\}$ with $3\pp = 0$ would satisfy
P1 and P2 but fail this proposition. P3 is the axiom that rules out wrap-around.
\end{remark}

\begin{remark}[Dependency]
This proof depends on Proposition~\ref{prop:tao-2-1-4} (so that $3 \in \mathbb{N}$
and P3 may be applied) and on Axiom P3.
It is used in turn by Proposition~\ref{prop:tao-2-1-8}: the descent argument for
$6 \neq 2$ eventually bottoms out here.
\end{remark}
