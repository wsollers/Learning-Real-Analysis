% --------------------------------------
% Proof: RA-TAO-C02-S2.2-E02
% --------------------------------------
\clearpage
\phantomsection
\hypertarget{proof-RA-TAO-C02-S2.2-E02}{}

\subsubsection[Predecessor Existence \& Uniqueness (L2.2.10)]{Proof --- RA-TAO-C02-S2.2-E02}

\noindent
\textbf{Source.}
\srccite{TaoAnalysis1}{Chapter~2, \S2.2, Exercise~2.2.2}.

\vspace{0.75em}

\noindent
\textbf{Goal.}
Prove Tao’s Lemma~2.2.10, namely:

\medskip
\noindent
\textbf{Lemma~2.2.10.}
Let $a$ be a positive natural number. Then there exists exactly one natural
number $b$ such that
\[
b\pp = a.
\]

\medskip
\emph{(Hint: use induction.)}

\vspace{0.75em}

\noindent
% =========================================================
% Proof: Lemma 2.2.10 — Predecessor Existence and Uniqueness
% =========================================================

\begin{lemma}[Predecessor exists and is unique]
\label{lem:predecessor}
Let $a$ be a positive natural number. Then there exists exactly one
natural number $b$ such that $b\pp = a$.
\end{lemma}

\begin{proof}
We prove existence and uniqueness separately.

% ---------------------------------------------------------
\medskip
\noindent\textbf{Part 1: Existence.}

We prove by induction on $a$ the statement:
\[
P(a) :\equiv \text{if } a \neq 0
\text{ then } \exists\, b \in \mathbb{N},\ b\pp = a.
\]

\medskip
\noindent\textbf{Base case} ($a = 0$).
$P(0)$ states: if $0 \neq 0$ then $\exists\, b,\ b\pp = 0$.
The hypothesis $0 \neq 0$ is false, so $P(0)$ is vacuously true.

\medskip
\noindent\textbf{Inductive step.}
Assume $P(a)$ holds. We show $P(a\pp)$.

$P(a\pp)$ states: if $a\pp \neq 0$ then $\exists\, b \in \mathbb{N},\
b\pp = a\pp$.

\noindent
\begin{tabular}{T S J}
\toprule
\textbf{Tag} &
\multicolumn{1}{p{0.44\textwidth}}{\textbf{Step}} &
\textbf{Justification} \\
\midrule
\addlinespace[4pt]

\tagTA
&
a\pp \neq 0
&
By P3: $n\pp \neq 0$ for all $n \in \mathbb{N}$.
The hypothesis of $P(a\pp)$ is satisfied.
\\[10pt]

\tagDU
&
\text{Take } b := a
&
We exhibit a witness. Since $a \in \mathbb{N}$, this is a valid
natural number.
\\[10pt]

\tagAM
&
b\pp = a\pp
&
Substituting $b = a$ gives $a\pp = a\pp$. \hfill$\blacksquare_{\exists}$
\\[6pt]

\bottomrule
\end{tabular}

\medskip
\noindent
Note: the inductive hypothesis $P(a)$ was not needed.
The witness $b = a$ satisfies $b\pp = a\pp$ directly,
regardless of $a$.
By induction (P5), $P(a)$ holds for all $a \in \mathbb{N}$.

% ---------------------------------------------------------
\medskip
\noindent\textbf{Part 2: Uniqueness.}

Suppose $b, c \in \mathbb{N}$ both satisfy $b\pp = a$ and $c\pp = a$.

\noindent
\begin{tabular}{T S J}
\toprule
\textbf{Tag} &
\multicolumn{1}{p{0.44\textwidth}}{\textbf{Step}} &
\textbf{Justification} \\
\midrule
\addlinespace[4pt]

\tagAM
&
b\pp = c\pp
&
Both equal $a$: $b\pp = a = c\pp$.
\\[10pt]

\tagTA
&
b = c
&
By P4: $n\pp = m\pp \Rightarrow n = m$.
Apply with $n = b$, $m = c$. \hfill$\blacksquare$
\\[6pt]

\bottomrule
\end{tabular}

\medskip
Existence and uniqueness together give $\exists!\, b \in \mathbb{N}$
such that $b\pp = a$.
\end{proof}

\begin{remark}[Vacuous base case]
The base case $P(0)$ is vacuously true because the hypothesis
$a \neq 0$ fails at $a = 0$. A vacuous base case is logically
valid and requires no further argument. It appears whenever the
claim has a precondition (here, positivity) that excludes the
base value.
\end{remark}

\begin{remark}[Inductive hypothesis not used]
The inductive step did not use $P(a)$ at all --- the witness
$b = a$ was available without it. This is unusual but not
contradictory. Induction was still the right proof structure
because the claim is universal over $\mathbb{N}$, and the
inductive step still needed to be verified for each $a$.
The vacuity of the hypothesis use simply means the result
holds uniformly, not by accumulation of prior cases.
\end{remark}

\begin{remark}[Uniqueness is one axiom]
The uniqueness part reduces entirely to P4 (injectivity of
successor) in two lines. Whenever a uniqueness claim involves
successors, P4 should be the first tool reached for.
\end{remark}

\vspace{1em}
