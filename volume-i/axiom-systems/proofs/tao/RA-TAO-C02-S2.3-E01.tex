% =========================================================
% Section 2.3 Exercises
% =========================================================

% --------------------------------------
% Proof: RA-TAO-C02-S2.3-E01
% --------------------------------------
\clearpage
\phantomsection
\hypertarget{proof-RA-TAO-C02-S2.3-E01}{}

\subsubsection[Multiplication is Commutative (L2.3.2)]{Proof --- RA-TAO-C02-S2.3-E01}

\noindent
\textbf{Source.}
\srccite{TaoAnalysis1}{Chapter~2, \S2.3, Exercise~2.3.1}.

\vspace{0.75em}

\noindent
\textbf{Goal.}
Prove Lemma~2.3.2 (Multiplication is commutative):

\medskip
\noindent
\textbf{Lemma~2.3.2.}
Let $n, m$ be natural numbers. Then $n \times m = m \times n$.

\vspace{0.75em}

\noindent
\begin{definition*}[Multiplication on $\mathbb{N}$ (recursive definition)]
Let $m \in \mathbb{N}$ be fixed. Define multiplication recursively in the left argument by:
\[
0 \cdot m := 0,
\qquad
(n^{++}) \cdot m := (n \cdot m) + m.
\]
We will write $nm$ in place of $n \cdot m$.
\end{definition*}

\begin{lemma*}[Right-successor rule for multiplication]
\label{lem:right-successor-mult}
For all $m,n \in \mathbb{N}$,
\[
m(n^{++}) = mn + m.
\]
\end{lemma*}

\begin{proof}
Fix $n \in \mathbb{N}$. We prove the statement by induction on $m$.

\medskip
\noindent\textbf{Base case: $m=0$.}
By definition of multiplication, $0(n^{++}) = 0$.
Also $0n=0$, hence $(0n)+0 = 0+0=0$.
Therefore $0(n^{++}) = 0n + 0$.

\medskip
\noindent\textbf{Inductive step.}
Assume $m(n^{++}) = mn + m$ for some $m \in \mathbb{N}$.
Then, by the recursive definition of multiplication (recursion in the left argument),
\[
(m^{++})(n^{++}) = m(n^{++}) + (n^{++}).
\]
Substituting the inductive hypothesis gives
\[
(m^{++})(n^{++}) = (mn + m) + (n^{++}).
\]
Using associativity of addition,
\[
(mn + m) + (n^{++}) = mn + \bigl(m + n^{++}\bigr).
\]
Reassociate and use $m^{++}=m+1$ (successor as ``add one'') to rewrite the right-hand side as
\[
mn + \bigl(m + n^{++}\bigr) = (mn + n) + m^{++}.
\]
By the recursive definition of multiplication,
\[
(m^{++})n = mn + n.
\]
Hence
\[
(m^{++})(n^{++}) = (m^{++})n + m^{++}.
\]
This is exactly the desired identity with $m$ replaced by $m^{++}$.
\end{proof}

\begin{theorem}[Multiplication is commutative]
\label{thm:mult-commutative}
For all $n,m \in \mathbb{N}$,
\[
nm = mn.
\]
\end{theorem}

\begin{proof}
We prove the statement by induction on $n$.
Define the induction predicate
\[
P(n) :\Longleftrightarrow \forall m \in \mathbb{N},\; nm = mn.
\]

\medskip
\noindent\textbf{Base case: $n=0$.}
Let $m \in \mathbb{N}$ be arbitrary.
By definition of multiplication, $0m = 0$.
We claim $m0=0$ for all $m$; this follows by induction on $m$ using the recursive definition
of multiplication (or can be cited as the standard ``times zero'' lemma proved earlier).
Thus $0m = 0 = m0$, and $P(0)$ holds.

\medskip
\noindent\textbf{Inductive step.}
Assume $P(n)$ holds, i.e. $\forall m,\; nm = mn$.
We prove $P(n^{++})$, i.e. $\forall m,\; (n^{++})m = m(n^{++})$.

Let $m \in \mathbb{N}$ be arbitrary.
By the recursive definition of multiplication,
\[
(n^{++})m = nm + m.
\]
By the inductive hypothesis $nm = mn$, so
\[
(n^{++})m = mn + m.
\]
By Lemma~\ref{lem:right-successor-mult},
\[
m(n^{++}) = mn + m.
\]
Therefore $(n^{++})m = m(n^{++})$.
Since $m$ was arbitrary, $P(n^{++})$ holds.

\medskip
By induction, $P(n)$ holds for all $n \in \mathbb{N}$, i.e. $nm=mn$ for all $n,m \in \mathbb{N}$.
\end{proof}



\begin{consequence*}[Symmetry of products]
For all $n,m \in \mathbb{N}$, the product $nm$ may be freely rewritten as $mn$ without
changing its value.
\end{consequence*}
\vspace{1em}
