% =========================================================
% Breadcrumb
% =========================================================
\begin{tcolorbox}[
  colback=gray!6,
  colframe=gray!40,
  arc=2pt,
  left=8pt, right=8pt, top=6pt, bottom=6pt,
  title={\small\textbf{Where You Are in the Journey}},
  fonttitle=\small\bfseries
]
\begin{center}
\small
Propositional Logic
$\;\to\;$ Predicate Calculus
$\;\to\;$ Sets \& Functions
$\;\to\;$ Proof Techniques
$\;\to\;$ \textbf{Axiom Systems}
$\;\to\;$ Natural Numbers ($\mathbb{N}$)
$\;\to\;$ $\mathbb{Z}$, $\mathbb{Q}$, $\mathbb{R}$
$\;\to\;$ $\cdots$
\end{center}

\medskip
\noindent\textbf{How we got here.}
Proof techniques gave us the tools to reason rigorously --- direct proof,
contradiction, induction.
We now turn those tools on mathematics itself: can we build the natural
numbers from nothing but logical axioms?
This chapter answers yes.

\medskip
\noindent\textbf{What this chapter builds.}
We lay down the five Peano axioms, which pin down the structure
of zero and the successor function.
We verify that small numerals ($1, 2, 3, \ldots$) exist, confirm
that the axioms rule out pathological models, and prove the
Recursion Theorem --- the logical guarantee that recursive
definitions actually produce well-defined functions.
These are the \emph{logical primitives}: the minimum needed for
$\mathbb{N}$ to exist.

\medskip
\noindent\textbf{Where this leads.}
Volume~II, Natural Numbers takes over immediately: it uses the Peano
axioms and the Recursion Theorem as its foundation and builds the full
arithmetic of $\mathbb{N}$ --- addition, order, multiplication, and
exponentiation --- from scratch.
Nothing from that development is assumed here.
\end{tcolorbox}
\vspace{1em}

% =========================================================
% Structural Roadmap
% =========================================================
\subsection*{Structural Roadmap}

This chapter covers the \emph{logical skeleton} of $\mathbb{N}$ only.
Arithmetic (addition, order, multiplication, exponentiation) lives in
\textbf{Volume~II, Natural Numbers}.

\begin{center}
\textbf{Axioms $\longrightarrow$ Numerals
$\longrightarrow$ Induction principle $\longrightarrow$ Recursion theorem}
\end{center}

The global progression is:
\begin{enumerate}
  \item \textbf{Peano axioms} (P1--P5): the five axioms that characterise
        $\mathbb{N}$ up to isomorphism. Includes the induction axiom
        and a discussion of what the axioms exclude.
  \item \textbf{Numerals}: defining $1 := 0\pp$, $2 := 1\pp$, \ldots{}
        and verifying they are distinct natural numbers.
  \item \textbf{Recursion theorem} (Tao Prop.~2.1.16):
        for any set $X$, base value $a \in X$, and function $f : X \to X$,
        there exists a unique $g : \mathbb{N} \to X$ satisfying
        $g(0) = a$ and $g(n\pp) = f(g(n))$.
        This theorem is what makes recursive \emph{definitions} legitimate.
\end{enumerate}

\vspace{1em}

\begin{remark}[Scope boundary]
The Peano axioms tell us $\mathbb{N}$ \emph{exists} and has the right
logical shape. The Recursion Theorem tells us recursive definitions
are \emph{valid}. Everything else --- what $\mathbb{N}$ can \emph{do}
arithmetically --- is in Volume~II.
\end{remark}

\begin{remark}[Primary source]
Tao, \textit{Analysis~I}, Chapter~2, §2.1.
\end{remark}

% =========================================================
% Content
% =========================================================
% =========================================================
% Peano Axioms and Construction of N
% =========================================================
\subsubsection{The Peano Axioms}

% =========================================================
% The Peano Axioms (Tao §2.1)
% =========================================================

\begin{tcolorbox}[colback=propbox, colframe=propborder, arc=2pt,
  left=6pt, right=6pt, top=4pt, bottom=4pt,
  title={\small\textbf{The Peano Axioms}},
  fonttitle=\small\bfseries]
\begin{enumerate}
  \item[\textbf{P1.}] $0 \in \mathbb{N}$ \quad (zero is a natural number)
  \item[\textbf{P2.}] $n \in \mathbb{N} \Rightarrow n\pp \in \mathbb{N}$
    \quad (closure under successor)
  \item[\textbf{P3.}] $n\pp \neq 0$ for all $n \in \mathbb{N}$
    \quad (zero is not a successor)
  \item[\textbf{P4.}] $n\pp = m\pp \Rightarrow n = m$
    \quad (successor is injective)
  \item[\textbf{P5.}] \emph{Induction}: if $P(0)$ is true and
    $P(n) \Rightarrow P(n\pp)$ for all $n \in \mathbb{N}$,
    then $P(n)$ is true for all $n \in \mathbb{N}$.
\end{enumerate}
\end{tcolorbox}

\begin{remark}[Why five axioms?]
P1--P2 build $\mathbb{N}$ upward from $0$.
P3 prevents wrap-around (e.g.\ $3\pp = 0$).
P4 prevents collapse (e.g.\ $4\pp = 2\pp$ with $4 \neq 2$).
P5 excludes rogue elements (e.g.\ $0.5 \in \mathbb{N}$) by
asserting $\mathbb{N}$ is the \emph{smallest} set satisfying P1--P2.
\end{remark}

\begin{remark}[P5 is second-order]
P5 quantifies over \emph{properties} $P$, not just elements.
This makes it an axiom schema (one axiom per property $P$),
not a single first-order axiom.
\end{remark}

\begin{remark}[Axioms define structure, not objects]
The Peano axioms do not say what $0$ or $n\pp$ \emph{are} ---
they say what properties they \emph{have}.
$0$ could be the empty set $\varnothing$, or the singleton
$\{\varnothing\}$, or an abstract symbol with no set-theoretic
content. None of this matters. What matters is that the five
axioms are satisfied. Any two systems satisfying P1--P5 are
isomorphic: there is a unique structure-preserving bijection
between them. This is why we speak of \emph{the} natural numbers
rather than \emph{a} natural number system --- all valid models
are interchangeable for mathematical purposes.
This is what it means for an axiom system to be
\emph{categorical} (in second-order logic).
In first-order logic the situation is different: first-order
Peano arithmetic admits nonstandard models containing
``infinite'' natural numbers not reachable from $0$ by
successors. Second-order induction (P5 as stated) rules
these out.
\end{remark}

\begin{remark}[Tao's notation]
Tao writes $n\pp$ for the successor of $n$ (increment), following
computer-language convention. We adopt this notation throughout.
$n\pp$ and $S(n)$ refer to the same object.
\end{remark}
% =========================================================
% Numerals and Basic Definitions (Tao §2.1)
% =========================================================

\begin{definition}[Standard numerals]
\label{def:numerals}
We define:
\[
1 := 0\pp, \quad
2 := 1\pp = (0\pp)\pp, \quad
3 := 2\pp, \quad \text{etc.}
\]
In general, each standard numeral is the successor of the previous one.
\end{definition}

\begin{proposition}[Tao 2.1.4]
\label{prop:tao-2-1-4}
$3$ is a natural number.
\end{proposition}

\begin{remark}[Proof]
See \hyperref[prf:tao-2-1-4]{Proof $\to$ Tao 2.1.4 --- $3$ is a natural number}.
\end{remark}


\begin{proposition}[Tao 2.1.6 — $4 \neq 0$]
\label{prop:tao-2-1-6}
$4$ is not equal to $0$.
\end{proposition}

\begin{remark}[Proof]
See \hyperref[prf:tao-2-1-6]{Proof $\to$ Tao 2.1.6 --- $4 \neq 0$}.
\end{remark}

\begin{remark}[Why prove $4 \neq 0$?]
Without P3, a wrap-around system where $3\pp = 0$ (like a 4-clock)
satisfies P1 and P2. P3 rules this out, and the proof above shows
P3 doing its work explicitly.
\end{remark}

\begin{proposition}[Tao 2.1.8 — $6 \neq 2$]
\label{prop:tao-2-1-8}
$6$ is not equal to $2$.
\end{proposition}

\begin{remark}[Proof]
See \hyperref[prf:tao-2-1-8]{Proof $\to$ Tao 2.1.8 --- $6 \neq 2$}.
\end{remark}

\begin{remark}[Pattern]
Distinctness proofs always proceed by descending through P4 until
reaching $4 \neq 0$ (established via P3). P3 provides the
``ground'' that stops the descent.
\end{remark}

% =========================================================
% Recursive Definitions (Tao §2.1, Prop. 2.1.16)
% =========================================================

\begin{tcolorbox}[colback=propbox, colframe=propborder, arc=2pt,
  left=6pt, right=6pt, top=4pt, bottom=4pt,
  title={\small\textbf{Theorem (Recursion on $\mathbb{N}$)}},
  fonttitle=\small\bfseries]
Let $X$ be a set, let $a \in X$, and let $f : X \to X$ be a function.
Then there exists a \textbf{unique} function $g : \mathbb{N} \to X$ such that:
\[
g(0) = a
\quad \text{and} \quad
g(n\pp) = f(g(n)) \quad \text{for all } n \in \mathbb{N}.
\]
\end{tcolorbox}

\begin{remark}[Reading the theorem]
The theorem has two parts.
\emph{Existence}: the recursive rule actually defines a function ---
there are no gaps (every $n$ gets a value) and no contradictions
(no $n$ gets two different values).
\emph{Uniqueness}: there is only one such function. Two functions
satisfying the same recursive rule must agree everywhere, proved
by induction on $n$.
\end{remark}

\begin{remark}[Tao's version]
Tao states this as Proposition~2.1.16 in a slightly more general
form: instead of a single function $f : X \to X$, he allows a
family $f_n : X \to X$ indexed by $n$. The version above is the
standard form and suffices for all applications here.
\end{remark}

\begin{remark}[Why all five axioms are needed]
\begin{itemize}
  \item P1 anchors the recursion: $g(0) = a$ is well-defined.
  \item P2 guarantees $g(n\pp)$ is always reachable from $g(n)$.
  \item P3 ensures $g(0)$ is never overwritten by a step $g(n\pp)$,
    since $n\pp \neq 0$ for all $n$.
  \item P4 ensures no two steps conflict: if $m \neq n$ then
    $m\pp \neq n\pp$, so $g(m\pp)$ and $g(n\pp)$ are defined
    independently.
  \item P5 (induction) proves $g$ is defined at \emph{every}
    $n \in \mathbb{N}$, not just the ones we can reach by hand.
\end{itemize}
Remove any one axiom and the construction breaks.
\end{remark}

\begin{remark}[Recursive definitions as the engine of arithmetic]
Every arithmetic operation on $\mathbb{N}$ is an instance of this
theorem with $X = \mathbb{N}$:
\begin{itemize}
  \item \textbf{Addition} ($+\, m$): $g(0) = m$, $f(x) = x\pp$.
    This gives $g(n) = n + m$.
  \item \textbf{Multiplication} ($\times\, m$): $g(0) = 0$,
    $f(x) = x + m$. This gives $g(n) = n \times m$.
  \item \textbf{Exponentiation} ($m^{\,\cdot}$): $g(0) = 1$,
    $f(x) = x \times m$. This gives $g(n) = m^n$.
\end{itemize}
Each new operation is defined in terms of the previous one.
The recursion theorem guarantees all three are well-defined and unique.
\end{remark}

