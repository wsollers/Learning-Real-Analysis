% =========================================================
% Proof Techniques — Notes Assembly
% =========================================================
% =========================================================
% Tao Analysis I — Chapter 2 Reference Sheet
% Available axioms, definitions, and lemmas
% =========================================================

\subsection*{Peano Axioms}

\begin{itemize}
  \item[\textbf{P1.}] $0 \in \mathbb{N}$
  \item[\textbf{P2.}] $n \in \mathbb{N} \Rightarrow n\pp \in \mathbb{N}$
  \item[\textbf{P3.}] $n\pp \neq 0$ for all $n \in \mathbb{N}$
  \item[\textbf{P4.}] $n\pp = m\pp \Rightarrow n = m$ \quad (successor is injective)
  \item[\textbf{P5.}] \emph{Induction axiom}: if $P(0)$ holds and
    $P(n) \Rightarrow P(n\pp)$ for all $n \in \mathbb{N}$,
    then $P(n)$ holds for all $n \in \mathbb{N}$.
\end{itemize}

\subsection*{Definition of Addition (Recursive)}

\begin{itemize}
  \item[\textbf{A1.}] $0 + m := m$
  \item[\textbf{A2.}] $(n\pp) + m := (n + m)\pp$
\end{itemize}

\noindent\textit{Note}: recursion is on the \textbf{left} argument only.
$n + m$ is defined by recursing on $n$, not $m$.

\subsection*{Available Results}

\begin{itemize}
  \item[\textbf{L2.2.2.}] $n + 0 = n$ \quad for all $n \in \mathbb{N}$
  \item[\textbf{L2.2.3.}] $n + (m\pp) = (n + m)\pp$ \quad
    for all $n, m \in \mathbb{N}$
  \item[\textbf{P2.2.4.}] $n + m = m + n$ \quad
    for all $n, m \in \mathbb{N}$ \quad (commutativity)
\end{itemize}





\begin{tcolorbox}[
  colback=gray!6, colframe=gray!40, arc=2pt,
  left=8pt, right=8pt, top=6pt, bottom=6pt,
  title={\small\textbf{Where You Are in the Journey}},
  fonttitle=\small\bfseries
]
\begin{center}
\small
Propositional Logic $\;\to\;$ Predicate Calculus
$\;\to\;$ Sets \& Functions
$\;\to\;$ \textbf{Proof Techniques}
$\;\to\;$ Real Analysis $\;\to\;$ Algebraic Structures
$\;\to\;$ Linear Algebra $\;\to\;$ $\cdots$
\end{center}
\medskip
\noindent\textbf{How we got here.}
Logic gave us the language of valid inference and sets gave us the
objects we reason about. This chapter asks the operational question:
how do we build a valid mathematical argument from start to finish?

\medskip
\noindent\textbf{What this chapter builds.}
A complete architecture for proof writing: how to read a statement
and identify its proof strategy immediately; how to execute that
strategy step by step; how to use induction to reason about
well-ordered and recursive structures; and how to deploy specific
algebraic tactics when progress stalls.

\medskip
\noindent\textbf{Where this leads.}
Every proof in every subsequent chapter uses this toolkit.
Epsilon-delta analysis uses direct proof and contradiction.
Algebraic structure proofs use the uniqueness and
satisfy-and-cite patterns. Induction appears throughout
number theory, algebra, and combinatorics.
\end{tcolorbox}
\vspace{1em}

\subsection*{Structural Roadmap}

This chapter is organized in six sections of increasing specificity.

\begin{center}
\textbf{Architecture $\longrightarrow$ Construction
$\longrightarrow$ Structures $\longrightarrow$ Induction
$\longrightarrow$ Tactics $\longrightarrow$ Reference}
\end{center}

Sections 1--4 should be read sequentially on first pass.
Sections 5--6 are reference material to consult when working a
proof and progress stalls.

\begin{remark}[Four distinct questions]
\textbf{Architecture} answers: what kind of proof is this and what
skeleton should it have?
\textbf{Construction} answers: how do I execute that skeleton
line by line?
\textbf{Structures} answers: what do the standard proof patterns
look like in practice?
\textbf{Tactics} answers: once a proof is in progress,
what specific moves are available?
These are four different questions. Conflating them is the most
common source of proof-writing paralysis.
\end{remark}

% =========================================================
% Section 1: Proof Architecture
% =========================================================
\subsubsection{Proof Architecture}
\label{sec:proof-architecture}

% =========================================================
% The Two Questions
% =========================================================
\paragraph{The Two Questions Before You Write Anything.}

Before writing a single symbol, ask two questions in order.

\begin{tcolorbox}[colback=propbox, colframe=propborder, arc=2pt,
  left=6pt, right=6pt, top=4pt, bottom=4pt]
\textbf{Question 1.} \emph{What is the logical form of the statement?}

Read the claim and identify its outermost logical structure:
universal, conditional, biconditional, existential, uniqueness,
equality, or structural assertion.

\medskip
\textbf{Question 2.} \emph{What proof strategy does this form suggest?}

The form determines the skeleton of the proof mechanically,
before any mathematics is done.
\end{tcolorbox}

\begin{remark}[Why order matters]
Question 2 cannot be answered before Question 1.
The most common proof error is reaching for a familiar strategy
--- contradiction is the usual default --- without checking
whether it matches the statement form. Read the statement first.
The form is not preliminary; it is the first mathematical act
of the proof.
\end{remark}

\begin{example}[Applying the two questions]
Statement: \emph{The identity element of a group is unique.}

\medskip
\noindent\textbf{Q1.} Logical form:
$\forall G$ (group), $\forall e, e' \in G$,
[both satisfy the identity axiom] $\Rightarrow e = e'$.
This is a universal statement with a uniqueness conclusion.

\medskip
\noindent\textbf{Q2.} Strategy: introduce an arbitrary group $G$,
assume two identity elements $e$ and $e'$ exist, show $e = e'$.
This is the \emph{assume-two-and-compare} pattern
(see Section~\ref{sec:proof-structures}).

\medskip
The strategy is forced entirely by the form.
No creative insight is required to select it.
\end{example}

% =========================================================
% Statement Forms → Proof Strategies Map
% =========================================================
\paragraph{Statement Forms and Their Proof Strategies.}

The following table maps every common statement form to the proof
strategy it demands. This table should be consulted at the start
of every proof until the correspondences are automatic.

\begin{center}
\renewcommand{\arraystretch}{1.4}
\begin{tabular}{p{0.30\textwidth} p{0.30\textwidth} p{0.30\textwidth}}
\toprule
\textbf{Statement Form} & \textbf{Proof Strategy} & \textbf{Opening Move} \\
\midrule
$\forall x,\; P(x)$ &
  Introduce arbitrary element &
  ``Let $x$ be arbitrary.'' \\

$P \Rightarrow Q$ &
  Direct proof &
  ``Assume $P$.'' \\

$P \Rightarrow Q$ &
  Contrapositive &
  ``Assume $\neg Q$.'' \\

$\neg Q \Rightarrow \neg P$\newline (same as above) &
  Contradiction &
  ``Assume $P$ and $\neg Q$.'' \\

$P \Leftrightarrow Q$ &
  Two directions &
  Prove $P \Rightarrow Q$, then $Q \Rightarrow P$. \\

$\exists x,\; P(x)$ &
  Construct a witness &
  ``Define $x := \ldots$ and verify $P(x)$.'' \\

$\exists! x,\; P(x)$ &
  Existence + uniqueness &
  Construct witness; then assume-two-and-compare. \\

$A = B$ (sets) &
  Double inclusion &
  Prove $A \subseteq B$ and $B \subseteq A$. \\

$a = b$ (elements) &
  Satisfy-and-cite or algebra &
  Show $a$ satisfies the defining property of $b$. \\

$P(n)$ for all $n \in \mathbb{N}$ &
  Induction &
  Base case; inductive step. \\

``$X$ is unique'' &
  Assume-two-and-compare &
  ``Let $x, y$ both satisfy the definition.'' \\
\bottomrule
\end{tabular}
\end{center}

\begin{remark}[Multiple strategies for one form]
Some statement forms admit more than one strategy.
$P \Rightarrow Q$ can be proved directly, by contrapositive,
or by contradiction. The table lists the most natural default.
Direct proof should be attempted first; contrapositive and
contradiction are reached for when the hypothesis is hard to
use forward or when the negation of the conclusion is
more tractable.
\end{remark}

\begin{remark}[The equality forms are different]
Notice that $A = B$ (set equality) and $a = b$ (element equality)
suggest different strategies. Set equality uses double inclusion
because sets are defined by their members.
Element equality in an algebraic context uses satisfy-and-cite:
show that $a$ satisfies the same defining property as $b$,
then invoke a previously proved uniqueness theorem.
See Section~\ref{sec:satisfy-cite}.
\end{remark}

% =========================================================
% The Five Archetypes
% =========================================================
\paragraph{The Five Proof Archetypes.}

At the highest level of compression, every mathematical proof
reduces to one of five archetypes. Everything else is refinement
of these five.

\begin{tcolorbox}[colback=propbox, colframe=propborder, arc=2pt,
  left=6pt, right=6pt, top=4pt, bottom=4pt,
  title={\small\textbf{The Five Archetypes}}, fonttitle=\small\bfseries]
\begin{enumerate}
  \item \textbf{Construct something.}
    Exhibit a concrete object and verify it has the required property.
    Used for all existence claims.

  \item \textbf{Assume two and compare.}
    Assume two objects both satisfy a definition; derive that they are equal.
    Used for all uniqueness claims.

  \item \textbf{Take a minimal element.}
    Apply the well-ordering principle to a nonempty set;
    exploit the minimality of the least element.
    Used in number theory, induction, and contradiction arguments.

  \item \textbf{Induct.}
    Verify a base case; show the property propagates to successors.
    Used whenever the claim concerns all natural numbers or a
    recursively defined structure.

  \item \textbf{Chase an arbitrary element.}
    Take an arbitrary element of one set or satisfying one property;
    track it through definitions to land in the target.
    Used for set inclusions, function properties, and structural claims. Also may be used for direct proofs when the logic flows in a straight line.
\end{enumerate}
\end{tcolorbox}

\begin{remark}[Using the archetypes]
When starting a proof, identify which archetype applies before
consulting the detailed strategy map. The archetype tells you the
global shape of the argument. The statement map tells you
the specific opening move. The construction algorithm
(Section~\ref{sec:construction}) tells you how to fill in each line.
\end{remark}

\begin{remark}[Archetypes can nest]
A single proof may use more than one archetype. A common pattern
in algebra is to construct a witness (archetype 1) and then
invoke a uniqueness theorem (archetype 2) to conclude that the
constructed object equals a previously named one.
This nesting is the \emph{satisfy-and-cite} tactic developed
in Section~\ref{sec:satisfy-cite}.
\end{remark}

% =========================================================
% Satisfy-and-Cite Uniqueness
% =========================================================
\paragraph{The Satisfy-and-Cite Pattern.}
\label{sec:satisfy-cite}

The most commonly missing pattern at the undergraduate level is one
that does not appear in any of the standard proof archetypes by name,
yet appears constantly in abstract algebra and analysis.

\begin{definition}[Satisfy-and-Cite Pattern]
To prove that $a = b$ in a context where $b$ is a \emph{uniquely
determined} object (the unique identity, the unique inverse, the
unique limit, the unique fixed point), proceed as follows:
\begin{enumerate}
  \item Show that $a$ satisfies the defining property of $b$.
  \item Cite the previously proved uniqueness theorem for $b$.
  \item Conclude $a = b$.
\end{enumerate}
\end{definition}

\begin{remark}[Why this is different from assume-two-and-compare]
The assume-two-and-compare pattern \emph{proves} uniqueness:
it shows that any two objects satisfying a definition are equal.

The satisfy-and-cite pattern \emph{uses} a previously proved
uniqueness result: it shows that a specific object $a$ equals
a specific named object $b$ by showing $a$ satisfies $b$'s
defining property, then invoking uniqueness to collapse $a$ to $b$.

Assume-two-and-compare appears in the proof that the identity is unique.
Satisfy-and-cite appears in every proof that uses that uniqueness
theorem afterward.
\end{remark}

\begin{example}[Socks-shoes property: $(ab)^{-1} = b^{-1}a^{-1}$]
We want to show $(ab)^{-1} = b^{-1}a^{-1}$.

The inverse of $ab$ is the unique element $x$ such that
$(ab)x = e$ and $x(ab) = e$.

We compute:
\[
(ab)(b^{-1}a^{-1})
= a(bb^{-1})a^{-1}
= a e a^{-1}
= aa^{-1} = e.
\]
Similarly $(b^{-1}a^{-1})(ab) = e$.

So $b^{-1}a^{-1}$ \emph{satisfies the defining property of
$(ab)^{-1}$}. By uniqueness of inverses (previously proved),
$(ab)^{-1} = b^{-1}a^{-1}$. \qed

\medskip
\noindent Notice: we never assumed two inverses existed and compared
them. We showed one specific element satisfies the inverse definition,
then cited uniqueness.
\end{example}

\begin{example}[Ring theorem: $a \cdot (-b) = -(ab)$]
We want to show $a(-b) = -(ab)$.

The additive inverse $-(ab)$ is the unique element $x$ such that
$ab + x = 0$.

We compute:
\[
ab + a(-b) = a(b + (-b)) = a \cdot 0 = 0.
\]
So $a(-b)$ \emph{satisfies the defining property of $-(ab)$}.
By uniqueness of additive inverses, $a(-b) = -(ab)$. \qed

\medskip
\noindent Same pattern: compute, satisfy, cite.
\end{example}

\begin{remark}[The pattern across mathematics]
Satisfy-and-cite appears wherever unique objects are defined:
\begin{itemize}
  \item Algebra: inverses, identities, quotients, kernels
  \item Analysis: limits, suprema, infima, fixed points
  \item Linear algebra: zero vector, additive inverses,
    linear transformations determined by a basis
\end{itemize}
Each time a uniqueness theorem is proved, it creates a new
tool that can be activated by the satisfy-and-cite pattern
for every future proof in that structure.
\end{remark}


% =========================================================
% Section 2: Proof Construction Algorithm
% =========================================================
\subsubsection{The Proof Construction Algorithm}
\label{sec:construction}

% =========================================================
% Proof Construction Algorithm
% (Preserved from original proof-techniques notes)
% =========================================================

\begin{remark}[Purpose of this section]
This section records a general proof-writing procedure intended
to be applied consciously until it becomes internalized.
It answers the question: once the architecture is chosen
(Section~\ref{sec:proof-architecture}), how is the argument
executed line by line?
\end{remark}

\paragraph{Step 0: Classify the Statement.}

Determine the logical form of the claim. Common forms include:
\begin{itemize}
  \item Universal: $\forall x\; P(x)$
  \item Conditional: $P \rightarrow Q$
  \item Biconditional: $P \leftrightarrow Q$
  \item Existential: $\exists x\; P(x)$
  \item Equality: $X = Y$
  \item Set equality: $A = B$
  \item Structural: ``$R$ is an equivalence relation'',
    ``$f$ is injective''
\end{itemize}
The logical form determines the overall structure of the proof.

\paragraph{Step 1: Restate the Givens.}

Explicitly record what is given. Introduce all sets, relations,
and functions, and note any structural properties assumed.

\begin{quote}
Let $A$ be a set and let $R$ be an equivalence relation on $A$.
\end{quote}

This licenses later use of reflexivity, symmetry, and transitivity
without reintroducing them each time.

\paragraph{Step 2: Identify Objects and Their Types.}

Before reasoning begins, identify the \emph{type} of each object:
is it an element, a set, a function, or a relation?
What is its ambient universe?

\[
x \in A,\quad f : A \to B,\quad R \subseteq A \times A.
\]

Many logical errors arise from confusing equality, membership,
and inclusion. Every object used in a proof must have a declared type.

\paragraph{Step 3: Introduce Arbitrary Elements.}

If the claim is universal, immediately introduce arbitrary elements.

\[
\text{Let } a, b \in A \text{ be arbitrary.}
\]

This step enables general reasoning and avoids illegal specialization.

\paragraph{Step 4: Expand the Goal.}

Rewrite the conclusion using definitions rather than named concepts.
\begin{itemize}
  \item To prove $f$ injective: expand the definition of injectivity.
  \item To prove $A = B$: prove $A \subseteq B$ and $B \subseteq A$.
  \item To prove $x \in A \cup B$: rewrite as $x \in A \lor x \in B$.
\end{itemize}
Never attempt to prove a named concept directly without first
expanding its definition.

\paragraph{Step 5: Introduce Helper Objects.}

Introduce auxiliary objects needed for the argument: witnesses,
intermediate elements, bounds, images under functions.

\[
\text{Let } x \in A,\quad
\text{let } P_a \text{ be the block containing } a,\quad
\text{let } b := f(a).
\]

No object should appear in a proof without being explicitly introduced.

\paragraph{Step 6: Apply One Definition or Property at a Time.}

Each step follows from exactly one of:
\begin{itemize}
  \item a definition,
  \item a hypothesis,
  \item a previously proved theorem,
  \item a basic logical rule.
\end{itemize}

If progress stalls, identify which definition has not yet been unpacked.

\paragraph{Step 7: Use Forward and Backward Reasoning.}

Proofs often alternate between two modes:
\begin{itemize}
  \item \emph{Forward}: deduce consequences from the hypotheses.
  \item \emph{Backward}: rewrite the goal to determine what
    would suffice to prove it.
\end{itemize}
Backward reasoning is effective when the conclusion involves
nested definitions.

\paragraph{Step 8: Handle Cases Explicitly.}

If a statement splits into cases, enumerate and exhaust all of them.

\[
\text{Either } x \in A \text{ or } x \notin A.
\]

Each case is treated separately; together they cover all possibilities.

\paragraph{Step 9: Close the Argument.}

Once the desired conclusion is reached, state it explicitly.
\begin{itemize}
  \item ``Thus $x \in B$, so $A \subseteq B$.''
  \item ``Hence $f$ is injective.''
  \item ``Therefore the two sets are equal.''
\end{itemize}

\paragraph{Step 10: Signal Completion.}

Conclude with $\qed$ or an explicit statement that the proof is complete.

\bigskip

\paragraph{Legal Moves.}

\begin{itemize}
  \item An existential witness may not be chosen before existence is proved.
  \item An arbitrary element may only be introduced under universal scope.
  \item The conclusion may never be assumed.
  \item Definitions must be applied in full, not partially.
\end{itemize}

\paragraph{Line-by-Line Discipline.}

For each line, silently check:
\begin{enumerate}[label=(\roman*)]
  \item Has every object in this line been defined?
  \item Which definition, hypothesis, or theorem justifies this step?
  \item Does this step move the argument closer to the stated goal?
\end{enumerate}

\paragraph{Stop Conditions.}

A proof is complete when:
\begin{itemize}
  \item a universal claim has been shown for an arbitrary element;
  \item an existential claim has produced a valid witness;
  \item a set equality has established both inclusions;
  \item each direction of a biconditional has been proved.
\end{itemize}

\begin{remark}[On internalizing the algorithm]
This procedure is intentionally explicit and mechanical.
With practice, these steps become automatic and are applied
subconsciously. Experienced mathematicians follow the same process
but omit intermediate steps once correctness is assured.
Until fluency is achieved, longer proofs with explicit steps
are preferable to shorter proofs with implicit gaps.
\end{remark}


% =========================================================
% Section 3: Proof Structures
% =========================================================
\subsubsection{Proof Structures}
\label{sec:proof-structures}

% =========================================================
% Direct Proof
% =========================================================
\paragraph{Direct Proof.}

A direct proof of $P \Rightarrow Q$ assumes $P$ and derives $Q$
through a chain of valid inferences.

\begin{tcolorbox}[colback=propbox, colframe=propborder, arc=2pt,
  left=6pt, right=6pt, top=4pt, bottom=4pt,
  title={\small\textbf{Template: Direct Proof}}, fonttitle=\small\bfseries]
\textbf{Given.} $P$.\\
\textbf{Goal.} $Q$.\\[4pt]
Assume $P$.\\
$\vdots$ \quad [expand definitions; derive consequences]\\
Therefore $Q$. \qed
\end{tcolorbox}

\begin{remark}[When to use direct proof]
Direct proof is the default. Reach for it first.
It is appropriate whenever the hypothesis gives you something
concrete to work with and the conclusion follows by expanding
definitions and applying theorems.
Reserve contrapositive and contradiction for cases where the
hypothesis is hard to use forward.
\end{remark}

\begin{example}[Direct proof in a group]
\textbf{Claim.} In a group $G$, if $a^2 = e$ then $a = a^{-1}$.

\medskip
\noindent\textit{Proof.}
Assume $a^2 = e$, i.e., $aa = e$.
Multiply both sides on the right by $a^{-1}$:
\[
aa \cdot a^{-1} = e \cdot a^{-1}.
\]
By associativity, $a(aa^{-1}) = a^{-1}$, so $ae = a^{-1}$,
so $a = a^{-1}$. \qed
\end{example}

% =========================================================
% Contrapositive
% =========================================================
\paragraph{Proof by Contrapositive.}

The contrapositive of $P \Rightarrow Q$ is $\neg Q \Rightarrow \neg P$.
These are logically equivalent. A proof by contrapositive proves
$\neg Q \Rightarrow \neg P$ directly.

\begin{tcolorbox}[colback=propbox, colframe=propborder, arc=2pt,
  left=6pt, right=6pt, top=4pt, bottom=4pt,
  title={\small\textbf{Template: Contrapositive}}, fonttitle=\small\bfseries]
\textbf{Goal.} $P \Rightarrow Q$.\\
\textbf{Equivalent goal.} $\neg Q \Rightarrow \neg P$.\\[4pt]
Assume $\neg Q$.\\
$\vdots$ \quad [derive $\neg P$]\\
Therefore $\neg P$.
Hence $P \Rightarrow Q$. \qed
\end{tcolorbox}

\begin{remark}[When to use contrapositive]
Use contrapositive when the negation of $Q$ is more informative
or easier to manipulate than $P$ itself.
Injectivity proofs are a canonical example: to prove
$f(a) = f(b) \Rightarrow a = b$, it is often cleaner to prove
the contrapositive $a \neq b \Rightarrow f(a) \neq f(b)$.
\end{remark}

\begin{remark}[Contrapositive vs contradiction]
Contrapositive: assume $\neg Q$, derive $\neg P$, done.
Contradiction: assume both $P$ and $\neg Q$, derive any
absurdity, done.
Contrapositive is cleaner when it works because the assumption
is $\neg Q$ alone, not $P \land \neg Q$.
Try contrapositive before reaching for contradiction.
\end{remark}

\begin{example}[Contrapositive for divisibility]
\textbf{Claim.} If $n^2$ is even then $n$ is even.

\medskip
\noindent\textit{Proof (contrapositive).}
Assume $n$ is odd. Then $n = 2k+1$ for some $k \in \mathbb{Z}$.
\[
n^2 = (2k+1)^2 = 4k^2 + 4k + 1 = 2(2k^2+2k)+1,
\]
which is odd. Hence $n$ odd $\Rightarrow$ $n^2$ odd, i.e.,
$n^2$ even $\Rightarrow$ $n$ even. \qed
\end{example}

% =========================================================
% Proof by Contradiction
% =========================================================
\paragraph{Proof by Contradiction.}

To prove $P$, assume $\neg P$ and derive a statement that
contradicts a known truth: a hypothesis, an axiom, or a
previously proved theorem.

\begin{tcolorbox}[colback=propbox, colframe=propborder, arc=2pt,
  left=6pt, right=6pt, top=4pt, bottom=4pt,
  title={\small\textbf{Template: Contradiction}}, fonttitle=\small\bfseries]
\textbf{Goal.} $P$.\\[4pt]
Suppose for contradiction that $\neg P$.\\
$\vdots$ \quad [derive a contradiction]\\
This contradicts [name the violated fact].\\
Therefore $P$. \qed
\end{tcolorbox}

\begin{remark}[When to use contradiction]
Contradiction is best reserved for claims whose negation has
strong consequences. It is the natural tool for:
\begin{itemize}
  \item Irrationality proofs ($\neg P$ gives a fraction in lowest
    terms and leads to a parity contradiction).
  \item Infinitude of primes ($\neg P$ gives a finite list;
    constructing a number not on the list contradicts completeness).
  \item Statements involving nonexistence.
\end{itemize}
Avoid using contradiction as a default. When a direct proof or
contrapositive works, it is cleaner.
\end{remark}

\begin{remark}[Name the contradiction explicitly]
A well-written contradiction proof always names what is contradicted.
``This contradicts [G2: identity axiom]'' is better than
``contradiction.'' Naming the violated fact confirms that the
contradiction is genuine and locates the logical break.
\end{remark}

\begin{example}[$\sqrt{2}$ is irrational]
\textit{Proof.}
Suppose for contradiction that $\sqrt{2} \in \mathbb{Q}$.
Write $\sqrt{2} = p/q$ with $p, q \in \mathbb{Z}$, $q \neq 0$,
and $\gcd(p,q) = 1$.
Then $2q^2 = p^2$, so $p^2$ is even, so $p$ is even (write $p = 2k$).
Then $2q^2 = 4k^2$, so $q^2 = 2k^2$, so $q$ is even.
But then $\gcd(p,q) \geq 2$, contradicting $\gcd(p,q) = 1$. \qed
\end{example}

% =========================================================
% Case Analysis
% =========================================================
\paragraph{Proof by Cases.}

If the hypotheses or conclusion naturally divide into exhaustive
and mutually exclusive alternatives, handle each case separately.

\begin{tcolorbox}[colback=propbox, colframe=propborder, arc=2pt,
  left=6pt, right=6pt, top=4pt, bottom=4pt,
  title={\small\textbf{Template: Proof by Cases}}, fonttitle=\small\bfseries]
\textbf{Goal:} Prove $Q$, given that $P_1 \lor P_2 \lor \cdots \lor P_k$
is exhaustive.

\medskip
\textit{Verify exhaustiveness:} Every possibility falls under some $P_i$.

\medskip
\textbf{Case 1:} Assume $P_1$. \quad [Argue.] \quad Hence $Q$.\\[4pt]
\textbf{Case 2:} Assume $P_2$. \quad [Argue.] \quad Hence $Q$.\\[4pt]
\hspace{1.2em}$\vdots$\\[4pt]
\textbf{Case $k$:} Assume $P_k$. \quad [Argue.] \quad Hence $Q$.

\medskip
Since $P_1 \lor P_2 \lor \cdots \lor P_k$ is exhaustive and $Q$
holds in each case, $Q$ holds. \qed
\end{tcolorbox}

\begin{remark}[The exhaustiveness obligation]
A case proof is only valid if the cases together cover every
possibility. Always verify this. Common exhaustive splits:
\begin{itemize}
  \item $n$ even or $n$ odd
  \item $a > 0$, $a = 0$, or $a < 0$
  \item $a \mid b$ or $a \nmid b$
  \item $x \in A$ or $x \notin A$
\end{itemize}
\end{remark}

\begin{remark}[Cases can reduce to earlier cases]
In a two-case proof, if Case 2 is symmetric to Case 1, it is
acceptable to write: ``Case 2 is symmetric.'' But only when
the symmetry is genuinely complete --- when the argument for
Case 2 is obtained from Case 1 by renaming variables.
\end{remark}

\begin{example}[Parity of $n(n+1)$]
\textbf{Claim.} For every $n \in \mathbb{Z}$,
$n(n+1)$ is even.

\medskip
\noindent\textit{Proof.}
\textbf{Case 1:} $n$ is even. Write $n = 2k$. Then
$n(n+1) = 2k(n+1)$, which is even.

\textbf{Case 2:} $n$ is odd. Write $n+1 = 2k$. Then
$n(n+1) = n \cdot 2k$, which is even.

In both cases $n(n+1)$ is even. Since every integer is even
or odd, the proof is complete. \qed
\end{example}
% =========================================================
% Existence and Uniqueness Patterns
% =========================================================
\paragraph{Existence and Uniqueness.}

The statement $\exists!\, x\; P(x)$ has two parts:
existence ($\exists x\; P(x)$) and uniqueness ($\exists!\, x$).
They are proved separately.

\begin{tcolorbox}[colback=propbox, colframe=propborder, arc=2pt,
  left=6pt, right=6pt, top=4pt, bottom=4pt,
  title={\small\textbf{Template: Existence and Uniqueness ($\exists!\, x\; P(x)$)}},
  fonttitle=\small\bfseries]
\textbf{Step 1 — Existence.}\\
Define $x := [\text{explicit construction or witness}]$.\\
Verify that $P(x)$ holds.

\medskip
\textbf{Step 2 — Uniqueness (assume-two-and-compare).}\\
Let $x$ and $y$ be arbitrary objects both satisfying $P$.\\
\textit{[Apply the definition of $x$ with input $y$,
and the definition of $y$ with input $x$.
The two applications collapse.]}\\
Conclude $x = y$.

\medskip
\textbf{Step 3 — Conclude.}\\
By Steps~1 and~2, there exists exactly one object satisfying $P$. \qed
\end{tcolorbox}

\begin{remark}[Existence must come first]
The uniqueness argument assumes that at least one object satisfying
$P$ exists; otherwise ``$x$ and $y$ both satisfy $P$'' has no
content. Always prove existence before uniqueness.
\end{remark}

\begin{remark}[The key move in assume-two-and-compare]
The proof that $x = y$ almost always uses $x$ and $y$ against
each other: apply the definition of $x$ as an identity (or inverse,
or fixed point) with $y$ as the input, and simultaneously apply
the definition of $y$ with $x$ as the input.
The two applications collapse to give $x = y$.

This move appears in:
uniqueness of group identity, uniqueness of inverses,
uniqueness of the zero vector, uniqueness of limits,
uniqueness of the supremum.
\end{remark}

\begin{example}[Uniqueness of group identity]
\textit{Proof.}
Suppose $e$ and $e'$ both satisfy the identity axiom in $G$.
Then:
\[
e = e \cdot e' = e'.
\]
The first equality uses $e'$ as an identity; the second uses
$e$ as an identity. Hence $e = e'$. \qed
\end{example}

\begin{remark}[Uniqueness creates a tool]
Every uniqueness theorem, once proved, activates the
satisfy-and-cite tactic (Section~\ref{sec:satisfy-cite}).
Proving uniqueness is not just an end in itself ---
it is the creation of a new proof tool that applies to all
subsequent arguments in the same structure.
\end{remark}

% =========================================================
% Section 4: Mathematical Induction
% =========================================================
\subsubsection{Mathematical Induction}
\label{sec:induction}

% =========================================================
% Why Induction Works
% =========================================================
\paragraph{Why Induction Works.}

Induction feels like a technique. It is actually a theorem.
It follows from the axiomatic structure of $\mathbb{N}$.

\begin{remark}[The Peano axiom behind induction]
The Peano axioms characterize $\mathbb{N}$ as the smallest
inductive set: a set containing $0$ and closed under the
successor function $S$. Formally, Axiom P5 states:

\medskip
If $A \subseteq \mathbb{N}$ satisfies (i) $0 \in A$ and
(ii) $n \in A \Rightarrow S(n) \in A$, then $A = \mathbb{N}$.

\medskip
The induction principle is a direct restatement: if $P(0)$ holds
and $P(n) \Rightarrow P(S(n))$, then defining
$A = \{n \in \mathbb{N} : P(n)\}$ gives an inductive set,
hence $A = \mathbb{N}$, hence $P(n)$ holds for all $n$.
The full treatment is in the Axiom Systems chapter.
\end{remark}

\begin{remark}[Induction and well-ordering are equivalent]
The induction principle and the well-ordering principle for
$\mathbb{N}$ are logically equivalent. The well-ordering principle
states: every nonempty subset of $\mathbb{N}$ has a least element.

This equivalence means every induction argument can be rephrased
as a minimal counterexample argument, and vice versa.
Both are available as proof tools; the choice between them
is a matter of which formulation makes the argument cleaner.
The equivalence is proved in Section~\ref{sec:wellordering}.
\end{remark}

\begin{remark}[The domino picture is misleading]
The common intuition ``induction is like an infinite row of
dominoes'' is useful but imprecise. It suggests that induction
requires each domino to knock over the next, which captures the
inductive step but misses the point that the base case is what
starts the chain. More importantly, it does not explain why
the argument is valid --- it is valid because $\mathbb{N}$ is
defined to be the smallest inductive set, not because of any
physical metaphor. The domino picture is a mnemonic, not a proof.
\end{remark}

% =========================================================
% Weak Induction
% =========================================================
\begin{tcolorbox}[colback=propbox, colframe=propborder, arc=2pt,
  left=6pt, right=6pt, top=4pt, bottom=4pt,
  title={\small\textbf{Theorem: Weak Induction}},
  fonttitle=\small\bfseries]
Let $P(n)$ be a statement depending on $n \in \mathbb{N}$.
Suppose:
\begin{enumerate}
  \item \textbf{Base case.} $P(0)$ is true.
  \item \textbf{Inductive step.} For all $n \in \mathbb{N}$,
    $P(n) \Rightarrow P(n+1)$.
\end{enumerate}
Then $P(n)$ is true for all $n \in \mathbb{N}$.
\end{tcolorbox}

\begin{tcolorbox}[colback=gray!4, colframe=gray!35, arc=2pt,
  left=6pt, right=6pt, top=4pt, bottom=4pt,
  title={\small\textbf{Template: Weak Induction Proof}},
  fonttitle=\small\bfseries]
Let $P(n)$ denote the statement: [\emph{write it out explicitly}].

\medskip
\noindent\textbf{Base case} ($n = [\text{start}]$).
[\emph{Verify $P(\text{start})$ directly.}]
Hence $P(\text{start})$ holds.

\medskip
\noindent\textbf{Inductive step.}
Let $n \in \mathbb{N}$ and assume $P(n)$ holds.
[\emph{This is the inductive hypothesis --- state it explicitly.}]
We show $P(n+1)$ holds.

[\emph{Work. Use $P(n)$ at least once.}]

Hence $P(n+1)$ holds.

\medskip
\noindent By induction, $P(n)$ holds for all $n \geq [\text{start}]$. \qed
\end{tcolorbox}

\begin{remark}[The base case is not trivial]
The base case is logically necessary, not a formality.
A proof that omits it is invalid, regardless of how clean the
inductive step is. However, the base case is usually the easiest
part. Do not spend more than a line or two on it.
\end{remark}

\begin{remark}[State the inductive hypothesis explicitly]
The single most important discipline in induction proofs is
writing the inductive hypothesis as a full sentence before
the inductive step begins. ``Assume $P(n)$ holds'' is not
sufficient. Write: ``Assume $\sum_{k=1}^n k = \frac{n(n+1)}{2}$.''
This forces you to know what you are allowed to use,
and it is the step most students skip.
\end{remark}

\begin{example}[Sum formula]
\textbf{Proposition.}
For all $n \geq 1$:
$\displaystyle\sum_{k=1}^n k = \frac{n(n+1)}{2}$.

\medskip
\noindent\textit{Proof.}
Let $P(n)$ denote the statement
$\displaystyle\sum_{k=1}^n k = \frac{n(n+1)}{2}$.

\medskip
\noindent\textbf{Base case} ($n=1$).
$\displaystyle\sum_{k=1}^1 k = 1 = \frac{1 \cdot 2}{2}$.
Hence $P(1)$ holds.

\medskip
\noindent\textbf{Inductive step.}
Let $n \geq 1$ and assume
$\displaystyle\sum_{k=1}^n k = \frac{n(n+1)}{2}$. \quad
\textit{(Inductive hypothesis.)}

We show $\displaystyle\sum_{k=1}^{n+1} k = \frac{(n+1)(n+2)}{2}$.

\begin{align*}
\sum_{k=1}^{n+1} k
&= \left(\sum_{k=1}^n k\right) + (n+1) \\
&= \frac{n(n+1)}{2} + (n+1)
  \tag{by inductive hypothesis} \\
&= (n+1)\left(\frac{n}{2} + 1\right)
  = \frac{(n+1)(n+2)}{2}.
\end{align*}

Hence $P(n+1)$ holds.

By induction, $P(n)$ holds for all $n \geq 1$. \qed
\end{example}

\begin{example}[Divisibility]
\textbf{Proposition.}
For all $n \geq 0$, $3 \mid (4^n - 1)$.

\medskip
\noindent\textit{Proof.}
Let $P(n)$ denote ``$3 \mid (4^n - 1)$''.

\medskip
\noindent\textbf{Base case} ($n=0$).
$4^0 - 1 = 0 = 3 \cdot 0$. Hence $3 \mid 0$, so $P(0)$ holds.

\medskip
\noindent\textbf{Inductive step.}
Assume $3 \mid (4^n - 1)$, i.e., $4^n - 1 = 3m$ for some
$m \in \mathbb{Z}$.

We compute:
\[
4^{n+1} - 1 = 4 \cdot 4^n - 1
= 4(4^n - 1) + 4 - 1
= 4 \cdot 3m + 3
= 3(4m + 1).
\]
Hence $3 \mid (4^{n+1}-1)$, so $P(n+1)$ holds.

By induction, $P(n)$ holds for all $n \geq 0$. \qed
\end{example}

\begin{remark}[Anatomy of both examples]
In both examples, the inductive step follows a single pattern:
\emph{rewrite $P(n+1)$ to expose $P(n)$ inside it, then substitute
the inductive hypothesis.} This rewriting step is the core skill
of weak induction. When it is not obvious how to expose $P(n)$,
the statement $P(n)$ may need to be strengthened ---
see Section~\ref{sec:choosing-pn}.
\end{remark}

% =========================================================
% Choosing P(n): The Hard Part
% =========================================================
\paragraph{Choosing $P(n)$: The Hard Part.}
\label{sec:choosing-pn}

The statement you induct on is a choice, not given data.
For textbook sum formulas, $P(n)$ is handed to you.
In original proofs, you must \emph{invent} it.
Sometimes the obvious choice does not work and must be strengthened.

\begin{remark}[The strengthening principle]
If the inductive step fails with $P(n)$, the fix is usually to
\emph{add more to $P(n)$}, not to try a different proof strategy.
This is counterintuitive: the stronger claim $P'(n)$
(which implies $P(n)$) is easier to induct on because the
inductive hypothesis $P'(n)$ gives you more to work with.
\end{remark}

\begin{example}[Strengthening is necessary]
\textbf{Goal.}
Show that the Fibonacci sequence $F_n$ defined by $F_1 = F_2 = 1$,
$F_{n+2} = F_{n+1} + F_n$ satisfies
$F_n < 2^n$ for all $n \geq 1$.

\medskip
\noindent\textbf{Naive attempt.}
Let $P(n)$: ``$F_n < 2^n$''.

Inductive step: assume $F_n < 2^n$; show $F_{n+1} < 2^{n+1}$.

We have $F_{n+1} = F_n + F_{n-1}$.
We know $F_n < 2^n$ by hypothesis. But what is $F_{n-1}$?
We have no bound on $F_{n-1}$ from $P(n)$ alone.
The step is stuck.

\medskip
\noindent\textbf{Fix: use strong induction.}
Assume $F_k < 2^k$ for all $k \leq n$.
Then:
\[
F_{n+1} = F_n + F_{n-1} < 2^n + 2^{n-1} < 2^n + 2^n = 2^{n+1}.
\]
The step goes through because the stronger hypothesis provides
$F_{n-1} < 2^{n-1}$.
This is the canonical signal that strong induction is needed.
\end{example}

\begin{example}[Strengthening the claim itself]
\textbf{Goal.}
Prove that for all $n \geq 1$, the sum
$\displaystyle S_n = \sum_{k=1}^n \frac{1}{k(k+1)}$
equals $\displaystyle 1 - \frac{1}{n+1}$.

\medskip
\noindent\textbf{What $P(n)$ to choose?}
The statement is handed to us: $P(n)$: $S_n = 1 - \frac{1}{n+1}$.
Here the step is: $S_{n+1} = S_n + \frac{1}{(n+1)(n+2)}$.
Substitute $S_n = 1 - \frac{1}{n+1}$ and simplify.
The inductive step works because $P(n)$ gives the exact
closed form needed.

\medskip
\noindent\textbf{Lesson.}
When the claim is a closed-form identity, $P(n)$ is the identity
itself. The step is always: split off the last term, apply the
inductive hypothesis, simplify algebraically.
\end{example}

\begin{remark}[Diagnostic questions for choosing $P(n)$]
When stuck choosing $P(n)$:
\begin{enumerate}
  \item What exactly do I need to prove for $n = 0, 1, 2, 3$?
    Write it out. The pattern of $P(n)$ will appear.
  \item Does the inductive step require knowing $P(n-1)$ as well
    as $P(n)$? If yes, switch to strong induction.
  \item Does the inductive step produce a bound or equation that
    is slightly weaker than $P(n+1)$? If yes, strengthen $P(n)$.
\end{enumerate}
\end{remark}

% =========================================================
% Choosing P(n): Axiomatic and Recursive Settings
% =========================================================

\paragraph{Choosing $P(n)$ in Axiomatic and Recursive Settings.}
\label{sec:choosing-pn-axiomatic}

The examples above assume $P(n)$ is a formula handed to you or
readable from a sequence. In axiomatic settings (e.g., Tao
Chapter~2), the goal is stated as a universal claim about a
recursively defined operation, and $P(n)$ must be
\emph{constructed} from the logical form of the goal.
Two rules govern this construction.

\begin{remark}[Rule 1: $P(n)$ mirrors the logical form of the goal]
Every universal claim has the form $\forall n,\ \Phi(n)$.
Strip the universal quantifier. What remains is $P(n)$.

\medskip
\noindent The shape of $P(n)$ is determined by $\Phi$:

\medskip
\noindent
\begin{tabular}{ll}
\toprule
\textbf{Goal form} & \textbf{Shape of $P(n)$} \\
\midrule
$\forall n,\ f(n) = g(n)$ & Equality: $P(n) :\equiv f(n) = g(n)$ \\[4pt]
$\forall n,\ A(n) \Rightarrow B(n)$ & Conditional: $P(n) :\equiv A(n) \Rightarrow B(n)$ \\[4pt]
$\forall n,\ A(n) \Leftrightarrow B(n)$ & Biconditional: $P(n) :\equiv A(n) \Leftrightarrow B(n)$ \\[4pt]
$\forall n,\ \varphi(n)$ & Simple property: $P(n) :\equiv \varphi(n)$ \\
\bottomrule
\end{tabular}

\medskip
\noindent Do not add or remove hypotheses. Do not simplify.
Transcribe the logical form exactly.
\end{remark}

\begin{remark}[Rule 2: Induct on the recursive variable]
When the goal involves a recursively defined operation,
identify which variable the definition recurses on.
Induct on that variable and hold all others fixed.

\medskip
\noindent To find the recursive variable: locate the definition
of the operation and identify which variable is reduced in the
recursive clause.

\medskip
\noindent
\begin{tabular}{lll}
\toprule
\textbf{Definition} & \textbf{Recursive clause} & \textbf{Induct on} \\
\midrule
Addition (Tao A1--A2) & $(n\pp) + m := (n + m)\pp$ & $n$ \\[4pt]
Multiplication (Tao M1--M2) & $(n\pp) \times m := (n \times m) + m$ & $n$ \\[4pt]
Exponentiation (Tao) & $m^{n\pp} := m^n \times m$ & $n$ \\
\bottomrule
\end{tabular}

\medskip
\noindent The non-recursive variable is held fixed throughout
the induction and treated as an arbitrary element of $\mathbb{N}$.
\end{remark}

\begin{example}[Constructing $P(n)$ for a conditional goal]
\textbf{Goal.} Prove Lemma~2.3.3: for all $n, m \in \mathbb{N}$,
if $n > 0$ and $m > 0$ then $nm > 0$.

\medskip
\noindent\textbf{Step 1 — Read the logical form.}
The goal is a universal conditional:
$\forall n,\ (n > 0 \wedge m > 0) \Rightarrow nm > 0$.
Strip the quantifier:
\[
P(n) :\equiv \text{if } n > 0 \text{ and } m > 0
\text{ then } n \times m > 0.
\]

\noindent\textbf{Step 2 — Identify the recursive variable.}
The operation is multiplication. M2 recurses on $n$:
$(n\pp) \times m := (n \times m) + m$.
Induct on $n$, hold $m$ fixed.

\medskip
\noindent\textbf{Step 3 — Check the base case form.}
$P(0)$ states: if $0 > 0$ and $m > 0$ then $0 \times m > 0$.
The hypothesis $0 > 0$ is false, so $P(0)$ is vacuously true.
A conditional $P(n)$ will often produce a vacuous base case
when the hypothesis excludes $n = 0$.
\end{example}

\begin{remark}[Vacuous base cases are a signal, not a problem]
When $P(n)$ is a conditional and its hypothesis fails at the base
value, the base case is vacuously true by definition.
This is expected and requires no further argument.
It signals that the claim only has content for values of $n$
where the hypothesis holds --- positivity, nonzero-ness, a bound,
or any other precondition that excludes the base case.
The induction then establishes the claim for all $n$ where the
hypothesis is satisfied.
\end{remark}


\begin{remark}[Diagnostic questions for axiomatic settings]
When constructing $P(n)$ in an axiomatic proof:
\begin{enumerate}
  \item What is the logical form of the goal?
    Write it out with explicit quantifiers before choosing $P(n)$.
  \item Which operation is involved?
    Find its recursive definition and identify the recursive variable.
  \item Does the base case require $n = 0$ to satisfy a hypothesis?
    If the hypothesis fails at $n = 0$, expect a vacuous base case.
  \item After writing $P(n\pp)$, can I apply the recursive definition
    to expose $P(n)$ or an earlier lemma?
    If not, the wrong variable may have been chosen.
\end{enumerate}
\end{remark}

% =========================================================
% Strong Induction
% =========================================================
\begin{tcolorbox}[colback=propbox, colframe=propborder, arc=2pt,
  left=6pt, right=6pt, top=4pt, bottom=4pt,
  title={\small\textbf{Theorem: Strong Induction}},
  fonttitle=\small\bfseries]
Let $P(n)$ be a statement depending on $n \in \mathbb{N}$.
Suppose that for every $n \in \mathbb{N}$:
\[
\Big(\forall k < n,\; P(k)\Big) \Rightarrow P(n).
\]
Then $P(n)$ holds for all $n \in \mathbb{N}$.
\end{tcolorbox}

\begin{remark}[The base case in strong induction]
When $n = 0$, the hypothesis $\forall k < 0,\; P(k)$ is vacuously
true (no $k < 0$ exists in $\mathbb{N}$). So the step for $n = 0$
reduces to showing $P(0)$ unconditionally --- this is the base case,
and it is implicit in the universal statement rather than listed
separately. In practice, still write the base case explicitly.
\end{remark}

\begin{remark}[When to use strong induction]
Use strong induction when the proof of $P(n+1)$ requires knowing
$P(k)$ for some $k < n$ that is not specifically $k = n$.

The diagnostic: in the inductive step, write
$P(n+1) = f(P(?), P(?), \ldots)$ and ask what values of $?$ you need.
If the answer includes anything other than $n$, use strong induction.

Canonical situations:
\begin{itemize}
  \item Recursive sequences: $a_{n+1} = a_n + a_{n-1}$ (need $n-1$).
  \item Prime factorization: $n+1 = ab$ requires $P(a)$ and $P(b)$
    where $a, b < n+1$ but are otherwise arbitrary.
  \item Any argument that breaks $n+1$ into parts.
\end{itemize}
\end{remark}

\begin{tcolorbox}[colback=gray!4, colframe=gray!35, arc=2pt,
  left=6pt, right=6pt, top=4pt, bottom=4pt,
  title={\small\textbf{Template: Strong Induction Proof}},
  fonttitle=\small\bfseries]
Let $P(n)$ denote: [\emph{write it out}].

\medskip
\noindent\textbf{Base case} ($n = [\text{start}]$).
[\emph{Verify directly.}]

\medskip
\noindent\textbf{Inductive step.}
Let $n > [\text{start}]$ and assume $P(k)$ holds for all
$[\text{start}] \leq k < n$.
\textit{(Strong inductive hypothesis.)}
We show $P(n)$ holds.

[\emph{Work. Use $P(k)$ for some specific $k < n$.}]

Hence $P(n)$ holds.

By strong induction, $P(n)$ holds for all $n \geq [\text{start}]$. \qed
\end{tcolorbox}

\begin{example}[Every integer $n \geq 2$ has a prime factor]
\textit{Proof.}
Let $P(n)$: ``$n$ has a prime factor''.

\medskip
\noindent\textbf{Base case} ($n = 2$).
$2$ is prime, so $2$ is a prime factor of itself.

\medskip
\noindent\textbf{Inductive step.}
Let $n > 2$ and assume $P(k)$ holds for all $2 \leq k < n$.

\textbf{Case 1}: $n$ is prime. Then $n$ is its own prime factor.

\textbf{Case 2}: $n$ is composite. Then $n = ab$ for some
$2 \leq a, b < n$.
By the strong inductive hypothesis applied to $a$,
$a$ has a prime factor $p$.
Then $p \mid a$ and $a \mid n$, so $p \mid n$.

In both cases $n$ has a prime factor. By strong induction,
$P(n)$ holds for all $n \geq 2$. \qed
\end{example}

\begin{remark}[Why weak induction cannot do this directly]
In the prime factor proof, the composite case splits $n$ into $a < n$.
We need $P(a)$, not $P(n-1)$. Weak induction only provides $P(n-1)$,
which is useless if $a \neq n-1$.
Strong induction provides $P(k)$ for all $k < n$,
covering whatever value $a$ takes.
\end{remark}

% =========================================================
% Well-Ordering and Equivalence
% =========================================================
\noindent\begin{tcolorbox}[colback=propbox, colframe=propborder, arc=2pt,
  left=6pt, right=6pt, top=4pt, bottom=4pt,
  title={\small\textbf{Well-Ordering Principle}},
  fonttitle=\small\bfseries]
Every nonempty subset of $\mathbb{N}$ has a least element.
\end{tcolorbox}
\label{sec:wellordering}

\begin{remark}[Well-ordering as a proof engine]
The well-ordering principle drives the \emph{minimal counterexample}
argument, which is the well-ordering dual of induction:

\begin{enumerate}
  \item Suppose $P(n)$ fails for some $n$.
  \item Let $S = \{n \in \mathbb{N} : \neg P(n)\}$.
    By assumption $S \neq \emptyset$.
  \item By well-ordering, $S$ has a least element $m$.
  \item Derive a contradiction: show either $P(m)$ holds
    (contradicting $m \in S$) or some $k < m$ satisfies
    $k \in S$ (contradicting minimality of $m$).
\end{enumerate}

This is structurally the same as induction but runs in the
opposite direction: instead of climbing up from a base case,
you descend to a minimal failure and derive a contradiction.
\end{remark}

\begin{proposition}[Equivalence of induction and well-ordering]
The following are equivalent over the axioms of $\mathbb{N}$
without P5:
\begin{enumerate}
  \item The induction principle (weak induction).
  \item The well-ordering principle.
  \item The strong induction principle.
\end{enumerate}
\end{proposition}

\begin{remark}[Proof sketch of equivalence]
\textbf{Induction $\Rightarrow$ Well-ordering.}
Suppose $S \subseteq \mathbb{N}$ has no least element.
Define $P(n)$: ``$n \notin S$''. Then $P(0)$ holds
(else $0 \in S$ would be the least element). If $P(k)$ holds
for all $k \leq n$ then $n+1 \notin S$ (else $n+1$ would be
a least element). By induction, $P(n)$ holds for all $n$,
so $S = \emptyset$.

\textbf{Well-ordering $\Rightarrow$ Induction.}
Suppose the base case and inductive step hold.
Let $S = \{n : \neg P(n)\}$. If $S \neq \emptyset$, let $m$ be
its least element. Then $m \neq 0$ (base case), so $m \geq 1$,
so $m - 1 < m$ satisfies $P(m-1)$ (minimality of $m$).
But by the inductive step, $P(m-1) \Rightarrow P(m)$,
contradicting $m \in S$. Hence $S = \emptyset$.
\end{remark}

\begin{tcolorbox}[colback=gray!4, colframe=gray!35, arc=2pt,
  left=6pt, right=6pt, top=4pt, bottom=4pt,
  title={\small\textbf{Template: Minimal Counterexample}},
  fonttitle=\small\bfseries]
Suppose for contradiction that $P(n)$ fails for some $n \in \mathbb{N}$.

Let $S = \{n \in \mathbb{N} : \neg P(n)\}$.
By assumption $S \neq \emptyset$.
By well-ordering, $S$ has a least element $m$.

[\emph{Derive a contradiction using minimality of $m$.}]

This contradicts [minimality of $m$ / the base case].
Therefore $P(n)$ holds for all $n \in \mathbb{N}$. \qed
\end{tcolorbox}

\begin{example}[Minimal counterexample: divisibility]
\textbf{Claim.} For all $n \geq 1$, $2^n > n$.

\medskip
\noindent\textit{Proof (minimal counterexample).}
Suppose the claim fails. Let
$S = \{n \geq 1 : 2^n \leq n\}$.
By assumption $S \neq \emptyset$; let $m$ be its least element.

Then $m \geq 2$ (since $2^1 = 2 > 1$), so $m - 1 \geq 1$
and $m - 1 < m$, hence $m - 1 \notin S$, so $2^{m-1} > m - 1$.
Then:
\[
2^m = 2 \cdot 2^{m-1} > 2(m-1) = 2m - 2 \geq m
\]
(since $m \geq 2$), contradicting $m \in S$.
Therefore $S = \emptyset$ and the claim holds. \qed
\end{example}

% =========================================================
% Structural Induction
% =========================================================
\paragraph{Structural Induction (Preview).}

Induction on $\mathbb{N}$ is a special case of a more general
principle that applies to any recursively defined structure.

\begin{remark}[The general principle]
Whenever a set $X$ is defined by:
\begin{enumerate}
  \item a collection of \emph{base cases} (atomic elements), and
  \item a collection of \emph{constructor rules} (ways of building
    larger elements from smaller ones),
\end{enumerate}
structural induction is available: to prove $P(x)$ for all $x \in X$,
prove $P$ for all base cases, and prove that if $P$ holds for the
parts then $P$ holds for the whole.
\end{remark}

\begin{example}[Induction on algebraic expressions]
Define an \emph{algebraic term} over a ring $R$ by:
\begin{itemize}
  \item Base: any $r \in R$ is a term.
  \item Constructor: if $s, t$ are terms, so are $s + t$ and $s \cdot t$.
\end{itemize}
To prove a property $P$ holds for all terms:
\begin{enumerate}
  \item Prove $P(r)$ for all $r \in R$ (base cases).
  \item Prove: if $P(s)$ and $P(t)$ hold, then $P(s+t)$ holds.
  \item Prove: if $P(s)$ and $P(t)$ hold, then $P(s \cdot t)$ holds.
\end{enumerate}
This is structural induction on the recursive definition of terms.
\end{example}

\begin{remark}[Where structural induction appears in this project]
\begin{itemize}
  \item \textbf{Axiom Systems}: induction on $\mathbb{N}$ is proved
    from Peano P5, which is exactly the structural induction
    principle for $(\mathbb{N}, 0, S)$.
  \item \textbf{Abstract Algebra}: properties of polynomials over a ring
    are proved by induction on degree (structural induction on
    the recursive definition of polynomials).
  \item \textbf{Computer Science}: correctness of recursive algorithms
    is proved by structural induction on the input data structure
    (lists, trees, expressions).
\end{itemize}
The full development of structural induction appears in the
Abstract Algebra chapter.
\end{remark}

% =========================================================
% Common Induction Mistakes
% =========================================================
\paragraph{Common Induction Mistakes.}

\begin{remark}[Mistake 1: Missing base case]
A proof that establishes the inductive step but not the base case
proves nothing. The inductive step alone says: \emph{if} $P(n)$
holds for some $n$, then $P(n+1)$ holds. Without a base case,
there is no $n$ for which $P(n)$ is known to hold, so the
chain never starts.

\textbf{Classic fake theorem}: ``All horses are the same colour.''
\textit{Proof attempt}: Base case $n=1$: one horse is trivially
the same colour as itself. Inductive step: [flawed argument].
The flaw is in the inductive step, but the lesson is that
even a correct base case does not rescue a bad step.
Scrutinize both parts independently.
\end{remark}

\begin{remark}[Mistake 2: Circular inductive step]
The inductive step must prove $P(n+1)$ using only $P(n)$ (or
earlier cases in strong induction) and previously established
facts. It is illegal to assume $P(n+1)$ anywhere in the proof
of $P(n+1)$.

This error is most common when the student ``derives'' both
sides of an equation and meets in the middle, inadvertently
assuming the conclusion. Always work in one direction: from
$P(n)$, derive $P(n+1)$.
\end{remark}

\begin{remark}[Mistake 3: Wrong or missing inductive hypothesis]
The most common error in practice is writing ``assume $P(n)$''
without stating what $P(n)$ says. This is a sign that $P(n)$
has not been written down as a full sentence. Write it out.
If you cannot state $P(n)$ as a complete mathematical sentence,
you do not yet know what you are inducting on.
\end{remark}

\begin{remark}[Mistake 4: Off-by-one base case]
Induction proves $P(n)$ for all $n \geq n_0$, where $n_0$ is
the base case. If the claim is only true for $n \geq 2$,
the base case $n = 0$ may fail, and the proof should start at
$n = 2$.

Check: what is the smallest $n$ for which the claim is true?
That is your base case. Do not default to $n = 0$ or $n = 1$
without verifying.
\end{remark}

\begin{remark}[Mistake 5: Using strong induction when weak suffices]
Strong induction is not more powerful than weak induction in the
sense that they prove the same class of statements. But a proof
that uses the full strength of the strong hypothesis when only
$P(n)$ is needed is unnecessarily complicated. Use weak induction
unless the step genuinely requires knowing $P(k)$ for some
$k < n$ other than $k = n - 1$.
\end{remark}

\begin{remark}[Mistake 6: Forgetting vacuity in strong induction]
In strong induction, the case $n = 0$ has a vacuously true
hypothesis $\forall k < 0,\; P(k)$, since no natural number
is less than $0$. This means $P(0)$ must be proved directly,
not derived from the inductive hypothesis. Many students
write ``by the inductive hypothesis applied to $n-1$'' when
$n = 0$ has no predecessor. The base case is always proved
from scratch.
\end{remark}


% =========================================================
% Section 5: Algebraic Tactics
% =========================================================
\subsubsection{Algebraic Tactics}
\label{sec:tactics}

% =========================================================
% DU/TA/AM Framework
% =========================================================
\paragraph{The DU/TA/AM Framework.}

Every line of an algebraic proof is justified by exactly one
of three sources. This framework formalizes the tagging system
used throughout these notes.

\begin{definition}[Proof line tags]
Each step in a three-column proof carries one of three tags:
\begin{itemize}
  \item \tagDU\ \textbf{(Definition / Unpack)}.
    The step applies a definition or unpacks a named concept
    into its explicit conditions.
    \textit{Example}: expanding ``$e$ is an identity'' into
    ``$ea = ae = a$ for all $a \in G$''.

  \item \tagTA\ \textbf{(Theorem / Axiom)}.
    The step cites a previously proved proposition or a structural
    axiom of the system (group axiom, ring axiom, field axiom,
    or Peano axiom).
    \textit{Example}: citing G1 (associativity) to regroup
    $(ab)c = a(bc)$.

  \item \tagAM\ \textbf{(Algebraic Manipulation)}.
    The step performs a calculation whose justification is
    itself a combination of DU and TA moves, compressed for
    readability.
    \textit{Example}: simplifying $a \cdot e = a$ in one step
    after the identity axiom has already been cited.
\end{itemize}
\end{definition}

\begin{remark}[Why tag every line?]
The DU/TA/AM system enforces the most important discipline in
proof writing: every step must be justified by something.
A line with no tag is a line whose justification has not been
identified. When a proof stalls, examining the tags reveals
which definition has not been unpacked or which axiom has
not been cited.

At the learning stage, every line should carry an explicit tag.
As fluency increases, AM steps can absorb multiple tagged moves.
But the underlying DU and TA structure is always present even
when not written.
\end{remark}

\begin{remark}[The hierarchy of tags]
DU and TA are the atomic justifications. AM is a shorthand for
a sequence of DU and TA steps that have been verified and
compressed. A fully formal proof contains only DU and TA steps.
The three-column format used in these notes puts the statement
in the middle column and the tag in the left column, forcing
explicit justification of every line.
\end{remark}

% =========================================================
% Tactics Catalog
% =========================================================
\paragraph{Catalog of Algebraic Tactics.}

The following tactics are the move-level tools available once
a proof is in progress and the architecture has been chosen.
Each entry states the tactic, its precondition (what must
already be in place), and its effect (what it achieves).

\begin{center}
\renewcommand{\arraystretch}{1.5}
\begin{tabular}{p{0.20\textwidth} p{0.35\textwidth} p{0.35\textwidth}}
\toprule
\textbf{Tactic} & \textbf{Precondition} & \textbf{Effect} \\
\midrule

\textbf{Multiply by inverse} &
  $a \neq 0$ in a field; or $a \in G$ a group element &
  Multiply both sides by $a^{-1}$; use associativity to collapse
  $a^{-1} a = e$; isolates the target variable. \\

\textbf{Cancellation} &
  $ac = bc$ or $ca = cb$ in a group or integral domain &
  Conclude $a = b$. In a group: multiply by $c^{-1}$.
  In an integral domain: cite no-zero-divisors. \\

\textbf{Distributivity bridge} &
  Multiplication by $0$, $-1$, or a negative element &
  Rewrite the product as a sum using distributivity,
  then use additive group structure to collapse.
  The bridge crosses from $\times$ to $+$. \\

\textbf{Add zero} &
  Need to introduce a term without changing the expression &
  Write $a = a + 0 = a + (b + (-b))$; regroup to expose
  needed structure. \\

\textbf{Satisfy-and-cite} &
  A uniqueness theorem for some object $b$ is in scope &
  Show $a$ satisfies $b$'s defining property; cite uniqueness;
  conclude $a = b$. Avoids assume-two-and-compare entirely. \\

\textbf{Double negation} &
  $-(- a)$ appears or is the goal &
  $-a$ is the unique element $x$ with $a + x = 0$;
  $-(-a)$ satisfies this with $x = a$; by uniqueness,
  $-(-a) = a$. \\

\textbf{Absorb into axiom} &
  A sub-expression matches the left-hand side of an axiom &
  Rewrite using the axiom. Tag as \tagTA. \\

\bottomrule
\end{tabular}
\end{center}

\begin{remark}[Distributivity bridge in detail]
The distributivity bridge is the central tactic for ring and
field zero/negative proofs. The pattern is always:
\begin{enumerate}
  \item Introduce $a \cdot 0$ or $a \cdot (-b)$.
  \item Write $0 = x + (-x)$ or $-b = b + (-b) + (-b)$\,---\,use
    the additive inverse definition to expand.
  \item Apply distributivity: $a(x + y) = ax + ay$.
  \item Use uniqueness of additive inverse to identify the result.
\end{enumerate}
Every instance of ``$a \cdot 0 = 0$'' and ``$a(-b) = -(ab)$''
uses this bridge. Once seen once, it is immediately recognizable.
\end{remark}

\begin{remark}[Cancellation: two different theorems]
``Cancellation'' names two distinct results that must not be conflated:
\begin{itemize}
  \item \textbf{Group cancellation}: in any group, $ac = bc$
    implies $a = b$. Proved by multiplying by $c^{-1}$, which
    exists by the group inverse axiom.
  \item \textbf{Domain cancellation}: in an integral domain,
    $ac = bc$ and $c \neq 0$ imply $a = b$. Proved by rewriting
    as $c(a - b) = 0$ and citing no zero divisors.
\end{itemize}
The preconditions are different. In a group, $c$ need not be
specified as nonzero (inverses always exist). In an integral domain,
$c \neq 0$ is essential.
\end{remark}


% =========================================================
% Section 6: Proof Strategies Reference
% =========================================================
\subsubsection{Proof Strategies Reference}
\label{sec:reference}

% =========================================================
% Proof Strategies Lookup Table
% =========================================================
\paragraph{Lookup Table: Stuck Situations and Strategies.}

\begin{remark}[How to use this table]
When progress on a proof stalls, locate the column that describes
your situation. The middle column names the strategy. The right
column gives the specific first move and a cross-reference to
a worked example.
\end{remark}

\begin{center}
\renewcommand{\arraystretch}{1.6}
\begin{tabular}{p{0.27\textwidth} p{0.22\textwidth} p{0.40\textwidth}}
\toprule
\textbf{Stuck because\ldots} &
\textbf{Strategy} &
\textbf{First move / See} \\
\midrule

I need to prove $P$ but don't know where to start &
  Two-question check &
  Write the logical form of $P$. Consult the statement map.
  \S\ref{sec:proof-architecture} \\

I need to show two things are equal and one is uniquely defined &
  Satisfy-and-cite &
  Show the first satisfies the definition of the second.
  Cite the uniqueness theorem.
  \S\ref{sec:satisfy-cite} \\

I need to show something is the only object of its kind &
  Assume-two-and-compare &
  ``Let $x, y$ both satisfy the definition. Show $x = y$.''
  \S\ref{sec:proof-structures} \\

I have a conditional $P \Rightarrow Q$ and $P$ is hard to use &
  Contrapositive &
  ``Assume $\neg Q$.'' Derive $\neg P$.
  \S\ref{sec:proof-structures} \\

I have a conditional and the conclusion is a negation or impossible &
  Contradiction &
  ``Assume $P$ and $\neg Q$.'' Derive any contradiction.
  \S\ref{sec:proof-structures} \\

The claim splits naturally by cases &
  Case analysis &
  List all cases explicitly. Verify they are exhaustive.
  \S\ref{sec:proof-structures} \\

The claim is about all $n \in \mathbb{N}$ &
  Induction &
  Write $P(n)$ as a full sentence. Prove base case.
  \S\ref{sec:induction} \\

The inductive step needs $P(k)$ for some $k < n$ (not just $n-1$) &
  Strong induction &
  Replace ``assume $P(n)$'' with
  ``assume $P(k)$ for all $k < n$''.
  \S\ref{sec:induction} \\

The claim is about all $n \geq n_0$ and feels easier to negate &
  Minimal counterexample &
  ``Suppose $S = \{n : \neg P(n)\} \neq \emptyset$. Let $m = \min S$.''
  \S\ref{sec:wellordering} \\

I need to prove $a \cdot 0 = 0$ or $a(-b) = -(ab)$ &
  Distributivity bridge &
  Write $0 = x + (-x)$; apply distributivity; use
  uniqueness of additive inverse.
  \S\ref{sec:tactics} \\

I have $ac = bc$ and need $a = b$ &
  Cancellation &
  Group: multiply by $c^{-1}$.
  Domain: write $(a-b)c = 0$; cite no zero divisors.
  \S\ref{sec:tactics} \\

I need to set two sets equal &
  Double inclusion &
  Prove $A \subseteq B$: chase an arbitrary $x \in A$.
  Then prove $B \subseteq A$.
  \S\ref{sec:proof-architecture} \\

A definition has not been unpacked &
  DU move &
  Identify the named concept and write out its definition.
  Tag the line \tagDU.
  \S\ref{sec:tactics} \\

A step uses an axiom or theorem by name &
  TA move &
  Cite the axiom or theorem explicitly by label.
  Tag the line \tagTA.
  \S\ref{sec:tactics} \\

\bottomrule
\end{tabular}
\end{center}

\begin{remark}[Cross-references to worked examples]
The following table maps each strategy to a worked example
in these notes where it is the primary tool.

\begin{center}
\renewcommand{\arraystretch}{1.4}
\begin{tabular}{p{0.35\textwidth} p{0.55\textwidth}}
\toprule
\textbf{Strategy} & \textbf{Primary worked examples} \\
\midrule
Assume-two-and-compare &
  Prop.~\ref{prop:group-identity-unique},
  Prop.~\ref{prop:group-inverse-unique} \\
Satisfy-and-cite &
  Prop.~\ref{prop:group-socks-shoes},
  Prop.~\ref{prop:ring-mult-neg} \\
Distributivity bridge &
  Prop.~\ref{prop:ring-mult-zero},
  Prop.~\ref{prop:ring-mult-neg} \\
Cancellation (group) &
  Prop.~\ref{prop:group-cancellation} \\
Cancellation (domain) &
  Prop.~\ref{prop:domain-cancellation} \\
Multiply by inverse + no zero divisors &
  Prop.~\ref{prop:field-is-domain} \\
Weak induction &
  Sum formula, divisibility examples (\S\ref{sec:induction}) \\
Strong induction &
  Prime factor theorem (\S\ref{sec:induction}) \\
Minimal counterexample &
  $2^n > n$ example (\S\ref{sec:wellordering}) \\
Contrapositive &
  $n^2$ even $\Rightarrow$ $n$ even (\S\ref{sec:proof-structures}) \\
Contradiction &
  $\sqrt{2} \notin \mathbb{Q}$ (\S\ref{sec:proof-structures}) \\
\bottomrule
\end{tabular}
\end{center}
\end{remark}


