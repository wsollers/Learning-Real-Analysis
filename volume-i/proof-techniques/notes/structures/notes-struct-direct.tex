% =========================================================
% Direct Proof
% =========================================================
\paragraph{Direct Proof.}

A direct proof of $P \Rightarrow Q$ assumes $P$ and derives $Q$
through a chain of valid inferences.

\begin{tcolorbox}[colback=propbox, colframe=propborder, arc=2pt,
  left=6pt, right=6pt, top=4pt, bottom=4pt,
  title={\small\textbf{Template: Direct Proof}}, fonttitle=\small\bfseries]
\textbf{Given.} $P$.\\
\textbf{Goal.} $Q$.\\[4pt]
Assume $P$.\\
$\vdots$ \quad [expand definitions; derive consequences]\\
Therefore $Q$. \qed
\end{tcolorbox}

\begin{remark}[When to use direct proof]
Direct proof is the default. Reach for it first.
It is appropriate whenever the hypothesis gives you something
concrete to work with and the conclusion follows by expanding
definitions and applying theorems.
Reserve contrapositive and contradiction for cases where the
hypothesis is hard to use forward.
\end{remark}

\begin{example}[Direct proof in a group]
\textbf{Claim.} In a group $G$, if $a^2 = e$ then $a = a^{-1}$.

\medskip
\noindent\textit{Proof.}
Assume $a^2 = e$, i.e., $aa = e$.
Multiply both sides on the right by $a^{-1}$:
\[
aa \cdot a^{-1} = e \cdot a^{-1}.
\]
By associativity, $a(aa^{-1}) = a^{-1}$, so $ae = a^{-1}$,
so $a = a^{-1}$. \qed
\end{example}
