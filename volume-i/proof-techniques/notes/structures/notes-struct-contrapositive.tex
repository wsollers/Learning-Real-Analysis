% =========================================================
% Contrapositive
% =========================================================
\paragraph{Proof by Contrapositive.}

The contrapositive of $P \Rightarrow Q$ is $\neg Q \Rightarrow \neg P$.
These are logically equivalent. A proof by contrapositive proves
$\neg Q \Rightarrow \neg P$ directly.

\begin{tcolorbox}[colback=propbox, colframe=propborder, arc=2pt,
  left=6pt, right=6pt, top=4pt, bottom=4pt,
  title={\small\textbf{Template: Contrapositive}}, fonttitle=\small\bfseries]
\textbf{Goal.} $P \Rightarrow Q$.\\
\textbf{Equivalent goal.} $\neg Q \Rightarrow \neg P$.\\[4pt]
Assume $\neg Q$.\\
$\vdots$ \quad [derive $\neg P$]\\
Therefore $\neg P$.
Hence $P \Rightarrow Q$. \qed
\end{tcolorbox}

\begin{remark}[When to use contrapositive]
Use contrapositive when the negation of $Q$ is more informative
or easier to manipulate than $P$ itself.
Injectivity proofs are a canonical example: to prove
$f(a) = f(b) \Rightarrow a = b$, it is often cleaner to prove
the contrapositive $a \neq b \Rightarrow f(a) \neq f(b)$.
\end{remark}

\begin{remark}[Contrapositive vs contradiction]
Contrapositive: assume $\neg Q$, derive $\neg P$, done.
Contradiction: assume both $P$ and $\neg Q$, derive any
absurdity, done.
Contrapositive is cleaner when it works because the assumption
is $\neg Q$ alone, not $P \land \neg Q$.
Try contrapositive before reaching for contradiction.
\end{remark}

\begin{example}[Contrapositive for divisibility]
\textbf{Claim.} If $n^2$ is even then $n$ is even.

\medskip
\noindent\textit{Proof (contrapositive).}
Assume $n$ is odd. Then $n = 2k+1$ for some $k \in \mathbb{Z}$.
\[
n^2 = (2k+1)^2 = 4k^2 + 4k + 1 = 2(2k^2+2k)+1,
\]
which is odd. Hence $n$ odd $\Rightarrow$ $n^2$ odd, i.e.,
$n^2$ even $\Rightarrow$ $n$ even. \qed
\end{example}
