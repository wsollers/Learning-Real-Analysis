% =========================================================
% DU/TA/AM Framework
% =========================================================
\paragraph{The DU/TA/AM Framework.}

Every line of an algebraic proof is justified by exactly one
of three sources. This framework formalizes the tagging system
used throughout these notes.

\begin{definition}[Proof line tags]
Each step in a three-column proof carries one of three tags:
\begin{itemize}
  \item \tagDU\ \textbf{(Definition / Unpack)}.
    The step applies a definition or unpacks a named concept
    into its explicit conditions.
    \textit{Example}: expanding ``$e$ is an identity'' into
    ``$ea = ae = a$ for all $a \in G$''.

  \item \tagTA\ \textbf{(Theorem / Axiom)}.
    The step cites a previously proved proposition or a structural
    axiom of the system (group axiom, ring axiom, field axiom,
    or Peano axiom).
    \textit{Example}: citing G1 (associativity) to regroup
    $(ab)c = a(bc)$.

  \item \tagAM\ \textbf{(Algebraic Manipulation)}.
    The step performs a calculation whose justification is
    itself a combination of DU and TA moves, compressed for
    readability.
    \textit{Example}: simplifying $a \cdot e = a$ in one step
    after the identity axiom has already been cited.
\end{itemize}
\end{definition}

\begin{remark}[Why tag every line?]
The DU/TA/AM system enforces the most important discipline in
proof writing: every step must be justified by something.
A line with no tag is a line whose justification has not been
identified. When a proof stalls, examining the tags reveals
which definition has not been unpacked or which axiom has
not been cited.

At the learning stage, every line should carry an explicit tag.
As fluency increases, AM steps can absorb multiple tagged moves.
But the underlying DU and TA structure is always present even
when not written.
\end{remark}

\begin{remark}[The hierarchy of tags]
DU and TA are the atomic justifications. AM is a shorthand for
a sequence of DU and TA steps that have been verified and
compressed. A fully formal proof contains only DU and TA steps.
The three-column format used in these notes puts the statement
in the middle column and the tag in the left column, forcing
explicit justification of every line.
\end{remark}
