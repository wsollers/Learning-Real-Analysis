% =========================================================
% Tactics Catalog
% =========================================================
\paragraph{Catalog of Algebraic Tactics.}

The following tactics are the move-level tools available once
a proof is in progress and the architecture has been chosen.
Each entry states the tactic, its precondition (what must
already be in place), and its effect (what it achieves).

\begin{center}
\renewcommand{\arraystretch}{1.5}
\begin{tabular}{p{0.20\textwidth} p{0.35\textwidth} p{0.35\textwidth}}
\toprule
\textbf{Tactic} & \textbf{Precondition} & \textbf{Effect} \\
\midrule

\textbf{Multiply by inverse} &
  $a \neq 0$ in a field; or $a \in G$ a group element &
  Multiply both sides by $a^{-1}$; use associativity to collapse
  $a^{-1} a = e$; isolates the target variable. \\

\textbf{Cancellation} &
  $ac = bc$ or $ca = cb$ in a group or integral domain &
  Conclude $a = b$. In a group: multiply by $c^{-1}$.
  In an integral domain: cite no-zero-divisors. \\

\textbf{Distributivity bridge} &
  Multiplication by $0$, $-1$, or a negative element &
  Rewrite the product as a sum using distributivity,
  then use additive group structure to collapse.
  The bridge crosses from $\times$ to $+$. \\

\textbf{Add zero} &
  Need to introduce a term without changing the expression &
  Write $a = a + 0 = a + (b + (-b))$; regroup to expose
  needed structure. \\

\textbf{Satisfy-and-cite} &
  A uniqueness theorem for some object $b$ is in scope &
  Show $a$ satisfies $b$'s defining property; cite uniqueness;
  conclude $a = b$. Avoids assume-two-and-compare entirely. \\

\textbf{Double negation} &
  $-(- a)$ appears or is the goal &
  $-a$ is the unique element $x$ with $a + x = 0$;
  $-(-a)$ satisfies this with $x = a$; by uniqueness,
  $-(-a) = a$. \\

\textbf{Absorb into axiom} &
  A sub-expression matches the left-hand side of an axiom &
  Rewrite using the axiom. Tag as \tagTA. \\

\bottomrule
\end{tabular}
\end{center}

\begin{remark}[Distributivity bridge in detail]
The distributivity bridge is the central tactic for ring and
field zero/negative proofs. The pattern is always:
\begin{enumerate}
  \item Introduce $a \cdot 0$ or $a \cdot (-b)$.
  \item Write $0 = x + (-x)$ or $-b = b + (-b) + (-b)$\,---\,use
    the additive inverse definition to expand.
  \item Apply distributivity: $a(x + y) = ax + ay$.
  \item Use uniqueness of additive inverse to identify the result.
\end{enumerate}
Every instance of ``$a \cdot 0 = 0$'' and ``$a(-b) = -(ab)$''
uses this bridge. Once seen once, it is immediately recognizable.
\end{remark}

\begin{remark}[Cancellation: two different theorems]
``Cancellation'' names two distinct results that must not be conflated:
\begin{itemize}
  \item \textbf{Group cancellation}: in any group, $ac = bc$
    implies $a = b$. Proved by multiplying by $c^{-1}$, which
    exists by the group inverse axiom.
  \item \textbf{Domain cancellation}: in an integral domain,
    $ac = bc$ and $c \neq 0$ imply $a = b$. Proved by rewriting
    as $c(a - b) = 0$ and citing no zero divisors.
\end{itemize}
The preconditions are different. In a group, $c$ need not be
specified as nonzero (inverses always exist). In an integral domain,
$c \neq 0$ is essential.
\end{remark}
