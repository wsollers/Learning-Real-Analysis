% =========================================================
% Choosing P(n): The Hard Part
% =========================================================
\paragraph{Choosing $P(n)$: The Hard Part.}
\label{sec:choosing-pn}

The statement you induct on is a choice, not given data.
For textbook sum formulas, $P(n)$ is handed to you.
In original proofs, you must \emph{invent} it.
Sometimes the obvious choice does not work and must be strengthened.

\begin{remark}[The strengthening principle]
If the inductive step fails with $P(n)$, the fix is usually to
\emph{add more to $P(n)$}, not to try a different proof strategy.
This is counterintuitive: the stronger claim $P'(n)$
(which implies $P(n)$) is easier to induct on because the
inductive hypothesis $P'(n)$ gives you more to work with.
\end{remark}

\begin{example}[Strengthening is necessary]
\textbf{Goal.}
Show that the Fibonacci sequence $F_n$ defined by $F_1 = F_2 = 1$,
$F_{n+2} = F_{n+1} + F_n$ satisfies
$F_n < 2^n$ for all $n \geq 1$.

\medskip
\noindent\textbf{Naive attempt.}
Let $P(n)$: ``$F_n < 2^n$''.

Inductive step: assume $F_n < 2^n$; show $F_{n+1} < 2^{n+1}$.

We have $F_{n+1} = F_n + F_{n-1}$.
We know $F_n < 2^n$ by hypothesis. But what is $F_{n-1}$?
We have no bound on $F_{n-1}$ from $P(n)$ alone.
The step is stuck.

\medskip
\noindent\textbf{Fix: use strong induction.}
Assume $F_k < 2^k$ for all $k \leq n$.
Then:
\[
F_{n+1} = F_n + F_{n-1} < 2^n + 2^{n-1} < 2^n + 2^n = 2^{n+1}.
\]
The step goes through because the stronger hypothesis provides
$F_{n-1} < 2^{n-1}$.
This is the canonical signal that strong induction is needed.
\end{example}

\begin{example}[Strengthening the claim itself]
\textbf{Goal.}
Prove that for all $n \geq 1$, the sum
$\displaystyle S_n = \sum_{k=1}^n \frac{1}{k(k+1)}$
equals $\displaystyle 1 - \frac{1}{n+1}$.

\medskip
\noindent\textbf{What $P(n)$ to choose?}
The statement is handed to us: $P(n)$: $S_n = 1 - \frac{1}{n+1}$.
Here the step is: $S_{n+1} = S_n + \frac{1}{(n+1)(n+2)}$.
Substitute $S_n = 1 - \frac{1}{n+1}$ and simplify.
The inductive step works because $P(n)$ gives the exact
closed form needed.

\medskip
\noindent\textbf{Lesson.}
When the claim is a closed-form identity, $P(n)$ is the identity
itself. The step is always: split off the last term, apply the
inductive hypothesis, simplify algebraically.
\end{example}

\begin{remark}[Diagnostic questions for choosing $P(n)$]
When stuck choosing $P(n)$:
\begin{enumerate}
  \item What exactly do I need to prove for $n = 0, 1, 2, 3$?
    Write it out. The pattern of $P(n)$ will appear.
  \item Does the inductive step require knowing $P(n-1)$ as well
    as $P(n)$? If yes, switch to strong induction.
  \item Does the inductive step produce a bound or equation that
    is slightly weaker than $P(n+1)$? If yes, strengthen $P(n)$.
\end{enumerate}
\end{remark}

% =========================================================
% Choosing P(n): Axiomatic and Recursive Settings
% =========================================================

\paragraph{Choosing $P(n)$ in Axiomatic and Recursive Settings.}
\label{sec:choosing-pn-axiomatic}

The examples above assume $P(n)$ is a formula handed to you or
readable from a sequence. In axiomatic settings (e.g., Tao
Chapter~2), the goal is stated as a universal claim about a
recursively defined operation, and $P(n)$ must be
\emph{constructed} from the logical form of the goal.
Two rules govern this construction.

\begin{remark}[Rule 1: $P(n)$ mirrors the logical form of the goal]
Every universal claim has the form $\forall n,\ \Phi(n)$.
Strip the universal quantifier. What remains is $P(n)$.

\medskip
\noindent The shape of $P(n)$ is determined by $\Phi$:

\medskip
\noindent
\begin{tabular}{ll}
\toprule
\textbf{Goal form} & \textbf{Shape of $P(n)$} \\
\midrule
$\forall n,\ f(n) = g(n)$ & Equality: $P(n) :\equiv f(n) = g(n)$ \\[4pt]
$\forall n,\ A(n) \Rightarrow B(n)$ & Conditional: $P(n) :\equiv A(n) \Rightarrow B(n)$ \\[4pt]
$\forall n,\ A(n) \Leftrightarrow B(n)$ & Biconditional: $P(n) :\equiv A(n) \Leftrightarrow B(n)$ \\[4pt]
$\forall n,\ \varphi(n)$ & Simple property: $P(n) :\equiv \varphi(n)$ \\
\bottomrule
\end{tabular}

\medskip
\noindent Do not add or remove hypotheses. Do not simplify.
Transcribe the logical form exactly.
\end{remark}

\begin{remark}[Rule 2: Induct on the recursive variable]
When the goal involves a recursively defined operation,
identify which variable the definition recurses on.
Induct on that variable and hold all others fixed.

\medskip
\noindent To find the recursive variable: locate the definition
of the operation and identify which variable is reduced in the
recursive clause.

\medskip
\noindent
\begin{tabular}{lll}
\toprule
\textbf{Definition} & \textbf{Recursive clause} & \textbf{Induct on} \\
\midrule
Addition (Tao A1--A2) & $(n\pp) + m := (n + m)\pp$ & $n$ \\[4pt]
Multiplication (Tao M1--M2) & $(n\pp) \times m := (n \times m) + m$ & $n$ \\[4pt]
Exponentiation (Tao) & $m^{n\pp} := m^n \times m$ & $n$ \\
\bottomrule
\end{tabular}

\medskip
\noindent The non-recursive variable is held fixed throughout
the induction and treated as an arbitrary element of $\mathbb{N}$.
\end{remark}

\begin{example}[Constructing $P(n)$ for a conditional goal]
\textbf{Goal.} Prove Lemma~2.3.3: for all $n, m \in \mathbb{N}$,
if $n > 0$ and $m > 0$ then $nm > 0$.

\medskip
\noindent\textbf{Step 1 — Read the logical form.}
The goal is a universal conditional:
$\forall n,\ (n > 0 \wedge m > 0) \Rightarrow nm > 0$.
Strip the quantifier:
\[
P(n) :\equiv \text{if } n > 0 \text{ and } m > 0
\text{ then } n \times m > 0.
\]

\noindent\textbf{Step 2 — Identify the recursive variable.}
The operation is multiplication. M2 recurses on $n$:
$(n\pp) \times m := (n \times m) + m$.
Induct on $n$, hold $m$ fixed.

\medskip
\noindent\textbf{Step 3 — Check the base case form.}
$P(0)$ states: if $0 > 0$ and $m > 0$ then $0 \times m > 0$.
The hypothesis $0 > 0$ is false, so $P(0)$ is vacuously true.
A conditional $P(n)$ will often produce a vacuous base case
when the hypothesis excludes $n = 0$.
\end{example}

\begin{remark}[Vacuous base cases are a signal, not a problem]
When $P(n)$ is a conditional and its hypothesis fails at the base
value, the base case is vacuously true by definition.
This is expected and requires no further argument.
It signals that the claim only has content for values of $n$
where the hypothesis holds --- positivity, nonzero-ness, a bound,
or any other precondition that excludes the base case.
The induction then establishes the claim for all $n$ where the
hypothesis is satisfied.
\end{remark}


\begin{remark}[Diagnostic questions for axiomatic settings]
When constructing $P(n)$ in an axiomatic proof:
\begin{enumerate}
  \item What is the logical form of the goal?
    Write it out with explicit quantifiers before choosing $P(n)$.
  \item Which operation is involved?
    Find its recursive definition and identify the recursive variable.
  \item Does the base case require $n = 0$ to satisfy a hypothesis?
    If the hypothesis fails at $n = 0$, expect a vacuous base case.
  \item After writing $P(n\pp)$, can I apply the recursive definition
    to expose $P(n)$ or an earlier lemma?
    If not, the wrong variable may have been chosen.
\end{enumerate}
\end{remark}
