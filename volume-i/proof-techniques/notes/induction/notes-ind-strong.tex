% =========================================================
% Strong Induction
% =========================================================
\begin{tcolorbox}[colback=propbox, colframe=propborder, arc=2pt,
  left=6pt, right=6pt, top=4pt, bottom=4pt,
  title={\small\textbf{Theorem: Strong Induction}},
  fonttitle=\small\bfseries]
Let $P(n)$ be a statement depending on $n \in \mathbb{N}$.
Suppose that for every $n \in \mathbb{N}$:
\[
\Big(\forall k < n,\; P(k)\Big) \Rightarrow P(n).
\]
Then $P(n)$ holds for all $n \in \mathbb{N}$.
\end{tcolorbox}

\begin{remark}[The base case in strong induction]
When $n = 0$, the hypothesis $\forall k < 0,\; P(k)$ is vacuously
true (no $k < 0$ exists in $\mathbb{N}$). So the step for $n = 0$
reduces to showing $P(0)$ unconditionally --- this is the base case,
and it is implicit in the universal statement rather than listed
separately. In practice, still write the base case explicitly.
\end{remark}

\begin{remark}[When to use strong induction]
Use strong induction when the proof of $P(n+1)$ requires knowing
$P(k)$ for some $k < n$ that is not specifically $k = n$.

The diagnostic: in the inductive step, write
$P(n+1) = f(P(?), P(?), \ldots)$ and ask what values of $?$ you need.
If the answer includes anything other than $n$, use strong induction.

Canonical situations:
\begin{itemize}
  \item Recursive sequences: $a_{n+1} = a_n + a_{n-1}$ (need $n-1$).
  \item Prime factorization: $n+1 = ab$ requires $P(a)$ and $P(b)$
    where $a, b < n+1$ but are otherwise arbitrary.
  \item Any argument that breaks $n+1$ into parts.
\end{itemize}
\end{remark}

\begin{tcolorbox}[colback=gray!4, colframe=gray!35, arc=2pt,
  left=6pt, right=6pt, top=4pt, bottom=4pt,
  title={\small\textbf{Template: Strong Induction Proof}},
  fonttitle=\small\bfseries]
Let $P(n)$ denote: [\emph{write it out}].

\medskip
\noindent\textbf{Base case} ($n = [\text{start}]$).
[\emph{Verify directly.}]

\medskip
\noindent\textbf{Inductive step.}
Let $n > [\text{start}]$ and assume $P(k)$ holds for all
$[\text{start}] \leq k < n$.
\textit{(Strong inductive hypothesis.)}
We show $P(n)$ holds.

[\emph{Work. Use $P(k)$ for some specific $k < n$.}]

Hence $P(n)$ holds.

By strong induction, $P(n)$ holds for all $n \geq [\text{start}]$. \qed
\end{tcolorbox}

\begin{example}[Every integer $n \geq 2$ has a prime factor]
\textit{Proof.}
Let $P(n)$: ``$n$ has a prime factor''.

\medskip
\noindent\textbf{Base case} ($n = 2$).
$2$ is prime, so $2$ is a prime factor of itself.

\medskip
\noindent\textbf{Inductive step.}
Let $n > 2$ and assume $P(k)$ holds for all $2 \leq k < n$.

\textbf{Case 1}: $n$ is prime. Then $n$ is its own prime factor.

\textbf{Case 2}: $n$ is composite. Then $n = ab$ for some
$2 \leq a, b < n$.
By the strong inductive hypothesis applied to $a$,
$a$ has a prime factor $p$.
Then $p \mid a$ and $a \mid n$, so $p \mid n$.

In both cases $n$ has a prime factor. By strong induction,
$P(n)$ holds for all $n \geq 2$. \qed
\end{example}

\begin{remark}[Why weak induction cannot do this directly]
In the prime factor proof, the composite case splits $n$ into $a < n$.
We need $P(a)$, not $P(n-1)$. Weak induction only provides $P(n-1)$,
which is useless if $a \neq n-1$.
Strong induction provides $P(k)$ for all $k < n$,
covering whatever value $a$ takes.
\end{remark}
