% =========================================================
% Well-Ordering and Equivalence
% =========================================================
\noindent\begin{tcolorbox}[colback=propbox, colframe=propborder, arc=2pt,
  left=6pt, right=6pt, top=4pt, bottom=4pt,
  title={\small\textbf{Well-Ordering Principle}},
  fonttitle=\small\bfseries]
Every nonempty subset of $\mathbb{N}$ has a least element.
\end{tcolorbox}
\label{sec:wellordering}

\begin{remark}[Well-ordering as a proof engine]
The well-ordering principle drives the \emph{minimal counterexample}
argument, which is the well-ordering dual of induction:

\begin{enumerate}
  \item Suppose $P(n)$ fails for some $n$.
  \item Let $S = \{n \in \mathbb{N} : \neg P(n)\}$.
    By assumption $S \neq \emptyset$.
  \item By well-ordering, $S$ has a least element $m$.
  \item Derive a contradiction: show either $P(m)$ holds
    (contradicting $m \in S$) or some $k < m$ satisfies
    $k \in S$ (contradicting minimality of $m$).
\end{enumerate}

This is structurally the same as induction but runs in the
opposite direction: instead of climbing up from a base case,
you descend to a minimal failure and derive a contradiction.
\end{remark}

\begin{proposition}[Equivalence of induction and well-ordering]
The following are equivalent over the axioms of $\mathbb{N}$
without P5:
\begin{enumerate}
  \item The induction principle (weak induction).
  \item The well-ordering principle.
  \item The strong induction principle.
\end{enumerate}
\end{proposition}

\begin{remark}[Proof sketch of equivalence]
\textbf{Induction $\Rightarrow$ Well-ordering.}
Suppose $S \subseteq \mathbb{N}$ has no least element.
Define $P(n)$: ``$n \notin S$''. Then $P(0)$ holds
(else $0 \in S$ would be the least element). If $P(k)$ holds
for all $k \leq n$ then $n+1 \notin S$ (else $n+1$ would be
a least element). By induction, $P(n)$ holds for all $n$,
so $S = \emptyset$.

\textbf{Well-ordering $\Rightarrow$ Induction.}
Suppose the base case and inductive step hold.
Let $S = \{n : \neg P(n)\}$. If $S \neq \emptyset$, let $m$ be
its least element. Then $m \neq 0$ (base case), so $m \geq 1$,
so $m - 1 < m$ satisfies $P(m-1)$ (minimality of $m$).
But by the inductive step, $P(m-1) \Rightarrow P(m)$,
contradicting $m \in S$. Hence $S = \emptyset$.
\end{remark}

\begin{tcolorbox}[colback=gray!4, colframe=gray!35, arc=2pt,
  left=6pt, right=6pt, top=4pt, bottom=4pt,
  title={\small\textbf{Template: Minimal Counterexample}},
  fonttitle=\small\bfseries]
Suppose for contradiction that $P(n)$ fails for some $n \in \mathbb{N}$.

Let $S = \{n \in \mathbb{N} : \neg P(n)\}$.
By assumption $S \neq \emptyset$.
By well-ordering, $S$ has a least element $m$.

[\emph{Derive a contradiction using minimality of $m$.}]

This contradicts [minimality of $m$ / the base case].
Therefore $P(n)$ holds for all $n \in \mathbb{N}$. \qed
\end{tcolorbox}

\begin{example}[Minimal counterexample: divisibility]
\textbf{Claim.} For all $n \geq 1$, $2^n > n$.

\medskip
\noindent\textit{Proof (minimal counterexample).}
Suppose the claim fails. Let
$S = \{n \geq 1 : 2^n \leq n\}$.
By assumption $S \neq \emptyset$; let $m$ be its least element.

Then $m \geq 2$ (since $2^1 = 2 > 1$), so $m - 1 \geq 1$
and $m - 1 < m$, hence $m - 1 \notin S$, so $2^{m-1} > m - 1$.
Then:
\[
2^m = 2 \cdot 2^{m-1} > 2(m-1) = 2m - 2 \geq m
\]
(since $m \geq 2$), contradicting $m \in S$.
Therefore $S = \emptyset$ and the claim holds. \qed
\end{example}
