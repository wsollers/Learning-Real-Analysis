% =========================================================
% Common Induction Mistakes
% =========================================================
\paragraph{Common Induction Mistakes.}

\begin{remark}[Mistake 1: Missing base case]
A proof that establishes the inductive step but not the base case
proves nothing. The inductive step alone says: \emph{if} $P(n)$
holds for some $n$, then $P(n+1)$ holds. Without a base case,
there is no $n$ for which $P(n)$ is known to hold, so the
chain never starts.

\textbf{Classic fake theorem}: ``All horses are the same colour.''
\textit{Proof attempt}: Base case $n=1$: one horse is trivially
the same colour as itself. Inductive step: [flawed argument].
The flaw is in the inductive step, but the lesson is that
even a correct base case does not rescue a bad step.
Scrutinize both parts independently.
\end{remark}

\begin{remark}[Mistake 2: Circular inductive step]
The inductive step must prove $P(n+1)$ using only $P(n)$ (or
earlier cases in strong induction) and previously established
facts. It is illegal to assume $P(n+1)$ anywhere in the proof
of $P(n+1)$.

This error is most common when the student ``derives'' both
sides of an equation and meets in the middle, inadvertently
assuming the conclusion. Always work in one direction: from
$P(n)$, derive $P(n+1)$.
\end{remark}

\begin{remark}[Mistake 3: Wrong or missing inductive hypothesis]
The most common error in practice is writing ``assume $P(n)$''
without stating what $P(n)$ says. This is a sign that $P(n)$
has not been written down as a full sentence. Write it out.
If you cannot state $P(n)$ as a complete mathematical sentence,
you do not yet know what you are inducting on.
\end{remark}

\begin{remark}[Mistake 4: Off-by-one base case]
Induction proves $P(n)$ for all $n \geq n_0$, where $n_0$ is
the base case. If the claim is only true for $n \geq 2$,
the base case $n = 0$ may fail, and the proof should start at
$n = 2$.

Check: what is the smallest $n$ for which the claim is true?
That is your base case. Do not default to $n = 0$ or $n = 1$
without verifying.
\end{remark}

\begin{remark}[Mistake 5: Using strong induction when weak suffices]
Strong induction is not more powerful than weak induction in the
sense that they prove the same class of statements. But a proof
that uses the full strength of the strong hypothesis when only
$P(n)$ is needed is unnecessarily complicated. Use weak induction
unless the step genuinely requires knowing $P(k)$ for some
$k < n$ other than $k = n - 1$.
\end{remark}

\begin{remark}[Mistake 6: Forgetting vacuity in strong induction]
In strong induction, the case $n = 0$ has a vacuously true
hypothesis $\forall k < 0,\; P(k)$, since no natural number
is less than $0$. This means $P(0)$ must be proved directly,
not derived from the inductive hypothesis. Many students
write ``by the inductive hypothesis applied to $n-1$'' when
$n = 0$ has no predecessor. The base case is always proved
from scratch.
\end{remark}
