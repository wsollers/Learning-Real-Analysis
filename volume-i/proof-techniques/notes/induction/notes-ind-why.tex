% =========================================================
% Why Induction Works
% =========================================================
\paragraph{Why Induction Works.}

Induction feels like a technique. It is actually a theorem.
It follows from the axiomatic structure of $\mathbb{N}$.

\begin{remark}[The Peano axiom behind induction]
The Peano axioms characterize $\mathbb{N}$ as the smallest
inductive set: a set containing $0$ and closed under the
successor function $S$. Formally, Axiom P5 states:

\medskip
If $A \subseteq \mathbb{N}$ satisfies (i) $0 \in A$ and
(ii) $n \in A \Rightarrow S(n) \in A$, then $A = \mathbb{N}$.

\medskip
The induction principle is a direct restatement: if $P(0)$ holds
and $P(n) \Rightarrow P(S(n))$, then defining
$A = \{n \in \mathbb{N} : P(n)\}$ gives an inductive set,
hence $A = \mathbb{N}$, hence $P(n)$ holds for all $n$.
The full treatment is in the Axiom Systems chapter.
\end{remark}

\begin{remark}[Induction and well-ordering are equivalent]
The induction principle and the well-ordering principle for
$\mathbb{N}$ are logically equivalent. The well-ordering principle
states: every nonempty subset of $\mathbb{N}$ has a least element.

This equivalence means every induction argument can be rephrased
as a minimal counterexample argument, and vice versa.
Both are available as proof tools; the choice between them
is a matter of which formulation makes the argument cleaner.
The equivalence is proved in Section~\ref{sec:wellordering}.
\end{remark}

\begin{remark}[The domino picture is misleading]
The common intuition ``induction is like an infinite row of
dominoes'' is useful but imprecise. It suggests that induction
requires each domino to knock over the next, which captures the
inductive step but misses the point that the base case is what
starts the chain. More importantly, it does not explain why
the argument is valid --- it is valid because $\mathbb{N}$ is
defined to be the smallest inductive set, not because of any
physical metaphor. The domino picture is a mnemonic, not a proof.
\end{remark}
