% =========================================================
% Weak Induction
% =========================================================
\begin{tcolorbox}[colback=propbox, colframe=propborder, arc=2pt,
  left=6pt, right=6pt, top=4pt, bottom=4pt,
  title={\small\textbf{Theorem: Weak Induction}},
  fonttitle=\small\bfseries]
Let $P(n)$ be a statement depending on $n \in \mathbb{N}$.
Suppose:
\begin{enumerate}
  \item \textbf{Base case.} $P(0)$ is true.
  \item \textbf{Inductive step.} For all $n \in \mathbb{N}$,
    $P(n) \Rightarrow P(n+1)$.
\end{enumerate}
Then $P(n)$ is true for all $n \in \mathbb{N}$.
\end{tcolorbox}

\begin{tcolorbox}[colback=gray!4, colframe=gray!35, arc=2pt,
  left=6pt, right=6pt, top=4pt, bottom=4pt,
  title={\small\textbf{Template: Weak Induction Proof}},
  fonttitle=\small\bfseries]
Let $P(n)$ denote the statement: [\emph{write it out explicitly}].

\medskip
\noindent\textbf{Base case} ($n = [\text{start}]$).
[\emph{Verify $P(\text{start})$ directly.}]
Hence $P(\text{start})$ holds.

\medskip
\noindent\textbf{Inductive step.}
Let $n \in \mathbb{N}$ and assume $P(n)$ holds.
[\emph{This is the inductive hypothesis --- state it explicitly.}]
We show $P(n+1)$ holds.

[\emph{Work. Use $P(n)$ at least once.}]

Hence $P(n+1)$ holds.

\medskip
\noindent By induction, $P(n)$ holds for all $n \geq [\text{start}]$. \qed
\end{tcolorbox}

\begin{remark}[The base case is not trivial]
The base case is logically necessary, not a formality.
A proof that omits it is invalid, regardless of how clean the
inductive step is. However, the base case is usually the easiest
part. Do not spend more than a line or two on it.
\end{remark}

\begin{remark}[State the inductive hypothesis explicitly]
The single most important discipline in induction proofs is
writing the inductive hypothesis as a full sentence before
the inductive step begins. ``Assume $P(n)$ holds'' is not
sufficient. Write: ``Assume $\sum_{k=1}^n k = \frac{n(n+1)}{2}$.''
This forces you to know what you are allowed to use,
and it is the step most students skip.
\end{remark}

\begin{example}[Sum formula]
\textbf{Proposition.}
For all $n \geq 1$:
$\displaystyle\sum_{k=1}^n k = \frac{n(n+1)}{2}$.

\medskip
\noindent\textit{Proof.}
Let $P(n)$ denote the statement
$\displaystyle\sum_{k=1}^n k = \frac{n(n+1)}{2}$.

\medskip
\noindent\textbf{Base case} ($n=1$).
$\displaystyle\sum_{k=1}^1 k = 1 = \frac{1 \cdot 2}{2}$.
Hence $P(1)$ holds.

\medskip
\noindent\textbf{Inductive step.}
Let $n \geq 1$ and assume
$\displaystyle\sum_{k=1}^n k = \frac{n(n+1)}{2}$. \quad
\textit{(Inductive hypothesis.)}

We show $\displaystyle\sum_{k=1}^{n+1} k = \frac{(n+1)(n+2)}{2}$.

\begin{align*}
\sum_{k=1}^{n+1} k
&= \left(\sum_{k=1}^n k\right) + (n+1) \\
&= \frac{n(n+1)}{2} + (n+1)
  \tag{by inductive hypothesis} \\
&= (n+1)\left(\frac{n}{2} + 1\right)
  = \frac{(n+1)(n+2)}{2}.
\end{align*}

Hence $P(n+1)$ holds.

By induction, $P(n)$ holds for all $n \geq 1$. \qed
\end{example}

\begin{example}[Divisibility]
\textbf{Proposition.}
For all $n \geq 0$, $3 \mid (4^n - 1)$.

\medskip
\noindent\textit{Proof.}
Let $P(n)$ denote ``$3 \mid (4^n - 1)$''.

\medskip
\noindent\textbf{Base case} ($n=0$).
$4^0 - 1 = 0 = 3 \cdot 0$. Hence $3 \mid 0$, so $P(0)$ holds.

\medskip
\noindent\textbf{Inductive step.}
Assume $3 \mid (4^n - 1)$, i.e., $4^n - 1 = 3m$ for some
$m \in \mathbb{Z}$.

We compute:
\[
4^{n+1} - 1 = 4 \cdot 4^n - 1
= 4(4^n - 1) + 4 - 1
= 4 \cdot 3m + 3
= 3(4m + 1).
\]
Hence $3 \mid (4^{n+1}-1)$, so $P(n+1)$ holds.

By induction, $P(n)$ holds for all $n \geq 0$. \qed
\end{example}

\begin{remark}[Anatomy of both examples]
In both examples, the inductive step follows a single pattern:
\emph{rewrite $P(n+1)$ to expose $P(n)$ inside it, then substitute
the inductive hypothesis.} This rewriting step is the core skill
of weak induction. When it is not obvious how to expose $P(n)$,
the statement $P(n)$ may need to be strengthened ---
see Section~\ref{sec:choosing-pn}.
\end{remark}
