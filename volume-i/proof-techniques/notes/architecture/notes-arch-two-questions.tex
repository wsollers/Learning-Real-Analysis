% =========================================================
% The Two Questions
% =========================================================
\paragraph{The Two Questions Before You Write Anything.}

Before writing a single symbol, ask two questions in order.

\begin{tcolorbox}[colback=propbox, colframe=propborder, arc=2pt,
  left=6pt, right=6pt, top=4pt, bottom=4pt]
\textbf{Question 1.} \emph{What is the logical form of the statement?}

Read the claim and identify its outermost logical structure:
universal, conditional, biconditional, existential, uniqueness,
equality, or structural assertion.

\medskip
\textbf{Question 2.} \emph{What proof strategy does this form suggest?}

The form determines the skeleton of the proof mechanically,
before any mathematics is done.
\end{tcolorbox}

\begin{remark}[Why order matters]
Question 2 cannot be answered before Question 1.
The most common proof error is reaching for a familiar strategy
--- contradiction is the usual default --- without checking
whether it matches the statement form. Read the statement first.
The form is not preliminary; it is the first mathematical act
of the proof.
\end{remark}

\begin{example}[Applying the two questions]
Statement: \emph{The identity element of a group is unique.}

\medskip
\noindent\textbf{Q1.} Logical form:
$\forall G$ (group), $\forall e, e' \in G$,
[both satisfy the identity axiom] $\Rightarrow e = e'$.
This is a universal statement with a uniqueness conclusion.

\medskip
\noindent\textbf{Q2.} Strategy: introduce an arbitrary group $G$,
assume two identity elements $e$ and $e'$ exist, show $e = e'$.
This is the \emph{assume-two-and-compare} pattern
(see Section~\ref{sec:proof-structures}).

\medskip
The strategy is forced entirely by the form.
No creative insight is required to select it.
\end{example}
