% =========================================================
% Statement Forms → Proof Strategies Map
% =========================================================
\paragraph{Statement Forms and Their Proof Strategies.}

The following table maps every common statement form to the proof
strategy it demands. This table should be consulted at the start
of every proof until the correspondences are automatic.

\begin{center}
\renewcommand{\arraystretch}{1.4}
\begin{tabular}{p{0.30\textwidth} p{0.30\textwidth} p{0.30\textwidth}}
\toprule
\textbf{Statement Form} & \textbf{Proof Strategy} & \textbf{Opening Move} \\
\midrule
$\forall x,\; P(x)$ &
  Introduce arbitrary element &
  ``Let $x$ be arbitrary.'' \\

$P \Rightarrow Q$ &
  Direct proof &
  ``Assume $P$.'' \\

$P \Rightarrow Q$ &
  Contrapositive &
  ``Assume $\neg Q$.'' \\

$\neg Q \Rightarrow \neg P$\newline (same as above) &
  Contradiction &
  ``Assume $P$ and $\neg Q$.'' \\

$P \Leftrightarrow Q$ &
  Two directions &
  Prove $P \Rightarrow Q$, then $Q \Rightarrow P$. \\

$\exists x,\; P(x)$ &
  Construct a witness &
  ``Define $x := \ldots$ and verify $P(x)$.'' \\

$\exists! x,\; P(x)$ &
  Existence + uniqueness &
  Construct witness; then assume-two-and-compare. \\

$A = B$ (sets) &
  Double inclusion &
  Prove $A \subseteq B$ and $B \subseteq A$. \\

$a = b$ (elements) &
  Satisfy-and-cite or algebra &
  Show $a$ satisfies the defining property of $b$. \\

$P(n)$ for all $n \in \mathbb{N}$ &
  Induction &
  Base case; inductive step. \\

``$X$ is unique'' &
  Assume-two-and-compare &
  ``Let $x, y$ both satisfy the definition.'' \\
\bottomrule
\end{tabular}
\end{center}

\begin{remark}[Multiple strategies for one form]
Some statement forms admit more than one strategy.
$P \Rightarrow Q$ can be proved directly, by contrapositive,
or by contradiction. The table lists the most natural default.
Direct proof should be attempted first; contrapositive and
contradiction are reached for when the hypothesis is hard to
use forward or when the negation of the conclusion is
more tractable.
\end{remark}

\begin{remark}[The equality forms are different]
Notice that $A = B$ (set equality) and $a = b$ (element equality)
suggest different strategies. Set equality uses double inclusion
because sets are defined by their members.
Element equality in an algebraic context uses satisfy-and-cite:
show that $a$ satisfies the same defining property as $b$,
then invoke a previously proved uniqueness theorem.
See Section~\ref{sec:satisfy-cite}.
\end{remark}
