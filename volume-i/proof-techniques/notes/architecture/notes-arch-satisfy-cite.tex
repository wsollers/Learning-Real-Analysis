% =========================================================
% Satisfy-and-Cite Uniqueness
% =========================================================
\paragraph{The Satisfy-and-Cite Pattern.}
\label{sec:satisfy-cite}

The most commonly missing pattern at the undergraduate level is one
that does not appear in any of the standard proof archetypes by name,
yet appears constantly in abstract algebra and analysis.

\begin{definition}[Satisfy-and-Cite Pattern]
To prove that $a = b$ in a context where $b$ is a \emph{uniquely
determined} object (the unique identity, the unique inverse, the
unique limit, the unique fixed point), proceed as follows:
\begin{enumerate}
  \item Show that $a$ satisfies the defining property of $b$.
  \item Cite the previously proved uniqueness theorem for $b$.
  \item Conclude $a = b$.
\end{enumerate}
\end{definition}

\begin{remark}[Why this is different from assume-two-and-compare]
The assume-two-and-compare pattern \emph{proves} uniqueness:
it shows that any two objects satisfying a definition are equal.

The satisfy-and-cite pattern \emph{uses} a previously proved
uniqueness result: it shows that a specific object $a$ equals
a specific named object $b$ by showing $a$ satisfies $b$'s
defining property, then invoking uniqueness to collapse $a$ to $b$.

Assume-two-and-compare appears in the proof that the identity is unique.
Satisfy-and-cite appears in every proof that uses that uniqueness
theorem afterward.
\end{remark}

\begin{example}[Socks-shoes property: $(ab)^{-1} = b^{-1}a^{-1}$]
We want to show $(ab)^{-1} = b^{-1}a^{-1}$.

The inverse of $ab$ is the unique element $x$ such that
$(ab)x = e$ and $x(ab) = e$.

We compute:
\[
(ab)(b^{-1}a^{-1})
= a(bb^{-1})a^{-1}
= a e a^{-1}
= aa^{-1} = e.
\]
Similarly $(b^{-1}a^{-1})(ab) = e$.

So $b^{-1}a^{-1}$ \emph{satisfies the defining property of
$(ab)^{-1}$}. By uniqueness of inverses (previously proved),
$(ab)^{-1} = b^{-1}a^{-1}$. \qed

\medskip
\noindent Notice: we never assumed two inverses existed and compared
them. We showed one specific element satisfies the inverse definition,
then cited uniqueness.
\end{example}

\begin{example}[Ring theorem: $a \cdot (-b) = -(ab)$]
We want to show $a(-b) = -(ab)$.

The additive inverse $-(ab)$ is the unique element $x$ such that
$ab + x = 0$.

We compute:
\[
ab + a(-b) = a(b + (-b)) = a \cdot 0 = 0.
\]
So $a(-b)$ \emph{satisfies the defining property of $-(ab)$}.
By uniqueness of additive inverses, $a(-b) = -(ab)$. \qed

\medskip
\noindent Same pattern: compute, satisfy, cite.
\end{example}

\begin{remark}[The pattern across mathematics]
Satisfy-and-cite appears wherever unique objects are defined:
\begin{itemize}
  \item Algebra: inverses, identities, quotients, kernels
  \item Analysis: limits, suprema, infima, fixed points
  \item Linear algebra: zero vector, additive inverses,
    linear transformations determined by a basis
\end{itemize}
Each time a uniqueness theorem is proved, it creates a new
tool that can be activated by the satisfy-and-cite pattern
for every future proof in that structure.
\end{remark}
