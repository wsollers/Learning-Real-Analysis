% =========================================================
% Proof Strategies Lookup Table
% =========================================================
\paragraph{Lookup Table: Stuck Situations and Strategies.}

\begin{remark}[How to use this table]
When progress on a proof stalls, locate the column that describes
your situation. The middle column names the strategy. The right
column gives the specific first move and a cross-reference to
a worked example.
\end{remark}

\begin{center}
\renewcommand{\arraystretch}{1.6}
\begin{tabular}{p{0.27\textwidth} p{0.22\textwidth} p{0.40\textwidth}}
\toprule
\textbf{Stuck because\ldots} &
\textbf{Strategy} &
\textbf{First move / See} \\
\midrule

I need to prove $P$ but don't know where to start &
  Two-question check &
  Write the logical form of $P$. Consult the statement map.
  \S\ref{sec:proof-architecture} \\

I need to show two things are equal and one is uniquely defined &
  Satisfy-and-cite &
  Show the first satisfies the definition of the second.
  Cite the uniqueness theorem.
  \S\ref{sec:satisfy-cite} \\

I need to show something is the only object of its kind &
  Assume-two-and-compare &
  ``Let $x, y$ both satisfy the definition. Show $x = y$.''
  \S\ref{sec:proof-structures} \\

I have a conditional $P \Rightarrow Q$ and $P$ is hard to use &
  Contrapositive &
  ``Assume $\neg Q$.'' Derive $\neg P$.
  \S\ref{sec:proof-structures} \\

I have a conditional and the conclusion is a negation or impossible &
  Contradiction &
  ``Assume $P$ and $\neg Q$.'' Derive any contradiction.
  \S\ref{sec:proof-structures} \\

The claim splits naturally by cases &
  Case analysis &
  List all cases explicitly. Verify they are exhaustive.
  \S\ref{sec:proof-structures} \\

The claim is about all $n \in \mathbb{N}$ &
  Induction &
  Write $P(n)$ as a full sentence. Prove base case.
  \S\ref{sec:induction} \\

The inductive step needs $P(k)$ for some $k < n$ (not just $n-1$) &
  Strong induction &
  Replace ``assume $P(n)$'' with
  ``assume $P(k)$ for all $k < n$''.
  \S\ref{sec:induction} \\

The claim is about all $n \geq n_0$ and feels easier to negate &
  Minimal counterexample &
  ``Suppose $S = \{n : \neg P(n)\} \neq \emptyset$. Let $m = \min S$.''
  \S\ref{sec:wellordering} \\

I need to prove $a \cdot 0 = 0$ or $a(-b) = -(ab)$ &
  Distributivity bridge &
  Write $0 = x + (-x)$; apply distributivity; use
  uniqueness of additive inverse.
  \S\ref{sec:tactics} \\

I have $ac = bc$ and need $a = b$ &
  Cancellation &
  Group: multiply by $c^{-1}$.
  Domain: write $(a-b)c = 0$; cite no zero divisors.
  \S\ref{sec:tactics} \\

I need to set two sets equal &
  Double inclusion &
  Prove $A \subseteq B$: chase an arbitrary $x \in A$.
  Then prove $B \subseteq A$.
  \S\ref{sec:proof-architecture} \\

A definition has not been unpacked &
  DU move &
  Identify the named concept and write out its definition.
  Tag the line \tagDU.
  \S\ref{sec:tactics} \\

A step uses an axiom or theorem by name &
  TA move &
  Cite the axiom or theorem explicitly by label.
  Tag the line \tagTA.
  \S\ref{sec:tactics} \\

\bottomrule
\end{tabular}
\end{center}

\begin{remark}[Cross-references to worked examples]
The following table maps each strategy to a worked example
in these notes where it is the primary tool.

\begin{center}
\renewcommand{\arraystretch}{1.4}
\begin{tabular}{p{0.35\textwidth} p{0.55\textwidth}}
\toprule
\textbf{Strategy} & \textbf{Primary worked examples} \\
\midrule
Assume-two-and-compare &
  Prop.~\ref{prop:group-identity-unique},
  Prop.~\ref{prop:group-inverse-unique} \\
Satisfy-and-cite &
  Prop.~\ref{prop:group-socks-shoes},
  Prop.~\ref{prop:ring-mult-neg} \\
Distributivity bridge &
  Prop.~\ref{prop:ring-mult-zero},
  Prop.~\ref{prop:ring-mult-neg} \\
Cancellation (group) &
  Prop.~\ref{prop:group-cancellation} \\
Cancellation (domain) &
  Prop.~\ref{prop:domain-cancellation} \\
Multiply by inverse + no zero divisors &
  Prop.~\ref{prop:field-is-domain} \\
Weak induction &
  Sum formula, divisibility examples (\S\ref{sec:induction}) \\
Strong induction &
  Prime factor theorem (\S\ref{sec:induction}) \\
Minimal counterexample &
  $2^n > n$ example (\S\ref{sec:wellordering}) \\
Contrapositive &
  $n^2$ even $\Rightarrow$ $n$ even (\S\ref{sec:proof-structures}) \\
Contradiction &
  $\sqrt{2} \notin \mathbb{Q}$ (\S\ref{sec:proof-structures}) \\
\bottomrule
\end{tabular}
\end{center}
\end{remark}
