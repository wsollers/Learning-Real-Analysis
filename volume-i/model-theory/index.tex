% =========================================================
% Model Theory — Planned
% =========================================================
\chapter{Model Theory}

\begin{tcolorbox}[
  colback=gray!6,
  colframe=gray!40,
  arc=2pt,
  left=8pt, right=8pt, top=6pt, bottom=6pt,
  title={\small\textbf{Where You Are in the Journey}},
  fonttitle=\small\bfseries
]
\begin{center}
\small
Propositional Logic
$\;\to\;$ Predicate Calculus
$\;\to\;$ Sets \& Functions
$\;\to\;$ Proof Techniques
$\;\to\;$ Axiom Systems
$\;\to\;$ \textbf{Model Theory}
$\;\to\;$ $\cdots$
\end{center}

\medskip
\noindent\textbf{How we got here.}
Predicate logic gave us the syntax and proof theory of first-order
languages. Model theory studies the relationship between formal theories
and their interpretations (models): which structures satisfy a given
set of axioms?

\medskip
\noindent\textbf{What this chapter will build.}
The compactness theorem, the Löwenheim--Skolem theorems, and the
study of complete and categorical theories.

\medskip
\noindent\textbf{Where this leads.}
Model theory provides the metamathematical foundations for understanding
how algebraic structures (groups, fields, rings) relate to their axioms.
\end{tcolorbox}

\vspace{1em}
\begin{tcolorbox}[
  colback=gray!6, colframe=gray!40, arc=2pt,
  left=8pt, right=8pt, top=6pt, bottom=6pt,
  title={\small\textbf{Status: Planned}},
  fonttitle=\small\bfseries
]
\begin{center}\Large\bfseries Coming Soon\end{center}
\vspace{6pt}
\noindent Notes, proofs, and exercises will appear here in a future revision.
\end{tcolorbox}

\section{Notes}
\textit{To be populated.}

\section{Proofs}
\textit{To be populated.}

\section{Capstone}
\textit{To be populated.}
