% =========================================================
% Breadcrumb
% =========================================================
\begin{tcolorbox}[
  colback=gray!6,
  colframe=gray!40,
  arc=2pt,
  left=8pt, right=8pt, top=6pt, bottom=6pt,
  title={\small\textbf{Where You Are in the Journey}},
  fonttitle=\small\bfseries
]
\begin{center}
\small
\textbf{Propositional Logic}
$\;\to\;$ Predicate Calculus
$\;\to\;$ Sets \& Functions
$\;\to\;$ Proof Techniques
$\;\to\;$ Axiom Systems
$\;\to\;$ $\mathbb{N}$, $\mathbb{Z}$, $\mathbb{Q}$, $\mathbb{R}$
$\;\to\;$ Real Analysis
$\;\to\;$ $\cdots$
\end{center}

\medskip
\noindent\textbf{How we got here.}
This is the starting point. Mathematics needs a precise language for
reasoning --- a system for forming statements, combining them with
connectives, and determining their truth mechanically.
Propositional logic provides that foundation.

\medskip
\noindent\textbf{What this chapter builds.}
We develop the syntax of propositional formulas, their semantics via
truth assignments, and the proof theory of natural deduction.
The central results --- soundness, completeness, and compactness ---
characterise what is provable and what follows logically.

\medskip
\noindent\textbf{Where this leads.}
Predicate logic extends this language with quantifiers, allowing us to
talk about objects and their properties rather than just truth values.
The metatheoretic pattern (syntax, semantics, soundness, completeness)
established here repeats at every level of the logical hierarchy.
\end{tcolorbox}
\vspace{1em}

% =========================================================
% Structural Roadmap
% =========================================================
\subsection*{Structural Roadmap}

Each major topic is organised as:
\begin{center}
\textbf{Syntax $\longrightarrow$ Semantics
$\longrightarrow$ Proof Theory $\longrightarrow$ Metatheory}
\end{center}

The global progression is:
\begin{enumerate}
  \item Syntax: propositional variables, connectives, well-formed formulas
  \item Semantics: truth assignments, validity, logical equivalence, normal forms
  \item Proof theory: inference rules, natural deduction, derivability
  \item Metatheory: functional completeness, compactness, Craig's interpolation
  \item Reference: fallacies and summary tables
\end{enumerate}

\vspace{1em}

\begin{remark}[Primary source]
The primary driver is Bjørndahl's \textit{Logic and Proof},
supplemented by Suppes' \textit{Introduction to Logic}.
\end{remark}

% =========================================================
% Content
% =========================================================
% =========================================================
% Propositional Logic Notes — Master Orchestrator
% =========================================================
% Subject subfolder structure:
%
%   syntax/        ← propositional variables, connectives, WFF,
%                     unique readability, parse tree, subformula,
%                     depth, precedence; XOR, NAND, NOR
%
%   semantics/     ← truth assignments, satisfaction, tautology/
%                     contradiction/contingency, logical consequence,
%                     logical equivalence; all equivalence laws;
%                     normal forms (NNF, CNF, DNF)
%
%   proof-theory/  ← inference rules (MP, MT, HS, DS, conj, disj,
%                     cond, bicond, neg, dilemma, structural);
%                     resolution (rule, refutation, soundness,
%                     completeness, strategies);
%                     proof systems (derivability, soundness,
%                     completeness, syntactic vs.\ semantic)
%
%   metatheory/    ← functional completeness (truth functions,
%                     adequate sets, Post's theorem);
%                     compactness (statement, consequence form,
%                     graph coloring application);
%                     interpolation (Craig's theorem, significance)
%
%   reference/     ← errors and fallacies; summary tables
%
% Dependency order: syntax → semantics → proof-theory → metatheory
% Reference is self-contained.
% =========================================================

% =========================================================
% Syntax — Orchestrator
% =========================================================
% Dependency order: variables and connectives must precede WFFs;
% unique readability follows WFF definition; subformulas and depth
% follow from the recursive WFF structure.

% =========================================================
% Syntax of Propositional Logic
% =========================================================

\subsection{Syntax of Propositional Logic}

% ---------------------------------------------------------
% TOOLKIT
% ---------------------------------------------------------
\begin{tcolorbox}[colback=gray!6, colframe=gray!40, arc=2pt,
  left=6pt, right=6pt, top=4pt, bottom=4pt,
  title={\small\textbf{Syntax Toolkit — Quick Reference}},
  fonttitle=\small\bfseries]
\small
\begin{tabular}{l l l}
\toprule
\textbf{Concept} & \textbf{Meaning} & \textbf{Detail} \\
\midrule
Propositional variable & Atomic symbol with no internal structure & \hyperref[def:prop-var]{↓ Def} \\
Logical connective     & Symbol forming compound formulas & \hyperref[def:connectives]{↓ Def} \\
Well-formed formula    & Recursively constructed expression & \hyperref[def:wff]{↓ Def} \\
Unique readability     & Every wff has exactly one parse tree & \hyperref[thm:unique-read]{↓ Thm} \\
Parse tree             & Tree representation of formula structure & \hyperref[def:parse-tree]{↓ Def} \\
Subformula             & Constituents of a formula & \hyperref[def:subformula]{↓ Def} \\
Formula depth          & Nesting depth of connectives & \hyperref[def:depth]{↓ Def} \\
Operator precedence    & Disambiguation convention & \hyperref[def:precedence]{↓ Def} \\
\bottomrule
\end{tabular}
\end{tcolorbox}

\vspace{1em}

% ---------------------------------------------------------
% Propositional Variables
% ---------------------------------------------------------
\begin{tcolorbox}[colback=propbox, colframe=propborder, arc=2pt,
  left=6pt, right=6pt, top=4pt, bottom=4pt,
  title={\small\textbf{Definition (Propositional Variable)}},
  fonttitle=\small\bfseries]
\label{def:prop-var}
A \emph{propositional variable} (or \emph{atomic proposition}) is a primitive
symbol representing a statement that is either true or false.

The set of all propositional variables is denoted
\[
\mathsf{Prop} = \{P_1, P_2, P_3, \dots\}
\]
or informally by letters $P, Q, R, S, \dots$
\end{tcolorbox}

\begin{remark}[English reading]
A propositional variable is the smallest meaningful unit in propositional logic.
It names a proposition — ``It is raining,'' ``The number is prime'' — but has
no further decomposable structure. Atomicity is the defining feature: unlike
predicate logic, we cannot look inside a variable and ask \emph{who} or
\emph{what} it talks about.
\end{remark}

\begin{remark}[Fully formal statement]
$\mathsf{Prop}$ is a countably infinite set of symbols, pairwise distinct.
A propositional variable is any element $P_i \in \mathsf{Prop}$.
No propositional variable is a connective, a parenthesis, or a formula built
from other symbols.
\end{remark}

\begin{remark}[Consequence for proof strategy]
Proofs about \emph{all} propositional variables are trivial base cases in
structural induction: if $\varphi \in \mathsf{Prop}$, the property holds by
the atomic case of the induction.
\end{remark}

% ---------------------------------------------------------
% Logical Connectives
% ---------------------------------------------------------
\begin{tcolorbox}[colback=propbox, colframe=propborder, arc=2pt,
  left=6pt, right=6pt, top=4pt, bottom=4pt,
  title={\small\textbf{Definition (Logical Connectives)}},
  fonttitle=\small\bfseries]
\label{def:connectives}
The \emph{logical connectives} of propositional logic are:

\begin{center}
\renewcommand{\arraystretch}{1.3}
\begin{tabular}{cccl}
\toprule
\textbf{Symbol} & \textbf{Name} & \textbf{Arity} & \textbf{Reading} \\
\midrule
$\neg$           & Negation       & Unary  & ``not $P$'' \\
$\wedge$         & Conjunction    & Binary & ``$P$ and $Q$'' \\
$\vee$           & Disjunction    & Binary & ``$P$ or $Q$'' \\
$\rightarrow$    & Conditional    & Binary & ``if $P$ then $Q$'' \\
$\leftrightarrow$& Biconditional  & Binary & ``$P$ if and only if $Q$'' \\
\bottomrule
\end{tabular}
\end{center}
\end{tcolorbox}

\begin{remark}[English reading]
Connectives are the glue. They take propositions as input and produce
propositions as output. $\neg$ flips a truth value; binary connectives
combine two truth values into one according to a fixed truth table.
\end{remark}

\begin{remark}[Notation variants across sources]
$\sim$ or $!$ for $\neg$;\quad $\&$, $\cdot$, or $\&\&$ for $\wedge$;\quad
$|$, $+$ for $\vee$;\quad $\supset$, $\Rightarrow$ for $\rightarrow$;\quad
$\equiv$, $\Leftrightarrow$ for $\leftrightarrow$.
Bjørndahl uses $\neg, \wedge, \vee, \rightarrow$. Suppes \& Hill use similar.
Always check the source's symbol table.
\end{remark}

\begin{remark}[Logical implication]
Five connectives suffice for propositional logic, but not all are independent.
$\leftrightarrow$ is definable as $(P \rightarrow Q) \wedge (Q \rightarrow P)$;
$\rightarrow$ is definable as $\neg P \vee Q$. The standard five are chosen for
readability, not minimality.
\end{remark}

% ---------------------------------------------------------
% Well-Formed Formulas
% ---------------------------------------------------------
\begin{tcolorbox}[colback=propbox, colframe=propborder, arc=2pt,
  left=6pt, right=6pt, top=4pt, bottom=4pt,
  title={\small\textbf{Definition (Well-Formed Formula)}},
  fonttitle=\small\bfseries]
\label{def:wff}
Let $\mathcal{L}$ be a propositional language. The set
$\mathsf{WFF}_{\mathcal{L}}$ of \emph{well-formed formulas} is the smallest set
satisfying:

\begin{enumerate}
  \item \textbf{Atomic:} Every $P \in \mathsf{Prop}$ is a wff.
  \item \textbf{Negation:} If $\varphi$ is a wff, so is $\neg\varphi$.
  \item \textbf{Binary:} If $\varphi, \psi$ are wffs, so are
        $(\varphi \wedge \psi)$, $(\varphi \vee \psi)$,
        $(\varphi \rightarrow \psi)$, $(\varphi \leftrightarrow \psi)$.
  \item \textbf{Closure:} Nothing else is a wff.
\end{enumerate}
\end{tcolorbox}

\begin{remark}[English reading]
A wff is any expression that can be produced by starting with atomic variables
and repeatedly applying connectives according to the rules above. The closure
clause (rule 4) is crucial: it rules out nonsense strings like
$\rightarrow P \wedge$ that no finite application of rules~1--3 can produce.
\end{remark}

\begin{remark}[Fully quantified form]
$\mathsf{WFF}$ is the intersection of all sets $X$ satisfying:
(i) $\mathsf{Prop} \subseteq X$;
(ii) $\varphi \in X \Rightarrow \neg\varphi \in X$;
(iii) $\varphi, \psi \in X \Rightarrow (\varphi \circ \psi) \in X$ for each
binary connective $\circ$.
This is the standard \emph{inductive definition} / least fixed-point construction.
\end{remark}

\begin{remark}[Consequence — structural induction]
Because $\mathsf{WFF}$ is inductively defined, \emph{structural induction}
applies: to prove property $\mathcal{P}(\varphi)$ for all wffs, prove it for
atomic $\varphi$, and show that each formation rule preserves $\mathcal{P}$.
This is the standard proof technique for syntactic results in logic.
\end{remark}

\begin{definition}[Atomic and Molecular Formulas]
A formula is \emph{atomic} if it is a propositional variable.
A formula is \emph{molecular} (or \emph{compound}) if it is not atomic.
\end{definition}

% ---------------------------------------------------------
% Unique Readability
% ---------------------------------------------------------
\begin{tcolorbox}[colback=thmbox, colframe=thmborder, arc=2pt,
  left=6pt, right=6pt, top=4pt, bottom=4pt,
  title={\small\textbf{Theorem (Unique Readability)}},
  fonttitle=\small\bfseries]
\label{thm:unique-read}
Every well-formed formula $\varphi$ can be constructed in exactly one way.
Formally, exactly one of the following holds:
\begin{enumerate}
  \item $\varphi \in \mathsf{Prop}$ (atomic case).
  \item $\varphi = \neg\psi$ for a unique wff $\psi$.
  \item $\varphi = (\psi \circ \chi)$ for a unique binary connective $\circ$
        and unique wffs $\psi$, $\chi$.
\end{enumerate}
\end{tcolorbox}

\begin{remark}[Why this theorem is a theorem]
The recursive definition of wffs gives many ways to \emph{generate} formulas.
Unique readability asserts that no two generation paths produce the same string.
This is non-trivial: it fails for most context-free grammars. Propositional
logic avoids ambiguity by requiring explicit parentheses around every binary
connective application.
\end{remark}

\begin{remark}[Consequence — well-defined semantics]
Truth evaluation $v(\varphi)$ is defined recursively by cases on the structure
of $\varphi$. If $\varphi$ had two structures, $v(\varphi)$ might be
ill-defined. Unique readability guarantees that the recursive definition of
truth is coherent.
\end{remark}

\begin{remark}[Consequence — structural induction is safe]
Structural induction proceeds by case analysis on the unique form of $\varphi$.
Without unique readability, cases could overlap and the induction would break.
\end{remark}

% ---------------------------------------------------------
% Parse Tree
% ---------------------------------------------------------
\begin{tcolorbox}[colback=propbox, colframe=propborder, arc=2pt,
  left=6pt, right=6pt, top=4pt, bottom=4pt,
  title={\small\textbf{Definition (Parse Tree)}},
  fonttitle=\small\bfseries]
\label{def:parse-tree}
The \emph{parse tree} (or \emph{formation tree}) of a wff $\varphi$ is a
labeled binary tree where:
\begin{itemize}
  \item Each \textbf{leaf} is labeled with a propositional variable.
  \item Each \textbf{internal node} is labeled with a connective.
  \item A node labeled $\neg$ has \textbf{one child}; a node labeled with a
        binary connective has \textbf{two children}.
\end{itemize}
\end{tcolorbox}

\begin{remark}[Intuition]
The parse tree externalizes the unique recursive structure of a formula. Reading
it top-down reconstructs the formula; reading it bottom-up shows how sub-results
combine into the whole truth value.
\end{remark}

\begin{remark}[Consequence]
Unique readability guarantees every wff has a unique parse tree. This makes
structural induction synonymous with tree induction: induction on the height of
the parse tree.
\end{remark}

% ---------------------------------------------------------
% Subformula
% ---------------------------------------------------------
\begin{tcolorbox}[colback=propbox, colframe=propborder, arc=2pt,
  left=6pt, right=6pt, top=4pt, bottom=4pt,
  title={\small\textbf{Definition (Subformula)}},
  fonttitle=\small\bfseries]
\label{def:subformula}
The set of \emph{subformulas} of $\varphi$, written $\mathrm{Sub}(\varphi)$,
is defined recursively:
\begin{enumerate}
  \item $\varphi$ atomic: $\mathrm{Sub}(\varphi) = \{\varphi\}$.
  \item $\varphi = \neg\psi$: $\mathrm{Sub}(\varphi) = \{\varphi\} \cup \mathrm{Sub}(\psi)$.
  \item $\varphi = (\psi \circ \chi)$:
        $\mathrm{Sub}(\varphi) = \{\varphi\} \cup \mathrm{Sub}(\psi) \cup \mathrm{Sub}(\chi)$.
\end{enumerate}
A \emph{proper subformula} is any element of $\mathrm{Sub}(\varphi)$ other
than $\varphi$ itself.
\end{tcolorbox}

\begin{remark}[Intuition]
Every node in the parse tree of $\varphi$ corresponds to exactly one subformula.
The subformulas are the ``parts'' of $\varphi$ visible at each level of nesting.
\end{remark}

\begin{remark}[Consequence]
Many logical properties — tautology, satisfiability — are defined on
whole formulas but proved by induction on subformulas. A property that holds
for all subformulas of $\varphi$ and is preserved by connectives holds for
$\varphi$.
\end{remark}

% ---------------------------------------------------------
% Formula Depth
% ---------------------------------------------------------
\begin{tcolorbox}[colback=propbox, colframe=propborder, arc=2pt,
  left=6pt, right=6pt, top=4pt, bottom=4pt,
  title={\small\textbf{Definition (Formula Depth)}},
  fonttitle=\small\bfseries]
\label{def:depth}
The \emph{depth} (or \emph{complexity}) of a formula $\varphi$:
\begin{enumerate}
  \item $\varphi$ atomic: $\mathrm{depth}(\varphi) = 0$.
  \item $\varphi = \neg\psi$: $\mathrm{depth}(\varphi) = \mathrm{depth}(\psi) + 1$.
  \item $\varphi = (\psi \circ \chi)$:
        $\mathrm{depth}(\varphi) = \max\{\mathrm{depth}(\psi),\, \mathrm{depth}(\chi)\} + 1$.
\end{enumerate}
\end{tcolorbox}

\begin{remark}[Intuition]
Depth measures how deeply nested the connectives are — equivalently, the height
of the parse tree. A depth-0 formula has no connectives; a depth-$n$ formula
has a subformula at nesting level $n$.
\end{remark}

\begin{remark}[Consequence — strong induction]
Induction on depth is the canonical way to prove properties of all wffs when
you need the inductive hypothesis to apply to all formulas of \emph{strictly
smaller} depth, not just immediate subformulas.
\end{remark}

% ---------------------------------------------------------
% Precedence
% ---------------------------------------------------------
\begin{tcolorbox}[colback=propbox, colframe=propborder, arc=2pt,
  left=6pt, right=6pt, top=4pt, bottom=4pt,
  title={\small\textbf{Definition (Operator Precedence)}},
  fonttitle=\small\bfseries]
\label{def:precedence}
Standard precedence (highest binds tightest):
\begin{enumerate}
  \item Parentheses (override all)
  \item $\neg$ (negation)
  \item $\wedge$ (conjunction)
  \item $\vee$ (disjunction)
  \item $\rightarrow$ (conditional, right-associative)
  \item $\leftrightarrow$ (biconditional)
\end{enumerate}
\end{tcolorbox}

\begin{remark}[English reading]
Precedence is a convention for omitting parentheses without creating ambiguity.
$\neg P \wedge Q$ means $(\neg P) \wedge Q$ because $\neg$ binds tighter than
$\wedge$. The conditional is right-associative: $P \rightarrow Q \rightarrow R$
means $P \rightarrow (Q \rightarrow R)$.
\end{remark}

\begin{remark}[Common error]
$\neg P \wedge Q$ is \emph{not} $\neg(P \wedge Q)$. Negation applies to the
smallest formula immediately to its right. When in doubt, use explicit
parentheses.
\end{remark}

\begin{example}
Using standard precedence:
\begin{itemize}
  \item $\neg P \wedge Q$ means $(\neg P) \wedge Q$.
  \item $P \vee Q \wedge R$ means $P \vee (Q \wedge R)$ if $\wedge > \vee$.
  \item $P \rightarrow Q \vee R$ means $P \rightarrow (Q \vee R)$.
\end{itemize}
\end{example}

% =========================================================
% Additional Connectives (XOR, NAND, NOR)
% =========================================================

\subsection{Additional Connectives}

% ---------------------------------------------------------
% TOOLKIT
% ---------------------------------------------------------
\begin{tcolorbox}[colback=gray!6, colframe=gray!40, arc=2pt,
  left=6pt, right=6pt, top=4pt, bottom=4pt,
  title={\small\textbf{Additional Connectives — Quick Reference}},
  fonttitle=\small\bfseries]
\small
\begin{tabular}{l l l l}
\toprule
\textbf{Symbol} & \textbf{Name} & \textbf{Equivalent} & \textbf{Detail} \\
\midrule
$P \oplus Q$    & XOR (Exclusive Or)  & $\neg(P \leftrightarrow Q)$  & \hyperref[def:xor]{↓ Def} \\
$P \uparrow Q$  & NAND (Sheffer stroke) & $\neg(P \wedge Q)$          & \hyperref[def:nand]{↓ Def} \\
$P \downarrow Q$& NOR (Peirce arrow)  & $\neg(P \vee Q)$             & \hyperref[def:nor]{↓ Def} \\
\midrule
$\{\uparrow\}$  & NAND alone is adequate & — & \hyperref[prop:nand-complete]{↓ Prop} \\
$\{\downarrow\}$& NOR alone is adequate  & — & \hyperref[prop:nor-complete]{↓ Prop} \\
\bottomrule
\end{tabular}
\end{tcolorbox}

\vspace{0.5em}

\begin{center}
\small
\renewcommand{\arraystretch}{1.2}
\begin{tabular}{cc|ccc}
\toprule
$P$ & $Q$ & $P \oplus Q$ & $P \uparrow Q$ & $P \downarrow Q$ \\
\midrule
T & T & F & F & F \\
T & F & T & T & F \\
F & T & T & T & F \\
F & F & F & T & T \\
\bottomrule
\end{tabular}
\end{center}

\vspace{1em}

% ---------------------------------------------------------
% XOR
% ---------------------------------------------------------
\begin{tcolorbox}[colback=propbox, colframe=propborder, arc=2pt,
  left=6pt, right=6pt, top=4pt, bottom=4pt,
  title={\small\textbf{Definition (Exclusive Or — XOR)}},
  fonttitle=\small\bfseries]
\label{def:xor}
The \emph{exclusive or}, denoted $P \oplus Q$ (also $P \veebar Q$), is true
when exactly one of $P$, $Q$ is true:
\[
P \oplus Q \;\equiv\; \neg(P \leftrightarrow Q)
\]
Equivalently: $(P \vee Q) \wedge \neg(P \wedge Q)$, and also
$(P \wedge \neg Q) \vee (\neg P \wedge Q)$.
\end{tcolorbox}

\begin{remark}[English reading]
XOR captures the natural-language ``either … or … but not both.'' It differs
from inclusive or ($\vee$) only in the $(T, T)$ row: $P \vee Q$ is true there,
$P \oplus Q$ is false.
\end{remark}

\begin{remark}[Fully quantified properties]
For all $P, Q, R$:
\begin{align*}
P \oplus Q &\equiv Q \oplus P \quad\text{(commutativity)}\\
(P \oplus Q) \oplus R &\equiv P \oplus (Q \oplus R) \quad\text{(associativity)}\\
P \oplus \bot &\equiv P \quad\text{(identity)}\\
P \oplus \top &\equiv \neg P \quad\text{(negation)}\\
P \oplus P &\equiv \bot \quad\text{(self-inverse)}\\
P \wedge (Q \oplus R) &\equiv (P \wedge Q) \oplus (P \wedge R) \quad\text{(distributivity of $\wedge$ over $\oplus$)}
\end{align*}
The self-inverse law makes $(\mathbb{B}, \oplus)$ a group — a fact that
resurfaces in abstract algebra.
\end{remark}

\begin{remark}[Consequence]
XOR is definable from standard connectives, so it adds no expressive power.
It is included for notational convenience and because it has a clean algebraic
structure (Boolean ring addition).
\end{remark}

% ---------------------------------------------------------
% NAND
% ---------------------------------------------------------
\begin{tcolorbox}[colback=propbox, colframe=propborder, arc=2pt,
  left=6pt, right=6pt, top=4pt, bottom=4pt,
  title={\small\textbf{Definition (NAND — Sheffer Stroke)}},
  fonttitle=\small\bfseries]
\label{def:nand}
The \emph{NAND connective} (Sheffer stroke), denoted $P \uparrow Q$ or $P | Q$,
is false only when both operands are true:
\[
P \uparrow Q \;\equiv\; \neg(P \wedge Q)
\]
\end{tcolorbox}

\begin{remark}[English reading]
``Not both $P$ and $Q$.'' NAND is true in every row of the truth table except
when both inputs are true. It is the most common single connective in digital
circuit design (NAND gates are universal).
\end{remark}

\begin{remark}[Fully quantified — expressing all connectives via NAND]
\begin{align*}
\neg P &\equiv P \uparrow P \\
P \wedge Q &\equiv (P \uparrow Q) \uparrow (P \uparrow Q) \\
P \vee Q &\equiv (P \uparrow P) \uparrow (Q \uparrow Q) \\
P \rightarrow Q &\equiv P \uparrow (Q \uparrow Q)
\end{align*}
\end{remark}

\begin{proposition}[Non-Associativity of NAND]
\label{prop:nand-complete}
NAND is \textbf{not associative}: $(P \uparrow Q) \uparrow R \not\equiv P \uparrow (Q \uparrow R)$.
\end{proposition}

\begin{remark}[Consequence]
Non-associativity means NAND expressions require careful parenthesization.
Despite this, $\{\uparrow\}$ alone is functionally complete — see the
Functional Completeness section.
\end{remark}

% ---------------------------------------------------------
% NOR
% ---------------------------------------------------------
\begin{tcolorbox}[colback=propbox, colframe=propborder, arc=2pt,
  left=6pt, right=6pt, top=4pt, bottom=4pt,
  title={\small\textbf{Definition (NOR — Peirce Arrow)}},
  fonttitle=\small\bfseries]
\label{def:nor}
The \emph{NOR connective} (Peirce arrow), denoted $P \downarrow Q$, is true
only when both operands are false:
\[
P \downarrow Q \;\equiv\; \neg(P \vee Q)
\]
\end{tcolorbox}

\begin{remark}[English reading]
``Neither $P$ nor $Q$.'' NOR is the dual of NAND (swap $\wedge \leftrightarrow \vee$
and $\top \leftrightarrow \bot$). It is also a universal gate in digital circuits.
\end{remark}

\begin{remark}[Fully quantified — expressing all connectives via NOR]
\begin{align*}
\neg P &\equiv P \downarrow P \\
P \vee Q &\equiv (P \downarrow Q) \downarrow (P \downarrow Q) \\
P \wedge Q &\equiv (P \downarrow P) \downarrow (Q \downarrow Q) \\
P \rightarrow Q &\equiv ((P \downarrow P) \downarrow Q) \downarrow ((P \downarrow P) \downarrow Q)
\end{align*}
\end{remark}

\begin{proposition}[Non-Associativity of NOR]
\label{prop:nor-complete}
NOR is \textbf{not associative}: $(P \downarrow Q) \downarrow R \not\equiv P \downarrow (Q \downarrow R)$.
\end{proposition}

\begin{remark}[Duality with NAND]
NAND and NOR are De Morgan duals: replacing each by the other and swapping
$\wedge \leftrightarrow \vee$ transforms any NAND circuit into a NOR circuit
computing the same function. They share identical functional completeness
properties.
\end{remark}


% =========================================================
% Semantics — Orchestrator
% =========================================================
% =========================================================
% Semantics of Propositional Logic
% =========================================================

\subsection{Semantics of Propositional Logic}

% ---------------------------------------------------------
% TOOLKIT
% ---------------------------------------------------------
\begin{tcolorbox}[colback=gray!6, colframe=gray!40, arc=2pt,
  left=6pt, right=6pt, top=4pt, bottom=4pt,
  title={\small\textbf{Semantics Toolkit — Quick Reference}},
  fonttitle=\small\bfseries]
\small
\begin{tabular}{l l l}
\toprule
\textbf{Concept} & \textbf{Meaning} & \textbf{Detail} \\
\midrule
Truth assignment   & Function $v: \mathsf{Prop} \to \{\mathsf{T},\mathsf{F}\}$ & \hyperref[def:truth-assign]{↓ Def} \\
Truth under $v$    & $v \models \varphi$ iff $v(\varphi) = \mathsf{T}$ & \hyperref[def:satisfaction]{↓ Def} \\
Tautology          & True under all assignments ($\models\varphi$) & \hyperref[def:tautology]{↓ Def} \\
Contradiction      & False under all assignments & \hyperref[def:contradiction]{↓ Def} \\
Satisfiable        & True under at least one assignment & \hyperref[def:satisfiable]{↓ Def} \\
Contingency        & Neither tautology nor contradiction & \hyperref[def:contingency]{↓ Def} \\
Logical consequence & $\Gamma \models \varphi$ & \hyperref[def:consequence]{↓ Def} \\
Logical equivalence & $\varphi \equiv \psi$ ($\varphi \leftrightarrow \psi$ is a tautology) & \hyperref[def:log-equiv]{↓ Def} \\
\bottomrule
\end{tabular}
\end{tcolorbox}

\vspace{1em}

% ---------------------------------------------------------
% Truth Assignment and Extension
% ---------------------------------------------------------
\begin{tcolorbox}[colback=propbox, colframe=propborder, arc=2pt,
  left=6pt, right=6pt, top=4pt, bottom=4pt,
  title={\small\textbf{Definition (Truth Assignment)}},
  fonttitle=\small\bfseries]
\label{def:truth-assign}
A \emph{truth assignment} (or \emph{valuation}) is a function
\[
v : \mathsf{Prop} \to \{\mathsf{T}, \mathsf{F}\}.
\]
Every truth assignment extends \emph{uniquely} to
$\hat{v} : \mathsf{WFF} \to \{\mathsf{T}, \mathsf{F}\}$ by:
\begin{enumerate}
  \item $\hat{v}(P) = v(P)$ for $P \in \mathsf{Prop}$.
  \item $\hat{v}(\neg\varphi) = \mathsf{T}$ iff $\hat{v}(\varphi) = \mathsf{F}$.
  \item $\hat{v}(\varphi \wedge \psi) = \mathsf{T}$ iff $\hat{v}(\varphi) = \hat{v}(\psi) = \mathsf{T}$.
  \item $\hat{v}(\varphi \vee \psi) = \mathsf{T}$ iff $\hat{v}(\varphi) = \mathsf{T}$ or $\hat{v}(\psi) = \mathsf{T}$.
  \item $\hat{v}(\varphi \rightarrow \psi) = \mathsf{T}$ iff $\hat{v}(\varphi) = \mathsf{F}$ or $\hat{v}(\psi) = \mathsf{T}$.
  \item $\hat{v}(\varphi \leftrightarrow \psi) = \mathsf{T}$ iff $\hat{v}(\varphi) = \hat{v}(\psi)$.
\end{enumerate}
We write $v(\varphi)$ for $\hat{v}(\varphi)$.
\end{tcolorbox}

\begin{remark}[English reading]
A truth assignment is an interpretation: it gives a definite truth value to
every atomic proposition, and the extension rules then compute the truth value
of every compound formula mechanically. There is no ambiguity once $v$ is fixed.
\end{remark}

\begin{remark}[Fully quantified]
$\hat{v}$ is the unique function $\mathsf{WFF} \to \{\mathsf{T},\mathsf{F}\}$
extending $v$ and compatible with the recursive structure of wffs.
Uniqueness follows from unique readability: each wff has exactly one form,
so the recursive cases never conflict.
\end{remark}

\begin{remark}[Truth table connection]
A formula with $n$ distinct variables has exactly $2^n$ distinct truth
assignments. The truth table lists all of them, one per row.
\end{remark}

\begin{center}
\small\textbf{Truth Tables for Standard Connectives}

\vspace{0.5em}
\renewcommand{\arraystretch}{1.2}
\begin{tabular}{cc|ccccc}
\toprule
$P$ & $Q$ & $\neg P$ & $P \wedge Q$ & $P \vee Q$ & $P \rightarrow Q$ & $P \leftrightarrow Q$ \\
\midrule
T & T & F & T & T & T & T \\
T & F & F & F & T & F & F \\
F & T & T & F & T & T & F \\
F & F & T & F & F & T & T \\
\bottomrule
\end{tabular}
\end{center}

% ---------------------------------------------------------
% Satisfaction
% ---------------------------------------------------------
\begin{tcolorbox}[colback=propbox, colframe=propborder, arc=2pt,
  left=6pt, right=6pt, top=4pt, bottom=4pt,
  title={\small\textbf{Definition (Satisfaction)}},
  fonttitle=\small\bfseries]
\label{def:satisfaction}
A truth assignment $v$ \emph{satisfies} formula $\varphi$, written
$v \models \varphi$, if $v(\varphi) = \mathsf{T}$.

$v$ satisfies a set $\Gamma$ if $v \models \varphi$ for every $\varphi \in \Gamma$.
\end{tcolorbox}

\begin{remark}[Intuition]
$v \models \varphi$ is the semantic turnstile: $v$ makes $\varphi$ true.
The symbol $\models$ without a subscript on the left will be reused for
logical consequence ($\Gamma \models \varphi$) — context determines which reading.
\end{remark}

% ---------------------------------------------------------
% Tautology / Contradiction / Satisfiable / Contingency
% ---------------------------------------------------------
\begin{tcolorbox}[colback=propbox, colframe=propborder, arc=2pt,
  left=6pt, right=6pt, top=4pt, bottom=4pt,
  title={\small\textbf{Definitions (Formula Classification)}},
  fonttitle=\small\bfseries]
\begin{description}
  \item[Tautology\label{def:tautology}] $\varphi$ is a \emph{tautology} (written $\models\varphi$)
    if $v \models \varphi$ for every truth assignment $v$.
  \item[Contradiction\label{def:contradiction}] $\varphi$ is a \emph{contradiction} (or \emph{unsatisfiable})
    if $v \not\models \varphi$ for every $v$.
  \item[Satisfiable\label{def:satisfiable}] $\varphi$ is \emph{satisfiable} if there exists some $v$ with $v \models \varphi$.
  \item[Contingency\label{def:contingency}] $\varphi$ is a \emph{contingency} if it is neither a tautology nor a contradiction.
\end{description}
\end{tcolorbox}

\begin{remark}[Mutual exclusivity and exhaustiveness]
Every formula belongs to exactly one category:
\begin{center}
\renewcommand{\arraystretch}{1.2}
\begin{tabular}{lcc}
\toprule
\textbf{Category} & \textbf{True under all $v$?} & \textbf{True under some $v$?} \\
\midrule
Tautology     & Yes & Yes (trivially) \\
Contradiction & No  & No \\
Contingency   & No  & Yes \\
\bottomrule
\end{tabular}
\end{center}
\end{remark}

\begin{remark}[Key logical implication]
A formula is satisfiable if and only if it is \emph{not} a contradiction.
This equivalence is the bridge between satisfiability and unsatisfiability,
and is the basis for refutation proofs (prove $\varphi$ by deriving a
contradiction from $\neg\varphi$).
\end{remark}

\begin{example}
\begin{itemize}
  \item $P \vee \neg P$ — tautology (law of excluded middle).
  \item $P \wedge \neg P$ — contradiction.
  \item $P \rightarrow Q$ — contingency (false when $P = \mathsf{T}$, $Q = \mathsf{F}$).
\end{itemize}
\end{example}

% ---------------------------------------------------------
% Logical Consequence
% ---------------------------------------------------------
\begin{tcolorbox}[colback=propbox, colframe=propborder, arc=2pt,
  left=6pt, right=6pt, top=4pt, bottom=4pt,
  title={\small\textbf{Definition (Logical Consequence)}},
  fonttitle=\small\bfseries]
\label{def:consequence}
$\varphi$ is a \emph{logical consequence} of $\Gamma$, written $\Gamma \models \varphi$,
if every truth assignment satisfying $\Gamma$ also satisfies $\varphi$.

Formally: $\forall v,\; (v \models \Gamma) \Rightarrow (v \models \varphi)$.
\end{tcolorbox}

\begin{remark}[English reading]
$\Gamma \models \varphi$ means: there is no way to make all of $\Gamma$ true
while making $\varphi$ false. The premise set $\Gamma$ semantically forces $\varphi$.
\end{remark}

\begin{remark}[Special cases]
\begin{itemize}
  \item $\emptyset \models \varphi$ iff $\varphi$ is a tautology.
  \item $\Gamma \models \varphi$ for any $\Gamma$ containing a contradiction.
  \item $\{P\} \models Q$ iff $P \rightarrow Q$ is a tautology
        (\emph{tautological implication}).
\end{itemize}
\end{remark}

\begin{remark}[Connection to proof systems]
The fundamental theorem of propositional logic:
$\Gamma \models \varphi \;\Leftrightarrow\; \Gamma \vdash \varphi$ (soundness and completeness).
The left side is semantic; the right side is syntactic. They coincide for
propositional logic.
\end{remark}

% ---------------------------------------------------------
% Logical Equivalence
% ---------------------------------------------------------
\begin{tcolorbox}[colback=propbox, colframe=propborder, arc=2pt,
  left=6pt, right=6pt, top=4pt, bottom=4pt,
  title={\small\textbf{Definition (Logical Equivalence)}},
  fonttitle=\small\bfseries]
\label{def:log-equiv}
$\varphi$ and $\psi$ are \emph{logically equivalent}, written $\varphi \equiv \psi$,
if they have the same truth value under every truth assignment. Equivalently:
\begin{itemize}
  \item $\varphi \models \psi$ and $\psi \models \varphi$, or
  \item $\varphi \leftrightarrow \psi$ is a tautology, or
  \item $\varphi$ and $\psi$ have identical truth tables.
\end{itemize}
\end{tcolorbox}

\begin{remark}[Intuition]
Logically equivalent formulas say the same thing — they agree on every possible
state of affairs. They can be freely substituted for one another in any context.
\end{remark}

\begin{remark}[Equivalence is an equivalence relation]
$\equiv$ is reflexive ($\varphi \equiv \varphi$), symmetric, and transitive.
Moreover, it is a \emph{congruence}: if $\varphi \equiv \psi$, then
$(\varphi \circ \chi) \equiv (\psi \circ \chi)$ for any connective $\circ$ and
formula $\chi$. This substitution property is what makes equivalences useful
for formula simplification.
\end{remark}

% =========================================================
% Logical Equivalences
% =========================================================

\subsection{Logical Equivalences}

% ---------------------------------------------------------
% TOOLKIT — the summary table replaces the 12 bare theorems
% ---------------------------------------------------------
\begin{tcolorbox}[colback=gray!6, colframe=gray!40, arc=2pt,
  left=6pt, right=6pt, top=4pt, bottom=4pt,
  title={\small\textbf{Logical Equivalences — Quick Reference}},
  fonttitle=\small\bfseries]
\small
\begin{tabular}{l l l}
\toprule
\textbf{Name} & \textbf{Equivalence(s)} & \textbf{Detail} \\
\midrule
Double Negation      & $\neg\neg P \equiv P$ & \hyperref[eq:double-neg]{↓} \\
De Morgan (pair)     & $\neg(P\wedge Q)\equiv\neg P\vee\neg Q$;\quad $\neg(P\vee Q)\equiv\neg P\wedge\neg Q$ & \hyperref[eq:demorgan]{↓} \\
Commutativity        & $P\wedge Q\equiv Q\wedge P$;\quad $P\vee Q\equiv Q\vee P$ & \hyperref[eq:comm]{↓} \\
Associativity        & $(P\wedge Q)\wedge R\equiv P\wedge(Q\wedge R)$;\; and $\vee$ & \hyperref[eq:assoc]{↓} \\
Distributivity       & $P\wedge(Q\vee R)\equiv(P\wedge Q)\vee(P\wedge R)$;\; and dual & \hyperref[eq:dist]{↓} \\
Idempotence          & $P\wedge P\equiv P$;\quad $P\vee P\equiv P$ & \hyperref[eq:idemp]{↓} \\
Absorption           & $P\wedge(P\vee Q)\equiv P$;\quad $P\vee(P\wedge Q)\equiv P$ & \hyperref[eq:absorb]{↓} \\
Identity             & $P\wedge\top\equiv P$;\quad $P\vee\bot\equiv P$ & \hyperref[eq:identity]{↓} \\
Domination           & $P\vee\top\equiv\top$;\quad $P\wedge\bot\equiv\bot$ & \hyperref[eq:domination]{↓} \\
Negation Laws        & $P\vee\neg P\equiv\top$;\quad $P\wedge\neg P\equiv\bot$ & \hyperref[eq:negation]{↓} \\
Material Implication & $P\rightarrow Q\equiv\neg P\vee Q$ & \hyperref[eq:mat-impl]{↓} \\
Contraposition       & $P\rightarrow Q\equiv\neg Q\rightarrow\neg P$ & \hyperref[eq:contra]{↓} \\
Exportation          & $(P\wedge Q)\rightarrow R\equiv P\rightarrow(Q\rightarrow R)$ & \hyperref[eq:export]{↓} \\
Neg.\ Conditional    & $\neg(P\rightarrow Q)\equiv P\wedge\neg Q$ & \hyperref[eq:neg-cond]{↓} \\
Biconditional Exp.   & $P\leftrightarrow Q\equiv(P\rightarrow Q)\wedge(Q\rightarrow P)$ & \hyperref[eq:bicond-exp]{↓} \\
Duality Principle    & $\varphi\equiv\psi \Rightarrow \varphi^d\equiv\psi^d$ & \hyperref[thm:duality]{↓ Thm} \\
\bottomrule
\end{tabular}
\end{tcolorbox}

\vspace{1em}

% ---------------------------------------------------------
% Details
% ---------------------------------------------------------

\subsubsection*{Core Equivalences}

\begin{proposition}[Double Negation]\label{eq:double-neg}
$\neg\neg P \;\equiv\; P$
\end{proposition}
\begin{remark}[Intuition] Negating twice returns to the original truth value.
In classical logic, $\neg\neg P$ and $P$ are interchangeable everywhere.
In intuitionistic logic this fails — $\neg\neg P \Rightarrow P$ is not provable
without the law of excluded middle.
\end{remark}
\begin{remark}[Proof strategy] Truth table: 2 rows, 2 columns; inspect directly.
\end{remark}

\begin{proposition}[De Morgan's Laws]\label{eq:demorgan}
\begin{align*}
\neg(P \wedge Q) &\;\equiv\; \neg P \vee \neg Q \\
\neg(P \vee Q)   &\;\equiv\; \neg P \wedge \neg Q
\end{align*}
\end{proposition}
\begin{remark}[Intuition] Negation distributes through conjunction/disjunction
by flipping the connective. ``Not both'' = ``not the first or not the second.''
``Neither'' = ``not the first and not the second.''
\end{remark}
\begin{remark}[Consequence] De Morgan's laws are the primary tool for pushing
negations inward (step 2 of NNF conversion) and for expanding $\neg$ through
complex formulas.
\end{remark}
\begin{remark}[Logical implication — duality]
The two De Morgan laws are duals of each other under the duality principle
(swap $\wedge\leftrightarrow\vee$, $\top\leftrightarrow\bot$).
\end{remark}

\begin{proposition}[Commutativity]\label{eq:comm}
$P \wedge Q \equiv Q \wedge P$;\qquad $P \vee Q \equiv Q \vee P$
\end{proposition}
\begin{remark}[Intuition] Order of operands does not matter for $\wedge$ and $\vee$.
Note: $\rightarrow$ is \emph{not} commutative ($P \rightarrow Q \not\equiv Q \rightarrow P$
in general).
\end{remark}

\begin{proposition}[Associativity]\label{eq:assoc}
$(P \wedge Q) \wedge R \equiv P \wedge (Q \wedge R)$;\qquad
$(P \vee Q) \vee R \equiv P \vee (Q \vee R)$
\end{proposition}
\begin{remark}[Intuition] Parenthesization does not affect truth value for chains
of $\wedge$ or $\vee$. This justifies writing $P \wedge Q \wedge R$ without
parentheses.
\end{remark}
\begin{remark}[Common error] $\rightarrow$ is right-associative by convention
but \emph{not} associative as an equivalence: $(P \rightarrow Q) \rightarrow R
\not\equiv P \rightarrow (Q \rightarrow R)$ in general.
\end{remark}

\begin{proposition}[Distributivity]\label{eq:dist}
$P \wedge (Q \vee R) \equiv (P \wedge Q) \vee (P \wedge R)$;\qquad
$P \vee (Q \wedge R) \equiv (P \vee Q) \wedge (P \vee R)$
\end{proposition}
\begin{remark}[Intuition] $\wedge$ distributes over $\vee$ and $\vee$ distributes
over $\wedge$ — unlike in arithmetic where only multiplication distributes over
addition, here both distribute over the other.
\end{remark}
\begin{remark}[Consequence] Distributivity is the key step in CNF and DNF
conversion. DNF uses $\wedge$ over $\vee$; CNF uses $\vee$ over $\wedge$.
\end{remark}

\begin{proposition}[Idempotence]\label{eq:idemp}
$P \wedge P \equiv P$;\qquad $P \vee P \equiv P$
\end{proposition}
\begin{remark}[Intuition] Repeating a formula under $\wedge$ or $\vee$ adds no
new information. Useful for simplification.
\end{remark}

\begin{proposition}[Absorption]\label{eq:absorb}
$P \wedge (P \vee Q) \equiv P$;\qquad $P \vee (P \wedge Q) \equiv P$
\end{proposition}
\begin{remark}[Intuition] The stronger condition absorbs the weaker. If $P$ is
true, then $P \vee Q$ is certainly true, so $P \wedge (P \vee Q)$ is just $P$.
\end{remark}

\begin{proposition}[Identity Laws]\label{eq:identity}
$P \wedge \top \equiv P$;\qquad $P \vee \bot \equiv P$
\end{proposition}
\begin{remark}[Intuition] $\top$ is the identity for $\wedge$; $\bot$ is the
identity for $\vee$. They play the role of 1 and 0 in Boolean algebra.
\end{remark}

\begin{proposition}[Domination Laws]\label{eq:domination}
$P \vee \top \equiv \top$;\qquad $P \wedge \bot \equiv \bot$
\end{proposition}
\begin{remark}[Intuition] $\top$ dominates $\vee$; $\bot$ dominates $\wedge$.
A disjunction containing $\top$ is always true; a conjunction containing $\bot$
is always false.
\end{remark}

\begin{proposition}[Negation Laws]\label{eq:negation}
$P \vee \neg P \equiv \top$ \quad (Law of Excluded Middle);\qquad
$P \wedge \neg P \equiv \bot$ \quad (Law of Non-Contradiction)
\end{proposition}
\begin{remark}[Logical significance] These are the two fundamental laws of
classical logic. Excluded middle ($P \vee \neg P$) is rejected by intuitionists.
Non-contradiction is accepted universally. They are duals of each other.
\end{remark}

\subsubsection*{Conditional Equivalences}

\begin{proposition}[Material Implication]\label{eq:mat-impl}
$P \rightarrow Q \;\equiv\; \neg P \vee Q$
\end{proposition}
\begin{remark}[Intuition] The conditional is only false when the antecedent is
true and the consequent false. In all other cases — including when $P$ is false —
it is vacuously true. This often surprises newcomers.
\end{remark}
\begin{remark}[Consequence] Material implication eliminates $\rightarrow$ in
favor of $\neg$ and $\vee$, the first step in NNF conversion.
\end{remark}

\begin{proposition}[Contraposition]\label{eq:contra}
$P \rightarrow Q \;\equiv\; \neg Q \rightarrow \neg P$
\end{proposition}
\begin{remark}[Proof strategy] Proof by contraposition: to prove $P \rightarrow Q$,
it is equivalent to prove $\neg Q \rightarrow \neg P$. Choose whichever direction
is easier to argue.
\end{remark}

\begin{proposition}[Exportation / Importation]\label{eq:export}
$(P \wedge Q) \rightarrow R \;\equiv\; P \rightarrow (Q \rightarrow R)$
\end{proposition}
\begin{remark}[Intuition] A proof from two premises can be restructured as a
chain of single-premise implications. This is the basis for currying in
functional programming and natural deduction.
\end{remark}

\begin{proposition}[Negation of Conditional]\label{eq:neg-cond}
$\neg(P \rightarrow Q) \;\equiv\; P \wedge \neg Q$
\end{proposition}
\begin{remark}[Intuition] The only way a conditional is false is if the antecedent
holds but the consequent fails.
\end{remark}
\begin{remark}[Consequence] Used in refutation proofs: to show $P \rightarrow Q$,
assume $\neg(P \rightarrow Q) \equiv P \wedge \neg Q$ and derive a contradiction.
\end{remark}

\begin{proposition}[Biconditional Expansion]\label{eq:bicond-exp}
$P \leftrightarrow Q \;\equiv\; (P \rightarrow Q) \wedge (Q \rightarrow P)$
\end{proposition}
\begin{remark}[Proof strategy] To prove $P \leftrightarrow Q$, prove both
$P \rightarrow Q$ and $Q \rightarrow P$ separately. This is the standard
``iff proof'' structure.
\end{remark}

\subsubsection*{The Duality Principle}

\begin{tcolorbox}[colback=propbox, colframe=propborder, arc=2pt,
  left=6pt, right=6pt, top=4pt, bottom=4pt,
  title={\small\textbf{Definition (Dual Formula)}},
  fonttitle=\small\bfseries]
The \emph{dual} of $\varphi$, written $\varphi^d$, is obtained by simultaneously
replacing $\wedge \leftrightarrow \vee$ and $\top \leftrightarrow \bot$, leaving
all variables and negations unchanged.
\end{tcolorbox}

\begin{tcolorbox}[colback=thmbox, colframe=thmborder, arc=2pt,
  left=6pt, right=6pt, top=4pt, bottom=4pt,
  title={\small\textbf{Theorem (Duality Principle)}},
  fonttitle=\small\bfseries]
\label{thm:duality}
If $\varphi \equiv \psi$ is a logical equivalence involving only
$\neg, \wedge, \vee, \top, \bot$, then $\varphi^d \equiv \psi^d$.
\end{tcolorbox}

\begin{remark}[Intuition] Every valid equivalence has a ``mirror image'' obtained
by swapping $\wedge\leftrightarrow\vee$ and $\top\leftrightarrow\bot$. De Morgan's
laws are each other's duals. Identity and Domination are each other's duals.
\end{remark}
\begin{remark}[Consequence] Duality halves the number of equivalences to memorize.
Learn one form; the dual comes for free.
\end{remark}
\begin{remark}[Formally]
For any formula $\varphi$ with variables $P_1,\dots,P_n$:
$\varphi^d \equiv \neg\varphi[\neg P_1/P_1, \dots, \neg P_n/P_n]$,
where $\neg P_i / P_i$ denotes simultaneous substitution of $\neg P_i$ for $P_i$.
\end{remark}

% =========================================================
% Normal Forms
% =========================================================

\subsection{Normal Forms}

% ---------------------------------------------------------
% TOOLKIT
% ---------------------------------------------------------
\begin{tcolorbox}[colback=gray!6, colframe=gray!40, arc=2pt,
  left=6pt, right=6pt, top=4pt, bottom=4pt,
  title={\small\textbf{Normal Forms — Quick Reference}},
  fonttitle=\small\bfseries]
\small
\begin{tabular}{l l l}
\toprule
\textbf{Form} & \textbf{Structure} & \textbf{Detail} \\
\midrule
Literal      & $P$ or $\neg P$ & \hyperref[def:literal]{↓ Def} \\
Clause (disjunctive) & Disjunction of literals & \hyperref[def:clause]{↓ Def} \\
NNF          & Negations only on variables; uses $\neg,\wedge,\vee$ only & \hyperref[def:nnf]{↓ Def} \\
CNF          & Conjunction of disjunctions of literals $\bigwedge_i\bigvee_j L_{ij}$ & \hyperref[def:cnf]{↓ Def} \\
DNF          & Disjunction of conjunctions of literals $\bigvee_i\bigwedge_j L_{ij}$ & \hyperref[def:dnf]{↓ Def} \\
\midrule
NNF existence & Every formula $\equiv$ some NNF formula & \hyperref[prop:nnf-exists]{↓ Prop} \\
CNF existence & Every formula $\equiv$ some CNF formula & \hyperref[prop:cnf-exists]{↓ Prop} \\
DNF existence & Every formula $\equiv$ some DNF formula & \hyperref[prop:dnf-exists]{↓ Prop} \\
\bottomrule
\end{tabular}

\vspace{0.5em}
\textbf{Conversion pipeline:} Original formula $\xrightarrow{\text{elim } \rightarrow,\leftrightarrow}$
NNF $\xrightarrow{\text{distribute } \vee/\wedge}$ CNF or DNF
\end{tcolorbox}

\vspace{1em}

% ---------------------------------------------------------
% Literals and Clauses
% ---------------------------------------------------------
\begin{tcolorbox}[colback=propbox, colframe=propborder, arc=2pt,
  left=6pt, right=6pt, top=4pt, bottom=4pt,
  title={\small\textbf{Definitions (Literal and Clause)}},
  fonttitle=\small\bfseries]
\label{def:literal}
A \emph{literal} is a propositional variable ($P$, a \emph{positive literal})
or the negation of one ($\neg P$, a \emph{negative literal}).

\label{def:clause}
A \emph{(disjunctive) clause} is a disjunction of literals:
$L_1 \vee L_2 \vee \cdots \vee L_k$.

A \emph{conjunctive clause} (or \emph{term}) is a conjunction of literals:
$L_1 \wedge L_2 \wedge \cdots \wedge L_k$.
\end{tcolorbox}

\begin{remark}[Why literals matter]
CNF and DNF are defined in terms of literals, not arbitrary formulas. Literals
are the atoms of normal form theory. Every formula in NNF is a tree whose
leaves are literals.
\end{remark}

% ---------------------------------------------------------
% NNF
% ---------------------------------------------------------
\begin{tcolorbox}[colback=propbox, colframe=propborder, arc=2pt,
  left=6pt, right=6pt, top=4pt, bottom=4pt,
  title={\small\textbf{Definition (Negation Normal Form — NNF)}},
  fonttitle=\small\bfseries]
\label{def:nnf}
A formula is in \emph{negation normal form} (NNF) if:
\begin{enumerate}
  \item Negation ($\neg$) is applied only to propositional variables (no negations of compound formulas).
  \item The only connectives used are $\neg$, $\wedge$, $\vee$.
\end{enumerate}
\end{tcolorbox}

\begin{proposition}\label{prop:nnf-exists}
Every propositional formula is logically equivalent to a formula in NNF.
\end{proposition}

\begin{remark}[Conversion procedure]
\begin{enumerate}
  \item Eliminate $\rightarrow$: replace $P \rightarrow Q$ with $\neg P \vee Q$.
  \item Eliminate $\leftrightarrow$: replace $P \leftrightarrow Q$ with
        $(P \wedge Q) \vee (\neg P \wedge \neg Q)$.
  \item Push $\neg$ inward via De Morgan and double negation:
        $\neg(P \wedge Q) \to \neg P \vee \neg Q$;\quad
        $\neg(P \vee Q) \to \neg P \wedge \neg Q$;\quad
        $\neg\neg P \to P$.
\end{enumerate}
\end{remark}

\begin{remark}[Why NNF first]
NNF is the prerequisite for CNF/DNF conversion. Once in NNF, the only remaining
structural operations are distributing $\wedge$ over $\vee$ (for CNF) or
$\vee$ over $\wedge$ (for DNF).
\end{remark}

% ---------------------------------------------------------
% CNF
% ---------------------------------------------------------
\begin{tcolorbox}[colback=propbox, colframe=propborder, arc=2pt,
  left=6pt, right=6pt, top=4pt, bottom=4pt,
  title={\small\textbf{Definition (Conjunctive Normal Form — CNF)}},
  fonttitle=\small\bfseries]
\label{def:cnf}
A formula is in \emph{conjunctive normal form} (CNF) if it is a conjunction of
disjunctive clauses:
\[
\bigwedge_{i=1}^{m} \bigvee_{j=1}^{n_i} L_{ij}
\]
where each $L_{ij}$ is a literal.
\end{tcolorbox}

\begin{proposition}\label{prop:cnf-exists}
Every propositional formula is logically equivalent to a formula in CNF.
\end{proposition}

\begin{remark}[CNF conversion from NNF]
Distribute $\vee$ over $\wedge$ repeatedly:
$P \vee (Q \wedge R) \equiv (P \vee Q) \wedge (P \vee R)$.
Repeat until no $\vee$ sits above a $\wedge$ in the parse tree.
\end{remark}

\begin{remark}[Truth-table method for CNF]
For each row where the formula is \emph{false}, form a disjunctive clause
using positive literals for variables assigned $\mathsf{F}$ and negative literals
for variables assigned $\mathsf{T}$. Take the conjunction of all such clauses.
\end{remark}

\begin{remark}[Consequence — resolution]
CNF is the required input format for the resolution proof procedure.
Every satisfiability solver (SAT solver) works on CNF.
\end{remark}

\begin{example}
$(P \vee Q) \wedge (\neg P \vee R) \wedge (Q \vee \neg R)$ is in CNF.
\end{example}

% ---------------------------------------------------------
% DNF
% ---------------------------------------------------------
\begin{tcolorbox}[colback=propbox, colframe=propborder, arc=2pt,
  left=6pt, right=6pt, top=4pt, bottom=4pt,
  title={\small\textbf{Definition (Disjunctive Normal Form — DNF)}},
  fonttitle=\small\bfseries]
\label{def:dnf}
A formula is in \emph{disjunctive normal form} (DNF) if it is a disjunction of
conjunctive clauses:
\[
\bigvee_{i=1}^{m} \bigwedge_{j=1}^{n_i} L_{ij}
\]
where each $L_{ij}$ is a literal.
\end{tcolorbox}

\begin{proposition}\label{prop:dnf-exists}
Every propositional formula is logically equivalent to a formula in DNF.
\end{proposition}

\begin{remark}[DNF conversion from NNF]
Distribute $\wedge$ over $\vee$ repeatedly:
$P \wedge (Q \vee R) \equiv (P \wedge Q) \vee (P \wedge R)$.
\end{remark}

\begin{remark}[Truth-table method for DNF]
For each row where the formula is \emph{true}, form a conjunctive clause
using positive literals for variables assigned $\mathsf{T}$ and negative literals
for variables assigned $\mathsf{F}$. Take the disjunction of all such clauses.
\end{remark}

\begin{remark}[CNF vs.\ DNF trade-off]
CNF and DNF can both be exponentially larger than the original formula.
Neither is always more compact. The choice depends on application:
resolution needs CNF; model enumeration benefits from DNF.
\end{remark}


% =========================================================
% Proof Theory — Orchestrator
% =========================================================
% Dependency order: inference rules precede resolution (resolution is a
% specific rule system); proof systems follow both (they contextualize
% derivability, soundness, completeness).

% =========================================================
% Inference Rules
% =========================================================

\subsection{Inference Rules}

% ---------------------------------------------------------
% TOOLKIT
% ---------------------------------------------------------
\begin{tcolorbox}[colback=gray!6, colframe=gray!40, arc=2pt,
  left=6pt, right=6pt, top=4pt, bottom=4pt,
  title={\small\textbf{Inference Rules — Quick Reference}},
  fonttitle=\small\bfseries]
\small
\begin{tabular}{l l l l}
\toprule
\textbf{Rule} & \textbf{Premises} & \textbf{Conclusion} & \textbf{Detail} \\
\midrule
Modus Ponens (MP)         & $P\to Q,\; P$              & $Q$                & \hyperref[def:mp]{↓} \\
Modus Tollens (MT)        & $P\to Q,\; \neg Q$         & $\neg P$           & \hyperref[def:mt]{↓} \\
Hypothetical Syllogism    & $P\to Q,\; Q\to R$         & $P\to R$           & \hyperref[def:hs]{↓} \\
Disjunctive Syllogism     & $P\vee Q,\; \neg P$        & $Q$                & \hyperref[def:ds]{↓} \\
Conj.\ Introduction       & $P,\; Q$                   & $P\wedge Q$        & \hyperref[def:conj-i]{↓} \\
Conj.\ Elimination        & $P\wedge Q$                & $P$ (or $Q$)       & \hyperref[def:conj-e]{↓} \\
Disj.\ Introduction       & $P$                        & $P\vee Q$          & \hyperref[def:disj-i]{↓} \\
Disj.\ Elimination        & $P\vee Q,\;[P\vdash R],\;[Q\vdash R]$ & $R$  & \hyperref[def:disj-e]{↓} \\
Cond.\ Introduction (CP)  & $[P \vdash Q]$             & $P\to Q$           & \hyperref[def:cond-i]{↓} \\
Bicond.\ Introduction     & $P\to Q,\; Q\to P$         & $P\leftrightarrow Q$ & \hyperref[def:bicond-i]{↓} \\
Bicond.\ Elimination      & $P\leftrightarrow Q$       & $P\to Q$ (or $Q\to P$) & \hyperref[def:bicond-e]{↓} \\
Neg.\ Introduction (RAA)  & $[P\vdash\bot]$            & $\neg P$           & \hyperref[def:neg-i]{↓} \\
Neg.\ Elimination         & $[\neg P\vdash\bot]$       & $P$                & \hyperref[def:neg-e]{↓} \\
Double Neg.\ Elim.        & $\neg\neg P$               & $P$                & \hyperref[def:dne]{↓} \\
Constructive Dilemma      & $P\vee Q,\;P\to R,\;Q\to S$ & $R\vee S$        & \hyperref[def:cd]{↓} \\
Destructive Dilemma       & $P\to R,\;Q\to S,\;\neg R\vee\neg S$ & $\neg P\vee\neg Q$ & \hyperref[def:dd]{↓} \\
\bottomrule
\end{tabular}
\end{tcolorbox}

\vspace{1em}

% ---------------------------------------------------------
% Details
% ---------------------------------------------------------

\subsubsection*{Conditional Rules}

\begin{definition}[Modus Ponens]\label{def:mp}
From a conditional and its antecedent, infer the consequent:
\[
\frac{P \rightarrow Q \qquad P}{Q}
\]
\end{definition}

\begin{remark}[Intuition] ``If it rains the ground is wet. It is raining.
Therefore, the ground is wet.'' The most fundamental inference rule. All
proof systems for classical logic include MP or an equivalent.
\end{remark}
\begin{remark}[Fully quantified]
$\forall P, Q:\; [(P \rightarrow Q) \wedge P] \models Q$.
\end{remark}
\begin{remark}[Common error — Affirming the Consequent]
From $P \rightarrow Q$ and $Q$, one \emph{cannot} infer $P$. This is the
fallacy of affirming the consequent.
\end{remark}

\begin{definition}[Modus Tollens]\label{def:mt}
From a conditional and the negation of its consequent, infer the negation of
the antecedent:
\[
\frac{P \rightarrow Q \qquad \neg Q}{\neg P}
\]
\end{definition}

\begin{remark}[Intuition] ``If it rains the ground is wet. The ground is not wet.
Therefore, it did not rain.'' MT is logically equivalent to applying MP to
the contrapositive $\neg Q \rightarrow \neg P$.
\end{remark}
\begin{remark}[Common error — Denying the Antecedent]
From $P \rightarrow Q$ and $\neg P$, one \emph{cannot} infer $\neg Q$.
\end{remark}

\begin{definition}[Hypothetical Syllogism]\label{def:hs}
From two conditionals chained, infer the composed conditional:
\[
\frac{P \rightarrow Q \qquad Q \rightarrow R}{P \rightarrow R}
\]
\end{definition}

\begin{remark}[Intuition] Transitivity of implication. Chains of reasoning can
be composed into a single step.
\end{remark}
\begin{remark}[Consequence] Hypothetical syllogism makes $\rightarrow$ transitive,
allowing arbitrarily long inference chains to be compressed.
\end{remark}

\begin{definition}[Conditional Introduction (Conditional Proof)]\label{def:cond-i}
If assuming $P$ in a subproof leads to $Q$, then discharge the assumption and
infer $P \rightarrow Q$:
\[
\frac{[P \vdash Q]}{P \rightarrow Q}
\]
\end{definition}

\begin{remark}[Intuition] The standard method for proving a conditional: assume
the antecedent, derive the consequent, discharge. This corresponds to
exportation: a proof from hypothesis $P$ of $Q$ is exactly a proof of $P \to Q$.
\end{remark}

\subsubsection*{Conjunction Rules}

\begin{definition}[Conjunction Introduction]\label{def:conj-i}
From two formulas, infer their conjunction:
\[
\frac{P \qquad Q}{P \wedge Q}
\]
\end{definition}

\begin{definition}[Conjunction Elimination]\label{def:conj-e}
From a conjunction, infer either conjunct:
\[
\frac{P \wedge Q}{P} \qquad \frac{P \wedge Q}{Q}
\]
\end{definition}

\begin{remark}[Intuition] Introduction: combining two independent truths.
Elimination: extracting a component from a joint assertion.
\end{remark}

\subsubsection*{Disjunction Rules}

\begin{definition}[Disjunction Introduction]\label{def:disj-i}
From a proposition, infer any disjunction containing it:
\[
\frac{P}{P \vee Q}
\]
\end{definition}

\begin{remark}[Intuition] ``It's raining, so it's raining or snowing.'' Adding
a disjunct weakens the claim.
\end{remark}

\begin{definition}[Disjunction Elimination (Proof by Cases)]\label{def:disj-e}
From a disjunction, if the same conclusion $R$ follows from each disjunct, infer $R$:
\[
\frac{P \vee Q \qquad [P \vdash R] \qquad [Q \vdash R]}{R}
\]
\end{definition}

\begin{remark}[Proof strategy] Case analysis: prove $R$ holds in case $P$;
prove $R$ holds in case $Q$; since one of $P, Q$ must hold, conclude $R$.
\end{remark}

\subsubsection*{Biconditional Rules}

\begin{definition}[Biconditional Introduction]\label{def:bicond-i}
\[
\frac{P \rightarrow Q \qquad Q \rightarrow P}{P \leftrightarrow Q}
\]
\end{definition}

\begin{definition}[Biconditional Elimination]\label{def:bicond-e}
\[
\frac{P \leftrightarrow Q}{P \rightarrow Q} \qquad \frac{P \leftrightarrow Q}{Q \rightarrow P}
\]
\end{definition}

\begin{remark}[Proof strategy] Every iff proof has two halves: prove $(\Rightarrow)$
and prove $(\Leftarrow)$.
\end{remark}

\subsubsection*{Negation Rules}

\begin{definition}[Negation Introduction (Reductio ad Absurdum)]\label{def:neg-i}
If assuming $P$ leads to contradiction $\bot$, infer $\neg P$:
\[
\frac{[P \vdash \bot]}{\neg P}
\]
\end{definition}

\begin{definition}[Negation Elimination (Indirect Proof)]\label{def:neg-e}
If assuming $\neg P$ leads to $\bot$, infer $P$:
\[
\frac{[\neg P \vdash \bot]}{P}
\]
\end{definition}

\begin{definition}[Double Negation Elimination]\label{def:dne}
\[
\frac{\neg\neg P}{P}
\]
\end{definition}

\begin{remark}[Classical vs.\ intuitionistic logic]
Negation elimination and double negation elimination are classical rules;
they fail in intuitionistic logic, where $\neg\neg P \Rightarrow P$ is not
provable without the law of excluded middle. All three rules are equivalent
to LEM in a suitable sense.
\end{remark}

\subsubsection*{Dilemma Rules}

\begin{definition}[Constructive Dilemma]\label{def:cd}
\[
\frac{P \vee Q \qquad P \rightarrow R \qquad Q \rightarrow S}{R \vee S}
\]
\end{definition}

\begin{definition}[Destructive Dilemma]\label{def:dd}
\[
\frac{P \rightarrow R \qquad Q \rightarrow S \qquad \neg R \vee \neg S}{\neg P \vee \neg Q}
\]
\end{definition}

\begin{remark}[Intuition] Constructive dilemma applies two implications to a
disjunction, deriving a disjunction of the two conclusions. Destructive dilemma
is its contrapositive form.
\end{remark}

% =========================================================
% Resolution
% =========================================================

\subsection{Resolution}

% ---------------------------------------------------------
% TOOLKIT
% ---------------------------------------------------------
\begin{tcolorbox}[colback=gray!6, colframe=gray!40, arc=2pt,
  left=6pt, right=6pt, top=4pt, bottom=4pt,
  title={\small\textbf{Resolution — Quick Reference}},
  fonttitle=\small\bfseries]
\small
\begin{tabular}{l l l}
\toprule
\textbf{Concept} & \textbf{Key fact} & \textbf{Detail} \\
\midrule
Resolution rule      & From $C_1\vee L$ and $C_2\vee\neg L$, derive $C_1\vee C_2$ & \hyperref[def:resolution]{↓ Def} \\
Empty clause $\square$ & Always false; represents a contradiction & \hyperref[def:empty-clause]{↓ Def} \\
Resolution refutation & Derive $\square$ from $\Gamma$ — proves $\Gamma$ unsatisfiable & \hyperref[def:refutation]{↓ Def} \\
Soundness            & $\Gamma\vdash_{\mathrm{res}}\square \Rightarrow \Gamma$ unsat. & \hyperref[thm:res-sound]{↓ Thm} \\
Completeness         & $\Gamma$ unsat. $\Rightarrow \Gamma\vdash_{\mathrm{res}}\square$ & \hyperref[thm:res-complete]{↓ Thm} \\
Validity testing     & $\Gamma\models\varphi$ iff $\Gamma\cup\{\neg\varphi\}$ derives $\square$ & \hyperref[rem:res-validity]{↓} \\
\midrule
Strategies & Unit res., set of support, linear, input & \hyperref[rem:res-strat]{↓} \\
\bottomrule
\end{tabular}
\end{tcolorbox}

\vspace{1em}

% ---------------------------------------------------------
% Resolution Rule
% ---------------------------------------------------------
\begin{tcolorbox}[colback=propbox, colframe=propborder, arc=2pt,
  left=6pt, right=6pt, top=4pt, bottom=4pt,
  title={\small\textbf{Definition (Resolution Rule)}},
  fonttitle=\small\bfseries]
\label{def:resolution}
Given two clauses containing complementary literals $L$ and $\neg L$:
\[
\frac{C_1 \vee L \qquad C_2 \vee \neg L}{C_1 \vee C_2}
\]
The derived clause $C_1 \vee C_2$ is the \emph{resolvent}. The literal $L$
is \emph{resolved upon} (eliminated). $C_1$, $C_2$ are (possibly empty)
disjunctions of literals.
\end{tcolorbox}

\begin{remark}[English reading]
Resolution combines two clauses that disagree on exactly one variable,
producing a clause that contains everything both clauses said, minus the
disagreement. It is the only inference rule in the resolution proof system.
\end{remark}

\begin{remark}[Fully quantified]
For all clauses $C_1, C_2$ and literal $L$:
$(C_1 \vee L) \wedge (C_2 \vee \neg L) \models (C_1 \vee C_2)$.
Resolution is sound: the resolvent is a semantic consequence of the parent clauses.
\end{remark}

\begin{remark}[Requires CNF]
Resolution applies only to clauses (disjunctions of literals). The input set
$\Gamma$ must be in CNF. Converting to CNF is always possible (by the CNF
existence theorem).
\end{remark}

\begin{example}
Resolving $(P \vee Q)$ and $(\neg P \vee R)$ on $P$:
\[
\frac{P \vee Q \qquad \neg P \vee R}{Q \vee R}
\]
Resolving $(P)$ and $(\neg P)$: derive the empty clause $\square$.
\end{example}

% ---------------------------------------------------------
% Empty Clause
% ---------------------------------------------------------
\begin{tcolorbox}[colback=propbox, colframe=propborder, arc=2pt,
  left=6pt, right=6pt, top=4pt, bottom=4pt,
  title={\small\textbf{Definition (Empty Clause)}},
  fonttitle=\small\bfseries]
\label{def:empty-clause}
The \emph{empty clause}, denoted $\square$ (or $\bot$), is the disjunction of
no literals. It is always false (unsatisfiable by any truth assignment).
\end{tcolorbox}

\begin{remark}[Intuition]
Deriving $\square$ is the resolution analogue of deriving a contradiction.
Since $\square$ is false under every assignment, its derivation from $\Gamma$
proves $\Gamma$ cannot be satisfied.
\end{remark}

% ---------------------------------------------------------
% Resolution Refutation
% ---------------------------------------------------------
\begin{tcolorbox}[colback=propbox, colframe=propborder, arc=2pt,
  left=6pt, right=6pt, top=4pt, bottom=4pt,
  title={\small\textbf{Definition (Resolution Refutation)}},
  fonttitle=\small\bfseries]
\label{def:refutation}
A \emph{resolution refutation} of a set of clauses $\Gamma$ is a finite
sequence of resolution steps deriving $\square$ from $\Gamma$.

If a resolution refutation exists, then $\Gamma$ is unsatisfiable.
\end{tcolorbox}

% ---------------------------------------------------------
% Soundness and Completeness
% ---------------------------------------------------------
\begin{tcolorbox}[colback=thmbox, colframe=thmborder, arc=2pt,
  left=6pt, right=6pt, top=4pt, bottom=4pt,
  title={\small\textbf{Theorem (Soundness and Completeness of Resolution)}},
  fonttitle=\small\bfseries]
\label{thm:res-sound}\label{thm:res-complete}
A set of clauses $\Gamma$ is unsatisfiable if and only if the empty clause
$\square$ can be derived from $\Gamma$ by resolution:
\[
\Gamma \text{ unsatisfiable} \;\Longleftrightarrow\; \Gamma \vdash_{\mathrm{res}} \square
\]
\end{tcolorbox}

\begin{remark}[Soundness direction ($\Rightarrow$)]
If $\square$ is derived, then $\Gamma$ is unsatisfiable. This follows because
each resolution step is semantically valid (preserves consequence), and
$\square$ is always false.
\end{remark}

\begin{remark}[Completeness direction ($\Leftarrow$)]
If $\Gamma$ is unsatisfiable, resolution will eventually produce $\square$.
Resolution is \emph{refutation-complete}: it can detect every unsatisfiable
set, but it is not designed to enumerate all consequences.
\end{remark}

\begin{remark}[Logical implication]\label{rem:res-validity}
To test $\Gamma \models \varphi$:
\begin{enumerate}
  \item Negate: $\neg\varphi$.
  \item Convert $\Gamma \cup \{\neg\varphi\}$ to CNF.
  \item Run resolution. If $\square$ is derived, then $\Gamma \models \varphi$.
\end{enumerate}
Validity reduces to unsatisfiability, which resolution decides.
\end{remark}

\begin{remark}[Resolution strategies]\label{rem:res-strat}
Naive resolution may generate many irrelevant clauses. Strategies restrict
which pairs are resolved:
\begin{itemize}
  \item \textbf{Unit resolution:} Always resolve with a unit clause (single literal).
  \item \textbf{Set of support:} At least one parent must come from the
        negated goal (set of support). Keeps the search goal-directed.
  \item \textbf{Linear resolution:} Each step uses the most recently derived
        clause as one parent.
  \item \textbf{Input resolution:} At least one parent is always an original
        input clause.
\end{itemize}
All strategies preserve completeness for refutation.
\end{remark}

% =========================================================
% Proof Systems
% =========================================================

\subsection{Proof Systems}

% ---------------------------------------------------------
% TOOLKIT
% ---------------------------------------------------------
\begin{tcolorbox}[colback=gray!6, colframe=gray!40, arc=2pt,
  left=6pt, right=6pt, top=4pt, bottom=4pt,
  title={\small\textbf{Proof Systems — Quick Reference}},
  fonttitle=\small\bfseries]
\small
\begin{tabular}{l l l}
\toprule
\textbf{Concept} & \textbf{Meaning} & \textbf{Detail} \\
\midrule
Derivability $\Gamma\vdash\varphi$ & $\varphi$ provable from $\Gamma$ syntactically & \hyperref[def:derivability]{↓ Def} \\
Soundness    & $\Gamma\vdash\varphi \Rightarrow \Gamma\models\varphi$ & \hyperref[def:soundness]{↓ Def} \\
Completeness & $\Gamma\models\varphi \Rightarrow \Gamma\vdash\varphi$ & \hyperref[def:completeness]{↓ Def} \\
$\vdash\leftrightarrow\models$ & For PL: derivable iff semantically valid & \hyperref[thm:sound-complete]{↓ Thm} \\
\midrule
Semantic ($\models$) & Truth-assignment based; model-theoretic & \\
Syntactic ($\vdash$) & Rule-application based; proof-theoretic & \\
\bottomrule
\end{tabular}
\end{tcolorbox}

\vspace{1em}

% ---------------------------------------------------------
% Derivability
% ---------------------------------------------------------
\begin{tcolorbox}[colback=propbox, colframe=propborder, arc=2pt,
  left=6pt, right=6pt, top=4pt, bottom=4pt,
  title={\small\textbf{Definition (Derivability)}},
  fonttitle=\small\bfseries]
\label{def:derivability}
$\varphi$ is \emph{derivable} from $\Gamma$ in a proof system $\mathcal{P}$,
written $\Gamma \vdash \varphi$, if there exists a finite sequence of formulas
ending in $\varphi$, each of which is either:
\begin{itemize}
  \item an element of $\Gamma$ (an assumption), or
  \item obtained from earlier formulas by an inference rule of $\mathcal{P}$.
\end{itemize}
\end{tcolorbox}

\begin{remark}[English reading]
Derivability is a purely syntactic notion: it depends on the proof system's rules
and makes no reference to truth or models. A derivation is a finite certificate
that can be checked mechanically, step by step.
\end{remark}

\begin{remark}[Proof systems for propositional logic]
Common proof systems include:
\begin{itemize}
  \item \textbf{Natural deduction} — introduction and elimination rules for each
        connective; subproofs; discharge of assumptions. Closest to how
        mathematicians actually reason.
  \item \textbf{Sequent calculus} — operates on \emph{sequents} $\Gamma\vdash\Delta$;
        left and right rules; useful for proof theory.
  \item \textbf{Hilbert systems} — many axiom schemas, few rules (typically MP
        only). Compact to define, tedious to use.
  \item \textbf{Resolution} — single rule; requires CNF input; optimized for
        automated theorem proving.
\end{itemize}
All systems for classical propositional logic prove the same theorems.
\end{remark}

% ---------------------------------------------------------
% Soundness and Completeness
% ---------------------------------------------------------
\begin{tcolorbox}[colback=propbox, colframe=propborder, arc=2pt,
  left=6pt, right=6pt, top=4pt, bottom=4pt,
  title={\small\textbf{Definitions (Soundness and Completeness)}},
  fonttitle=\small\bfseries]
\label{def:soundness}
A proof system is \emph{sound} if every derivable formula is semantically valid:
\[
\Gamma \vdash \varphi \;\Rightarrow\; \Gamma \models \varphi.
\]

\label{def:completeness}
A proof system is \emph{complete} if every semantically valid formula is
derivable:
\[
\Gamma \models \varphi \;\Rightarrow\; \Gamma \vdash \varphi.
\]
\end{tcolorbox}

\begin{remark}[Why soundness matters]
Soundness prevents the proof system from proving false things. Without soundness,
a formal proof is worthless — it might ``prove'' $P \wedge \neg P$.
\end{remark}

\begin{remark}[Why completeness matters]
Completeness ensures the proof system is powerful enough: every logical truth
has a proof. Without completeness, the system might miss valid arguments.
\end{remark}

\begin{tcolorbox}[colback=thmbox, colframe=thmborder, arc=2pt,
  left=6pt, right=6pt, top=4pt, bottom=4pt,
  title={\small\textbf{Theorem (Soundness and Completeness of Propositional Logic)}},
  fonttitle=\small\bfseries]
\label{thm:sound-complete}
For any standard proof system for classical propositional logic (natural
deduction, Hilbert system, sequent calculus, resolution):
\[
\Gamma \vdash \varphi \;\Longleftrightarrow\; \Gamma \models \varphi.
\]
\end{tcolorbox}

\begin{remark}[Logical significance]
This equivalence is the central metatheorem of propositional logic. It means
the syntactic and semantic notions of entailment coincide perfectly.
A formula is provable if and only if it is true under all models.
\end{remark}

\begin{remark}[Contrast with first-order logic]
In first-order logic, completeness still holds (Gödel's completeness theorem,
1930). But for arithmetic (Peano arithmetic), Gödel's \emph{incompleteness}
theorems show that some true statements are unprovable. Propositional logic
avoids incompleteness because it is decidable — truth tables provide a
decision procedure.
\end{remark}

\begin{center}
\small
\renewcommand{\arraystretch}{1.3}
\begin{tabular}{lll}
\toprule
\textbf{Notion} & \textbf{Symbol} & \textbf{Nature} \\
\midrule
Tautological implication & $P \models_{\mathrm{taut}} Q$ & Semantic \\
Logical consequence & $\Gamma \models \varphi$ & Semantic, model-theoretic \\
Derivability & $\Gamma \vdash \varphi$ & Syntactic, proof-theoretic \\
\bottomrule
\end{tabular}
\end{center}


% =========================================================
% Metatheory — Orchestrator
% =========================================================
% Dependency order: functional completeness is self-contained;
% compactness and interpolation both require knowledge of satisfiability
% and logical consequence (from semantics/); interpolation follows
% compactness (both are advanced metatheorems).

% =========================================================
% Functional Completeness
% =========================================================

\subsection{Functional Completeness}

% ---------------------------------------------------------
% TOOLKIT
% ---------------------------------------------------------
\begin{tcolorbox}[colback=gray!6, colframe=gray!40, arc=2pt,
  left=6pt, right=6pt, top=4pt, bottom=4pt,
  title={\small\textbf{Functional Completeness — Quick Reference}},
  fonttitle=\small\bfseries]
\small
\begin{tabular}{l l l}
\toprule
\textbf{Concept} & \textbf{Key fact} & \textbf{Detail} \\
\midrule
Truth function         & $f:\{T,F\}^n\to\{T,F\}$; exactly $2^{2^n}$ of them & \hyperref[def:truth-fn]{↓ Def} \\
Functional completeness & Every truth function expressible from $S$ & \hyperref[def:func-complete]{↓ Def} \\
Adequate set           & Synonym for functionally complete set & \hyperref[def:adequate]{↓ Def} \\
Minimal adequate set   & No proper subset is adequate & \hyperref[def:minimal]{↓ Def} \\
\midrule
$\{\neg,\wedge\}$       & Adequate — expresses $\vee$ via De Morgan & \hyperref[thm:std-adequate]{↓ Thm} \\
$\{\neg,\vee\}$         & Adequate & \hyperref[thm:std-adequate]{↓ Thm} \\
$\{\neg,\rightarrow\}$  & Adequate & \hyperref[thm:std-adequate]{↓ Thm} \\
$\{\uparrow\}$ NAND     & Adequate (singleton!) & \hyperref[thm:nand-adequate]{↓ Thm} \\
$\{\downarrow\}$ NOR    & Adequate (singleton!) & \hyperref[thm:nor-adequate]{↓ Thm} \\
$\{\wedge,\vee\}$       & \textbf{Not} adequate — cannot express $\neg$ & \hyperref[thm:non-adequate]{↓ Thm} \\
Post's theorem          & Iff criterion for adequacy via 5 classes & \hyperref[thm:post]{↓ Thm} \\
\bottomrule
\end{tabular}

\vspace{0.4em}
\small $n$-ary truth functions: $n=1 \Rightarrow 4$ functions;\quad $n=2 \Rightarrow 16$;\quad $n=3 \Rightarrow 256$
\end{tcolorbox}

\vspace{1em}

% ---------------------------------------------------------
% Truth Function
% ---------------------------------------------------------
\begin{tcolorbox}[colback=propbox, colframe=propborder, arc=2pt,
  left=6pt, right=6pt, top=4pt, bottom=4pt,
  title={\small\textbf{Definition (Truth Function)}},
  fonttitle=\small\bfseries]
\label{def:truth-fn}
An $n$-ary \emph{truth function} is a function
\[
f : \{\mathsf{T}, \mathsf{F}\}^n \to \{\mathsf{T}, \mathsf{F}\}.
\]
There are exactly $2^{2^n}$ distinct $n$-ary truth functions.
\end{tcolorbox}

\begin{remark}[English reading]
A truth function is the abstract input-output behavior of a formula: it maps
each possible combination of truth values for $n$ variables to a single output.
Two formulas with the same truth table compute the same truth function.
\end{remark}

\begin{remark}[Counting]
For $n$ inputs, there are $2^n$ possible input rows. Each row independently
maps to $\mathsf{T}$ or $\mathsf{F}$, giving $2^{2^n}$ total functions.
For $n=1$: 4 functions (constant-T, identity, negation, constant-F).
For $n=2$: 16 functions, including $\wedge, \vee, \rightarrow, \leftrightarrow, \oplus, \uparrow, \downarrow$, and 9 others.
\end{remark}

\begin{remark}[Consequence]
Functional completeness asks: does a set of connectives generate \emph{all}
$2^{2^n}$ truth functions for every $n$? If yes, no expressible truth function
is beyond the reach of the set.
\end{remark}

\begin{example}
The 4 unary truth functions for input $P$:
\begin{center}
\renewcommand{\arraystretch}{1.2}
\begin{tabular}{c|cccc}
\toprule
$P$ & $\bot$ & $P$ & $\neg P$ & $\top$ \\
\midrule
T & F & T & F & T \\
F & F & F & T & T \\
\bottomrule
\end{tabular}
\end{center}
\end{example}

% ---------------------------------------------------------
% Functional Completeness
% ---------------------------------------------------------
\begin{tcolorbox}[colback=propbox, colframe=propborder, arc=2pt,
  left=6pt, right=6pt, top=4pt, bottom=4pt,
  title={\small\textbf{Definition (Functional Completeness)}},
  fonttitle=\small\bfseries]
\label{def:func-complete}
A set $S$ of logical connectives is \emph{functionally complete} (or
\emph{adequate}) if for every $n \geq 1$ and every $n$-ary truth function $f$,
there exists a formula $\varphi$ built using only connectives from $S$ whose
truth table computes $f$.
\end{tcolorbox}

\begin{remark}[Intuition]
$S$ is adequate if the logic built from $S$ can describe every possible
truth-functional relationship. No truth function escapes it.
\end{remark}

\begin{remark}[Fully quantified]
$S$ is functionally complete $\Longleftrightarrow$
$\forall n \geq 1,\; \forall f : \{T,F\}^n \to \{T,F\},\;
\exists \varphi \in \mathsf{WFF}(S),\;
\forall v : \{T,F\}^n,\; v(\varphi) = f(v)$.
\end{remark}

\begin{tcolorbox}[colback=propbox, colframe=propborder, arc=2pt,
  left=6pt, right=6pt, top=4pt, bottom=4pt,
  title={\small\textbf{Definition (Adequate Set / Minimal Adequate Set)}},
  fonttitle=\small\bfseries]
\label{def:adequate}\label{def:minimal}
An \emph{adequate set} is a functionally complete set.
A \emph{minimal adequate set} is an adequate set no proper subset of which is adequate.
\end{tcolorbox}

% ---------------------------------------------------------
% Standard Adequate Sets
% ---------------------------------------------------------
\begin{tcolorbox}[colback=thmbox, colframe=thmborder, arc=2pt,
  left=6pt, right=6pt, top=4pt, bottom=4pt,
  title={\small\textbf{Theorem (Standard Functionally Complete Sets)}},
  fonttitle=\small\bfseries]
\label{thm:std-adequate}
The following sets of connectives are functionally complete:
\begin{enumerate}
  \item $\{\neg, \wedge\}$
  \item $\{\neg, \vee\}$
  \item $\{\neg, \rightarrow\}$
  \item $\{\bot, \rightarrow\}$ \quad (constant false + conditional)
  \item $\{\neg, \wedge, \vee, \rightarrow, \leftrightarrow\}$ \quad (all standard connectives)
\end{enumerate}
\end{tcolorbox}

\begin{remark}[Proof sketch for $\{\neg, \wedge\}$]
Every truth function can be expressed in DNF using $\neg$, $\wedge$, $\vee$.
Since $P \vee Q \equiv \neg(\neg P \wedge \neg Q)$ (De Morgan), $\vee$ is
definable from $\{\neg, \wedge\}$. Therefore $\{\neg, \wedge\}$ expresses all
DNF formulas, hence all truth functions.
\end{remark}

\begin{remark}[Why $\{\neg, \wedge\}$ is minimal]
Remove either element: $\{\wedge\}$ alone cannot express negation (it
preserves falsity); $\{\neg\}$ alone cannot combine propositions (it never
increases the number of variables). Neither singleton is adequate. So
$\{\neg, \wedge\}$ has no redundant member.
\end{remark}

% ---------------------------------------------------------
% NAND and NOR
% ---------------------------------------------------------
\begin{tcolorbox}[colback=thmbox, colframe=thmborder, arc=2pt,
  left=6pt, right=6pt, top=4pt, bottom=4pt,
  title={\small\textbf{Theorem (NAND is Functionally Complete)}},
  fonttitle=\small\bfseries]
\label{thm:nand-adequate}
$\{\uparrow\}$ (NAND alone) is functionally complete.
\end{tcolorbox}



\begin{tcolorbox}[colback=thmbox, colframe=thmborder, arc=2pt,
  left=6pt, right=6pt, top=4pt, bottom=4pt,
  title={\small\textbf{Theorem (NOR is Functionally Complete)}},
  fonttitle=\small\bfseries]
\label{thm:nor-adequate}
$\{\downarrow\}$ (NOR alone) is functionally complete.
\end{tcolorbox}



\begin{remark}[Consequence — circuit design]
NAND and NOR are each singleton adequate sets. In digital hardware, any
logic circuit can be built entirely from NAND gates or entirely from NOR
gates. This is why NAND/NOR are called \emph{universal gates}.
\end{remark}

% ---------------------------------------------------------
% Non-adequate Sets
% ---------------------------------------------------------
\begin{tcolorbox}[colback=thmbox, colframe=thmborder, arc=2pt,
  left=6pt, right=6pt, top=4pt, bottom=4pt,
  title={\small\textbf{Theorem (Non-Adequate Sets)}},
  fonttitle=\small\bfseries]
\label{thm:non-adequate}
The following sets are \emph{not} functionally complete:
\begin{enumerate}
  \item $\{\wedge, \vee\}$ — cannot express $\neg$ (or any function that
        is false when all inputs are true).
  \item $\{\rightarrow\}$ alone — cannot express functions false when all
        inputs are false.
  \item $\{\neg\}$ alone — unary only; cannot build any binary function.
  \item $\{\leftrightarrow, \neg\}$ — cannot express $\wedge$ (this set
        generates only affine/XOR-based functions under close examination).
\end{enumerate}
\end{tcolorbox}

\begin{remark}[How to prove non-adequacy]
Identify a \emph{closure property} shared by all functions in $S$ that is not
shared by some target truth function. Common properties:
truth-preserving ($f(T,\ldots,T)=T$), falsity-preserving ($f(F,\ldots,F)=F$),
monotonic, affine (XOR-representable), self-dual.
If all members of $S$ share the property, $S$ cannot express functions lacking it.
For $\{\wedge,\vee\}$: both are truth-preserving and falsity-preserving, so the
set cannot express $\neg$ (which is neither).
\end{remark}

% ---------------------------------------------------------
% Post's Theorem
% ---------------------------------------------------------
\begin{tcolorbox}[colback=thmbox, colframe=thmborder, arc=2pt,
  left=6pt, right=6pt, top=4pt, bottom=4pt,
  title={\small\textbf{Theorem (Post's Functional Completeness Theorem)}},
  fonttitle=\small\bfseries]
\label{thm:post}
A set $S$ of Boolean functions is functionally complete if and only if, for
each of the following five \emph{Post classes}, $S$ contains at least one
function \emph{not} in that class:
\begin{enumerate}
  \item $\mathbf{T_0}$: \textbf{Truth-preserving} — $f(T,\ldots,T)=T$.
  \item $\mathbf{T_1}$: \textbf{Falsity-preserving} — $f(F,\ldots,F)=F$.
  \item $\mathbf{M}$: \textbf{Monotonic} — $x_i \leq y_i\;\forall i \Rightarrow f(\mathbf{x}) \leq f(\mathbf{y})$ (where $F<T$).
  \item $\mathbf{A}$: \textbf{Affine (linear)} — $f$ expressible as $a_0 \oplus a_1 x_1 \oplus \cdots \oplus a_n x_n$.
  \item $\mathbf{S}$: \textbf{Self-dual} — $f(x_1,\ldots,x_n) = \neg f(\neg x_1,\ldots,\neg x_n)$.
\end{enumerate}
\end{tcolorbox}

\begin{remark}[How to apply Post's theorem]
For each of the 5 classes, check whether $S$ contains a function outside that
class. If yes for all 5, $S$ is adequate. If $S$ misses any class entirely
(all members of $S$ lie within some class), $S$ is not adequate.
\end{remark}

\begin{remark}[Logical significance]
Post's theorem (Emil Post, 1941) gives a complete, finite, mechanical criterion
for adequacy. It is one of the deepest results of classical propositional logic,
characterizing the entire lattice of clones of Boolean functions.
\end{remark}

\begin{example}[Applying Post's theorem to $\{\neg, \wedge\}$]
\begin{center}
\renewcommand{\arraystretch}{1.3}
\begin{tabular}{lccc}
\toprule
\textbf{Class} & $\neg \in$ class? & $\wedge \in$ class? & \textbf{Some member outside?} \\
\midrule
$T_0$ truth-pres.  & No ($\neg T=F$) & Yes & Yes ($\neg$) \\
$T_1$ falsity-pres.& No ($\neg F=T$) & Yes & Yes ($\neg$) \\
$M$ monotone       & No             & Yes & Yes ($\neg$) \\
$A$ affine         & Yes ($\neg P = T\oplus P$) & No & Yes ($\wedge$) \\
$S$ self-dual      & Yes            & No  & Yes ($\wedge$) \\
\bottomrule
\end{tabular}
\end{center}
All 5 classes have a member of $S$ outside them. By Post's theorem,
$\{\neg,\wedge\}$ is functionally complete. $\checkmark$
\end{example}

% =========================================================
% Compactness Theorem
% =========================================================

\subsection{Compactness Theorem}

% ---------------------------------------------------------
% TOOLKIT
% ---------------------------------------------------------
\begin{tcolorbox}[colback=gray!6, colframe=gray!40, arc=2pt,
  left=6pt, right=6pt, top=4pt, bottom=4pt,
  title={\small\textbf{Compactness — Quick Reference}},
  fonttitle=\small\bfseries]
\small
\begin{tabular}{l l l}
\toprule
\textbf{Concept} & \textbf{Key fact} & \textbf{Detail} \\
\midrule
Compactness theorem & $\Gamma$ sat. $\Longleftrightarrow$ every finite $\Gamma_0\subseteq\Gamma$ sat. & \hyperref[thm:compactness]{↓ Thm} \\
Consequence form    & $\Gamma\models\varphi \Rightarrow$ some finite $\Gamma_0\models\varphi$ & \hyperref[cor:consequence-compact]{↓ Cor} \\
Graph coloring app. & Infinite graph $k$-colorable iff all finite subgraphs are & \hyperref[ex:graph-color]{↓ Ex} \\
Proof methods       & Via completeness; direct construction; ultraproducts & \hyperref[rem:compact-proof]{↓} \\
\bottomrule
\end{tabular}
\end{tcolorbox}

\vspace{1em}

% ---------------------------------------------------------
% Compactness Theorem
% ---------------------------------------------------------
\begin{tcolorbox}[colback=thmbox, colframe=thmborder, arc=2pt,
  left=6pt, right=6pt, top=4pt, bottom=4pt,
  title={\small\textbf{Theorem (Compactness of Propositional Logic)}},
  fonttitle=\small\bfseries]
\label{thm:compactness}
A set of formulas $\Gamma$ is satisfiable if and only if every finite subset
of $\Gamma$ is satisfiable.

Equivalently: if $\Gamma$ is unsatisfiable, some \emph{finite} subset of
$\Gamma$ is already unsatisfiable.
\end{tcolorbox}

\begin{remark}[English reading]
Compactness says that satisfiability is a \emph{finitary} property: an infinite
set of premises can only be contradictory if some finite fragment is already
contradictory. You can never need infinitely many premises to derive a
contradiction; finitely many always suffice.
\end{remark}

\begin{remark}[Fully quantified form]
$\forall \Gamma \subseteq \mathsf{WFF}:\;
\bigl(\exists v,\; v \models \Gamma\bigr)
\;\Longleftrightarrow\;
\bigl(\forall \Gamma_0 \subseteq_{\mathrm{fin}} \Gamma,\;
\exists v,\; v \models \Gamma_0\bigr)$.
\end{remark}

\begin{remark}[Which direction is trivial]
($\Rightarrow$): If $v$ satisfies $\Gamma$, it satisfies every subset. Trivial.

($\Leftarrow$): The substantive direction. If every finite subset is satisfiable,
is the whole set? Intuition says ``yes'' but it is not obvious — an infinite set
might encode constraints that collectively rule out every assignment even though
no finite fragment does.
\end{remark}

\begin{remark}[Why this is a \emph{meta}theorem]
Compactness is a statement about the entire logic, not about any particular
formula. It belongs to metatheory — it tells us something about the structure
of propositional logic itself.
\end{remark}

\begin{remark}[Proof methods]\label{rem:compact-proof}
Three standard approaches:
\begin{enumerate}
  \item \textbf{Via completeness:} If every finite subset of $\Gamma$ is
        satisfiable, no finite subset derives $\bot$. By completeness, $\Gamma$
        does not derive $\bot$. By soundness-completeness, $\Gamma$ is consistent,
        hence satisfiable. (Circular-looking but rigorous with careful bookkeeping.)
  \item \textbf{Direct construction (König's lemma):} Build a satisfying
        assignment by extending partial assignments level by level through all
        formulas in $\Gamma$, using compactness of $\{T,F\}^\omega$ (Tychonoff).
  \item \textbf{Ultraproducts:} Take an ultraproduct of the satisfying assignments
        for all finite subsets.
\end{enumerate}
\end{remark}

% ---------------------------------------------------------
% Consequence form
% ---------------------------------------------------------
\begin{corollary}[Compactness of Logical Consequence]\label{cor:consequence-compact}
If $\Gamma \models \varphi$, then there exists a \emph{finite} subset
$\Gamma_0 \subseteq \Gamma$ with $\Gamma_0 \models \varphi$.

Equivalently: logical consequence from an infinite premise set always depends
on only finitely many premises.
\end{corollary}



\begin{remark}[Practical implication]
In any propositional proof, one uses only finitely many formulas from $\Gamma$.
Compactness tells us this is not a limitation — we never \emph{need} infinitely
many premises for a single conclusion.
\end{remark}

% ---------------------------------------------------------
% Applications
% ---------------------------------------------------------
\begin{example}[Graph Coloring]\label{ex:graph-color}
\textbf{Claim:} An infinite graph $G$ is $k$-colorable if and only if every
finite subgraph of $G$ is $k$-colorable.

\textbf{Encoding:} For each vertex $v$ and color $i \in \{1,\ldots,k\}$,
introduce propositional variable $C_{v,i}$ (``vertex $v$ has color $i$'').
Add formulas:
\begin{itemize}
  \item For each $v$: $C_{v,1} \vee \cdots \vee C_{v,k}$ (every vertex has some color).
  \item For each $v$, distinct $i,j$: $\neg(C_{v,i} \wedge C_{v,j})$ (at most one color).
  \item For each edge $(u,v)$, each $i$: $\neg(C_{u,i} \wedge C_{v,i})$ (adjacent vertices differ).
\end{itemize}
Each finite subset of this formula set involves only finitely many vertices —
a finite subgraph. If every finite subgraph is $k$-colorable, every finite
subset is satisfiable. By compactness, the full set is satisfiable, giving a
$k$-coloring of $G$.
\end{example}

\begin{remark}[Why compactness is surprising here]
The graph coloring argument works even for uncountably infinite graphs.
Compactness is doing real work: it converts local, finite satisfiability into
global satisfiability for a system with infinitely many variables.
\end{remark}

% =========================================================
% Craig's Interpolation Theorem
% =========================================================

\subsection{Craig's Interpolation Theorem}

% ---------------------------------------------------------
% TOOLKIT
% ---------------------------------------------------------
\begin{tcolorbox}[colback=gray!6, colframe=gray!40, arc=2pt,
  left=6pt, right=6pt, top=4pt, bottom=4pt,
  title={\small\textbf{Interpolation — Quick Reference}},
  fonttitle=\small\bfseries]
\small
\begin{tabular}{l l l}
\toprule
\textbf{Concept} & \textbf{Key fact} & \textbf{Detail} \\
\midrule
Common language     & Variables appearing in both $\varphi$ and $\psi$ & \hyperref[def:common-lang]{↓ Def} \\
Interpolant         & $\theta$ s.t.\ $\varphi\models\theta\models\psi$ with vars in both & \hyperref[thm:craig]{↓ Thm} \\
Craig's theorem     & If $\varphi\models\psi$ (shared var), interpolant exists & \hyperref[thm:craig]{↓ Thm} \\
Trivial cases       & $\varphi\equiv\bot$: use $\bot$;\quad $\psi\equiv\top$: use $\top$ & \hyperref[rem:trivial]{↓} \\
Significance        & Modularity of reasoning; Beth definability; program verification & \hyperref[rem:significance]{↓} \\
\bottomrule
\end{tabular}
\end{tcolorbox}

\vspace{1em}

% ---------------------------------------------------------
% Common Language
% ---------------------------------------------------------
\begin{tcolorbox}[colback=propbox, colframe=propborder, arc=2pt,
  left=6pt, right=6pt, top=4pt, bottom=4pt,
  title={\small\textbf{Definition (Common Language)}},
  fonttitle=\small\bfseries]
\label{def:common-lang}
The \emph{common language} of formulas $\varphi$ and $\psi$ is the set of
propositional variables appearing in \emph{both} $\varphi$ and $\psi$:
\[
\mathsf{Var}(\varphi) \cap \mathsf{Var}(\psi).
\]
\end{tcolorbox}

\begin{remark}[Intuition]
Two formulas may talk about overlapping but distinct vocabularies. The common
language is the shared vocabulary — the concepts both formulas discuss.
Interpolation says: when $\varphi$ forces $\psi$, the reason can always be
expressed entirely in the shared vocabulary.
\end{remark}

% ---------------------------------------------------------
% Craig's Theorem
% ---------------------------------------------------------
\begin{tcolorbox}[colback=thmbox, colframe=thmborder, arc=2pt,
  left=6pt, right=6pt, top=4pt, bottom=4pt,
  title={\small\textbf{Theorem (Craig's Interpolation Theorem)}},
  fonttitle=\small\bfseries]
\label{thm:craig}
Let $\varphi$ and $\psi$ be propositional formulas with $\varphi \models \psi$
and $\mathsf{Var}(\varphi) \cap \mathsf{Var}(\psi) \neq \emptyset$.

Then there exists a formula $\theta$ (an \emph{interpolant}) such that:
\begin{enumerate}
  \item $\varphi \models \theta$
  \item $\theta \models \psi$
  \item $\mathsf{Var}(\theta) \subseteq \mathsf{Var}(\varphi) \cap \mathsf{Var}(\psi)$
\end{enumerate}
\end{tcolorbox}

\begin{remark}[English reading]
If $\varphi$ entails $\psi$, there is an intermediate formula $\theta$ that:
(1) follows from $\varphi$; (2) implies $\psi$; and (3) uses only variables
common to both. The interpolant $\theta$ captures exactly the logically relevant
content that $\varphi$ passes to $\psi$, in their shared vocabulary.
\end{remark}

\begin{remark}[Fully quantified form]
$\forall \varphi, \psi \in \mathsf{WFF}$:
if $\varphi \models \psi$ and $\mathsf{Var}(\varphi)\cap\mathsf{Var}(\psi)\neq\emptyset$,
then $\exists \theta \in \mathsf{WFF}$ such that
$\varphi\models\theta$, $\theta\models\psi$, and
$\mathsf{Var}(\theta)\subseteq\mathsf{Var}(\varphi)\cap\mathsf{Var}(\psi)$.
\end{remark}

\begin{remark}[Trivial cases]\label{rem:trivial}
\begin{itemize}
  \item If $\varphi$ is a contradiction, take $\theta = \bot$ (any $\psi$ follows
        from $\bot$, and $\bot$ uses no variables).
  \item If $\psi$ is a tautology, take $\theta = \top$ ($\varphi \models \top$
        always, and $\top \models \psi$ when $\psi$ is a tautology).
  \item If $\mathsf{Var}(\varphi) \cap \mathsf{Var}(\psi) = \emptyset$ and
        $\varphi \models \psi$, then $\varphi$ must be a contradiction or $\psi$
        a tautology (since no shared variable can transmit information). The
        hypothesis requires at least one shared variable to avoid this.
\end{itemize}
\end{remark}

\begin{example}
Let $\varphi = P \wedge Q$ and $\psi = P \vee R$. Then $\varphi \models \psi$
(since $P \wedge Q$ implies $P$, which implies $P \vee R$).
Shared variable: $P$.

Interpolant: $\theta = P$, since $P \wedge Q \models P$ and $P \models P \vee R$,
and $\mathsf{Var}(\theta) = \{P\} \subseteq \{P,Q\} \cap \{P,R\}$. $\checkmark$
\end{example}

\begin{example}
Let $\varphi = P \rightarrow Q$ and $\psi = \neg Q \rightarrow \neg P$
(contrapositive). They are logically equivalent, so $\varphi \models \psi$.
Shared variables: $\{P,Q\}$.

Interpolants include $\theta = P \rightarrow Q$, $\theta = \neg P \vee Q$, or
$\theta = \top$ (since $\psi$ is a tautology given $\varphi \equiv \psi$).
\end{example}

% ---------------------------------------------------------
% Significance
% ---------------------------------------------------------
\begin{remark}[Logical and mathematical significance]\label{rem:significance}
Craig's theorem (William Craig, 1957) has several important consequences:
\begin{enumerate}
  \item \textbf{Modularity of reasoning:} Whenever $\varphi \models \psi$, the
        ``reason'' can be expressed using only the shared vocabulary. This
        justifies modular reasoning in large formal systems — you never need to
        bring in private vocabulary of one module to pass a conclusion to another.
  \item \textbf{Beth definability:} A predicate is implicitly definable in a
        theory if and only if it is explicitly definable. In propositional terms:
        if a variable is determined by the rest of a formula, an explicit
        definition exists. Interpolation is the key proof tool for this.
  \item \textbf{Constructive proof via sequent calculus:} Interpolation can be
        proved constructively by tracing a cut-elimination proof in the sequent
        calculus. The interpolant is built simultaneously with the proof.
  \item \textbf{Program verification and model checking:} Interpolants are used
        to compute abstractions of reachability proofs in software model checking.
        Given a failing execution trace, an interpolant separates initial states
        from bad states.
\end{enumerate}
\end{remark}

\begin{remark}[Interpolation in first-order logic]
Craig's theorem extends to first-order logic: if $\varphi \models \psi$ in FOL,
an interpolant exists in their common language. The proof is harder, requiring
cut elimination in first-order sequent calculus. In higher-order logic,
interpolation can fail.
\end{remark}


% =========================================================
% Reference — Orchestrator
% =========================================================
% =========================================================
% Common Errors and Fallacies
% =========================================================

\subsection{Common Errors and Fallacies}

% ---------------------------------------------------------
% TOOLKIT
% ---------------------------------------------------------
\begin{tcolorbox}[colback=gray!6, colframe=gray!40, arc=2pt,
  left=6pt, right=6pt, top=4pt, bottom=4pt,
  title={\small\textbf{Errors and Fallacies — Quick Reference}},
  fonttitle=\small\bfseries]
\small
\begin{tabular}{l l l l}
\toprule
\textbf{Error} & \textbf{Form} & \textbf{Valid?} & \textbf{Detail} \\
\midrule
Modus Ponens          & $P\to Q,\;P \;\therefore Q$          & \textbf{Yes} & \hyperref[def:mp-valid]{↓} \\
Modus Tollens         & $P\to Q,\;\neg Q \;\therefore \neg P$ & \textbf{Yes} & \hyperref[def:mt-valid]{↓} \\
Affirming Consequent  & $P\to Q,\;Q \;\therefore P$          & No           & \hyperref[def:aff-cons]{↓} \\
Denying Antecedent    & $P\to Q,\;\neg P \;\therefore \neg Q$ & No           & \hyperref[def:deny-ant]{↓} \\
\midrule
Conditional misread   & $P\to Q$ is false when $P$ true, $Q$ false only & \hyperref[rem:cond-error]{↓} \\
Inclusive vs.\ exclusive $\vee$ & Logic uses inclusive or & \hyperref[rem:or-error]{↓} \\
Missing rows          & $n$ vars $\Rightarrow$ $2^n$ rows needed & \hyperref[rem:rows-error]{↓} \\
Precedence errors     & $\neg P\wedge Q \neq \neg(P\wedge Q)$ & \hyperref[rem:prec-error]{↓} \\
\bottomrule
\end{tabular}
\end{tcolorbox}

\vspace{1em}

\subsubsection*{Formal Fallacies}

\begin{tcolorbox}[colback=propbox, colframe=propborder, arc=2pt,
  left=6pt, right=6pt, top=4pt, bottom=4pt,
  title={\small\textbf{Definition (Affirming the Consequent)}},
  fonttitle=\small\bfseries]
\label{def:aff-cons}
The following inference is \textbf{invalid}:
\[
\frac{P \rightarrow Q \qquad Q}{P \quad \text{(INVALID)}}
\]
\end{tcolorbox}

\begin{remark}[Why it fails]
$P \rightarrow Q$ says: whenever $P$ is true, $Q$ must be true. It says nothing
about what happens when $Q$ is true. $Q$ may hold for reasons entirely unrelated
to $P$. Countermodel: $P = F$, $Q = T$. Then $P \rightarrow Q$ is true and $Q$
is true, but $P$ is false.
\end{remark}

\begin{remark}[English example]
``If it rains, the ground is wet. The ground is wet. Therefore, it rained.''
— Invalid: a sprinkler could explain the wet ground.
\end{remark}

\begin{remark}[Fully quantified diagnostic]
$\not\models [(P \rightarrow Q) \wedge Q] \rightarrow P$.
Check: set $P = F$, $Q = T$. The antecedent is $T \wedge T = T$; the consequent
is $F$. The whole formula is false under this assignment. Not a tautology.
\end{remark}

\begin{tcolorbox}[colback=propbox, colframe=propborder, arc=2pt,
  left=6pt, right=6pt, top=4pt, bottom=4pt,
  title={\small\textbf{Definition (Denying the Antecedent)}},
  fonttitle=\small\bfseries]
\label{def:deny-ant}
The following inference is \textbf{invalid}:
\[
\frac{P \rightarrow Q \qquad \neg P}{\neg Q \quad \text{(INVALID)}}
\]
\end{tcolorbox}

\begin{remark}[Why it fails]
$P \rightarrow Q$ only guarantees $Q$ when $P$ holds. When $P$ is false, $Q$
may be true or false — the conditional is silent. Countermodel: $P = F$, $Q = T$.
\end{remark}

\begin{remark}[English example]
``If it rains, the ground is wet. It did not rain. Therefore, the ground is not
wet.'' — Invalid: a sprinkler could still make the ground wet.
\end{remark}

\subsubsection*{Valid Inference Patterns for Comparison}

\begin{remark}\label{def:mp-valid}\label{def:mt-valid}
The two valid patterns involving a conditional:
\begin{center}
\renewcommand{\arraystretch}{1.3}
\begin{tabular}{llll}
\toprule
\textbf{Pattern} & \textbf{Form} & \textbf{Valid?} & \textbf{Why} \\
\midrule
Modus Ponens & $P\to Q,\; P \;\therefore Q$ & Yes & $\models[(P\to Q)\wedge P]\to Q$ \\
Modus Tollens & $P\to Q,\;\neg Q \;\therefore\neg P$ & Yes & Contrapositive of MP \\
Affirming Consequent & $P\to Q,\; Q \;\therefore P$ & \textbf{No} & Countermodel: $P=F, Q=T$ \\
Denying Antecedent & $P\to Q,\;\neg P \;\therefore\neg Q$ & \textbf{No} & Countermodel: $P=F, Q=T$ \\
\bottomrule
\end{tabular}
\end{center}
The two fallacies are easy to confuse because they \emph{look like} the valid
patterns with one element swapped.
\end{remark}

\subsubsection*{Common Truth Table Errors}

\begin{remark}[Misreading the conditional]\label{rem:cond-error}
The conditional $P \rightarrow Q$ is false \emph{only} in the case
$(P, Q) = (T, F)$. In all other cases it is true, including when $P$ is false.
This ``vacuous truth'' is counterintuitive but essential. The mistake is treating
the conditional as false whenever the antecedent is false.
\end{remark}

\begin{remark}[Confusing $\rightarrow$ with $\leftrightarrow$]\
$P \rightarrow Q$ is \emph{not} symmetric. $P \rightarrow Q$ and $Q \rightarrow P$
are different (they are converses). Treating a conditional as a biconditional
is a form of affirming the consequent.
\end{remark}

\begin{remark}[Inclusive vs.\ exclusive or]\label{rem:or-error}
In formal logic, $\vee$ is \emph{inclusive}: $P \vee Q$ is true when both
$P$ and $Q$ are true. Natural language ``or'' is often exclusive (``tea or
coffee'' usually means one but not both). When formalizing English, always
check which reading is intended.
\end{remark}

\begin{remark}[Forgetting rows]\label{rem:rows-error}
A formula with $n$ distinct variables has $2^n$ rows. For 3 variables: 8 rows.
Omitting rows leads to incorrect classification (e.g., calling a contingency a
tautology because you only checked favorable assignments).
\end{remark}

\begin{remark}[Operator precedence errors]\label{rem:prec-error}
\begin{itemize}
  \item $\neg P \wedge Q$ means $(\neg P) \wedge Q$, \textbf{not} $\neg(P \wedge Q)$.
  \item $P \vee Q \wedge R$ means $P \vee (Q \wedge R)$ if $\wedge$ binds tighter.
  \item When in doubt: add explicit parentheses.
\end{itemize}
\end{remark}

\subsubsection*{Fallacy Checklist}

\begin{center}
\renewcommand{\arraystretch}{1.4}
\begin{tabular}{p{4.5cm} p{8.5cm}}
\toprule
\textbf{Potential error} & \textbf{Diagnostic question} \\
\midrule
Affirming the consequent & Was $Q$ used to infer $P$ from $P \rightarrow Q$? \\
Denying the antecedent & Was $\neg P$ used to infer $\neg Q$ from $P \rightarrow Q$? \\
Conditional misread & Was the conditional treated as false when the antecedent is false? \\
Confusing direction & Was $P \rightarrow Q$ treated as equivalent to $Q \rightarrow P$? \\
Missing rows & Were all $2^n$ truth-table rows considered? \\
Precedence error & Were all connectives properly parenthesized before evaluating? \\
Inclusive/exclusive confusion & Was $\vee$ read as exclusive when inclusive was intended? \\
\bottomrule
\end{tabular}
\end{center}

% =========================================================
% Summary Tables
% =========================================================

\subsection{Summary Tables}

\subsubsection*{All Connectives — Truth Tables}

\begin{center}
\renewcommand{\arraystretch}{1.2}
\begin{tabular}{cc|ccccccc}
\toprule
$P$ & $Q$ & $\neg P$ & $P{\wedge}Q$ & $P{\vee}Q$ & $P{\to}Q$ & $P{\leftrightarrow}Q$ & $P{\oplus}Q$ & $P{\uparrow}Q$ \\
\midrule
T & T & F & T & T & T & T & F & F \\
T & F & F & F & T & F & F & T & T \\
F & T & T & F & T & T & F & T & T \\
F & F & T & F & F & T & T & F & T \\
\bottomrule
\end{tabular}

\vspace{0.4em}
$P \downarrow Q = \neg(P \vee Q)$: T only in the $F,F$ row.
\end{center}

\subsubsection*{Formula Classification}

\begin{center}
\renewcommand{\arraystretch}{1.3}
\begin{tabular}{l l l l}
\toprule
\textbf{Class} & \textbf{True under} & \textbf{Satisfiable?} & \textbf{Example} \\
\midrule
Tautology     & All assignments     & Yes (trivially)       & $P \vee \neg P$ \\
Contradiction & No assignment       & No                    & $P \wedge \neg P$ \\
Contingency   & Some, not all       & Yes                   & $P \rightarrow Q$ \\
\midrule
Satisfiable   & $\geq 1$ assignment & — (broader category)  & $P$, $P \vee Q$ \\
\bottomrule
\end{tabular}
\end{center}

\begin{remark}[Key relationship]
Satisfiable $=$ not a contradiction.
Every tautology is satisfiable; every contingency is satisfiable.
The three primary classes (tautology, contradiction, contingency) are mutually
exclusive and exhaustive.
\end{remark}

\subsubsection*{Logical Equivalences — Complete Reference}

\begin{center}
\renewcommand{\arraystretch}{1.3}
\begin{tabular}{l l}
\toprule
\textbf{Name} & \textbf{Equivalence} \\
\midrule
Double Negation       & $\neg\neg P \equiv P$ \\
De Morgan (1)         & $\neg(P \wedge Q) \equiv \neg P \vee \neg Q$ \\
De Morgan (2)         & $\neg(P \vee Q) \equiv \neg P \wedge \neg Q$ \\
Commutativity         & $P \wedge Q \equiv Q \wedge P$;\quad $P \vee Q \equiv Q \vee P$ \\
Associativity         & $(P \wedge Q) \wedge R \equiv P \wedge (Q \wedge R)$ (and for $\vee$) \\
Distributivity (1)    & $P \wedge (Q \vee R) \equiv (P \wedge Q) \vee (P \wedge R)$ \\
Distributivity (2)    & $P \vee (Q \wedge R) \equiv (P \vee Q) \wedge (P \vee R)$ \\
Idempotence           & $P \wedge P \equiv P$;\quad $P \vee P \equiv P$ \\
Absorption            & $P \wedge (P \vee Q) \equiv P$;\quad $P \vee (P \wedge Q) \equiv P$ \\
Identity              & $P \wedge \top \equiv P$;\quad $P \vee \bot \equiv P$ \\
Domination            & $P \vee \top \equiv \top$;\quad $P \wedge \bot \equiv \bot$ \\
Excluded Middle       & $P \vee \neg P \equiv \top$ \\
Non-Contradiction     & $P \wedge \neg P \equiv \bot$ \\
Material Implication  & $P \rightarrow Q \equiv \neg P \vee Q$ \\
Contraposition        & $P \rightarrow Q \equiv \neg Q \rightarrow \neg P$ \\
Exportation           & $(P \wedge Q) \rightarrow R \equiv P \rightarrow (Q \rightarrow R)$ \\
Neg.\ Conditional     & $\neg(P \rightarrow Q) \equiv P \wedge \neg Q$ \\
Bicond.\ Expansion    & $P \leftrightarrow Q \equiv (P \rightarrow Q) \wedge (Q \rightarrow P)$ \\
Bicond.\ Disjunction  & $P \leftrightarrow Q \equiv (P \wedge Q) \vee (\neg P \wedge \neg Q)$ \\
Neg.\ Biconditional   & $\neg(P \leftrightarrow Q) \equiv P \oplus Q \equiv P \leftrightarrow \neg Q$ \\
\bottomrule
\end{tabular}
\end{center}

\subsubsection*{Inference Rules — Valid vs.\ Invalid Patterns}

\begin{center}
\renewcommand{\arraystretch}{1.3}
\begin{tabular}{l l l}
\toprule
\textbf{Rule} & \textbf{Form} & \textbf{Valid?} \\
\midrule
Modus Ponens         & $P \rightarrow Q,\; P \;\therefore Q$                & \textbf{Yes} \\
Modus Tollens        & $P \rightarrow Q,\; \neg Q \;\therefore \neg P$      & \textbf{Yes} \\
Hypothetical Syllog. & $P \rightarrow Q,\; Q \rightarrow R \;\therefore P \rightarrow R$ & \textbf{Yes} \\
Disjunctive Syllog.  & $P \vee Q,\; \neg P \;\therefore Q$                  & \textbf{Yes} \\
Conj.\ Intro         & $P,\; Q \;\therefore P \wedge Q$                     & \textbf{Yes} \\
Disj.\ Intro         & $P \;\therefore P \vee Q$                            & \textbf{Yes} \\
\midrule
Affirm.\ Consequent  & $P \rightarrow Q,\; Q \;\therefore P$                & \textbf{No} \\
Denying Antecedent   & $P \rightarrow Q,\; \neg P \;\therefore \neg Q$      & \textbf{No} \\
\bottomrule
\end{tabular}
\end{center}

\subsubsection*{Functionally Complete Sets}

\begin{center}
\renewcommand{\arraystretch}{1.3}
\begin{tabular}{l c l}
\toprule
\textbf{Set} & \textbf{Adequate?} & \textbf{Remark} \\
\midrule
$\{\neg, \wedge\}$        & Yes & Minimal \\
$\{\neg, \vee\}$          & Yes & Minimal \\
$\{\neg, \rightarrow\}$   & Yes & Minimal \\
$\{\bot, \rightarrow\}$   & Yes & Minimal \\
$\{\uparrow\}$ (NAND)     & Yes & Minimal singleton \\
$\{\downarrow\}$ (NOR)    & Yes & Minimal singleton \\
$\{\neg,\wedge,\vee,\rightarrow,\leftrightarrow\}$ & Yes & Not minimal \\
\midrule
$\{\wedge, \vee\}$        & No  & Cannot express $\neg$ \\
$\{\rightarrow\}$         & No  & Cannot express falsity-preserving functions \\
$\{\neg\}$                & No  & Unary only \\
$\{\neg, \leftrightarrow\}$ & No & Cannot express $\wedge$ \\
\bottomrule
\end{tabular}
\end{center}

\subsubsection*{Key Metatheorems}

\begin{center}
\renewcommand{\arraystretch}{1.3}
\begin{tabular}{l p{9.5cm}}
\toprule
\textbf{Theorem} & \textbf{Statement} \\
\midrule
Soundness     & If $\Gamma \vdash \varphi$ then $\Gamma \models \varphi$ \\
Completeness  & If $\Gamma \models \varphi$ then $\Gamma \vdash \varphi$ \\
Sound+Complete & $\Gamma \vdash \varphi \Longleftrightarrow \Gamma \models \varphi$ \\
Compactness   & $\Gamma$ satisfiable $\Longleftrightarrow$ every finite $\Gamma_0 \subseteq \Gamma$ is satisfiable \\
Craig's Interp.& If $\varphi \models \psi$ (shared vars), $\exists\theta$ with $\varphi\models\theta\models\psi$ and $\mathsf{Var}(\theta)\subseteq\mathsf{Var}(\varphi)\cap\mathsf{Var}(\psi)$ \\
Post's Theorem & $S$ adequate $\Longleftrightarrow$ for each of 5 Post classes, some member of $S$ lies outside it \\
Unique Readability & Every wff has a unique parse tree \\
\bottomrule
\end{tabular}
\end{center}

\subsubsection*{Propositional Logic vs.\ Predicate Logic}

\begin{center}
\renewcommand{\arraystretch}{1.3}
\begin{tabular}{p{3.5cm} p{5cm} p{5cm}}
\toprule
\textbf{Aspect} & \textbf{Propositional Logic} & \textbf{Predicate Logic} \\
\midrule
Atomic formulas & Propositional variables ($P, Q, R$) & Predicate symbols applied to terms \\
Internal structure & None & Terms, variables, functions \\
Quantifiers & None & $\forall$, $\exists$ \\
Semantics & Truth assignments $v: \mathsf{Prop}\to\{T,F\}$ & Structures + variable assignments \\
Decision problem & Decidable (truth tables, $O(2^n)$) & Undecidable in general \\
Compactness & Yes & Yes (Gödel, 1930) \\
Interpolation & Yes & Yes \\
Soundness/Completeness & Yes & Yes (Gödel, 1930) \\
Incompleteness & No (decidable) & Yes for arithmetic (Gödel, 1931) \\
\bottomrule
\end{tabular}
\end{center}



