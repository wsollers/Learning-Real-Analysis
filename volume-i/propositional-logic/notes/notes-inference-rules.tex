% =========================================================
% Inference Rules
% =========================================================

\subsection{Inference Rules}
% ---------------------------------------------------------

\subsubsection{Basic Inference Rules}

\begin{definition}[Modus Ponens]
\textbf{Modus Ponens} (Law of Detachment): From a conditional and its antecedent,
infer the consequent.
\[
\begin{array}{l}
P \rightarrow Q \\
P \\ \hline
Q
\end{array}
\]
\end{definition}

\begin{definition}[Modus Tollens]
\textbf{Modus Tollens} (Denying the Consequent): From a conditional and the
negation of its consequent, infer the negation of the antecedent.
\[
\begin{array}{l}
P \rightarrow Q \\
\neg Q \\ \hline
\neg P
\end{array}
\]
\end{definition}

\begin{definition}[Hypothetical Syllogism]
\textbf{Hypothetical Syllogism} (Chain Rule): From two conditionals where the
consequent of the first is the antecedent of the second, infer a conditional
linking the first antecedent to the second consequent.
\[
\begin{array}{l}
P \rightarrow Q \\
Q \rightarrow R \\ \hline
P \rightarrow R
\end{array}
\]
\end{definition}

\begin{definition}[Disjunctive Syllogism]
\textbf{Disjunctive Syllogism} (Modus Tollendo Ponens): From a disjunction and the
negation of one disjunct, infer the other disjunct.
\[
\begin{array}{l}
P \vee Q \\
\neg P \\ \hline
Q
\end{array}
\qquad\text{or}\qquad
\begin{array}{l}
P \vee Q \\
\neg Q \\ \hline
P
\end{array}
\]
\end{definition}

\subsubsection{Conjunction Rules}

\begin{definition}[Conjunction Introduction]
\textbf{Conjunction Introduction} (Adjunction): From two propositions, infer
their conjunction.
\[
\begin{array}{l}
P \\
Q \\ \hline
P \wedge Q
\end{array}
\]
\end{definition}

\begin{definition}[Conjunction Elimination]
\textbf{Conjunction Elimination} (Simplification): From a conjunction, infer
either conjunct.
\[
\begin{array}{l}
P \wedge Q \\ \hline
P
\end{array}
\qquad\text{or}\qquad
\begin{array}{l}
P \wedge Q \\ \hline
Q
\end{array}
\]
\end{definition}

\subsubsection{Disjunction Rules}

\begin{definition}[Disjunction Introduction]
\textbf{Disjunction Introduction} (Addition): From a proposition, infer any
disjunction containing it.
\[
\begin{array}{l}
P \\ \hline
P \vee Q
\end{array}
\]
\end{definition}

\begin{definition}[Disjunction Elimination]
\textbf{Disjunction Elimination} (Proof by Cases): If a disjunction is true and
the same conclusion follows from each disjunct, then that conclusion may be
inferred.
\[
\begin{array}{l}
P \vee Q \\
\begin{array}{l}
\text{Assume } P \\ \vdots \\ R
\end{array}
\quad
\begin{array}{l}
\text{Assume } Q \\ \vdots \\ R
\end{array}
\\ \hline
R
\end{array}
\]
\end{definition}

\subsubsection{Conditional Rules}

\begin{definition}[Conditional Introduction]
\textbf{Conditional Introduction} (Conditional Proof): If assuming $P$ leads to
$Q$, then infer $P \rightarrow Q$.
\[
\begin{array}{l}
\text{Assume } P \\ \vdots \\ Q \\ \hline
P \rightarrow Q
\end{array}
\]
\end{definition}

\subsubsection{Biconditional Rules}

\begin{definition}[Biconditional Introduction]
\textbf{Biconditional Introduction}: From two conditionals in opposite directions,
infer their biconditional.
\[
\begin{array}{l}
P \rightarrow Q \\
Q \rightarrow P \\ \hline
P \leftrightarrow Q
\end{array}
\]
\end{definition}

\begin{definition}[Biconditional Elimination]
\textbf{Biconditional Elimination}: From a biconditional, infer either component
conditional.
\[
\begin{array}{l}
P \leftrightarrow Q \\ \hline
P \rightarrow Q
\end{array}
\qquad
\begin{array}{l}
P \leftrightarrow Q \\ \hline
Q \rightarrow P
\end{array}
\]
\end{definition}

\subsubsection{Negation Rules}

\begin{definition}[Negation Introduction]
\textbf{Negation Introduction} (Reductio ad Absurdum): If assuming $P$ leads to
a contradiction, then infer $\neg P$.
\[
\begin{array}{l}
\text{Assume } P \\ \vdots \\ \bot \\ \hline
\neg P
\end{array}
\]
\end{definition}

\begin{definition}[Negation Elimination]
\textbf{Negation Elimination} (Indirect Proof): If assuming $\neg P$ leads to a
contradiction, then infer $P$.
\[
\begin{array}{l}
\text{Assume } \neg P \\ \vdots \\ \bot \\ \hline
P
\end{array}
\]
\end{definition}

\begin{definition}[Double Negation Elimination]
\textbf{Double Negation Elimination}: From a double negation, infer the original
proposition.
\[
\begin{array}{l}
\neg\neg P \\ \hline
P
\end{array}
\]
\end{definition}

\subsubsection{Dilemma Rules}

\begin{definition}[Constructive Dilemma]
\textbf{Constructive Dilemma}: From a disjunction and two conditionals whose
antecedents are the disjuncts, infer the disjunction of the consequents.
\[
\begin{array}{l}
P \vee Q \\
P \rightarrow R \\
Q \rightarrow S \\ \hline
R \vee S
\end{array}
\]
\end{definition}

\begin{definition}[Destructive Dilemma]
\textbf{Destructive Dilemma}: From two conditionals and the disjunction of their
negated consequents, infer the disjunction of their negated antecedents.
\[
\begin{array}{l}
P \rightarrow R \\
Q \rightarrow S \\
\neg R \vee \neg S \\ \hline
\neg P \vee \neg Q
\end{array}
\]
\end{definition}

\subsubsection{Structural Rules}

\begin{definition}[Reiteration]
\textbf{Reiteration}: Any previously derived statement may be restated.
\[
\begin{array}{l}
P \\ \hline
P
\end{array}
\]
\end{definition}

\begin{definition}[Assumption]
\textbf{Assumption Rule}: A proposition may be temporarily assumed for the
purpose of a subproof. Assumptions must be discharged before the proof is
complete.
\end{definition}

\subsubsection{Summary of Inference Rules}

\begin{center}
\renewcommand{\arraystretch}{1.3}
\begin{tabular}{|l|l|l|}
\hline
\textbf{Rule} & \textbf{Premises} & \textbf{Conclusion} \\
\hline
Modus Ponens & $P \rightarrow Q$, $P$ & $Q$ \\
Modus Tollens & $P \rightarrow Q$, $\neg Q$ & $\neg P$ \\
Hypothetical Syllogism & $P \rightarrow Q$, $Q \rightarrow R$ & $P \rightarrow R$ \\
Disjunctive Syllogism & $P \vee Q$, $\neg P$ & $Q$ \\
Conjunction Introduction & $P$, $Q$ & $P \wedge Q$ \\
Conjunction Elimination & $P \wedge Q$ & $P$ (or $Q$) \\
Disjunction Introduction & $P$ & $P \vee Q$ \\
Constructive Dilemma & $P \vee Q$, $P \rightarrow R$, $Q \rightarrow S$ & $R \vee S$ \\
Destructive Dilemma & $P \rightarrow R$, $Q \rightarrow S$, $\neg R \vee \neg S$ & $\neg P \vee \neg Q$ \\
Biconditional Introduction & $P \rightarrow Q$, $Q \rightarrow P$ & $P \leftrightarrow Q$ \\
Biconditional Elimination & $P \leftrightarrow Q$ & $P \rightarrow Q$ \\
Double Negation Elimination & $\neg\neg P$ & $P$ \\
\hline
\end{tabular}
\end{center}

% ---------------------------------------------------------
