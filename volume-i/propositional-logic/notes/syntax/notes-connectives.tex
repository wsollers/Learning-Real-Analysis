% =========================================================
% Additional Connectives (XOR, NAND, NOR)
% =========================================================

\subsection{Additional Connectives}

% ---------------------------------------------------------
% TOOLKIT
% ---------------------------------------------------------
\begin{tcolorbox}[colback=gray!6, colframe=gray!40, arc=2pt,
  left=6pt, right=6pt, top=4pt, bottom=4pt,
  title={\small\textbf{Additional Connectives — Quick Reference}},
  fonttitle=\small\bfseries]
\small
\begin{tabular}{l l l l}
\toprule
\textbf{Symbol} & \textbf{Name} & \textbf{Equivalent} & \textbf{Detail} \\
\midrule
$P \oplus Q$    & XOR (Exclusive Or)  & $\neg(P \leftrightarrow Q)$  & \hyperref[def:xor]{↓ Def} \\
$P \uparrow Q$  & NAND (Sheffer stroke) & $\neg(P \wedge Q)$          & \hyperref[def:nand]{↓ Def} \\
$P \downarrow Q$& NOR (Peirce arrow)  & $\neg(P \vee Q)$             & \hyperref[def:nor]{↓ Def} \\
\midrule
$\{\uparrow\}$  & NAND alone is adequate & — & \hyperref[prop:nand-complete]{↓ Prop} \\
$\{\downarrow\}$& NOR alone is adequate  & — & \hyperref[prop:nor-complete]{↓ Prop} \\
\bottomrule
\end{tabular}
\end{tcolorbox}

\vspace{0.5em}

\begin{center}
\small
\renewcommand{\arraystretch}{1.2}
\begin{tabular}{cc|ccc}
\toprule
$P$ & $Q$ & $P \oplus Q$ & $P \uparrow Q$ & $P \downarrow Q$ \\
\midrule
T & T & F & F & F \\
T & F & T & T & F \\
F & T & T & T & F \\
F & F & F & T & T \\
\bottomrule
\end{tabular}
\end{center}

\vspace{1em}

% ---------------------------------------------------------
% XOR
% ---------------------------------------------------------
\begin{tcolorbox}[colback=propbox, colframe=propborder, arc=2pt,
  left=6pt, right=6pt, top=4pt, bottom=4pt,
  title={\small\textbf{Definition (Exclusive Or — XOR)}},
  fonttitle=\small\bfseries]
\label{def:xor}
The \emph{exclusive or}, denoted $P \oplus Q$ (also $P \veebar Q$), is true
when exactly one of $P$, $Q$ is true:
\[
P \oplus Q \;\equiv\; \neg(P \leftrightarrow Q)
\]
Equivalently: $(P \vee Q) \wedge \neg(P \wedge Q)$, and also
$(P \wedge \neg Q) \vee (\neg P \wedge Q)$.
\end{tcolorbox}

\begin{remark}[English reading]
XOR captures the natural-language ``either … or … but not both.'' It differs
from inclusive or ($\vee$) only in the $(T, T)$ row: $P \vee Q$ is true there,
$P \oplus Q$ is false.
\end{remark}

\begin{remark}[Fully quantified properties]
For all $P, Q, R$:
\begin{align*}
P \oplus Q &\equiv Q \oplus P \quad\text{(commutativity)}\\
(P \oplus Q) \oplus R &\equiv P \oplus (Q \oplus R) \quad\text{(associativity)}\\
P \oplus \bot &\equiv P \quad\text{(identity)}\\
P \oplus \top &\equiv \neg P \quad\text{(negation)}\\
P \oplus P &\equiv \bot \quad\text{(self-inverse)}\\
P \wedge (Q \oplus R) &\equiv (P \wedge Q) \oplus (P \wedge R) \quad\text{(distributivity of $\wedge$ over $\oplus$)}
\end{align*}
The self-inverse law makes $(\mathbb{B}, \oplus)$ a group — a fact that
resurfaces in abstract algebra.
\end{remark}

\begin{remark}[Consequence]
XOR is definable from standard connectives, so it adds no expressive power.
It is included for notational convenience and because it has a clean algebraic
structure (Boolean ring addition).
\end{remark}

% ---------------------------------------------------------
% NAND
% ---------------------------------------------------------
\begin{tcolorbox}[colback=propbox, colframe=propborder, arc=2pt,
  left=6pt, right=6pt, top=4pt, bottom=4pt,
  title={\small\textbf{Definition (NAND — Sheffer Stroke)}},
  fonttitle=\small\bfseries]
\label{def:nand}
The \emph{NAND connective} (Sheffer stroke), denoted $P \uparrow Q$ or $P | Q$,
is false only when both operands are true:
\[
P \uparrow Q \;\equiv\; \neg(P \wedge Q)
\]
\end{tcolorbox}

\begin{remark}[English reading]
``Not both $P$ and $Q$.'' NAND is true in every row of the truth table except
when both inputs are true. It is the most common single connective in digital
circuit design (NAND gates are universal).
\end{remark}

\begin{remark}[Fully quantified — expressing all connectives via NAND]
\begin{align*}
\neg P &\equiv P \uparrow P \\
P \wedge Q &\equiv (P \uparrow Q) \uparrow (P \uparrow Q) \\
P \vee Q &\equiv (P \uparrow P) \uparrow (Q \uparrow Q) \\
P \rightarrow Q &\equiv P \uparrow (Q \uparrow Q)
\end{align*}
\end{remark}

\begin{proposition}[Non-Associativity of NAND]
\label{prop:nand-complete}
NAND is \textbf{not associative}: $(P \uparrow Q) \uparrow R \not\equiv P \uparrow (Q \uparrow R)$.
\end{proposition}

\begin{remark}[Consequence]
Non-associativity means NAND expressions require careful parenthesization.
Despite this, $\{\uparrow\}$ alone is functionally complete — see the
Functional Completeness section.
\end{remark}

% ---------------------------------------------------------
% NOR
% ---------------------------------------------------------
\begin{tcolorbox}[colback=propbox, colframe=propborder, arc=2pt,
  left=6pt, right=6pt, top=4pt, bottom=4pt,
  title={\small\textbf{Definition (NOR — Peirce Arrow)}},
  fonttitle=\small\bfseries]
\label{def:nor}
The \emph{NOR connective} (Peirce arrow), denoted $P \downarrow Q$, is true
only when both operands are false:
\[
P \downarrow Q \;\equiv\; \neg(P \vee Q)
\]
\end{tcolorbox}

\begin{remark}[English reading]
``Neither $P$ nor $Q$.'' NOR is the dual of NAND (swap $\wedge \leftrightarrow \vee$
and $\top \leftrightarrow \bot$). It is also a universal gate in digital circuits.
\end{remark}

\begin{remark}[Fully quantified — expressing all connectives via NOR]
\begin{align*}
\neg P &\equiv P \downarrow P \\
P \vee Q &\equiv (P \downarrow Q) \downarrow (P \downarrow Q) \\
P \wedge Q &\equiv (P \downarrow P) \downarrow (Q \downarrow Q) \\
P \rightarrow Q &\equiv ((P \downarrow P) \downarrow Q) \downarrow ((P \downarrow P) \downarrow Q)
\end{align*}
\end{remark}

\begin{proposition}[Non-Associativity of NOR]
\label{prop:nor-complete}
NOR is \textbf{not associative}: $(P \downarrow Q) \downarrow R \not\equiv P \downarrow (Q \downarrow R)$.
\end{proposition}

\begin{remark}[Duality with NAND]
NAND and NOR are De Morgan duals: replacing each by the other and swapping
$\wedge \leftrightarrow \vee$ transforms any NAND circuit into a NOR circuit
computing the same function. They share identical functional completeness
properties.
\end{remark}
