% =========================================================
% Proof Systems
% =========================================================

\subsection{Proof Systems}
% ---------------------------------------------------------

\subsubsection{Derivability}

\begin{definition}[Derivability]
Let $\Gamma$ be a set of formulas and $\varphi$ a formula.
We say that $\varphi$ is \emph{derivable} from $\Gamma$, written
\[
\Gamma \vdash \varphi,
\]
if there exists a formal proof of $\varphi$ from the assumptions in $\Gamma$
using the rules of a specified deductive system.
\end{definition}

\begin{remark}
Derivability is a \emph{syntactic} notion. It depends on a chosen proof system
and concerns what can be obtained by applying inference rules step by step.
Unlike logical consequence, derivability makes no direct reference to truth or
interpretations.
\end{remark}

\subsubsection{Soundness and Completeness}

\begin{definition}[Soundness]
A proof system is \emph{sound} if every derivable formula is valid.

Formally, if $\Gamma \vdash \varphi$, then $\Gamma \models \varphi$.
\end{definition}

\begin{definition}[Completeness]
A proof system is \emph{complete} if every valid formula is derivable.

Formally, if $\Gamma \models \varphi$, then $\Gamma \vdash \varphi$.
\end{definition}

\begin{theorem}[Soundness and Completeness of Propositional Logic]
The standard proof systems for propositional logic (natural deduction, sequent
calculus, Hilbert systems, resolution) are both sound and complete.

Thus, for propositional logic:
\[
\Gamma \vdash \varphi \quad\Longleftrightarrow\quad \Gamma \models \varphi.
\]
\end{theorem}

\begin{remark}
Soundness ensures that proofs do not lead us astray: we cannot prove false
statements from true premises. Completeness ensures that proofs are powerful
enough: every valid argument can be formally verified.
\end{remark}

\subsubsection{Comparison: Semantic vs.\ Syntactic Notions}

\begin{center}
\renewcommand{\arraystretch}{1.3}
\begin{tabular}{|l|l|l|}
\hline
\textbf{Notion} & \textbf{Symbol} & \textbf{Nature} \\
\hline
Tautological implication & $P \models_{\mathsf{taut}} Q$ & Semantic, truth-table based \\
Logical consequence & $\Gamma \models \varphi$ & Semantic, model-theoretic \\
Derivability & $\Gamma \vdash \varphi$ & Syntactic, proof-theoretic \\
\hline
\end{tabular}
\end{center}

\begin{remark}
The distinction between $\models$ (semantic entailment) and $\vdash$ (syntactic
derivability) is central to logic. Soundness and completeness theorems establish
the correspondence between these two perspectives.
\end{remark}

% ---------------------------------------------------------
