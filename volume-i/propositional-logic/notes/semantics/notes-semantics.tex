% =========================================================
% Semantics of Propositional Logic
% =========================================================

\subsection{Semantics of Propositional Logic}

% ---------------------------------------------------------
% TOOLKIT
% ---------------------------------------------------------
\begin{tcolorbox}[colback=gray!6, colframe=gray!40, arc=2pt,
  left=6pt, right=6pt, top=4pt, bottom=4pt,
  title={\small\textbf{Semantics Toolkit — Quick Reference}},
  fonttitle=\small\bfseries]
\small
\begin{tabular}{l l l}
\toprule
\textbf{Concept} & \textbf{Meaning} & \textbf{Detail} \\
\midrule
Truth assignment   & Function $v: \mathsf{Prop} \to \{\mathsf{T},\mathsf{F}\}$ & \hyperref[def:truth-assign]{↓ Def} \\
Truth under $v$    & $v \models \varphi$ iff $v(\varphi) = \mathsf{T}$ & \hyperref[def:satisfaction]{↓ Def} \\
Tautology          & True under all assignments ($\models\varphi$) & \hyperref[def:tautology]{↓ Def} \\
Contradiction      & False under all assignments & \hyperref[def:contradiction]{↓ Def} \\
Satisfiable        & True under at least one assignment & \hyperref[def:satisfiable]{↓ Def} \\
Contingency        & Neither tautology nor contradiction & \hyperref[def:contingency]{↓ Def} \\
Logical consequence & $\Gamma \models \varphi$ & \hyperref[def:consequence]{↓ Def} \\
Logical equivalence & $\varphi \equiv \psi$ ($\varphi \leftrightarrow \psi$ is a tautology) & \hyperref[def:log-equiv]{↓ Def} \\
\bottomrule
\end{tabular}
\end{tcolorbox}

\vspace{1em}

% ---------------------------------------------------------
% Truth Assignment and Extension
% ---------------------------------------------------------
\begin{tcolorbox}[colback=propbox, colframe=propborder, arc=2pt,
  left=6pt, right=6pt, top=4pt, bottom=4pt,
  title={\small\textbf{Definition (Truth Assignment)}},
  fonttitle=\small\bfseries]
\label{def:truth-assign}
A \emph{truth assignment} (or \emph{valuation}) is a function
\[
v : \mathsf{Prop} \to \{\mathsf{T}, \mathsf{F}\}.
\]
Every truth assignment extends \emph{uniquely} to
$\hat{v} : \mathsf{WFF} \to \{\mathsf{T}, \mathsf{F}\}$ by:
\begin{enumerate}
  \item $\hat{v}(P) = v(P)$ for $P \in \mathsf{Prop}$.
  \item $\hat{v}(\neg\varphi) = \mathsf{T}$ iff $\hat{v}(\varphi) = \mathsf{F}$.
  \item $\hat{v}(\varphi \wedge \psi) = \mathsf{T}$ iff $\hat{v}(\varphi) = \hat{v}(\psi) = \mathsf{T}$.
  \item $\hat{v}(\varphi \vee \psi) = \mathsf{T}$ iff $\hat{v}(\varphi) = \mathsf{T}$ or $\hat{v}(\psi) = \mathsf{T}$.
  \item $\hat{v}(\varphi \rightarrow \psi) = \mathsf{T}$ iff $\hat{v}(\varphi) = \mathsf{F}$ or $\hat{v}(\psi) = \mathsf{T}$.
  \item $\hat{v}(\varphi \leftrightarrow \psi) = \mathsf{T}$ iff $\hat{v}(\varphi) = \hat{v}(\psi)$.
\end{enumerate}
We write $v(\varphi)$ for $\hat{v}(\varphi)$.
\end{tcolorbox}

\begin{remark}[English reading]
A truth assignment is an interpretation: it gives a definite truth value to
every atomic proposition, and the extension rules then compute the truth value
of every compound formula mechanically. There is no ambiguity once $v$ is fixed.
\end{remark}

\begin{remark}[Fully quantified]
$\hat{v}$ is the unique function $\mathsf{WFF} \to \{\mathsf{T},\mathsf{F}\}$
extending $v$ and compatible with the recursive structure of wffs.
Uniqueness follows from unique readability: each wff has exactly one form,
so the recursive cases never conflict.
\end{remark}

\begin{remark}[Truth table connection]
A formula with $n$ distinct variables has exactly $2^n$ distinct truth
assignments. The truth table lists all of them, one per row.
\end{remark}

\begin{center}
\small\textbf{Truth Tables for Standard Connectives}

\vspace{0.5em}
\renewcommand{\arraystretch}{1.2}
\begin{tabular}{cc|ccccc}
\toprule
$P$ & $Q$ & $\neg P$ & $P \wedge Q$ & $P \vee Q$ & $P \rightarrow Q$ & $P \leftrightarrow Q$ \\
\midrule
T & T & F & T & T & T & T \\
T & F & F & F & T & F & F \\
F & T & T & F & T & T & F \\
F & F & T & F & F & T & T \\
\bottomrule
\end{tabular}
\end{center}

% ---------------------------------------------------------
% Satisfaction
% ---------------------------------------------------------
\begin{tcolorbox}[colback=propbox, colframe=propborder, arc=2pt,
  left=6pt, right=6pt, top=4pt, bottom=4pt,
  title={\small\textbf{Definition (Satisfaction)}},
  fonttitle=\small\bfseries]
\label{def:satisfaction}
A truth assignment $v$ \emph{satisfies} formula $\varphi$, written
$v \models \varphi$, if $v(\varphi) = \mathsf{T}$.

$v$ satisfies a set $\Gamma$ if $v \models \varphi$ for every $\varphi \in \Gamma$.
\end{tcolorbox}

\begin{remark}[Intuition]
$v \models \varphi$ is the semantic turnstile: $v$ makes $\varphi$ true.
The symbol $\models$ without a subscript on the left will be reused for
logical consequence ($\Gamma \models \varphi$) — context determines which reading.
\end{remark}

% ---------------------------------------------------------
% Tautology / Contradiction / Satisfiable / Contingency
% ---------------------------------------------------------
\begin{tcolorbox}[colback=propbox, colframe=propborder, arc=2pt,
  left=6pt, right=6pt, top=4pt, bottom=4pt,
  title={\small\textbf{Definitions (Formula Classification)}},
  fonttitle=\small\bfseries]
\begin{description}
  \item[Tautology\label{def:tautology}] $\varphi$ is a \emph{tautology} (written $\models\varphi$)
    if $v \models \varphi$ for every truth assignment $v$.
  \item[Contradiction\label{def:contradiction}] $\varphi$ is a \emph{contradiction} (or \emph{unsatisfiable})
    if $v \not\models \varphi$ for every $v$.
  \item[Satisfiable\label{def:satisfiable}] $\varphi$ is \emph{satisfiable} if there exists some $v$ with $v \models \varphi$.
  \item[Contingency\label{def:contingency}] $\varphi$ is a \emph{contingency} if it is neither a tautology nor a contradiction.
\end{description}
\end{tcolorbox}

\begin{remark}[Mutual exclusivity and exhaustiveness]
Every formula belongs to exactly one category:
\begin{center}
\renewcommand{\arraystretch}{1.2}
\begin{tabular}{lcc}
\toprule
\textbf{Category} & \textbf{True under all $v$?} & \textbf{True under some $v$?} \\
\midrule
Tautology     & Yes & Yes (trivially) \\
Contradiction & No  & No \\
Contingency   & No  & Yes \\
\bottomrule
\end{tabular}
\end{center}
\end{remark}

\begin{remark}[Key logical implication]
A formula is satisfiable if and only if it is \emph{not} a contradiction.
This equivalence is the bridge between satisfiability and unsatisfiability,
and is the basis for refutation proofs (prove $\varphi$ by deriving a
contradiction from $\neg\varphi$).
\end{remark}

\begin{example}
\begin{itemize}
  \item $P \vee \neg P$ — tautology (law of excluded middle).
  \item $P \wedge \neg P$ — contradiction.
  \item $P \rightarrow Q$ — contingency (false when $P = \mathsf{T}$, $Q = \mathsf{F}$).
\end{itemize}
\end{example}

% ---------------------------------------------------------
% Logical Consequence
% ---------------------------------------------------------
\begin{tcolorbox}[colback=propbox, colframe=propborder, arc=2pt,
  left=6pt, right=6pt, top=4pt, bottom=4pt,
  title={\small\textbf{Definition (Logical Consequence)}},
  fonttitle=\small\bfseries]
\label{def:consequence}
$\varphi$ is a \emph{logical consequence} of $\Gamma$, written $\Gamma \models \varphi$,
if every truth assignment satisfying $\Gamma$ also satisfies $\varphi$.

Formally: $\forall v,\; (v \models \Gamma) \Rightarrow (v \models \varphi)$.
\end{tcolorbox}

\begin{remark}[English reading]
$\Gamma \models \varphi$ means: there is no way to make all of $\Gamma$ true
while making $\varphi$ false. The premise set $\Gamma$ semantically forces $\varphi$.
\end{remark}

\begin{remark}[Special cases]
\begin{itemize}
  \item $\emptyset \models \varphi$ iff $\varphi$ is a tautology.
  \item $\Gamma \models \varphi$ for any $\Gamma$ containing a contradiction.
  \item $\{P\} \models Q$ iff $P \rightarrow Q$ is a tautology
        (\emph{tautological implication}).
\end{itemize}
\end{remark}

\begin{remark}[Connection to proof systems]
The fundamental theorem of propositional logic:
$\Gamma \models \varphi \;\Leftrightarrow\; \Gamma \vdash \varphi$ (soundness and completeness).
The left side is semantic; the right side is syntactic. They coincide for
propositional logic.
\end{remark}

% ---------------------------------------------------------
% Logical Equivalence
% ---------------------------------------------------------
\begin{tcolorbox}[colback=propbox, colframe=propborder, arc=2pt,
  left=6pt, right=6pt, top=4pt, bottom=4pt,
  title={\small\textbf{Definition (Logical Equivalence)}},
  fonttitle=\small\bfseries]
\label{def:log-equiv}
$\varphi$ and $\psi$ are \emph{logically equivalent}, written $\varphi \equiv \psi$,
if they have the same truth value under every truth assignment. Equivalently:
\begin{itemize}
  \item $\varphi \models \psi$ and $\psi \models \varphi$, or
  \item $\varphi \leftrightarrow \psi$ is a tautology, or
  \item $\varphi$ and $\psi$ have identical truth tables.
\end{itemize}
\end{tcolorbox}

\begin{remark}[Intuition]
Logically equivalent formulas say the same thing — they agree on every possible
state of affairs. They can be freely substituted for one another in any context.
\end{remark}

\begin{remark}[Equivalence is an equivalence relation]
$\equiv$ is reflexive ($\varphi \equiv \varphi$), symmetric, and transitive.
Moreover, it is a \emph{congruence}: if $\varphi \equiv \psi$, then
$(\varphi \circ \chi) \equiv (\psi \circ \chi)$ for any connective $\circ$ and
formula $\chi$. This substitution property is what makes equivalences useful
for formula simplification.
\end{remark}
