% =========================================================
% Normal Forms
% =========================================================

\subsection{Normal Forms}

% ---------------------------------------------------------
% TOOLKIT
% ---------------------------------------------------------
\begin{tcolorbox}[colback=gray!6, colframe=gray!40, arc=2pt,
  left=6pt, right=6pt, top=4pt, bottom=4pt,
  title={\small\textbf{Normal Forms — Quick Reference}},
  fonttitle=\small\bfseries]
\small
\begin{tabular}{l l l}
\toprule
\textbf{Form} & \textbf{Structure} & \textbf{Detail} \\
\midrule
Literal      & $P$ or $\neg P$ & \hyperref[def:literal]{↓ Def} \\
Clause (disjunctive) & Disjunction of literals & \hyperref[def:clause]{↓ Def} \\
NNF          & Negations only on variables; uses $\neg,\wedge,\vee$ only & \hyperref[def:nnf]{↓ Def} \\
CNF          & Conjunction of disjunctions of literals $\bigwedge_i\bigvee_j L_{ij}$ & \hyperref[def:cnf]{↓ Def} \\
DNF          & Disjunction of conjunctions of literals $\bigvee_i\bigwedge_j L_{ij}$ & \hyperref[def:dnf]{↓ Def} \\
\midrule
NNF existence & Every formula $\equiv$ some NNF formula & \hyperref[prop:nnf-exists]{↓ Prop} \\
CNF existence & Every formula $\equiv$ some CNF formula & \hyperref[prop:cnf-exists]{↓ Prop} \\
DNF existence & Every formula $\equiv$ some DNF formula & \hyperref[prop:dnf-exists]{↓ Prop} \\
\bottomrule
\end{tabular}

\vspace{0.5em}
\textbf{Conversion pipeline:} Original formula $\xrightarrow{\text{elim } \rightarrow,\leftrightarrow}$
NNF $\xrightarrow{\text{distribute } \vee/\wedge}$ CNF or DNF
\end{tcolorbox}

\vspace{1em}

% ---------------------------------------------------------
% Literals and Clauses
% ---------------------------------------------------------
\begin{tcolorbox}[colback=propbox, colframe=propborder, arc=2pt,
  left=6pt, right=6pt, top=4pt, bottom=4pt,
  title={\small\textbf{Definitions (Literal and Clause)}},
  fonttitle=\small\bfseries]
\label{def:literal}
A \emph{literal} is a propositional variable ($P$, a \emph{positive literal})
or the negation of one ($\neg P$, a \emph{negative literal}).

\label{def:clause}
A \emph{(disjunctive) clause} is a disjunction of literals:
$L_1 \vee L_2 \vee \cdots \vee L_k$.

A \emph{conjunctive clause} (or \emph{term}) is a conjunction of literals:
$L_1 \wedge L_2 \wedge \cdots \wedge L_k$.
\end{tcolorbox}

\begin{remark}[Why literals matter]
CNF and DNF are defined in terms of literals, not arbitrary formulas. Literals
are the atoms of normal form theory. Every formula in NNF is a tree whose
leaves are literals.
\end{remark}

% ---------------------------------------------------------
% NNF
% ---------------------------------------------------------
\begin{tcolorbox}[colback=propbox, colframe=propborder, arc=2pt,
  left=6pt, right=6pt, top=4pt, bottom=4pt,
  title={\small\textbf{Definition (Negation Normal Form — NNF)}},
  fonttitle=\small\bfseries]
\label{def:nnf}
A formula is in \emph{negation normal form} (NNF) if:
\begin{enumerate}
  \item Negation ($\neg$) is applied only to propositional variables (no negations of compound formulas).
  \item The only connectives used are $\neg$, $\wedge$, $\vee$.
\end{enumerate}
\end{tcolorbox}

\begin{proposition}\label{prop:nnf-exists}
Every propositional formula is logically equivalent to a formula in NNF.
\end{proposition}

\begin{remark}[Conversion procedure]
\begin{enumerate}
  \item Eliminate $\rightarrow$: replace $P \rightarrow Q$ with $\neg P \vee Q$.
  \item Eliminate $\leftrightarrow$: replace $P \leftrightarrow Q$ with
        $(P \wedge Q) \vee (\neg P \wedge \neg Q)$.
  \item Push $\neg$ inward via De Morgan and double negation:
        $\neg(P \wedge Q) \to \neg P \vee \neg Q$;\quad
        $\neg(P \vee Q) \to \neg P \wedge \neg Q$;\quad
        $\neg\neg P \to P$.
\end{enumerate}
\end{remark}

\begin{remark}[Why NNF first]
NNF is the prerequisite for CNF/DNF conversion. Once in NNF, the only remaining
structural operations are distributing $\wedge$ over $\vee$ (for CNF) or
$\vee$ over $\wedge$ (for DNF).
\end{remark}

% ---------------------------------------------------------
% CNF
% ---------------------------------------------------------
\begin{tcolorbox}[colback=propbox, colframe=propborder, arc=2pt,
  left=6pt, right=6pt, top=4pt, bottom=4pt,
  title={\small\textbf{Definition (Conjunctive Normal Form — CNF)}},
  fonttitle=\small\bfseries]
\label{def:cnf}
A formula is in \emph{conjunctive normal form} (CNF) if it is a conjunction of
disjunctive clauses:
\[
\bigwedge_{i=1}^{m} \bigvee_{j=1}^{n_i} L_{ij}
\]
where each $L_{ij}$ is a literal.
\end{tcolorbox}

\begin{proposition}\label{prop:cnf-exists}
Every propositional formula is logically equivalent to a formula in CNF.
\end{proposition}

\begin{remark}[CNF conversion from NNF]
Distribute $\vee$ over $\wedge$ repeatedly:
$P \vee (Q \wedge R) \equiv (P \vee Q) \wedge (P \vee R)$.
Repeat until no $\vee$ sits above a $\wedge$ in the parse tree.
\end{remark}

\begin{remark}[Truth-table method for CNF]
For each row where the formula is \emph{false}, form a disjunctive clause
using positive literals for variables assigned $\mathsf{F}$ and negative literals
for variables assigned $\mathsf{T}$. Take the conjunction of all such clauses.
\end{remark}

\begin{remark}[Consequence — resolution]
CNF is the required input format for the resolution proof procedure.
Every satisfiability solver (SAT solver) works on CNF.
\end{remark}

\begin{example}
$(P \vee Q) \wedge (\neg P \vee R) \wedge (Q \vee \neg R)$ is in CNF.
\end{example}

% ---------------------------------------------------------
% DNF
% ---------------------------------------------------------
\begin{tcolorbox}[colback=propbox, colframe=propborder, arc=2pt,
  left=6pt, right=6pt, top=4pt, bottom=4pt,
  title={\small\textbf{Definition (Disjunctive Normal Form — DNF)}},
  fonttitle=\small\bfseries]
\label{def:dnf}
A formula is in \emph{disjunctive normal form} (DNF) if it is a disjunction of
conjunctive clauses:
\[
\bigvee_{i=1}^{m} \bigwedge_{j=1}^{n_i} L_{ij}
\]
where each $L_{ij}$ is a literal.
\end{tcolorbox}

\begin{proposition}\label{prop:dnf-exists}
Every propositional formula is logically equivalent to a formula in DNF.
\end{proposition}

\begin{remark}[DNF conversion from NNF]
Distribute $\wedge$ over $\vee$ repeatedly:
$P \wedge (Q \vee R) \equiv (P \wedge Q) \vee (P \wedge R)$.
\end{remark}

\begin{remark}[Truth-table method for DNF]
For each row where the formula is \emph{true}, form a conjunctive clause
using positive literals for variables assigned $\mathsf{T}$ and negative literals
for variables assigned $\mathsf{F}$. Take the disjunction of all such clauses.
\end{remark}

\begin{remark}[CNF vs.\ DNF trade-off]
CNF and DNF can both be exponentially larger than the original formula.
Neither is always more compact. The choice depends on application:
resolution needs CNF; model enumeration benefits from DNF.
\end{remark}
