% =========================================================
% Craig's Interpolation Theorem
% =========================================================

\subsection{Craig's Interpolation Theorem}
% ---------------------------------------------------------

\subsubsection{Statement of Craig's Theorem}

\begin{definition}[Common Language]
Given formulas $\varphi$ and $\psi$, the \emph{common language} of $\varphi$ and
$\psi$ consists of the propositional variables that occur in both formulas.
\end{definition}

\begin{theorem}[Craig's Interpolation Theorem]
Let $\varphi$ and $\psi$ be propositional formulas such that $\varphi \models \psi$
and $\varphi$ and $\psi$ share at least one propositional variable.

Then there exists a formula $\theta$ (called an \emph{interpolant}) such that:
\begin{enumerate}
  \item $\varphi \models \theta$
  \item $\theta \models \psi$
  \item Every propositional variable in $\theta$ occurs in both $\varphi$ and $\psi$.
\end{enumerate}
\end{theorem}

\begin{remark}
The interpolant $\theta$ serves as a ``bridge'' between $\varphi$ and $\psi$,
capturing the logically relevant content that $\varphi$ and $\psi$ have in common.
\end{remark}

\subsubsection{Examples of Interpolation}

\begin{example}
Let $\varphi = P \wedge Q$ and $\psi = P \vee R$.

We have $\varphi \models \psi$ since $(P \wedge Q) \models P$ and $P \models (P \vee R)$.

An interpolant is $\theta = P$:
\begin{itemize}
  \item $(P \wedge Q) \models P$ \checkmark
  \item $P \models (P \vee R)$ \checkmark
  \item $P$ contains only the variable $P$, which appears in both $\varphi$ and $\psi$ \checkmark
\end{itemize}
\end{example}

\begin{example}
Let $\varphi = (P \wedge Q) \vee (P \wedge R)$ and $\psi = P \vee S$.

We have $\varphi \models \psi$ since $\varphi \models P$ and $P \models (P \vee S)$.

An interpolant is $\theta = P$:
\begin{itemize}
  \item $((P \wedge Q) \vee (P \wedge R)) \models P$ \checkmark
  \item $P \models (P \vee S)$ \checkmark
  \item $P$ appears in both $\varphi$ and $\psi$ \checkmark
\end{itemize}
\end{example}

\begin{example}
Let $\varphi = P \rightarrow Q$ and $\psi = \neg Q \rightarrow \neg P$.

These are logically equivalent (contraposition), so $\varphi \models \psi$.

The common variables are $P$ and $Q$. Possible interpolants include:
\begin{itemize}
  \item $\theta = P \rightarrow Q$ (which equals $\varphi$)
  \item $\theta = \neg P \vee Q$
\end{itemize}
\end{example}

\subsubsection{Significance of Interpolation}

\begin{remark}[Logical Significance]
Craig's interpolation theorem has several important consequences:

\begin{enumerate}
  \item \textbf{Modularity of reasoning:} If $\varphi \models \psi$, the reason
        can be expressed using only the common vocabulary.

  \item \textbf{Definability:} Interpolation is closely related to Beth's
        definability theorem, which concerns when implicit definitions can be
        made explicit.

  \item \textbf{Proof theory:} Interpolation can be proved constructively using
        proof systems like sequent calculus, giving an algorithm to find
        interpolants.

  \item \textbf{Computer science applications:} Interpolation is used in model
        checking, program verification, and automated reasoning.
\end{enumerate}
\end{remark}

\begin{remark}[Non-trivial Cases]
When $\varphi$ is a tautology, we can take $\theta = \top$.
When $\psi$ is a tautology, we can take $\theta = \top$.
When $\varphi$ is a contradiction, we can take $\theta = \bot$.

The theorem is most interesting when $\varphi$ and $\psi$ are contingent formulas
with genuine logical content.
\end{remark}

% ---------------------------------------------------------
