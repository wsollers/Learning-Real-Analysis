% =========================================================
% Resolution
% =========================================================

\subsection{Resolution}
% ---------------------------------------------------------

\subsubsection{The Resolution Rule}

\begin{definition}[Resolution]
\textbf{Resolution} is an inference rule that operates on clauses (disjunctions
of literals). Given two clauses, one containing a literal $L$ and the other
containing its complement $\neg L$, resolution produces a new clause called the
\emph{resolvent}.

\[
\begin{array}{l}
C_1 \vee L \\
C_2 \vee \neg L \\ \hline
C_1 \vee C_2
\end{array}
\]

where $C_1$ and $C_2$ are (possibly empty) disjunctions of literals, and $L$ is
a literal.
\end{definition}

\begin{definition}[Complementary Literals]
Two literals are \emph{complementary} if one is the negation of the other. That
is, $L$ and $\neg L$ are complementary.

In resolution, we \emph{resolve} on a pair of complementary literals, eliminating
them from the resolvent.
\end{definition}

\begin{example}
Resolving the clauses $(P \vee Q)$ and $(\neg P \vee R)$ on $P$:
\[
\begin{array}{l}
P \vee Q \\
\neg P \vee R \\ \hline
Q \vee R
\end{array}
\]
\end{example}

\begin{example}
Resolving $(P \vee Q \vee R)$ and $(\neg Q \vee S)$ on $Q$:
\[
\begin{array}{l}
P \vee Q \vee R \\
\neg Q \vee S \\ \hline
P \vee R \vee S
\end{array}
\]
\end{example}

\subsubsection{The Empty Clause}

\begin{definition}[Empty Clause]
The \emph{empty clause}, denoted $\square$ or $\bot$, is the disjunction of no
literals. The empty clause is always false (it represents a contradiction).
\end{definition}

\begin{example}
Resolving the unit clauses $P$ and $\neg P$:
\[
\begin{array}{l}
P \\
\neg P \\ \hline
\square
\end{array}
\]
\end{example}

\subsubsection{Resolution Refutation}

\begin{definition}[Resolution Refutation]
A \emph{resolution refutation} (or \emph{resolution proof}) of a set of clauses
$\Gamma$ is a derivation of the empty clause $\square$ from $\Gamma$ using only
the resolution rule.

If a resolution refutation exists, then $\Gamma$ is unsatisfiable.
\end{definition}

\begin{theorem}[Soundness of Resolution]
If the empty clause can be derived from a set of clauses $\Gamma$ by resolution,
then $\Gamma$ is unsatisfiable.
\end{theorem}

\begin{theorem}[Completeness of Resolution]
If a set of clauses $\Gamma$ is unsatisfiable, then the empty clause can be
derived from $\Gamma$ by resolution.
\end{theorem}

\begin{remark}
Resolution is a \emph{refutation-complete} proof system: it can derive the empty
clause from any unsatisfiable set of clauses. However, it is not designed to
derive arbitrary consequences, only to detect unsatisfiability.
\end{remark}

\subsubsection{Using Resolution to Test Validity}

\begin{remark}[Testing Validity by Refutation]
To test whether $\Gamma \models \varphi$:
\begin{enumerate}
  \item Negate the conclusion: form $\neg\varphi$.
  \item Convert all formulas in $\Gamma \cup \{\neg\varphi\}$ to CNF.
  \item Apply resolution repeatedly.
  \item If the empty clause is derived, then $\Gamma \models \varphi$.
  \item If no new clauses can be derived and $\square$ is not among them, then
        $\Gamma \not\models \varphi$.
\end{enumerate}
\end{remark}

\begin{example}[Resolution Refutation]
Show that $\{P \rightarrow Q, Q \rightarrow R\} \models P \rightarrow R$.

\textbf{Step 1: Negate the conclusion.}
\[
\neg(P \rightarrow R) \equiv P \wedge \neg R
\]

\textbf{Step 2: Convert to CNF.}
\begin{align*}
P \rightarrow Q &\equiv \neg P \vee Q \\
Q \rightarrow R &\equiv \neg Q \vee R \\
P \wedge \neg R &\text{ gives clauses } P \text{ and } \neg R
\end{align*}

Clauses: $\{\neg P \vee Q, \; \neg Q \vee R, \; P, \; \neg R\}$

\textbf{Step 3: Apply resolution.}
\begin{enumerate}
  \item Resolve $(\neg P \vee Q)$ and $P$ on $P$: derive $Q$.
  \item Resolve $(\neg Q \vee R)$ and $Q$ on $Q$: derive $R$.
  \item Resolve $R$ and $\neg R$ on $R$: derive $\square$.
\end{enumerate}

Since $\square$ is derived, the original set is unsatisfiable, so
$\{P \rightarrow Q, Q \rightarrow R\} \models P \rightarrow R$.
\end{example}

\subsubsection{Resolution Strategies}

\begin{remark}[Resolution Strategies]
Various strategies can make resolution more efficient:

\begin{itemize}
  \item \textbf{Unit resolution:} Always resolve with a unit clause (a clause
        with exactly one literal) when possible.

  \item \textbf{Set of support:} Maintain a ``set of support'' (initially the
        negated conclusion) and require that at least one parent clause in each
        resolution step comes from this set.

  \item \textbf{Linear resolution:} Each resolution step uses the most recently
        derived clause as one of its parents.

  \item \textbf{Input resolution:} At least one parent in each resolution must
        be an original input clause.
\end{itemize}
\end{remark}

% ---------------------------------------------------------
