% =========================================================
% Proof Systems
% =========================================================

\subsection{Proof Systems}

% ---------------------------------------------------------
% TOOLKIT
% ---------------------------------------------------------
\begin{tcolorbox}[colback=gray!6, colframe=gray!40, arc=2pt,
  left=6pt, right=6pt, top=4pt, bottom=4pt,
  title={\small\textbf{Proof Systems — Quick Reference}},
  fonttitle=\small\bfseries]
\small
\begin{tabular}{l l l}
\toprule
\textbf{Concept} & \textbf{Meaning} & \textbf{Detail} \\
\midrule
Derivability $\Gamma\vdash\varphi$ & $\varphi$ provable from $\Gamma$ syntactically & \hyperref[def:derivability]{↓ Def} \\
Soundness    & $\Gamma\vdash\varphi \Rightarrow \Gamma\models\varphi$ & \hyperref[def:soundness]{↓ Def} \\
Completeness & $\Gamma\models\varphi \Rightarrow \Gamma\vdash\varphi$ & \hyperref[def:completeness]{↓ Def} \\
$\vdash\leftrightarrow\models$ & For PL: derivable iff semantically valid & \hyperref[thm:sound-complete]{↓ Thm} \\
\midrule
Semantic ($\models$) & Truth-assignment based; model-theoretic & \\
Syntactic ($\vdash$) & Rule-application based; proof-theoretic & \\
\bottomrule
\end{tabular}
\end{tcolorbox}

\vspace{1em}

% ---------------------------------------------------------
% Derivability
% ---------------------------------------------------------
\begin{tcolorbox}[colback=propbox, colframe=propborder, arc=2pt,
  left=6pt, right=6pt, top=4pt, bottom=4pt,
  title={\small\textbf{Definition (Derivability)}},
  fonttitle=\small\bfseries]
\label{def:derivability}
$\varphi$ is \emph{derivable} from $\Gamma$ in a proof system $\mathcal{P}$,
written $\Gamma \vdash \varphi$, if there exists a finite sequence of formulas
ending in $\varphi$, each of which is either:
\begin{itemize}
  \item an element of $\Gamma$ (an assumption), or
  \item obtained from earlier formulas by an inference rule of $\mathcal{P}$.
\end{itemize}
\end{tcolorbox}

\begin{remark}[English reading]
Derivability is a purely syntactic notion: it depends on the proof system's rules
and makes no reference to truth or models. A derivation is a finite certificate
that can be checked mechanically, step by step.
\end{remark}

\begin{remark}[Proof systems for propositional logic]
Common proof systems include:
\begin{itemize}
  \item \textbf{Natural deduction} — introduction and elimination rules for each
        connective; subproofs; discharge of assumptions. Closest to how
        mathematicians actually reason.
  \item \textbf{Sequent calculus} — operates on \emph{sequents} $\Gamma\vdash\Delta$;
        left and right rules; useful for proof theory.
  \item \textbf{Hilbert systems} — many axiom schemas, few rules (typically MP
        only). Compact to define, tedious to use.
  \item \textbf{Resolution} — single rule; requires CNF input; optimized for
        automated theorem proving.
\end{itemize}
All systems for classical propositional logic prove the same theorems.
\end{remark}

% ---------------------------------------------------------
% Soundness and Completeness
% ---------------------------------------------------------
\begin{tcolorbox}[colback=propbox, colframe=propborder, arc=2pt,
  left=6pt, right=6pt, top=4pt, bottom=4pt,
  title={\small\textbf{Definitions (Soundness and Completeness)}},
  fonttitle=\small\bfseries]
\label{def:soundness}
A proof system is \emph{sound} if every derivable formula is semantically valid:
\[
\Gamma \vdash \varphi \;\Rightarrow\; \Gamma \models \varphi.
\]

\label{def:completeness}
A proof system is \emph{complete} if every semantically valid formula is
derivable:
\[
\Gamma \models \varphi \;\Rightarrow\; \Gamma \vdash \varphi.
\]
\end{tcolorbox}

\begin{remark}[Why soundness matters]
Soundness prevents the proof system from proving false things. Without soundness,
a formal proof is worthless — it might ``prove'' $P \wedge \neg P$.
\end{remark}

\begin{remark}[Why completeness matters]
Completeness ensures the proof system is powerful enough: every logical truth
has a proof. Without completeness, the system might miss valid arguments.
\end{remark}

\begin{tcolorbox}[colback=thmbox, colframe=thmborder, arc=2pt,
  left=6pt, right=6pt, top=4pt, bottom=4pt,
  title={\small\textbf{Theorem (Soundness and Completeness of Propositional Logic)}},
  fonttitle=\small\bfseries]
\label{thm:sound-complete}
For any standard proof system for classical propositional logic (natural
deduction, Hilbert system, sequent calculus, resolution):
\[
\Gamma \vdash \varphi \;\Longleftrightarrow\; \Gamma \models \varphi.
\]
\end{tcolorbox}

\begin{remark}[Logical significance]
This equivalence is the central metatheorem of propositional logic. It means
the syntactic and semantic notions of entailment coincide perfectly.
A formula is provable if and only if it is true under all models.
\end{remark}

\begin{remark}[Contrast with first-order logic]
In first-order logic, completeness still holds (Gödel's completeness theorem,
1930). But for arithmetic (Peano arithmetic), Gödel's \emph{incompleteness}
theorems show that some true statements are unprovable. Propositional logic
avoids incompleteness because it is decidable — truth tables provide a
decision procedure.
\end{remark}

\begin{center}
\small
\renewcommand{\arraystretch}{1.3}
\begin{tabular}{lll}
\toprule
\textbf{Notion} & \textbf{Symbol} & \textbf{Nature} \\
\midrule
Tautological implication & $P \models_{\mathrm{taut}} Q$ & Semantic \\
Logical consequence & $\Gamma \models \varphi$ & Semantic, model-theoretic \\
Derivability & $\Gamma \vdash \varphi$ & Syntactic, proof-theoretic \\
\bottomrule
\end{tabular}
\end{center}
