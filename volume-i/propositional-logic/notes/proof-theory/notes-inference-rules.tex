% =========================================================
% Inference Rules
% =========================================================

\subsection{Inference Rules}

% ---------------------------------------------------------
% TOOLKIT
% ---------------------------------------------------------
\begin{tcolorbox}[colback=gray!6, colframe=gray!40, arc=2pt,
  left=6pt, right=6pt, top=4pt, bottom=4pt,
  title={\small\textbf{Inference Rules — Quick Reference}},
  fonttitle=\small\bfseries]
\small
\begin{tabular}{l l l l}
\toprule
\textbf{Rule} & \textbf{Premises} & \textbf{Conclusion} & \textbf{Detail} \\
\midrule
Modus Ponens (MP)         & $P\to Q,\; P$              & $Q$                & \hyperref[def:mp]{↓} \\
Modus Tollens (MT)        & $P\to Q,\; \neg Q$         & $\neg P$           & \hyperref[def:mt]{↓} \\
Hypothetical Syllogism    & $P\to Q,\; Q\to R$         & $P\to R$           & \hyperref[def:hs]{↓} \\
Disjunctive Syllogism     & $P\vee Q,\; \neg P$        & $Q$                & \hyperref[def:ds]{↓} \\
Conj.\ Introduction       & $P,\; Q$                   & $P\wedge Q$        & \hyperref[def:conj-i]{↓} \\
Conj.\ Elimination        & $P\wedge Q$                & $P$ (or $Q$)       & \hyperref[def:conj-e]{↓} \\
Disj.\ Introduction       & $P$                        & $P\vee Q$          & \hyperref[def:disj-i]{↓} \\
Disj.\ Elimination        & $P\vee Q,\;[P\vdash R],\;[Q\vdash R]$ & $R$  & \hyperref[def:disj-e]{↓} \\
Cond.\ Introduction (CP)  & $[P \vdash Q]$             & $P\to Q$           & \hyperref[def:cond-i]{↓} \\
Bicond.\ Introduction     & $P\to Q,\; Q\to P$         & $P\leftrightarrow Q$ & \hyperref[def:bicond-i]{↓} \\
Bicond.\ Elimination      & $P\leftrightarrow Q$       & $P\to Q$ (or $Q\to P$) & \hyperref[def:bicond-e]{↓} \\
Neg.\ Introduction (RAA)  & $[P\vdash\bot]$            & $\neg P$           & \hyperref[def:neg-i]{↓} \\
Neg.\ Elimination         & $[\neg P\vdash\bot]$       & $P$                & \hyperref[def:neg-e]{↓} \\
Double Neg.\ Elim.        & $\neg\neg P$               & $P$                & \hyperref[def:dne]{↓} \\
Constructive Dilemma      & $P\vee Q,\;P\to R,\;Q\to S$ & $R\vee S$        & \hyperref[def:cd]{↓} \\
Destructive Dilemma       & $P\to R,\;Q\to S,\;\neg R\vee\neg S$ & $\neg P\vee\neg Q$ & \hyperref[def:dd]{↓} \\
\bottomrule
\end{tabular}
\end{tcolorbox}

\vspace{1em}

% ---------------------------------------------------------
% Details
% ---------------------------------------------------------

\subsubsection*{Conditional Rules}

\begin{definition}[Modus Ponens]\label{def:mp}
From a conditional and its antecedent, infer the consequent:
\[
\frac{P \rightarrow Q \qquad P}{Q}
\]
\end{definition}

\begin{remark}[Intuition] ``If it rains the ground is wet. It is raining.
Therefore, the ground is wet.'' The most fundamental inference rule. All
proof systems for classical logic include MP or an equivalent.
\end{remark}
\begin{remark}[Fully quantified]
$\forall P, Q:\; [(P \rightarrow Q) \wedge P] \models Q$.
\end{remark}
\begin{remark}[Common error — Affirming the Consequent]
From $P \rightarrow Q$ and $Q$, one \emph{cannot} infer $P$. This is the
fallacy of affirming the consequent.
\end{remark}

\begin{definition}[Modus Tollens]\label{def:mt}
From a conditional and the negation of its consequent, infer the negation of
the antecedent:
\[
\frac{P \rightarrow Q \qquad \neg Q}{\neg P}
\]
\end{definition}

\begin{remark}[Intuition] ``If it rains the ground is wet. The ground is not wet.
Therefore, it did not rain.'' MT is logically equivalent to applying MP to
the contrapositive $\neg Q \rightarrow \neg P$.
\end{remark}
\begin{remark}[Common error — Denying the Antecedent]
From $P \rightarrow Q$ and $\neg P$, one \emph{cannot} infer $\neg Q$.
\end{remark}

\begin{definition}[Hypothetical Syllogism]\label{def:hs}
From two conditionals chained, infer the composed conditional:
\[
\frac{P \rightarrow Q \qquad Q \rightarrow R}{P \rightarrow R}
\]
\end{definition}

\begin{remark}[Intuition] Transitivity of implication. Chains of reasoning can
be composed into a single step.
\end{remark}
\begin{remark}[Consequence] Hypothetical syllogism makes $\rightarrow$ transitive,
allowing arbitrarily long inference chains to be compressed.
\end{remark}

\begin{definition}[Conditional Introduction (Conditional Proof)]\label{def:cond-i}
If assuming $P$ in a subproof leads to $Q$, then discharge the assumption and
infer $P \rightarrow Q$:
\[
\frac{[P \vdash Q]}{P \rightarrow Q}
\]
\end{definition}

\begin{remark}[Intuition] The standard method for proving a conditional: assume
the antecedent, derive the consequent, discharge. This corresponds to
exportation: a proof from hypothesis $P$ of $Q$ is exactly a proof of $P \to Q$.
\end{remark}

\subsubsection*{Conjunction Rules}

\begin{definition}[Conjunction Introduction]\label{def:conj-i}
From two formulas, infer their conjunction:
\[
\frac{P \qquad Q}{P \wedge Q}
\]
\end{definition}

\begin{definition}[Conjunction Elimination]\label{def:conj-e}
From a conjunction, infer either conjunct:
\[
\frac{P \wedge Q}{P} \qquad \frac{P \wedge Q}{Q}
\]
\end{definition}

\begin{remark}[Intuition] Introduction: combining two independent truths.
Elimination: extracting a component from a joint assertion.
\end{remark}

\subsubsection*{Disjunction Rules}

\begin{definition}[Disjunction Introduction]\label{def:disj-i}
From a proposition, infer any disjunction containing it:
\[
\frac{P}{P \vee Q}
\]
\end{definition}

\begin{remark}[Intuition] ``It's raining, so it's raining or snowing.'' Adding
a disjunct weakens the claim.
\end{remark}

\begin{definition}[Disjunction Elimination (Proof by Cases)]\label{def:disj-e}
From a disjunction, if the same conclusion $R$ follows from each disjunct, infer $R$:
\[
\frac{P \vee Q \qquad [P \vdash R] \qquad [Q \vdash R]}{R}
\]
\end{definition}

\begin{remark}[Proof strategy] Case analysis: prove $R$ holds in case $P$;
prove $R$ holds in case $Q$; since one of $P, Q$ must hold, conclude $R$.
\end{remark}

\subsubsection*{Biconditional Rules}

\begin{definition}[Biconditional Introduction]\label{def:bicond-i}
\[
\frac{P \rightarrow Q \qquad Q \rightarrow P}{P \leftrightarrow Q}
\]
\end{definition}

\begin{definition}[Biconditional Elimination]\label{def:bicond-e}
\[
\frac{P \leftrightarrow Q}{P \rightarrow Q} \qquad \frac{P \leftrightarrow Q}{Q \rightarrow P}
\]
\end{definition}

\begin{remark}[Proof strategy] Every iff proof has two halves: prove $(\Rightarrow)$
and prove $(\Leftarrow)$.
\end{remark}

\subsubsection*{Negation Rules}

\begin{definition}[Negation Introduction (Reductio ad Absurdum)]\label{def:neg-i}
If assuming $P$ leads to contradiction $\bot$, infer $\neg P$:
\[
\frac{[P \vdash \bot]}{\neg P}
\]
\end{definition}

\begin{definition}[Negation Elimination (Indirect Proof)]\label{def:neg-e}
If assuming $\neg P$ leads to $\bot$, infer $P$:
\[
\frac{[\neg P \vdash \bot]}{P}
\]
\end{definition}

\begin{definition}[Double Negation Elimination]\label{def:dne}
\[
\frac{\neg\neg P}{P}
\]
\end{definition}

\begin{remark}[Classical vs.\ intuitionistic logic]
Negation elimination and double negation elimination are classical rules;
they fail in intuitionistic logic, where $\neg\neg P \Rightarrow P$ is not
provable without the law of excluded middle. All three rules are equivalent
to LEM in a suitable sense.
\end{remark}

\subsubsection*{Dilemma Rules}

\begin{definition}[Constructive Dilemma]\label{def:cd}
\[
\frac{P \vee Q \qquad P \rightarrow R \qquad Q \rightarrow S}{R \vee S}
\]
\end{definition}

\begin{definition}[Destructive Dilemma]\label{def:dd}
\[
\frac{P \rightarrow R \qquad Q \rightarrow S \qquad \neg R \vee \neg S}{\neg P \vee \neg Q}
\]
\end{definition}

\begin{remark}[Intuition] Constructive dilemma applies two implications to a
disjunction, deriving a disjunction of the two conclusions. Destructive dilemma
is its contrapositive form.
\end{remark}
