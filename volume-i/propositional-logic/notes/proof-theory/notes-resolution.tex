% =========================================================
% Resolution
% =========================================================

\subsection{Resolution}

% ---------------------------------------------------------
% TOOLKIT
% ---------------------------------------------------------
\begin{tcolorbox}[colback=gray!6, colframe=gray!40, arc=2pt,
  left=6pt, right=6pt, top=4pt, bottom=4pt,
  title={\small\textbf{Resolution — Quick Reference}},
  fonttitle=\small\bfseries]
\small
\begin{tabular}{l l l}
\toprule
\textbf{Concept} & \textbf{Key fact} & \textbf{Detail} \\
\midrule
Resolution rule      & From $C_1\vee L$ and $C_2\vee\neg L$, derive $C_1\vee C_2$ & \hyperref[def:resolution]{↓ Def} \\
Empty clause $\square$ & Always false; represents a contradiction & \hyperref[def:empty-clause]{↓ Def} \\
Resolution refutation & Derive $\square$ from $\Gamma$ — proves $\Gamma$ unsatisfiable & \hyperref[def:refutation]{↓ Def} \\
Soundness            & $\Gamma\vdash_{\mathrm{res}}\square \Rightarrow \Gamma$ unsat. & \hyperref[thm:res-sound]{↓ Thm} \\
Completeness         & $\Gamma$ unsat. $\Rightarrow \Gamma\vdash_{\mathrm{res}}\square$ & \hyperref[thm:res-complete]{↓ Thm} \\
Validity testing     & $\Gamma\models\varphi$ iff $\Gamma\cup\{\neg\varphi\}$ derives $\square$ & \hyperref[rem:res-validity]{↓} \\
\midrule
Strategies & Unit res., set of support, linear, input & \hyperref[rem:res-strat]{↓} \\
\bottomrule
\end{tabular}
\end{tcolorbox}

\vspace{1em}

% ---------------------------------------------------------
% Resolution Rule
% ---------------------------------------------------------
\begin{tcolorbox}[colback=propbox, colframe=propborder, arc=2pt,
  left=6pt, right=6pt, top=4pt, bottom=4pt,
  title={\small\textbf{Definition (Resolution Rule)}},
  fonttitle=\small\bfseries]
\label{def:resolution}
Given two clauses containing complementary literals $L$ and $\neg L$:
\[
\frac{C_1 \vee L \qquad C_2 \vee \neg L}{C_1 \vee C_2}
\]
The derived clause $C_1 \vee C_2$ is the \emph{resolvent}. The literal $L$
is \emph{resolved upon} (eliminated). $C_1$, $C_2$ are (possibly empty)
disjunctions of literals.
\end{tcolorbox}

\begin{remark}[English reading]
Resolution combines two clauses that disagree on exactly one variable,
producing a clause that contains everything both clauses said, minus the
disagreement. It is the only inference rule in the resolution proof system.
\end{remark}

\begin{remark}[Fully quantified]
For all clauses $C_1, C_2$ and literal $L$:
$(C_1 \vee L) \wedge (C_2 \vee \neg L) \models (C_1 \vee C_2)$.
Resolution is sound: the resolvent is a semantic consequence of the parent clauses.
\end{remark}

\begin{remark}[Requires CNF]
Resolution applies only to clauses (disjunctions of literals). The input set
$\Gamma$ must be in CNF. Converting to CNF is always possible (by the CNF
existence theorem).
\end{remark}

\begin{example}
Resolving $(P \vee Q)$ and $(\neg P \vee R)$ on $P$:
\[
\frac{P \vee Q \qquad \neg P \vee R}{Q \vee R}
\]
Resolving $(P)$ and $(\neg P)$: derive the empty clause $\square$.
\end{example}

% ---------------------------------------------------------
% Empty Clause
% ---------------------------------------------------------
\begin{tcolorbox}[colback=propbox, colframe=propborder, arc=2pt,
  left=6pt, right=6pt, top=4pt, bottom=4pt,
  title={\small\textbf{Definition (Empty Clause)}},
  fonttitle=\small\bfseries]
\label{def:empty-clause}
The \emph{empty clause}, denoted $\square$ (or $\bot$), is the disjunction of
no literals. It is always false (unsatisfiable by any truth assignment).
\end{tcolorbox}

\begin{remark}[Intuition]
Deriving $\square$ is the resolution analogue of deriving a contradiction.
Since $\square$ is false under every assignment, its derivation from $\Gamma$
proves $\Gamma$ cannot be satisfied.
\end{remark}

% ---------------------------------------------------------
% Resolution Refutation
% ---------------------------------------------------------
\begin{tcolorbox}[colback=propbox, colframe=propborder, arc=2pt,
  left=6pt, right=6pt, top=4pt, bottom=4pt,
  title={\small\textbf{Definition (Resolution Refutation)}},
  fonttitle=\small\bfseries]
\label{def:refutation}
A \emph{resolution refutation} of a set of clauses $\Gamma$ is a finite
sequence of resolution steps deriving $\square$ from $\Gamma$.

If a resolution refutation exists, then $\Gamma$ is unsatisfiable.
\end{tcolorbox}

% ---------------------------------------------------------
% Soundness and Completeness
% ---------------------------------------------------------
\begin{tcolorbox}[colback=thmbox, colframe=thmborder, arc=2pt,
  left=6pt, right=6pt, top=4pt, bottom=4pt,
  title={\small\textbf{Theorem (Soundness and Completeness of Resolution)}},
  fonttitle=\small\bfseries]
\label{thm:res-sound}\label{thm:res-complete}
A set of clauses $\Gamma$ is unsatisfiable if and only if the empty clause
$\square$ can be derived from $\Gamma$ by resolution:
\[
\Gamma \text{ unsatisfiable} \;\Longleftrightarrow\; \Gamma \vdash_{\mathrm{res}} \square
\]
\end{tcolorbox}

\begin{remark}[Soundness direction ($\Rightarrow$)]
If $\square$ is derived, then $\Gamma$ is unsatisfiable. This follows because
each resolution step is semantically valid (preserves consequence), and
$\square$ is always false.
\end{remark}

\begin{remark}[Completeness direction ($\Leftarrow$)]
If $\Gamma$ is unsatisfiable, resolution will eventually produce $\square$.
Resolution is \emph{refutation-complete}: it can detect every unsatisfiable
set, but it is not designed to enumerate all consequences.
\end{remark}

\begin{remark}[Logical implication]\label{rem:res-validity}
To test $\Gamma \models \varphi$:
\begin{enumerate}
  \item Negate: $\neg\varphi$.
  \item Convert $\Gamma \cup \{\neg\varphi\}$ to CNF.
  \item Run resolution. If $\square$ is derived, then $\Gamma \models \varphi$.
\end{enumerate}
Validity reduces to unsatisfiability, which resolution decides.
\end{remark}

\begin{remark}[Resolution strategies]\label{rem:res-strat}
Naive resolution may generate many irrelevant clauses. Strategies restrict
which pairs are resolved:
\begin{itemize}
  \item \textbf{Unit resolution:} Always resolve with a unit clause (single literal).
  \item \textbf{Set of support:} At least one parent must come from the
        negated goal (set of support). Keeps the search goal-directed.
  \item \textbf{Linear resolution:} Each step uses the most recently derived
        clause as one parent.
  \item \textbf{Input resolution:} At least one parent is always an original
        input clause.
\end{itemize}
All strategies preserve completeness for refutation.
\end{remark}
