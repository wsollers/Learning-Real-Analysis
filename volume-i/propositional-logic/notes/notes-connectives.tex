% =========================================================
% Additional Connectives
% =========================================================

\subsection{Additional Connectives}
% ---------------------------------------------------------

\subsubsection{Exclusive Or (XOR)}

\begin{definition}[Exclusive Or]
The \emph{exclusive or} (or \emph{exclusive disjunction}), denoted $\oplus$ or
$\veebar$, is a binary connective that is true when exactly one of its operands
is true.

\begin{center}
\renewcommand{\arraystretch}{1.2}
\begin{tabular}{|c|c||c|}
\hline
$P$ & $Q$ & $P \oplus Q$ \\
\hline
T & T & F \\
T & F & T \\
F & T & T \\
F & F & F \\
\hline
\end{tabular}
\end{center}
\end{definition}

\begin{theorem}[XOR Equivalences]
The exclusive or can be expressed using standard connectives:
\begin{align*}
P \oplus Q &\;\equiv\; (P \vee Q) \wedge \neg(P \wedge Q) \\
P \oplus Q &\;\equiv\; (P \wedge \neg Q) \vee (\neg P \wedge Q) \\
P \oplus Q &\;\equiv\; \neg(P \leftrightarrow Q)
\end{align*}
\end{theorem}

\begin{remark}
The exclusive or differs from the inclusive or ($\vee$) only when both operands
are true: $P \vee Q$ is true in this case, but $P \oplus Q$ is false.

In natural language, ``or'' is often ambiguous between inclusive and exclusive
readings. ``Would you like coffee or tea?'' typically means exclusive or, while
``Students who have taken calculus or linear algebra may enroll'' typically
means inclusive or.
\end{remark}

\begin{theorem}[Properties of XOR]
\begin{align*}
P \oplus Q &\;\equiv\; Q \oplus P && \text{(Commutativity)} \\
(P \oplus Q) \oplus R &\;\equiv\; P \oplus (Q \oplus R) && \text{(Associativity)} \\
P \oplus \bot &\;\equiv\; P && \text{(Identity)} \\
P \oplus \top &\;\equiv\; \neg P && \text{(Negation)} \\
P \oplus P &\;\equiv\; \bot && \text{(Self-inverse)} \\
P \oplus \neg P &\;\equiv\; \top && \\
P \wedge (Q \oplus R) &\;\equiv\; (P \wedge Q) \oplus (P \wedge R) && \text{(Distributivity of $\wedge$ over $\oplus$)}
\end{align*}
\end{theorem}

\subsubsection{NAND (Sheffer Stroke)}

\begin{definition}[NAND]
The \emph{NAND} connective (also called the \emph{Sheffer stroke}), denoted
$\uparrow$ or $|$, is the negation of conjunction. It is false only when both
operands are true.

\begin{center}
\renewcommand{\arraystretch}{1.2}
\begin{tabular}{|c|c||c|}
\hline
$P$ & $Q$ & $P \uparrow Q$ \\
\hline
T & T & F \\
T & F & T \\
F & T & T \\
F & F & T \\
\hline
\end{tabular}
\end{center}
\end{definition}

\begin{theorem}[NAND Definition]
\[
P \uparrow Q \;\equiv\; \neg(P \wedge Q)
\]
\end{theorem}

\begin{theorem}[Expressing Connectives with NAND]
All standard connectives can be expressed using only NAND:
\begin{align*}
\neg P &\;\equiv\; P \uparrow P \\
P \wedge Q &\;\equiv\; (P \uparrow Q) \uparrow (P \uparrow Q) \\
P \vee Q &\;\equiv\; (P \uparrow P) \uparrow (Q \uparrow Q) \\
P \rightarrow Q &\;\equiv\; P \uparrow (Q \uparrow Q)
\end{align*}
\end{theorem}

\begin{remark}
The NAND connective is not associative:
\[
(P \uparrow Q) \uparrow R \;\not\equiv\; P \uparrow (Q \uparrow R)
\]
\end{remark}

\subsubsection{NOR (Peirce Arrow)}

\begin{definition}[NOR]
The \emph{NOR} connective (also called the \emph{Peirce arrow} or \emph{Quine
dagger}), denoted $\downarrow$, is the negation of disjunction. It is true only
when both operands are false.

\begin{center}
\renewcommand{\arraystretch}{1.2}
\begin{tabular}{|c|c||c|}
\hline
$P$ & $Q$ & $P \downarrow Q$ \\
\hline
T & T & F \\
T & F & F \\
F & T & F \\
F & F & T \\
\hline
\end{tabular}
\end{center}
\end{definition}

\begin{theorem}[NOR Definition]
\[
P \downarrow Q \;\equiv\; \neg(P \vee Q)
\]
\end{theorem}

\begin{theorem}[Expressing Connectives with NOR]
All standard connectives can be expressed using only NOR:
\begin{align*}
\neg P &\;\equiv\; P \downarrow P \\
P \vee Q &\;\equiv\; (P \downarrow Q) \downarrow (P \downarrow Q) \\
P \wedge Q &\;\equiv\; (P \downarrow P) \downarrow (Q \downarrow Q) \\
P \rightarrow Q &\;\equiv\; ((P \downarrow P) \downarrow Q) \downarrow ((P \downarrow P) \downarrow Q)
\end{align*}
\end{theorem}

\begin{remark}
Like NAND, NOR is not associative:
\[
(P \downarrow Q) \downarrow R \;\not\equiv\; P \downarrow (Q \downarrow R)
\]
\end{remark}

\subsubsection{Summary of Additional Connectives}

\begin{center}
\renewcommand{\arraystretch}{1.2}
\begin{tabular}{|c|c||c|c|c|}
\hline
$P$ & $Q$ & $P \oplus Q$ & $P \uparrow Q$ & $P \downarrow Q$ \\
\hline
T & T & F & F & F \\
T & F & T & T & F \\
F & T & T & T & F \\
F & F & F & T & T \\
\hline
\end{tabular}
\end{center}

\begin{center}
\renewcommand{\arraystretch}{1.3}
\begin{tabular}{|c|c|c|c|}
\hline
\textbf{Connective} & \textbf{Symbol} & \textbf{Name} & \textbf{Equivalent} \\
\hline
Exclusive Or & $\oplus$, $\veebar$ & XOR & $\neg(P \leftrightarrow Q)$ \\
NAND & $\uparrow$, $|$ & Sheffer stroke & $\neg(P \wedge Q)$ \\
NOR & $\downarrow$ & Peirce arrow & $\neg(P \vee Q)$ \\
\hline
\end{tabular}
\end{center}

% ---------------------------------------------------------
