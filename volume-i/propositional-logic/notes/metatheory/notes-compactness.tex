% =========================================================
% Compactness Theorem
% =========================================================

\subsection{Compactness Theorem}

% ---------------------------------------------------------
% TOOLKIT
% ---------------------------------------------------------
\begin{tcolorbox}[colback=gray!6, colframe=gray!40, arc=2pt,
  left=6pt, right=6pt, top=4pt, bottom=4pt,
  title={\small\textbf{Compactness — Quick Reference}},
  fonttitle=\small\bfseries]
\small
\begin{tabular}{l l l}
\toprule
\textbf{Concept} & \textbf{Key fact} & \textbf{Detail} \\
\midrule
Compactness theorem & $\Gamma$ sat. $\Longleftrightarrow$ every finite $\Gamma_0\subseteq\Gamma$ sat. & \hyperref[thm:compactness]{↓ Thm} \\
Consequence form    & $\Gamma\models\varphi \Rightarrow$ some finite $\Gamma_0\models\varphi$ & \hyperref[cor:consequence-compact]{↓ Cor} \\
Graph coloring app. & Infinite graph $k$-colorable iff all finite subgraphs are & \hyperref[ex:graph-color]{↓ Ex} \\
Proof methods       & Via completeness; direct construction; ultraproducts & \hyperref[rem:compact-proof]{↓} \\
\bottomrule
\end{tabular}
\end{tcolorbox}

\vspace{1em}

% ---------------------------------------------------------
% Compactness Theorem
% ---------------------------------------------------------
\begin{tcolorbox}[colback=thmbox, colframe=thmborder, arc=2pt,
  left=6pt, right=6pt, top=4pt, bottom=4pt,
  title={\small\textbf{Theorem (Compactness of Propositional Logic)}},
  fonttitle=\small\bfseries]
\label{thm:compactness}
A set of formulas $\Gamma$ is satisfiable if and only if every finite subset
of $\Gamma$ is satisfiable.

Equivalently: if $\Gamma$ is unsatisfiable, some \emph{finite} subset of
$\Gamma$ is already unsatisfiable.
\end{tcolorbox}

\begin{remark}[English reading]
Compactness says that satisfiability is a \emph{finitary} property: an infinite
set of premises can only be contradictory if some finite fragment is already
contradictory. You can never need infinitely many premises to derive a
contradiction; finitely many always suffice.
\end{remark}

\begin{remark}[Fully quantified form]
$\forall \Gamma \subseteq \mathsf{WFF}:\;
\bigl(\exists v,\; v \models \Gamma\bigr)
\;\Longleftrightarrow\;
\bigl(\forall \Gamma_0 \subseteq_{\mathrm{fin}} \Gamma,\;
\exists v,\; v \models \Gamma_0\bigr)$.
\end{remark}

\begin{remark}[Which direction is trivial]
($\Rightarrow$): If $v$ satisfies $\Gamma$, it satisfies every subset. Trivial.

($\Leftarrow$): The substantive direction. If every finite subset is satisfiable,
is the whole set? Intuition says ``yes'' but it is not obvious — an infinite set
might encode constraints that collectively rule out every assignment even though
no finite fragment does.
\end{remark}

\begin{remark}[Why this is a \emph{meta}theorem]
Compactness is a statement about the entire logic, not about any particular
formula. It belongs to metatheory — it tells us something about the structure
of propositional logic itself.
\end{remark}

\begin{remark}[Proof methods]\label{rem:compact-proof}
Three standard approaches:
\begin{enumerate}
  \item \textbf{Via completeness:} If every finite subset of $\Gamma$ is
        satisfiable, no finite subset derives $\bot$. By completeness, $\Gamma$
        does not derive $\bot$. By soundness-completeness, $\Gamma$ is consistent,
        hence satisfiable. (Circular-looking but rigorous with careful bookkeeping.)
  \item \textbf{Direct construction (König's lemma):} Build a satisfying
        assignment by extending partial assignments level by level through all
        formulas in $\Gamma$, using compactness of $\{T,F\}^\omega$ (Tychonoff).
  \item \textbf{Ultraproducts:} Take an ultraproduct of the satisfying assignments
        for all finite subsets.
\end{enumerate}
\end{remark}

% ---------------------------------------------------------
% Consequence form
% ---------------------------------------------------------
\begin{corollary}[Compactness of Logical Consequence]\label{cor:consequence-compact}
If $\Gamma \models \varphi$, then there exists a \emph{finite} subset
$\Gamma_0 \subseteq \Gamma$ with $\Gamma_0 \models \varphi$.

Equivalently: logical consequence from an infinite premise set always depends
on only finitely many premises.
\end{corollary}



\begin{remark}[Practical implication]
In any propositional proof, one uses only finitely many formulas from $\Gamma$.
Compactness tells us this is not a limitation — we never \emph{need} infinitely
many premises for a single conclusion.
\end{remark}

% ---------------------------------------------------------
% Applications
% ---------------------------------------------------------
\begin{example}[Graph Coloring]\label{ex:graph-color}
\textbf{Claim:} An infinite graph $G$ is $k$-colorable if and only if every
finite subgraph of $G$ is $k$-colorable.

\textbf{Encoding:} For each vertex $v$ and color $i \in \{1,\ldots,k\}$,
introduce propositional variable $C_{v,i}$ (``vertex $v$ has color $i$'').
Add formulas:
\begin{itemize}
  \item For each $v$: $C_{v,1} \vee \cdots \vee C_{v,k}$ (every vertex has some color).
  \item For each $v$, distinct $i,j$: $\neg(C_{v,i} \wedge C_{v,j})$ (at most one color).
  \item For each edge $(u,v)$, each $i$: $\neg(C_{u,i} \wedge C_{v,i})$ (adjacent vertices differ).
\end{itemize}
Each finite subset of this formula set involves only finitely many vertices —
a finite subgraph. If every finite subgraph is $k$-colorable, every finite
subset is satisfiable. By compactness, the full set is satisfiable, giving a
$k$-coloring of $G$.
\end{example}

\begin{remark}[Why compactness is surprising here]
The graph coloring argument works even for uncountably infinite graphs.
Compactness is doing real work: it converts local, finite satisfiability into
global satisfiability for a system with infinitely many variables.
\end{remark}
