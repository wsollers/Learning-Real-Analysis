% =========================================================
% Functional Completeness
% =========================================================

\subsection{Functional Completeness}

% ---------------------------------------------------------
% TOOLKIT
% ---------------------------------------------------------
\begin{tcolorbox}[colback=gray!6, colframe=gray!40, arc=2pt,
  left=6pt, right=6pt, top=4pt, bottom=4pt,
  title={\small\textbf{Functional Completeness — Quick Reference}},
  fonttitle=\small\bfseries]
\small
\begin{tabular}{l l l}
\toprule
\textbf{Concept} & \textbf{Key fact} & \textbf{Detail} \\
\midrule
Truth function         & $f:\{T,F\}^n\to\{T,F\}$; exactly $2^{2^n}$ of them & \hyperref[def:truth-fn]{↓ Def} \\
Functional completeness & Every truth function expressible from $S$ & \hyperref[def:func-complete]{↓ Def} \\
Adequate set           & Synonym for functionally complete set & \hyperref[def:adequate]{↓ Def} \\
Minimal adequate set   & No proper subset is adequate & \hyperref[def:minimal]{↓ Def} \\
\midrule
$\{\neg,\wedge\}$       & Adequate — expresses $\vee$ via De Morgan & \hyperref[thm:std-adequate]{↓ Thm} \\
$\{\neg,\vee\}$         & Adequate & \hyperref[thm:std-adequate]{↓ Thm} \\
$\{\neg,\rightarrow\}$  & Adequate & \hyperref[thm:std-adequate]{↓ Thm} \\
$\{\uparrow\}$ NAND     & Adequate (singleton!) & \hyperref[thm:nand-adequate]{↓ Thm} \\
$\{\downarrow\}$ NOR    & Adequate (singleton!) & \hyperref[thm:nor-adequate]{↓ Thm} \\
$\{\wedge,\vee\}$       & \textbf{Not} adequate — cannot express $\neg$ & \hyperref[thm:non-adequate]{↓ Thm} \\
Post's theorem          & Iff criterion for adequacy via 5 classes & \hyperref[thm:post]{↓ Thm} \\
\bottomrule
\end{tabular}

\vspace{0.4em}
\small $n$-ary truth functions: $n=1 \Rightarrow 4$ functions;\quad $n=2 \Rightarrow 16$;\quad $n=3 \Rightarrow 256$
\end{tcolorbox}

\vspace{1em}

% ---------------------------------------------------------
% Truth Function
% ---------------------------------------------------------
\begin{tcolorbox}[colback=propbox, colframe=propborder, arc=2pt,
  left=6pt, right=6pt, top=4pt, bottom=4pt,
  title={\small\textbf{Definition (Truth Function)}},
  fonttitle=\small\bfseries]
\label{def:truth-fn}
An $n$-ary \emph{truth function} is a function
\[
f : \{\mathsf{T}, \mathsf{F}\}^n \to \{\mathsf{T}, \mathsf{F}\}.
\]
There are exactly $2^{2^n}$ distinct $n$-ary truth functions.
\end{tcolorbox}

\begin{remark}[English reading]
A truth function is the abstract input-output behavior of a formula: it maps
each possible combination of truth values for $n$ variables to a single output.
Two formulas with the same truth table compute the same truth function.
\end{remark}

\begin{remark}[Counting]
For $n$ inputs, there are $2^n$ possible input rows. Each row independently
maps to $\mathsf{T}$ or $\mathsf{F}$, giving $2^{2^n}$ total functions.
For $n=1$: 4 functions (constant-T, identity, negation, constant-F).
For $n=2$: 16 functions, including $\wedge, \vee, \rightarrow, \leftrightarrow, \oplus, \uparrow, \downarrow$, and 9 others.
\end{remark}

\begin{remark}[Consequence]
Functional completeness asks: does a set of connectives generate \emph{all}
$2^{2^n}$ truth functions for every $n$? If yes, no expressible truth function
is beyond the reach of the set.
\end{remark}

\begin{example}
The 4 unary truth functions for input $P$:
\begin{center}
\renewcommand{\arraystretch}{1.2}
\begin{tabular}{c|cccc}
\toprule
$P$ & $\bot$ & $P$ & $\neg P$ & $\top$ \\
\midrule
T & F & T & F & T \\
F & F & F & T & T \\
\bottomrule
\end{tabular}
\end{center}
\end{example}

% ---------------------------------------------------------
% Functional Completeness
% ---------------------------------------------------------
\begin{tcolorbox}[colback=propbox, colframe=propborder, arc=2pt,
  left=6pt, right=6pt, top=4pt, bottom=4pt,
  title={\small\textbf{Definition (Functional Completeness)}},
  fonttitle=\small\bfseries]
\label{def:func-complete}
A set $S$ of logical connectives is \emph{functionally complete} (or
\emph{adequate}) if for every $n \geq 1$ and every $n$-ary truth function $f$,
there exists a formula $\varphi$ built using only connectives from $S$ whose
truth table computes $f$.
\end{tcolorbox}

\begin{remark}[Intuition]
$S$ is adequate if the logic built from $S$ can describe every possible
truth-functional relationship. No truth function escapes it.
\end{remark}

\begin{remark}[Fully quantified]
$S$ is functionally complete $\Longleftrightarrow$
$\forall n \geq 1,\; \forall f : \{T,F\}^n \to \{T,F\},\;
\exists \varphi \in \mathsf{WFF}(S),\;
\forall v : \{T,F\}^n,\; v(\varphi) = f(v)$.
\end{remark}

\begin{tcolorbox}[colback=propbox, colframe=propborder, arc=2pt,
  left=6pt, right=6pt, top=4pt, bottom=4pt,
  title={\small\textbf{Definition (Adequate Set / Minimal Adequate Set)}},
  fonttitle=\small\bfseries]
\label{def:adequate}\label{def:minimal}
An \emph{adequate set} is a functionally complete set.
A \emph{minimal adequate set} is an adequate set no proper subset of which is adequate.
\end{tcolorbox}

% ---------------------------------------------------------
% Standard Adequate Sets
% ---------------------------------------------------------
\begin{tcolorbox}[colback=thmbox, colframe=thmborder, arc=2pt,
  left=6pt, right=6pt, top=4pt, bottom=4pt,
  title={\small\textbf{Theorem (Standard Functionally Complete Sets)}},
  fonttitle=\small\bfseries]
\label{thm:std-adequate}
The following sets of connectives are functionally complete:
\begin{enumerate}
  \item $\{\neg, \wedge\}$
  \item $\{\neg, \vee\}$
  \item $\{\neg, \rightarrow\}$
  \item $\{\bot, \rightarrow\}$ \quad (constant false + conditional)
  \item $\{\neg, \wedge, \vee, \rightarrow, \leftrightarrow\}$ \quad (all standard connectives)
\end{enumerate}
\end{tcolorbox}

\begin{remark}[Proof sketch for $\{\neg, \wedge\}$]
Every truth function can be expressed in DNF using $\neg$, $\wedge$, $\vee$.
Since $P \vee Q \equiv \neg(\neg P \wedge \neg Q)$ (De Morgan), $\vee$ is
definable from $\{\neg, \wedge\}$. Therefore $\{\neg, \wedge\}$ expresses all
DNF formulas, hence all truth functions.
\end{remark}

\begin{remark}[Why $\{\neg, \wedge\}$ is minimal]
Remove either element: $\{\wedge\}$ alone cannot express negation (it
preserves falsity); $\{\neg\}$ alone cannot combine propositions (it never
increases the number of variables). Neither singleton is adequate. So
$\{\neg, \wedge\}$ has no redundant member.
\end{remark}

% ---------------------------------------------------------
% NAND and NOR
% ---------------------------------------------------------
\begin{tcolorbox}[colback=thmbox, colframe=thmborder, arc=2pt,
  left=6pt, right=6pt, top=4pt, bottom=4pt,
  title={\small\textbf{Theorem (NAND is Functionally Complete)}},
  fonttitle=\small\bfseries]
\label{thm:nand-adequate}
$\{\uparrow\}$ (NAND alone) is functionally complete.
\end{tcolorbox}



\begin{tcolorbox}[colback=thmbox, colframe=thmborder, arc=2pt,
  left=6pt, right=6pt, top=4pt, bottom=4pt,
  title={\small\textbf{Theorem (NOR is Functionally Complete)}},
  fonttitle=\small\bfseries]
\label{thm:nor-adequate}
$\{\downarrow\}$ (NOR alone) is functionally complete.
\end{tcolorbox}



\begin{remark}[Consequence — circuit design]
NAND and NOR are each singleton adequate sets. In digital hardware, any
logic circuit can be built entirely from NAND gates or entirely from NOR
gates. This is why NAND/NOR are called \emph{universal gates}.
\end{remark}

% ---------------------------------------------------------
% Non-adequate Sets
% ---------------------------------------------------------
\begin{tcolorbox}[colback=thmbox, colframe=thmborder, arc=2pt,
  left=6pt, right=6pt, top=4pt, bottom=4pt,
  title={\small\textbf{Theorem (Non-Adequate Sets)}},
  fonttitle=\small\bfseries]
\label{thm:non-adequate}
The following sets are \emph{not} functionally complete:
\begin{enumerate}
  \item $\{\wedge, \vee\}$ — cannot express $\neg$ (or any function that
        is false when all inputs are true).
  \item $\{\rightarrow\}$ alone — cannot express functions false when all
        inputs are false.
  \item $\{\neg\}$ alone — unary only; cannot build any binary function.
  \item $\{\leftrightarrow, \neg\}$ — cannot express $\wedge$ (this set
        generates only affine/XOR-based functions under close examination).
\end{enumerate}
\end{tcolorbox}

\begin{remark}[How to prove non-adequacy]
Identify a \emph{closure property} shared by all functions in $S$ that is not
shared by some target truth function. Common properties:
truth-preserving ($f(T,\ldots,T)=T$), falsity-preserving ($f(F,\ldots,F)=F$),
monotonic, affine (XOR-representable), self-dual.
If all members of $S$ share the property, $S$ cannot express functions lacking it.
For $\{\wedge,\vee\}$: both are truth-preserving and falsity-preserving, so the
set cannot express $\neg$ (which is neither).
\end{remark}

% ---------------------------------------------------------
% Post's Theorem
% ---------------------------------------------------------
\begin{tcolorbox}[colback=thmbox, colframe=thmborder, arc=2pt,
  left=6pt, right=6pt, top=4pt, bottom=4pt,
  title={\small\textbf{Theorem (Post's Functional Completeness Theorem)}},
  fonttitle=\small\bfseries]
\label{thm:post}
A set $S$ of Boolean functions is functionally complete if and only if, for
each of the following five \emph{Post classes}, $S$ contains at least one
function \emph{not} in that class:
\begin{enumerate}
  \item $\mathbf{T_0}$: \textbf{Truth-preserving} — $f(T,\ldots,T)=T$.
  \item $\mathbf{T_1}$: \textbf{Falsity-preserving} — $f(F,\ldots,F)=F$.
  \item $\mathbf{M}$: \textbf{Monotonic} — $x_i \leq y_i\;\forall i \Rightarrow f(\mathbf{x}) \leq f(\mathbf{y})$ (where $F<T$).
  \item $\mathbf{A}$: \textbf{Affine (linear)} — $f$ expressible as $a_0 \oplus a_1 x_1 \oplus \cdots \oplus a_n x_n$.
  \item $\mathbf{S}$: \textbf{Self-dual} — $f(x_1,\ldots,x_n) = \neg f(\neg x_1,\ldots,\neg x_n)$.
\end{enumerate}
\end{tcolorbox}

\begin{remark}[How to apply Post's theorem]
For each of the 5 classes, check whether $S$ contains a function outside that
class. If yes for all 5, $S$ is adequate. If $S$ misses any class entirely
(all members of $S$ lie within some class), $S$ is not adequate.
\end{remark}

\begin{remark}[Logical significance]
Post's theorem (Emil Post, 1941) gives a complete, finite, mechanical criterion
for adequacy. It is one of the deepest results of classical propositional logic,
characterizing the entire lattice of clones of Boolean functions.
\end{remark}

\begin{example}[Applying Post's theorem to $\{\neg, \wedge\}$]
\begin{center}
\renewcommand{\arraystretch}{1.3}
\begin{tabular}{lccc}
\toprule
\textbf{Class} & $\neg \in$ class? & $\wedge \in$ class? & \textbf{Some member outside?} \\
\midrule
$T_0$ truth-pres.  & No ($\neg T=F$) & Yes & Yes ($\neg$) \\
$T_1$ falsity-pres.& No ($\neg F=T$) & Yes & Yes ($\neg$) \\
$M$ monotone       & No             & Yes & Yes ($\neg$) \\
$A$ affine         & Yes ($\neg P = T\oplus P$) & No & Yes ($\wedge$) \\
$S$ self-dual      & Yes            & No  & Yes ($\wedge$) \\
\bottomrule
\end{tabular}
\end{center}
All 5 classes have a member of $S$ outside them. By Post's theorem,
$\{\neg,\wedge\}$ is functionally complete. $\checkmark$
\end{example}
