% =========================================================
% Craig's Interpolation Theorem
% =========================================================

\subsection{Craig's Interpolation Theorem}

% ---------------------------------------------------------
% TOOLKIT
% ---------------------------------------------------------
\begin{tcolorbox}[colback=gray!6, colframe=gray!40, arc=2pt,
  left=6pt, right=6pt, top=4pt, bottom=4pt,
  title={\small\textbf{Interpolation — Quick Reference}},
  fonttitle=\small\bfseries]
\small
\begin{tabular}{l l l}
\toprule
\textbf{Concept} & \textbf{Key fact} & \textbf{Detail} \\
\midrule
Common language     & Variables appearing in both $\varphi$ and $\psi$ & \hyperref[def:common-lang]{↓ Def} \\
Interpolant         & $\theta$ s.t.\ $\varphi\models\theta\models\psi$ with vars in both & \hyperref[thm:craig]{↓ Thm} \\
Craig's theorem     & If $\varphi\models\psi$ (shared var), interpolant exists & \hyperref[thm:craig]{↓ Thm} \\
Trivial cases       & $\varphi\equiv\bot$: use $\bot$;\quad $\psi\equiv\top$: use $\top$ & \hyperref[rem:trivial]{↓} \\
Significance        & Modularity of reasoning; Beth definability; program verification & \hyperref[rem:significance]{↓} \\
\bottomrule
\end{tabular}
\end{tcolorbox}

\vspace{1em}

% ---------------------------------------------------------
% Common Language
% ---------------------------------------------------------
\begin{tcolorbox}[colback=propbox, colframe=propborder, arc=2pt,
  left=6pt, right=6pt, top=4pt, bottom=4pt,
  title={\small\textbf{Definition (Common Language)}},
  fonttitle=\small\bfseries]
\label{def:common-lang}
The \emph{common language} of formulas $\varphi$ and $\psi$ is the set of
propositional variables appearing in \emph{both} $\varphi$ and $\psi$:
\[
\mathsf{Var}(\varphi) \cap \mathsf{Var}(\psi).
\]
\end{tcolorbox}

\begin{remark}[Intuition]
Two formulas may talk about overlapping but distinct vocabularies. The common
language is the shared vocabulary — the concepts both formulas discuss.
Interpolation says: when $\varphi$ forces $\psi$, the reason can always be
expressed entirely in the shared vocabulary.
\end{remark}

% ---------------------------------------------------------
% Craig's Theorem
% ---------------------------------------------------------
\begin{tcolorbox}[colback=thmbox, colframe=thmborder, arc=2pt,
  left=6pt, right=6pt, top=4pt, bottom=4pt,
  title={\small\textbf{Theorem (Craig's Interpolation Theorem)}},
  fonttitle=\small\bfseries]
\label{thm:craig}
Let $\varphi$ and $\psi$ be propositional formulas with $\varphi \models \psi$
and $\mathsf{Var}(\varphi) \cap \mathsf{Var}(\psi) \neq \emptyset$.

Then there exists a formula $\theta$ (an \emph{interpolant}) such that:
\begin{enumerate}
  \item $\varphi \models \theta$
  \item $\theta \models \psi$
  \item $\mathsf{Var}(\theta) \subseteq \mathsf{Var}(\varphi) \cap \mathsf{Var}(\psi)$
\end{enumerate}
\end{tcolorbox}

\begin{remark}[English reading]
If $\varphi$ entails $\psi$, there is an intermediate formula $\theta$ that:
(1) follows from $\varphi$; (2) implies $\psi$; and (3) uses only variables
common to both. The interpolant $\theta$ captures exactly the logically relevant
content that $\varphi$ passes to $\psi$, in their shared vocabulary.
\end{remark}

\begin{remark}[Fully quantified form]
$\forall \varphi, \psi \in \mathsf{WFF}$:
if $\varphi \models \psi$ and $\mathsf{Var}(\varphi)\cap\mathsf{Var}(\psi)\neq\emptyset$,
then $\exists \theta \in \mathsf{WFF}$ such that
$\varphi\models\theta$, $\theta\models\psi$, and
$\mathsf{Var}(\theta)\subseteq\mathsf{Var}(\varphi)\cap\mathsf{Var}(\psi)$.
\end{remark}

\begin{remark}[Trivial cases]\label{rem:trivial}
\begin{itemize}
  \item If $\varphi$ is a contradiction, take $\theta = \bot$ (any $\psi$ follows
        from $\bot$, and $\bot$ uses no variables).
  \item If $\psi$ is a tautology, take $\theta = \top$ ($\varphi \models \top$
        always, and $\top \models \psi$ when $\psi$ is a tautology).
  \item If $\mathsf{Var}(\varphi) \cap \mathsf{Var}(\psi) = \emptyset$ and
        $\varphi \models \psi$, then $\varphi$ must be a contradiction or $\psi$
        a tautology (since no shared variable can transmit information). The
        hypothesis requires at least one shared variable to avoid this.
\end{itemize}
\end{remark}

\begin{example}
Let $\varphi = P \wedge Q$ and $\psi = P \vee R$. Then $\varphi \models \psi$
(since $P \wedge Q$ implies $P$, which implies $P \vee R$).
Shared variable: $P$.

Interpolant: $\theta = P$, since $P \wedge Q \models P$ and $P \models P \vee R$,
and $\mathsf{Var}(\theta) = \{P\} \subseteq \{P,Q\} \cap \{P,R\}$. $\checkmark$
\end{example}

\begin{example}
Let $\varphi = P \rightarrow Q$ and $\psi = \neg Q \rightarrow \neg P$
(contrapositive). They are logically equivalent, so $\varphi \models \psi$.
Shared variables: $\{P,Q\}$.

Interpolants include $\theta = P \rightarrow Q$, $\theta = \neg P \vee Q$, or
$\theta = \top$ (since $\psi$ is a tautology given $\varphi \equiv \psi$).
\end{example}

% ---------------------------------------------------------
% Significance
% ---------------------------------------------------------
\begin{remark}[Logical and mathematical significance]\label{rem:significance}
Craig's theorem (William Craig, 1957) has several important consequences:
\begin{enumerate}
  \item \textbf{Modularity of reasoning:} Whenever $\varphi \models \psi$, the
        ``reason'' can be expressed using only the shared vocabulary. This
        justifies modular reasoning in large formal systems — you never need to
        bring in private vocabulary of one module to pass a conclusion to another.
  \item \textbf{Beth definability:} A predicate is implicitly definable in a
        theory if and only if it is explicitly definable. In propositional terms:
        if a variable is determined by the rest of a formula, an explicit
        definition exists. Interpolation is the key proof tool for this.
  \item \textbf{Constructive proof via sequent calculus:} Interpolation can be
        proved constructively by tracing a cut-elimination proof in the sequent
        calculus. The interpolant is built simultaneously with the proof.
  \item \textbf{Program verification and model checking:} Interpolants are used
        to compute abstractions of reachability proofs in software model checking.
        Given a failing execution trace, an interpolant separates initial states
        from bad states.
\end{enumerate}
\end{remark}

\begin{remark}[Interpolation in first-order logic]
Craig's theorem extends to first-order logic: if $\varphi \models \psi$ in FOL,
an interpolant exists in their common language. The proof is harder, requiring
cut elimination in first-order sequent calculus. In higher-order logic,
interpolation can fail.
\end{remark}
