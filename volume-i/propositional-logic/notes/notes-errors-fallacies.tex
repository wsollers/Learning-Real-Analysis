% =========================================================
% Common Errors and Fallacies
% =========================================================

\subsection{Common Errors and Fallacies}
% ---------------------------------------------------------

\subsubsection{Formal Fallacies}

\begin{definition}[Affirming the Consequent]
\textbf{Affirming the Consequent} is the invalid inference:
\[
\begin{array}{l}
P \rightarrow Q \\
Q \\ \hline
P \quad \text{(INVALID)}
\end{array}
\]

This is fallacious because $Q$ being true does not guarantee that $P$ is true;
$Q$ might be true for other reasons.
\end{definition}

\begin{example}
``If it rains, the ground is wet. The ground is wet. Therefore, it rained.''

This is invalid: the ground might be wet because of a sprinkler.
\end{example}

\begin{definition}[Denying the Antecedent]
\textbf{Denying the Antecedent} is the invalid inference:
\[
\begin{array}{l}
P \rightarrow Q \\
\neg P \\ \hline
\neg Q \quad \text{(INVALID)}
\end{array}
\]

This is fallacious because the conditional $P \rightarrow Q$ says nothing about
what happens when $P$ is false.
\end{definition}

\begin{example}
``If it rains, the ground is wet. It did not rain. Therefore, the ground is not wet.''

This is invalid: the ground might be wet for other reasons.
\end{example}

\subsubsection{Comparison: Valid vs.\ Invalid Inference Patterns}

\begin{center}
\renewcommand{\arraystretch}{1.3}
\begin{tabular}{|l|c|l|}
\hline
\textbf{Pattern} & \textbf{Valid?} & \textbf{Form} \\
\hline
Modus Ponens & Yes & $P \rightarrow Q$, $P$ $\therefore$ $Q$ \\
Modus Tollens & Yes & $P \rightarrow Q$, $\neg Q$ $\therefore$ $\neg P$ \\
Affirming the Consequent & No & $P \rightarrow Q$, $Q$ $\therefore$ $P$ \\
Denying the Antecedent & No & $P \rightarrow Q$, $\neg P$ $\therefore$ $\neg Q$ \\
\hline
\end{tabular}
\end{center}

\subsubsection{Common Truth Table Errors}

\begin{remark}[Common Errors]
\begin{itemize}
  \item \textbf{Misunderstanding the conditional:} The conditional
        $P \rightarrow Q$ is false only when $P$ is true and $Q$ is false. It is
        true in all other cases, including when $P$ is false.

  \item \textbf{Confusing $\rightarrow$ with $\leftrightarrow$:} The conditional
        is not symmetric; $P \rightarrow Q$ is not equivalent to $Q \rightarrow P$.

  \item \textbf{Confusing ``or'' meanings:} In logic, $\vee$ is inclusive
        (true when at least one disjunct is true). Exclusive or (true when
        exactly one is true) is written $P \oplus Q$ or $(P \vee Q) \wedge \neg(P \wedge Q)$.

  \item \textbf{Forgetting rows:} A truth table for $n$ variables must have
        $2^n$ rows.

  \item \textbf{Operator precedence:} Failing to account for precedence leads
        to incorrect parsing of formulas.
\end{itemize}
\end{remark}

\subsubsection{Fallacy Checklist}

\begin{center}
\renewcommand{\arraystretch}{1.4}
\begin{tabular}{|p{5cm}|p{8cm}|}
\hline
\textbf{Fallacy} & \textbf{Diagnostic Question} \\
\hline
Affirming the Consequent & Was $Q$ used to infer $P$ from $P \rightarrow Q$? \\
\hline
Denying the Antecedent & Was $\neg P$ used to infer $\neg Q$ from $P \rightarrow Q$? \\
\hline
Misreading Conditional & Was the conditional treated as false when the antecedent is false? \\
\hline
Confusing Implication Direction & Was $P \rightarrow Q$ confused with $Q \rightarrow P$? \\
\hline
Missing Cases & Were all $2^n$ rows of the truth table considered? \\
\hline
\end{tabular}
\end{center}

% ---------------------------------------------------------
