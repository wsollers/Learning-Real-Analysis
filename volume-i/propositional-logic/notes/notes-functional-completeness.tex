% =========================================================
% Functional Completeness
% =========================================================

\subsection{Functional Completeness}
% ---------------------------------------------------------

\subsubsection{Truth Functions}

\begin{definition}[Truth Function]
An $n$-ary \emph{truth function} is a function
\[
f : \{\mathsf{T}, \mathsf{F}\}^n \to \{\mathsf{T}, \mathsf{F}\}.
\]

There are exactly $2^{2^n}$ distinct $n$-ary truth functions.
\end{definition}

\begin{example}
For $n = 1$, there are $2^{2^1} = 4$ unary truth functions:
\begin{center}
\renewcommand{\arraystretch}{1.2}
\begin{tabular}{|c||c|c|c|c|}
\hline
$P$ & $f_1$ & $f_2$ & $f_3$ & $f_4$ \\
\hline
T & T & T & F & F \\
F & T & F & T & F \\
\hline
\end{tabular}
\end{center}
These correspond to: $f_1 = \top$ (constant true), $f_2 = P$ (identity),
$f_3 = \neg P$ (negation), $f_4 = \bot$ (constant false).
\end{example}

\begin{example}
For $n = 2$, there are $2^{2^2} = 16$ binary truth functions, including
conjunction, disjunction, conditional, biconditional, NAND, NOR, XOR, and others.
\end{example}

\subsubsection{Functional Completeness}

\begin{definition}[Functional Completeness]
A set $S$ of logical connectives is \emph{functionally complete} (or
\emph{adequate}) if every truth function can be expressed by a formula using
only connectives from $S$.

Equivalently, $S$ is functionally complete if for every truth function $f$,
there exists a formula $\varphi$ built using only connectives in $S$ such that
$\varphi$ computes $f$.
\end{definition}

\begin{theorem}[Standard Functionally Complete Sets]
The following sets of connectives are functionally complete:
\begin{enumerate}
  \item $\{\neg, \wedge\}$
  \item $\{\neg, \vee\}$
  \item $\{\neg, \rightarrow\}$
  \item $\{\bot, \rightarrow\}$ where $\bot$ is the constant false
  \item $\{\neg, \wedge, \vee\}$
  \item $\{\neg, \wedge, \vee, \rightarrow, \leftrightarrow\}$
\end{enumerate}
\end{theorem}



\begin{theorem}[NAND is Functionally Complete]
The singleton set $\{\uparrow\}$ (NAND alone) is functionally complete.
\end{theorem}



\begin{theorem}[NOR is Functionally Complete]
The singleton set $\{\downarrow\}$ (NOR alone) is functionally complete.
\end{theorem}



\subsubsection{Adequate Sets of Connectives}

\begin{definition}[Adequate Set]
An \emph{adequate set} of connectives is a functionally complete set. A
\emph{minimal adequate set} is an adequate set such that no proper subset is
adequate.
\end{definition}

\begin{theorem}[Minimal Adequate Sets]
The minimal adequate sets of connectives are:
\begin{enumerate}
  \item $\{\uparrow\}$ (NAND alone)
  \item $\{\downarrow\}$ (NOR alone)
  \item $\{\neg, \wedge\}$
  \item $\{\neg, \vee\}$
  \item $\{\neg, \rightarrow\}$
  \item $\{\bot, \rightarrow\}$
\end{enumerate}
\end{theorem}

\begin{theorem}[Non-Adequate Sets]
The following sets are \emph{not} functionally complete:
\begin{enumerate}
  \item $\{\wedge, \vee\}$ — cannot express negation
  \item $\{\wedge, \vee, \rightarrow, \leftrightarrow\}$ — cannot express negation
  \item $\{\neg\}$ — cannot combine propositions
  \item $\{\rightarrow\}$ — cannot express functions that are false when all
        inputs are false
  \item $\{\leftrightarrow, \neg\}$ — cannot express functions that are true
        for an odd number of true inputs other than exactly one
\end{enumerate}
\end{theorem}

\begin{remark}
To show that a set $S$ is not functionally complete, one typically identifies
a property preserved by all connectives in $S$ that is not preserved by some
truth function. Common properties used include:
\begin{itemize}
  \item Preserving truth (true when all inputs are true)
  \item Preserving falsity (false when all inputs are false)
  \item Monotonicity
  \item Affine/linear (expressible as XOR of inputs and constants)
  \item Self-duality
\end{itemize}
\end{remark}

\subsubsection{Post's Functional Completeness Theorem}

\begin{theorem}[Post's Theorem]
A set $S$ of Boolean functions is functionally complete if and only if for each
of the following five properties, $S$ contains at least one function that does
not have that property:

\begin{enumerate}
  \item \textbf{Truth-preserving:} $f(\mathsf{T}, \dots, \mathsf{T}) = \mathsf{T}$
  \item \textbf{Falsity-preserving:} $f(\mathsf{F}, \dots, \mathsf{F}) = \mathsf{F}$
  \item \textbf{Monotonic:} If $x_i \leq y_i$ for all $i$ (where $\mathsf{F} < \mathsf{T}$),
        then $f(x_1, \dots, x_n) \leq f(y_1, \dots, y_n)$
  \item \textbf{Affine:} $f$ can be written as $a_0 \oplus a_1 x_1 \oplus \cdots \oplus a_n x_n$
        for constants $a_i \in \{\mathsf{T}, \mathsf{F}\}$
  \item \textbf{Self-dual:} $f(x_1, \dots, x_n) = \neg f(\neg x_1, \dots, \neg x_n)$
\end{enumerate}
\end{theorem}

\begin{example}[Applying Post's Theorem]
To verify that $\{\neg, \wedge\}$ is functionally complete:

\begin{center}
\renewcommand{\arraystretch}{1.3}
\begin{tabular}{|l|c|c|}
\hline
\textbf{Property} & $\neg$ & $\wedge$ \\
\hline
Truth-preserving & No ($\neg\mathsf{T} = \mathsf{F}$) & Yes \\
Falsity-preserving & No ($\neg\mathsf{F} = \mathsf{T}$) & Yes \\
Monotonic & No & Yes \\
Affine & Yes ($\neg P \equiv \mathsf{T} \oplus P$) & No \\
Self-dual & Yes ($\neg P = \neg(\neg(\neg P))$) & No \\
\hline
\end{tabular}
\end{center}

For each property, at least one of $\{\neg, \wedge\}$ lacks it, so by Post's
theorem the set is functionally complete.
\end{example}

% ---------------------------------------------------------
