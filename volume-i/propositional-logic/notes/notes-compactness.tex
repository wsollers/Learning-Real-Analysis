% =========================================================
% Compactness Theorem
% =========================================================

\subsection{Compactness Theorem}
% ---------------------------------------------------------

\subsubsection{Statement of Compactness}

\begin{theorem}[Compactness Theorem for Propositional Logic]
A set of formulas $\Gamma$ is satisfiable if and only if every finite subset of
$\Gamma$ is satisfiable.

Equivalently:
\begin{itemize}
  \item If $\Gamma$ is unsatisfiable, then some finite subset of $\Gamma$ is
        unsatisfiable.
  \item If every finite subset of $\Gamma$ is satisfiable, then $\Gamma$ is
        satisfiable.
\end{itemize}
\end{theorem}

\begin{remark}
The ``if'' direction is trivial: if $\Gamma$ is satisfiable, then any truth
assignment satisfying $\Gamma$ also satisfies every subset of $\Gamma$.

The ``only if'' direction (the substantive content) states that satisfiability
of an infinite set reduces to satisfiability of its finite subsets.
\end{remark}

\subsubsection{Consequence Form of Compactness}

\begin{corollary}[Compactness for Logical Consequence]
If $\Gamma \models \varphi$, then there exists a finite subset
$\Gamma_0 \subseteq \Gamma$ such that $\Gamma_0 \models \varphi$.

Equivalently, logical consequence from an infinite set of premises always
depends on only finitely many of those premises.
\end{corollary}



\subsubsection{Applications of Compactness}

\begin{example}[Graph Coloring]
Let $G$ be an infinite graph. If every finite subgraph of $G$ is $k$-colorable,
then $G$ is $k$-colorable.

\textbf{Proof sketch:} For each vertex $v$ and color $i \in \{1, \dots, k\}$,
introduce a propositional variable $C_{v,i}$ meaning ``vertex $v$ has color $i$.''

Add formulas:
\begin{itemize}
  \item For each vertex $v$: $C_{v,1} \vee C_{v,2} \vee \cdots \vee C_{v,k}$
        (every vertex has some color).
  \item For each vertex $v$ and distinct colors $i, j$:
        $\neg(C_{v,i} \wedge C_{v,j})$ (no vertex has two colors).
  \item For each edge $(u, v)$ and each color $i$:
        $\neg(C_{u,i} \wedge C_{v,i})$ (adjacent vertices have different colors).
\end{itemize}

Each finite subset of these formulas involves only finitely many vertices, hence
a finite subgraph. If every finite subgraph is $k$-colorable, every finite subset
is satisfiable. By compactness, the entire set is satisfiable, giving a
$k$-coloring of $G$.
\end{example}

\begin{example}[Nonstandard Models]
Compactness can be used to construct ``nonstandard'' models with unexpected
properties by adding infinitely many formulas that are individually consistent
with a theory but together force new elements.
\end{example}

\subsubsection{Proof of Compactness}

\begin{remark}[Proof Methods]
The compactness theorem can be proved in several ways:

\begin{enumerate}
  \item \textbf{Via completeness:} If every finite subset of $\Gamma$ is
        satisfiable, then no finite subset derives a contradiction. By
        completeness, $\Gamma$ itself does not derive a contradiction, so
        $\Gamma$ is consistent, hence (by completeness again) satisfiable.

  \item \textbf{Direct construction:} Build a satisfying assignment for $\Gamma$
        by systematically extending partial assignments, ensuring consistency
        with all formulas in $\Gamma$. This typically uses König's lemma or
        Zorn's lemma.

  \item \textbf{Ultraproducts:} Use ultraproducts of finite satisfying
        assignments to construct a satisfying assignment for $\Gamma$.
\end{enumerate}
\end{remark}

% ---------------------------------------------------------
