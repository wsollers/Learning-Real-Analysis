% =========================================================
% Normal Forms
% =========================================================

\subsection{Normal Forms}
% ---------------------------------------------------------

\subsubsection{Literals and Clauses}

\begin{definition}[Literal]
A \emph{literal} is either a propositional variable or the negation of a
propositional variable.

A \emph{positive literal} is a propositional variable $P$.
A \emph{negative literal} is the negation of a propositional variable $\neg P$.
\end{definition}

\begin{definition}[Clause]
A \emph{clause} is a disjunction of literals.

A \emph{conjunctive clause} (or \emph{term}) is a conjunction of literals.
\end{definition}

\subsubsection{Negation Normal Form}

\begin{definition}[Negation Normal Form]
A formula is in \emph{negation normal form} (NNF) if:
\begin{enumerate}
  \item Negation ($\neg$) is applied only to propositional variables.
  \item The only connectives used are $\neg$, $\wedge$, and $\vee$.
\end{enumerate}
\end{definition}

\begin{theorem}[NNF Conversion]
Every propositional formula is logically equivalent to a formula in negation
normal form.
\end{theorem}

\begin{remark}[Procedure for NNF Conversion]
\begin{enumerate}
  \item Eliminate $\rightarrow$ and $\leftrightarrow$ using:
  \begin{align*}
    P \rightarrow Q &\equiv \neg P \vee Q \\
    P \leftrightarrow Q &\equiv (P \wedge Q) \vee (\neg P \wedge \neg Q)
  \end{align*}
  \item Push negations inward using De Morgan's laws and double negation:
  \begin{align*}
    \neg(P \wedge Q) &\equiv \neg P \vee \neg Q \\
    \neg(P \vee Q) &\equiv \neg P \wedge \neg Q \\
    \neg\neg P &\equiv P
  \end{align*}
\end{enumerate}
\end{remark}

\subsubsection{Conjunctive Normal Form}

\begin{definition}[Conjunctive Normal Form]
A formula is in \emph{conjunctive normal form} (CNF) if it is a conjunction of
clauses; that is, a conjunction of disjunctions of literals:
\[
\bigwedge_{i} \bigvee_{j} L_{ij}
\]
where each $L_{ij}$ is a literal.
\end{definition}

\begin{theorem}[CNF Existence]
Every propositional formula is logically equivalent to a formula in conjunctive
normal form.
\end{theorem}

\begin{example}
The formula $(P \vee Q) \wedge (\neg P \vee R) \wedge (Q \vee \neg R)$ is in CNF.
\end{example}

\subsubsection{Disjunctive Normal Form}

\begin{definition}[Disjunctive Normal Form]
A formula is in \emph{disjunctive normal form} (DNF) if it is a disjunction of
conjunctive clauses; that is, a disjunction of conjunctions of literals:
\[
\bigvee_{i} \bigwedge_{j} L_{ij}
\]
where each $L_{ij}$ is a literal.
\end{definition}

\begin{theorem}[DNF Existence]
Every propositional formula is logically equivalent to a formula in disjunctive
normal form.
\end{theorem}

\begin{example}
The formula $(P \wedge Q) \vee (\neg P \wedge R) \vee (Q \wedge \neg R)$ is in DNF.
\end{example}

\subsubsection{Conversion Procedures}

\begin{remark}[CNF Conversion Procedure]
To convert a formula to CNF:
\begin{enumerate}
  \item Convert to negation normal form.
  \item Apply distribution of $\vee$ over $\wedge$:
  \[
  P \vee (Q \wedge R) \equiv (P \vee Q) \wedge (P \vee R)
  \]
  \item Repeat until the formula is a conjunction of disjunctions.
\end{enumerate}
\end{remark}

\begin{remark}[DNF Conversion Procedure]
To convert a formula to DNF:
\begin{enumerate}
  \item Convert to negation normal form.
  \item Apply distribution of $\wedge$ over $\vee$:
  \[
  P \wedge (Q \vee R) \equiv (P \wedge Q) \vee (P \wedge R)
  \]
  \item Repeat until the formula is a disjunction of conjunctions.
\end{enumerate}
\end{remark}

\begin{remark}[Truth Table Method]
CNF and DNF can also be constructed directly from a truth table:
\begin{itemize}
  \item \textbf{DNF:} For each row where the formula is true, form a conjunctive
        clause using positive literals for variables assigned T and negative
        literals for variables assigned F. Take the disjunction of all such clauses.
  \item \textbf{CNF:} For each row where the formula is false, form a disjunctive
        clause using positive literals for variables assigned F and negative
        literals for variables assigned T. Take the conjunction of all such clauses.
\end{itemize}
\end{remark}

% ---------------------------------------------------------
