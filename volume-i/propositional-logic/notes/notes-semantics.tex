% =========================================================
% Semantics of Propositional Logic
% =========================================================

\subsection{Semantics of Propositional Logic}
% ---------------------------------------------------------

\subsubsection{Truth Assignments}

\begin{definition}[Truth Assignment]
A \emph{truth assignment} (or \emph{valuation}) is a function
\[
v : \mathsf{Prop} \to \{\mathsf{T}, \mathsf{F}\}
\]
that assigns a truth value to each propositional variable.
\end{definition}

\begin{definition}[Extension to All Formulas]
Every truth assignment $v$ extends uniquely to a function
$\hat{v} : \mathsf{WFF} \to \{\mathsf{T}, \mathsf{F}\}$ defined recursively:

\begin{enumerate}
  \item If $\varphi \in \mathsf{Prop}$, then $\hat{v}(\varphi) = v(\varphi)$.

  \item $\hat{v}(\neg\varphi) = \mathsf{T}$ iff $\hat{v}(\varphi) = \mathsf{F}$.

  \item $\hat{v}(\varphi \wedge \psi) = \mathsf{T}$ iff
        $\hat{v}(\varphi) = \mathsf{T}$ and $\hat{v}(\psi) = \mathsf{T}$.

  \item $\hat{v}(\varphi \vee \psi) = \mathsf{T}$ iff
        $\hat{v}(\varphi) = \mathsf{T}$ or $\hat{v}(\psi) = \mathsf{T}$.

  \item $\hat{v}(\varphi \rightarrow \psi) = \mathsf{T}$ iff
        $\hat{v}(\varphi) = \mathsf{F}$ or $\hat{v}(\psi) = \mathsf{T}$.

  \item $\hat{v}(\varphi \leftrightarrow \psi) = \mathsf{T}$ iff
        $\hat{v}(\varphi) = \hat{v}(\psi)$.
\end{enumerate}

We typically write $v(\varphi)$ instead of $\hat{v}(\varphi)$.
\end{definition}

\subsubsection{Truth Tables}

\begin{definition}[Truth Table]
A \emph{truth table} is a tabular representation of all possible truth values
of a formula under all possible truth assignments to its propositional variables.
\end{definition}

\begin{center}
\textbf{Truth Tables for the Basic Logical Connectives}

\vspace{0.5em}
\renewcommand{\arraystretch}{1.2}
\begin{tabular}{|c|c||c|c|c|c|c|}
\hline
$P$ & $Q$ & $\neg P$ & $P \wedge Q$ & $P \vee Q$ & $P \rightarrow Q$ & $P \leftrightarrow Q$ \\
\hline
T & T & F & T & T & T & T \\
T & F & F & F & T & F & F \\
F & T & T & F & T & T & F \\
F & F & T & F & F & T & T \\
\hline
\end{tabular}
\end{center}

\begin{remark}
A formula with $n$ distinct propositional variables has $2^n$ rows in its truth
table, corresponding to the $2^n$ possible truth assignments.
\end{remark}

\subsubsection{Satisfaction and Truth}

\begin{definition}[Satisfaction]
A truth assignment $v$ \emph{satisfies} a formula $\varphi$, written
$v \models \varphi$, if $v(\varphi) = \mathsf{T}$.

A truth assignment $v$ satisfies a set of formulas $\Gamma$ if $v$ satisfies
every formula in $\Gamma$.
\end{definition}

\begin{definition}[Truth under an Assignment]
A formula $\varphi$ is \emph{true under} $v$ if $v \models \varphi$.
A formula $\varphi$ is \emph{false under} $v$ if $v \not\models \varphi$.
\end{definition}

\subsubsection{Validity, Satisfiability, and Contradiction}

\begin{definition}[Tautology]
A formula $\varphi$ is a \emph{tautology} (or is \emph{valid}) if it is true
under every truth assignment.

We write $\models \varphi$ to indicate that $\varphi$ is a tautology.

Tautologies are sometimes denoted by the symbol $\top$.
\end{definition}

\begin{definition}[Contradiction]
A formula $\varphi$ is a \emph{contradiction} (or is \emph{unsatisfiable}) if
it is false under every truth assignment.

Contradictions are sometimes denoted by the symbol $\bot$.
\end{definition}

\begin{definition}[Satisfiable Formula]
A formula $\varphi$ is \emph{satisfiable} if there exists at least one truth
assignment under which $\varphi$ is true.

A formula is satisfiable if and only if it is not a contradiction.
\end{definition}

\begin{definition}[Contingency]
A formula $\varphi$ is a \emph{contingency} if it is neither a tautology nor a
contradiction; that is, it is true under some assignments and false under others.
\end{definition}

\begin{remark}
Every formula falls into exactly one of these three categories:
\begin{itemize}
  \item Tautology: true under all assignments.
  \item Contradiction: false under all assignments.
  \item Contingency: true under some, false under others.
\end{itemize}
\end{remark}

\begin{example}
\begin{itemize}
  \item $P \vee \neg P$ is a tautology (law of excluded middle).
  \item $P \wedge \neg P$ is a contradiction.
  \item $P \rightarrow Q$ is a contingency.
\end{itemize}
\end{example}

\subsubsection{Logical Consequence}

\begin{definition}[Logical Consequence]
Let $\Gamma$ be a set of formulas and $\varphi$ a formula.
We say that $\varphi$ is a \emph{logical consequence} of $\Gamma$, written
\[
\Gamma \models \varphi,
\]
if every truth assignment that satisfies all formulas in $\Gamma$ also satisfies
$\varphi$.

Equivalently, there is no truth assignment under which all formulas in $\Gamma$
are true and $\varphi$ is false.
\end{definition}

\begin{definition}[Tautological Implication]
Let $P$ and $Q$ be formulas.
We say that $P$ \emph{tautologically implies} $Q$, written
\[
P \models_{\mathsf{taut}} Q,
\]
if and only if the formula $P \rightarrow Q$ is a tautology.

Equivalently, $\{P\} \models Q$.
\end{definition}

\begin{remark}
Tautological implication is a special case of logical consequence where the
premise set is a singleton.
\end{remark}

\subsubsection{Logical Equivalence}

\begin{definition}[Logical Equivalence]
Two formulas $\varphi$ and $\psi$ are \emph{logically equivalent}, written
\[
\varphi \equiv \psi,
\]
if they have the same truth value under every truth assignment.

Equivalently:
\begin{itemize}
  \item $\varphi \models \psi$ and $\psi \models \varphi$.
  \item $\varphi \leftrightarrow \psi$ is a tautology.
  \item $\varphi$ and $\psi$ have identical truth tables.
\end{itemize}
\end{definition}

\begin{remark}
Logically equivalent formulas may be substituted for one another in any context
without changing the truth value of the containing formula.
\end{remark}

% ---------------------------------------------------------
