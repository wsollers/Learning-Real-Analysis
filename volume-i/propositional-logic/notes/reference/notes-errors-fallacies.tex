% =========================================================
% Common Errors and Fallacies
% =========================================================

\subsection{Common Errors and Fallacies}

% ---------------------------------------------------------
% TOOLKIT
% ---------------------------------------------------------
\begin{tcolorbox}[colback=gray!6, colframe=gray!40, arc=2pt,
  left=6pt, right=6pt, top=4pt, bottom=4pt,
  title={\small\textbf{Errors and Fallacies — Quick Reference}},
  fonttitle=\small\bfseries]
\small
\begin{tabular}{l l l l}
\toprule
\textbf{Error} & \textbf{Form} & \textbf{Valid?} & \textbf{Detail} \\
\midrule
Modus Ponens          & $P\to Q,\;P \;\therefore Q$          & \textbf{Yes} & \hyperref[def:mp-valid]{↓} \\
Modus Tollens         & $P\to Q,\;\neg Q \;\therefore \neg P$ & \textbf{Yes} & \hyperref[def:mt-valid]{↓} \\
Affirming Consequent  & $P\to Q,\;Q \;\therefore P$          & No           & \hyperref[def:aff-cons]{↓} \\
Denying Antecedent    & $P\to Q,\;\neg P \;\therefore \neg Q$ & No           & \hyperref[def:deny-ant]{↓} \\
\midrule
Conditional misread   & $P\to Q$ is false when $P$ true, $Q$ false only & \hyperref[rem:cond-error]{↓} \\
Inclusive vs.\ exclusive $\vee$ & Logic uses inclusive or & \hyperref[rem:or-error]{↓} \\
Missing rows          & $n$ vars $\Rightarrow$ $2^n$ rows needed & \hyperref[rem:rows-error]{↓} \\
Precedence errors     & $\neg P\wedge Q \neq \neg(P\wedge Q)$ & \hyperref[rem:prec-error]{↓} \\
\bottomrule
\end{tabular}
\end{tcolorbox}

\vspace{1em}

\subsubsection*{Formal Fallacies}

\begin{tcolorbox}[colback=propbox, colframe=propborder, arc=2pt,
  left=6pt, right=6pt, top=4pt, bottom=4pt,
  title={\small\textbf{Definition (Affirming the Consequent)}},
  fonttitle=\small\bfseries]
\label{def:aff-cons}
The following inference is \textbf{invalid}:
\[
\frac{P \rightarrow Q \qquad Q}{P \quad \text{(INVALID)}}
\]
\end{tcolorbox}

\begin{remark}[Why it fails]
$P \rightarrow Q$ says: whenever $P$ is true, $Q$ must be true. It says nothing
about what happens when $Q$ is true. $Q$ may hold for reasons entirely unrelated
to $P$. Countermodel: $P = F$, $Q = T$. Then $P \rightarrow Q$ is true and $Q$
is true, but $P$ is false.
\end{remark}

\begin{remark}[English example]
``If it rains, the ground is wet. The ground is wet. Therefore, it rained.''
— Invalid: a sprinkler could explain the wet ground.
\end{remark}

\begin{remark}[Fully quantified diagnostic]
$\not\models [(P \rightarrow Q) \wedge Q] \rightarrow P$.
Check: set $P = F$, $Q = T$. The antecedent is $T \wedge T = T$; the consequent
is $F$. The whole formula is false under this assignment. Not a tautology.
\end{remark}

\begin{tcolorbox}[colback=propbox, colframe=propborder, arc=2pt,
  left=6pt, right=6pt, top=4pt, bottom=4pt,
  title={\small\textbf{Definition (Denying the Antecedent)}},
  fonttitle=\small\bfseries]
\label{def:deny-ant}
The following inference is \textbf{invalid}:
\[
\frac{P \rightarrow Q \qquad \neg P}{\neg Q \quad \text{(INVALID)}}
\]
\end{tcolorbox}

\begin{remark}[Why it fails]
$P \rightarrow Q$ only guarantees $Q$ when $P$ holds. When $P$ is false, $Q$
may be true or false — the conditional is silent. Countermodel: $P = F$, $Q = T$.
\end{remark}

\begin{remark}[English example]
``If it rains, the ground is wet. It did not rain. Therefore, the ground is not
wet.'' — Invalid: a sprinkler could still make the ground wet.
\end{remark}

\subsubsection*{Valid Inference Patterns for Comparison}

\begin{remark}\label{def:mp-valid}\label{def:mt-valid}
The two valid patterns involving a conditional:
\begin{center}
\renewcommand{\arraystretch}{1.3}
\begin{tabular}{llll}
\toprule
\textbf{Pattern} & \textbf{Form} & \textbf{Valid?} & \textbf{Why} \\
\midrule
Modus Ponens & $P\to Q,\; P \;\therefore Q$ & Yes & $\models[(P\to Q)\wedge P]\to Q$ \\
Modus Tollens & $P\to Q,\;\neg Q \;\therefore\neg P$ & Yes & Contrapositive of MP \\
Affirming Consequent & $P\to Q,\; Q \;\therefore P$ & \textbf{No} & Countermodel: $P=F, Q=T$ \\
Denying Antecedent & $P\to Q,\;\neg P \;\therefore\neg Q$ & \textbf{No} & Countermodel: $P=F, Q=T$ \\
\bottomrule
\end{tabular}
\end{center}
The two fallacies are easy to confuse because they \emph{look like} the valid
patterns with one element swapped.
\end{remark}

\subsubsection*{Common Truth Table Errors}

\begin{remark}[Misreading the conditional]\label{rem:cond-error}
The conditional $P \rightarrow Q$ is false \emph{only} in the case
$(P, Q) = (T, F)$. In all other cases it is true, including when $P$ is false.
This ``vacuous truth'' is counterintuitive but essential. The mistake is treating
the conditional as false whenever the antecedent is false.
\end{remark}

\begin{remark}[Confusing $\rightarrow$ with $\leftrightarrow$]\
$P \rightarrow Q$ is \emph{not} symmetric. $P \rightarrow Q$ and $Q \rightarrow P$
are different (they are converses). Treating a conditional as a biconditional
is a form of affirming the consequent.
\end{remark}

\begin{remark}[Inclusive vs.\ exclusive or]\label{rem:or-error}
In formal logic, $\vee$ is \emph{inclusive}: $P \vee Q$ is true when both
$P$ and $Q$ are true. Natural language ``or'' is often exclusive (``tea or
coffee'' usually means one but not both). When formalizing English, always
check which reading is intended.
\end{remark}

\begin{remark}[Forgetting rows]\label{rem:rows-error}
A formula with $n$ distinct variables has $2^n$ rows. For 3 variables: 8 rows.
Omitting rows leads to incorrect classification (e.g., calling a contingency a
tautology because you only checked favorable assignments).
\end{remark}

\begin{remark}[Operator precedence errors]\label{rem:prec-error}
\begin{itemize}
  \item $\neg P \wedge Q$ means $(\neg P) \wedge Q$, \textbf{not} $\neg(P \wedge Q)$.
  \item $P \vee Q \wedge R$ means $P \vee (Q \wedge R)$ if $\wedge$ binds tighter.
  \item When in doubt: add explicit parentheses.
\end{itemize}
\end{remark}

\subsubsection*{Fallacy Checklist}

\begin{center}
\renewcommand{\arraystretch}{1.4}
\begin{tabular}{p{4.5cm} p{8.5cm}}
\toprule
\textbf{Potential error} & \textbf{Diagnostic question} \\
\midrule
Affirming the consequent & Was $Q$ used to infer $P$ from $P \rightarrow Q$? \\
Denying the antecedent & Was $\neg P$ used to infer $\neg Q$ from $P \rightarrow Q$? \\
Conditional misread & Was the conditional treated as false when the antecedent is false? \\
Confusing direction & Was $P \rightarrow Q$ treated as equivalent to $Q \rightarrow P$? \\
Missing rows & Were all $2^n$ truth-table rows considered? \\
Precedence error & Were all connectives properly parenthesized before evaluating? \\
Inclusive/exclusive confusion & Was $\vee$ read as exclusive when inclusive was intended? \\
\bottomrule
\end{tabular}
\end{center}
