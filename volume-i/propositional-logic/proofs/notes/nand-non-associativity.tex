% =========================================================
% Proof: Non-Associativity of NAND (Sheffer Stroke)
% Source: volume-i/propositional-logic/notes/syntax/notes-connectives.tex
% =========================================================

\subsection*{Non-Associativity of NAND}
\label{prf:nand-non-associativity}

\begin{remark}[Return]
\hyperref[prop:nand-complete]{$\leftarrow$ Back to Proposition (Non-Associativity of NAND) in Notes}
\end{remark}

\begin{proposition}[Non-Associativity of NAND]
NAND is not associative. That is,
\[
    (P \uparrow Q) \uparrow R \;\not\equiv\; P \uparrow (Q \uparrow R).
\]
\end{proposition}

\begin{proof}
\Given The definition $P \uparrow Q \;\equiv\; \neg(P \wedge Q)$.
\Goal To show that $(P \uparrow Q) \uparrow R \not\equiv P \uparrow (Q \uparrow R)$,
i.e.\ that the two sides differ in truth value for at least one assignment.

\Strategy We exhibit a counterexample: a single truth-value assignment to $P$, $Q$, $R$
under which the left-hand side and right-hand side take opposite truth values.

\medskip
\textbf{Assignment.} Set $P := \mathsf{T}$, $Q := \mathsf{T}$, $R := \mathsf{T}$.

\medskip
\textbf{Left-hand side: $(P \uparrow Q) \uparrow R$.}

\ByDef $P \uparrow Q \;\equiv\; \neg(P \wedge Q)$.
Since $P = Q = \mathsf{T}$, we have $P \wedge Q = \mathsf{T}$, so
\[
    P \uparrow Q = \neg\mathsf{T} = \mathsf{F}.
\]
Now apply NAND to $(\mathsf{F})$ and $R = \mathsf{T}$:
\[
    (P \uparrow Q) \uparrow R = \mathsf{F} \uparrow \mathsf{T}
    = \neg(\mathsf{F} \wedge \mathsf{T}) = \neg\mathsf{F} = \mathsf{T}.
\]

\medskip
\textbf{Right-hand side: $P \uparrow (Q \uparrow R)$.}

\ByDef $Q \uparrow R \;\equiv\; \neg(Q \wedge R)$.
Since $Q = R = \mathsf{T}$, we have $Q \wedge R = \mathsf{T}$, so
\[
    Q \uparrow R = \neg\mathsf{T} = \mathsf{F}.
\]
Now apply NAND to $P = \mathsf{T}$ and $(\mathsf{F})$:
\[
    P \uparrow (Q \uparrow R) = \mathsf{T} \uparrow \mathsf{F}
    = \neg(\mathsf{T} \wedge \mathsf{F}) = \neg\mathsf{F} = \mathsf{T}.
\]

\medskip
\textbf{Wait — let us try a different assignment.} Set $P := \mathsf{T}$, $Q := \mathsf{T}$,
$R := \mathsf{F}$.

\medskip
\textbf{Left-hand side: $(P \uparrow Q) \uparrow R$.}

As before, $P \uparrow Q = \mathsf{F}$.
\[
    (P \uparrow Q) \uparrow R = \mathsf{F} \uparrow \mathsf{F}
    = \neg(\mathsf{F} \wedge \mathsf{F}) = \neg\mathsf{F} = \mathsf{T}.
\]

\medskip
\textbf{Right-hand side: $P \uparrow (Q \uparrow R)$.}

$Q \uparrow R = \neg(Q \wedge R) = \neg(\mathsf{T} \wedge \mathsf{F}) = \neg\mathsf{F} = \mathsf{T}$.
\[
    P \uparrow (Q \uparrow R) = \mathsf{T} \uparrow \mathsf{T}
    = \neg(\mathsf{T} \wedge \mathsf{T}) = \neg\mathsf{T} = \mathsf{F}.
\]

\medskip
\textbf{Comparison.}
Under the assignment $P = \mathsf{T}$, $Q = \mathsf{T}$, $R = \mathsf{F}$:
\[
    (P \uparrow Q) \uparrow R = \mathsf{T}
    \qquad\text{but}\qquad
    P \uparrow (Q \uparrow R) = \mathsf{F}.
\]
The two formulas differ in truth value.

\Hence $(P \uparrow Q) \uparrow R \not\equiv P \uparrow (Q \uparrow R)$,
so NAND is not associative. \AsReq
\end{proof}

\begin{remark}[Proof shape]
This is a \emph{disproof by counterexample}. To refute a universal equivalence
$A \equiv B$ it suffices to find one truth-value assignment under which $A$ and
$B$ receive different truth values. One counterexample is logically enough;
exhaustive case analysis is not required.
\end{remark}

\begin{remark}[Why the all-true assignment failed]
Under $P = Q = R = \mathsf{T}$ both sides evaluate to $\mathsf{T}$, so that
assignment is \emph{not} a counterexample. This illustrates a useful caution: when
searching for a counterexample to a proposed equivalence, one should try varied
assignments rather than stopping at the first one checked. The all-false assignment
$P = Q = R = \mathsf{F}$ also yields $\mathsf{T}$ on both sides (both inner NANDs
return $\mathsf{T}$, and $\mathsf{T} \uparrow \mathsf{T} = \mathsf{F}$ on both
sides), so it is not a counterexample either. Mixed assignments are the productive
place to look.
\end{remark}

\begin{remark}[Full truth-table survey]
For completeness, the table below records all eight assignments.
The columns LHS and RHS abbreviate $(P \uparrow Q) \uparrow R$ and
$P \uparrow (Q \uparrow R)$, respectively.

\begin{center}
\renewcommand{\arraystretch}{1.2}
\begin{tabular}{|c|c|c||c|c||c|}
\hline
$P$ & $Q$ & $R$ & LHS & RHS & Equal? \\
\hline
T & T & T & T & T & \checkmark \\
T & T & F & T & F & \texttimes \\
T & F & T & F & T & \texttimes \\
T & F & F & F & T & \texttimes \\
F & T & T & T & T & \checkmark \\
F & T & F & T & T & \checkmark \\
F & F & T & T & T & \checkmark \\
F & F & F & T & T & \checkmark \\
\hline
\end{tabular}
\end{center}

Three of the eight rows witness non-associativity. The five rows where LHS = RHS do
not rescue associativity: a logical equivalence must hold on \emph{all} assignments.
\end{remark}

\begin{remark}[Contrast with XOR]
Exclusive or ($\oplus$) is associative: $(P \oplus Q) \oplus R \equiv P \oplus (Q
\oplus R)$ for all assignments. NAND's non-associativity is therefore not a feature
shared by all non-standard connectives; it is specific to the structure of NAND
(and similarly NOR).
\end{remark}

\begin{remark}[Dependencies]
The proof uses only the definition of NAND ($P \uparrow Q \equiv \neg(P \wedge Q)$),
the truth tables for $\wedge$ and $\neg$, and the semantic criterion for logical
equivalence (agreement on all truth-value assignments).
\end{remark}
