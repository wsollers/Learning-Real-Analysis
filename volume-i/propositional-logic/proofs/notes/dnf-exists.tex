% =========================================================
% Proof: Every Formula Has a DNF
% Source: volume-i/propositional-logic/notes/semantics/notes-normal-forms.tex
% =========================================================

\subsection*{Every Formula Has a DNF}
\label{prf:dnf-exists}

\begin{remark}[Return]
\hyperref[prop:dnf-exists]{$\leftarrow$ Back to Proposition (Every Formula Has a DNF) in Notes}
\end{remark}

\begin{proposition}[Every Formula Has a DNF]
Every propositional formula is logically equivalent to a formula in disjunctive
normal form (DNF).
\end{proposition}

\begin{proof}
\Given The definition of DNF: a formula is in DNF if it is a disjunction of
conjunctive clauses $\bigvee_i \bigwedge_j L_{ij}$, where each $L_{ij}$ is a
literal. The definition of NNF: negations appear only on variables; connectives
are restricted to $\neg$, $\wedge$, $\vee$. The dual distribution law:
$P \wedge (Q \vee R) \;\equiv\; (P \wedge Q) \vee (P \wedge R)$, and its
symmetric form $(P \vee Q) \wedge R \;\equiv\; (P \wedge R) \vee (Q \wedge R)$.
The NNF existence result (\hyperref[prop:nnf-exists]{Every Formula Has an NNF}).

\Goal To show that for every propositional formula $\varphi$ there exists a
formula $\varphi'$ in DNF such that $\varphi \equiv \varphi'$.

\Strategy The proof is the exact dual of the CNF existence proof
(\hyperref[prop:cnf-exists]{Every Formula Has a CNF}), with $\wedge$ and $\vee$
swapped throughout. It proceeds in two stages. \textbf{Stage 1:} apply the NNF
existence theorem to obtain an NNF equivalent $\varphi_{\mathrm{NNF}}$ of
$\varphi$. \textbf{Stage 2:} convert $\varphi_{\mathrm{NNF}}$ to DNF by
structural induction, using the distribution law to push $\wedge$ inside $\vee$
whenever a conjunction sits above a disjunction in the parse tree. Both stages
preserve logical equivalence, so the final result is equivalent to $\varphi$.

\medskip
\noindent\textbf{Stage 1: Obtain an NNF equivalent.}

By \hyperref[prop:nnf-exists]{Every Formula Has an NNF}, there exists a formula
$\varphi_{\mathrm{NNF}}$ in NNF such that $\varphi \equiv
\varphi_{\mathrm{NNF}}$. It suffices to show that $\varphi_{\mathrm{NNF}}$ is
equivalent to some DNF formula; the full result then follows by transitivity of
$\equiv$.

\medskip
\noindent\textbf{Stage 2: Convert the NNF formula to DNF.}

We prove by structural induction on $\varphi_{\mathrm{NNF}}$ that every NNF
formula is equivalent to some DNF formula.

\medskip
\noindent\textit{Base cases.}

\medskip
\textit{Case 1: $\varphi_{\mathrm{NNF}} = P$ (a propositional variable).}
$P$ is a single positive literal. A single literal is a degenerate conjunctive
clause ($k = 1$), and a single clause is a degenerate disjunction ($m = 1$).
Hence $P$ is already in DNF.

\medskip
\textit{Case 2: $\varphi_{\mathrm{NNF}} = \neg P$ (negation of a variable).}
$\neg P$ is a single negative literal, and the same reasoning as Case 1 applies.
$\neg P$ is already in DNF.

\medskip
\noindent\textit{Inductive step.}

Assume (induction hypothesis) that every NNF formula of strictly smaller
complexity than $\varphi_{\mathrm{NNF}}$ is equivalent to some DNF formula. We
consider the two compound cases that can arise in an NNF formula ($\rightarrow$
and $\leftrightarrow$ are absent).

\medskip
\textit{Case 3: $\varphi_{\mathrm{NNF}} = \alpha \vee \beta$.}

By the induction hypothesis, there exist DNF formulas $\alpha'$ and $\beta'$
with $\alpha \equiv \alpha'$ and $\beta \equiv \beta'$. Write
\[
    \alpha' = T_1 \vee T_2 \vee \cdots \vee T_m, \qquad
    \beta'  = U_1 \vee U_2 \vee \cdots \vee U_n,
\]
where each $T_i$ and $U_j$ is a conjunctive clause. Then
\[
    \alpha \vee \beta \;\equiv\; \alpha' \vee \beta'
    \;=\; T_1 \vee \cdots \vee T_m \vee U_1 \vee \cdots \vee U_n.
\]
This is a disjunction of conjunctive clauses, hence in DNF. Take $\varphi' =
T_1 \vee \cdots \vee T_m \vee U_1 \vee \cdots \vee U_n$.

\medskip
\textit{Case 4: $\varphi_{\mathrm{NNF}} = \alpha \wedge \beta$.}

By the induction hypothesis, there exist DNF formulas $\alpha'$ and $\beta'$
with $\alpha \equiv \alpha'$ and $\beta \equiv \beta'$. Write
\[
    \alpha' = T_1 \vee T_2 \vee \cdots \vee T_m, \qquad
    \beta'  = U_1 \vee U_2 \vee \cdots \vee U_n,
\]
where each $T_i$ and $U_j$ is a conjunctive clause. Then
\[
    \alpha \wedge \beta \;\equiv\; \alpha' \wedge \beta'
    \;=\; (T_1 \vee \cdots \vee T_m) \wedge (U_1 \vee \cdots \vee U_n).
\]
This is a conjunction of two disjunctions, which is not yet in DNF. We apply
the dual distribution law repeatedly. Distributing $\wedge$ over $\vee$ across
all pairs $(T_i, U_j)$ gives:
\[
    \alpha' \wedge \beta' \;\equiv\;
    \bigvee_{i=1}^{m} \bigvee_{j=1}^{n} (T_i \wedge U_j).
\]
Each factor $T_i \wedge U_j$ is a conjunction of two conjunctive clauses, hence
itself a conjunctive clause (conjunction is associative and flattens). The
result is a disjunction of $mn$ conjunctive clauses, hence in DNF. Take
$\varphi' = \bigvee_{i,j}(T_i \wedge U_j)$.

\medskip
In every case an equivalent DNF formula exists. By structural induction, every
NNF formula is equivalent to some DNF formula. Combined with Stage 1, every
propositional formula is equivalent to some DNF formula. \AsReq
\end{proof}

\begin{remark}[Proof shape]
The proof is structurally identical to the CNF existence proof
(\hyperref[prf:cnf-exists]{Every Formula Has a CNF}), with $\wedge$ and $\vee$
— and conjunctive and disjunctive clauses — swapped throughout. This
duality is not accidental: CNF and DNF are De Morgan duals of each other.
The two-stage structure (NNF first, then distribute) is the same; only the
direction of distribution differs. In CNF we push $\vee$ inside $\wedge$; in
DNF we push $\wedge$ inside $\vee$.
\end{remark}

\begin{remark}[The duality with CNF made explicit]
If $\varphi'$ is the CNF of $\varphi$, then $\neg\varphi'$ — obtained by
applying De Morgan to every clause — is the DNF of $\neg\varphi$. This means
that CNF and DNF are related by negation: computing one gives the other for
free via De Morgan's laws. In practice this means that any algorithm for CNF
conversion can be adapted to DNF conversion simply by dualising.
\end{remark}

\begin{remark}[Exponential size blowup]
The distribution step in Case 4 replaces $m + n$ clauses with $mn$ clauses.
As in the CNF case, repeated applications across a formula of depth $d$ can
yield a DNF with exponentially many clauses. This blowup is unavoidable in the
worst case for logical equivalence; compact representations (BDDs, decision
trees) exist at the cost of a different normal form altogether.
\end{remark}

\begin{remark}[Common error]
The most frequent mistake is conflating the conjunction and disjunction cases.
In DNF, the top-level connective is $\vee$ (disjunction of terms), so the
\emph{easy} case is $\alpha \vee \beta$ (concatenate term lists) and the
\emph{hard} case is $\alpha \wedge \beta$ (distribute). This is the reverse of
the CNF situation, where $\wedge$ is easy and $\vee$ is hard. Mixing up which
case requires distribution is the most common source of error when working
through both proofs in the same session.
\end{remark}

\begin{remark}[Dependencies]
The proof depends on: the definition of DNF (\hyperref[def:dnf]{Def.\ DNF}),
the definition of NNF (\hyperref[def:nnf]{Def.\ NNF}),
\hyperref[prop:nnf-exists]{Every Formula Has an NNF},
the dual distribution law ($P \wedge (Q \vee R) \equiv (P \wedge Q) \vee (P
\wedge R)$), associativity of $\wedge$ (to flatten clause conjunctions), and
the principle of structural induction on propositional formulas.
\end{remark}
