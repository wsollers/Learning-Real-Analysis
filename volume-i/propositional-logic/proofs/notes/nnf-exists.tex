% =========================================================
% Proof: Every Formula Has an NNF
% Source: volume-i/propositional-logic/notes/semantics/notes-normal-forms.tex
% =========================================================

\subsection*{Every Formula Has an NNF}
\label{prf:nnf-exists}

\begin{remark}[Return]
\hyperref[prop:nnf-exists]{$\leftarrow$ Back to Proposition (Every Formula Has an NNF) in Notes}
\end{remark}

\begin{proposition}[Every Formula Has an NNF]
Every propositional formula is logically equivalent to a formula in negation
normal form (NNF).
\end{proposition}

\begin{proof}
\Given The definition of NNF: a formula is in NNF if (i) negation is applied
only to propositional variables, and (ii) the only connectives present are
$\neg$, $\wedge$, $\vee$. The standard logical equivalences: double negation
($\neg\neg P \equiv P$), De Morgan's laws ($\neg(P \wedge Q) \equiv \neg P
\vee \neg Q$, $\neg(P \vee Q) \equiv \neg P \wedge \neg Q$), and the
eliminations $P \rightarrow Q \equiv \neg P \vee Q$ and $P \leftrightarrow Q
\equiv (P \wedge Q) \vee (\neg P \wedge \neg Q)$.

\Goal To show that for every propositional formula $\varphi$ there exists a
formula $\varphi'$ in NNF such that $\varphi \equiv \varphi'$.

\Strategy We proceed by structural induction on $\varphi$. The base cases
cover atomic formulas (variables and their negations), which are already in
NNF. The inductive step shows that if each subformula can be converted to NNF,
then the whole formula can be converted, using the equivalences above to
eliminate $\rightarrow$ and $\leftrightarrow$ and to push any remaining
$\neg$ inward until it rests on variables alone.

\medskip
\noindent\textbf{Base cases.}

\medskip
\textit{Case 1: $\varphi = P$ (a propositional variable).}
$P$ is a positive literal and therefore already in NNF. Take $\varphi' = P$.

\medskip
\textit{Case 2: $\varphi = \neg P$ (negation of a variable).}
$\neg P$ is a negative literal and therefore already in NNF. Take $\varphi' = \neg P$.

\medskip
\noindent\textbf{Inductive step.}

Suppose the claim holds for all proper subformulas of $\varphi$ (induction
hypothesis): each subformula is equivalent to some NNF formula. We show
$\varphi$ itself is equivalent to an NNF formula by cases on the outermost
connective.

\medskip
\textit{Case 3: $\varphi = \alpha \wedge \beta$.}
By the induction hypothesis, there exist NNF formulas $\alpha'$ and $\beta'$
with $\alpha \equiv \alpha'$ and $\beta \equiv \beta'$. Then
\[
    \alpha \wedge \beta \;\equiv\; \alpha' \wedge \beta'.
\]
Since $\alpha'$ and $\beta'$ are in NNF and $\wedge$ is a permitted
connective, $\alpha' \wedge \beta'$ is in NNF. Take $\varphi' = \alpha'
\wedge \beta'$.

\medskip
\textit{Case 4: $\varphi = \alpha \vee \beta$.}
Identical to Case 3 with $\vee$ in place of $\wedge$. Take $\varphi' =
\alpha' \vee \beta'$.

\medskip
\textit{Case 5: $\varphi = \alpha \rightarrow \beta$.}
Eliminate the implication:
\[
    \alpha \rightarrow \beta \;\equiv\; \neg\alpha \vee \beta.
\]
Now $\neg\alpha$ is a formula of strictly smaller complexity than $\varphi$,
so by the induction hypothesis applied to $\neg\alpha$ and $\beta$, there
exist NNF formulas $(\neg\alpha)'$ and $\beta'$ equivalent to them. Then
\[
    \alpha \rightarrow \beta \;\equiv\; (\neg\alpha)' \vee \beta',
\]
which is in NNF. (The inductive hypothesis applies to $\neg\alpha$ because
the conversion of $\neg\alpha$ to NNF proceeds by the same structural
induction — see Case 6.)

\medskip
\textit{Case 6: $\varphi = \neg\psi$ (negation of a compound formula).}
We sub-case on the outermost connective of $\psi$.

\begin{itemize}
  \item \textit{$\psi = \neg\chi$ (double negation):}
        $\neg\neg\chi \equiv \chi$. Apply the induction hypothesis to $\chi$
        to get an NNF formula $\chi' \equiv \chi$. Take $\varphi' = \chi'$.

  \item \textit{$\psi = \alpha \wedge \beta$:}
        By De Morgan, $\neg(\alpha \wedge \beta) \equiv \neg\alpha \vee
        \neg\beta$. Apply the induction hypothesis to $\neg\alpha$ and
        $\neg\beta$ (both strictly smaller than $\varphi$) to get NNF
        formulas $(\neg\alpha)' \equiv \neg\alpha$ and $(\neg\beta)' \equiv
        \neg\beta$. Take $\varphi' = (\neg\alpha)' \vee (\neg\beta)'$.

  \item \textit{$\psi = \alpha \vee \beta$:}
        By De Morgan, $\neg(\alpha \vee \beta) \equiv \neg\alpha \wedge
        \neg\beta$. Proceed as in the previous sub-case with $\wedge$ in
        place of $\vee$.

  \item \textit{$\psi = \alpha \rightarrow \beta$:}
        $\neg(\alpha \rightarrow \beta) \equiv \neg(\neg\alpha \vee \beta)
        \equiv \alpha \wedge \neg\beta$. Apply the induction hypothesis to
        $\alpha$ and $\neg\beta$ to get NNF equivalents. Take $\varphi' =
        \alpha' \wedge (\neg\beta)'$.

  \item \textit{$\psi = \alpha \leftrightarrow \beta$:}
        $\neg(\alpha \leftrightarrow \beta) \equiv \neg\bigl((P \wedge Q)
        \vee (\neg P \wedge \neg Q)\bigr)$. Apply De Morgan and double
        negation to reduce fully to $\neg$, $\wedge$, $\vee$ on subformulas
        of strictly smaller complexity, then apply the induction hypothesis
        to each.
\end{itemize}

\medskip
\textit{Case 7: $\varphi = \alpha \leftrightarrow \beta$.}
Eliminate the biconditional:
\[
    \alpha \leftrightarrow \beta \;\equiv\; (\alpha \wedge \beta) \vee
    (\neg\alpha \wedge \neg\beta).
\]
Apply the induction hypothesis to $\alpha$, $\beta$, $\neg\alpha$,
$\neg\beta$ to obtain NNF equivalents, combine with $\wedge$ and $\vee$
(permitted connectives), and take the result as $\varphi'$.

\medskip
In every case, $\varphi'$ is in NNF and $\varphi \equiv \varphi'$. By
structural induction, the claim holds for all propositional formulas. \AsReq
\end{proof}

\begin{remark}[Proof shape]
The proof is by \emph{structural induction} on the formula $\varphi$. The
base cases are the two kinds of literal (positive and negative), both of
which are already in NNF. The inductive step handles each connective
separately: $\wedge$ and $\vee$ are preserved directly; $\rightarrow$ and
$\leftrightarrow$ are eliminated by standard equivalences; and $\neg$
applied to a compound formula is pushed inward using De Morgan's laws and
double-negation elimination until it reaches variables. The induction
terminates because each rewriting step reduces formula complexity: De Morgan
distributes $\neg$ to strictly smaller subformulas, double-negation
elimination removes two layers of $\neg$, and implication/biconditional
elimination replaces the formula with subformulas of smaller connective
count.
\end{remark}

\begin{remark}[Why the induction is well-founded]
Structural induction on propositional formulas is well-founded because the
grammar is defined inductively: every compound formula is built from strictly
shorter subformulas. There is no circular construction; the base cases
(variables, their negations) have no subformulas, so the induction always
reaches ground.
\end{remark}

\begin{remark}[The result is constructive]
The proof does not merely assert that an NNF equivalent exists — it
\emph{constructs} one by providing the explicit rewriting procedure: (1)
eliminate $\rightarrow$ and $\leftrightarrow$; (2) push $\neg$ inward via De
Morgan; (3) cancel double negations. Each step preserves logical equivalence,
so the result is sound. The procedure terminates because every application of
De Morgan strictly decreases the nesting depth of $\neg$.
\end{remark}

\begin{remark}[Common error]
A frequent mistake is to attempt the proof by cases on the outermost
connective without recognising that the $\neg$ case requires its own
sub-induction on what is being negated. Treating $\neg\psi$ as a single case
without splitting on the structure of $\psi$ leaves the double-negation and
De Morgan steps unjustified.
\end{remark}

\begin{remark}[Dependencies]
The proof depends on: the definition of NNF (\hyperref[def:nnf]{Def.\ NNF}),
the principle of structural induction on propositional formulas, the
double-negation law ($\neg\neg P \equiv P$), De Morgan's laws ($\neg(P
\wedge Q) \equiv \neg P \vee \neg Q$ and $\neg(P \vee Q) \equiv \neg P
\wedge \neg Q$), and the equivalences eliminating $\rightarrow$ and
$\leftrightarrow$.
\end{remark}
