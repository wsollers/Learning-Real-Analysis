% =========================================================
% Proof: Non-Associativity of NOR (Peirce Arrow)
% Source: volume-i/propositional-logic/notes/syntax/notes-connectives.tex
% =========================================================

\subsection*{Non-Associativity of NOR}
\label{prf:nor-non-associativity}

\begin{remark}[Return]
\hyperref[prop:nor-complete]{$\leftarrow$ Back to Proposition (Non-Associativity of NOR) in Notes}
\end{remark}

\begin{proof}
\Claim NOR is not associative: $(P \downarrow Q) \downarrow R \;\not\equiv\; P \downarrow (Q \downarrow R)$.

\Given The definition $P \downarrow Q \;\equiv\; \neg(P \vee Q)$.
\Goal To show that $(P \downarrow Q) \downarrow R \not\equiv P \downarrow (Q \downarrow R)$,
i.e.\ that the two sides differ in truth value for at least one assignment.

\Strategy We exhibit a counterexample: a single truth-value assignment to $P$, $Q$, $R$
under which the left-hand side and right-hand side take opposite truth values.

\medskip
\textbf{Assignment.} Set $P := \mathsf{F}$, $Q := \mathsf{F}$, $R := \mathsf{T}$.

\medskip
\textbf{Left-hand side: $(P \downarrow Q) \downarrow R$.}

\ByDef $P \downarrow Q \;\equiv\; \neg(P \vee Q)$.
Since $P = Q = \mathsf{F}$, we have $P \vee Q = \mathsf{F}$, so
\[
    P \downarrow Q = \neg\mathsf{F} = \mathsf{T}.
\]
Now apply NOR to $(\mathsf{T})$ and $R = \mathsf{T}$:
\[
    (P \downarrow Q) \downarrow R = \mathsf{T} \downarrow \mathsf{T}
    = \neg(\mathsf{T} \vee \mathsf{T}) = \neg\mathsf{T} = \mathsf{F}.
\]

\medskip
\textbf{Right-hand side: $P \downarrow (Q \downarrow R)$.}

\ByDef $Q \downarrow R \;\equiv\; \neg(Q \vee R)$.
Since $Q = \mathsf{F}$ and $R = \mathsf{T}$, we have $Q \vee R = \mathsf{T}$, so
\[
    Q \downarrow R = \neg\mathsf{T} = \mathsf{F}.
\]
Now apply NOR to $P = \mathsf{F}$ and $(\mathsf{F})$:
\[
    P \downarrow (Q \downarrow R) = \mathsf{F} \downarrow \mathsf{F}
    = \neg(\mathsf{F} \vee \mathsf{F}) = \neg\mathsf{F} = \mathsf{T}.
\]

\medskip
\textbf{Comparison.}
Under the assignment $P = \mathsf{F}$, $Q = \mathsf{F}$, $R = \mathsf{T}$:
\[
    (P \downarrow Q) \downarrow R = \mathsf{F}
    \qquad\text{but}\qquad
    P \downarrow (Q \downarrow R) = \mathsf{T}.
\]
The two formulas differ in truth value.

\Hence $(P \downarrow Q) \downarrow R \not\equiv P \downarrow (Q \downarrow R)$,
so NOR is not associative. \AsReq
\end{proof}

\begin{remark}[Proof shape]
This is a \emph{disproof by counterexample}. To refute a universal equivalence
$A \equiv B$ it suffices to find one truth-value assignment under which $A$ and
$B$ receive different truth values. One counterexample is logically sufficient;
exhaustive case analysis is not required.
\end{remark}

\begin{remark}[Why all-false fails]
Under $P = Q = R = \mathsf{F}$, both inner NORs yield $\mathsf{T}$, so both
sides evaluate to $\neg(\mathsf{T} \vee \mathsf{F}) = \mathsf{F}$ on the left
and $\neg(\mathsf{F} \vee \mathsf{T}) = \mathsf{F}$ on the right — they agree.
All-false is therefore not a counterexample. This illustrates why one must try
varied assignments rather than stopping at the first case checked. Because NOR
produces $\mathsf{T}$ only when \emph{both} inputs are false, the most
productive assignments are those where the inner NOR fires (producing
$\mathsf{T}$), creating an asymmetry between the two groupings.
\end{remark}

\begin{remark}[Full truth-table survey]
For completeness, the table below records all eight assignments.
The columns LHS and RHS abbreviate $(P \downarrow Q) \downarrow R$ and
$P \downarrow (Q \downarrow R)$, respectively.

\begin{center}
\renewcommand{\arraystretch}{1.2}
\begin{tabular}{|c|c|c||c|c||c|}
\hline
$P$ & $Q$ & $R$ & LHS & RHS & Equal? \\
\hline
T & T & T & F & F & \checkmark \\
T & T & F & F & F & \checkmark \\
T & F & T & F & F & \checkmark \\
T & F & F & F & T & \texttimes \\
F & T & T & F & F & \checkmark \\
F & T & F & F & T & \texttimes \\
F & F & T & F & T & \texttimes \\
F & F & F & T & T & \checkmark \\
\hline
\end{tabular}
\end{center}

Three of the eight rows witness non-associativity. The five rows where LHS = RHS do
not rescue associativity: a logical equivalence must hold on \emph{all} assignments.
\end{remark}

\begin{remark}[Duality with NAND]
NOR is the De Morgan dual of NAND. The non-associativity of NOR mirrors that of NAND
exactly: both fail associativity for the same structural reason — the inner
connective can produce $\mathsf{T}$ in one grouping but not the other, and the outer
connective then sees a different input. The counterexample assignments differ
(NAND uses a mixed assignment with at least one $\mathsf{T}$ input; NOR uses a mostly
$\mathsf{F}$ assignment) but the logical shape of the argument is identical.
\end{remark}

\begin{remark}[Dependencies]
The proof uses only the definition of NOR ($P \downarrow Q \equiv \neg(P \vee Q)$),
the truth tables for $\vee$ and $\neg$, and the semantic criterion for logical
equivalence (agreement on all truth-value assignments).
\end{remark}
