% =========================================================
% Proof: Every Formula Has a CNF
% Source: volume-i/propositional-logic/notes/semantics/notes-normal-forms.tex
% =========================================================

\subsection*{Every Formula Has a CNF}
\label{prf:cnf-exists}

\begin{remark}[Return]
\hyperref[prop:cnf-exists]{$\leftarrow$ Back to Proposition (Every Formula Has a CNF) in Notes}
\end{remark}

\begin{proof}
\Claim Every propositional formula is logically equivalent to a formula in conjunctive normal form (CNF).

\Given The definition of CNF: a formula is in CNF if it is a conjunction of
disjunctive clauses $\bigwedge_i \bigvee_j L_{ij}$, where each $L_{ij}$ is a
literal. The definition of NNF: negations appear only on variables; connectives
are restricted to $\neg$, $\wedge$, $\vee$. The distribution law:
$P \vee (Q \wedge R) \;\equiv\; (P \vee Q) \wedge (P \vee R)$, and its
symmetric form $(P \wedge Q) \vee R \;\equiv\; (P \vee R) \wedge (Q \vee R)$.
The NNF existence result (\hyperref[prop:nnf-exists]{Every Formula Has an NNF}).

\Goal To show that for every propositional formula $\varphi$ there exists a
formula $\varphi'$ in CNF such that $\varphi \equiv \varphi'$.

\Strategy The proof proceeds in two stages. \textbf{Stage 1:} apply the NNF
existence theorem to obtain an NNF equivalent $\varphi_{\mathrm{NNF}}$ of
$\varphi$. \textbf{Stage 2:} convert $\varphi_{\mathrm{NNF}}$ to CNF by
structural induction, using the distribution law to push $\vee$ inside $\wedge$
whenever a disjunction sits above a conjunction in the parse tree. Since Stage 1
preserves logical equivalence and Stage 2 preserves logical equivalence at every
step, the final result is equivalent to $\varphi$.

\medskip
\noindent\textbf{Stage 1: Obtain an NNF equivalent.}

By \hyperref[prop:nnf-exists]{Every Formula Has an NNF}, there exists a formula
$\varphi_{\mathrm{NNF}}$ in NNF such that $\varphi \equiv
\varphi_{\mathrm{NNF}}$. It suffices to show that $\varphi_{\mathrm{NNF}}$ is
equivalent to some CNF formula; the full result then follows by transitivity of
$\equiv$.

\medskip
\noindent\textbf{Stage 2: Convert the NNF formula to CNF.}

We prove by structural induction on $\varphi_{\mathrm{NNF}}$ that every NNF
formula is equivalent to some CNF formula.

\medskip
\noindent\textit{Base cases.}

\medskip
\textit{Case 1: $\varphi_{\mathrm{NNF}} = P$ (a propositional variable).}
$P$ is a single positive literal. A single literal is a degenerate disjunctive
clause ($k = 1$), and a single clause is a degenerate conjunction ($m = 1$).
Hence $P$ is already in CNF.

\medskip
\textit{Case 2: $\varphi_{\mathrm{NNF}} = \neg P$ (negation of a variable).}
$\neg P$ is a single negative literal, and the same reasoning as Case 1 applies.
$\neg P$ is already in CNF.

\medskip
\noindent\textit{Inductive step.}

Assume (induction hypothesis) that every NNF formula of strictly smaller
complexity than $\varphi_{\mathrm{NNF}}$ is equivalent to some CNF formula. We
consider the two compound cases that can arise in an NNF formula ($\rightarrow$
and $\leftrightarrow$ are absent).

\medskip
\textit{Case 3: $\varphi_{\mathrm{NNF}} = \alpha \wedge \beta$.}

By the induction hypothesis, there exist CNF formulas $\alpha'$ and $\beta'$
with $\alpha \equiv \alpha'$ and $\beta \equiv \beta'$. Write
\[
    \alpha' = C_1 \wedge C_2 \wedge \cdots \wedge C_m, \qquad
    \beta'  = D_1 \wedge D_2 \wedge \cdots \wedge D_n,
\]
where each $C_i$ and $D_j$ is a disjunctive clause. Then
\[
    \alpha \wedge \beta \;\equiv\; \alpha' \wedge \beta'
    \;=\; C_1 \wedge \cdots \wedge C_m \wedge D_1 \wedge \cdots \wedge D_n.
\]
This is a conjunction of disjunctive clauses, hence in CNF. Take $\varphi' =
C_1 \wedge \cdots \wedge C_m \wedge D_1 \wedge \cdots \wedge D_n$.

\medskip
\textit{Case 4: $\varphi_{\mathrm{NNF}} = \alpha \vee \beta$.}

By the induction hypothesis, there exist CNF formulas $\alpha'$ and $\beta'$
with $\alpha \equiv \alpha'$ and $\beta \equiv \beta'$. Write
\[
    \alpha' = C_1 \wedge C_2 \wedge \cdots \wedge C_m, \qquad
    \beta'  = D_1 \wedge D_2 \wedge \cdots \wedge D_n.
\]
Then
\[
    \alpha \vee \beta \;\equiv\; \alpha' \vee \beta'
    \;=\; (C_1 \wedge \cdots \wedge C_m) \vee (D_1 \wedge \cdots \wedge D_n).
\]
This is a disjunction of two conjunctions, which is not yet in CNF. We apply
the distribution law repeatedly. For each pair $(C_i, D_j)$, distributing
$\vee$ over $\wedge$ gives:
\[
    \alpha' \vee \beta' \;\equiv\;
    \bigwedge_{i=1}^{m} \bigwedge_{j=1}^{n} (C_i \vee D_j).
\]
Each factor $C_i \vee D_j$ is a disjunction of two disjunctive clauses, hence
itself a disjunctive clause (disjunction is associative and flattens). The
result is a conjunction of $mn$ disjunctive clauses, hence in CNF. Take
$\varphi' = \bigwedge_{i,j}(C_i \vee D_j)$.

\medskip
In every case an equivalent CNF formula exists. By structural induction, every
NNF formula is equivalent to some CNF formula. Combined with Stage 1, every
propositional formula is equivalent to some CNF formula. \AsReq
\end{proof}

\begin{remark}[Proof shape]
The proof is a \emph{two-stage reduction}. Stage 1 is a direct appeal to the
already-proved NNF existence theorem: we do not reprove that result here, but
inherit it. Stage 2 is a structural induction on NNF formulas. The base cases
(literals) are trivial; the conjunction case is immediate (concatenate clause
lists); the hard case is disjunction, which requires distributing $\vee$ over
$\wedge$ to flatten a disjunction of conjunctions into a conjunction of
disjunctions. This distribution step is the logical core of the proof and is
the step that can cause exponential blowup in formula size.
\end{remark}

\begin{remark}[Why NNF is an intermediate step]
The NNF stage eliminates $\rightarrow$ and $\leftrightarrow$ and pushes $\neg$
to the leaves. This ensures that in Stage 2 the only connectives present are
$\wedge$ and $\vee$ on compound subformulas, and $\neg$ only on variables.
Without NNF as an intermediate form, the induction in Stage 2 would require
handling all five connectives rather than just two, making the case analysis
considerably more involved.
\end{remark}

\begin{remark}[Exponential size blowup]
The distribution step in Case 4 replaces a formula of size $m + n$ clauses with
one of size $mn$ clauses. Repeated applications across a formula of depth $d$
can yield a CNF with exponentially many clauses. This is not a defect in the
proof but a genuine feature of the CNF representation: some formulas have no
polynomial-size CNF equivalent (unless $\mathrm{P} = \mathrm{NP}$). The
Tseitin transformation avoids this blowup by introducing auxiliary variables,
but at the cost of producing a formula that is only \emph{equisatisfiable}
rather than logically equivalent.
\end{remark}

\begin{remark}[Common error]
A frequent mistake is to attempt to convert directly to CNF without first
going through NNF. This fails because distributing $\vee$ over $\wedge$ in the
presence of $\neg(\alpha \wedge \beta)$ or $\rightarrow$ produces intermediate
forms that are neither CNF nor NNF, making the induction ill-defined. The NNF
stage is not just convenient — it is what makes the induction in Stage 2 clean
and well-founded.
\end{remark}

\begin{remark}[Dependencies]
The proof depends on: the definition of CNF (\hyperref[def:cnf]{Def.\ CNF}),
the definition of NNF (\hyperref[def:nnf]{Def.\ NNF}),
\hyperref[prop:nnf-exists]{Every Formula Has an NNF},
the distribution law ($P \vee (Q \wedge R) \equiv (P \vee Q) \wedge (P \vee
R)$), associativity of $\vee$ (to flatten clause disjunctions), and the
principle of structural induction on propositional formulas.
\end{remark}
