% ======================================
% Proof: PL-BJO-C01-E02a
% ======================================
\clearpage
\phantomsection
\hypertarget{proof-PL-BJO-C01-E02a}{}

\section*{Proof — PL-BJO-C01-E02(a)}

}

\noindent
\textbf{Source.}
Adam Bjørndahl,
\textit{An Introduction to Classical and Modal Logics: The Outlines of Knowledge},
Chapter 1, Exercise 2(a).

\vspace{0.5em}

\noindent
\textbf{Statement.}
Provide a recursive definition of the function $\mathrm{conn}(\varphi)$ returning the
number of connectives occurring in $\varphi$ (counting duplicates).

\vspace{1em}

\noindent



\begin{definition}
Let $\mathcal{L}(\neg,\wedge,\vee,\to)$ be the set of all well-formed propositional
formulas over the connectives $\neg$, $\wedge$, $\vee$, and $\to$. Atomic formulas
are propositional variables.

Define a function
\[
\mathrm{conn} \colon \mathcal{L}(\neg,\wedge,\vee,\to) \to \mathbb{N}
\]
recursively as follows:
\begin{align*}
\text{If } \varphi \text{ is an atomic formula,}
&\quad \mathrm{conn}(\varphi)=0. \\[0.75em]
\text{If } \varphi = \neg \psi,
&\quad \mathrm{conn}(\varphi)=1+\mathrm{conn}(\psi). \\[0.75em]
\text{If } \varphi = (\psi \circ \chi) \text{ where } \circ \in \{\wedge,\vee,\to\},
&\quad \mathrm{conn}(\varphi)=1+\mathrm{conn}(\psi)+\mathrm{conn}(\chi).
\end{align*}
\end{definition}





\vspace{1em}

\noindent
\textbf{Remarks.}
\begin{itemize}
  \item conn is a recursively defined function that counts the number of logical operators.
  \item Proof strategy:
  \item Logical form:
  \item Mistakes to avoid:
\end{itemize}

\vspace{1em}

}
