% ======================================
% Proof: PL-SUP-C02-S2.2-P04
% ======================================
\clearpage
\phantomsection
\hypertarget{proof-PL-SUP-C02-S2.2-P04}{}

\section*{Proof — PL-SUP-C02-S2.2-P04}

}

\noindent
\textbf{Source.}
\srccite{SuppesHillFirstCourseLogic}{Chapter~2, \S2}.

\vspace{0.75em}

\noindent
\textbf{Goal.} Prove $M \vee N$.

\vspace{0.75em}

\noindent
\textbf{Given (Premises).}
\[
\begin{array}{r l l}
(1) & \neg J \rightarrow (M \vee N) & \text{P} \\
(2) & (F \vee G) \rightarrow \neg J & \text{P} \\
(3) & F \vee G & \text{P} \\
\end{array}
\]

\vspace{0.75em}

\noindent
\textbf{Proof (stub).}

\[\begin{array}
{r l l}
(1) & \neg J \rightarrow (M \vee N) & \text{P} \\
(2) & (F \vee G) \rightarrow \neg J & \text{P} \\
(3) & F \vee G & \text{P} \\
(4) & \neg J & \text{MP} \\
(5) & (M \vee N) & \text{MP} \\

\end{array}
\]

\vspace{0.75em}

\noindent
\textbf{Remarks.}
\begin{itemize}
  \item Target: $M \vee N$.
  \item Look for an intermediate statement that triggers detachment into the goal.
\end{itemize}

\vspace{1em}

}