% =========================================================
% Equivalence Classes and Partitions
% =========================================================

\subsection{Equivalence Classes and Partitions}

% ---------------------------------------------------------
% TOOLKIT
% ---------------------------------------------------------
\begin{tcolorbox}[colback=gray!6, colframe=gray!40, arc=2pt,
  left=6pt, right=6pt, top=4pt, bottom=4pt,
  title={\small\textbf{Equivalence and Partitions — Quick Reference}},
  fonttitle=\small\bfseries]
\small
\begin{tabular}{l l l}
\toprule
\textbf{Concept} & \textbf{Meaning} & \textbf{Detail} \\
\midrule
Equivalence class  & $[a]_R = \{x \in A \mid (a,x)\in R\}$                        & \hyperref[def:equiv-class]{↓ Def} \\
Quotient set       & $A/R$: all equivalence classes                                & \hyperref[def:quotient-set]{↓ Def} \\
Index of $R$       & Cardinality of $A/R$                                          & \hyperref[def:index]{↓ Def} \\
Canonical surjection & $\pi : A \to A/R$, $\pi(a) = [a]$                          & \hyperref[def:canonical-surj]{↓ Def} \\
Partition          & Nonempty, disjoint, covering collection of subsets            & \hyperref[def:partition]{↓ Def} \\
Rep.\ independence & $[a]=[b] \iff (a,b)\in R$                                    & \hyperref[lem:rep-independence]{↓ Lem} \\
Equiv.--partition correspondence & Bijection between equiv.\ relations and partitions & \hyperref[thm:equiv-partition]{↓ Thm} \\
\bottomrule
\end{tabular}
\end{tcolorbox}

\vspace{1em}

\begin{tcolorbox}[colback=propbox, colframe=propborder, arc=2pt,
  left=6pt, right=6pt, top=4pt, bottom=4pt,
  title={\small\textbf{Definition (Equivalence Class)}},
  fonttitle=\small\bfseries]
\label{def:equiv-class}
Let $R$ be an equivalence relation on $A$. For $a \in A$, the
\emph{equivalence class} of $a$ is:
\[
[a]_R \;:=\; \{\, x \in A \mid (a,x) \in R \,\}.
\]
When $R$ is clear from context, we write $[a]$.
\end{tcolorbox}

\begin{remark}[English reading]
$[a]_R$ is the set of all elements that $R$ declares ``the same as $a$.''
Two elements lie in the same class iff they are related: this is precisely
the Representative Independence Lemma below.
\end{remark}

\begin{tcolorbox}[colback=propbox, colframe=propborder, arc=2pt,
  left=6pt, right=6pt, top=4pt, bottom=4pt,
  title={\small\textbf{Definition (Quotient Set)}},
  fonttitle=\small\bfseries]
\label{def:quotient-set}
The \emph{quotient set} of $A$ by $R$ is the set of all equivalence classes:
\[
A / R \;:=\; \{\, [a]_R \mid a \in A \,\}.
\]
\end{tcolorbox}

\begin{definition}[Index of an Equivalence Relation]\label{def:index}
The \emph{index} of $R$ on $A$ is the cardinality $|A/R|$, i.e.\ the number
of equivalence classes.
\end{definition}

\begin{tcolorbox}[colback=propbox, colframe=propborder, arc=2pt,
  left=6pt, right=6pt, top=4pt, bottom=4pt,
  title={\small\textbf{Definition (Canonical Surjection)}},
  fonttitle=\small\bfseries]
\label{def:canonical-surj}
The \emph{canonical surjection} (quotient map) is the function
\[
\pi : A \to A/R,
\qquad
\pi(a) := [a].
\]
\end{tcolorbox}

\begin{remark}[Properties of $\pi$]
$\pi$ is surjective by construction. Elements $a, b \in A$ satisfy
$\pi(a) = \pi(b)$ iff $(a,b) \in R$. The canonical surjection is the
prototype for all quotient constructions: it reappears as the quotient
homomorphism in algebra and the quotient map in topology.
\end{remark}

\begin{tcolorbox}[colback=propbox, colframe=propborder, arc=2pt,
  left=6pt, right=6pt, top=4pt, bottom=4pt,
  title={\small\textbf{Definition (Partition)}},
  fonttitle=\small\bfseries]
\label{def:partition}
A \emph{partition} of $A$ is a collection $\mathcal{P}$ of subsets of $A$ such
that:
\begin{enumerate}[label=(\roman*)]
\item every block is nonempty: $\forall P \in \mathcal{P},\; P \neq \varnothing$;
\item distinct blocks are disjoint:
$\forall P,Q \in \mathcal{P},\; P \neq Q \rightarrow P \cap Q = \varnothing$;
\item the blocks cover $A$: $\bigcup_{P \in \mathcal{P}} P = A$.
\end{enumerate}
The sets $P \in \mathcal{P}$ are called the \emph{blocks} of the partition.
\end{tcolorbox}

\begin{remark}[Partition vs.\ cover]
Every partition of $A$ is a cover of $A$ whose members are nonempty and
pairwise disjoint. Partitions are precisely the covers satisfying the
disjointness condition.
\end{remark}

% ---------------------------------------------------------
% Core Lemma and Theorem
% ---------------------------------------------------------

\begin{lemma}[Representative Independence Lemma]\label{lem:rep-independence}
Let $R$ be an equivalence relation on $A$. For any $a,b \in A$,
\[
[a] = [b]
\;\;\Longleftrightarrow\;\;
(a,b) \in R.
\]
\end{lemma}



\begin{remark}[Consequence for quotient maps]
This lemma is the key fact underlying well-definedness of functions on quotient
sets: a function $f : A \to B$ defined by $f([a]) = \cdots$ is well-defined
iff the formula gives the same output for all representatives of $[a]$.
\end{remark}

\begin{theorem}[Equivalence Relations and Partitions]\label{thm:equiv-partition}
Let $A$ be a set.
\begin{enumerate}[label=(\roman*)]
\item If $R$ is an equivalence relation on $A$, then $A/R$ is a partition of $A$.
\item If $\mathcal{P}$ is a partition of $A$, then the relation
$R_{\mathcal{P}}$ defined by
\[
(a,b) \in R_{\mathcal{P}} \;\Longleftrightarrow\;
\exists P \in \mathcal{P} \text{ with } a \in P \text{ and } b \in P
\]
is an equivalence relation on $A$.
\item These constructions are inverse: $R_{A/R} = R$ and $A/R_{\mathcal{P}} = \mathcal{P}$.
\end{enumerate}
\end{theorem}



\begin{remark}[Significance]
This theorem establishes a bijection between equivalence relations on $A$
and partitions of $A$. The two perspectives---``same block'' (partition) and
``related'' (equivalence relation)---are interchangeable and each is more
natural in different contexts.
\end{remark}

\begin{example}[Extremal equivalence relations]
On any set $A$:
\begin{enumerate}
\item The \emph{equality relation} $(a,b) \in R \iff a=b$ gives singleton
classes $[a] = \{a\}$ and the finest partition of $A$.
\item The \emph{universal relation} $R = A \times A$ gives $[a] = A$ for all
$a$, and the coarsest partition (one block).
\end{enumerate}
All other equivalence relations on $A$ lie strictly between these extremes.
\end{example}
