% =========================================================
% Indexed Families and Arbitrary Cartesian Products
% =========================================================

\subsection{Indexed Families of Sets}

% ---------------------------------------------------------
% TOOLKIT
% ---------------------------------------------------------
\begin{tcolorbox}[colback=gray!6, colframe=gray!40, arc=2pt,
  left=6pt, right=6pt, top=4pt, bottom=4pt,
  title={\small\textbf{Indexed Families — Quick Reference}},
  fonttitle=\small\bfseries]
\small
\begin{tabular}{l l l}
\toprule
\textbf{Concept} & \textbf{Meaning} & \textbf{Detail} \\
\midrule
Indexed family      & Function $F : I \to \mathcal{P}(U)$; written $\{A_i\}_{i \in I}$ & \hyperref[def:indexed-family]{↓ Def} \\
Indexed union       & $\bigcup_{i \in I} A_i$: in at least one $A_i$                    & \hyperref[def:indexed-union]{↓ Def} \\
Indexed intersection & $\bigcap_{i \in I} A_i$: in every $A_i$                         & \hyperref[def:indexed-intersection]{↓ Def} \\
Pairwise disjoint   & $i \neq j \Rightarrow A_i \cap A_j = \varnothing$                  & \hyperref[def:pairwise-disjoint]{↓ Def} \\
Cover               & $\bigcup_{C \in \mathcal{C}} C = A$                               & \hyperref[def:cover]{↓ Def} \\
Arbitrary product   & $\prod_{i \in I} A_i$: choice functions $f : I \to \bigcup A_i$   & \hyperref[def:arbitrary-product]{↓ Def} \\
\bottomrule
\end{tabular}
\end{tcolorbox}

\vspace{1em}

\begin{remark}[Motivation]
Many set-theoretic constructions require not just pairs of sets but
infinite families. Indexed families provide the formal language for
partitions, equivalence classes, unions over countably or uncountably many
sets, and products indexed by arbitrary index sets.
\end{remark}

\begin{tcolorbox}[colback=propbox, colframe=propborder, arc=2pt,
  left=6pt, right=6pt, top=4pt, bottom=4pt,
  title={\small\textbf{Definition (Indexed Family of Sets)}},
  fonttitle=\small\bfseries]
\label{def:indexed-family}
Let $I$ be a set (the \emph{index set}) and $U$ a universe. An
\emph{indexed family of sets} is a function
\[
F : I \to \mathcal{P}(U),
\]
with $F(i)$ typically written $A_i$. The family is denoted $\{A_i\}_{i \in I}$.
\end{tcolorbox}

\begin{remark}[Family vs.\ set of sets]
An indexed family is formally a \emph{function}, not a set of sets.
Different indices may correspond to the same set: $i \neq j$ does not imply
$A_i \neq A_j$. This distinction matters for equivalence class constructions.
\end{remark}

\begin{tcolorbox}[colback=propbox, colframe=propborder, arc=2pt,
  left=6pt, right=6pt, top=4pt, bottom=4pt,
  title={\small\textbf{Definition (Indexed Union)}},
  fonttitle=\small\bfseries]
\label{def:indexed-union}
The \emph{union} of the indexed family $\{A_i\}_{i \in I}$ is:
\[
\bigcup_{i \in I} A_i
\;:=\;
\{\, x \mid \exists i \in I \text{ such that } x \in A_i \,\}.
\]
\end{tcolorbox}

\begin{tcolorbox}[colback=propbox, colframe=propborder, arc=2pt,
  left=6pt, right=6pt, top=4pt, bottom=4pt,
  title={\small\textbf{Definition (Indexed Intersection)}},
  fonttitle=\small\bfseries]
\label{def:indexed-intersection}
For $I \neq \varnothing$, the \emph{intersection} of $\{A_i\}_{i \in I}$ is:
\[
\bigcap_{i \in I} A_i
\;:=\;
\{\, x \mid \forall i \in I,\; x \in A_i \,\}.
\]
\end{tcolorbox}

\begin{remark}[Why $I \neq \varnothing$ is required]
If $I = \varnothing$, then $\forall i \in I, x \in A_i$ holds vacuously for
every $x$, making the intersection the entire universe $U$. This is normally
left undefined or requires specifying a background universe explicitly.
\end{remark}

\begin{tcolorbox}[colback=propbox, colframe=propborder, arc=2pt,
  left=6pt, right=6pt, top=4pt, bottom=4pt,
  title={\small\textbf{Definition (Pairwise Disjoint Family)}},
  fonttitle=\small\bfseries]
\label{def:pairwise-disjoint}
The family $\{A_i\}_{i \in I}$ is \emph{pairwise disjoint} if
\[
\forall i,j \in I,\;
i \neq j \rightarrow A_i \cap A_j = \varnothing.
\]
\end{tcolorbox}

\begin{definition}[Cover]\label{def:cover}
A collection $\mathcal{C} \subseteq \mathcal{P}(A)$ is a \emph{cover} of $A$ if
\[
\bigcup_{C \in \mathcal{C}} C = A.
\]
\end{definition}

\vspace{1em}
\subsection{Arbitrary Cartesian Products}

\begin{tcolorbox}[colback=propbox, colframe=propborder, arc=2pt,
  left=6pt, right=6pt, top=4pt, bottom=4pt,
  title={\small\textbf{Definition (Arbitrary Cartesian Product)}},
  fonttitle=\small\bfseries]
\label{def:arbitrary-product}
Let $\{A_i\}_{i \in I}$ be an indexed family of sets. The \emph{Cartesian
product} is:
\[
\prod_{i \in I} A_i
\;:=\;
\Bigl\{
f : I \to \bigcup_{i \in I} A_i
\;\Bigm|\;
\forall i \in I,\; f(i) \in A_i
\Bigr\}.
\]
\end{tcolorbox}

\begin{remark}[Elements as choice functions]
An element of $\prod_{i \in I} A_i$ is a \emph{choice function}: a function
that assigns to each index $i$ an element of the corresponding set $A_i$.
For finite $I = \{1,\dots,n\}$, this reduces to the familiar $n$-tuple
$(a_1,\dots,a_n)$.
\end{remark}

\begin{remark}[Axiom of Choice connection]
For infinite $I$, the product $\prod_{i \in I} A_i$ is nonempty iff a choice
function exists. This existence is not guaranteed by the other ZFC axioms:
it is equivalent to the Axiom of Choice. Thus the Axiom of Choice is
precisely the assertion that arbitrary products of nonempty sets are nonempty.
\end{remark}
