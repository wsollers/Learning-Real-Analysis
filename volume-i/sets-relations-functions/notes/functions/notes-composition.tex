% =========================================================
% Composition, Inverses, and Image / Preimage Laws
% =========================================================

\subsection{Composition, Inverses, and Set-Image Laws}

% ---------------------------------------------------------
% TOOLKIT
% ---------------------------------------------------------
\begin{tcolorbox}[colback=gray!6, colframe=gray!40, arc=2pt,
  left=6pt, right=6pt, top=4pt, bottom=4pt,
  title={\small\textbf{Composition \& Inverses — Quick Reference}},
  fonttitle=\small\bfseries]
\small
\begin{tabular}{l l l l}
\toprule
\textbf{Result} & \textbf{Statement} & \textbf{Proof method} & \textbf{Detail} \\
\midrule
Associativity of $\circ$ & $h \circ (g \circ f) = (h \circ g) \circ f$ & element-wise & \hyperref[thm:comp-assoc]{↓ Thm} \\
Identity and $\circ$     & $f \circ \id_A = \id_B \circ f = f$          & element-wise & \hyperref[thm:comp-id]{↓ Thm} \\
Inj/Surj under $\circ$   & Closed under composition; partial converses  & element-wise & \hyperref[thm:comp-inj-surj]{↓ Thm} \\
Inverse characterization & $f^{-1}\circ f = \id_A$, $f \circ f^{-1} = \id_B$ & bijectivity & \hyperref[thm:inverse-char]{↓ Thm} \\
Inverse of composition   & $(g \circ f)^{-1} = f^{-1} \circ g^{-1}$    & verify both sides & \hyperref[thm:inverse-comp]{↓ Thm} \\
One-sided inverses       & Left inverse $\iff$ injective; right $\iff$ surjective & & \hyperref[thm:one-sided]{↓ Thm} \\
Preimage preserves ops   & $f^{-1}$ commutes with $\cup,\cap,\setminus,c$ & element-wise & \hyperref[thm:preimage-ops]{↓ Thm} \\
Image and set ops        & Images preserve $\cup$; contain $\subseteq$ for $\cap$ & counterexample & \hyperref[thm:image-ops]{↓ Thm} \\
\bottomrule
\end{tabular}
\end{tcolorbox}

\vspace{1em}

\begin{tcolorbox}[colback=propbox, colframe=propborder, arc=2pt,
  left=6pt, right=6pt, top=4pt, bottom=4pt,
  title={\small\textbf{Definition (Composition)}},
  fonttitle=\small\bfseries]
\label{def:composition}
Let $f : A \to B$ and $g : B \to C$. The \emph{composition} $g \circ f : A \to C$ is:
\[
(g \circ f)(a) := g\bigl(f(a)\bigr)
\quad\text{for all } a \in A.
\]
\end{tcolorbox}

\begin{remark}[Non-commutativity]
Composition is generally not commutative: $g \circ f \neq f \circ g$ even
when both are defined.
\end{remark}

\begin{theorem}[Associativity of Composition]\label{thm:comp-assoc}
For $f : A \to B$, $g : B \to C$, $h : C \to D$:
\[
h \circ (g \circ f) = (h \circ g) \circ f.
\]
\end{theorem}



\begin{theorem}[Identity and Composition]\label{thm:comp-id}
For any $f : A \to B$:
\[
f \circ \id_A = f
\quad\text{and}\quad
\id_B \circ f = f.
\]
\end{theorem}

\begin{theorem}[Injectivity and Surjectivity Under Composition]\label{thm:comp-inj-surj}
Let $f : A \to B$ and $g : B \to C$.
\begin{enumerate}[label=(\roman*)]
\item If $f$ and $g$ are injective, then $g \circ f$ is injective.
\item If $f$ and $g$ are surjective, then $g \circ f$ is surjective.
\item If $g \circ f$ is injective, then $f$ is injective.
\item If $g \circ f$ is surjective, then $g$ is surjective.
\end{enumerate}
\end{theorem}

\begin{remark}[Partial converses]
In (iii), $g$ need not be injective. In (iv), $f$ need not be surjective.
Bijectivity of $g \circ f$ does not imply bijectivity of either $f$ or $g$
individually.
\end{remark}

% ---------------------------------------------------------
% Inverse functions
% ---------------------------------------------------------
\begin{tcolorbox}[colback=propbox, colframe=propborder, arc=2pt,
  left=6pt, right=6pt, top=4pt, bottom=4pt,
  title={\small\textbf{Definition (Inverse Function)}},
  fonttitle=\small\bfseries]
\label{def:inverse}
Let $f : A \to B$ be bijective. The \emph{inverse function} is
$f^{-1} : B \to A$ defined by: $f^{-1}(b) = a$ iff $f(a) = b$.
\end{tcolorbox}

\begin{theorem}[Characterization of Inverse Functions]\label{thm:inverse-char}
Let $f : A \to B$ be bijective. Then
\[
f^{-1} \circ f = \id_A
\quad\text{and}\quad
f \circ f^{-1} = \id_B.
\]
Conversely, a function admits an inverse iff it is bijective.
\end{theorem}

\begin{theorem}[Inverse of a Composition]\label{thm:inverse-comp}
Let $f : A \to B$ and $g : B \to C$ be bijective. Then $g \circ f$ is
bijective and
\[
(g \circ f)^{-1} = f^{-1} \circ g^{-1}.
\]
\end{theorem}

% ---------------------------------------------------------
% One-sided inverses
% ---------------------------------------------------------
\begin{tcolorbox}[colback=propbox, colframe=propborder, arc=2pt,
  left=6pt, right=6pt, top=4pt, bottom=4pt,
  title={\small\textbf{Definition (Left and Right Inverses)}},
  fonttitle=\small\bfseries]
\label{def:left-right-inverse}
Let $f : A \to B$.

A \emph{left inverse} of $f$ is a function $g : B \to A$ with $g \circ f = \id_A$.

A \emph{right inverse} (or \emph{section}) of $f$ is a function $h : B \to A$
with $f \circ h = \id_B$.
\end{tcolorbox}

\begin{theorem}[One-Sided Inverses and Function Properties]\label{thm:one-sided}
Let $f : A \to B$.
\begin{enumerate}[label=(\roman*)]
\item $f$ has a left inverse $\iff$ $f$ is injective (and $A \neq \varnothing$ or $A=B=\varnothing$).
\item $f$ has a right inverse $\iff$ $f$ is surjective.
\item If $f$ has both a left inverse $g$ and a right inverse $h$, then $g = h$
and $f$ is bijective.
\end{enumerate}
\end{theorem}

\begin{remark}[Axiom of Choice]
The existence of a right inverse for every surjective function is equivalent
to the Axiom of Choice.
\end{remark}

% ---------------------------------------------------------
% Restriction and Extension
% ---------------------------------------------------------
\begin{tcolorbox}[colback=propbox, colframe=propborder, arc=2pt,
  left=6pt, right=6pt, top=4pt, bottom=4pt,
  title={\small\textbf{Definition (Restriction and Extension)}},
  fonttitle=\small\bfseries]
\label{def:restriction}
Let $f : A \to B$ and $C \subseteq A$. The \emph{restriction} of $f$ to $C$
is $f|_C : C \to B$, $f|_C(a) = f(a)$ for all $a \in C$.

Let $A \subseteq A'$, and let $f : A \to B$. A function $g : A' \to B$ is an
\emph{extension} of $f$ if $g|_A = f$.
\end{tcolorbox}

\begin{remark}[Extensions not unique]
\label{def:extension}
An extension agrees with $f$ on all of $A$ but may be defined on a larger
domain $A'$. Extensions are generally not unique; uniqueness requires additional
constraints (such as continuity in analysis or linearity in algebra), leading to
important results like the Tietze Extension Theorem and the Hahn--Banach
Theorem.
\end{remark}

% ---------------------------------------------------------
% Image and Preimage Laws
% ---------------------------------------------------------
\begin{theorem}[Preimages Preserve Set Operations]\label{thm:preimage-ops}
Let $f : A \to B$ and $S, T \subseteq B$. Then
\begin{enumerate}[label=(\roman*)]
\item $f^{-1}(S \cup T) = f^{-1}(S) \cup f^{-1}(T)$,
\item $f^{-1}(S \cap T) = f^{-1}(S) \cap f^{-1}(T)$,
\item $f^{-1}(S \setminus T) = f^{-1}(S) \setminus f^{-1}(T)$,
\item $f^{-1}(S^c) = (f^{-1}(S))^c$.
\end{enumerate}
\end{theorem}

\begin{remark}[Why preimages are well-behaved]
Preimages preserve all set-theoretic operations. This property makes preimages
central in topology and measure theory, where properties such as continuity
and measurability are defined via preimages.
\end{remark}

\begin{theorem}[Images and Set Operations]\label{thm:image-ops}
Let $f : A \to B$ and $S, T \subseteq A$. Then
\begin{enumerate}[label=(\roman*)]
\item $f(S \cup T) = f(S) \cup f(T)$,
\item $f(S \cap T) \subseteq f(S) \cap f(T)$,
\item $f(S \setminus T) \supseteq f(S) \setminus f(T)$.
\end{enumerate}
Equality holds in (ii) and (iii) for all $S,T$ iff $f$ is injective.
\end{theorem}

\begin{remark}[Asymmetry between images and preimages]
Forward images preserve unions but generally not intersections or complements.
This asymmetry between images and preimages is fundamental and explains why
preimages appear more naturally in topology and measure theory.
\end{remark}
