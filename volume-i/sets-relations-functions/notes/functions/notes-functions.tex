% =========================================================
% Functions: Definitions and Basic Properties
% =========================================================

\subsection{Functions}

% ---------------------------------------------------------
% TOOLKIT
% ---------------------------------------------------------
\begin{tcolorbox}[colback=gray!6, colframe=gray!40, arc=2pt,
  left=6pt, right=6pt, top=4pt, bottom=4pt,
  title={\small\textbf{Functions — Quick Reference}},
  fonttitle=\small\bfseries]
\small
\begin{tabular}{l l l}
\toprule
\textbf{Concept} & \textbf{Meaning} & \textbf{Detail} \\
\midrule
Function         & Left-total, right-unique relation                            & \hyperref[def:function]{↓ Def} \\
Domain / Codomain & $\dom(f) = A$,\; $\cod(f) = B$ for $f : A \to B$            & \hyperref[def:domain-codomain]{↓ Def} \\
Image of function & $\im(f) = \{f(a) \mid a \in A\}$                            & \hyperref[def:image-function]{↓ Def} \\
Image of set      & $f(S) = \{f(a) \mid a \in S\}$ for $S \subseteq A$          & \hyperref[def:image-set]{↓ Def} \\
Preimage          & $f^{-1}(T) = \{a \in A \mid f(a) \in T\}$ for $T \subseteq B$ & \hyperref[def:preimage]{↓ Def} \\
Fiber             & $f^{-1}(\{b\})$: preimage of a singleton                    & \hyperref[def:fiber]{↓ Def} \\
Graph             & $\{(a,b) \in A\times B \mid b = f(a)\}$                     & \hyperref[def:graph]{↓ Def} \\
Injective         & Distinct inputs $\Rightarrow$ distinct outputs               & \hyperref[def:injective]{↓ Def} \\
Surjective        & Every codomain element is achieved                          & \hyperref[def:surjective]{↓ Def} \\
Bijective         & Injective and surjective                                    & \hyperref[def:bijective]{↓ Def} \\
Identity          & $\id_A(a) = a$                                              & \hyperref[def:identity]{↓ Def} \\
Inclusion map     & $\iota : A \hookrightarrow B$, $\iota(a) = a$ for $A \subseteq B$ & \hyperref[def:inclusion]{↓ Def} \\
Composition       & $(g \circ f)(a) = g(f(a))$                                  & \hyperref[def:composition]{↓ Def} \\
Inverse           & Defined for bijections; $f^{-1} \circ f = \id_A$            & \hyperref[def:inverse]{↓ Def} \\
Left / Right inverse & Section and retraction                                  & \hyperref[def:left-right-inverse]{↓ Def} \\
Restriction       & $f|_C : C \to B$                                            & \hyperref[def:restriction]{↓ Def} \\
Extension         & $g : A' \to B$ with $g|_A = f$                              & \hyperref[def:extension]{↓ Def} \\
\bottomrule
\end{tabular}
\end{tcolorbox}

\vspace{1em}

% ---------------------------------------------------------
% Function definition
% ---------------------------------------------------------
\begin{tcolorbox}[colback=propbox, colframe=propborder, arc=2pt,
  left=6pt, right=6pt, top=4pt, bottom=4pt,
  title={\small\textbf{Definition (Function)}},
  fonttitle=\small\bfseries]
\label{def:function}
Let $A$ and $B$ be sets. A \emph{function} from $A$ to $B$ is a relation
$f \subseteq A \times B$ such that:
\begin{enumerate}[label=(\roman*)]
\item (\emph{Existence / left-total}) for every $a \in A$, there exists $b \in B$
  with $(a,b) \in f$;
\item (\emph{Uniqueness / right-unique}) if $(a,b_1) \in f$ and $(a,b_2) \in f$
  then $b_1 = b_2$.
\end{enumerate}
If $(a,b) \in f$ we write $f(a) = b$.
\end{tcolorbox}

\begin{remark}[English reading]
A function is a rule assigning to each input exactly one output. The two
conditions formalize ``defined everywhere'' (existence) and ``single-valued''
(uniqueness).
\end{remark}

\begin{remark}[Function as relation]
Every function is a relation, but not every relation is a function. The two
conditions that distinguish functions are left-totality and right-uniqueness.
\end{remark}

\begin{tcolorbox}[colback=propbox, colframe=propborder, arc=2pt,
  left=6pt, right=6pt, top=4pt, bottom=4pt,
  title={\small\textbf{Definition (Domain and Codomain)}},
  fonttitle=\small\bfseries]
\label{def:domain-codomain}
If $f$ is a function from $A$ to $B$, we write $f : A \to B$, where $A$ is
the \emph{domain} $\dom(f)$ and $B$ the \emph{codomain} $\cod(f)$.
\end{tcolorbox}

\begin{tcolorbox}[colback=propbox, colframe=propborder, arc=2pt,
  left=6pt, right=6pt, top=4pt, bottom=4pt,
  title={\small\textbf{Definition (Image of a Function)}},
  fonttitle=\small\bfseries]
\label{def:image-function}
The \emph{image} (or \emph{range}) of $f : A \to B$ is:
\[
\im(f) \;:=\; \{\, b \in B \mid \exists a \in A,\ f(a)=b \,\}.
\]
The image may be a proper subset of the codomain.
\end{tcolorbox}

\begin{tcolorbox}[colback=propbox, colframe=propborder, arc=2pt,
  left=6pt, right=6pt, top=4pt, bottom=4pt,
  title={\small\textbf{Definition (Image of a Set)}},
  fonttitle=\small\bfseries]
\label{def:image-set}
For $S \subseteq A$, the \emph{image of $S$ under $f$} is:
\[
f(S) \;:=\; \{\, f(a) \mid a \in S \,\}.
\]
Note $f(A) = \im(f)$.
\end{tcolorbox}

\begin{tcolorbox}[colback=propbox, colframe=propborder, arc=2pt,
  left=6pt, right=6pt, top=4pt, bottom=4pt,
  title={\small\textbf{Definition (Preimage)}},
  fonttitle=\small\bfseries]
\label{def:preimage}
For $T \subseteq B$, the \emph{preimage} (inverse image) of $T$ under $f$ is:
\[
f^{-1}(T) \;:=\; \{\, a \in A \mid f(a) \in T \,\}.
\]
\end{tcolorbox}

\begin{remark}[Preimage notation warning]
$f^{-1}(T)$ does \emph{not} denote an inverse function. It is defined for any
function $f$, regardless of whether $f$ is injective or bijective.
\end{remark}

\begin{tcolorbox}[colback=propbox, colframe=propborder, arc=2pt,
  left=6pt, right=6pt, top=4pt, bottom=4pt,
  title={\small\textbf{Definition (Fiber)}},
  fonttitle=\small\bfseries]
\label{def:fiber}
The \emph{fiber} of $f$ over $b \in B$ is the preimage of the singleton:
\[
f^{-1}(\{b\}) \;=\; \{\, a \in A \mid f(a) = b \,\}.
\]
\end{tcolorbox}

\begin{remark}[Fibers partition the domain]
The collection $\{f^{-1}(\{b\}) \mid b \in \im(f)\}$ is a partition of $A$.
Every function thus induces an equivalence relation $a_1 \sim_f a_2 \iff f(a_1) = f(a_2)$,
whose equivalence classes are precisely the fibers.
\end{remark}

\begin{tcolorbox}[colback=propbox, colframe=propborder, arc=2pt,
  left=6pt, right=6pt, top=4pt, bottom=4pt,
  title={\small\textbf{Definition (Graph of a Function)}},
  fonttitle=\small\bfseries]
\label{def:graph}
The \emph{graph} of $f : A \to B$ is:
\[
\operatorname{Graph}(f)
\;:=\;
\{\, (a,b) \in A \times B \mid b = f(a) \,\}.
\]
\end{tcolorbox}

\begin{remark}[Function as graph]
A relation $G \subseteq A \times B$ is the graph of a function $f : A \to B$
iff $G$ is left-total and right-unique. Functions may therefore be identified
with their graphs: a function \emph{is} a special kind of relation.
\end{remark}

% ---------------------------------------------------------
% Injectivity / Surjectivity / Bijectivity
% ---------------------------------------------------------
\begin{tcolorbox}[colback=propbox, colframe=propborder, arc=2pt,
  left=6pt, right=6pt, top=4pt, bottom=4pt,
  title={\small\textbf{Definition (Injective Function)}},
  fonttitle=\small\bfseries]
\label{def:injective}
$f : A \to B$ is \emph{injective} (one-to-one) if distinct elements of $A$
have distinct images:
\[
\forall a_1,a_2 \in A,\;
f(a_1)=f(a_2) \;\Longrightarrow\; a_1=a_2.
\]
\end{tcolorbox}

\begin{tcolorbox}[colback=propbox, colframe=propborder, arc=2pt,
  left=6pt, right=6pt, top=4pt, bottom=4pt,
  title={\small\textbf{Definition (Surjective Function)}},
  fonttitle=\small\bfseries]
\label{def:surjective}
$f : A \to B$ is \emph{surjective} (onto) if every element of $B$ is achieved:
\[
\forall b \in B,\; \exists a \in A \text{ such that } f(a)=b.
\]
Equivalently, $\im(f) = B$.
\end{tcolorbox}

\begin{tcolorbox}[colback=propbox, colframe=propborder, arc=2pt,
  left=6pt, right=6pt, top=4pt, bottom=4pt,
  title={\small\textbf{Definition (Bijective Function)}},
  fonttitle=\small\bfseries]
\label{def:bijective}
$f : A \to B$ is \emph{bijective} if it is both injective and surjective.
\end{tcolorbox}

\begin{remark}[Fiber interpretation]
Injectivity: each fiber has at most one element. Surjectivity: every fiber is
nonempty. Bijectivity: every fiber has exactly one element.
\end{remark}

% ---------------------------------------------------------
% Special functions
% ---------------------------------------------------------
\begin{tcolorbox}[colback=propbox, colframe=propborder, arc=2pt,
  left=6pt, right=6pt, top=4pt, bottom=4pt,
  title={\small\textbf{Definition (Identity and Inclusion)}},
  fonttitle=\small\bfseries]
\label{def:identity}
The \emph{identity function} on $A$ is $\id_A : A \to A$, $\id_A(a) = a$.

If $A \subseteq B$, the \emph{inclusion map} is $\iota : A \hookrightarrow B$,
$\iota(a) = a$.
\end{tcolorbox}

\begin{remark}[Identity vs.\ inclusion]
\label{def:inclusion}
Both send each element to itself, but the inclusion map has a strictly larger
codomain when $A \subsetneq B$. The inclusion map is always injective.
\end{remark}

\begin{definition}[Constant Function]\label{def:constant}
$f : A \to B$ is \emph{constant} if there exists $b_0 \in B$ with
$f(a) = b_0$ for all $a \in A$.
\end{definition}
