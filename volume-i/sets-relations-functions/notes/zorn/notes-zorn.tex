% =========================================================
% Zorn's Lemma and the Axiom of Choice
% Source: Mark Dean, Order Theory (Brown University, 2015)
% =========================================================

\subsection{Zorn's Lemma and the Axiom of Choice}

% ---------------------------------------------------------
% TOOLKIT
% ---------------------------------------------------------
\begin{tcolorbox}[colback=gray!6, colframe=gray!40, arc=2pt,
  left=6pt, right=6pt, top=4pt, bottom=4pt,
  title={\small\textbf{AC Equivalences — Quick Reference}},
  fonttitle=\small\bfseries]
\small
\begin{tabular}{l l}
\toprule
\textbf{Statement} & \textbf{Role} \\
\midrule
Axiom of Choice (AC)         & Choice function from any collection of nonempty sets \\
Product form of AC           & Cartesian product of nonempty sets is nonempty \\
Zorn's Lemma                 & Poset with every chain bounded $\Rightarrow$ maximal element \\
Hausdorff Maximum Principle  & Every poset has a $\subseteq$-maximal chain \\
Well-Ordering Theorem        & Every set admits a well-ordering \\
\bottomrule
\end{tabular}

\medskip
All five statements are equivalent (over ZF set theory).
\end{tcolorbox}

\vspace{1em}

% ---------------------------------------------------------
% Axiom of Choice
% ---------------------------------------------------------
\begin{tcolorbox}[colback=propbox, colframe=propborder, arc=2pt,
  left=6pt, right=6pt, top=4pt, bottom=4pt,
  title={\small\textbf{Axiom of Choice}},
  fonttitle=\small\bfseries]
\label{ax:ac}
Let $\mathcal{A}$ be any collection of nonempty sets. Then there exists a
function
\[
  f : \mathcal{A} \to \bigcup \mathcal{A}
\]
such that $f(A) \in A$ for all $A \in \mathcal{A}$.

\medskip
\noindent\textbf{Equivalent form.} The Cartesian product of an arbitrary
collection of nonempty sets is nonempty.
\end{tcolorbox}

\begin{remark}[Why AC is non-constructive]
For finite collections, the axiom is provable without extra assumptions
(see SRF-STO-C08-S8.1). For infinite collections, no construction can
produce the choice function: AC asserts its existence without exhibiting it.
This non-constructive character is both AC's power and the source of the
foundational controversy surrounding it.
\end{remark}

% ---------------------------------------------------------
% Zorn's Lemma
% ---------------------------------------------------------
\begin{tcolorbox}[colback=thmbox, colframe=propborder, arc=2pt,
  left=6pt, right=6pt, top=4pt, bottom=4pt,
  title={\small\textbf{Zorn's Lemma}},
  fonttitle=\small\bfseries]
\label{thm:zorn}
Let $(P, \leq)$ be a poset in which every chain (loset) has an upper bound
in $P$. Then $P$ has at least one maximal element.
\end{tcolorbox}

\begin{remark}[Reading Zorn's Lemma]
The hypothesis ``every chain has an upper bound'' does \emph{not} require
the bound to lie in the chain itself. The conclusion gives a maximal
element: an element above which nothing in $P$ sits. In a partial order,
there may be many maximal elements; Zorn guarantees at least one.
\end{remark}

\begin{remark}[Typical proof pattern using Zorn's Lemma]
To show an object of type $X$ exists with a maximal property:
\begin{enumerate}
  \item Partially order candidate objects by inclusion (or extension).
  \item Verify every chain of candidates has an upper bound (usually its union).
  \item Conclude by Zorn that a maximal candidate exists.
  \item Show the maximal candidate has the desired property (often: if it
        lacked it, it could be enlarged, contradicting maximality).
\end{enumerate}
This pattern recurs throughout algebra and analysis: maximal ideals,
algebraic bases (Hamel bases), maximal consistent sets, and Szpilrajn's Theorem.
\end{remark}

% ---------------------------------------------------------
% Hausdorff Maximum Principle
% ---------------------------------------------------------
\begin{tcolorbox}[colback=thmbox, colframe=propborder, arc=2pt,
  left=6pt, right=6pt, top=4pt, bottom=4pt,
  title={\small\textbf{Hausdorff Maximum Principle}},
  fonttitle=\small\bfseries]
\label{thm:hausdorff}
In every poset $(P, \leq)$ there exists a $\subseteq$-maximal chain; that
is, a chain $C \subseteq P$ such that no chain $C' \subseteq P$ strictly
contains $C$.
\end{tcolorbox}

\begin{remark}[Relation to Zorn]
The Hausdorff Maximum Principle is an immediate corollary of Zorn's Lemma:
take the poset of chains ordered by inclusion; every chain of chains is
bounded by its union; Zorn gives a maximal chain. Conversely, Hausdorff
implies Zorn. Both are equivalent to AC over ZF.
\end{remark}

\begin{remark}[Application: Szpilrajn via Hausdorff]
In the proof of Szpilrajn's Theorem (\hyperref[thm:szpilrajn]{↑ Thm}), let
$T_X$ be the set of all partial orders on $X$ extending $\succsim$, ordered
by $\subseteq$. By the Hausdorff Maximum Principle, $T_X$ has a maximal chain
$A$; its union $\succsim^* = \bigcup A$ is a partial order extending
$\succsim$. If $\succsim^*$ were not total, there would exist incomparable
$x, y$ and one could extend $\succsim^*$ by adding $(x,y)$ (closing under
transitivity), contradicting maximality of $A$. Hence $\succsim^*$ is a
linear order. \qed
\end{remark}

\begin{remark}[Transition]
Zorn's Lemma is the workhorse for existence proofs throughout abstract
algebra and analysis. It first appears in the project context here and at
the proof stubs in \S11 of the Chapter~VIII exercises.
\end{remark}
