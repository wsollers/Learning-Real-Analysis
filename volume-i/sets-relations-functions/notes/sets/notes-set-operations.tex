% =========================================================
% Derived Constructions and Set Operations
% =========================================================

\subsection{Set Constructions and Operations}

% ---------------------------------------------------------
% TOOLKIT
% ---------------------------------------------------------
\begin{tcolorbox}[colback=gray!6, colframe=gray!40, arc=2pt,
  left=6pt, right=6pt, top=4pt, bottom=4pt,
  title={\small\textbf{Set Operations — Quick Reference}},
  fonttitle=\small\bfseries]
\small
\begin{tabular}{l l l l}
\toprule
\textbf{Concept} & \textbf{Notation} & \textbf{Membership condition} & \textbf{Detail} \\
\midrule
Empty set        & $\varnothing$             & $x \in \varnothing$ never         & \hyperref[def:empty-set]{↓ Def} \\
Subset           & $A \subseteq B$           & $x \in A \Rightarrow x \in B$     & \hyperref[def:subset]{↓ Def} \\
Proper subset    & $A \subsetneq B$          & $A \subseteq B$ and $A \neq B$    & \hyperref[def:proper-subset]{↓ Def} \\
Set equality     & $A = B$                   & same elements both ways            & \hyperref[def:set-equality]{↓ Def} \\
Union            & $A \cup B$                & in $A$ or in $B$                  & \hyperref[def:union]{↓ Def} \\
Intersection     & $A \cap B$                & in $A$ and in $B$                 & \hyperref[def:intersection]{↓ Def} \\
Set difference   & $A \setminus B$           & in $A$ but not $B$                & \hyperref[def:set-difference]{↓ Def} \\
Symmetric diff.  & $A \triangle B$           & in exactly one of $A$, $B$        & \hyperref[def:sym-diff]{↓ Def} \\
Complement       & $A^c = U \setminus A$     & in $U$ but not $A$                & \hyperref[def:complement]{↓ Def} \\
Cartesian product& $A \times B$              & all ordered pairs $(a,b)$         & \hyperref[def:cartesian-product]{↓ Def} \\
Power set        & $\mathcal{P}(A)$          & all subsets of $A$                & \hyperref[def:power-set]{↓ Def} \\
De Morgan        & $(A \cup B)^c = A^c \cap B^c$ & complement of union        & \hyperref[thm:de-morgan]{↓ Thm} \\
\bottomrule
\end{tabular}
\end{tcolorbox}

\vspace{1em}

% ---------------------------------------------------------
% Empty Set
% ---------------------------------------------------------
\begin{tcolorbox}[colback=propbox, colframe=propborder, arc=2pt,
  left=6pt, right=6pt, top=4pt, bottom=4pt,
  title={\small\textbf{Definition (Empty Set)}},
  fonttitle=\small\bfseries]
\label{def:empty-set}
The unique set with no elements is called the \emph{empty set} and is denoted
$\varnothing$.
\end{tcolorbox}

\begin{remark}[Vacuous truth]
Statements of the form $\forall x \in \varnothing,\; P(x)$ are
\emph{vacuously true}: there is no element $x \in \varnothing$ for which
$P(x)$ could fail. This is not a special convention but a consequence of
how universal quantification is defined.
\end{remark}

% ---------------------------------------------------------
% Subset and Set Equality
% ---------------------------------------------------------
\begin{tcolorbox}[colback=propbox, colframe=propborder, arc=2pt,
  left=6pt, right=6pt, top=4pt, bottom=4pt,
  title={\small\textbf{Definition (Subset)}},
  fonttitle=\small\bfseries]
\label{def:subset}
Let $A$ and $B$ be sets. We say $A$ is a \emph{subset} of $B$, written
$A \subseteq B$, if every element of $A$ is also an element of $B$:
\[
A \subseteq B \;\;\Longleftrightarrow\;\; \forall x \, (x \in A \rightarrow x \in B).
\]
\end{tcolorbox}

\begin{tcolorbox}[colback=propbox, colframe=propborder, arc=2pt,
  left=6pt, right=6pt, top=4pt, bottom=4pt,
  title={\small\textbf{Definition (Proper Subset)}},
  fonttitle=\small\bfseries]
\label{def:proper-subset}
$A$ is a \emph{proper subset} of $B$, written $A \subsetneq B$, if
$A \subseteq B$ and $A \neq B$.
\end{tcolorbox}

\begin{remark}[Notation convention]
Some authors write $A \subset B$ for a proper subset; others use it to mean
$A \subseteq B$. In these notes, $\subseteq$ always denotes subset (possibly
equal) and $\subsetneq$ always denotes proper subset.
\end{remark}

\begin{tcolorbox}[colback=propbox, colframe=propborder, arc=2pt,
  left=6pt, right=6pt, top=4pt, bottom=4pt,
  title={\small\textbf{Definition (Set Equality)}},
  fonttitle=\small\bfseries]
\label{def:set-equality}
Two sets $A$ and $B$ are \emph{equal}, written $A = B$, if they have the same
elements:
\[
A = B \;\;\Longleftrightarrow\;\; \forall x \, (x \in A \leftrightarrow x \in B).
\]
Equivalently, $A = B \iff (A \subseteq B \land B \subseteq A)$.
\end{tcolorbox}

\begin{remark}[Proof strategy]
In practice, to prove $A = B$ one shows mutual inclusion: first $A \subseteq B$
(take arbitrary $x \in A$ and deduce $x \in B$), then $B \subseteq A$.
This two-step structure appears throughout set-theoretic proofs.
\end{remark}

% ---------------------------------------------------------
% Set Operations
% ---------------------------------------------------------
\begin{tcolorbox}[colback=propbox, colframe=propborder, arc=2pt,
  left=6pt, right=6pt, top=4pt, bottom=4pt,
  title={\small\textbf{Definition (Union)}},
  fonttitle=\small\bfseries]
\label{def:union}
The \emph{union} of $A$ and $B$, denoted $A \cup B$, is the set of all
elements belonging to at least one of the two sets:
\[
A \cup B \;=\; \{\, x \mid x \in A \lor x \in B \,\}.
\]
\end{tcolorbox}

\begin{center}
\begin{tikzpicture}[scale=0.85]
  \fill[pattern=north east lines] (-1,0) circle (1.5);
  \fill[pattern=north east lines] ( 1,0) circle (1.5);
  \draw (-1,0) circle (1.5);
  \draw ( 1,0) circle (1.5);
  \node at (-2.2,-1.6) {$A$};
  \node at ( 2.2,-1.6) {$B$};
\end{tikzpicture}
\end{center}

\begin{tcolorbox}[colback=propbox, colframe=propborder, arc=2pt,
  left=6pt, right=6pt, top=4pt, bottom=4pt,
  title={\small\textbf{Definition (Intersection)}},
  fonttitle=\small\bfseries]
\label{def:intersection}
The \emph{intersection} of $A$ and $B$, denoted $A \cap B$, is the set of all
elements common to both:
\[
A \cap B \;=\; \{\, x \mid x \in A \land x \in B \,\}.
\]
\end{tcolorbox}

\begin{center}
\begin{tikzpicture}[scale=0.85]
  \begin{scope}
    \clip (-1,0) circle (1.5);
    \fill[pattern=north east lines] (1,0) circle (1.5);
  \end{scope}
  \draw (-1,0) circle (1.5);
  \draw ( 1,0) circle (1.5);
  \node at (-2.2,-1.6) {$A$};
  \node at ( 2.2,-1.6) {$B$};
\end{tikzpicture}
\end{center}

\begin{tcolorbox}[colback=propbox, colframe=propborder, arc=2pt,
  left=6pt, right=6pt, top=4pt, bottom=4pt,
  title={\small\textbf{Definition (Set Difference)}},
  fonttitle=\small\bfseries]
\label{def:set-difference}
The \emph{set difference} $A \setminus B$ is the set of elements in $A$ but
not in $B$:
\[
A \setminus B \;=\; \{\, x \mid x \in A \land x \notin B \,\}.
\]
\end{tcolorbox}

\begin{center}
\begin{tikzpicture}[scale=0.85]
  \fill[pattern=north east lines] (-1,0) circle (1.5);
  \begin{scope}
    \clip (1,0) circle (1.5);
    \fill[white] (-1,0) circle (1.5);
  \end{scope}
  \draw (-1,0) circle (1.5);
  \draw ( 1,0) circle (1.5);
  \node at (-2.2,-1.6) {$A$};
  \node at ( 2.2,-1.6) {$B$};
\end{tikzpicture}
\end{center}

\begin{tcolorbox}[colback=propbox, colframe=propborder, arc=2pt,
  left=6pt, right=6pt, top=4pt, bottom=4pt,
  title={\small\textbf{Definition (Symmetric Difference)}},
  fonttitle=\small\bfseries]
\label{def:sym-diff}
The \emph{symmetric difference} $A \triangle B$ is the set of elements
belonging to exactly one of $A$ and $B$:
\[
A \triangle B
\;=\;
(A \setminus B) \cup (B \setminus A)
\;=\;
\{\, x \mid (x \in A \lor x \in B) \land x \notin A \cap B \,\}.
\]
\end{tcolorbox}

\begin{center}
\begin{tikzpicture}[scale=0.85]
  \fill[pattern=north east lines] (-1,0) circle (1.5);
  \fill[pattern=north east lines] ( 1,0) circle (1.5);
  \begin{scope}
    \clip (-1,0) circle (1.5);
    \fill[white] (1,0) circle (1.5);
  \end{scope}
  \draw (-1,0) circle (1.5);
  \draw ( 1,0) circle (1.5);
  \node at (-2.2,-1.6) {$A$};
  \node at ( 2.2,-1.6) {$B$};
\end{tikzpicture}
\end{center}

\begin{remark}[Algebraic structure of $\triangle$]
The symmetric difference is commutative and associative. Under $\triangle$,
the power set $\mathcal{P}(U)$ forms an abelian group with identity
$\varnothing$ and every element its own inverse ($A \triangle A = \varnothing$).
\end{remark}

\begin{tcolorbox}[colback=propbox, colframe=propborder, arc=2pt,
  left=6pt, right=6pt, top=4pt, bottom=4pt,
  title={\small\textbf{Definition (Complement)}},
  fonttitle=\small\bfseries]
\label{def:complement}
Let $U$ be a fixed universe and $A \subseteq U$. The \emph{complement} of $A$
relative to $U$, denoted $A^c$, is:
\[
A^c \;=\; U \setminus A.
\]
\end{tcolorbox}

\begin{tcolorbox}[colback=propbox, colframe=propborder, arc=2pt,
  left=6pt, right=6pt, top=4pt, bottom=4pt,
  title={\small\textbf{Definition (Cartesian Product)}},
  fonttitle=\small\bfseries]
\label{def:cartesian-product}
The \emph{Cartesian product} of sets $A$ and $B$ is:
\[
A \times B = \{\, (a,b) \mid a \in A \text{ and } b \in B \,\}.
\]
\end{tcolorbox}

\begin{remark}[Ordered pairs]
The ordered pair $(a,b)$ is formally defined as $\{\{a\},\{a,b\}\}$.
This Kuratowski encoding ensures the key property: $(a,b) = (c,d)$ iff
$a = c$ and $b = d$.
\end{remark}

\begin{remark}[Non-commutativity]
In general, $A \times B \neq B \times A$. The product is however associative
up to canonical isomorphism: $(A \times B) \times C \cong A \times (B \times C)$.
\end{remark}

\begin{tcolorbox}[colback=propbox, colframe=propborder, arc=2pt,
  left=6pt, right=6pt, top=4pt, bottom=4pt,
  title={\small\textbf{Definition (Power Set)}},
  fonttitle=\small\bfseries]
\label{def:power-set}
The \emph{power set} of $A$, denoted $\mathcal{P}(A)$, is the set of all
subsets of $A$:
\[
\mathcal{P}(A) \;:=\; \{\, S \mid S \subseteq A \,\}.
\]
\end{tcolorbox}

\begin{remark}[Size]
If $A$ has $n$ elements, then $|\mathcal{P}(A)| = 2^n$. For infinite $A$,
Cantor's theorem shows $|\mathcal{P}(A)| > |A|$ strictly.
\end{remark}

% ---------------------------------------------------------
% De Morgan — Theorem box (named, major result of section)
% ---------------------------------------------------------
\begin{tcolorbox}[colback=thmbox, colframe=thmborder, arc=2pt,
  left=6pt, right=6pt, top=4pt, bottom=4pt,
  title={\small\textbf{Theorem (De Morgan's Laws)}},
  fonttitle=\small\bfseries]
\label{thm:de-morgan}
Let $U$ be a universe and $A,B \subseteq U$. Then
\[
(A \cup B)^c = A^c \cap B^c
\quad\text{and}\quad
(A \cap B)^c = A^c \cup B^c.
\]
\end{tcolorbox}

\begin{proof}
We prove the first identity; the second follows by symmetry (or set duality).

Let $x \in U$. Then
\[
\begin{aligned}
x \in (A \cup B)^c
&\iff x \notin A \cup B \\
&\iff (x \notin A \land x \notin B) \\
&\iff (x \in A^c \land x \in B^c) \\
&\iff x \in A^c \cap B^c.
\end{aligned}
\]
Since membership is equivalent for all $x \in U$, the sets are equal.
\end{proof}

\begin{remark}[Logical analogy]
De Morgan's laws mirror the propositional logic identities $\neg(P \lor Q) \equiv \neg P \land \neg Q$ and $\neg(P \land Q) \equiv \neg P \lor \neg Q$ under the correspondence $\cap \leftrightarrow \land$, $\cup \leftrightarrow \lor$, $A^c \leftrightarrow \neg A$.
\end{remark}

\begin{definition}[Set Duality]\label{def:set-duality}
Two set-theoretic expressions over a fixed universe $U$ are \emph{dual} if one
is obtained from the other by simultaneously replacing
$\cup \leftrightarrow \cap$ and $\varnothing \leftrightarrow U$, with
complements unchanged.
\end{definition}

\begin{corollary}[Principle of Set Duality]\label{cor:set-duality}
Any identity involving $\cup$, $\cap$, $\varnothing$, and $U$ that holds for
all subsets of a universe remains valid when each operation and constant is
replaced by its dual.
\end{corollary}

\begin{remark}[Using duality in proofs]
To prove a statement involving unions and intersections, it often suffices to
prove one version; the dual statement follows immediately.
\end{remark}

\begin{example}[Distributive law via duality]
The two distributive laws
\[
A \cap (B \cup C) = (A \cap B) \cup (A \cap C)
\quad\text{and}\quad
A \cup (B \cap C) = (A \cup B) \cap (A \cup C)
\]
are duals of each other. Proving either one and applying the Principle of Set
Duality yields the other immediately.
\end{example}

\begin{remark}[Transition to relations]
The Cartesian product provides a way to encode ordered information.
Relations and functions will be defined as special subsets of Cartesian
products, so they require no new foundational objects.
\end{remark}
