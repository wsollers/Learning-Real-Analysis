% =========================================================
% Sets, Membership, and ZFC Axioms
% =========================================================

\subsection{Sets, Membership, and ZFC Axioms}

% ---------------------------------------------------------
% TOOLKIT
% ---------------------------------------------------------
\begin{tcolorbox}[colback=gray!6, colframe=gray!40, arc=2pt,
  left=6pt, right=6pt, top=4pt, bottom=4pt,
  title={\small\textbf{Foundations — Quick Reference}},
  fonttitle=\small\bfseries]
\small
\begin{tabular}{l l l}
\toprule
\textbf{Concept} & \textbf{Meaning} & \textbf{Detail} \\
\midrule
Set, membership        & Primitive notions; meaning fixed by axioms & \hyperref[def:set-membership]{↓ Def} \\
Extensionality         & Sets equal iff same elements & \hyperref[ax:extensionality]{↓ Ax} \\
Empty Set              & Exists a set with no elements & \hyperref[ax:empty-set]{↓ Ax} \\
Pairing                & Any two sets can be collected & \hyperref[ax:pairing]{↓ Ax} \\
Union                  & Elements of sets in a family & \hyperref[ax:union]{↓ Ax} \\
Power Set              & All subsets form a set & \hyperref[ax:power-set]{↓ Ax} \\
Infinity               & An infinite set exists & \hyperref[ax:infinity]{↓ Ax} \\
Separation             & Subsets by definable property & \hyperref[ax:separation]{↓ Ax} \\
Replacement            & Functional image of a set is a set & \hyperref[ax:replacement]{↓ Ax} \\
Foundation             & No infinite descending $\in$-chains & \hyperref[ax:foundation]{↓ Ax} \\
Choice                 & Selection function exists & \hyperref[ax:choice]{↓ Ax} \\
\bottomrule
\end{tabular}
\end{tcolorbox}

\vspace{1em}

% ---------------------------------------------------------
% Primitive Notions
% ---------------------------------------------------------
\begin{tcolorbox}[colback=propbox, colframe=propborder, arc=2pt,
  left=6pt, right=6pt, top=4pt, bottom=4pt,
  title={\small\textbf{Definition (Set and Membership)}},
  fonttitle=\small\bfseries]
\label{def:set-membership}
In axiomatic set theory, the notions of \emph{set} and \emph{membership} are
\emph{primitive}.

\begin{itemize}
  \item A \emph{set} is an object.
  \item \emph{Membership} is a binary relation, denoted by $\in$, between objects.
\end{itemize}

If $x$ is an object and $A$ is a set, the statement $x \in A$ is read as
``$x$ is an element of $A$.'' No definition of ``set'' or ``$\in$'' is given
in more basic terms. Their meaning is determined entirely by the axioms
governing them.
\end{tcolorbox}

\begin{remark}[English reading]
Primitive means we do not define these in terms of simpler concepts.
Rather, we fix their meaning implicitly by stating the axioms they must obey.
This is the standard approach in formal mathematics: rules of behaviour replace
informal definitions.
\end{remark}

\begin{remark}[Consequence]
All subsequent notions---subsets, ordered pairs, relations, functions,
number systems---are \emph{definitions} introduced within this axiomatic
framework. Every proved result is a \emph{theorem} derived logically from the
axioms via the rules of inference.
\end{remark}

% ---------------------------------------------------------
% ZFC AXIOMS — single axiom box
% ---------------------------------------------------------
\begin{tcolorbox}[colback=axiombox, colframe=axiomborder, arc=2pt,
  left=6pt, right=6pt, top=4pt, bottom=4pt,
  title={\small\textbf{Axiom System (ZFC)}},
  fonttitle=\small\bfseries]

\textbf{Axiom of Extensionality.}\label{ax:extensionality}\\
Two sets are equal iff they have the same elements.
\[
\forall A \, \forall B \,
\Bigl( A = B \iff \forall x \, (x \in A \leftrightarrow x \in B) \Bigr).
\]

\medskip
\textbf{Axiom of Empty Set.}\label{ax:empty-set}\\
There exists a set with no elements.
\[
\exists A \, \forall x \, (x \notin A).
\]

\medskip
\textbf{Axiom of Pairing.}\label{ax:pairing}\\
For any two sets, there exists a set containing exactly those two sets.
\[
\forall A \, \forall B \, \exists C \,
\forall x \, (x \in C \leftrightarrow (x = A \lor x = B)).
\]

\medskip
\textbf{Axiom of Union.}\label{ax:union}\\
For any family of sets, there exists a set containing exactly the elements of
those sets.
\[
\forall A \, \exists U \,
\forall x \,
\bigl( x \in U \leftrightarrow \exists B \, (B \in A \land x \in B) \bigr).
\]

\medskip
\textbf{Axiom of Power Set.}\label{ax:power-set}\\
For any set, there exists a set of all its subsets.
\[
\forall A \, \exists P \,
\forall x \, (x \in P \leftrightarrow x \subseteq A).
\]

\medskip
\textbf{Axiom of Infinity.}\label{ax:infinity}\\
There exists an infinite set.
\[
\exists A \,
\Bigl( \varnothing \in A \land
\forall x \, (x \in A \rightarrow x \cup \{x\} \in A) \Bigr).
\]

\medskip
\textbf{Axiom Schema of Separation.}\label{ax:separation}\\
Given a set and a property, there exists a subset containing exactly the
elements satisfying that property. For any formula $\varphi(x)$,
\[
\forall A \, \exists B \,
\forall x \, (x \in B \leftrightarrow (x \in A \land \varphi(x))).
\]

\medskip
\textbf{Axiom Schema of Replacement.}\label{ax:replacement}\\
If $\varphi(x,y)$ defines a functional relation on a set $A$, then the image
of $A$ under $\varphi$ is a set. For any formula $\varphi(x,y)$,
\[
\forall A\,
\Bigl(
\bigl(\forall x \in A\, \exists! y\, \varphi(x,y)\bigr)
\rightarrow
\exists B\, \forall y\,
\bigl(y \in B \leftrightarrow \exists x \in A\, \varphi(x,y)\bigr)
\Bigr).
\]

\medskip
\textbf{Axiom of Foundation.}\label{ax:foundation}\\
Every nonempty set has an $\in$-minimal element.
\[
\forall A \,
\Bigl( A \neq \varnothing \rightarrow
\exists x \in A \, (x \cap A = \varnothing) \Bigr).
\]

\medskip
\textbf{Axiom of Choice.}\label{ax:choice}\\
For any family of nonempty sets, there exists a selection function.
\[
\forall A \,
\Bigl(
(\forall B \in A \, (B \neq \varnothing))
\rightarrow
\exists f \,
\forall B \in A \, (f(B) \in B)
\Bigr).
\]
\end{tcolorbox}

\begin{remark}[Role of the axioms]
The axioms are not statements to be proved, but rules specifying how $\in$
behaves and which sets are permitted to exist. From this point onward, set
theory functions as the underlying language of mathematics: all reasoning
about mathematical objects is ultimately grounded in these axioms, even when
they are not cited explicitly.
\end{remark}

\begin{remark}[Separation vs.\ Replacement]
Separation is a schema: one axiom instance per definable property $\varphi$.
It carves out subsets of an already-existing set. Replacement is strictly
stronger: it can produce sets whose elements are not already contained in any
pre-existing set, allowing the construction of large stages of the cumulative
hierarchy.
\end{remark}

\begin{remark}[Axiom of Choice and right inverses]
The Axiom of Choice is equivalent to many statements used later in analysis,
including Zorn's Lemma, the well-ordering theorem, and the statement that every
surjective function has a right inverse. It is independent of the other ZFC
axioms.
\end{remark}
