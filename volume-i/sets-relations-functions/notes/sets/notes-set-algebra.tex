% =========================================================
% Algebraic Laws of Set Operations
% =========================================================

\subsection{Algebraic Laws of Set Operations}

% ---------------------------------------------------------
% TOOLKIT
% ---------------------------------------------------------
\begin{tcolorbox}[colback=gray!6, colframe=gray!40, arc=2pt,
  left=6pt, right=6pt, top=4pt, bottom=4pt,
  title={\small\textbf{Set Algebra Laws — Quick Reference}},
  fonttitle=\small\bfseries]
\small
\begin{tabular}{l l l l}
\toprule
\textbf{Law} & \textbf{Union form} & \textbf{Intersection form} & \textbf{Detail} \\
\midrule
Commutativity  & $A \cup B = B \cup A$          & $A \cap B = B \cap A$          & \hyperref[thm:commutativity]{↓ Thm} \\
Associativity  & $(A \cup B) \cup C = A \cup (B \cup C)$  & analogous   & \hyperref[thm:associativity]{↓ Thm} \\
Distributivity & $A \cap (B \cup C) = (A \cap B) \cup (A \cap C)$ & dual   & \hyperref[thm:distributivity]{↓ Thm} \\
Identity       & $A \cup \varnothing = A$        & $A \cap U = A$                 & \hyperref[thm:identity-absorption]{↓ Thm} \\
Absorption     & $A \cup (A \cap B) = A$         & $A \cap (A \cup B) = A$        & \hyperref[thm:identity-absorption]{↓ Thm} \\
Involution     & $(A^c)^c = A$                  & ---                            & \hyperref[thm:involution]{↓ Thm} \\
\bottomrule
\end{tabular}
\end{tcolorbox}

\vspace{1em}

\begin{remark}[Algebraic structure of set operations]
The operations $\cup$, $\cap$, and complement satisfy algebraic laws analogous
to those of logical connectives. Together they endow $\mathcal{P}(U)$ with
the structure of a \emph{Boolean algebra}. These laws justify manipulation
of set expressions in later proofs, particularly those involving equivalence
classes, partitions, and functions.
\end{remark}

\begin{theorem}[Commutativity of Union and Intersection]\label{thm:commutativity}
Let $A,B$ be sets. Then
\[
A \cup B = B \cup A
\quad\text{and}\quad
A \cap B = B \cap A.
\]
\end{theorem}



\begin{theorem}[Associativity of Union and Intersection]\label{thm:associativity}
Let $A,B,C$ be sets. Then
\[
(A \cup B) \cup C = A \cup (B \cup C)
\quad\text{and}\quad
(A \cap B) \cap C = A \cap (B \cap C).
\]
\end{theorem}



\begin{theorem}[Distributive Laws]\label{thm:distributivity}
Let $A,B,C$ be sets. Then
\[
A \cap (B \cup C) = (A \cap B) \cup (A \cap C),
\]
\[
A \cup (B \cap C) = (A \cup B) \cap (A \cup C).
\]
\end{theorem}



\begin{theorem}[Identity and Absorption Laws]\label{thm:identity-absorption}
Let $A,B$ be sets and $U$ a universe with $A \subseteq U$. Then
\[
A \cup \varnothing = A,
\qquad
A \cap U = A,
\]
\[
A \cup (A \cap B) = A,
\qquad
A \cap (A \cup B) = A.
\]
\end{theorem}

\begin{theorem}[Involution of Complement]\label{thm:involution}
For any $A \subseteq U$,
\[
(A^c)^c = A.
\]
\end{theorem}

\begin{remark}[Non-commutative and non-associative operations]
Set difference $\setminus$ is neither commutative nor associative:
\[
A \setminus B \neq B \setminus A,
\qquad
(A \setminus B) \setminus C \neq A \setminus (B \setminus C) \quad \text{in general}.
\]
It interacts with union and intersection via:
\[
A \setminus (B \cup C) = (A \setminus B) \cap (A \setminus C),
\qquad
A \setminus (B \cap C) = (A \setminus B) \cup (A \setminus C).
\]
These follow from $A \setminus B = A \cap B^c$ together with De Morgan's laws.
\end{remark}

\begin{remark}[Cartesian product]
The Cartesian product $\times$ is not commutative, is associative only up to
canonical isomorphism, and distributes over union in each coordinate:
\[
A \times (B \cup C) = (A \times B) \cup (A \times C).
\]
\end{remark}
