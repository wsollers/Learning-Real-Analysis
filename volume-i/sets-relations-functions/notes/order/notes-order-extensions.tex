% =========================================================
% Order Extensions
% Source: Mark Dean, Order Theory (Brown University, 2015)
% =========================================================

\subsection{Order Extensions}

% ---------------------------------------------------------
% TOOLKIT
% ---------------------------------------------------------
\begin{tcolorbox}[colback=gray!6, colframe=gray!40, arc=2pt,
  left=6pt, right=6pt, top=4pt, bottom=4pt,
  title={\small\textbf{Order Extensions — Quick Reference}},
  fonttitle=\small\bfseries]
\small
\begin{tabular}{l l l}
\toprule
\textbf{Concept} & \textbf{Meaning} & \textbf{Detail} \\
\midrule
Preorder            & Reflexive, transitive (not necessarily antisymmetric)         & \hyperref[def:preorder]{↓ Def} \\
Loset               & Linear order = complete partial order                        & \hyperref[def:loset]{↓ Def} \\
Symmetric part      & $xIy \iff xRy \land yRx$                                     & \hyperref[def:sym-asym]{↓ Def} \\
Asymmetric part     & $xPy \iff xRy \land \lnot(yRx)$                              & \hyperref[def:sym-asym]{↓ Def} \\
Transitive closure  & Smallest transitive relation containing $R$                  & \hyperref[def:trans-closure]{↓ Def} \\
Extension           & $\succsim'$ agrees with $\succsim$ on strict and weak ranks  & \hyperref[def:extension]{↓ Def} \\
OWC                 & $xT(\succsim)y \Rightarrow \lnot(y \succ x)$                 & \hyperref[def:owc]{↓ Def} \\
Szpilrajn's Thm     & Every partial order extends to a linear order                & \hyperref[thm:szpilrajn]{↓ Thm} \\
\bottomrule
\end{tabular}
\end{tcolorbox}

\vspace{1em}

% ---------------------------------------------------------
% Symmetric and asymmetric parts
% ---------------------------------------------------------
\begin{tcolorbox}[colback=propbox, colframe=propborder, arc=2pt,
  left=6pt, right=6pt, top=4pt, bottom=4pt,
  title={\small\textbf{Definition (Symmetric and Asymmetric Parts)}},
  fonttitle=\small\bfseries]
\label{def:sym-asym}
For any binary relation $R$ on $X$, define:
\begin{itemize}
  \item the \emph{symmetric part}: $xIy \iff xRy \text{ and } yRx$
  \item the \emph{asymmetric part}: $xPy \iff xRy \text{ but not } yRx$
\end{itemize}
Note that $R = P \cup I$.
\end{tcolorbox}

\begin{remark}[Preference-theoretic reading]
In decision theory, $R$ encodes a preference relation: $xRy$ means ``$x$ is
at least as good as $y$.'' Then $I$ is the \emph{indifference} relation
(equally good) and $P$ is the \emph{strict preference} relation (strictly
better). This decomposition is fundamental to utility representation theory.
\end{remark}

% ---------------------------------------------------------
% Preorder and loset
% ---------------------------------------------------------
\begin{tcolorbox}[colback=propbox, colframe=propborder, arc=2pt,
  left=6pt, right=6pt, top=4pt, bottom=4pt,
  title={\small\textbf{Definition (Preorder and Loset)}},
  fonttitle=\small\bfseries]
\label{def:preorder}
A binary relation $R$ on $X$ is a \emph{preorder} if it is reflexive and
transitive.

\medskip
\label{def:loset}
A \emph{loset} (linearly ordered set) is a pair $(X, R)$ where $R$ is a
\emph{linear order}: a complete partial order ($R$ is reflexive, transitive,
antisymmetric, and complete).
\end{tcolorbox}

\begin{remark}[Preorder vs.\ partial order]
A preorder need not be antisymmetric: two distinct elements may satisfy
$xRy$ and $yRx$ simultaneously (they are ``indifferent'' but not equal).
Adding antisymmetry promotes a preorder to a partial order. Every partial
order is a preorder, but not conversely.
\end{remark}

% ---------------------------------------------------------
% Maximal elements vs. upper bounds
% ---------------------------------------------------------
\begin{tcolorbox}[colback=propbox, colframe=propborder, arc=2pt,
  left=6pt, right=6pt, top=4pt, bottom=4pt,
  title={\small\textbf{Definition (Maximal Elements and Upper Bounds)}},
  fonttitle=\small\bfseries]
\label{def:maximal-upper}
Let $\succsim$ be a binary relation on $X$.
\begin{itemize}
  \item The set of \emph{maximal elements} of $X$ is
    \[
      \mathrm{Max}(X, \succsim)
      = \{ x \in X \mid y \succsim x \text{ for no } y \in X \text{ with }
          y \neq x \}.
    \]
  \item The set of \emph{upper bounds} of $X$ is
    \[
      M(X, \succsim)
      = \{ x \in X \mid x \succsim y \text{ for all } y \in X \}.
    \]
\end{itemize}
\end{tcolorbox}

\begin{remark}[Maximal vs.\ greatest]
A maximal element has nothing strictly above it; an upper bound (greatest
element) is above everything. In a partial order, the greatest element is
unique if it exists and is always maximal, but maximal elements need not be
greatest. In a linear order, maximal and greatest coincide.
\end{remark}

% ---------------------------------------------------------
% Transitive closure
% ---------------------------------------------------------
\begin{tcolorbox}[colback=propbox, colframe=propborder, arc=2pt,
  left=6pt, right=6pt, top=4pt, bottom=4pt,
  title={\small\textbf{Definition (Transitive Closure)}},
  fonttitle=\small\bfseries]
\label{def:trans-closure}
The \emph{transitive closure} $T(R)$ of a binary relation $R$ on $X$ is the
smallest transitive relation containing $R$: that is, $T(R)$ is transitive,
$xRy \Rightarrow xT(R)y$, and no strictly smaller relation is both transitive
and contains $R$.
\end{tcolorbox}

\begin{remark}[Constructive description]
Define $R^0 = R$ and $xR^m y$ if there exist $z_1, \ldots, z_m \in X$ such
that $xRz_1 R \cdots R z_m Ry$. Then
\[
  T(R) = R \;\cup\; \bigcup_{m \in \mathbb{N}} R^m.
\]
Existence follows from the fact that the intersection of any collection of
transitive relations is transitive, so the intersection of all transitive
supersets of $R$ is the smallest such relation.
\end{remark}

% ---------------------------------------------------------
% Extensions of preorders
% ---------------------------------------------------------
\begin{tcolorbox}[colback=propbox, colframe=propborder, arc=2pt,
  left=6pt, right=6pt, top=4pt, bottom=4pt,
  title={\small\textbf{Definition (Extension of a Preorder)}},
  fonttitle=\small\bfseries]
\label{def:extension}
Let $\succsim$ be a preorder on $X$. A preorder $\succsim'$ is an
\emph{extension} of $\succsim$ if
\[
  x \succsim y \;\Rightarrow\; x \succsim' y,
  \qquad
  x \succ y \;\Rightarrow\; x \succ' y.
\]
A \emph{complete extension} is an extension that is also complete.
\end{tcolorbox}

\begin{remark}[What extensions do]
An extension of $\succsim$ ``fills in'' the comparisons left undecided by
$\succsim$ without reversing any existing strict comparison. The goal is to
promote a partial ranking to a total ranking while respecting the original
judgements.
\end{remark}

% ---------------------------------------------------------
% Szpilrajn's Theorem
% ---------------------------------------------------------
\begin{tcolorbox}[colback=thmbox, colframe=propborder, arc=2pt,
  left=6pt, right=6pt, top=4pt, bottom=4pt,
  title={\small\textbf{Theorem (Szpilrajn, 1930)}},
  fonttitle=\small\bfseries]
\label{thm:szpilrajn}
For any nonempty set $X$ and partial order $\succsim$ on $X$, there exists a
linear order that is an extension of $\succsim$.
\end{tcolorbox}

\begin{remark}[Proof sketch via Hausdorff Maximum Principle]
Let $T_X$ be the set of all partial orders on $X$ extending $\succsim$,
ordered by inclusion. By the Hausdorff Maximum Principle (equivalent to
AC), $T_X$ has a maximal chain; its union $\succsim^*$ is a partial order
extending $\succsim$. If $\succsim^*$ were not complete, one could enlarge
it via transitive closure, contradicting maximality. Hence $\succsim^*$ is a
linear order extending $\succsim$.

\medskip
\noindent\textit{Dependence on AC.} The theorem is equivalent to the Axiom
of Choice for infinite sets; no proof avoiding AC is known.
\end{remark}

\begin{corollary}[Complete preorder extension]
For any nonempty set $X$ and preorder $\succsim$ on $X$, there exists a
complete preorder that is an extension of $\succsim$.
\end{corollary}

\begin{remark}[Proof]
Pass to the quotient $X/{\sim}$ (where $\sim$ is the symmetric part), apply
Szpilrajn's Theorem to obtain a linear order on $X/{\sim}$, then pull back
to a complete preorder on $X$.
\end{remark}

% ---------------------------------------------------------
% Only Weak Cycles (OWC)
% ---------------------------------------------------------
\begin{tcolorbox}[colback=propbox, colframe=propborder, arc=2pt,
  left=6pt, right=6pt, top=4pt, bottom=4pt,
  title={\small\textbf{Definition and Proposition (Only Weak Cycles)}},
  fonttitle=\small\bfseries]
\label{def:owc}
Let $\succsim$ be a binary relation on $X$ with asymmetric part $\succ$ and
transitive closure $T(\succsim)$. We say $\succsim$ satisfies \emph{only weak
cycles} (OWC) if
\[
  xT(\succsim)y \;\Rightarrow\; \lnot(y \succ x).
\]

\medskip
\noindent\textbf{Proposition.} A binary relation $\succsim$ on a nonempty
set $X$ can be extended to a complete preorder if and only if it satisfies
OWC.
\end{tcolorbox}

\begin{remark}[Intuition for OWC]
OWC prohibits the following: there is a ``chain'' of weak preferences
$x_1 \succsim x_2 \succsim \cdots \succsim x_n$ but a strict reversal
$x_n \succ x_1$. Such a configuration cannot be extended to a complete
preorder because the chain forces $x_1 \succsim' x_n$ in any extension, but
strict preference $x_n \succ x_1$ in the extension would mean
$x_n \succsim' x_1$ but not $x_1 \succsim' x_n$ — contradicting
$x_1 \succsim' x_n$.
\end{remark}

\begin{remark}[Transition to AC]
Szpilrajn's Theorem relies on Zorn's Lemma / Hausdorff Maximum Principle,
which are equivalent to the Axiom of Choice. The next section develops
these equivalences directly.
\end{remark}
