% =========================================================
% Induced Orders and Order Embeddings
% =========================================================

\subsection{Induced Orders and Order Embeddings}

% ---------------------------------------------------------
% TOOLKIT
% ---------------------------------------------------------
\begin{tcolorbox}[colback=gray!6, colframe=gray!40, arc=2pt,
  left=6pt, right=6pt, top=4pt, bottom=4pt,
  title={\small\textbf{Induced Orders and Order Embeddings --- Quick Reference}},
  fonttitle=\small\bfseries]
\small
\begin{tabular}{l l l}
\toprule
\textbf{Concept} & \textbf{Meaning} & \textbf{Detail} \\
\midrule
Induced preorder    & $x \leq_f y \;\Leftrightarrow\; f(x) \leq' f(y)$; always a preorder  & \hyperref[def:induced-order]{↓ Def} \\
Induced partial order & Induced order is a partial order iff $f$ is injective              & \hyperref[prop:induced-poset]{↓ Prop} \\
$f$-indistinguishable & $x \sim_f y$: both $x \leq_f y$ and $y \leq_f x$                   & \hyperref[def:f-indist]{↓ Def} \\
Order embedding     & Injective, order-preserving, and order-reflecting                     & \hyperref[def:order-embedding]{↓ Def} \\
Embedding vs.\ isomorphism & Embedding is iso onto image; isomorphism is surjective too & \hyperref[prop:embedding-iso]{↓ Prop} \\
Suborder            & Restriction of $\leq'$ to a subset $S \subseteq B$                   & \hyperref[def:suborder]{↓ Def} \\
\midrule
\multicolumn{2}{l}{\textit{Key results:}} & \\
\midrule
\multicolumn{2}{l}{$\leq_f$ preorder: always. \quad Antisymmetry: iff $f$ injective.}       & \hyperref[prop:induced-poset]{↓} \\
\multicolumn{2}{l}{$f$ order embedding $\Rightarrow$ $(A,\leq_f) \cong (f(A), \leq')$.}    & \hyperref[prop:embedding-iso]{↓} \\
\bottomrule
\end{tabular}
\end{tcolorbox}

\vspace{1em}

% ---------------------------------------------------------
% The construction
% ---------------------------------------------------------

The simplest way to put an order on a set $A$ is to compare elements of $A$
\emph{indirectly} through a function into an already-ordered set. This
construction, called the \emph{induced order} or \emph{pullback order},
is ubiquitous: it explains how the usual order on $\mathbb{Z}$ restricts
to $\mathbb{N}$, how functions on a common domain acquire a pointwise order,
and how subsets inherit the order of the ambient space.

\begin{tcolorbox}[colback=propbox, colframe=propborder, arc=2pt,
  left=6pt, right=6pt, top=4pt, bottom=4pt,
  title={\small\textbf{Definition (Induced Order)}},
  fonttitle=\small\bfseries]
\label{def:induced-order}
Let $(B, \leq')$ be a partially ordered set and let $f : A \to B$ be a
function. The \emph{order induced by $f$} (or \emph{pullback order along
$f$}) is the relation $\leq_f$ on $A$ defined by:
\[
x \;\leq_f\; y
\;\;\Longleftrightarrow\;\;
f(x) \;\leq'\; f(y).
\]
\end{tcolorbox}

\begin{remark}[Intuition]
Two elements of $A$ are compared by sending them through $f$ and comparing
their images in $(B, \leq')$. The order on $A$ is entirely inherited from
$B$ via $f$: we never compare elements of $A$ directly, only their
$f$-values.
\end{remark}

\begin{remark}[Fully quantified form]
$\forall x, y \in A,\quad x \leq_f y \;\Longleftrightarrow\; f(x) \leq' f(y)$.
\end{remark}

% ---------------------------------------------------------
% Reflexivity and transitivity: always free
% ---------------------------------------------------------

\begin{proposition}[$\leq_f$ is always a preorder]\label{prop:induced-preorder}
For any function $f : A \to B$ and partial order $(B, \leq')$, the relation
$\leq_f$ is a preorder on $A$: it is reflexive and transitive.
\end{proposition}



\begin{remark}[Proof strategy]
Both properties are inherited directly from $(B,\leq')$ by pulling back
through $f$: reflexivity at $x$ uses reflexivity at $f(x)$, and transitivity
along $x,y,z$ uses transitivity along $f(x),f(y),f(z)$.
\end{remark}

% ---------------------------------------------------------
% Antisymmetry: requires injectivity
% ---------------------------------------------------------

\begin{tcolorbox}[colback=propbox, colframe=propborder, arc=2pt,
  left=6pt, right=6pt, top=4pt, bottom=4pt,
  title={\small\textbf{Definition ($f$-Indistinguishable Elements)}},
  fonttitle=\small\bfseries]
\label{def:f-indist}
Let $f : A \to B$ and $\leq_f$ be the induced order. Two elements
$x, y \in A$ are \emph{$f$-indistinguishable}, written $x \sim_f y$,
if both $x \leq_f y$ and $y \leq_f x$, i.e.\ if
\[
f(x) \leq' f(y) \quad \text{and} \quad f(y) \leq' f(x).
\]
\end{tcolorbox}

\begin{remark}[Intuition]
Two elements are $f$-indistinguishable when they map to the same position
in the $\leq'$-order — not necessarily to the same \emph{point}, but to
mutually comparable points. When $\leq'$ is a partial order (antisymmetric),
$x \sim_f y$ forces $f(x) = f(y)$, making $f$ non-injective the only
obstruction to antisymmetry.
\end{remark}

\begin{proposition}[$\leq_f$ is a partial order iff $f$ is injective]\label{prop:induced-poset}
Let $(B, \leq')$ be a partially ordered set and $f : A \to B$.
The induced order $\leq_f$ is a partial order on $A$ if and only if
$f$ is injective.
\end{proposition}



\begin{remark}[Common error]
It is tempting to think the induced order is automatically a partial order
``because $\leq'$ is one.'' It is not. The issue is antisymmetry: if $f$
collapses two distinct points $x \neq y$ to the same or comparable images,
we get $x \leq_f y$ and $y \leq_f x$ with $x \neq y$, violating
antisymmetry. Reflexivity and transitivity lift freely; antisymmetry does
not.
\end{remark}

\begin{example}[Non-injective $f$ destroys antisymmetry]
Let $B = (\mathbb{N}, \leq)$ and let $f : \{a,b\} \to \mathbb{N}$ be the
constant function $f(a) = f(b) = 0$. Then $a \leq_f b$ (since $0 \leq 0$)
and $b \leq_f a$ (since $0 \leq 0$), but $a \neq b$. So $\leq_f$ is not
antisymmetric, confirming that non-injectivity forces a failure.
\end{example}

\begin{example}[Injective $f$ gives a genuine partial order]
Let $B = (\mathcal{P}(\{1,2,3\}), \subseteq)$ and let
$A = \{x, y, z\}$ with $f(x) = \{1\}$, $f(y) = \{2\}$, $f(z) = \{1,2\}$.
Then $f$ is injective. The induced order has $x \leq_f z$ (since
$\{1\} \subseteq \{1,2\}$) and $y \leq_f z$ (since $\{2\} \subseteq \{1,2\}$),
but $x$ and $y$ are incomparable. This is a genuine partial order on $A$.
\end{example}

% ---------------------------------------------------------
% Suborders
% ---------------------------------------------------------

\begin{tcolorbox}[colback=propbox, colframe=propborder, arc=2pt,
  left=6pt, right=6pt, top=4pt, bottom=4pt,
  title={\small\textbf{Definition (Suborder / Restriction)}},
  fonttitle=\small\bfseries]
\label{def:suborder}
Let $(B, \leq')$ be a partially ordered set and $S \subseteq B$. The
\emph{suborder} on $S$ (or \emph{induced order on $S$}) is the relation
$\leq'_S$ defined by:
\[
x \;\leq'_S\; y
\;\;\Longleftrightarrow\;\;
x \leq' y,
\quad \text{for } x, y \in S.
\]
\end{tcolorbox}

\begin{remark}[Connection to induced orders]
The suborder on $S \subseteq B$ is exactly the order induced by the
inclusion map $\iota : S \hookrightarrow B$, $\iota(x) = x$. Since $\iota$
is injective, the suborder is always a partial order whenever $(B,\leq')$
is. This justifies speaking of ``$S$ with the inherited order'' without
further verification.
\end{remark}

% ---------------------------------------------------------
% Order embeddings
% ---------------------------------------------------------

\begin{tcolorbox}[colback=propbox, colframe=propborder, arc=2pt,
  left=6pt, right=6pt, top=4pt, bottom=4pt,
  title={\small\textbf{Definition (Order Embedding)}},
  fonttitle=\small\bfseries]
\label{def:order-embedding}
Let $(A, \leq)$ and $(B, \leq')$ be partially ordered sets. A function
$f : A \to B$ is an \emph{order embedding} if for all $x, y \in A$:
\[
x \;\leq\; y
\;\;\Longleftrightarrow\;\;
f(x) \;\leq'\; f(y).
\]
\end{tcolorbox}

\begin{remark}[Two directions, one condition]
The $(\Rightarrow)$ direction says $f$ is order-preserving
(see \hyperref[def:order-pres]{Definition (Order-Preserving Map)}). The
$(\Leftarrow)$ direction --- $f(x) \leq' f(y) \Rightarrow x \leq y$ ---
is called \emph{order-reflection}. An order embedding both preserves and
reflects the order, so $f$ is a perfect local copy of $(A,\leq)$ inside
$(B,\leq')$.
\end{remark}

\begin{remark}[Relation to the induced order]
If $f : A \to B$ is an order embedding from $(A, \leq)$ to $(B, \leq')$,
then $\leq$ coincides with the induced order $\leq_f$:
\[
x \leq y \;\Longleftrightarrow\; f(x) \leq' f(y) \;\Longleftrightarrow\; x \leq_f y.
\]
In other words, an order embedding is exactly an injective function whose
induced order \emph{agrees with the existing order on $A$}.
\end{remark}

\begin{proposition}[Order embeddings are injective]\label{prop:embedding-injective}
Every order embedding $f : (A, \leq) \to (B, \leq')$ is injective.
\end{proposition}



\begin{remark}[Proof strategy]
Injectivity falls out of the reflection direction: if two elements have the
same image, reflection forces them to be mutually $\leq$-related, and
antisymmetry collapses them to equality.
\end{remark}

\begin{proposition}[Order embedding is isomorphism onto image]\label{prop:embedding-iso}
If $f : (A, \leq) \to (B, \leq')$ is an order embedding, then $f$ is an
order isomorphism from $(A, \leq)$ to the suborder $(f(A), \leq'_{f(A)})$.
\end{proposition}



\begin{remark}[Consequence]
This proposition is the order-theoretic analogue of the first isomorphism
theorem: an order embedding always realises $(A,\leq)$ as an isomorphic
copy of a sub-poset of $(B,\leq')$. When $f$ is also surjective, the
embedding becomes a full order isomorphism
(see \hyperref[def:order-iso]{Definition (Order Isomorphism)}).
\end{remark}

\begin{example}[Standard embeddings]
Each of the following is an order embedding:
\begin{enumerate}[label=(\roman*)]
  \item The inclusion $\mathbb{N} \hookrightarrow \mathbb{Z}$ with standard
        $\leq$ on both: $m \leq n \iff m \leq n$ trivially. The natural
        numbers sit inside the integers as an isomorphic copy of $(\mathbb{N},\leq)$.
  \item The function $f : \mathbb{N} \to \mathbb{N}$, $f(n) = 2n$, embeds
        $(\mathbb{N},\leq)$ into itself: $m \leq n \iff 2m \leq 2n$.
        The image is the even numbers with the inherited order.
  \item The map $A \mapsto A$ from $(\mathcal{P}(X), \subseteq)$ to itself
        is an order isomorphism (the identity), but any injective
        $g : \mathcal{P}(X) \to \mathcal{P}(X)$ that preserves and reflects
        $\subseteq$ is an order embedding.
\end{enumerate}
\end{example}

\begin{remark}[Transition]
Induced orders and embeddings are the order-theoretic counterpart of
substructures in algebra. Just as a subgroup is a subset closed under the
group operations, a sub-poset is a subset with the inherited order ---
and an order embedding is the morphism that witnesses this. In analysis,
these ideas appear when restricting an order from $\mathbb{R}$ to subsets
(intervals, sequences, function spaces), and when the completeness property
is transferred between ordered structures.
\end{remark}
