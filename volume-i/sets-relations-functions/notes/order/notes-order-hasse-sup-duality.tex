% =========================================================
% Hasse Diagrams, Supremum and Infimum, and the Duality Principle
% =========================================================

\subsection{Hasse Diagrams, Supremum and Infimum, and the Duality Principle}

% ---------------------------------------------------------
% TOOLKIT
% ---------------------------------------------------------
\begin{tcolorbox}[colback=gray!6, colframe=gray!40, arc=2pt,
  left=6pt, right=6pt, top=4pt, bottom=4pt,
  title={\small\textbf{Hasse Diagrams, Sup/Inf, Duality — Quick Reference}},
  fonttitle=\small\bfseries]
\small
\begin{tabular}{l l l}
\toprule
\textbf{Concept} & \textbf{Meaning} & \textbf{Detail} \\
\midrule
Hasse diagram      & Graph encoding a poset; edges only for \emph{cover} relations        & \hyperref[def:hasse]{↓ Def} \\
Cover relation     & $a \lessdot b$: $a < b$ with nothing strictly between                 & \hyperref[def:cover]{↓ Def} \\
Supremum (join)    & Least upper bound: $\sup S = \min\,\{\,u \mid u \geq s\ \forall s\in S\,\}$ & \hyperref[def:sup-inf]{↓ Def} \\
Infimum (meet)     & Greatest lower bound: $\inf S = \max\,\{\,\ell \mid \ell \leq s\ \forall s\in S\,\}$ & \hyperref[def:sup-inf]{↓ Def} \\
Dual poset         & $(A, \geq)$ obtained by reversing the order on $(A, \leq)$           & \hyperref[def:dual-poset]{↓ Def} \\
Duality principle  & Every theorem about posets yields a theorem in the dual              & \hyperref[thm:duality]{↓ Prop} \\
\bottomrule
\end{tabular}
\end{tcolorbox}

\vspace{1em}

% ---------------------------------------------------------
% Hasse Diagrams
% ---------------------------------------------------------

\begin{tcolorbox}[colback=propbox, colframe=propborder, arc=2pt,
  left=6pt, right=6pt, top=4pt, bottom=4pt,
  title={\small\textbf{Definition (Cover Relation)}},
  fonttitle=\small\bfseries]
\label{def:cover}
Let $(A, \leq)$ be a poset and $a, b \in A$. We say $b$ \emph{covers} $a$,
written $a \lessdot b$, if $a < b$ and there is no $c \in A$ with $a < c < b$.
\end{tcolorbox}

\begin{remark}[Intuition]
The cover relation strips the partial order down to its ``essential edges.''
If $a \lessdot b$, then $b$ is the immediate successor of $a$: you cannot
insert anything between them. In a finite poset every strict relation
$a < b$ factors as a finite chain of cover steps.
\end{remark}

\begin{tcolorbox}[colback=propbox, colframe=propborder, arc=2pt,
  left=6pt, right=6pt, top=4pt, bottom=4pt,
  title={\small\textbf{Definition (Hasse Diagram)}},
  fonttitle=\small\bfseries]
\label{def:hasse}
The \emph{Hasse diagram} of a finite poset $(A, \leq)$ is the directed graph
whose vertices are the elements of $A$, with an upward edge from $a$ to $b$
whenever $a \lessdot b$. Three graphical conventions apply:
\begin{enumerate}[label=(\roman*)]
  \item Reflexive loops ($a \leq a$) are suppressed.
  \item Edges implied by transitivity are suppressed: if $a \lessdot b$ and
        $b \lessdot c$, no direct edge from $a$ to $c$ is drawn.
  \item Direction is encoded by height: $a \lessdot b$ is drawn with $b$
        strictly above $a$, so arrowheads are unnecessary.
\end{enumerate}
\end{tcolorbox}

\begin{remark}[Why Hasse diagrams work]
Because we keep only cover edges, the diagram is uncluttered while still
encoding the full order: $a \leq b$ holds in $(A,\leq)$ if and only if there
is an upward path in the Hasse diagram from $a$ to $b$ (or $a = b$).
\end{remark}

\begin{example}[Divisibility on $\{1,2,3,4,6,12\}$]
Order the set $D = \{1, 2, 3, 4, 6, 12\}$ by divisibility ($a \leq b$ iff
$a \mid b$). The covers are:
\[
  1 \lessdot 2,\quad
  1 \lessdot 3,\quad
  2 \lessdot 4,\quad
  2 \lessdot 6,\quad
  3 \lessdot 6,\quad
  4 \lessdot 12,\quad
  6 \lessdot 12.
\]
The Hasse diagram places $1$ at the bottom, $12$ at the top, $2$ and $3$
on the next level up from $1$, $4$ and $6$ above those, and $12$ at the
apex. The edge $1 \to 4$ is \emph{not} drawn because $1 \lessdot 2 \lessdot 4$
already connects them through $2$.
\end{example}

\begin{example}[Power set $\mathcal{P}(\{a,b,c\})$ under inclusion]
The Hasse diagram has four levels: $\varnothing$ at the bottom; the three
singletons $\{a\},\{b\},\{c\}$ one level up; the three two-element sets
above those; and $\{a,b,c\}$ at the top. Each singleton is covered by the
two two-element sets containing it. This diagram is the Boolean lattice
$B_3$.
\end{example}

\begin{remark}[Limitation to finite posets]
For infinite posets such as $(\mathbb{N}, \leq)$, the cover relation is
well-defined ($n \lessdot n+1$) but the full diagram cannot be drawn. In
analysis, Hasse diagrams serve as a conceptual aid for finite or
schematic examples rather than as a proof tool.
\end{remark}

% ---------------------------------------------------------
% Supremum and Infimum (abstract)
% ---------------------------------------------------------

\begin{tcolorbox}[colback=propbox, colframe=propborder, arc=2pt,
  left=6pt, right=6pt, top=4pt, bottom=4pt,
  title={\small\textbf{Definition (Supremum and Infimum)}},
  fonttitle=\small\bfseries]
\label{def:sup-inf}
Let $(A, \leq)$ be a poset and $S \subseteq A$.

The \emph{supremum} (or \emph{least upper bound}) of $S$, written
$\sup S$, is an element $u^* \in A$ satisfying:
\begin{enumerate}[label=(\roman*)]
  \item $u^*$ is an upper bound of $S$: $\forall s \in S,\ s \leq u^*$;
  \item $u^*$ is least among upper bounds: if $u$ is any upper bound of
        $S$ then $u^* \leq u$.
\end{enumerate}

The \emph{infimum} (or \emph{greatest lower bound}) of $S$, written
$\inf S$, is an element $\ell^* \in A$ satisfying:
\begin{enumerate}[label=(\roman*)]
  \item $\ell^*$ is a lower bound of $S$: $\forall s \in S,\ \ell^* \leq s$;
  \item $\ell^*$ is greatest among lower bounds: if $\ell$ is any lower
        bound of $S$ then $\ell \leq \ell^*$.
\end{enumerate}

When they exist, $\sup S$ and $\inf S$ are unique.
\end{tcolorbox}

\begin{remark}[Uniqueness]
If $u^*$ and $v^*$ are both suprema of $S$, then $u^*$ is an upper bound
so $v^* \leq u^*$, and $v^*$ is an upper bound so $u^* \leq v^*$.
Antisymmetry gives $u^* = v^*$. The same argument applies to infima.
\end{remark}

\begin{remark}[Sup and inf need not lie in $S$]
The set $S = (0,1) \subseteq \mathbb{R}$ has $\sup S = 1$ and
$\inf S = 0$, neither of which belongs to $S$. Contrast with
$\max S$ and $\min S$, which are the supremum and infimum when they
happen to lie \emph{in} $S$. Every maximum is a supremum, but not
conversely.
\end{remark}

\begin{remark}[Sup and inf need not exist]
In the poset $(\mathbb{Q}, \leq)$, the set
$\{r \in \mathbb{Q} : r^2 < 2\}$ is bounded above (e.g.\ by $2$) but
has no supremum in $\mathbb{Q}$: the candidate $\sqrt{2}$ is not
rational. This failure is precisely what the completeness axiom for
$\mathbb{R}$ addresses: it asserts that every nonempty subset of
$\mathbb{R}$ that is bounded above \emph{does} have a supremum in
$\mathbb{R}$.
\end{remark}

\begin{remark}[Connection to greatest and least elements]
$\sup S$ is the \emph{least element of the set of upper bounds of $S$}
in $A$; $\inf S$ is the \emph{greatest element of the set of lower
bounds of $S$} in $A$. This is why they are alternatively called the
\emph{join} and \emph{meet} of $S$ in lattice theory.
\end{remark}

\begin{example}[Sup and inf in the divisibility poset]
In the divisibility poset $(D, \mid)$ with $D = \{1,2,3,4,6,12\}$, take
$S = \{4, 6\}$. The upper bounds of $S$ (elements divisible by both $4$
and $6$) are $\{12\}$, so $\sup S = 12$. The lower bounds (elements
dividing both $4$ and $6$) are $\{1, 2\}$; the greatest of these is
$2$, so $\inf S = 2$. Note: $\sup S = \mathrm{lcm}(4,6)$ and
$\inf S = \gcd(4,6)$, revealing that $\gcd$ and $\mathrm{lcm}$ are
order-theoretic concepts.
\end{example}

% ---------------------------------------------------------
% Dual Poset and Duality Principle
% ---------------------------------------------------------

\begin{tcolorbox}[colback=propbox, colframe=propborder, arc=2pt,
  left=6pt, right=6pt, top=4pt, bottom=4pt,
  title={\small\textbf{Definition (Dual Poset)}},
  fonttitle=\small\bfseries]
\label{def:dual-poset}
Let $(A, \leq)$ be a poset. The \emph{dual poset} is $(A, \geq)$, where
$a \geq b$ is defined to mean $b \leq a$. The dual is also written
$(A, \leq^{\mathrm{op}})$.
\end{tcolorbox}

\begin{remark}[Intuition]
The dual poset is obtained by flipping the Hasse diagram upside down.
Every structural feature is preserved but reflected: what was at the
top is now at the bottom. Minimal elements become maximal, upper bounds
become lower bounds, and the supremum becomes the infimum.
\end{remark}

\begin{proposition}[Duality Principle]\label{thm:duality}
Let $\Phi$ be any first-order statement about a poset $(A, \leq)$
expressed using only the relation $\leq$. Let $\Phi^*$ be the statement
obtained from $\Phi$ by replacing every occurrence of $\leq$ with $\geq$
(equivalently, working in the dual poset). Then:
\[
  (A, \leq) \models \Phi
  \;\;\Longleftrightarrow\;\;
  (A, \geq) \models \Phi.
\]
In particular, if $\Phi$ is a theorem about all posets, then so is
$\Phi^*$.
\end{proposition}



\begin{remark}[Practical use]
The duality principle means theorems come in pairs at no extra cost.
Some instances you will use repeatedly in analysis:
\begin{itemize}
  \item \emph{Supremum $\leftrightarrow$ Infimum.} Every theorem about
        $\sup$ yields a theorem about $\inf$ by duality. For instance,
        if $\sup$ of a bounded-above set exists, duality (applied to the
        negation of the order) gives that $\inf$ of a bounded-below set
        exists — and this is exactly how the infimum is deduced from the
        completeness axiom for $\mathbb{R}$.
  \item \emph{Minimal $\leftrightarrow$ Maximal.} A statement about
        minimal elements in a poset dualises to a statement about maximal
        elements.
  \item \emph{Well-ordering.} $(\mathbb{N}, \leq)$ is well-ordered (every
        nonempty subset has a least element); the dual
        $(\mathbb{N}, \geq)$ is \emph{not} well-ordered (the whole set has
        no greatest element). Duality preserves the \emph{form} of the
        definition but does not preserve well-ordering; this asymmetry is
        a key feature, not a failure of duality.
\end{itemize}
\end{remark}

\begin{remark}[Transition]
With Hasse diagrams, $\sup$/$\inf$, and duality in hand, the order
vocabulary of this chapter is complete. The next payoff is in
Volume~II: when $\mathbb{R}$ is constructed, the completeness axiom
is the assertion that every nonempty bounded-above subset of $\mathbb{R}$
has a supremum — an order-theoretic condition that cannot be satisfied
by $\mathbb{Q}$, as the example above shows.
\end{remark}
