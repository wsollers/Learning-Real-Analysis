% =========================================================
% Ordered Sets
% =========================================================

\subsection{Ordered Sets}

% ---------------------------------------------------------
% TOOLKIT
% ---------------------------------------------------------
\begin{tcolorbox}[colback=gray!6, colframe=gray!40, arc=2pt,
  left=6pt, right=6pt, top=4pt, bottom=4pt,
  title={\small\textbf{Ordered Sets — Quick Reference}},
  fonttitle=\small\bfseries]
\small
\begin{tabular}{l l l}
\toprule
\textbf{Concept} & \textbf{Meaning} & \textbf{Detail} \\
\midrule
Ordered set        & $(A, \leq)$: set with a partial order                        & \hyperref[def:ordered-set]{↓ Def} \\
Strict order       & $a < b \iff a \leq b \land a \neq b$                         & \hyperref[def:strict-order]{↓ Def} \\
Comparable / incomparable & $a \leq b$ or $b \leq a$ / neither              & \hyperref[def:comparable]{↓ Def} \\
Total (linear) order & Partial order with all pairs comparable                  & \hyperref[def:linear-order]{↓ Def} \\
Upper / lower bound & $u \geq s$ / $\ell \leq s$ for all $s \in S$               & \hyperref[def:bounds]{↓ Def} \\
Minimal / maximal  & No smaller/larger element in $S$                            & \hyperref[def:min-max]{↓ Def} \\
Least / greatest   & Smaller/larger than all elements of $S$                     & \hyperref[def:least-greatest]{↓ Def} \\
Order-preserving   & $a \leq b \Rightarrow f(a) \leq' f(b)$                       & \hyperref[def:order-pres]{↓ Def} \\
Order isomorphism  & Bijective order-preserving map with order-preserving inverse & \hyperref[def:order-iso]{↓ Def} \\
Well-ordered set   & Every nonempty subset has a least element                   & \hyperref[def:well-order]{↓ Def} \\
Chain / antichain  & All comparable / none comparable                            & \hyperref[def:chain]{↓ Def} \\
Initial segment    & Downward-closed subset                                      & \hyperref[def:initial-segment]{↓ Def} \\
\bottomrule
\end{tabular}
\end{tcolorbox}

\vspace{1em}

\begin{tcolorbox}[colback=propbox, colframe=propborder, arc=2pt,
  left=6pt, right=6pt, top=4pt, bottom=4pt,
  title={\small\textbf{Definition (Ordered Set)}},
  fonttitle=\small\bfseries]
\label{def:ordered-set}
An \emph{ordered set} (or \emph{partially ordered set}, \emph{poset}) is a
pair $(A,\leq)$ where $\leq$ is a partial order on $A$: a relation that is
reflexive, antisymmetric, and transitive.
\end{tcolorbox}

\begin{remark}[English reading]
An ordered set is a set equipped with a specified way to compare elements,
but without requiring all pairs to be comparable.
\end{remark}

\begin{tcolorbox}[colback=propbox, colframe=propborder, arc=2pt,
  left=6pt, right=6pt, top=4pt, bottom=4pt,
  title={\small\textbf{Definition (Strict Order)}},
  fonttitle=\small\bfseries]
\label{def:strict-order}
Let $(A,\leq)$ be an ordered set. The \emph{strict order} $<$ is defined by:
\[
a < b \;\;\Longleftrightarrow\;\; (a \leq b \land a \neq b).
\]
\end{tcolorbox}

\begin{remark}[Strict and non-strict are equivalent]
The relation $<$ is irreflexive and transitive. Conversely, given any strict
partial order $<$, one recovers a non-strict order by setting
$a \leq b \iff (a < b \lor a=b)$. The two formulations carry exactly the
same information.
\end{remark}

\begin{definition}[Comparable and Incomparable Elements]\label{def:comparable}
In an ordered set $(A,\leq)$, elements $a,b \in A$ are \emph{comparable} if
$a \leq b$ or $b \leq a$; they are \emph{incomparable} if neither holds.
\end{definition}

\begin{remark}[Incomparable elements]
Incomparable elements can only occur in partial orders. In a total order,
every pair is comparable by definition.
\end{remark}

\begin{tcolorbox}[colback=propbox, colframe=propborder, arc=2pt,
  left=6pt, right=6pt, top=4pt, bottom=4pt,
  title={\small\textbf{Definition (Total / Linear Order)}},
  fonttitle=\small\bfseries]
\label{def:linear-order}
An ordered set $(A,\leq)$ is a \emph{total order} (or \emph{linear order})
if every pair of elements is comparable:
\[
\forall a,b \in A,\; a \leq b \lor b \leq a.
\]
\end{tcolorbox}

\begin{remark}[Examples]
$(\mathbb{R},\leq)$ is a total order. $(\mathcal{P}(A), \subseteq)$ is a
partial order that is generally not total.
\end{remark}

% ---------------------------------------------------------
% Bounds and extremal elements
% ---------------------------------------------------------
\begin{tcolorbox}[colback=propbox, colframe=propborder, arc=2pt,
  left=6pt, right=6pt, top=4pt, bottom=4pt,
  title={\small\textbf{Definition (Upper and Lower Bounds)}},
  fonttitle=\small\bfseries]
\label{def:bounds}
Let $(A,\leq)$ be an ordered set and $S \subseteq A$.

An element $u \in A$ is an \emph{upper bound} of $S$ if $\forall s \in S,\; s \leq u$.

An element $\ell \in A$ is a \emph{lower bound} of $S$ if $\forall s \in S,\; \ell \leq s$.
\end{tcolorbox}

\begin{remark}[Consequence]
Bounds need not exist and need not be unique. They need not lie in $S$ itself.
Bounds are central to the definition of completeness for ordered fields.
\end{remark}

\begin{tcolorbox}[colback=propbox, colframe=propborder, arc=2pt,
  left=6pt, right=6pt, top=4pt, bottom=4pt,
  title={\small\textbf{Definition (Minimal, Maximal, Least, Greatest Elements)}},
  fonttitle=\small\bfseries]
\label{def:min-max}
Let $S \subseteq A$ in an ordered set $(A,\leq)$.

$m \in S$ is a \emph{minimal element} of $S$ if $\nexists s \in S$ with $s < m$.

$M \in S$ is a \emph{maximal element} of $S$ if $\nexists s \in S$ with $M < s$.

\smallskip
\label{def:least-greatest}
$\ell \in S$ is the \emph{least element} if $\forall s \in S,\; \ell \leq s$.

$g \in S$ is the \emph{greatest element} if $\forall s \in S,\; s \leq g$.
\end{tcolorbox}

\begin{remark}[Minimal vs.\ least]
A \emph{least} element is below every element of $S$; it must be unique if it
exists. A \emph{minimal} element merely has nothing below it within $S$; there
may be many. Every least element is minimal, but not conversely.
\end{remark}

% ---------------------------------------------------------
% Order maps and isomorphisms
% ---------------------------------------------------------
\begin{tcolorbox}[colback=propbox, colframe=propborder, arc=2pt,
  left=6pt, right=6pt, top=4pt, bottom=4pt,
  title={\small\textbf{Definition (Order-Preserving Map and Isomorphism)}},
  fonttitle=\small\bfseries]
\label{def:order-pres}
Let $(M,\leq)$ and $(M',\leq')$ be partially ordered sets.

A function $f : M \to M'$ is \emph{order-preserving} (or \emph{monotone}) if
\[
\forall a,b \in M,\quad a \leq b \;\Longrightarrow\; f(a) \leq' f(b).
\]

$f$ is an \emph{order isomorphism}\label{def:order-iso} if it is bijective and
\[
\forall a,b \in M,\quad a \leq b \;\Longleftrightarrow\; f(a) \leq' f(b).
\]
\end{tcolorbox}

\begin{remark}[Order-isomorphic posets]
An order isomorphism preserves and reflects the order structure exactly,
including comparability, minimal and maximal elements, and bounds. Two posets
are \emph{order-isomorphic} if such a map exists; they are structurally
identical from the order-theoretic viewpoint.
\end{remark}

% ---------------------------------------------------------
% Well-order
% ---------------------------------------------------------
\begin{tcolorbox}[colback=propbox, colframe=propborder, arc=2pt,
  left=6pt, right=6pt, top=4pt, bottom=4pt,
  title={\small\textbf{Definition (Well-Ordered Set)}},
  fonttitle=\small\bfseries]
\label{def:well-order}
An ordered set $(A,<)$ is \emph{well-ordered} if every nonempty subset
$S \subseteq A$ has a least element:
\[
\forall S \subseteq A,\;
\bigl(S \neq \varnothing \;\Rightarrow\; \exists m \in S \text{ s.t.\ }
m \leq s \;\forall s \in S\bigr).
\]
\end{tcolorbox}

\begin{remark}[Well-order implies total order]
Every well-ordered set is totally ordered. The well-ordering condition is
strictly stronger: it requires a least element in every nonempty subset,
not just in $A$ as a whole.
\end{remark}

\begin{example}[Well-order examples]
$(\mathbb{N},\leq)$ is well-ordered. $(\mathbb{Z},\leq)$ is not: the set
$\{\dots,-3,-2,-1\}$ has no least element. $(\mathbb{R},\leq)$ is not:
$(0,1)$ has no least element.
\end{example}

\begin{remark}[Connection to ordinal numbers]
An \emph{ordinal number} is defined as an equivalence class of well-ordered
sets under order isomorphism: it measures the \emph{order type} of a
well-ordered set, not merely its cardinality.
\end{remark}

% ---------------------------------------------------------
% Chains, antichains, initial segments
% ---------------------------------------------------------
\begin{definition}[Chain and Antichain]\label{def:chain}
A subset $C \subseteq A$ of a poset $(A,\leq)$ is a \emph{chain} if every
pair of elements in $C$ is comparable. It is an \emph{antichain} if no two
distinct elements of $C$ are comparable.
\end{definition}

\begin{definition}[Initial Segment]\label{def:initial-segment}
A subset $I \subseteq A$ is an \emph{initial segment} of $(A,\leq)$ if
\[
a \in I \text{ and } b \leq a \;\Rightarrow\; b \in I.
\]
\end{definition}

\begin{remark}[Transition]
Ordered sets provide the abstract framework for order structures on the real
numbers, function spaces, and metric spaces. In later sections, order
interacts with topology and analysis through intervals, monotone functions,
and the completeness axiom for $\mathbb{R}$.
\end{remark}
