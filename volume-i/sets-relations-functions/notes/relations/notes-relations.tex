% =========================================================
% Ordered Pairs and Relations
% =========================================================

\subsection{Ordered Pairs and Relations}

% ---------------------------------------------------------
% TOOLKIT
% ---------------------------------------------------------
\begin{tcolorbox}[colback=gray!6, colframe=gray!40, arc=2pt,
  left=6pt, right=6pt, top=4pt, bottom=4pt,
  title={\small\textbf{Relations — Quick Reference}},
  fonttitle=\small\bfseries]
\small
\begin{tabular}{l l l}
\toprule
\textbf{Concept} & \textbf{Meaning} & \textbf{Detail} \\
\midrule
Ordered pair     & $(a,b) := \{\{a\},\{a,b\}\}$; identity iff both coords equal  & \hyperref[def:ordered-pair]{↓ Def} \\
Cartesian product & $A \times B$: all ordered pairs from $A$ and $B$             & \hyperref[def:cartesian-product-rel]{↓ Def} \\
Relation         & Any subset $R \subseteq A \times B$                           & \hyperref[def:relation]{↓ Def} \\
\bottomrule
\end{tabular}
\end{tcolorbox}

\vspace{1em}

\begin{tcolorbox}[colback=propbox, colframe=propborder, arc=2pt,
  left=6pt, right=6pt, top=4pt, bottom=4pt,
  title={\small\textbf{Definition (Ordered Pair)}},
  fonttitle=\small\bfseries]
\label{def:ordered-pair}
Let $a$ and $b$ be sets. The \emph{ordered pair} $(a,b)$ is defined (Kuratowski)
as
\[
(a,b) \;:=\; \{\{a\}, \{a,b\}\}.
\]
\end{tcolorbox}

\begin{theorem}[Uniqueness of Ordered Pairs]\label{thm:ordered-pair-unique}
For any sets $a,b,c,d$,
\[
(a,b) = (c,d)
\;\;\Longleftrightarrow\;\;
(a = c \land b = d).
\]
\end{theorem}



\begin{remark}[Why this encoding works]
The purpose of the Kuratowski definition is purely to guarantee the uniqueness
theorem above. Once $(a,b) = (c,d) \iff a=c \land b=d$ is established,
we may treat ordered pairs as a primitive with this property and forget the
encoding.
\end{remark}

\begin{tcolorbox}[colback=propbox, colframe=propborder, arc=2pt,
  left=6pt, right=6pt, top=4pt, bottom=4pt,
  title={\small\textbf{Definition (Cartesian Product — as foundation for relations)}},
  fonttitle=\small\bfseries]
\label{def:cartesian-product-rel}
The \emph{Cartesian product} of sets $A$ and $B$ is:
\[
A \times B
\;:=\;
\{\, (a,b) \mid a \in A \text{ and } b \in B \,\}.
\]
\end{tcolorbox}

\begin{tcolorbox}[colback=propbox, colframe=propborder, arc=2pt,
  left=6pt, right=6pt, top=4pt, bottom=4pt,
  title={\small\textbf{Definition (Relation)}},
  fonttitle=\small\bfseries]
\label{def:relation}
A \emph{relation} from $A$ to $B$ is any subset $R \subseteq A \times B$.

If $(a,b) \in R$ we write $a \, R \, b$ and say ``$a$ is related to $b$.''
When $A = B$ we say $R$ is a \emph{relation on $A$}.
\end{tcolorbox}

\begin{remark}[Relations as structured sets]
Relations introduce no new foundational objects: they are simply sets of
ordered pairs, constructed using operations already available. Properties of
relations can therefore be studied with ordinary set-theoretic reasoning.
\end{remark}
