% =========================================================
% Breadcrumb
% =========================================================
\begin{tcolorbox}[
  colback=gray!6,
  colframe=gray!40,
  arc=2pt,
  left=8pt, right=8pt, top=6pt, bottom=6pt,
  title={\small\textbf{Where You Are in the Journey}},
  fonttitle=\small\bfseries
]
\begin{center}
\small
Propositional Logic
$\;\to\;$ Predicate Calculus
$\;\to\;$ \textbf{Sets \& Functions}
$\;\to\;$ Proof Techniques
$\;\to\;$ Axiom Systems
$\;\to\;$ $\mathbb{N}$, $\mathbb{Z}$, $\mathbb{Q}$, $\mathbb{R}$
$\;\to\;$ Real Analysis
$\;\to\;$ $\cdots$
\end{center}

\medskip
\noindent\textbf{How we got here.}
Logic gave us a precise language. We now need mathematical objects to
reason \emph{about}. Sets are the universe of mathematical objects;
functions are the structure-preserving maps between them.
Together they provide the language of all subsequent mathematics.

\medskip
\noindent\textbf{What this chapter builds.}
We develop naïve and axiomatic set theory: membership, subsets,
operations, power sets, and families. Relations --- including equivalence
relations and partial orders --- give us structure on sets.
Functions capture how sets relate to each other: injections, surjections,
bijections, and cardinality.

\medskip
\noindent\textbf{Where this leads.}
Every algebraic structure is a set with additional operations.
Every proof about $\mathbb{N}$, $\mathbb{Z}$, $\mathbb{R}$ uses
the function and relation vocabulary established here.
Equivalence relations appear again in the integer and rational constructions;
order relations appear throughout analysis.
\end{tcolorbox}
\vspace{1em}

% =========================================================
% Structural Roadmap
% =========================================================
\subsection*{Structural Roadmap}

Each major topic is organised as:
\begin{center}
\textbf{Definitions $\longrightarrow$ Main Theorems
$\longrightarrow$ Consequences and Structural Insight}
\end{center}

The global progression is:
\begin{enumerate}
  \item Sets: membership, subsets, operations (union, intersection, difference),
        power sets, the empty set
  \item Relations: binary relations, properties (reflexive, symmetric, transitive),
        composition
  \item Families: indexed families, arbitrary unions and intersections
  \item Equivalence relations: equivalence classes, partitions, quotient sets
  \item Functions: domain, codomain, image, injection, surjection, bijection,
        composition, inverse
  \item Order: partial orders, total orders, well-orders, Zorn's lemma
\end{enumerate}

\vspace{1em}

\begin{remark}[Primary source]
The primary driver is Suppes, \textit{Introduction to Logic},
supplemented by Gerstein, \textit{Introduction to Mathematical Structures
and Proofs}.
\end{remark}

% =========================================================
% Content
% =========================================================
% notes/index.tex — Sets, Relations, and Functions
% sets/index.tex
% =========================================================
% Sets, Membership, and ZFC Axioms
% =========================================================

\subsection{Sets, Membership, and ZFC Axioms}

% ---------------------------------------------------------
% TOOLKIT
% ---------------------------------------------------------
\begin{tcolorbox}[colback=gray!6, colframe=gray!40, arc=2pt,
  left=6pt, right=6pt, top=4pt, bottom=4pt,
  title={\small\textbf{Foundations — Quick Reference}},
  fonttitle=\small\bfseries]
\small
\begin{tabular}{l l l}
\toprule
\textbf{Concept} & \textbf{Meaning} & \textbf{Detail} \\
\midrule
Set, membership        & Primitive notions; meaning fixed by axioms & \hyperref[def:set-membership]{↓ Def} \\
Extensionality         & Sets equal iff same elements & \hyperref[ax:extensionality]{↓ Ax} \\
Empty Set              & Exists a set with no elements & \hyperref[ax:empty-set]{↓ Ax} \\
Pairing                & Any two sets can be collected & \hyperref[ax:pairing]{↓ Ax} \\
Union                  & Elements of sets in a family & \hyperref[ax:union]{↓ Ax} \\
Power Set              & All subsets form a set & \hyperref[ax:power-set]{↓ Ax} \\
Infinity               & An infinite set exists & \hyperref[ax:infinity]{↓ Ax} \\
Separation             & Subsets by definable property & \hyperref[ax:separation]{↓ Ax} \\
Replacement            & Functional image of a set is a set & \hyperref[ax:replacement]{↓ Ax} \\
Foundation             & No infinite descending $\in$-chains & \hyperref[ax:foundation]{↓ Ax} \\
Choice                 & Selection function exists & \hyperref[ax:choice]{↓ Ax} \\
\bottomrule
\end{tabular}
\end{tcolorbox}

\vspace{1em}

% ---------------------------------------------------------
% Primitive Notions
% ---------------------------------------------------------
\begin{tcolorbox}[colback=propbox, colframe=propborder, arc=2pt,
  left=6pt, right=6pt, top=4pt, bottom=4pt,
  title={\small\textbf{Definition (Set and Membership)}},
  fonttitle=\small\bfseries]
\label{def:set-membership}
In axiomatic set theory, the notions of \emph{set} and \emph{membership} are
\emph{primitive}.

\begin{itemize}
  \item A \emph{set} is an object.
  \item \emph{Membership} is a binary relation, denoted by $\in$, between objects.
\end{itemize}

If $x$ is an object and $A$ is a set, the statement $x \in A$ is read as
``$x$ is an element of $A$.'' No definition of ``set'' or ``$\in$'' is given
in more basic terms. Their meaning is determined entirely by the axioms
governing them.
\end{tcolorbox}

\begin{remark}[English reading]
Primitive means we do not define these in terms of simpler concepts.
Rather, we fix their meaning implicitly by stating the axioms they must obey.
This is the standard approach in formal mathematics: rules of behaviour replace
informal definitions.
\end{remark}

\begin{remark}[Consequence]
All subsequent notions---subsets, ordered pairs, relations, functions,
number systems---are \emph{definitions} introduced within this axiomatic
framework. Every proved result is a \emph{theorem} derived logically from the
axioms via the rules of inference.
\end{remark}

% ---------------------------------------------------------
% ZFC AXIOMS — single axiom box
% ---------------------------------------------------------
\begin{tcolorbox}[colback=axiombox, colframe=axiomborder, arc=2pt,
  left=6pt, right=6pt, top=4pt, bottom=4pt,
  title={\small\textbf{Axiom System (ZFC)}},
  fonttitle=\small\bfseries]

\textbf{Axiom of Extensionality.}\label{ax:extensionality}\\
Two sets are equal iff they have the same elements.
\[
\forall A \, \forall B \,
\Bigl( A = B \iff \forall x \, (x \in A \leftrightarrow x \in B) \Bigr).
\]

\medskip
\textbf{Axiom of Empty Set.}\label{ax:empty-set}\\
There exists a set with no elements.
\[
\exists A \, \forall x \, (x \notin A).
\]

\medskip
\textbf{Axiom of Pairing.}\label{ax:pairing}\\
For any two sets, there exists a set containing exactly those two sets.
\[
\forall A \, \forall B \, \exists C \,
\forall x \, (x \in C \leftrightarrow (x = A \lor x = B)).
\]

\medskip
\textbf{Axiom of Union.}\label{ax:union}\\
For any family of sets, there exists a set containing exactly the elements of
those sets.
\[
\forall A \, \exists U \,
\forall x \,
\bigl( x \in U \leftrightarrow \exists B \, (B \in A \land x \in B) \bigr).
\]

\medskip
\textbf{Axiom of Power Set.}\label{ax:power-set}\\
For any set, there exists a set of all its subsets.
\[
\forall A \, \exists P \,
\forall x \, (x \in P \leftrightarrow x \subseteq A).
\]

\medskip
\textbf{Axiom of Infinity.}\label{ax:infinity}\\
There exists an infinite set.
\[
\exists A \,
\Bigl( \varnothing \in A \land
\forall x \, (x \in A \rightarrow x \cup \{x\} \in A) \Bigr).
\]

\medskip
\textbf{Axiom Schema of Separation.}\label{ax:separation}\\
Given a set and a property, there exists a subset containing exactly the
elements satisfying that property. For any formula $\varphi(x)$,
\[
\forall A \, \exists B \,
\forall x \, (x \in B \leftrightarrow (x \in A \land \varphi(x))).
\]

\medskip
\textbf{Axiom Schema of Replacement.}\label{ax:replacement}\\
If $\varphi(x,y)$ defines a functional relation on a set $A$, then the image
of $A$ under $\varphi$ is a set. For any formula $\varphi(x,y)$,
\[
\forall A\,
\Bigl(
\bigl(\forall x \in A\, \exists! y\, \varphi(x,y)\bigr)
\rightarrow
\exists B\, \forall y\,
\bigl(y \in B \leftrightarrow \exists x \in A\, \varphi(x,y)\bigr)
\Bigr).
\]

\medskip
\textbf{Axiom of Foundation.}\label{ax:foundation}\\
Every nonempty set has an $\in$-minimal element.
\[
\forall A \,
\Bigl( A \neq \varnothing \rightarrow
\exists x \in A \, (x \cap A = \varnothing) \Bigr).
\]

\medskip
\textbf{Axiom of Choice.}\label{ax:choice}\\
For any family of nonempty sets, there exists a selection function.
\[
\forall A \,
\Bigl(
(\forall B \in A \, (B \neq \varnothing))
\rightarrow
\exists f \,
\forall B \in A \, (f(B) \in B)
\Bigr).
\]
\end{tcolorbox}

\begin{remark}[Role of the axioms]
The axioms are not statements to be proved, but rules specifying how $\in$
behaves and which sets are permitted to exist. From this point onward, set
theory functions as the underlying language of mathematics: all reasoning
about mathematical objects is ultimately grounded in these axioms, even when
they are not cited explicitly.
\end{remark}

\begin{remark}[Separation vs.\ Replacement]
Separation is a schema: one axiom instance per definable property $\varphi$.
It carves out subsets of an already-existing set. Replacement is strictly
stronger: it can produce sets whose elements are not already contained in any
pre-existing set, allowing the construction of large stages of the cumulative
hierarchy.
\end{remark}

\begin{remark}[Axiom of Choice and right inverses]
The Axiom of Choice is equivalent to many statements used later in analysis,
including Zorn's Lemma, the well-ordering theorem, and the statement that every
surjective function has a right inverse. It is independent of the other ZFC
axioms.
\end{remark}

% =========================================================
% Derived Constructions and Set Operations
% =========================================================

\subsection{Set Constructions and Operations}

% ---------------------------------------------------------
% TOOLKIT
% ---------------------------------------------------------
\begin{tcolorbox}[colback=gray!6, colframe=gray!40, arc=2pt,
  left=6pt, right=6pt, top=4pt, bottom=4pt,
  title={\small\textbf{Set Operations — Quick Reference}},
  fonttitle=\small\bfseries]
\small
\begin{tabular}{l l l l}
\toprule
\textbf{Concept} & \textbf{Notation} & \textbf{Membership condition} & \textbf{Detail} \\
\midrule
Empty set        & $\varnothing$             & $x \in \varnothing$ never         & \hyperref[def:empty-set]{↓ Def} \\
Subset           & $A \subseteq B$           & $x \in A \Rightarrow x \in B$     & \hyperref[def:subset]{↓ Def} \\
Proper subset    & $A \subsetneq B$          & $A \subseteq B$ and $A \neq B$    & \hyperref[def:proper-subset]{↓ Def} \\
Set equality     & $A = B$                   & same elements both ways            & \hyperref[def:set-equality]{↓ Def} \\
Union            & $A \cup B$                & in $A$ or in $B$                  & \hyperref[def:union]{↓ Def} \\
Intersection     & $A \cap B$                & in $A$ and in $B$                 & \hyperref[def:intersection]{↓ Def} \\
Set difference   & $A \setminus B$           & in $A$ but not $B$                & \hyperref[def:set-difference]{↓ Def} \\
Symmetric diff.  & $A \triangle B$           & in exactly one of $A$, $B$        & \hyperref[def:sym-diff]{↓ Def} \\
Complement       & $A^c = U \setminus A$     & in $U$ but not $A$                & \hyperref[def:complement]{↓ Def} \\
Cartesian product& $A \times B$              & all ordered pairs $(a,b)$         & \hyperref[def:cartesian-product]{↓ Def} \\
Power set        & $\mathcal{P}(A)$          & all subsets of $A$                & \hyperref[def:power-set]{↓ Def} \\
De Morgan        & $(A \cup B)^c = A^c \cap B^c$ & complement of union        & \hyperref[thm:de-morgan]{↓ Thm} \\
\bottomrule
\end{tabular}
\end{tcolorbox}

\vspace{1em}

% ---------------------------------------------------------
% Empty Set
% ---------------------------------------------------------
\begin{tcolorbox}[colback=propbox, colframe=propborder, arc=2pt,
  left=6pt, right=6pt, top=4pt, bottom=4pt,
  title={\small\textbf{Definition (Empty Set)}},
  fonttitle=\small\bfseries]
\label{def:empty-set}
The unique set with no elements is called the \emph{empty set} and is denoted
$\varnothing$.
\end{tcolorbox}

\begin{remark}[Vacuous truth]
Statements of the form $\forall x \in \varnothing,\; P(x)$ are
\emph{vacuously true}: there is no element $x \in \varnothing$ for which
$P(x)$ could fail. This is not a special convention but a consequence of
how universal quantification is defined.
\end{remark}

% ---------------------------------------------------------
% Subset and Set Equality
% ---------------------------------------------------------
\begin{tcolorbox}[colback=propbox, colframe=propborder, arc=2pt,
  left=6pt, right=6pt, top=4pt, bottom=4pt,
  title={\small\textbf{Definition (Subset)}},
  fonttitle=\small\bfseries]
\label{def:subset}
Let $A$ and $B$ be sets. We say $A$ is a \emph{subset} of $B$, written
$A \subseteq B$, if every element of $A$ is also an element of $B$:
\[
A \subseteq B \;\;\Longleftrightarrow\;\; \forall x \, (x \in A \rightarrow x \in B).
\]
\end{tcolorbox}

\begin{tcolorbox}[colback=propbox, colframe=propborder, arc=2pt,
  left=6pt, right=6pt, top=4pt, bottom=4pt,
  title={\small\textbf{Definition (Proper Subset)}},
  fonttitle=\small\bfseries]
\label{def:proper-subset}
$A$ is a \emph{proper subset} of $B$, written $A \subsetneq B$, if
$A \subseteq B$ and $A \neq B$.
\end{tcolorbox}

\begin{remark}[Notation convention]
Some authors write $A \subset B$ for a proper subset; others use it to mean
$A \subseteq B$. In these notes, $\subseteq$ always denotes subset (possibly
equal) and $\subsetneq$ always denotes proper subset.
\end{remark}

\begin{tcolorbox}[colback=propbox, colframe=propborder, arc=2pt,
  left=6pt, right=6pt, top=4pt, bottom=4pt,
  title={\small\textbf{Definition (Set Equality)}},
  fonttitle=\small\bfseries]
\label{def:set-equality}
Two sets $A$ and $B$ are \emph{equal}, written $A = B$, if they have the same
elements:
\[
A = B \;\;\Longleftrightarrow\;\; \forall x \, (x \in A \leftrightarrow x \in B).
\]
Equivalently, $A = B \iff (A \subseteq B \land B \subseteq A)$.
\end{tcolorbox}

\begin{remark}[Proof strategy]
In practice, to prove $A = B$ one shows mutual inclusion: first $A \subseteq B$
(take arbitrary $x \in A$ and deduce $x \in B$), then $B \subseteq A$.
This two-step structure appears throughout set-theoretic proofs.
\end{remark}

% ---------------------------------------------------------
% Set Operations
% ---------------------------------------------------------
\begin{tcolorbox}[colback=propbox, colframe=propborder, arc=2pt,
  left=6pt, right=6pt, top=4pt, bottom=4pt,
  title={\small\textbf{Definition (Union)}},
  fonttitle=\small\bfseries]
\label{def:union}
The \emph{union} of $A$ and $B$, denoted $A \cup B$, is the set of all
elements belonging to at least one of the two sets:
\[
A \cup B \;=\; \{\, x \mid x \in A \lor x \in B \,\}.
\]
\end{tcolorbox}

\begin{center}
\begin{tikzpicture}[scale=0.85]
  \fill[pattern=north east lines] (-1,0) circle (1.5);
  \fill[pattern=north east lines] ( 1,0) circle (1.5);
  \draw (-1,0) circle (1.5);
  \draw ( 1,0) circle (1.5);
  \node at (-2.2,-1.6) {$A$};
  \node at ( 2.2,-1.6) {$B$};
\end{tikzpicture}
\end{center}

\begin{tcolorbox}[colback=propbox, colframe=propborder, arc=2pt,
  left=6pt, right=6pt, top=4pt, bottom=4pt,
  title={\small\textbf{Definition (Intersection)}},
  fonttitle=\small\bfseries]
\label{def:intersection}
The \emph{intersection} of $A$ and $B$, denoted $A \cap B$, is the set of all
elements common to both:
\[
A \cap B \;=\; \{\, x \mid x \in A \land x \in B \,\}.
\]
\end{tcolorbox}

\begin{center}
\begin{tikzpicture}[scale=0.85]
  \begin{scope}
    \clip (-1,0) circle (1.5);
    \fill[pattern=north east lines] (1,0) circle (1.5);
  \end{scope}
  \draw (-1,0) circle (1.5);
  \draw ( 1,0) circle (1.5);
  \node at (-2.2,-1.6) {$A$};
  \node at ( 2.2,-1.6) {$B$};
\end{tikzpicture}
\end{center}

\begin{tcolorbox}[colback=propbox, colframe=propborder, arc=2pt,
  left=6pt, right=6pt, top=4pt, bottom=4pt,
  title={\small\textbf{Definition (Set Difference)}},
  fonttitle=\small\bfseries]
\label{def:set-difference}
The \emph{set difference} $A \setminus B$ is the set of elements in $A$ but
not in $B$:
\[
A \setminus B \;=\; \{\, x \mid x \in A \land x \notin B \,\}.
\]
\end{tcolorbox}

\begin{center}
\begin{tikzpicture}[scale=0.85]
  \fill[pattern=north east lines] (-1,0) circle (1.5);
  \begin{scope}
    \clip (1,0) circle (1.5);
    \fill[white] (-1,0) circle (1.5);
  \end{scope}
  \draw (-1,0) circle (1.5);
  \draw ( 1,0) circle (1.5);
  \node at (-2.2,-1.6) {$A$};
  \node at ( 2.2,-1.6) {$B$};
\end{tikzpicture}
\end{center}

\begin{tcolorbox}[colback=propbox, colframe=propborder, arc=2pt,
  left=6pt, right=6pt, top=4pt, bottom=4pt,
  title={\small\textbf{Definition (Symmetric Difference)}},
  fonttitle=\small\bfseries]
\label{def:sym-diff}
The \emph{symmetric difference} $A \triangle B$ is the set of elements
belonging to exactly one of $A$ and $B$:
\[
A \triangle B
\;=\;
(A \setminus B) \cup (B \setminus A)
\;=\;
\{\, x \mid (x \in A \lor x \in B) \land x \notin A \cap B \,\}.
\]
\end{tcolorbox}

\begin{center}
\begin{tikzpicture}[scale=0.85]
  \fill[pattern=north east lines] (-1,0) circle (1.5);
  \fill[pattern=north east lines] ( 1,0) circle (1.5);
  \begin{scope}
    \clip (-1,0) circle (1.5);
    \fill[white] (1,0) circle (1.5);
  \end{scope}
  \draw (-1,0) circle (1.5);
  \draw ( 1,0) circle (1.5);
  \node at (-2.2,-1.6) {$A$};
  \node at ( 2.2,-1.6) {$B$};
\end{tikzpicture}
\end{center}

\begin{remark}[Algebraic structure of $\triangle$]
The symmetric difference is commutative and associative. Under $\triangle$,
the power set $\mathcal{P}(U)$ forms an abelian group with identity
$\varnothing$ and every element its own inverse ($A \triangle A = \varnothing$).
\end{remark}

\begin{tcolorbox}[colback=propbox, colframe=propborder, arc=2pt,
  left=6pt, right=6pt, top=4pt, bottom=4pt,
  title={\small\textbf{Definition (Complement)}},
  fonttitle=\small\bfseries]
\label{def:complement}
Let $U$ be a fixed universe and $A \subseteq U$. The \emph{complement} of $A$
relative to $U$, denoted $A^c$, is:
\[
A^c \;=\; U \setminus A.
\]
\end{tcolorbox}

\begin{tcolorbox}[colback=propbox, colframe=propborder, arc=2pt,
  left=6pt, right=6pt, top=4pt, bottom=4pt,
  title={\small\textbf{Definition (Cartesian Product)}},
  fonttitle=\small\bfseries]
\label{def:cartesian-product}
The \emph{Cartesian product} of sets $A$ and $B$ is:
\[
A \times B = \{\, (a,b) \mid a \in A \text{ and } b \in B \,\}.
\]
\end{tcolorbox}

\begin{remark}[Ordered pairs]
The ordered pair $(a,b)$ is formally defined as $\{\{a\},\{a,b\}\}$.
This Kuratowski encoding ensures the key property: $(a,b) = (c,d)$ iff
$a = c$ and $b = d$.
\end{remark}

\begin{remark}[Non-commutativity]
In general, $A \times B \neq B \times A$. The product is however associative
up to canonical isomorphism: $(A \times B) \times C \cong A \times (B \times C)$.
\end{remark}

\begin{tcolorbox}[colback=propbox, colframe=propborder, arc=2pt,
  left=6pt, right=6pt, top=4pt, bottom=4pt,
  title={\small\textbf{Definition (Power Set)}},
  fonttitle=\small\bfseries]
\label{def:power-set}
The \emph{power set} of $A$, denoted $\mathcal{P}(A)$, is the set of all
subsets of $A$:
\[
\mathcal{P}(A) \;:=\; \{\, S \mid S \subseteq A \,\}.
\]
\end{tcolorbox}

\begin{remark}[Size]
If $A$ has $n$ elements, then $|\mathcal{P}(A)| = 2^n$. For infinite $A$,
Cantor's theorem shows $|\mathcal{P}(A)| > |A|$ strictly.
\end{remark}

% ---------------------------------------------------------
% De Morgan — Theorem box (named, major result of section)
% ---------------------------------------------------------
\begin{tcolorbox}[colback=thmbox, colframe=thmborder, arc=2pt,
  left=6pt, right=6pt, top=4pt, bottom=4pt,
  title={\small\textbf{Theorem (De Morgan's Laws)}},
  fonttitle=\small\bfseries]
\label{thm:de-morgan}
Let $U$ be a universe and $A,B \subseteq U$. Then
\[
(A \cup B)^c = A^c \cap B^c
\quad\text{and}\quad
(A \cap B)^c = A^c \cup B^c.
\]
\end{tcolorbox}

\begin{proof}
We prove the first identity; the second follows by symmetry (or set duality).

Let $x \in U$. Then
\[
\begin{aligned}
x \in (A \cup B)^c
&\iff x \notin A \cup B \\
&\iff (x \notin A \land x \notin B) \\
&\iff (x \in A^c \land x \in B^c) \\
&\iff x \in A^c \cap B^c.
\end{aligned}
\]
Since membership is equivalent for all $x \in U$, the sets are equal.
\end{proof}

\begin{remark}[Logical analogy]
De Morgan's laws mirror the propositional logic identities $\neg(P \lor Q) \equiv \neg P \land \neg Q$ and $\neg(P \land Q) \equiv \neg P \lor \neg Q$ under the correspondence $\cap \leftrightarrow \land$, $\cup \leftrightarrow \lor$, $A^c \leftrightarrow \neg A$.
\end{remark}

\begin{definition}[Set Duality]\label{def:set-duality}
Two set-theoretic expressions over a fixed universe $U$ are \emph{dual} if one
is obtained from the other by simultaneously replacing
$\cup \leftrightarrow \cap$ and $\varnothing \leftrightarrow U$, with
complements unchanged.
\end{definition}

\begin{corollary}[Principle of Set Duality]\label{cor:set-duality}
Any identity involving $\cup$, $\cap$, $\varnothing$, and $U$ that holds for
all subsets of a universe remains valid when each operation and constant is
replaced by its dual.
\end{corollary}

\begin{remark}[Using duality in proofs]
To prove a statement involving unions and intersections, it often suffices to
prove one version; the dual statement follows immediately.
\end{remark}

\begin{example}[Distributive law via duality]
The two distributive laws
\[
A \cap (B \cup C) = (A \cap B) \cup (A \cap C)
\quad\text{and}\quad
A \cup (B \cap C) = (A \cup B) \cap (A \cup C)
\]
are duals of each other. Proving either one and applying the Principle of Set
Duality yields the other immediately.
\end{example}

\begin{remark}[Transition to relations]
The Cartesian product provides a way to encode ordered information.
Relations and functions will be defined as special subsets of Cartesian
products, so they require no new foundational objects.
\end{remark}

% =========================================================
% Algebraic Laws of Set Operations
% =========================================================

\subsection{Algebraic Laws of Set Operations}

% ---------------------------------------------------------
% TOOLKIT
% ---------------------------------------------------------
\begin{tcolorbox}[colback=gray!6, colframe=gray!40, arc=2pt,
  left=6pt, right=6pt, top=4pt, bottom=4pt,
  title={\small\textbf{Set Algebra Laws — Quick Reference}},
  fonttitle=\small\bfseries]
\small
\begin{tabular}{l l l l}
\toprule
\textbf{Law} & \textbf{Union form} & \textbf{Intersection form} & \textbf{Detail} \\
\midrule
Commutativity  & $A \cup B = B \cup A$          & $A \cap B = B \cap A$          & \hyperref[thm:commutativity]{↓ Thm} \\
Associativity  & $(A \cup B) \cup C = A \cup (B \cup C)$  & analogous   & \hyperref[thm:associativity]{↓ Thm} \\
Distributivity & $A \cap (B \cup C) = (A \cap B) \cup (A \cap C)$ & dual   & \hyperref[thm:distributivity]{↓ Thm} \\
Identity       & $A \cup \varnothing = A$        & $A \cap U = A$                 & \hyperref[thm:identity-absorption]{↓ Thm} \\
Absorption     & $A \cup (A \cap B) = A$         & $A \cap (A \cup B) = A$        & \hyperref[thm:identity-absorption]{↓ Thm} \\
Involution     & $(A^c)^c = A$                  & ---                            & \hyperref[thm:involution]{↓ Thm} \\
\bottomrule
\end{tabular}
\end{tcolorbox}

\vspace{1em}

\begin{remark}[Algebraic structure of set operations]
The operations $\cup$, $\cap$, and complement satisfy algebraic laws analogous
to those of logical connectives. Together they endow $\mathcal{P}(U)$ with
the structure of a \emph{Boolean algebra}. These laws justify manipulation
of set expressions in later proofs, particularly those involving equivalence
classes, partitions, and functions.
\end{remark}

\begin{theorem}[Commutativity of Union and Intersection]\label{thm:commutativity}
Let $A,B$ be sets. Then
\[
A \cup B = B \cup A
\quad\text{and}\quad
A \cap B = B \cap A.
\]
\end{theorem}



\begin{theorem}[Associativity of Union and Intersection]\label{thm:associativity}
Let $A,B,C$ be sets. Then
\[
(A \cup B) \cup C = A \cup (B \cup C)
\quad\text{and}\quad
(A \cap B) \cap C = A \cap (B \cap C).
\]
\end{theorem}



\begin{theorem}[Distributive Laws]\label{thm:distributivity}
Let $A,B,C$ be sets. Then
\[
A \cap (B \cup C) = (A \cap B) \cup (A \cap C),
\]
\[
A \cup (B \cap C) = (A \cup B) \cap (A \cup C).
\]
\end{theorem}



\begin{theorem}[Identity and Absorption Laws]\label{thm:identity-absorption}
Let $A,B$ be sets and $U$ a universe with $A \subseteq U$. Then
\[
A \cup \varnothing = A,
\qquad
A \cap U = A,
\]
\[
A \cup (A \cap B) = A,
\qquad
A \cap (A \cup B) = A.
\]
\end{theorem}

\begin{theorem}[Involution of Complement]\label{thm:involution}
For any $A \subseteq U$,
\[
(A^c)^c = A.
\]
\end{theorem}

\begin{remark}[Non-commutative and non-associative operations]
Set difference $\setminus$ is neither commutative nor associative:
\[
A \setminus B \neq B \setminus A,
\qquad
(A \setminus B) \setminus C \neq A \setminus (B \setminus C) \quad \text{in general}.
\]
It interacts with union and intersection via:
\[
A \setminus (B \cup C) = (A \setminus B) \cap (A \setminus C),
\qquad
A \setminus (B \cap C) = (A \setminus B) \cup (A \setminus C).
\]
These follow from $A \setminus B = A \cap B^c$ together with De Morgan's laws.
\end{remark}

\begin{remark}[Cartesian product]
The Cartesian product $\times$ is not commutative, is associative only up to
canonical isomorphism, and distributes over union in each coordinate:
\[
A \times (B \cup C) = (A \times B) \cup (A \times C).
\]
\end{remark}


% relations/index.tex
% =========================================================
% Ordered Pairs and Relations
% =========================================================

\subsection{Ordered Pairs and Relations}

% ---------------------------------------------------------
% TOOLKIT
% ---------------------------------------------------------
\begin{tcolorbox}[colback=gray!6, colframe=gray!40, arc=2pt,
  left=6pt, right=6pt, top=4pt, bottom=4pt,
  title={\small\textbf{Relations — Quick Reference}},
  fonttitle=\small\bfseries]
\small
\begin{tabular}{l l l}
\toprule
\textbf{Concept} & \textbf{Meaning} & \textbf{Detail} \\
\midrule
Ordered pair     & $(a,b) := \{\{a\},\{a,b\}\}$; identity iff both coords equal  & \hyperref[def:ordered-pair]{↓ Def} \\
Cartesian product & $A \times B$: all ordered pairs from $A$ and $B$             & \hyperref[def:cartesian-product-rel]{↓ Def} \\
Relation         & Any subset $R \subseteq A \times B$                           & \hyperref[def:relation]{↓ Def} \\
\bottomrule
\end{tabular}
\end{tcolorbox}

\vspace{1em}

\begin{tcolorbox}[colback=propbox, colframe=propborder, arc=2pt,
  left=6pt, right=6pt, top=4pt, bottom=4pt,
  title={\small\textbf{Definition (Ordered Pair)}},
  fonttitle=\small\bfseries]
\label{def:ordered-pair}
Let $a$ and $b$ be sets. The \emph{ordered pair} $(a,b)$ is defined (Kuratowski)
as
\[
(a,b) \;:=\; \{\{a\}, \{a,b\}\}.
\]
\end{tcolorbox}

\begin{theorem}[Uniqueness of Ordered Pairs]\label{thm:ordered-pair-unique}
For any sets $a,b,c,d$,
\[
(a,b) = (c,d)
\;\;\Longleftrightarrow\;\;
(a = c \land b = d).
\]
\end{theorem}



\begin{remark}[Why this encoding works]
The purpose of the Kuratowski definition is purely to guarantee the uniqueness
theorem above. Once $(a,b) = (c,d) \iff a=c \land b=d$ is established,
we may treat ordered pairs as a primitive with this property and forget the
encoding.
\end{remark}

\begin{tcolorbox}[colback=propbox, colframe=propborder, arc=2pt,
  left=6pt, right=6pt, top=4pt, bottom=4pt,
  title={\small\textbf{Definition (Cartesian Product — as foundation for relations)}},
  fonttitle=\small\bfseries]
\label{def:cartesian-product-rel}
The \emph{Cartesian product} of sets $A$ and $B$ is:
\[
A \times B
\;:=\;
\{\, (a,b) \mid a \in A \text{ and } b \in B \,\}.
\]
\end{tcolorbox}

\begin{tcolorbox}[colback=propbox, colframe=propborder, arc=2pt,
  left=6pt, right=6pt, top=4pt, bottom=4pt,
  title={\small\textbf{Definition (Relation)}},
  fonttitle=\small\bfseries]
\label{def:relation}
A \emph{relation} from $A$ to $B$ is any subset $R \subseteq A \times B$.

If $(a,b) \in R$ we write $a \, R \, b$ and say ``$a$ is related to $b$.''
When $A = B$ we say $R$ is a \emph{relation on $A$}.
\end{tcolorbox}

\begin{remark}[Relations as structured sets]
Relations introduce no new foundational objects: they are simply sets of
ordered pairs, constructed using operations already available. Properties of
relations can therefore be studied with ordinary set-theoretic reasoning.
\end{remark}

% =========================================================
% Properties of Relations
% =========================================================

\subsection{Properties of Relations}

% ---------------------------------------------------------
% TOOLKIT
% ---------------------------------------------------------
\begin{tcolorbox}[colback=gray!6, colframe=gray!40, arc=2pt,
  left=6pt, right=6pt, top=4pt, bottom=4pt,
  title={\small\textbf{Relational Properties — Quick Reference}},
  fonttitle=\small\bfseries]
\small
\begin{tabular}{l l l l}
\toprule
\textbf{Property} & \textbf{Formal definition} & \textbf{Canonical example} & \textbf{Detail} \\
\midrule
Reflexive      & $\forall a,\; (a,a)\in R$                           & $=$ on any set      & \hyperref[def:reflexive]{↓ Def} \\
Irreflexive    & $\forall a,\; (a,a)\notin R$                        & $<$ on $\mathbb{N}$ & \hyperref[def:irreflexive]{↓ Def} \\
Symmetric      & $(a,b)\in R \Rightarrow (b,a)\in R$                 & ``same age as''     & \hyperref[def:symmetric]{↓ Def} \\
Antisymmetric  & $(a,b),(b,a)\in R \Rightarrow a=b$                  & $\leq$ on $\mathbb{N}$ & \hyperref[def:antisymmetric]{↓ Def} \\
Asymmetric     & $(a,b)\in R \Rightarrow (b,a)\notin R$              & $<$ on $\mathbb{N}$ & \hyperref[def:asymmetric]{↓ Def} \\
Transitive     & $(a,b),(b,c)\in R \Rightarrow (a,c)\in R$           & $\leq$ on $\mathbb{N}$ & \hyperref[def:transitive]{↓ Def} \\
Total (Connex) & $(a,b)\in R \lor (b,a)\in R$                        & $\leq$ on $\mathbb{R}$ & \hyperref[def:total-rel]{↓ Def} \\
\midrule
\multicolumn{4}{l}{\textit{Structural classes formed by combining properties:}} \\
\midrule
Equivalence    & Reflexive + Symmetric + Transitive                  & $=$ on any set      & \hyperref[def:equivalence-rel]{↓ Def} \\
Preorder       & Reflexive + Transitive                              & $\leq$ on $\mathbb{N}$ & \hyperref[def:preorder]{↓ Def} \\
Partial order  & Reflexive + Antisymmetric + Transitive              & $\subseteq$ on $\mathcal{P}(A)$ & \hyperref[def:partial-order]{↓ Def} \\
Total order    & Partial order + Total                               & $\leq$ on $\mathbb{R}$ & \hyperref[def:total-order]{↓ Def} \\
\bottomrule
\end{tabular}
\end{tcolorbox}

\vspace{1em}

% ---------------------------------------------------------
% Individual property definitions
% ---------------------------------------------------------
\begin{definition}[Reflexive Relation]\label{def:reflexive}
A relation $R$ on $A$ is \emph{reflexive} if every element is related to itself:
\[
\forall a \in A, \; (a,a) \in R.
\]
\end{definition}

\begin{definition}[Irreflexive Relation]\label{def:irreflexive}
$R$ is \emph{irreflexive} if no element is related to itself:
\[
\forall a \in A, \; (a,a) \notin R.
\]
\end{definition}

\begin{remark}[Independence of reflexive and irreflexive]
A relation may be neither reflexive nor irreflexive (if some but not all
elements satisfy $(a,a)\in R$). However, a relation cannot be both reflexive
and irreflexive unless $A = \varnothing$.
\end{remark}

\begin{definition}[Symmetric Relation]\label{def:symmetric}
$R$ is \emph{symmetric} if related pairs commute:
\[
\forall a,b \in A, \; (a,b) \in R \rightarrow (b,a) \in R.
\]
\end{definition}

\begin{definition}[Antisymmetric Relation]\label{def:antisymmetric}
$R$ is \emph{antisymmetric} if mutual relation implies equality:
\[
\forall a,b \in A, \;
\bigl( (a,b) \in R \land (b,a) \in R \bigr) \rightarrow a = b.
\]
\end{definition}

\begin{definition}[Asymmetric Relation]\label{def:asymmetric}
$R$ is \emph{asymmetric} if related pairs never commute:
\[
\forall a,b \in A, \;
(a,b) \in R \rightarrow (b,a) \notin R.
\]
\end{definition}

\begin{remark}[Asymmetric vs.\ antisymmetric]
Every asymmetric relation is antisymmetric, but not conversely. $\leq$ on
$\mathbb{R}$ is antisymmetric but not asymmetric (since $1 \leq 1$). Strict
order $<$ is both asymmetric and irreflexive.
\end{remark}

\begin{definition}[Transitive Relation]\label{def:transitive}
$R$ is \emph{transitive} if it extends along chains:
\[
\forall a,b,c \in A, \;
\bigl( (a,b) \in R \land (b,c) \in R \bigr) \rightarrow (a,c) \in R.
\]
\end{definition}

\begin{definition}[Total (Connex) Relation]\label{def:total-rel}
$R$ is \emph{total} (or \emph{connex}) if every pair is comparable:
\[
\forall a,b \in A, \;
(a,b) \in R \lor (b,a) \in R.
\]
\end{definition}

% ---------------------------------------------------------
% Structural classes
% ---------------------------------------------------------
\begin{tcolorbox}[colback=propbox, colframe=propborder, arc=2pt,
  left=6pt, right=6pt, top=4pt, bottom=4pt,
  title={\small\textbf{Definition (Equivalence Relation)}},
  fonttitle=\small\bfseries]
\label{def:equivalence-rel}
$R$ is an \emph{equivalence relation} on $A$ if it is reflexive, symmetric,
and transitive.
\end{tcolorbox}

\begin{remark}[Role in mathematics]
Equivalence relations axiomatize the idea of ``sameness up to a chosen
criterion.'' They partition sets into equivalence classes and are the basis
for quotient constructions throughout algebra, topology, and analysis.
\end{remark}

\begin{tcolorbox}[colback=propbox, colframe=propborder, arc=2pt,
  left=6pt, right=6pt, top=4pt, bottom=4pt,
  title={\small\textbf{Definition (Preorder)}},
  fonttitle=\small\bfseries]
\label{def:preorder}
$R$ is a \emph{preorder} on $A$ if it is reflexive and transitive.
\end{tcolorbox}

\begin{tcolorbox}[colback=propbox, colframe=propborder, arc=2pt,
  left=6pt, right=6pt, top=4pt, bottom=4pt,
  title={\small\textbf{Definition (Partial Order)}},
  fonttitle=\small\bfseries]
\label{def:partial-order}
$R$ is a \emph{partial order} on $A$ if it is reflexive, antisymmetric, and
transitive.
\end{tcolorbox}

\begin{tcolorbox}[colback=propbox, colframe=propborder, arc=2pt,
  left=6pt, right=6pt, top=4pt, bottom=4pt,
  title={\small\textbf{Definition (Total Order)}},
  fonttitle=\small\bfseries]
\label{def:total-order}
$R$ is a \emph{total order} on $A$ if it is a partial order and is total.
\end{tcolorbox}

\begin{remark}[Using structural properties in proofs]
When $R$ is asserted to be an equivalence relation or partial order, this is
shorthand for the conjunction of its defining properties. In proofs, cite
only the specific component needed: ``since $R$ is a partial order,
antisymmetry implies \dots'' rather than restating all properties.
\end{remark}

\begin{example}[Using a structural property in a proof]
Let $R$ be an equivalence relation on $A$. For any $a,b \in A$,
$(a,b) \in R \Rightarrow [a] = [b]$, where $[a] = \{x \in A \mid (a,x) \in R\}$.
\end{example}

\begin{proof}
Assume $(a,b) \in R$. By symmetry, $(b,a) \in R$.

\emph{$[a] \subseteq [b]$:} Let $x \in [a]$, so $(a,x) \in R$. Since
$(b,a) \in R$ and $R$ is transitive, $(b,x) \in R$, so $x \in [b]$.

\emph{$[b] \subseteq [a]$:} Symmetric argument using $(a,b) \in R$.

Therefore $[a] = [b]$.
\end{proof}

\begin{remark}[Properties are logically independent]
No basic property implies another in general. A relation may be transitive
without being reflexive, or symmetric without being transitive. The structural
classes are distinguished precisely by requiring specific combinations.
\end{remark}


% families/index.tex
% =========================================================
% Indexed Families and Arbitrary Cartesian Products
% =========================================================

\subsection{Indexed Families of Sets}

% ---------------------------------------------------------
% TOOLKIT
% ---------------------------------------------------------
\begin{tcolorbox}[colback=gray!6, colframe=gray!40, arc=2pt,
  left=6pt, right=6pt, top=4pt, bottom=4pt,
  title={\small\textbf{Indexed Families — Quick Reference}},
  fonttitle=\small\bfseries]
\small
\begin{tabular}{l l l}
\toprule
\textbf{Concept} & \textbf{Meaning} & \textbf{Detail} \\
\midrule
Indexed family      & Function $F : I \to \mathcal{P}(U)$; written $\{A_i\}_{i \in I}$ & \hyperref[def:indexed-family]{↓ Def} \\
Indexed union       & $\bigcup_{i \in I} A_i$: in at least one $A_i$                    & \hyperref[def:indexed-union]{↓ Def} \\
Indexed intersection & $\bigcap_{i \in I} A_i$: in every $A_i$                         & \hyperref[def:indexed-intersection]{↓ Def} \\
Pairwise disjoint   & $i \neq j \Rightarrow A_i \cap A_j = \varnothing$                  & \hyperref[def:pairwise-disjoint]{↓ Def} \\
Cover               & $\bigcup_{C \in \mathcal{C}} C = A$                               & \hyperref[def:cover]{↓ Def} \\
Arbitrary product   & $\prod_{i \in I} A_i$: choice functions $f : I \to \bigcup A_i$   & \hyperref[def:arbitrary-product]{↓ Def} \\
\bottomrule
\end{tabular}
\end{tcolorbox}

\vspace{1em}

\begin{remark}[Motivation]
Many set-theoretic constructions require not just pairs of sets but
infinite families. Indexed families provide the formal language for
partitions, equivalence classes, unions over countably or uncountably many
sets, and products indexed by arbitrary index sets.
\end{remark}

\begin{tcolorbox}[colback=propbox, colframe=propborder, arc=2pt,
  left=6pt, right=6pt, top=4pt, bottom=4pt,
  title={\small\textbf{Definition (Indexed Family of Sets)}},
  fonttitle=\small\bfseries]
\label{def:indexed-family}
Let $I$ be a set (the \emph{index set}) and $U$ a universe. An
\emph{indexed family of sets} is a function
\[
F : I \to \mathcal{P}(U),
\]
with $F(i)$ typically written $A_i$. The family is denoted $\{A_i\}_{i \in I}$.
\end{tcolorbox}

\begin{remark}[Family vs.\ set of sets]
An indexed family is formally a \emph{function}, not a set of sets.
Different indices may correspond to the same set: $i \neq j$ does not imply
$A_i \neq A_j$. This distinction matters for equivalence class constructions.
\end{remark}

\begin{tcolorbox}[colback=propbox, colframe=propborder, arc=2pt,
  left=6pt, right=6pt, top=4pt, bottom=4pt,
  title={\small\textbf{Definition (Indexed Union)}},
  fonttitle=\small\bfseries]
\label{def:indexed-union}
The \emph{union} of the indexed family $\{A_i\}_{i \in I}$ is:
\[
\bigcup_{i \in I} A_i
\;:=\;
\{\, x \mid \exists i \in I \text{ such that } x \in A_i \,\}.
\]
\end{tcolorbox}

\begin{tcolorbox}[colback=propbox, colframe=propborder, arc=2pt,
  left=6pt, right=6pt, top=4pt, bottom=4pt,
  title={\small\textbf{Definition (Indexed Intersection)}},
  fonttitle=\small\bfseries]
\label{def:indexed-intersection}
For $I \neq \varnothing$, the \emph{intersection} of $\{A_i\}_{i \in I}$ is:
\[
\bigcap_{i \in I} A_i
\;:=\;
\{\, x \mid \forall i \in I,\; x \in A_i \,\}.
\]
\end{tcolorbox}

\begin{remark}[Why $I \neq \varnothing$ is required]
If $I = \varnothing$, then $\forall i \in I, x \in A_i$ holds vacuously for
every $x$, making the intersection the entire universe $U$. This is normally
left undefined or requires specifying a background universe explicitly.
\end{remark}

\begin{tcolorbox}[colback=propbox, colframe=propborder, arc=2pt,
  left=6pt, right=6pt, top=4pt, bottom=4pt,
  title={\small\textbf{Definition (Pairwise Disjoint Family)}},
  fonttitle=\small\bfseries]
\label{def:pairwise-disjoint}
The family $\{A_i\}_{i \in I}$ is \emph{pairwise disjoint} if
\[
\forall i,j \in I,\;
i \neq j \rightarrow A_i \cap A_j = \varnothing.
\]
\end{tcolorbox}

\begin{definition}[Cover]\label{def:cover}
A collection $\mathcal{C} \subseteq \mathcal{P}(A)$ is a \emph{cover} of $A$ if
\[
\bigcup_{C \in \mathcal{C}} C = A.
\]
\end{definition}

\vspace{1em}
\subsection{Arbitrary Cartesian Products}

\begin{tcolorbox}[colback=propbox, colframe=propborder, arc=2pt,
  left=6pt, right=6pt, top=4pt, bottom=4pt,
  title={\small\textbf{Definition (Arbitrary Cartesian Product)}},
  fonttitle=\small\bfseries]
\label{def:arbitrary-product}
Let $\{A_i\}_{i \in I}$ be an indexed family of sets. The \emph{Cartesian
product} is:
\[
\prod_{i \in I} A_i
\;:=\;
\Bigl\{
f : I \to \bigcup_{i \in I} A_i
\;\Bigm|\;
\forall i \in I,\; f(i) \in A_i
\Bigr\}.
\]
\end{tcolorbox}

\begin{remark}[Elements as choice functions]
An element of $\prod_{i \in I} A_i$ is a \emph{choice function}: a function
that assigns to each index $i$ an element of the corresponding set $A_i$.
For finite $I = \{1,\dots,n\}$, this reduces to the familiar $n$-tuple
$(a_1,\dots,a_n)$.
\end{remark}

\begin{remark}[Axiom of Choice connection]
For infinite $I$, the product $\prod_{i \in I} A_i$ is nonempty iff a choice
function exists. This existence is not guaranteed by the other ZFC axioms:
it is equivalent to the Axiom of Choice. Thus the Axiom of Choice is
precisely the assertion that arbitrary products of nonempty sets are nonempty.
\end{remark}


% equivalence/index.tex
% =========================================================
% Equivalence Classes and Partitions
% =========================================================

\subsection{Equivalence Classes and Partitions}

% ---------------------------------------------------------
% TOOLKIT
% ---------------------------------------------------------
\begin{tcolorbox}[colback=gray!6, colframe=gray!40, arc=2pt,
  left=6pt, right=6pt, top=4pt, bottom=4pt,
  title={\small\textbf{Equivalence and Partitions — Quick Reference}},
  fonttitle=\small\bfseries]
\small
\begin{tabular}{l l l}
\toprule
\textbf{Concept} & \textbf{Meaning} & \textbf{Detail} \\
\midrule
Equivalence class  & $[a]_R = \{x \in A \mid (a,x)\in R\}$                        & \hyperref[def:equiv-class]{↓ Def} \\
Quotient set       & $A/R$: all equivalence classes                                & \hyperref[def:quotient-set]{↓ Def} \\
Index of $R$       & Cardinality of $A/R$                                          & \hyperref[def:index]{↓ Def} \\
Canonical surjection & $\pi : A \to A/R$, $\pi(a) = [a]$                          & \hyperref[def:canonical-surj]{↓ Def} \\
Partition          & Nonempty, disjoint, covering collection of subsets            & \hyperref[def:partition]{↓ Def} \\
Rep.\ independence & $[a]=[b] \iff (a,b)\in R$                                    & \hyperref[lem:rep-independence]{↓ Lem} \\
Equiv.--partition correspondence & Bijection between equiv.\ relations and partitions & \hyperref[thm:equiv-partition]{↓ Thm} \\
\bottomrule
\end{tabular}
\end{tcolorbox}

\vspace{1em}

\begin{tcolorbox}[colback=propbox, colframe=propborder, arc=2pt,
  left=6pt, right=6pt, top=4pt, bottom=4pt,
  title={\small\textbf{Definition (Equivalence Class)}},
  fonttitle=\small\bfseries]
\label{def:equiv-class}
Let $R$ be an equivalence relation on $A$. For $a \in A$, the
\emph{equivalence class} of $a$ is:
\[
[a]_R \;:=\; \{\, x \in A \mid (a,x) \in R \,\}.
\]
When $R$ is clear from context, we write $[a]$.
\end{tcolorbox}

\begin{remark}[English reading]
$[a]_R$ is the set of all elements that $R$ declares ``the same as $a$.''
Two elements lie in the same class iff they are related: this is precisely
the Representative Independence Lemma below.
\end{remark}

\begin{tcolorbox}[colback=propbox, colframe=propborder, arc=2pt,
  left=6pt, right=6pt, top=4pt, bottom=4pt,
  title={\small\textbf{Definition (Quotient Set)}},
  fonttitle=\small\bfseries]
\label{def:quotient-set}
The \emph{quotient set} of $A$ by $R$ is the set of all equivalence classes:
\[
A / R \;:=\; \{\, [a]_R \mid a \in A \,\}.
\]
\end{tcolorbox}

\begin{definition}[Index of an Equivalence Relation]\label{def:index}
The \emph{index} of $R$ on $A$ is the cardinality $|A/R|$, i.e.\ the number
of equivalence classes.
\end{definition}

\begin{tcolorbox}[colback=propbox, colframe=propborder, arc=2pt,
  left=6pt, right=6pt, top=4pt, bottom=4pt,
  title={\small\textbf{Definition (Canonical Surjection)}},
  fonttitle=\small\bfseries]
\label{def:canonical-surj}
The \emph{canonical surjection} (quotient map) is the function
\[
\pi : A \to A/R,
\qquad
\pi(a) := [a].
\]
\end{tcolorbox}

\begin{remark}[Properties of $\pi$]
$\pi$ is surjective by construction. Elements $a, b \in A$ satisfy
$\pi(a) = \pi(b)$ iff $(a,b) \in R$. The canonical surjection is the
prototype for all quotient constructions: it reappears as the quotient
homomorphism in algebra and the quotient map in topology.
\end{remark}

\begin{tcolorbox}[colback=propbox, colframe=propborder, arc=2pt,
  left=6pt, right=6pt, top=4pt, bottom=4pt,
  title={\small\textbf{Definition (Partition)}},
  fonttitle=\small\bfseries]
\label{def:partition}
A \emph{partition} of $A$ is a collection $\mathcal{P}$ of subsets of $A$ such
that:
\begin{enumerate}[label=(\roman*)]
\item every block is nonempty: $\forall P \in \mathcal{P},\; P \neq \varnothing$;
\item distinct blocks are disjoint:
$\forall P,Q \in \mathcal{P},\; P \neq Q \rightarrow P \cap Q = \varnothing$;
\item the blocks cover $A$: $\bigcup_{P \in \mathcal{P}} P = A$.
\end{enumerate}
The sets $P \in \mathcal{P}$ are called the \emph{blocks} of the partition.
\end{tcolorbox}

\begin{remark}[Partition vs.\ cover]
Every partition of $A$ is a cover of $A$ whose members are nonempty and
pairwise disjoint. Partitions are precisely the covers satisfying the
disjointness condition.
\end{remark}

% ---------------------------------------------------------
% Core Lemma and Theorem
% ---------------------------------------------------------

\begin{lemma}[Representative Independence Lemma]\label{lem:rep-independence}
Let $R$ be an equivalence relation on $A$. For any $a,b \in A$,
\[
[a] = [b]
\;\;\Longleftrightarrow\;\;
(a,b) \in R.
\]
\end{lemma}



\begin{remark}[Consequence for quotient maps]
This lemma is the key fact underlying well-definedness of functions on quotient
sets: a function $f : A \to B$ defined by $f([a]) = \cdots$ is well-defined
iff the formula gives the same output for all representatives of $[a]$.
\end{remark}

\begin{theorem}[Equivalence Relations and Partitions]\label{thm:equiv-partition}
Let $A$ be a set.
\begin{enumerate}[label=(\roman*)]
\item If $R$ is an equivalence relation on $A$, then $A/R$ is a partition of $A$.
\item If $\mathcal{P}$ is a partition of $A$, then the relation
$R_{\mathcal{P}}$ defined by
\[
(a,b) \in R_{\mathcal{P}} \;\Longleftrightarrow\;
\exists P \in \mathcal{P} \text{ with } a \in P \text{ and } b \in P
\]
is an equivalence relation on $A$.
\item These constructions are inverse: $R_{A/R} = R$ and $A/R_{\mathcal{P}} = \mathcal{P}$.
\end{enumerate}
\end{theorem}



\begin{remark}[Significance]
This theorem establishes a bijection between equivalence relations on $A$
and partitions of $A$. The two perspectives---``same block'' (partition) and
``related'' (equivalence relation)---are interchangeable and each is more
natural in different contexts.
\end{remark}

\begin{example}[Extremal equivalence relations]
On any set $A$:
\begin{enumerate}
\item The \emph{equality relation} $(a,b) \in R \iff a=b$ gives singleton
classes $[a] = \{a\}$ and the finest partition of $A$.
\item The \emph{universal relation} $R = A \times A$ gives $[a] = A$ for all
$a$, and the coarsest partition (one block).
\end{enumerate}
All other equivalence relations on $A$ lie strictly between these extremes.
\end{example}


% functions/index.tex
% =========================================================
% Functions: Definitions and Basic Properties
% =========================================================

\subsection{Functions}

% ---------------------------------------------------------
% TOOLKIT
% ---------------------------------------------------------
\begin{tcolorbox}[colback=gray!6, colframe=gray!40, arc=2pt,
  left=6pt, right=6pt, top=4pt, bottom=4pt,
  title={\small\textbf{Functions — Quick Reference}},
  fonttitle=\small\bfseries]
\small
\begin{tabular}{l l l}
\toprule
\textbf{Concept} & \textbf{Meaning} & \textbf{Detail} \\
\midrule
Function         & Left-total, right-unique relation                            & \hyperref[def:function]{↓ Def} \\
Domain / Codomain & $\dom(f) = A$,\; $\cod(f) = B$ for $f : A \to B$            & \hyperref[def:domain-codomain]{↓ Def} \\
Image of function & $\im(f) = \{f(a) \mid a \in A\}$                            & \hyperref[def:image-function]{↓ Def} \\
Image of set      & $f(S) = \{f(a) \mid a \in S\}$ for $S \subseteq A$          & \hyperref[def:image-set]{↓ Def} \\
Preimage          & $f^{-1}(T) = \{a \in A \mid f(a) \in T\}$ for $T \subseteq B$ & \hyperref[def:preimage]{↓ Def} \\
Fiber             & $f^{-1}(\{b\})$: preimage of a singleton                    & \hyperref[def:fiber]{↓ Def} \\
Graph             & $\{(a,b) \in A\times B \mid b = f(a)\}$                     & \hyperref[def:graph]{↓ Def} \\
Injective         & Distinct inputs $\Rightarrow$ distinct outputs               & \hyperref[def:injective]{↓ Def} \\
Surjective        & Every codomain element is achieved                          & \hyperref[def:surjective]{↓ Def} \\
Bijective         & Injective and surjective                                    & \hyperref[def:bijective]{↓ Def} \\
Identity          & $\id_A(a) = a$                                              & \hyperref[def:identity]{↓ Def} \\
Inclusion map     & $\iota : A \hookrightarrow B$, $\iota(a) = a$ for $A \subseteq B$ & \hyperref[def:inclusion]{↓ Def} \\
Composition       & $(g \circ f)(a) = g(f(a))$                                  & \hyperref[def:composition]{↓ Def} \\
Inverse           & Defined for bijections; $f^{-1} \circ f = \id_A$            & \hyperref[def:inverse]{↓ Def} \\
Left / Right inverse & Section and retraction                                  & \hyperref[def:left-right-inverse]{↓ Def} \\
Restriction       & $f|_C : C \to B$                                            & \hyperref[def:restriction]{↓ Def} \\
Extension         & $g : A' \to B$ with $g|_A = f$                              & \hyperref[def:extension]{↓ Def} \\
\bottomrule
\end{tabular}
\end{tcolorbox}

\vspace{1em}

% ---------------------------------------------------------
% Function definition
% ---------------------------------------------------------
\begin{tcolorbox}[colback=propbox, colframe=propborder, arc=2pt,
  left=6pt, right=6pt, top=4pt, bottom=4pt,
  title={\small\textbf{Definition (Function)}},
  fonttitle=\small\bfseries]
\label{def:function}
Let $A$ and $B$ be sets. A \emph{function} from $A$ to $B$ is a relation
$f \subseteq A \times B$ such that:
\begin{enumerate}[label=(\roman*)]
\item (\emph{Existence / left-total}) for every $a \in A$, there exists $b \in B$
  with $(a,b) \in f$;
\item (\emph{Uniqueness / right-unique}) if $(a,b_1) \in f$ and $(a,b_2) \in f$
  then $b_1 = b_2$.
\end{enumerate}
If $(a,b) \in f$ we write $f(a) = b$.
\end{tcolorbox}

\begin{remark}[English reading]
A function is a rule assigning to each input exactly one output. The two
conditions formalize ``defined everywhere'' (existence) and ``single-valued''
(uniqueness).
\end{remark}

\begin{remark}[Function as relation]
Every function is a relation, but not every relation is a function. The two
conditions that distinguish functions are left-totality and right-uniqueness.
\end{remark}

\begin{tcolorbox}[colback=propbox, colframe=propborder, arc=2pt,
  left=6pt, right=6pt, top=4pt, bottom=4pt,
  title={\small\textbf{Definition (Domain and Codomain)}},
  fonttitle=\small\bfseries]
\label{def:domain-codomain}
If $f$ is a function from $A$ to $B$, we write $f : A \to B$, where $A$ is
the \emph{domain} $\dom(f)$ and $B$ the \emph{codomain} $\cod(f)$.
\end{tcolorbox}

\begin{tcolorbox}[colback=propbox, colframe=propborder, arc=2pt,
  left=6pt, right=6pt, top=4pt, bottom=4pt,
  title={\small\textbf{Definition (Image of a Function)}},
  fonttitle=\small\bfseries]
\label{def:image-function}
The \emph{image} (or \emph{range}) of $f : A \to B$ is:
\[
\im(f) \;:=\; \{\, b \in B \mid \exists a \in A,\ f(a)=b \,\}.
\]
The image may be a proper subset of the codomain.
\end{tcolorbox}

\begin{tcolorbox}[colback=propbox, colframe=propborder, arc=2pt,
  left=6pt, right=6pt, top=4pt, bottom=4pt,
  title={\small\textbf{Definition (Image of a Set)}},
  fonttitle=\small\bfseries]
\label{def:image-set}
For $S \subseteq A$, the \emph{image of $S$ under $f$} is:
\[
f(S) \;:=\; \{\, f(a) \mid a \in S \,\}.
\]
Note $f(A) = \im(f)$.
\end{tcolorbox}

\begin{tcolorbox}[colback=propbox, colframe=propborder, arc=2pt,
  left=6pt, right=6pt, top=4pt, bottom=4pt,
  title={\small\textbf{Definition (Preimage)}},
  fonttitle=\small\bfseries]
\label{def:preimage}
For $T \subseteq B$, the \emph{preimage} (inverse image) of $T$ under $f$ is:
\[
f^{-1}(T) \;:=\; \{\, a \in A \mid f(a) \in T \,\}.
\]
\end{tcolorbox}

\begin{remark}[Preimage notation warning]
$f^{-1}(T)$ does \emph{not} denote an inverse function. It is defined for any
function $f$, regardless of whether $f$ is injective or bijective.
\end{remark}

\begin{tcolorbox}[colback=propbox, colframe=propborder, arc=2pt,
  left=6pt, right=6pt, top=4pt, bottom=4pt,
  title={\small\textbf{Definition (Fiber)}},
  fonttitle=\small\bfseries]
\label{def:fiber}
The \emph{fiber} of $f$ over $b \in B$ is the preimage of the singleton:
\[
f^{-1}(\{b\}) \;=\; \{\, a \in A \mid f(a) = b \,\}.
\]
\end{tcolorbox}

\begin{remark}[Fibers partition the domain]
The collection $\{f^{-1}(\{b\}) \mid b \in \im(f)\}$ is a partition of $A$.
Every function thus induces an equivalence relation $a_1 \sim_f a_2 \iff f(a_1) = f(a_2)$,
whose equivalence classes are precisely the fibers.
\end{remark}

\begin{tcolorbox}[colback=propbox, colframe=propborder, arc=2pt,
  left=6pt, right=6pt, top=4pt, bottom=4pt,
  title={\small\textbf{Definition (Graph of a Function)}},
  fonttitle=\small\bfseries]
\label{def:graph}
The \emph{graph} of $f : A \to B$ is:
\[
\operatorname{Graph}(f)
\;:=\;
\{\, (a,b) \in A \times B \mid b = f(a) \,\}.
\]
\end{tcolorbox}

\begin{remark}[Function as graph]
A relation $G \subseteq A \times B$ is the graph of a function $f : A \to B$
iff $G$ is left-total and right-unique. Functions may therefore be identified
with their graphs: a function \emph{is} a special kind of relation.
\end{remark}

% ---------------------------------------------------------
% Injectivity / Surjectivity / Bijectivity
% ---------------------------------------------------------
\begin{tcolorbox}[colback=propbox, colframe=propborder, arc=2pt,
  left=6pt, right=6pt, top=4pt, bottom=4pt,
  title={\small\textbf{Definition (Injective Function)}},
  fonttitle=\small\bfseries]
\label{def:injective}
$f : A \to B$ is \emph{injective} (one-to-one) if distinct elements of $A$
have distinct images:
\[
\forall a_1,a_2 \in A,\;
f(a_1)=f(a_2) \;\Longrightarrow\; a_1=a_2.
\]
\end{tcolorbox}

\begin{tcolorbox}[colback=propbox, colframe=propborder, arc=2pt,
  left=6pt, right=6pt, top=4pt, bottom=4pt,
  title={\small\textbf{Definition (Surjective Function)}},
  fonttitle=\small\bfseries]
\label{def:surjective}
$f : A \to B$ is \emph{surjective} (onto) if every element of $B$ is achieved:
\[
\forall b \in B,\; \exists a \in A \text{ such that } f(a)=b.
\]
Equivalently, $\im(f) = B$.
\end{tcolorbox}

\begin{tcolorbox}[colback=propbox, colframe=propborder, arc=2pt,
  left=6pt, right=6pt, top=4pt, bottom=4pt,
  title={\small\textbf{Definition (Bijective Function)}},
  fonttitle=\small\bfseries]
\label{def:bijective}
$f : A \to B$ is \emph{bijective} if it is both injective and surjective.
\end{tcolorbox}

\begin{remark}[Fiber interpretation]
Injectivity: each fiber has at most one element. Surjectivity: every fiber is
nonempty. Bijectivity: every fiber has exactly one element.
\end{remark}

% ---------------------------------------------------------
% Special functions
% ---------------------------------------------------------
\begin{tcolorbox}[colback=propbox, colframe=propborder, arc=2pt,
  left=6pt, right=6pt, top=4pt, bottom=4pt,
  title={\small\textbf{Definition (Identity and Inclusion)}},
  fonttitle=\small\bfseries]
\label{def:identity}
The \emph{identity function} on $A$ is $\id_A : A \to A$, $\id_A(a) = a$.

If $A \subseteq B$, the \emph{inclusion map} is $\iota : A \hookrightarrow B$,
$\iota(a) = a$.
\end{tcolorbox}

\begin{remark}[Identity vs.\ inclusion]
\label{def:inclusion}
Both send each element to itself, but the inclusion map has a strictly larger
codomain when $A \subsetneq B$. The inclusion map is always injective.
\end{remark}

\begin{definition}[Constant Function]\label{def:constant}
$f : A \to B$ is \emph{constant} if there exists $b_0 \in B$ with
$f(a) = b_0$ for all $a \in A$.
\end{definition}

% =========================================================
% Composition, Inverses, and Image / Preimage Laws
% =========================================================

\subsection{Composition, Inverses, and Set-Image Laws}

% ---------------------------------------------------------
% TOOLKIT
% ---------------------------------------------------------
\begin{tcolorbox}[colback=gray!6, colframe=gray!40, arc=2pt,
  left=6pt, right=6pt, top=4pt, bottom=4pt,
  title={\small\textbf{Composition \& Inverses — Quick Reference}},
  fonttitle=\small\bfseries]
\small
\begin{tabular}{l l l l}
\toprule
\textbf{Result} & \textbf{Statement} & \textbf{Proof method} & \textbf{Detail} \\
\midrule
Associativity of $\circ$ & $h \circ (g \circ f) = (h \circ g) \circ f$ & element-wise & \hyperref[thm:comp-assoc]{↓ Thm} \\
Identity and $\circ$     & $f \circ \id_A = \id_B \circ f = f$          & element-wise & \hyperref[thm:comp-id]{↓ Thm} \\
Inj/Surj under $\circ$   & Closed under composition; partial converses  & element-wise & \hyperref[thm:comp-inj-surj]{↓ Thm} \\
Inverse characterization & $f^{-1}\circ f = \id_A$, $f \circ f^{-1} = \id_B$ & bijectivity & \hyperref[thm:inverse-char]{↓ Thm} \\
Inverse of composition   & $(g \circ f)^{-1} = f^{-1} \circ g^{-1}$    & verify both sides & \hyperref[thm:inverse-comp]{↓ Thm} \\
One-sided inverses       & Left inverse $\iff$ injective; right $\iff$ surjective & & \hyperref[thm:one-sided]{↓ Thm} \\
Preimage preserves ops   & $f^{-1}$ commutes with $\cup,\cap,\setminus,c$ & element-wise & \hyperref[thm:preimage-ops]{↓ Thm} \\
Image and set ops        & Images preserve $\cup$; contain $\subseteq$ for $\cap$ & counterexample & \hyperref[thm:image-ops]{↓ Thm} \\
\bottomrule
\end{tabular}
\end{tcolorbox}

\vspace{1em}

\begin{tcolorbox}[colback=propbox, colframe=propborder, arc=2pt,
  left=6pt, right=6pt, top=4pt, bottom=4pt,
  title={\small\textbf{Definition (Composition)}},
  fonttitle=\small\bfseries]
\label{def:composition}
Let $f : A \to B$ and $g : B \to C$. The \emph{composition} $g \circ f : A \to C$ is:
\[
(g \circ f)(a) := g\bigl(f(a)\bigr)
\quad\text{for all } a \in A.
\]
\end{tcolorbox}

\begin{remark}[Non-commutativity]
Composition is generally not commutative: $g \circ f \neq f \circ g$ even
when both are defined.
\end{remark}

\begin{theorem}[Associativity of Composition]\label{thm:comp-assoc}
For $f : A \to B$, $g : B \to C$, $h : C \to D$:
\[
h \circ (g \circ f) = (h \circ g) \circ f.
\]
\end{theorem}



\begin{theorem}[Identity and Composition]\label{thm:comp-id}
For any $f : A \to B$:
\[
f \circ \id_A = f
\quad\text{and}\quad
\id_B \circ f = f.
\]
\end{theorem}

\begin{theorem}[Injectivity and Surjectivity Under Composition]\label{thm:comp-inj-surj}
Let $f : A \to B$ and $g : B \to C$.
\begin{enumerate}[label=(\roman*)]
\item If $f$ and $g$ are injective, then $g \circ f$ is injective.
\item If $f$ and $g$ are surjective, then $g \circ f$ is surjective.
\item If $g \circ f$ is injective, then $f$ is injective.
\item If $g \circ f$ is surjective, then $g$ is surjective.
\end{enumerate}
\end{theorem}

\begin{remark}[Partial converses]
In (iii), $g$ need not be injective. In (iv), $f$ need not be surjective.
Bijectivity of $g \circ f$ does not imply bijectivity of either $f$ or $g$
individually.
\end{remark}

% ---------------------------------------------------------
% Inverse functions
% ---------------------------------------------------------
\begin{tcolorbox}[colback=propbox, colframe=propborder, arc=2pt,
  left=6pt, right=6pt, top=4pt, bottom=4pt,
  title={\small\textbf{Definition (Inverse Function)}},
  fonttitle=\small\bfseries]
\label{def:inverse}
Let $f : A \to B$ be bijective. The \emph{inverse function} is
$f^{-1} : B \to A$ defined by: $f^{-1}(b) = a$ iff $f(a) = b$.
\end{tcolorbox}

\begin{theorem}[Characterization of Inverse Functions]\label{thm:inverse-char}
Let $f : A \to B$ be bijective. Then
\[
f^{-1} \circ f = \id_A
\quad\text{and}\quad
f \circ f^{-1} = \id_B.
\]
Conversely, a function admits an inverse iff it is bijective.
\end{theorem}

\begin{theorem}[Inverse of a Composition]\label{thm:inverse-comp}
Let $f : A \to B$ and $g : B \to C$ be bijective. Then $g \circ f$ is
bijective and
\[
(g \circ f)^{-1} = f^{-1} \circ g^{-1}.
\]
\end{theorem}

% ---------------------------------------------------------
% One-sided inverses
% ---------------------------------------------------------
\begin{tcolorbox}[colback=propbox, colframe=propborder, arc=2pt,
  left=6pt, right=6pt, top=4pt, bottom=4pt,
  title={\small\textbf{Definition (Left and Right Inverses)}},
  fonttitle=\small\bfseries]
\label{def:left-right-inverse}
Let $f : A \to B$.

A \emph{left inverse} of $f$ is a function $g : B \to A$ with $g \circ f = \id_A$.

A \emph{right inverse} (or \emph{section}) of $f$ is a function $h : B \to A$
with $f \circ h = \id_B$.
\end{tcolorbox}

\begin{theorem}[One-Sided Inverses and Function Properties]\label{thm:one-sided}
Let $f : A \to B$.
\begin{enumerate}[label=(\roman*)]
\item $f$ has a left inverse $\iff$ $f$ is injective (and $A \neq \varnothing$ or $A=B=\varnothing$).
\item $f$ has a right inverse $\iff$ $f$ is surjective.
\item If $f$ has both a left inverse $g$ and a right inverse $h$, then $g = h$
and $f$ is bijective.
\end{enumerate}
\end{theorem}

\begin{remark}[Axiom of Choice]
The existence of a right inverse for every surjective function is equivalent
to the Axiom of Choice.
\end{remark}

% ---------------------------------------------------------
% Restriction and Extension
% ---------------------------------------------------------
\begin{tcolorbox}[colback=propbox, colframe=propborder, arc=2pt,
  left=6pt, right=6pt, top=4pt, bottom=4pt,
  title={\small\textbf{Definition (Restriction and Extension)}},
  fonttitle=\small\bfseries]
\label{def:restriction}
Let $f : A \to B$ and $C \subseteq A$. The \emph{restriction} of $f$ to $C$
is $f|_C : C \to B$, $f|_C(a) = f(a)$ for all $a \in C$.

Let $A \subseteq A'$, and let $f : A \to B$. A function $g : A' \to B$ is an
\emph{extension} of $f$ if $g|_A = f$.
\end{tcolorbox}

\begin{remark}[Extensions not unique]
\label{def:extension}
An extension agrees with $f$ on all of $A$ but may be defined on a larger
domain $A'$. Extensions are generally not unique; uniqueness requires additional
constraints (such as continuity in analysis or linearity in algebra), leading to
important results like the Tietze Extension Theorem and the Hahn--Banach
Theorem.
\end{remark}

% ---------------------------------------------------------
% Image and Preimage Laws
% ---------------------------------------------------------
\begin{theorem}[Preimages Preserve Set Operations]\label{thm:preimage-ops}
Let $f : A \to B$ and $S, T \subseteq B$. Then
\begin{enumerate}[label=(\roman*)]
\item $f^{-1}(S \cup T) = f^{-1}(S) \cup f^{-1}(T)$,
\item $f^{-1}(S \cap T) = f^{-1}(S) \cap f^{-1}(T)$,
\item $f^{-1}(S \setminus T) = f^{-1}(S) \setminus f^{-1}(T)$,
\item $f^{-1}(S^c) = (f^{-1}(S))^c$.
\end{enumerate}
\end{theorem}

\begin{remark}[Why preimages are well-behaved]
Preimages preserve all set-theoretic operations. This property makes preimages
central in topology and measure theory, where properties such as continuity
and measurability are defined via preimages.
\end{remark}

\begin{theorem}[Images and Set Operations]\label{thm:image-ops}
Let $f : A \to B$ and $S, T \subseteq A$. Then
\begin{enumerate}[label=(\roman*)]
\item $f(S \cup T) = f(S) \cup f(T)$,
\item $f(S \cap T) \subseteq f(S) \cap f(T)$,
\item $f(S \setminus T) \supseteq f(S) \setminus f(T)$.
\end{enumerate}
Equality holds in (ii) and (iii) for all $S,T$ iff $f$ is injective.
\end{theorem}

\begin{remark}[Asymmetry between images and preimages]
Forward images preserve unions but generally not intersections or complements.
This asymmetry between images and preimages is fundamental and explains why
preimages appear more naturally in topology and measure theory.
\end{remark}


% order/index.tex
% =========================================================
% Ordered Sets
% =========================================================

\subsection{Ordered Sets}

% ---------------------------------------------------------
% TOOLKIT
% ---------------------------------------------------------
\begin{tcolorbox}[colback=gray!6, colframe=gray!40, arc=2pt,
  left=6pt, right=6pt, top=4pt, bottom=4pt,
  title={\small\textbf{Ordered Sets — Quick Reference}},
  fonttitle=\small\bfseries]
\small
\begin{tabular}{l l l}
\toprule
\textbf{Concept} & \textbf{Meaning} & \textbf{Detail} \\
\midrule
Ordered set        & $(A, \leq)$: set with a partial order                        & \hyperref[def:ordered-set]{↓ Def} \\
Strict order       & $a < b \iff a \leq b \land a \neq b$                         & \hyperref[def:strict-order]{↓ Def} \\
Comparable / incomparable & $a \leq b$ or $b \leq a$ / neither              & \hyperref[def:comparable]{↓ Def} \\
Total (linear) order & Partial order with all pairs comparable                  & \hyperref[def:linear-order]{↓ Def} \\
Upper / lower bound & $u \geq s$ / $\ell \leq s$ for all $s \in S$               & \hyperref[def:bounds]{↓ Def} \\
Minimal / maximal  & No smaller/larger element in $S$                            & \hyperref[def:min-max]{↓ Def} \\
Least / greatest   & Smaller/larger than all elements of $S$                     & \hyperref[def:least-greatest]{↓ Def} \\
Order-preserving   & $a \leq b \Rightarrow f(a) \leq' f(b)$                       & \hyperref[def:order-pres]{↓ Def} \\
Order isomorphism  & Bijective order-preserving map with order-preserving inverse & \hyperref[def:order-iso]{↓ Def} \\
Well-ordered set   & Every nonempty subset has a least element                   & \hyperref[def:well-order]{↓ Def} \\
Chain / antichain  & All comparable / none comparable                            & \hyperref[def:chain]{↓ Def} \\
Initial segment    & Downward-closed subset                                      & \hyperref[def:initial-segment]{↓ Def} \\
\bottomrule
\end{tabular}
\end{tcolorbox}

\vspace{1em}

\begin{tcolorbox}[colback=propbox, colframe=propborder, arc=2pt,
  left=6pt, right=6pt, top=4pt, bottom=4pt,
  title={\small\textbf{Definition (Ordered Set)}},
  fonttitle=\small\bfseries]
\label{def:ordered-set}
An \emph{ordered set} (or \emph{partially ordered set}, \emph{poset}) is a
pair $(A,\leq)$ where $\leq$ is a partial order on $A$: a relation that is
reflexive, antisymmetric, and transitive.
\end{tcolorbox}

\begin{remark}[English reading]
An ordered set is a set equipped with a specified way to compare elements,
but without requiring all pairs to be comparable.
\end{remark}

\begin{tcolorbox}[colback=propbox, colframe=propborder, arc=2pt,
  left=6pt, right=6pt, top=4pt, bottom=4pt,
  title={\small\textbf{Definition (Strict Order)}},
  fonttitle=\small\bfseries]
\label{def:strict-order}
Let $(A,\leq)$ be an ordered set. The \emph{strict order} $<$ is defined by:
\[
a < b \;\;\Longleftrightarrow\;\; (a \leq b \land a \neq b).
\]
\end{tcolorbox}

\begin{remark}[Strict and non-strict are equivalent]
The relation $<$ is irreflexive and transitive. Conversely, given any strict
partial order $<$, one recovers a non-strict order by setting
$a \leq b \iff (a < b \lor a=b)$. The two formulations carry exactly the
same information.
\end{remark}

\begin{definition}[Comparable and Incomparable Elements]\label{def:comparable}
In an ordered set $(A,\leq)$, elements $a,b \in A$ are \emph{comparable} if
$a \leq b$ or $b \leq a$; they are \emph{incomparable} if neither holds.
\end{definition}

\begin{remark}[Incomparable elements]
Incomparable elements can only occur in partial orders. In a total order,
every pair is comparable by definition.
\end{remark}

\begin{tcolorbox}[colback=propbox, colframe=propborder, arc=2pt,
  left=6pt, right=6pt, top=4pt, bottom=4pt,
  title={\small\textbf{Definition (Total / Linear Order)}},
  fonttitle=\small\bfseries]
\label{def:linear-order}
An ordered set $(A,\leq)$ is a \emph{total order} (or \emph{linear order})
if every pair of elements is comparable:
\[
\forall a,b \in A,\; a \leq b \lor b \leq a.
\]
\end{tcolorbox}

\begin{remark}[Examples]
$(\mathbb{R},\leq)$ is a total order. $(\mathcal{P}(A), \subseteq)$ is a
partial order that is generally not total.
\end{remark}

% ---------------------------------------------------------
% Bounds and extremal elements
% ---------------------------------------------------------
\begin{tcolorbox}[colback=propbox, colframe=propborder, arc=2pt,
  left=6pt, right=6pt, top=4pt, bottom=4pt,
  title={\small\textbf{Definition (Upper and Lower Bounds)}},
  fonttitle=\small\bfseries]
\label{def:bounds}
Let $(A,\leq)$ be an ordered set and $S \subseteq A$.

An element $u \in A$ is an \emph{upper bound} of $S$ if $\forall s \in S,\; s \leq u$.

An element $\ell \in A$ is a \emph{lower bound} of $S$ if $\forall s \in S,\; \ell \leq s$.
\end{tcolorbox}

\begin{remark}[Consequence]
Bounds need not exist and need not be unique. They need not lie in $S$ itself.
Bounds are central to the definition of completeness for ordered fields.
\end{remark}

\begin{tcolorbox}[colback=propbox, colframe=propborder, arc=2pt,
  left=6pt, right=6pt, top=4pt, bottom=4pt,
  title={\small\textbf{Definition (Minimal, Maximal, Least, Greatest Elements)}},
  fonttitle=\small\bfseries]
\label{def:min-max}
Let $S \subseteq A$ in an ordered set $(A,\leq)$.

$m \in S$ is a \emph{minimal element} of $S$ if $\nexists s \in S$ with $s < m$.

$M \in S$ is a \emph{maximal element} of $S$ if $\nexists s \in S$ with $M < s$.

\smallskip
\label{def:least-greatest}
$\ell \in S$ is the \emph{least element} if $\forall s \in S,\; \ell \leq s$.

$g \in S$ is the \emph{greatest element} if $\forall s \in S,\; s \leq g$.
\end{tcolorbox}

\begin{remark}[Minimal vs.\ least]
A \emph{least} element is below every element of $S$; it must be unique if it
exists. A \emph{minimal} element merely has nothing below it within $S$; there
may be many. Every least element is minimal, but not conversely.
\end{remark}

% ---------------------------------------------------------
% Order maps and isomorphisms
% ---------------------------------------------------------
\begin{tcolorbox}[colback=propbox, colframe=propborder, arc=2pt,
  left=6pt, right=6pt, top=4pt, bottom=4pt,
  title={\small\textbf{Definition (Order-Preserving Map and Isomorphism)}},
  fonttitle=\small\bfseries]
\label{def:order-pres}
Let $(M,\leq)$ and $(M',\leq')$ be partially ordered sets.

A function $f : M \to M'$ is \emph{order-preserving} (or \emph{monotone}) if
\[
\forall a,b \in M,\quad a \leq b \;\Longrightarrow\; f(a) \leq' f(b).
\]

$f$ is an \emph{order isomorphism}\label{def:order-iso} if it is bijective and
\[
\forall a,b \in M,\quad a \leq b \;\Longleftrightarrow\; f(a) \leq' f(b).
\]
\end{tcolorbox}

\begin{remark}[Order-isomorphic posets]
An order isomorphism preserves and reflects the order structure exactly,
including comparability, minimal and maximal elements, and bounds. Two posets
are \emph{order-isomorphic} if such a map exists; they are structurally
identical from the order-theoretic viewpoint.
\end{remark}

% ---------------------------------------------------------
% Well-order
% ---------------------------------------------------------
\begin{tcolorbox}[colback=propbox, colframe=propborder, arc=2pt,
  left=6pt, right=6pt, top=4pt, bottom=4pt,
  title={\small\textbf{Definition (Well-Ordered Set)}},
  fonttitle=\small\bfseries]
\label{def:well-order}
An ordered set $(A,<)$ is \emph{well-ordered} if every nonempty subset
$S \subseteq A$ has a least element:
\[
\forall S \subseteq A,\;
\bigl(S \neq \varnothing \;\Rightarrow\; \exists m \in S \text{ s.t.\ }
m \leq s \;\forall s \in S\bigr).
\]
\end{tcolorbox}

\begin{remark}[Well-order implies total order]
Every well-ordered set is totally ordered. The well-ordering condition is
strictly stronger: it requires a least element in every nonempty subset,
not just in $A$ as a whole.
\end{remark}

\begin{example}[Well-order examples]
$(\mathbb{N},\leq)$ is well-ordered. $(\mathbb{Z},\leq)$ is not: the set
$\{\dots,-3,-2,-1\}$ has no least element. $(\mathbb{R},\leq)$ is not:
$(0,1)$ has no least element.
\end{example}

\begin{remark}[Connection to ordinal numbers]
An \emph{ordinal number} is defined as an equivalence class of well-ordered
sets under order isomorphism: it measures the \emph{order type} of a
well-ordered set, not merely its cardinality.
\end{remark}

% ---------------------------------------------------------
% Chains, antichains, initial segments
% ---------------------------------------------------------
\begin{definition}[Chain and Antichain]\label{def:chain}
A subset $C \subseteq A$ of a poset $(A,\leq)$ is a \emph{chain} if every
pair of elements in $C$ is comparable. It is an \emph{antichain} if no two
distinct elements of $C$ are comparable.
\end{definition}

\begin{definition}[Initial Segment]\label{def:initial-segment}
A subset $I \subseteq A$ is an \emph{initial segment} of $(A,\leq)$ if
\[
a \in I \text{ and } b \leq a \;\Rightarrow\; b \in I.
\]
\end{definition}

\begin{remark}[Transition]
Ordered sets provide the abstract framework for order structures on the real
numbers, function spaces, and metric spaces. In later sections, order
interacts with topology and analysis through intervals, monotone functions,
and the completeness axiom for $\mathbb{R}$.
\end{remark}

% =========================================================
% Order Extensions
% Source: Mark Dean, Order Theory (Brown University, 2015)
% =========================================================

\subsection{Order Extensions}

% ---------------------------------------------------------
% TOOLKIT
% ---------------------------------------------------------
\begin{tcolorbox}[colback=gray!6, colframe=gray!40, arc=2pt,
  left=6pt, right=6pt, top=4pt, bottom=4pt,
  title={\small\textbf{Order Extensions — Quick Reference}},
  fonttitle=\small\bfseries]
\small
\begin{tabular}{l l l}
\toprule
\textbf{Concept} & \textbf{Meaning} & \textbf{Detail} \\
\midrule
Preorder            & Reflexive, transitive (not necessarily antisymmetric)         & \hyperref[def:preorder]{↓ Def} \\
Loset               & Linear order = complete partial order                        & \hyperref[def:loset]{↓ Def} \\
Symmetric part      & $xIy \iff xRy \land yRx$                                     & \hyperref[def:sym-asym]{↓ Def} \\
Asymmetric part     & $xPy \iff xRy \land \lnot(yRx)$                              & \hyperref[def:sym-asym]{↓ Def} \\
Transitive closure  & Smallest transitive relation containing $R$                  & \hyperref[def:trans-closure]{↓ Def} \\
Extension           & $\succsim'$ agrees with $\succsim$ on strict and weak ranks  & \hyperref[def:extension]{↓ Def} \\
OWC                 & $xT(\succsim)y \Rightarrow \lnot(y \succ x)$                 & \hyperref[def:owc]{↓ Def} \\
Szpilrajn's Thm     & Every partial order extends to a linear order                & \hyperref[thm:szpilrajn]{↓ Thm} \\
\bottomrule
\end{tabular}
\end{tcolorbox}

\vspace{1em}

% ---------------------------------------------------------
% Symmetric and asymmetric parts
% ---------------------------------------------------------
\begin{tcolorbox}[colback=propbox, colframe=propborder, arc=2pt,
  left=6pt, right=6pt, top=4pt, bottom=4pt,
  title={\small\textbf{Definition (Symmetric and Asymmetric Parts)}},
  fonttitle=\small\bfseries]
\label{def:sym-asym}
For any binary relation $R$ on $X$, define:
\begin{itemize}
  \item the \emph{symmetric part}: $xIy \iff xRy \text{ and } yRx$
  \item the \emph{asymmetric part}: $xPy \iff xRy \text{ but not } yRx$
\end{itemize}
Note that $R = P \cup I$.
\end{tcolorbox}

\begin{remark}[Preference-theoretic reading]
In decision theory, $R$ encodes a preference relation: $xRy$ means ``$x$ is
at least as good as $y$.'' Then $I$ is the \emph{indifference} relation
(equally good) and $P$ is the \emph{strict preference} relation (strictly
better). This decomposition is fundamental to utility representation theory.
\end{remark}

% ---------------------------------------------------------
% Preorder and loset
% ---------------------------------------------------------
\begin{tcolorbox}[colback=propbox, colframe=propborder, arc=2pt,
  left=6pt, right=6pt, top=4pt, bottom=4pt,
  title={\small\textbf{Definition (Preorder and Loset)}},
  fonttitle=\small\bfseries]
\label{def:preorder}
A binary relation $R$ on $X$ is a \emph{preorder} if it is reflexive and
transitive.

\medskip
\label{def:loset}
A \emph{loset} (linearly ordered set) is a pair $(X, R)$ where $R$ is a
\emph{linear order}: a complete partial order ($R$ is reflexive, transitive,
antisymmetric, and complete).
\end{tcolorbox}

\begin{remark}[Preorder vs.\ partial order]
A preorder need not be antisymmetric: two distinct elements may satisfy
$xRy$ and $yRx$ simultaneously (they are ``indifferent'' but not equal).
Adding antisymmetry promotes a preorder to a partial order. Every partial
order is a preorder, but not conversely.
\end{remark}

% ---------------------------------------------------------
% Maximal elements vs. upper bounds
% ---------------------------------------------------------
\begin{tcolorbox}[colback=propbox, colframe=propborder, arc=2pt,
  left=6pt, right=6pt, top=4pt, bottom=4pt,
  title={\small\textbf{Definition (Maximal Elements and Upper Bounds)}},
  fonttitle=\small\bfseries]
\label{def:maximal-upper}
Let $\succsim$ be a binary relation on $X$.
\begin{itemize}
  \item The set of \emph{maximal elements} of $X$ is
    \[
      \mathrm{Max}(X, \succsim)
      = \{ x \in X \mid y \succsim x \text{ for no } y \in X \text{ with }
          y \neq x \}.
    \]
  \item The set of \emph{upper bounds} of $X$ is
    \[
      M(X, \succsim)
      = \{ x \in X \mid x \succsim y \text{ for all } y \in X \}.
    \]
\end{itemize}
\end{tcolorbox}

\begin{remark}[Maximal vs.\ greatest]
A maximal element has nothing strictly above it; an upper bound (greatest
element) is above everything. In a partial order, the greatest element is
unique if it exists and is always maximal, but maximal elements need not be
greatest. In a linear order, maximal and greatest coincide.
\end{remark}

% ---------------------------------------------------------
% Transitive closure
% ---------------------------------------------------------
\begin{tcolorbox}[colback=propbox, colframe=propborder, arc=2pt,
  left=6pt, right=6pt, top=4pt, bottom=4pt,
  title={\small\textbf{Definition (Transitive Closure)}},
  fonttitle=\small\bfseries]
\label{def:trans-closure}
The \emph{transitive closure} $T(R)$ of a binary relation $R$ on $X$ is the
smallest transitive relation containing $R$: that is, $T(R)$ is transitive,
$xRy \Rightarrow xT(R)y$, and no strictly smaller relation is both transitive
and contains $R$.
\end{tcolorbox}

\begin{remark}[Constructive description]
Define $R^0 = R$ and $xR^m y$ if there exist $z_1, \ldots, z_m \in X$ such
that $xRz_1 R \cdots R z_m Ry$. Then
\[
  T(R) = R \;\cup\; \bigcup_{m \in \mathbb{N}} R^m.
\]
Existence follows from the fact that the intersection of any collection of
transitive relations is transitive, so the intersection of all transitive
supersets of $R$ is the smallest such relation.
\end{remark}

% ---------------------------------------------------------
% Extensions of preorders
% ---------------------------------------------------------
\begin{tcolorbox}[colback=propbox, colframe=propborder, arc=2pt,
  left=6pt, right=6pt, top=4pt, bottom=4pt,
  title={\small\textbf{Definition (Extension of a Preorder)}},
  fonttitle=\small\bfseries]
\label{def:extension}
Let $\succsim$ be a preorder on $X$. A preorder $\succsim'$ is an
\emph{extension} of $\succsim$ if
\[
  x \succsim y \;\Rightarrow\; x \succsim' y,
  \qquad
  x \succ y \;\Rightarrow\; x \succ' y.
\]
A \emph{complete extension} is an extension that is also complete.
\end{tcolorbox}

\begin{remark}[What extensions do]
An extension of $\succsim$ ``fills in'' the comparisons left undecided by
$\succsim$ without reversing any existing strict comparison. The goal is to
promote a partial ranking to a total ranking while respecting the original
judgements.
\end{remark}

% ---------------------------------------------------------
% Szpilrajn's Theorem
% ---------------------------------------------------------
\begin{tcolorbox}[colback=thmbox, colframe=propborder, arc=2pt,
  left=6pt, right=6pt, top=4pt, bottom=4pt,
  title={\small\textbf{Theorem (Szpilrajn, 1930)}},
  fonttitle=\small\bfseries]
\label{thm:szpilrajn}
For any nonempty set $X$ and partial order $\succsim$ on $X$, there exists a
linear order that is an extension of $\succsim$.
\end{tcolorbox}

\begin{remark}[Proof sketch via Hausdorff Maximum Principle]
Let $T_X$ be the set of all partial orders on $X$ extending $\succsim$,
ordered by inclusion. By the Hausdorff Maximum Principle (equivalent to
AC), $T_X$ has a maximal chain; its union $\succsim^*$ is a partial order
extending $\succsim$. If $\succsim^*$ were not complete, one could enlarge
it via transitive closure, contradicting maximality. Hence $\succsim^*$ is a
linear order extending $\succsim$.

\medskip
\noindent\textit{Dependence on AC.} The theorem is equivalent to the Axiom
of Choice for infinite sets; no proof avoiding AC is known.
\end{remark}

\begin{corollary}[Complete preorder extension]
For any nonempty set $X$ and preorder $\succsim$ on $X$, there exists a
complete preorder that is an extension of $\succsim$.
\end{corollary}

\begin{remark}[Proof]
Pass to the quotient $X/{\sim}$ (where $\sim$ is the symmetric part), apply
Szpilrajn's Theorem to obtain a linear order on $X/{\sim}$, then pull back
to a complete preorder on $X$.
\end{remark}

% ---------------------------------------------------------
% Only Weak Cycles (OWC)
% ---------------------------------------------------------
\begin{tcolorbox}[colback=propbox, colframe=propborder, arc=2pt,
  left=6pt, right=6pt, top=4pt, bottom=4pt,
  title={\small\textbf{Definition and Proposition (Only Weak Cycles)}},
  fonttitle=\small\bfseries]
\label{def:owc}
Let $\succsim$ be a binary relation on $X$ with asymmetric part $\succ$ and
transitive closure $T(\succsim)$. We say $\succsim$ satisfies \emph{only weak
cycles} (OWC) if
\[
  xT(\succsim)y \;\Rightarrow\; \lnot(y \succ x).
\]

\medskip
\noindent\textbf{Proposition.} A binary relation $\succsim$ on a nonempty
set $X$ can be extended to a complete preorder if and only if it satisfies
OWC.
\end{tcolorbox}

\begin{remark}[Intuition for OWC]
OWC prohibits the following: there is a ``chain'' of weak preferences
$x_1 \succsim x_2 \succsim \cdots \succsim x_n$ but a strict reversal
$x_n \succ x_1$. Such a configuration cannot be extended to a complete
preorder because the chain forces $x_1 \succsim' x_n$ in any extension, but
strict preference $x_n \succ x_1$ in the extension would mean
$x_n \succsim' x_1$ but not $x_1 \succsim' x_n$ — contradicting
$x_1 \succsim' x_n$.
\end{remark}

\begin{remark}[Transition to AC]
Szpilrajn's Theorem relies on Zorn's Lemma / Hausdorff Maximum Principle,
which are equivalent to the Axiom of Choice. The next section develops
these equivalences directly.
\end{remark}



