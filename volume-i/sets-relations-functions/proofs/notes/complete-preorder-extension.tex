% =========================================================
% Proof: Complete preorder extension
% Source: volume-i/sets-relations-functions/notes/order/notes-order-extensions.tex
% =========================================================

\subsection*{Complete Preorder Extension}
\label{prf:complete-preorder-extension}

\begin{remark}[Return]
$\leftarrow$ Back to Corollary (Complete preorder extension) in Notes
\end{remark}

\begin{corollary}[Complete preorder extension]
For any nonempty set $X$ and preorder $\succsim$ on $X$, there exists a
complete preorder that is an extension of $\succsim$.
\end{corollary}

\begin{proof}
\Given Szpilrajn's Theorem (\hyperref[thm:szpilrajn]{Theorem~(Szpilrajn,
1930)}): every partial order on a nonempty set extends to a linear order.
The definition of the symmetric part of $\succsim$: $x \sim y$ iff $x
\succsim y$ and $y \succsim x$. The quotient construction: $\sim$ is an
equivalence relation, and the strict part $\succ$ of $\succsim$ descends to a
well-defined strict partial order on the quotient $X/{\sim}$.

\Goal To construct a complete preorder on $X$ extending $\succsim$.

\Strategy Pass to the quotient, apply Szpilrajn's Theorem, then pull back.

\medskip
\noindent\textbf{Step 1: The quotient.} Define $x \sim y$ iff $x \succsim y$
and $y \succsim x$ (indifference). Since $\succsim$ is a preorder, $\sim$ is
an equivalence relation. Let $[x]$ denote the equivalence class of $x$ and
$X/{\sim}$ the quotient set.

\medskip
\noindent\textbf{Step 2: The induced order on the quotient.} Define $[x]
\leq^* [y]$ iff $x \succsim y$ on $X/{\sim}$. This is well-defined (the
relation $\succsim$ is constant on equivalence classes) and antisymmetric
(if $[x] \leq^* [y]$ and $[y] \leq^* [x]$, then $x \succsim y$ and $y
\succsim x$, so $x \sim y$, i.e.\ $[x] = [y]$). Since $\succsim$ is also
reflexive and transitive, $\leq^*$ is a partial order on $X/{\sim}$.

\medskip
\noindent\textbf{Step 3: Apply Szpilrajn.} By Szpilrajn's Theorem, there
exists a linear order $\leq^{**}$ on $X/{\sim}$ extending $\leq^*$.

\medskip
\noindent\textbf{Step 4: Pull back to $X$.} Define $x \succsim' y$ iff $[x]
\leq^{**} [y]$. We verify:

\textit{Complete preorder:} $\succsim'$ is reflexive (since $[x] \leq^{**}
[x]$), transitive (since $\leq^{**}$ is), and complete (since $\leq^{**}$ is
a linear order, any two classes are comparable, hence any two elements of $X$
are comparable under $\succsim'$).

\textit{Extension:} If $x \succsim y$, then $[x] \leq^* [y]$, so $[x]
\leq^{**} [y]$ (since $\leq^{**}$ extends $\leq^*$), so $x \succsim' y$. If
$x \succ y$ (i.e.\ $x \succsim y$ but not $y \succsim x$), then $[x] <^*
[y]$ strictly, so $[x] <^{**} [y]$ strictly (since $\leq^{**}$ extends the
strict part), so $x \succ' y$.

Hence $\succsim'$ is a complete preorder extending $\succsim$. \AsReq
\end{proof}

\begin{remark}[Proof shape]
The proof is a \emph{quotient-then-extend-then-pull-back} construction. The
difficulty with applying Szpilrajn directly to a preorder is that preorders are
not antisymmetric (indifferent elements are not forced to be equal). The
quotient removes this obstruction, turning the preorder into a genuine partial
order. Szpilrajn then provides a linear extension. Pulling back along the
quotient map reintroduces the indifference structure as a complete preorder.
\end{remark}

\begin{remark}[Dependence on the Axiom of Choice]
Szpilrajn's Theorem relies on Zorn's Lemma (or equivalently, the Axiom of
Choice). This corollary inherits that dependence. For finite sets, the Axiom
of Choice is not needed: one can construct a linear extension algorithmically.
\end{remark}

\begin{remark}[Dependencies]
The proof depends on: \hyperref[thm:szpilrajn]{Szpilrajn's Theorem}, the
definition of quotient by an equivalence relation, the definition of a complete
preorder (\hyperref[def:preorder-ext]{Definition~(Preorder and Loset)}), and
the definition of an extension of a preorder
(\hyperref[def:preorder-extension]{Definition~(Extension of a Preorder)}).
\end{remark}
