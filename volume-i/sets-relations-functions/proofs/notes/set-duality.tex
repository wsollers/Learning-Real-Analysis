% =========================================================
% Proof: Principle of Set Duality
% Source: volume-i/sets-relations-functions/notes/sets/notes-set-operations.tex
% =========================================================

\subsection*{Principle of Set Duality}
\label{prf:set-duality}

\begin{remark}[Return]
\hyperref[cor:set-duality]{$\leftarrow$ Back to Corollary (Principle of Set Duality) in Notes}
\end{remark}

\begin{proof}
\Claim Any set-theoretic identity involving $\cup$, $\cap$, $\varnothing$, and $U$ remains valid when each operation and constant is replaced by its dual: $\cup \leftrightarrow \cap$ and $\varnothing \leftrightarrow U$.

\Given De Morgan's laws (\hyperref[thm:de-morgan]{Theorem~(De Morgan's Laws)}):
$(A \cup B)^c = A^c \cap B^c$ and $(A \cap B)^c = A^c \cup B^c$. The
involution law: $(A^c)^c = A$ for all $A \subseteq U$. The boundary cases:
$\varnothing^c = U$ and $U^c = \varnothing$.

\Goal To show that if an identity $E = F$ holds for all subsets of $U$, then
its dual identity $E^* = F^*$ also holds for all subsets of $U$, where the
dual is obtained by swapping $\cup \leftrightarrow \cap$ and $\varnothing
\leftrightarrow U$.

\Strategy We show that if we take complements of both sides of $E = F$ and
apply De Morgan, we obtain $E^* = F^*$.

\medskip
Given the identity $E = F$, take complements of both sides: $E^c = F^c$.

Now apply De Morgan's laws to every occurrence of $\cup$ and $\cap$ in $E^c$
and $F^c$. The key substitutions are:
\begin{itemize}
  \item $(A \cup B)^c = A^c \cap B^c$ — complement turns $\cup$ into $\cap$.
  \item $(A \cap B)^c = A^c \cup B^c$ — complement turns $\cap$ into $\cup$.
  \item $\varnothing^c = U$ and $U^c = \varnothing$ — the boundary constants
        swap.
  \item $(A^c)^c = A$ — double complements cancel, restoring the original
        variables.
\end{itemize}

After applying these substitutions throughout $E^c$ and $F^c$, each variable
$A$ appears without complement (the two complementations cancel), and every
$\cup$ has become $\cap$ and vice versa. The constants $\varnothing$ and $U$
have swapped. The result is precisely $E^* = F^*$.

Since the identity $E = F$ holds for all subsets of $U$, so does $E^c = F^c$,
and therefore $E^* = F^*$ holds for all subsets of $U$. \AsReq
\end{proof}

\begin{remark}[Proof shape]
The argument is a \emph{structural transformation}: complement both sides of the
known identity, then use De Morgan to rewrite the complements of compound
expressions. The double-complement law cleans up the variables, and the
boundary swaps handle the constants. The result is the dual identity.
\end{remark}

\begin{remark}[This is a metatheorem]
The Principle of Set Duality is a statement about \emph{all} set-theoretic
identities, not about any particular one. It says we never need to prove both
of a dual pair: one proof suffices and the other is free. For example,
proving $A \cap (B \cup C) = (A \cap B) \cup (A \cap C)$ immediately gives
$A \cup (B \cap C) = (A \cup B) \cap (A \cup C)$ by duality.
\end{remark}

\begin{remark}[Dependencies]
The proof depends on: \hyperref[thm:de-morgan]{De Morgan's Laws}, the involution
of complement ($(A^c)^c = A$), and the boundary facts $\varnothing^c = U$ and
$U^c = \varnothing$.
\end{remark}
