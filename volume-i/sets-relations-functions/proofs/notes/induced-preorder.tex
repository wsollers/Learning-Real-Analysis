% =========================================================
% Proof: ≤_f is always a preorder
% Source: volume-i/sets-relations-functions/notes/order/notes-order-induced.tex
% =========================================================

\subsection*{$\leq_f$ is Always a Preorder}
\label{prf:induced-preorder}

\begin{remark}[Return]
\hyperref[prop:induced-preorder]{$\leftarrow$ Back to Proposition ($\leq_f$ is always a preorder) in Notes}
\end{remark}

\begin{proof}
\Claim For any function $f : A \to B$ and partial order $(B, \leq')$, the induced relation $\leq_f$ defined by $x \leq_f y \Leftrightarrow f(x) \leq' f(y)$ is a preorder on $A$.

\Given The induced order (\hyperref[def:induced-order]{Definition~(Induced Order)}):
$x \leq_f y \Leftrightarrow f(x) \leq' f(y)$. The partial order $(B, \leq')$:
$\leq'$ is reflexive, transitive, and antisymmetric.

\Goal To verify reflexivity and transitivity of $\leq_f$ on $A$.

\medskip
\noindent\textbf{Reflexivity.} Let $x \in A$. Since $\leq'$ is reflexive,
$f(x) \leq' f(x)$. By definition of $\leq_f$, $x \leq_f x$.

\medskip
\noindent\textbf{Transitivity.} Suppose $x \leq_f y$ and $y \leq_f z$. By
definition, $f(x) \leq' f(y)$ and $f(y) \leq' f(z)$. Since $\leq'$ is
transitive, $f(x) \leq' f(z)$. By definition, $x \leq_f z$. \AsReq
\end{proof}

\begin{remark}[Proof shape]
Both properties are one-line pullbacks. Reflexivity at $x \in A$ pulls back
reflexivity at $f(x) \in B$; transitivity along $(x, y, z)$ in $A$ pulls back
transitivity along $(f(x), f(y), f(z))$ in $B$. The structure of the proof
mirrors the structure of the definition: $\leq_f$ is defined by comparing
$f$-images, so properties of $\leq'$ transfer immediately to $\leq_f$.
\end{remark}

\begin{remark}[Antisymmetry does not lift freely]
Antisymmetry of $\leq'$ does not automatically give antisymmetry of $\leq_f$.
If $f$ is not injective, there may exist $x \neq y$ in $A$ with $f(x) = f(y)$,
giving $x \leq_f y$ and $y \leq_f x$ with $x \neq y$. This is why
$\leq_f$ is only guaranteed to be a preorder, not a partial order.
See \hyperref[prop:induced-poset]{$\leq_f$ is a partial order iff $f$ is injective}.
\end{remark}

\begin{remark}[Dependencies]
The proof depends on: the Definition of induced order
(\hyperref[def:induced-order]{Definition~(Induced Order)}) and reflexivity and
transitivity of $(B, \leq')$.
\end{remark}
