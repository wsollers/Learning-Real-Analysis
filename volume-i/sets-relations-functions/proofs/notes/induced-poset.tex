% =========================================================
% Proof: ≤_f is a partial order iff f is injective
% Source: volume-i/sets-relations-functions/notes/order/notes-order-induced.tex
% =========================================================

\subsection*{$\leq_f$ is a Partial Order iff $f$ is Injective}
\label{prf:induced-poset}

\begin{remark}[Return]
\hyperref[prop:induced-poset]{$\leftarrow$ Back to Proposition ($\leq_f$ is a partial order iff $f$ is injective) in Notes}
\end{remark}

\begin{proof}
\Claim Let $(B, \leq')$ be a partially ordered set and $f : A \to B$. The induced order $\leq_f$ is a partial order on $A$ if and only if $f$ is injective.

\Given $\leq_f$ is always a preorder
(\hyperref[prop:induced-preorder]{$\leq_f$ is always a preorder}). The
induced order definition: $x \leq_f y \Leftrightarrow f(x) \leq' f(y)$.
Antisymmetry of $\leq'$: $f(x) \leq' f(y)$ and $f(y) \leq' f(x)$ imply
$f(x) = f(y)$.

\Goal To show $\leq_f$ is antisymmetric if and only if $f$ is injective.
(Since $\leq_f$ is already a preorder, the only missing property for a partial
order is antisymmetry.)

\medskip
\noindent\textbf{($\Rightarrow$) If $\leq_f$ is antisymmetric, then $f$ is
injective.}

Suppose $f(x) = f(y)$. Then $f(x) \leq' f(y)$ (by reflexivity of $\leq'$)
and $f(y) \leq' f(x)$. By definition, $x \leq_f y$ and $y \leq_f x$. By
antisymmetry of $\leq_f$, $x = y$. So $f(x) = f(y) \Rightarrow x = y$, i.e.\
$f$ is injective.

\medskip
\noindent\textbf{($\Leftarrow$) If $f$ is injective, then $\leq_f$ is
antisymmetric.}

Suppose $x \leq_f y$ and $y \leq_f x$. By definition, $f(x) \leq' f(y)$ and
$f(y) \leq' f(x)$. By antisymmetry of $\leq'$, $f(x) = f(y)$. By injectivity
of $f$, $x = y$.

Therefore $\leq_f$ is antisymmetric, and since it is also a preorder, it is a
partial order. \AsReq
\end{proof}

\begin{remark}[Proof shape]
This is a clean biconditional proof. Each direction is one short chain:
$(\Rightarrow)$ shows that antisymmetry of $\leq_f$ forces injectivity by
constructing the right test case ($f(x) = f(y)$); $(\Leftarrow)$ shows
injectivity gives antisymmetry by tracing the chain $x \leq_f y \leq_f x
\Rightarrow f(x) = f(y) \Rightarrow x = y$.
\end{remark}

\begin{remark}[The exact obstruction]
The proposition identifies injectivity as the \emph{exact} condition for
antisymmetry to lift. Any non-injective $f$ (with $f(x) = f(y)$ for some $x
\neq y$) produces a violation: $x \leq_f y$ and $y \leq_f x$ but $x \neq y$.
Injectivity rules out all such pairs simultaneously.
\end{remark}

\begin{remark}[Dependencies]
The proof depends on: \hyperref[prop:induced-preorder]{$\leq_f$ is always a
preorder}, the Definition of induced order
(\hyperref[def:induced-order]{Definition~(Induced Order)}), antisymmetry of
$(B, \leq')$, and the definition of injectivity.
\end{remark}
