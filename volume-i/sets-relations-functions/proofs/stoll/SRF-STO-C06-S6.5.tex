% --------------------------------------
% Proof: SRF-STO-C06-S6.5
% --------------------------------------
\clearpage
\phantomsection
\hypertarget{proof-SRF-STO-C06-S6.5}{}

\subsubsection[Segments of a Chain are Comparable]{Proof --- SRF-STO-C06-S6.5}

\bigskip

\noindent
\textbf{Source.}
\srccite{StollSetTheoryLogic}{Chapter~2, \S6, Exercise~6.5}.

\vspace{0.75em}

\noindent
\textbf{Key definition.}
A \emph{segment} $B$ of a chain $A$ satisfies: if $b \in B$, $a \in A$,
and $a \leq b$, then $a \in B$. (Downward-closure with respect to the
order on $A$.)

\vspace{0.75em}

\noindent
\textbf{Goal.}
Let $A$ be a chain and let $B$, $C$ be segments of $A$. Prove that
$B \subseteq C$ or $C \subseteq B$.

\vspace{0.75em}

\noindent
\textbf{Logical form.}
$\forall$ chains $A$, $\forall$ segments $B, C$ of $A$:
$B \subseteq C \lor C \subseteq B$.

\vspace{0.75em}

\noindent
\textbf{Proof strategy.}
Proof by contradiction. Assume neither is a subset of the other; extract
witnesses; use totality of the chain order and the segment property to
derive a contradiction.

\vspace{0.75em}

\noindent
\textbf{Proof.}
\begin{proof}
We wish to show $B \subseteq C$ or $C \subseteq B$.

Suppose for contradiction that neither $B \subseteq C$ nor $C \subseteq B$.
Then there exist elements
\[
  b \in B \setminus C \quad \text{and} \quad c \in C \setminus B.
\]
Since $A$ is a chain (totally ordered), $b$ and $c$ are comparable.
Thus either $b \leq c$ or $c \leq b$.

\medskip
\textit{Case 1:} $b \leq c$.
Since $c \in C$ and $C$ is a segment of $A$, and $b \in A$ with
$b \leq c$, we conclude $b \in C$.
But $b \in B \setminus C$, a contradiction.

\medskip
\textit{Case 2:} $c \leq b$.
% Complete symmetrically.

\medskip
In both cases we reach a contradiction.
Therefore $B \subseteq C$ or $C \subseteq B$.
\end{proof}

\vspace{0.75em}

\noindent
\textbf{Remarks.}
\begin{itemize}
  \item \textbf{Mistakes to avoid:} Do not confuse ``segment'' with
    ``interval.'' The segment condition is a downward-closure condition;
    $\varnothing$ and $A$ itself are always segments.
\end{itemize}
