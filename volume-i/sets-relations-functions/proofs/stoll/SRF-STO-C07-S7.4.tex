% --------------------------------------
% Proof: SRF-STO-C07-S7.4
% --------------------------------------
\clearpage
\phantomsection
\hypertarget{proof-SRF-STO-C07-S7.4}{}

\subsubsection[Principle of Transfinite Induction]{Proof --- SRF-STO-C07-S7.4}

\bigskip

\noindent
\textbf{Source.}
\srccite{StollSetTheoryLogic}{Chapter~2, \S7, Exercise~7.4}.

\vspace{0.75em}

\noindent
\textbf{Goal.}
Let $\langle A, \leq \rangle$ be a well-ordered set and let $P(a)$ be a
property defined for elements $a \in A$. Suppose that for every $a \in A$:
if $P(b)$ holds for all $b < a$, then $P(a)$ holds. Prove that $P(a)$
holds for all $a \in A$.

\vspace{0.75em}

\noindent
\textbf{Logical form.}
$(\forall a \in A)[(\forall b < a)\, P(b) \Rightarrow P(a)]
\;\Rightarrow\; (\forall a \in A)\, P(a)$.

\vspace{0.75em}

\noindent
\textbf{Proof strategy.}
Proof by contradiction using the well-ordering principle. The set of
failures has a least element; the inductive hypothesis then forces that
element to satisfy $P$, yielding a contradiction.

\vspace{0.75em}

\noindent
\textbf{Proof.}
\begin{proof}
Suppose for contradiction that $P(a)$ fails for some $a \in A$. Let
\[
  F = \{ a \in A : P(a) \text{ is false} \}.
\]
By assumption, $F$ is nonempty. Since $A$ is well-ordered, $F$ has a
least element $a_0 \in F$.

Since $a_0$ is the least element of $F$, for every $b < a_0$ we have
$b \notin F$, i.e., $P(b)$ holds.

By the hypothesis of the theorem applied to $a_0$: since $P(b)$ holds for
all $b < a_0$, we conclude $P(a_0)$ holds.

But $a_0 \in F$ means $P(a_0)$ is false --- a contradiction.

Therefore $F = \varnothing$, and $P(a)$ holds for all $a \in A$.
\end{proof}

\vspace{0.75em}

\noindent
\textbf{Remarks.}
\begin{itemize}
  \item \textbf{Connection to ordinary induction:} When
    $A = \mathbb{N}$ with its standard order, this reduces to strong
    induction. Transfinite induction extends this to arbitrary well-ordered
    sets, including uncountable ordinals.
  \item \textbf{Capstone:} This is the payoff of the entire well-ordering
    development. Inductive arguments become available for \emph{any}
    well-ordered set, not just $\mathbb{N}$.
  \item \textbf{Note:} The inductive hypothesis here is the \emph{strong}
    form --- $P$ is assumed for \emph{all} predecessors, not just the
    immediate one.
\end{itemize}
