% --------------------------------------
% Proof: SRF-STO-C06-S6.6
% --------------------------------------
\clearpage
\phantomsection
\hypertarget{proof-SRF-STO-C06-S6.6}{}

\subsubsection[Chains of Order Type $\omega$]{Proof --- SRF-STO-C06-S6.6}

\bigskip

\noindent
\textbf{Source.}
\srccite{StollSetTheoryLogic}{Chapter~2, \S6, Exercise~6.6}.

\vspace{0.75em}

\noindent
\textbf{Goal.}
Prove that a chain is of order type $\omega$ if and only if it is
infinite and every proper segment is finite.

\vspace{0.75em}

\noindent
\textbf{Logical form.}
$\langle A, \leq \rangle$ has order type $\omega$
$\iff$ $A$ is infinite $\land$ every proper segment of $A$ is finite.

\vspace{0.75em}

\noindent
\textbf{Proof strategy.}
Biconditional --- prove each direction separately.
$(\Rightarrow)$: Use the explicit isomorphism to transfer finiteness of
proper initial segments of $\mathbb{N}$ to proper segments of $A$.
$(\Leftarrow)$: Construct the isomorphism by recursion, using finiteness
of proper segments to guarantee a least element at each step.

\vspace{0.75em}

\noindent
\textbf{Proof.}
\begin{proof}
Let $\langle A, \leq \rangle$ be a chain.

\medskip
\noindent $(\Rightarrow)$ Suppose $A$ has order type $\omega$, so $A$ is
order-isomorphic to $\langle \mathbb{N}, \leq \rangle$ via some bijection
$f : \mathbb{N} \to A$.

\textit{$A$ is infinite:} $\mathbb{N}$ is infinite and $f$ is a bijection,
so $A$ is infinite.

\textit{Every proper segment is finite:} Let $B \subsetneq A$ be a proper
segment.
% Argue that f^{-1}(B) is a proper initial segment of N, hence finite.

\medskip
\noindent $(\Leftarrow)$ Suppose $A$ is infinite and every proper segment
of $A$ is finite. We construct an order-isomorphism $f : \mathbb{N} \to A$.

% Define f by recursion: f(0) is the least element of A;
% f(n+1) is the least element of A \ {f(0),...,f(n)}.
% Verify f is well-defined, order-preserving, and bijective.
\end{proof}

\vspace{0.75em}

\noindent
\textbf{Remarks.}
\begin{itemize}
  \item \textbf{Connection to $\mathbb{N}$:} This result gives an
    order-theoretic fingerprint of $\mathbb{N}$: it is (up to isomorphism)
    the unique infinite chain with all proper segments finite.
  \item \textbf{Mistakes to avoid:} For the reverse direction, verify the
    recursion does not terminate prematurely, i.e., that
    $A \setminus \{f(0),\ldots,f(n)\}$ is always nonempty. This uses
    the assumption that $A$ is infinite.
\end{itemize}
