% --------------------------------------
% Proof: SRF-STO-C07-S7.1
% --------------------------------------
\clearpage
\phantomsection
\hypertarget{proof-SRF-STO-C07-S7.1}{}

\subsubsection[Antisymmetry and Minimum Element Imply Well-Ordering]{Proof --- SRF-STO-C07-S7.1}

\bigskip

\noindent
\textbf{Source.}
\srccite{StollSetTheoryLogic}{Chapter~2, \S7, Exercise~7.1}.

\vspace{0.75em}

\noindent
\textbf{Goal.}
Let $\rho$ be a relation on a set $A$. Suppose $\rho$ is antisymmetric
and for every nonempty subset $A_1 \subseteq A$ there exists an element
$a_1 \in A_1$ such that $a_1\,\rho\,b$ for every $b \in A_1$. Prove
that $\rho$ well-orders $A$.

\vspace{0.75em}

\noindent
\textbf{Logical form.}
$\rho$ antisymmetric $\land$ every nonempty $A_1 \subseteq A$ has a
$\rho$-minimum $\Rightarrow$ $\rho$ well-orders $A$.

\vspace{0.75em}

\noindent
\textbf{Proof strategy.}
The minimum-element condition is given directly. The work is verifying
the order axioms (reflexivity, transitivity, totality). Use singleton and
two-element subsets as test cases.

\vspace{0.75em}

\noindent
\textbf{Proof.}
\begin{proof}
To show $\rho$ well-orders $A$, we verify that $\rho$ is a total order
on $A$ and that every nonempty subset has a $\rho$-least element.
Condition (ii) is given directly; it remains to verify $\rho$ is a total
order.

\medskip
\textit{Reflexivity:} Let $a \in A$. Apply the hypothesis to the singleton
$A_1 = \{a\}$. There exists $a_1 \in \{a\}$ such that $a_1\,\rho\,b$ for
all $b \in \{a\}$. So $a_1 = a$ and $a\,\rho\,a$.

\medskip
\textit{Transitivity:} Let $a\,\rho\,b$ and $b\,\rho\,c$.
% Apply the hypothesis to A_1 = {a, b, c} to obtain a least element,
% then use antisymmetry and the given rho-relations to conclude a rho c.

\medskip
\textit{Totality:} Let $a, b \in A$ be arbitrary. Apply the hypothesis to
$A_1 = \{a, b\}$. There exists a least element $a_1 \in \{a, b\}$ with
$a_1\,\rho\,a$ and $a_1\,\rho\,b$. Either $a_1 = a$, giving $a\,\rho\,b$,
or $a_1 = b$, giving $b\,\rho\,a$. Hence $\rho$ is total.

\medskip
Combining reflexivity, transitivity, antisymmetry (given), and totality,
$\rho$ is a total order. Together with the minimum-element condition,
$\rho$ well-orders $A$.
\end{proof}

\vspace{0.75em}

\noindent
\textbf{Remarks.}
\begin{itemize}
  \item \textbf{Note:} The singleton-set trick for reflexivity is elegant
    and worth internalizing as a proof pattern.
  \item \textbf{Mistakes to avoid:} Do not assume transitivity from the
    hypothesis alone; it must be derived. The step requiring work is the
    transitivity case.
\end{itemize}
