% --------------------------------------
% Proof: SRF-STO-C07-S7.2
% --------------------------------------
\clearpage
\phantomsection
\hypertarget{proof-SRF-STO-C07-S7.2}{}

\subsubsection[Ordinal Similarity Preserves Well-Ordering]{Proof --- SRF-STO-C07-S7.2}

\bigskip

\noindent
\textbf{Source.}
\srccite{StollSetTheoryLogic}{Chapter~2, \S7, Exercise~7.2}.

\vspace{0.75em}

\noindent
\textbf{Goal.}
Prove that a simply ordered set which is ordinally similar
(order-isomorphic) to a well-ordered set is itself well-ordered.

\vspace{0.75em}

\noindent
\textbf{Logical form.}
$\langle A, \leq_A \rangle$ well-ordered $\land$
$f : A \xrightarrow{\sim} B$ order-isomorphism
$\Rightarrow$ $\langle B, \leq_B \rangle$ well-ordered.

\vspace{0.75em}

\noindent
\textbf{Proof strategy.}
Transfer structure through the isomorphism. For any nonempty $S \subseteq B$,
the preimage $f^{-1}(S)$ is nonempty (surjectivity of $f$); apply the
well-ordering of $A$ to get a least element; show its image under $f$ is
the least element of $S$.

\vspace{0.75em}

\noindent
\textbf{Proof.}
\begin{proof}
Let $\langle A, \leq_A \rangle$ be well-ordered and let
$f : A \to B$ be an order-isomorphism to the simply ordered set
$\langle B, \leq_B \rangle$.

We show every nonempty $S \subseteq B$ has a $\leq_B$-least element.

Let $S \subseteq B$ be nonempty. Since $f$ is surjective, $f^{-1}(S)$ is
nonempty. Since $A$ is well-ordered, $f^{-1}(S)$ has a least element
$a_0 \in f^{-1}(S)$.

% Show f(a_0) is the least element of S under leq_B.
% Use the order-preserving property of f in both directions.
\end{proof}

\vspace{0.75em}

\noindent
\textbf{Remarks.}
\begin{itemize}
  \item \textbf{Key insight:} Well-ordering is an order-theoretic property,
    hence invariant under order-isomorphism.
  \item \textbf{Mistakes to avoid:} Verify both that $f(a_0) \in S$ and
    that $f(a_0) \leq_B s$ for all $s \in S$.
\end{itemize}
