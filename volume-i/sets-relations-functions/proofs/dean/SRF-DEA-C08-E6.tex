% --------------------------------------
% Proof: SRF-DEA-C08-E6
% --------------------------------------
\clearpage
\phantomsection
\hypertarget{proof-SRF-DEA-C08-E6}{}

\subsubsection[Poset with Longest Chain $n$ Decomposes into $n$ Antichains]{Proof --- SRF-DEA-C08-E6}

\bigskip

\noindent
\textbf{Source.}
\srccite{Dean2015OrderTheory}{Chapter~VIII, Exercise~6}.

\vspace{0.75em}

\noindent
\textbf{Goal.}
Let $(X, \leq)$ be a poset in which the longest chain has length $n \in
\mathbb{N}$. Prove that $X$ can be written as the union of $n$ totally
unordered (antichain) subsets, and that $n$ is the minimum number of
antichains needed.

\vspace{0.75em}

\noindent
\textbf{Logical form.}
$\mathrm{length}(\text{longest chain in } X) = n
\Rightarrow X = A_1 \cup \cdots \cup A_n$ where each $A_i$ is an antichain,
and $n$ is optimal.

\vspace{0.75em}

\noindent
\textbf{Key definition.}
The \emph{length} of a chain is the number of elements minus 1 (or the
number of strict inequalities).

\vspace{0.75em}

\noindent
\textbf{Proof strategy.}
Define $A_k = \{x \in X : \text{the longest chain ending at } x \text{ has
length exactly } k-1\}$ for $k = 1, \ldots, n$. Prove:
(i) each $A_k$ is an antichain, (ii) $X = \bigcup A_k$, (iii) fewer than
$n$ antichains do not suffice.

\vspace{0.75em}

\noindent
\textbf{Proof.}
\begin{proof}
For each $x \in X$, define $\ell(x)$ as the length of the longest chain in
$X$ ending at $x$ (starting from a minimal element). Set
$A_k = \{x \in X : \ell(x) = k-1\}$ for $k = 1, \ldots, n$.

\textit{Each $A_k$ is an antichain:} Suppose $x, y \in A_k$ and $x < y$.
Then $\ell(y) \geq \ell(x) + 1 = k$, so $y \notin A_k$. Contradiction.

\textit{$X = \bigcup_{k=1}^n A_k$:} Since the longest chain has length
$n-1$ (i.e., $n$ elements), $\ell(x) \in \{0, 1, \ldots, n-1\}$ for all
$x$. Hence every $x$ belongs to some $A_k$.

\textit{Optimality:} Any chain of length $n-1$ has $n$ elements, each with
a distinct $\ell$-value, hence each in a different antichain. So at least $n$
antichains are required.
\end{proof}

\vspace{0.75em}

\noindent
\textbf{Remarks.}
\begin{itemize}
  \item \textbf{Mirsky's theorem:} This result is known as Mirsky's theorem
    (1971), the ``dual'' of Dilworth's theorem. Dilworth decomposes a poset
    into chains using the antichain width; Mirsky decomposes into antichains
    using the chain height.
  \item \textbf{Connection to E5:} If $X$ is infinite with no infinite
    antichain, then the longest chain is infinite (by E5) and this
    decomposition has infinitely many layers.
\end{itemize}
