% --------------------------------------
% Proof: SRF-DEA-C08-E13
% --------------------------------------
\clearpage
\phantomsection
\hypertarget{proof-SRF-DEA-C08-E13}{}

\subsubsection[A Well-Ordered Subset of $\mathbb{R}$ is Countable]{Proof --- SRF-DEA-C08-E13}

\bigskip

\noindent
\textbf{Source.}
\srccite{Dean2015OrderTheory}{Chapter~VIII, Exercise~13}.

\vspace{0.75em}

\noindent
\textbf{Goal.}
Suppose $G \subseteq \mathbb{R}$ and $G$ is well-ordered in the usual order
on $\mathbb{R}$. Prove that $G$ is countable.

\vspace{0.75em}

\noindent
\textbf{Logical form.}
$G \subseteq \mathbb{R}$, $G$ well-ordered $\Rightarrow$ $G$ countable.

\vspace{0.75em}

\noindent
\textbf{Proof strategy.}
Use the density of $\mathbb{Q}$ in $\mathbb{R}$. Between any two consecutive
elements of $G$ (i.e., $g < g'$ with no element of $G$ between them) there
is a rational. Assign a distinct rational to each such gap. Since $\mathbb{Q}$
is countable, the set of gaps is countable, and $G$ is countable.

\vspace{0.75em}

\noindent
\textbf{Proof.}
\begin{proof}
For each element $g \in G$ that is not the largest element of $G$ (if a
largest exists), define the \emph{successor} of $g$ in $G$ as the least
element of $G$ strictly greater than $g$ (which exists by well-ordering).
Call this $s(g)$.

For each $g$ with a successor, by density of $\mathbb{Q}$ in $\mathbb{R}$,
there exists $q_g \in \mathbb{Q}$ with $g < q_g < s(g)$.

The map $g \mapsto q_g$ is injective: if $g \neq h$ are both non-maximal
elements, then $(g, s(g))$ and $(h, s(h))$ are disjoint (consecutive
elements in a well-order have non-overlapping successor intervals), so
$q_g \neq q_h$.

% Therefore, all non-maximal elements of G inject into Q, so there are
% countably many. The maximal element (if it exists) contributes at most 1.
% Hence G is countable.
\end{proof}

\vspace{0.75em}

\noindent
\textbf{Remarks.}
\begin{itemize}
  \item \textbf{Why well-ordering is necessary:} The statement is false
    without the well-ordering hypothesis: $[0,1] \subset \mathbb{R}$ is
    uncountable but its usual order is not a well-order ($\varnothing \neq
    (0,1]$ has no least element in the subspace).
  \item \textbf{Contrast with general sets:} Uncountable well-ordered sets
    exist (e.g., $\omega_1$, the first uncountable ordinal) — but they
    cannot embed into $\mathbb{R}$ with its usual order, precisely because
    of this theorem.
\end{itemize}
