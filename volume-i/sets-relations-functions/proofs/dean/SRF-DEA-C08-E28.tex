% --------------------------------------
% Proof: SRF-DEA-C08-E28
% --------------------------------------
\clearpage
\phantomsection
\hypertarget{proof-SRF-DEA-C08-E28}{}

\subsubsection[Maximal vs.\ Largest: Countable Subsets of $\mathbb{R}$]{Proof --- SRF-DEA-C08-E28}

\bigskip

\noindent
\textbf{Source.}
\srccite{Dean2015OrderTheory}{Chapter~VIII, Exercise~28}.

\vspace{0.75em}

\noindent
\textbf{Goal.}
Resolve the following argument. Ray claims Zorn's Lemma gives a maximal
countable subset of $\mathbb{R}$ (in the poset of countably infinite subsets
ordered by inclusion). Debra observes that any countable set can be enlarged
by one element and still be countable, so no ``largest'' one exists. Ray
replies he said \emph{maximal}, not largest.

Explain: (1) Ray's Zorn argument is valid; (2) Debra's observation is also
correct; (3) there is no contradiction because maximal $\neq$ largest.

\vspace{0.75em}

\noindent
\textbf{Proof.}
\begin{proof}
Let $\mathcal{F}$ be the poset of all countably infinite subsets of
$\mathbb{R}$ ordered by inclusion.

\textit{Ray's argument is valid.}
Every chain in $\mathcal{F}$ (i.e., every collection of countably infinite
subsets totally ordered by inclusion) has an upper bound: the union of the
chain. A countable union of countable sets is countable, so the union is
countable; it is infinite since each set in the chain is infinite. Thus
every chain in $\mathcal{F}$ has an upper bound in $\mathcal{F}$, and by
Zorn's Lemma, $\mathcal{F}$ has a maximal element $M$.

\textit{Debra's observation is also correct.}
For any $A \in \mathcal{F}$ and any $r \in \mathbb{R} \setminus A$, the set
$A \cup \{r\}$ is also countably infinite, hence in $\mathcal{F}$, and
$A \subsetneq A \cup \{r\}$. So no element of $\mathcal{F}$ is a
\emph{largest} element.

\textit{Resolution: maximal $\neq$ largest.}
An element $M \in \mathcal{F}$ is \emph{maximal} if there is no
$A \in \mathcal{F}$ with $M \subsetneq A$. Since $\mathcal{F}$ is ordered
by inclusion and $\mathbb{R}$ is uncountable, for a maximal $M$ there
exist $r \in \mathbb{R} \setminus M$, and $M \cup \{r\}$ is countable but
could still contain $M$ as a proper subset — so $M$ need not block
enlargement. Wait: if $M \cup \{r\} \in \mathcal{F}$ and $M \subsetneq M
\cup \{r\}$, then $M$ is not maximal!

The resolution is: Zorn's Lemma gives a maximal element in $\mathcal{F}$,
but since $\mathbb{R}$ is uncountable and every element of $\mathcal{F}$ can
be enlarged by adding a new element (which stays countably infinite), \emph{no}
element of $\mathcal{F}$ is maximal in $\mathcal{F}$. This means Zorn's Lemma
\emph{cannot} be applied here: the union of an arbitrary chain is countable,
but chains can be uncountable in length if the sets have no ``last'' inclusion
step. In fact, \textbf{Ray's argument has a gap}: the union of an
uncountable chain of countably infinite sets can be uncountable, hence not in
$\mathcal{F}$.

\medskip
\textbf{Corrected conclusion:} Debra is right: there is no maximal countable
subset of $\mathbb{R}$, and Ray's Zorn argument is flawed. The upper bound of
a chain in $\mathcal{F}$ is not necessarily countable when the chain is
uncountable.
\end{proof}

\vspace{0.75em}

\noindent
\textbf{Remarks.}
\begin{itemize}
  \item \textbf{Lesson:} When applying Zorn's Lemma, verify that the union
    of every chain is actually in the poset. Here, an uncountable chain of
    countable sets can have an uncountable union — violating membership in
    $\mathcal{F}$.
  \item \textbf{Maximal vs.\ largest:} In a partial order, ``largest''
    (above all elements) implies maximal (nothing strictly above), but not
    conversely. Debra confuses the two.
\end{itemize}
