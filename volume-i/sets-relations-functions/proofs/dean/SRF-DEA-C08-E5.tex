% --------------------------------------
% Proof: SRF-DEA-C08-E5
% --------------------------------------
\clearpage
\phantomsection
\hypertarget{proof-SRF-DEA-C08-E5}{}

\subsubsection[Infinite Poset Has an Infinite Chain or an Infinite Antichain]{Proof --- SRF-DEA-C08-E5}

\bigskip

\noindent
\textbf{Source.}
\srccite{Dean2015OrderTheory}{Chapter~VIII, Exercise~5}.

\vspace{0.75em}

\noindent
\textbf{Goal.}
Let $(X, \leq)$ be an infinite poset. Prove that $X$ either has an infinite
chain or an infinite totally unordered (antichain) subset $C$.

\vspace{0.75em}

\noindent
\textbf{Logical form.}
$X$ infinite $\Rightarrow$ $\exists$ infinite chain $C \subseteq X$, or
$\exists$ infinite antichain $A \subseteq X$.

\vspace{0.75em}

\noindent
\textbf{Key background.}
This is an instance of the Erdős–Szekeres / Ramsey-type phenomenon: among
any infinite sequence, one must find a monotone infinite subsequence or an
antichain. It follows from Ramsey's theorem for pairs, or can be proved
directly by induction.

\vspace{0.75em}

\noindent
\textbf{Proof strategy.}
Enumerate $X = \{x_0, x_1, x_2, \ldots\}$. Define $x_n$ as ``good'' if
there is an infinite chain starting at $x_n$ in $\{x_n, x_{n+1}, \ldots\}$.

\textit{Case 1:} Infinitely many $x_n$ are good $\Rightarrow$ build an
infinite chain by choosing the next good element each time.

\textit{Case 2:} Only finitely many $x_n$ are good $\Rightarrow$ past a
certain index, no element starts an infinite chain; build an infinite
antichain from these ``bad'' elements.

\vspace{0.75em}

\noindent
\textbf{Proof.}
\begin{proof}
Enumerate $X = \{x_0, x_1, \ldots\}$. Call $x_n$ \emph{good} if the set
$\{x_m : m \geq n,\; x_n \leq x_m\}$ is infinite; otherwise call $x_n$
\emph{bad}.

\textit{Case 1: Infinitely many elements are good.}
Let $x_{n_0}$ be the first good element. Since $x_{n_0}$ is good, infinitely
many elements of $X$ are above $x_{n_0}$; let $x_{n_1}$ be the first good
element strictly above $x_{n_0}$ (good elements above $x_{n_0}$ exist since
the tail is infinite). Continue: $x_{n_0} < x_{n_1} < x_{n_2} < \cdots$
is an infinite chain.

\textit{Case 2: Only finitely many elements are good.}
Past some index $N$, every element is bad. We build an infinite antichain
by induction: take $a_0 = x_N$. Given $a_0, \ldots, a_k$ pairwise
incomparable, since $a_k$ is bad the set of elements above $a_k$ is finite;
thus infinitely many remaining elements are incomparable with $a_k$. Choose
$a_{k+1}$ among these that is also incomparable with $a_0, \ldots, a_{k-1}$
(by a finite intersection argument).
% Detail: formally, this requires a more careful induction; 
% the key is that "bad" elements do not start infinite chains, 
% so we can always find the next antichain element.
\end{proof}

\vspace{0.75em}

\noindent
\textbf{Remarks.}
\begin{itemize}
  \item \textbf{Dilworth's theorem (deeper result):} In a finite poset, the
    minimum number of chains needed to cover $X$ equals the maximum size of
    an antichain. Exercise~E6 uses a version of this.
  \item \textbf{Mistakes to avoid:} ``Infinite'' in the conclusion means
    there is no bound on the chain or antichain size — not just that a chain
    of every finite length exists (though these are equivalent by
    König's lemma / compactness in the finite case).
\end{itemize}
