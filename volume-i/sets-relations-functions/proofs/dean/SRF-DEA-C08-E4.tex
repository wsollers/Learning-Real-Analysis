% --------------------------------------
% Proof: SRF-DEA-C08-E4
% --------------------------------------
\clearpage
\phantomsection
\hypertarget{proof-SRF-DEA-C08-E4}{}

\subsubsection[Every Countable Chain Embeds in $(\mathbb{Q},\leq)$]{Proof --- SRF-DEA-C08-E4}

\bigskip

\noindent
\textbf{Source.}
\srccite{Dean2015OrderTheory}{Chapter~VIII, Exercise~4}.

\vspace{0.75em}

\noindent
\textbf{Goal.}
Prove that any countable chain $(L, \leq)$ is order-isomorphic to a subset
of $(\mathbb{Q}, \leq)$.

\vspace{0.75em}

\noindent
\textbf{Logical form.}
$\forall$ countable chain $(L, \leq)$, $\exists$ order-embedding
$f : L \hookrightarrow \mathbb{Q}$.

\vspace{0.75em}

\noindent
\textbf{Key background.}
Cantor's characterization of $\mathbb{Q}$ as an ordered set: $\mathbb{Q}$
is (up to order-isomorphism) the unique countable, dense linear order without
endpoints.

\vspace{0.75em}

\noindent
\textbf{Proof strategy.}
Enumerate $L = \{l_0, l_1, l_2, \ldots\}$. Define $f$ by induction: at
step $n$, assign $f(l_n) \in \mathbb{Q}$ so that the finite map $f$ on
$\{l_0, \ldots, l_n\}$ is order-preserving. The density of $\mathbb{Q}$
ensures a rational can always be inserted in the correct position.

\vspace{0.75em}

\noindent
\textbf{Proof.}
\begin{proof}
Let $L = \{l_0, l_1, l_2, \ldots\}$ be an enumeration of $L$.

\textit{Base case:} Set $f(l_0) = 0 \in \mathbb{Q}$.

\textit{Inductive step:} Suppose $f$ has been defined on $\{l_0, \ldots,
l_{n-1}\}$ so that $f$ is order-preserving on this finite set. We define
$f(l_n)$.

Let $A = \{f(l_i) : i < n,\; l_i < l_n\}$ and
$B = \{f(l_i) : i < n,\; l_i > l_n\}$.

Since $f$ is order-preserving on $\{l_0, \ldots, l_{n-1}\}$ and $(L,\leq)$
is a chain, every element of $A$ is strictly less than every element of $B$.
Since $\mathbb{Q}$ is dense and has no endpoints, there exists
$q \in \mathbb{Q}$ with $a < q < b$ for all $a \in A$, $b \in B$
(and $q > \max A$ if $B = \varnothing$, and $q < \min B$ if $A =
\varnothing$).

Set $f(l_n) = q$.

By induction, $f : L \to \mathbb{Q}$ is a well-defined order-embedding.
\end{proof}

\vspace{0.75em}

\noindent
\textbf{Remarks.}
\begin{itemize}
  \item \textbf{This is the ``back-and-forth'' method applied to one
    direction.} The full back-and-forth argument establishes the
    isomorphism $\mathbb{Q} \cong \mathbb{Q}$ (Cantor's theorem). Here
    only the ``forth'' direction is needed since we embed into (not onto)
    $\mathbb{Q}$.
  \item \textbf{Connection to Cantor's theorem:} $(\mathbb{Q}, \leq)$ is
    the unique countable dense linear order without endpoints. This
    universality among countable chains is the reason the embedding works.
\end{itemize}
