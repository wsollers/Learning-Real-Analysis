% --------------------------------------
% Proof: SRF-DEA-C08-E14
% --------------------------------------
\clearpage
\phantomsection
\hypertarget{proof-SRF-DEA-C08-E14}{}

\subsubsection[Characterizations of Well-Ordering]{Proof --- SRF-DEA-C08-E14}

\bigskip

\noindent
\textbf{Source.}
\srccite{Dean2015OrderTheory}{Chapter~VIII, Exercise~14}.

\vspace{0.75em}

\noindent
\textbf{Key definition.}
Let $\omega_0^*$ denote the order type of $(\mathbb{Z}_{\leq 0}, \leq)$,
the nonpositive integers with their usual order. A chain has order type
$\omega_0^*$ iff it is order-isomorphic to $\{\ldots, -3, -2, -1, 0\}$ —
an infinite descending sequence.

\vspace{0.75em}

\noindent
\textbf{Goal.}
\begin{enumerate}
  \item[(a)] A chain $(X, \leq)$ is well-ordered if and only if $X$ contains
    no subset of order type $\omega_0^*$.
  \item[(b)] If $(X, \leq)$ is a chain in which every countable subset is
    well-ordered, then $X$ is well-ordered.
  \item[(c)] Every infinite chain either has a subset of order type $\omega_0$
    or one of order type $\omega_0^*$.
\end{enumerate}

\vspace{0.75em}

\noindent
\textbf{Proof.}
\begin{proof}
\textbf{(a) ($\Rightarrow$)} If $X$ is well-ordered and contained a subset
$S$ of order type $\omega_0^*$, then $S$ would be a nonempty subset with no
least element (the descending chain never terminates). But well-ordering
requires every nonempty subset to have a least element. Contradiction.

$(\Leftarrow)$ Suppose $X$ has no subset of order type $\omega_0^*$, but
$X$ is not well-ordered. Then there exists a nonempty $S \subseteq X$ with
no least element. Starting with any $s_0 \in S$, since $s_0$ is not least,
there exists $s_1 \in S$ with $s_1 < s_0$. Continuing, build a strictly
decreasing sequence $s_0 > s_1 > s_2 > \cdots$, which is a subset of order
type $\omega_0^*$. Contradiction.

\textbf{(b)} Suppose $(X, \leq)$ is a chain in which every countable subset
is well-ordered. Let $S \subseteq X$ be nonempty. By (a), it suffices to
show $S$ contains no subset of order type $\omega_0^*$. Any such subset
would be countable (it is countably infinite), hence well-ordered by
hypothesis — but a subset of order type $\omega_0^*$ is not well-ordered.
Contradiction.

\textbf{(c)} Let $X$ be an infinite chain. By (a), either $X$ has a
descending sequence (order type $\omega_0^*$) or $X$ is well-ordered.
If $X$ is well-ordered and infinite, it contains a cofinal copy of
$\omega_0$: its least element $x_0$, then the least element of
$X \setminus \{x_0\}$, and so on, yielding an ascending sequence of order
type $\omega_0$.
\end{proof}

\vspace{0.75em}

\noindent
\textbf{Remarks.}
\begin{itemize}
  \item \textbf{Part (a)} gives an elegant negative characterization of
    well-ordering: the absence of infinite descending chains.
  \item \textbf{Part (c)} is the order-theoretic analogue of the Bolzano-
    Weierstrass flavor: every infinite sequence in a chain contains a
    monotone infinite subsequence.
\end{itemize}
