% --------------------------------------
% Proof: SRF-DEA-C08-E16
% --------------------------------------
\clearpage
\phantomsection
\hypertarget{proof-SRF-DEA-C08-E16}{}

\subsubsection[Unique Inductive Subset Implies Well-Ordering]{Proof --- SRF-DEA-C08-E16}

\bigskip

\noindent
\textbf{Source.}
\srccite{Dean2015OrderTheory}{Chapter~VIII, Exercise~16}.

\vspace{0.75em}

\noindent
\textbf{Key definition.}
Let $E$ be a chain. A subset $F \subseteq E$ is \emph{inductive} if
\[
  \forall\, t \in E,\quad \{a \in E : a < t\} \subseteq F
  \;\Rightarrow\; t \in F.
\]
(Every element whose proper predecessor set lies entirely in $F$ must itself
be in $F$.)

\vspace{0.75em}

\noindent
\textbf{Goal.}
Prove that if $E$ is the only inductive subset of $E$, then $E$ is
well-ordered.

\vspace{0.75em}

\noindent
\textbf{Logical form.}
$(E \text{ is the only inductive subset of } E)
\Rightarrow (E, \leq) \text{ is well-ordered}$.

\vspace{0.75em}

\noindent
\textbf{Proof strategy.}
Contrapositive. If $E$ is not well-ordered, exhibit a proper inductive subset
$F \subsetneq E$ (namely, the complement of a nonempty subset without a least
element).

\vspace{0.75em}

\noindent
\textbf{Proof.}
\begin{proof}
Suppose $E$ is not well-ordered. Then there exists a nonempty $S \subseteq E$
with no least element.

Let $F = E \setminus S$. We show $F$ is inductive.

Let $t \in E$ and suppose $\{a \in E : a < t\} \subseteq F$, i.e., every
element strictly below $t$ is in $F = E \setminus S$, hence not in $S$.

If $t \in S$, then since $S$ has no least element, there exists $s \in S$
with $s < t$. But then $s < t$ and $s \in S$, so $s \notin F$,
contradicting $\{a : a < t\} \subseteq F$. Therefore $t \notin S$,
i.e., $t \in F$.

Hence $F$ is inductive. But $F \neq E$ since $S \neq \varnothing$. So $E$
has a proper inductive subset, contradicting the hypothesis.

Contrapositive: if $E$ is the only inductive subset, then $E$ is
well-ordered.
\end{proof}

\vspace{0.75em}

\noindent
\textbf{Remarks.}
\begin{itemize}
  \item \textbf{This is transfinite induction in disguise:} The inductive
    condition on $F$ says exactly that $F$ is closed under the transfinite
    induction step. The theorem states that transfinite induction along $E$
    works (i.e., every inductively defined set is all of $E$) iff $E$ is
    well-ordered. Compare with SRF-STO-C07-S7.4.
  \item \textbf{Converse direction:} If $E$ is well-ordered, then $E$ is
    the only inductive subset. Proof: let $F$ be inductive. If $E \setminus F
    \neq \varnothing$, it has a least element $t$. All $a < t$ lie in $F$
    (by minimality of $t$), so $t \in F$ by inductivity — contradiction.
\end{itemize}
