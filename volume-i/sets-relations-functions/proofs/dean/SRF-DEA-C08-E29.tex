% --------------------------------------
% Proof: SRF-DEA-C08-E29
% --------------------------------------
\clearpage
\phantomsection
\hypertarget{proof-SRF-DEA-C08-E29}{}

\subsubsection[Every Partial Order Extends to a Linear Order]{Proof --- SRF-DEA-C08-E29}

\bigskip

\noindent
\textbf{Source.}
\srccite{Dean2015OrderTheory}{Chapter~VIII, Exercise~29}.

\vspace{0.75em}

\noindent
\textbf{Goal.}
Suppose $(T, \leq)$ is a poset. Prove that $\leq$ can be ``enlarged'' to a
relation $\leq^*$ such that $(T, \leq^*)$ is a chain, where
``enlarges'' means ${\leq} \subseteq {\leq^*} \subseteq T \times T$.

\vspace{0.75em}

\noindent
\textbf{Note.}
This is Szpilrajn's Theorem (see \hyperref[thm:szpilrajn]{notes-order-extensions})
in the form: every partial order extends to a linear order.

\vspace{0.75em}

\noindent
\textbf{Proof strategy.}
Apply Zorn's Lemma to the poset $\mathcal{E}$ of all partial orders on $T$
that extend $\leq$, ordered by $\subseteq$. Show every chain in $\mathcal{E}$
has an upper bound; conclude a maximal element $\leq^*$ exists; show $\leq^*$
must be total.

\vspace{0.75em}

\noindent
\textbf{Proof.}
\begin{proof}
Let $\mathcal{E} = \{\, R \subseteq T \times T : R \text{ is a partial order
on } T \text{ extending } \leq\,\}$, ordered by inclusion.

\textit{Chains have upper bounds.} Let $\mathcal{C} \subseteq \mathcal{E}$
be a chain. The union $R_0 = \bigcup \mathcal{C}$ is a relation extending
$\leq$. We verify $R_0$ is a partial order:
\begin{itemize}
  \item \emph{Reflexivity:} Each $R \in \mathcal{C}$ contains $(x,x)$, so
    $R_0$ does.
  \item \emph{Antisymmetry:} If $xR_0 y$ and $y R_0 x$, then $xRy$ for some
    $R \in \mathcal{C}$ and $ySx$ for some $S \in \mathcal{C}$. Since
    $\mathcal{C}$ is a chain, say $R \subseteq S$; then $xSy$ and $ySx$, so
    $x = y$ by antisymmetry of $S$.
  \item \emph{Transitivity:} If $xR_0 y$ and $y R_0 z$, choose $R, S \in
    \mathcal{C}$ with $xRy$, $ySz$. WLOG $R \subseteq S$; then $xSy$ and
    $ySz$, so $xSz$ by transitivity of $S$, giving $x R_0 z$.
\end{itemize}
So $R_0 \in \mathcal{E}$ is an upper bound for $\mathcal{C}$.

\textit{Maximal element.} By Zorn's Lemma, $\mathcal{E}$ has a maximal
element $\leq^*$.

\textit{$\leq^*$ is total.} Suppose $x, y \in T$ are incomparable under
$\leq^*$. Consider the relation $\leq^{**}$ defined as the transitive closure
of $\leq^* \cup \{(x,y)\}$. One checks $\leq^{**}$ is a partial order
extending $\leq^*$ (and hence $\leq$), with $\leq^* \subsetneq \leq^{**}$,
contradicting maximality of $\leq^*$.

Therefore $\leq^*$ is a linear order extending $\leq$.
\end{proof}

\vspace{0.75em}

\noindent
\textbf{Remarks.}
\begin{itemize}
  \item \textbf{Hint resolution:} The hint asks: if $\leq$ is already a
    linear order, can it be enlarged? No — a linear order on $T$ is already
    maximal in $\mathcal{E}$ (nothing can be added while keeping antisymmetry
    and transitivity). This confirms the maximal elements of $\mathcal{E}$
    are exactly the linear orders extending $\leq$.
\end{itemize}
