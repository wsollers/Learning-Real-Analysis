% --------------------------------------
% Proof: SRF-DEA-C08-E1
% --------------------------------------
\clearpage
\phantomsection
\hypertarget{proof-SRF-DEA-C08-E1}{}

\subsubsection[Symmetric and Antisymmetric Implies Equality]{Proof --- SRF-DEA-C08-E1}

\bigskip

\noindent
\textbf{Source.}
\srccite{Dean2015OrderTheory}{Chapter~VIII, Exercise~1}.

\vspace{0.75em}

\noindent
\textbf{Goal.}
Prove or disprove: if $V$ is a relation on $X$ that is both symmetric and
antisymmetric, then $V$ must be the equality relation $=$; that is,
$V = \{(x,x) : x \in X\}$.

\vspace{0.75em}

\noindent
\textbf{Logical form.}
$V$ symmetric $\land$ $V$ antisymmetric $\Rightarrow$
$\forall x, y \in X:\; xVy \iff x = y$.

\vspace{0.75em}

\noindent
\textbf{Proof strategy.}
Direct proof. Symmetry: $xVy \Rightarrow yVx$. Antisymmetry: $xVy \land yVx
\Rightarrow x = y$. Composing gives $xVy \Rightarrow x = y$. The converse
($x = y \Rightarrow xVx$) follows if $V$ is also reflexive — but note that
neither symmetry nor antisymmetry alone implies reflexivity. So the claim
needs care: the conclusion is that every pair satisfying $V$ satisfies
equality, not that $V$ contains all diagonal pairs.

\vspace{0.75em}

\noindent
\textbf{Proof.}
\begin{proof}
We prove that $V \subseteq \{(x,x) : x \in X\}$.

Let $xVy$. By symmetry of $V$, $yVx$. By antisymmetry of $V$, $x = y$.
Therefore $(x,y) = (x,x) \in \{(x,x) : x \in X\}$.

Hence $V \subseteq \{(x,x) : x \in X\}$.

% Remark: the reverse inclusion fails without reflexivity. E.g. V = empty
% relation is both symmetric and antisymmetric but is not the full equality
% relation (which contains all (x,x)).
\end{proof}

\vspace{0.75em}

\noindent
\textbf{Remarks.}
\begin{itemize}
  \item \textbf{Correction to naive statement:} The claim ``$V$ is the
    equality relation'' is slightly imprecise. The proof shows $V \subseteq
    \{(x,x)\}$, but $V$ could be a proper subset of the diagonal (e.g., the
    empty relation). The equality relation is the case when $V$ is also
    reflexive.
  \item \textbf{Counterexample flavor:} The empty relation $V = \varnothing$
    is both symmetric and antisymmetric but is strictly contained in $=$.
    So ``$V$ must be the equality relation'' is \emph{false} as literally
    stated; the proof establishes the weaker (and correct) conclusion
    $V \subseteq \mathord{=}$.
\end{itemize}
