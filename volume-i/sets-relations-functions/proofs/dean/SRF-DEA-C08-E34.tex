% --------------------------------------
% Proof: SRF-DEA-C08-E34
% --------------------------------------
\clearpage
\phantomsection
\hypertarget{proof-SRF-DEA-C08-E34}{}

\subsubsection[A Subset of $\mathbb{R}$ with Countably Many Translates]{Proof --- SRF-DEA-C08-E34}

\bigskip

\noindent
\textbf{Source.}
\srccite{Dean2015OrderTheory}{Chapter~VIII, Exercise~34}.

\vspace{0.75em}

\noindent
\textbf{Goal.}
Prove that there exists a subset $E \subseteq \mathbb{R}$ having only
countably many distinct translates, i.e., the set
$\{E + r : r \in \mathbb{R}\}$ is countable, where $E + r = \{a + r : a
\in E\}$.

\vspace{0.75em}

\noindent
\textbf{Key background.}
Treat $\mathbb{R}$ as a vector space over $\mathbb{Q}$. By Zorn's Lemma
(or equivalently AC), it has a Hamel basis $F$: a set $F \subseteq \mathbb{R}$
such that every $x \in \mathbb{R}$ is a unique finite $\mathbb{Q}$-linear
combination of elements of $F$.

\vspace{0.75em}

\noindent
\textbf{Proof strategy.}
Fix a basis element $b \in F$. Let $E$ consist of all reals whose Hamel
representation does not involve $b$. Show $E + r$ depends only on the
coefficient of $b$ in the Hamel expansion of $r$, and there are only countably
many such coefficients (members of $\mathbb{Q}$... but rationals are countable).

\vspace{0.75em}

\noindent
\textbf{Proof.}
\begin{proof}
By AC, $\mathbb{R}$ has a Hamel basis $F$ over $\mathbb{Q}$. Fix $b \in F$.
Define
\[
  E = \bigl\{ x \in \mathbb{R} : \text{the coefficient of } b \text{ in the
  Hamel expansion of } x \text{ is } 0 \bigr\}.
\]
$E$ is a $\mathbb{Q}$-subspace of $\mathbb{R}$ spanned by $F \setminus
\{b\}$.

For any $r \in \mathbb{R}$, write $r = qb + e$ where $q \in \mathbb{Q}$ is
the coefficient of $b$ in the Hamel expansion of $r$, and $e \in E$. Then
\[
  E + r = E + qb + e = E + qb
\]
since $E$ is a subspace and $e \in E$ so $E + e = E$.

Therefore, $E + r = E + qb$, which depends only on $q \in \mathbb{Q}$.
Since $\mathbb{Q}$ is countable, there are only countably many distinct
translates $E + r$.
\end{proof}

\vspace{0.75em}

\noindent
\textbf{Remarks.}
\begin{itemize}
  \item \textbf{Hamel basis:} The existence of a Hamel basis for $\mathbb{R}$
    over $\mathbb{Q}$ requires the Axiom of Choice; no explicit construction
    is known.
  \item \textbf{Connection to measure theory:} The set $E$ constructed here
    is a Vitali-type set: its translates by distinct rationals are disjoint.
    This is the classic construction of a non-Lebesgue-measurable set.
    The fact that $E$ has only countably many translates (one per rational
    multiple of $b$) is the key structural property.
\end{itemize}
