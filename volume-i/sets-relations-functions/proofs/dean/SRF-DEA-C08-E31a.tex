% --------------------------------------
% Proof: SRF-DEA-C08-E31a
% --------------------------------------
\clearpage
\phantomsection
\hypertarget{proof-SRF-DEA-C08-E31a}{}

\subsubsection[Maximal Almost-Disjoint Family of Infinite Subsets of $\mathbb{N}$]{Proof --- SRF-DEA-C08-E31a}

\bigskip

\noindent
\textbf{Source.}
\srccite{Dean2015OrderTheory}{Chapter~VIII, Exercise~31(a)}.

\vspace{0.75em}

\noindent
\textbf{Goal.}
Prove that there exists an infinite maximal family $\mathcal{T}$ of infinite
subsets of $\mathbb{N}$ such that the intersection of any two sets from
$\mathcal{T}$ is finite.

\vspace{0.75em}

\noindent
\textbf{Key definition.}
A family $\mathcal{T}$ of infinite subsets of $\mathbb{N}$ is
\emph{almost disjoint} if $|A \cap B| < \infty$ for all distinct $A, B \in
\mathcal{T}$. It is \emph{maximal almost disjoint} (MAD) if no infinite
subset of $\mathbb{N}$ can be added to $\mathcal{T}$ while preserving the
almost disjoint property.

\vspace{0.75em}

\noindent
\textbf{Proof strategy.}
Apply Zorn's Lemma to the poset of almost-disjoint families ordered by
inclusion. Show chains have upper bounds; a maximal element is a MAD family.
To ensure it is infinite, start with an infinite almost-disjoint family
(e.g., the rows of $\mathbb{N} \times \mathbb{N}$).

\vspace{0.75em}

\noindent
\textbf{Proof.}
\begin{proof}
\textit{Initial family.} For each $n \in \mathbb{N}$, let $A_n = \{n\}
\times \mathbb{N}$ (the $n$-th row in $\mathbb{N} \times \mathbb{N} \cong
\mathbb{N}$). Each $A_n$ is infinite, and $A_n \cap A_m = \varnothing$ for
$n \neq m$ (so in particular $|A_n \cap A_m| < \infty$). So
$\mathcal{F}_0 = \{A_n : n \in \mathbb{N}\}$ is an infinite almost-disjoint
family.

\textit{Zorn's Lemma.} Let $\mathcal{P}$ be the poset of all almost-disjoint
families $\mathcal{G}$ of infinite subsets of $\mathbb{N}$ with
$\mathcal{G} \supseteq \mathcal{F}_0$, ordered by inclusion.

Every chain $\mathcal{C}$ in $\mathcal{P}$ has an upper bound
$\mathcal{G}_0 = \bigcup \mathcal{C}$: since pairwise almost-disjointness
is preserved by unions of chains (any two sets $A, B \in \mathcal{G}_0$
belong to some common $\mathcal{G} \in \mathcal{C}$), $\mathcal{G}_0$ is
almost disjoint.

By Zorn's Lemma, $\mathcal{P}$ has a maximal element $\mathcal{T}$.

\textit{$\mathcal{T}$ is infinite.} It contains $\mathcal{F}_0$, which is
infinite.

\textit{$\mathcal{T}$ is maximal.} No infinite subset $B \subseteq \mathbb{N}$
can be added: if $B$ had finite intersection with every set in $\mathcal{T}$,
then $\mathcal{T} \cup \{B\}$ would be a larger almost-disjoint family,
contradicting maximality.
\end{proof}

\vspace{0.75em}

\noindent
\textbf{Remarks.}
\begin{itemize}
  \item MAD families are objects of significant interest in set-theoretic
    topology and combinatorics. The minimum cardinality of a MAD family
    is the cardinal $\mathfrak{a}$, one of the classical combinatorial
    cardinal invariants of the continuum.
\end{itemize}
