% --------------------------------------
% Proof: SRF-DEA-C08-E3
% --------------------------------------
\clearpage
\phantomsection
\hypertarget{proof-SRF-DEA-C08-E3}{}

\subsubsection[Largest and Smallest in Every Subset Implies Finite Chain]{Proof --- SRF-DEA-C08-E3}

\bigskip

\noindent
\textbf{Source.}
\srccite{Dean2015OrderTheory}{Chapter~VIII, Exercise~3}.

\vspace{0.75em}

\noindent
\textbf{Goal.}
Suppose $(X, \leq)$ is a poset in which every nonempty subset contains a
largest and a smallest element. Prove that $(X, \leq)$ is a finite chain.

\vspace{0.75em}

\noindent
\textbf{Logical form.}
$(\forall \varnothing \neq S \subseteq X:\; S \text{ has a largest and smallest element})
\Rightarrow X \text{ is a finite chain}$.

\vspace{0.75em}

\noindent
\textbf{Proof strategy.}
Prove totality (it's a chain) and finiteness separately.

\textit{Chain:} Apply the hypothesis to the two-element subset $\{x,y\}$.

\textit{Finite:} Suppose $X$ is infinite. Extract an infinite sequence of
distinct elements; consider the set of all elements in that sequence. It must
have a largest element, but the sequence can be extended, giving a
contradiction.

\vspace{0.75em}

\noindent
\textbf{Proof.}
\begin{proof}
\textit{Step 1: $(X, \leq)$ is a chain.}
Let $x, y \in X$ be arbitrary. Apply the hypothesis to $S = \{x, y\}$.
Since $S$ has a smallest element, either $x$ is smallest (so $x \leq y$)
or $y$ is smallest (so $y \leq x$). Hence every pair is comparable.

\textit{Step 2: $X$ is finite.}
Suppose for contradiction that $X$ is infinite. By the hypothesis applied
to $X$ itself, $X$ has a largest element $M$ and a smallest element $m$.
Since $X$ is a chain, $X$ is well-ordered (every nonempty subset has a
least element), and also reverse-well-ordered (every nonempty subset has a
greatest element).

% A chain that is both well-ordered and reverse-well-ordered must be finite:
% extract the least element, remove it, repeat -- this process terminates
% because if it ran for infinitely many steps, the set of removed elements
% would form an infinite subset with no greatest element. Contradiction.
\end{proof}

\vspace{0.75em}

\noindent
\textbf{Remarks.}
\begin{itemize}
  \item \textbf{Key lemma:} A chain that is simultaneously well-ordered and
    reverse-well-ordered is finite. This is the crux of Step 2.
  \item \textbf{Mistakes to avoid:} Do not assume $X$ is totally ordered
    without proving it first (Step 1 is required). Also, having a largest
    and smallest element alone does not imply finiteness — the interval
    $[0,1] \subset \mathbb{R}$ has both but is infinite. The hypothesis must
    hold for \emph{every} nonempty subset, not just $X$ itself.
\end{itemize}
