% --------------------------------------
% Proof: SRF-DEA-C08-E15
% --------------------------------------
\clearpage
\phantomsection
\hypertarget{proof-SRF-DEA-C08-E15}{}

\subsubsection[Basic Ordinal Arithmetic]{Proof --- SRF-DEA-C08-E15}

\bigskip

\noindent
\textbf{Source.}
\srccite{Dean2015OrderTheory}{Chapter~VIII, Exercise~15}.

\vspace{0.75em}

\noindent
\textbf{Goal.}
Prove the following facts about ordinal numbers $\alpha, \beta, \gamma$:
\begin{enumerate}
  \item[(a)] $\alpha > 0$ implies $\beta < \alpha + \beta$ for all $\beta > 0$.
  \item[(b)] If $\alpha < \beta$, then there exists a unique $\gamma$ such
    that $\alpha + \gamma = \beta$.
  \item[(c)] $\beta = \omega_0 + \alpha$ if and only if $\beta \geq \omega_0$
    and $\alpha < \omega_0$.
\end{enumerate}

\vspace{0.75em}

\noindent
\textbf{Key background.}
Ordinal addition: $\alpha + \beta$ is the ordinal corresponding to placing a
copy of $\beta$ after a copy of $\alpha$. It is \emph{not} commutative:
$1 + \omega_0 = \omega_0 \neq \omega_0 + 1$.

\vspace{0.75em}

\noindent
\textbf{Proof.}
\begin{proof}
\textbf{(a)} The ordinal $\alpha + \beta$ is the order type of $\alpha$
followed by $\beta$. The copy of $\beta$ begins strictly after all elements
of the $\alpha$-copy. Hence every element of the $\beta$-copy appears at a
position $\geq \alpha > 0$, so the initial $\alpha$-copy has order type
$\alpha$ as a proper initial segment of $\alpha + \beta$, giving
$\alpha \leq \alpha + \beta$; and since $\beta > 0$, the $\beta$-copy
contributes at least one additional element, so $\beta < \alpha + \beta$.

\textbf{(b)} Since $\alpha < \beta$, $\alpha$ is an initial segment of
$\beta$. The ordinal $\gamma$ is the order type of $\beta \setminus \alpha$
(the remainder after cutting off the initial $\alpha$-segment), giving
$\alpha + \gamma = \beta$. Uniqueness: if $\alpha + \gamma = \alpha + \gamma'
= \beta$, then both $\gamma$ and $\gamma'$ are order-isomorphic to
$\beta \setminus \alpha$, so $\gamma = \gamma'$.

\textbf{(c)} ($\Rightarrow$) If $\beta = \omega_0 + \alpha$, then
$\beta \geq \omega_0$ (by part (a) with the initial $\omega_0$-segment) and
$\alpha$ is the order type of the tail, which is the remainder after
$\omega_0$. Since $\omega_0 + \alpha$ lists $\omega_0$ first, the tail
$\alpha$ must be a finite ordinal; otherwise $\beta$ would contain $\omega_0$
as a non-initial proper segment, contradicting the representation.
% ($\Leftarrow$) If beta >= omega_0 and alpha < omega_0, then cut the initial
% omega_0-segment of beta; the remainder has order type some alpha' < omega_0
% (by the hypothesis that the total is at most omega_0 + finite).
\end{proof}

\vspace{0.75em}

\noindent
\textbf{Remarks.}
\begin{itemize}
  \item \textbf{Non-commutativity:} $1 + \omega_0 = \omega_0$ but $\omega_0
    + 1 > \omega_0$. Ordinal addition is only left-cancellable, not
    right-cancellable.
  \item \textbf{Part (c)} characterizes the ordinals in the ``block''
    $[\omega_0, \omega_0 \cdot 2)$ as exactly those of the form
    $\omega_0 + n$ for finite $n$.
\end{itemize}
