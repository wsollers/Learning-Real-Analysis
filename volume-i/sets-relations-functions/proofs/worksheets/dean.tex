\subsection{Dean}

\noindent\textbf{Source.} \cite{Dean2015OrderTheory}

\vspace{0.75em}
\begin{center}
\begin{tabular}{|p{5.0cm}|p{9.0cm}|}
\hline
\textbf{Problem ID} & \textbf{Exercise (descriptor)} \\
\hline
\phantomsection
\hypertarget{ws-SRF-DEA-C08-E1}{}
\hyperlink{proof-SRF-DEA-C08-E1}{\texttt{SRF-DEA-C08-E1}}
&
Ch.~VIII, Ex.~1 --- Symmetric and antisymmetric implies $V \subseteq {=}$.
\\ \hline
\phantomsection
\hypertarget{ws-SRF-DEA-C08-E2}{}
\hyperlink{proof-SRF-DEA-C08-E2}{\texttt{SRF-DEA-C08-E2}}
&
Ch.~VIII, Ex.~2 --- $\mathcal{P}(X)$ ordered by inclusion is a chain iff $|X| \leq 1$.
\\ \hline
\phantomsection
\hypertarget{ws-SRF-DEA-C08-E3}{}
\hyperlink{proof-SRF-DEA-C08-E3}{\texttt{SRF-DEA-C08-E3}}
&
Ch.~VIII, Ex.~3 --- Every nonempty subset has max and min implies finite chain.
\\ \hline
\phantomsection
\hypertarget{ws-SRF-DEA-C08-E4}{}
\hyperlink{proof-SRF-DEA-C08-E4}{\texttt{SRF-DEA-C08-E4}}
&
Ch.~VIII, Ex.~4 --- Every countable chain embeds order-isomorphically in $(\mathbb{Q},\leq)$.
\\ \hline
\phantomsection
\hypertarget{ws-SRF-DEA-C08-E5}{}
\hyperlink{proof-SRF-DEA-C08-E5}{\texttt{SRF-DEA-C08-E5}}
&
Ch.~VIII, Ex.~5 --- Infinite poset has an infinite chain or an infinite antichain.
\\ \hline
\phantomsection
\hypertarget{ws-SRF-DEA-C08-E6}{}
\hyperlink{proof-SRF-DEA-C08-E6}{\texttt{SRF-DEA-C08-E6}}
&
Ch.~VIII, Ex.~6 --- Poset with longest chain $n$ is union of $n$ antichains (Mirsky).
\\ \hline
\phantomsection
\hypertarget{ws-SRF-DEA-C08-E13}{}
\hyperlink{proof-SRF-DEA-C08-E13}{\texttt{SRF-DEA-C08-E13}}
&
Ch.~VIII, Ex.~13 --- A well-ordered subset of $\mathbb{R}$ is countable.
\\ \hline
\phantomsection
\hypertarget{ws-SRF-DEA-C08-E14}{}
\hyperlink{proof-SRF-DEA-C08-E14}{\texttt{SRF-DEA-C08-E14}}
&
Ch.~VIII, Ex.~14 --- Characterizations of well-ordering via $\omega_0^*$ and countable subsets.
\\ \hline
\phantomsection
\hypertarget{ws-SRF-DEA-C08-E15}{}
\hyperlink{proof-SRF-DEA-C08-E15}{\texttt{SRF-DEA-C08-E15}}
&
Ch.~VIII, Ex.~15 --- Basic ordinal arithmetic: $\beta < \alpha+\beta$, subtraction, $\omega_0$-blocks.
\\ \hline
\phantomsection
\hypertarget{ws-SRF-DEA-C08-E16}{}
\hyperlink{proof-SRF-DEA-C08-E16}{\texttt{SRF-DEA-C08-E16}}
&
Ch.~VIII, Ex.~16 --- Unique inductive subset implies well-ordering.
\\ \hline
\phantomsection
\hypertarget{ws-SRF-DEA-C08-E28}{}
\hyperlink{proof-SRF-DEA-C08-E28}{\texttt{SRF-DEA-C08-E28}}
&
Ch.~VIII, Ex.~28 --- Maximal vs.\ largest: why Ray's Zorn argument fails.
\\ \hline
\phantomsection
\hypertarget{ws-SRF-DEA-C08-E29}{}
\hyperlink{proof-SRF-DEA-C08-E29}{\texttt{SRF-DEA-C08-E29}}
&
Ch.~VIII, Ex.~29 --- Every partial order extends to a linear order (Szpilrajn via Zorn).
\\ \hline
\phantomsection
\hypertarget{ws-SRF-DEA-C08-E31a}{}
\hyperlink{proof-SRF-DEA-C08-E31a}{\texttt{SRF-DEA-C08-E31a}}
&
Ch.~VIII, Ex.~31(a) --- Existence of a maximal almost-disjoint family (MAD family).
\\ \hline
\phantomsection
\hypertarget{ws-SRF-DEA-C08-E34}{}
\hyperlink{proof-SRF-DEA-C08-E34}{\texttt{SRF-DEA-C08-E34}}
&
Ch.~VIII, Ex.~34 --- Subset of $\mathbb{R}$ with only countably many translates (Hamel basis).
\\ \hline
\end{tabular}
\end{center}
