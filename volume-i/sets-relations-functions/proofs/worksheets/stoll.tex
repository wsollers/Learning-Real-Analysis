\subsection{Stoll}

\noindent\textbf{Source.} \cite{StollSetTheoryLogic}

\vspace{0.75em}
\begin{center}
\begin{tabular}{|p{5.0cm}|p{9.0cm}|}
\hline
\textbf{Problem ID} & \textbf{Exercise (descriptor)} \\
\hline
\phantomsection
\hypertarget{ws-SRF-STO-CH01-1.1}{}
\hyperlink{proof-SRF-STO-CH01-1.1}{\texttt{SRF-STO-CH01-1.1}}
&
Stoll — Sets/relations/functions (placeholder).
\\ \hline
\phantomsection
\hypertarget{ws-SRF-STO-C06-S6.5}{}
\hyperlink{proof-SRF-STO-C06-S6.5}{\texttt{SRF-STO-C06-S6.5}}
&
Ch.~2, \S6, Ex.~6.5 --- Segments of a chain are comparable (\S6).
\\ \hline
\phantomsection
\hypertarget{ws-SRF-STO-C06-S6.6}{}
\hyperlink{proof-SRF-STO-C06-S6.6}{\texttt{SRF-STO-C06-S6.6}}
&
Ch.~2, \S6, Ex.~6.6 --- Chains of order type $\omega$ characterization.
\\ \hline
\phantomsection
\hypertarget{ws-SRF-STO-C07-S7.1}{}
\hyperlink{proof-SRF-STO-C07-S7.1}{\texttt{SRF-STO-C07-S7.1}}
&
Ch.~2, \S7, Ex.~7.1 --- Antisymmetry and minimum element imply well-ordering.
\\ \hline
\phantomsection
\hypertarget{ws-SRF-STO-C07-S7.2}{}
\hyperlink{proof-SRF-STO-C07-S7.2}{\texttt{SRF-STO-C07-S7.2}}
&
Ch.~2, \S7, Ex.~7.2 --- Ordinal similarity preserves well-ordering.
\\ \hline
\phantomsection
\hypertarget{ws-SRF-STO-C07-S7.4}{}
\hyperlink{proof-SRF-STO-C07-S7.4}{\texttt{SRF-STO-C07-S7.4}}
&
Ch.~2, \S7, Ex.~7.4 --- Principle of transfinite induction.
\\ \hline
\phantomsection
\hypertarget{ws-SRF-STO-C08-S8.1}{}
\hyperlink{proof-SRF-STO-C08-S8.1}{\texttt{SRF-STO-C08-S8.1}}
&
Ch.~2, \S8, Ex.~8.1 --- Finite axiom of choice by induction.
\\ \hline
\end{tabular}
\end{center}
