\clearpage
\phantomsection
\hypertarget{proof-SRF-GER-C02-E16}{}

\section*{Proof — SRF-GER-C02-E16}

\noindent
\textbf{Source.} Gerstein \cite{GersteinMathStructures}, Chapter 2, Exercise 2.16.

\medskip
\noindent
\hyperlink{ws-SRF-GER-C02-E16}{\textbf{← Back to worksheet}}

\medskip

\noindent
\textbf{Assignment (verbatim).}
An ordering $R$ on a set $S$ (see Exercise 15) is a \emph{linear ordering} (or \emph{total ordering})
if every two elements of $S$ are comparable; that is, for $a,b \in S$ either $aRb$ or $bRa$.
\begin{enumerate}[label=(\alph*), leftmargin=*]
  \item Show that neither ordering in Exercise 15 is a total ordering. (Assume that $A$ has at least two elements.)
  \item Let $W$ be the set of English words. To use a dictionary one must know the alphabetical (or lexicographic)
  ordering of $W$, which we denote here by $L$. Define $L$ precisely. (That is, carefully state the conditions
  under which one word precedes another in the dictionary.)
\end{enumerate}

\medskip
\noindent
\textbf{Proof.}
% Write proof here.

\qed

\medskip
\noindent
\textbf{Remarks.}
\begin{itemize}
  \item
\end{itemize}
