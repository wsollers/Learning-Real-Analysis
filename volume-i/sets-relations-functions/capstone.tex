% ======================================================
% Capstone Assessment — Sets, Relations, and Functions
% ======================================================

\section{Capstone Assessment: Sets, Relations, and Functions}

\noindent
\textbf{Purpose.}
This capstone assesses mastery of elementary set theory,
relations, and functions as used in analysis and abstract mathematics.
All proofs must be written using precise definitions and logical reasoning.

\vspace{1em}

\noindent
\textbf{Instructions.}
Each problem requires a complete proof.
You must explicitly invoke definitions (e.g.\ function, equivalence relation)
when they are used.
No appeal to diagrams or intuition is permitted.

\vspace{1.5em}

% --------------------------------------------------
\subsection*{Problem 1 — Equivalence Relations}

Let $\sim$ be a relation on a set $A$.
Prove that $\sim$ is an equivalence relation if and only if
its equivalence classes form a partition of $A$.

Your proof must establish both directions.

\vspace{1.5em}

% --------------------------------------------------
\subsection*{Problem 2 — Images and Preimages}

Let $f : A \to B$ be a function and let $S \subseteq A$.
Prove that
\[
S \subseteq f^{-1}(f(S)).
\]

Give an example where equality does not hold.

\vspace{1.5em}

% --------------------------------------------------
\subsection*{Problem 3 — Injectivity and Left Inverses}

Prove that a function $f : A \to B$ is injective
if and only if there exists a function $g : B \to A$
such that
\[
g \circ f = \mathrm{id}_A.
\]

Your proof must clearly indicate where injectivity is used.

\vspace{1.5em}

% --------------------------------------------------
\subsection*{Problem 4 — Composition of Relations}

Let $R \subseteq A \times B$ and $S \subseteq B \times C$
be relations.
Prove that if both $R$ and $S$ are transitive relations
(on their respective domains), then their composition
need not be transitive.

Your proof must include a concrete counterexample.

\vspace{1.5em}

% --------------------------------------------------
\subsection*{Problem 5 — Order Relations}

Let $(A,\le)$ be a partially ordered set.
Prove that $\le$ is antisymmetric if and only if
\[
(x \le y \;\land\; y \le x) \;\rightarrow\; x = y
\]
holds for all $x,y \in A$.

Your proof must explicitly use the definition of antisymmetry.

\vspace{1em}

\noindent
\textbf{Completion Criterion.}
You have mastered sets, relations, and functions if all five proofs:
\begin{itemize}
  \item correctly invoke definitions,
  \item handle element-wise reasoning rigorously,
  \item distinguish relations from functions,
  \item and avoid implicit assumptions.
\end{itemize}

\noindent
Successful completion certifies readiness to proceed to
foundations of the real line and completeness.
