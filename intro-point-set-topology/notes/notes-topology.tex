% =========================================================
% Topology of a Metric Space
% =========================================================

\subsection{Topology of the Real Metric Space}

% =========================================================
\subsubsection{Basic Definitions}
% =========================================================


\begin{remark}[Why topology appears here]
The metric space and topological definitions introduced in this section ---
open sets, closed sets, neighborhoods, limit points, and compactness ---
are stated in their natural generality, but their real force only becomes
visible once sequences and limits are in hand.

In particular:
\begin{itemize}
  \item A point $x$ is a \emph{limit point} of a set $A$ if and only if
        some sequence in $A \setminus \{x\}$ converges to $x$.
  \item A set $F$ is \emph{closed} if and only if it contains the limits
        of all convergent sequences in $F$.
  \item A set $K$ is \emph{sequentially compact} if and only if every
        sequence in $K$ has a subsequence converging to a point in $K$.
        By the Heine--Borel theorem, this coincides with compactness for
        subsets of $\mathbb{R}^n$.
\end{itemize}

These equivalences are not incidental --- they reveal that the topological
language and the sequential language are two dialects describing the same
structure. The sequence-based characterizations are generally easier to
work with in proofs, while the open-set definitions generalize more
cleanly to spaces where sequences are insufficient (e.g.\ uncountable
products).

The results that follow --- convergence, Cauchy sequences,
Bolzano--Weierstrass, and subsequence theory --- should be read as
building the sequential side of this correspondence. The topological
interpretations will be noted where they arise.
\end{remark}


\begin{definition}[$\varepsilon$-neighborhood]
Let $x_0\in\mathbb{R}$ and $\varepsilon>0$.
\[
N_\varepsilon(x_0)
:=
\{x\in\mathbb{R} : |x-x_0|<\varepsilon\}.
\]
\end{definition}

\begin{definition}[Open ball]
Let $(X,d)$ be a metric space.
\[
B_\varepsilon(x_0)
:=
\{x\in X : d(x,x_0)<\varepsilon\}.
\]
\end{definition}

\begin{definition}[Closed ball]
\[
\overline{B}_\varepsilon(x_0)
:=
\{x\in X : d(x,x_0)\le\varepsilon\}.
\]
\end{definition}

\begin{definition}[Open set]
A set $U\subseteq X$ is open if
\[
\forall x\in U\;\exists \varepsilon>0
\quad\text{such that}\quad
B_\varepsilon(x)\subseteq U.
\]
\end{definition}

\begin{definition}[Closed set]
A set $F\subseteq X$ is closed if $X\setminus F$ is open.
\end{definition}

\begin{definition}[Open cover]
A family $\{U_\alpha\}_{\alpha\in I}$ of open sets is an open cover of $A\subseteq X$ if
\[
A \subseteq \bigcup_{\alpha\in I} U_\alpha.
\]
\end{definition}

\begin{definition}[Closure]
\[
\overline{A}
=
\bigcap\{F\subseteq X : F \text{ closed and } A\subseteq F\}.
\]
\end{definition}

\begin{definition}[Interior]
\[
A^\circ
=
\bigcup\{U\subseteq X : U \text{ open and } U\subseteq A\}.
\]
\end{definition}

\begin{definition}[Limit point]
\[
\forall \varepsilon>0,\quad
(B_\varepsilon(x)\setminus\{x\})\cap A \neq \varnothing.
\]
\end{definition}

\begin{definition}[Compact set]
Every open cover admits a finite subcover.
\end{definition}

\begin{definition}[Sequential compactness]
Every sequence in $K$ has a convergent subsequence whose limit lies in $K$.
\end{definition}

\begin{definition}[Bounded set]
\[
A \subseteq B_R(x_0)
\quad\text{for some } x_0,R.
\]
\end{definition}

% =========================================================
\subsubsection{Main Theorems}
% =========================================================

\begin{theorem}[Neighborhood = ball in $\mathbb{R}$]
With $d(x,y)=|x-y|$,
\[
N_\varepsilon(x_0)=B_\varepsilon(x_0).
\]
\end{theorem}

\begin{theorem}[Ball characterization of closure]
\[
x\in \overline{A}
\iff
\forall \varepsilon>0,\;
B_\varepsilon(x)\cap A\neq\varnothing.
\]
\end{theorem}

\begin{theorem}[Ball characterization of interior]
\[
x\in A^\circ
\iff
\exists \varepsilon>0,\;
B_\varepsilon(x)\subseteq A.
\]
\end{theorem}

\begin{theorem}[Heine--Borel Theorem on $\mathbb{R}^n$]
For $K\subseteq\mathbb{R}^n$, the following are equivalent:

\begin{enumerate}
\item $K$ compact
\item $K$ sequentially compact
\item Every infinite subset has a limit point in $K$
\item $K$ closed and bounded
\end{enumerate}

Implication cycle:
\[
(1)\Rightarrow(2)\Rightarrow(3)\Rightarrow(4)\Rightarrow(1).
\]
\end{theorem}

\begin{proof}[Proof sketch]
(1)$\Rightarrow$(2) compact $\Rightarrow$ sequentially compact in metric spaces.

(2)$\Rightarrow$(3) subsequence convergence gives limit point.

(3)$\Rightarrow$(4) rules out unboundedness and non-closedness.

(4)$\Rightarrow$(1) classical Heine--Borel theorem in $\mathbb{R}^n$.
\end{proof}

\begin{remark}
Closed and bounded implies compact only in finite-dimensional spaces.
\end{remark}

% =========================================================
\subsubsection{Geometric Illustration}
% =========================================================

\begin{center}
\begin{tikzpicture}[scale=1.05, line cap=round, line join=round]

\def\rBig{3.0}
\def\rSmall{1.4}

\coordinate (a)  at (0,0);
\coordinate (x0) at (-1.4,0);

\fill[blue!25] (a) circle (\rBig);
\fill[red!30]  (x0) circle (\rSmall);

\draw[blue!70!black, dashed, thick] (a) circle (\rBig);
\draw[red!70!black,  dashed, thick] (x0) circle (\rSmall);

\fill (x0) circle (1.3pt);
\fill (a)  circle (1.3pt);

\node[above left] at (x0) {$x_0$};
\node[above right] at (a) {$a$};

\coordinate (epspt) at ($(x0)+(1.1,0)$);
\draw[thick] (x0) -- (epspt);
\node[above] at ($(x0)!0.55!(epspt)$) {$\varepsilon$};

\node at ($(x0)+(0.2,-0.8)$) {$B(x_0,\varepsilon)$};
\node at ($(a)+(1.6,-1.9)$) {$B(a,r)$};

\end{tikzpicture}
\end{center}

% =========================================================
\subsubsection{Consequences}
% =========================================================

\begin{remark}[Logical Structure]
\[
\text{Metric}
\Rightarrow
\text{Open Balls}
\Rightarrow
\text{Open Sets}
\Rightarrow
\text{Closure / Interior}
\Rightarrow
\text{Compactness}
\]
\end{remark}

\begin{remark}
In $\mathbb{R}^n$, compactness is equivalent to closed and bounded.
This depends critically on completeness.
\end{remark}
