% =========================================================
% Topology — Structural Roadmap
% File: 07-topology/topology.tex
% =========================================================

\subsection{Introduction to Topology}

% =========================================================
% Structural Roadmap
% =========================================================
\subsubsection*{Structural Roadmap}

The development of topology follows the same
definition--theorem--structure architecture used throughout
this project, but now abstracts the notion of
\emph{neighborhood structure} independently of distance.

Each major topic is organized as:
\begin{center}
\textbf{Definitions $\longrightarrow$ Main Theorems $\longrightarrow$ Consequences and Logical Structure}
\end{center}

The global progression is:

\begin{enumerate}
  \item Definition of a topological space
  \item Open and closed sets
  \item Interior, closure, and boundary
  \item Accumulation (limit) points
  \item Subspace topology
  \item Compactness via open covers
  \item Sequential vs. open-cover compactness
\end{enumerate}

\vspace{1em}

\begin{remark}
Topology isolates the concept of
\emph{open sets} and neighborhood structure
without reference to algebra, order, or distance.
\end{remark}

\begin{remark}
Many notions previously defined using sequences
(convergence, compactness, limit points)
are reinterpreted here in terms of open sets.
\end{remark}

% =========================================================
% I. Topological Structure
% =========================================================

% =========================================================
% Topology of a Metric Space
% =========================================================

\subsection{Topology of the Real Metric Space}

% =========================================================
\subsubsection{Basic Definitions}
% =========================================================


\begin{remark}[Why topology appears here]
The metric space and topological definitions introduced in this section ---
open sets, closed sets, neighborhoods, limit points, and compactness ---
are stated in their natural generality, but their real force only becomes
visible once sequences and limits are in hand.

In particular:
\begin{itemize}
  \item A point $x$ is a \emph{limit point} of a set $A$ if and only if
        some sequence in $A \setminus \{x\}$ converges to $x$.
  \item A set $F$ is \emph{closed} if and only if it contains the limits
        of all convergent sequences in $F$.
  \item A set $K$ is \emph{sequentially compact} if and only if every
        sequence in $K$ has a subsequence converging to a point in $K$.
        By the Heine--Borel theorem, this coincides with compactness for
        subsets of $\mathbb{R}^n$.
\end{itemize}

These equivalences are not incidental --- they reveal that the topological
language and the sequential language are two dialects describing the same
structure. The sequence-based characterizations are generally easier to
work with in proofs, while the open-set definitions generalize more
cleanly to spaces where sequences are insufficient (e.g.\ uncountable
products).

The results that follow --- convergence, Cauchy sequences,
Bolzano--Weierstrass, and subsequence theory --- should be read as
building the sequential side of this correspondence. The topological
interpretations will be noted where they arise.
\end{remark}


\begin{definition}[$\varepsilon$-neighborhood]
Let $x_0\in\mathbb{R}$ and $\varepsilon>0$.
\[
N_\varepsilon(x_0)
:=
\{x\in\mathbb{R} : |x-x_0|<\varepsilon\}.
\]
\end{definition}

\begin{definition}[Open ball]
Let $(X,d)$ be a metric space.
\[
B_\varepsilon(x_0)
:=
\{x\in X : d(x,x_0)<\varepsilon\}.
\]
\end{definition}

\begin{definition}[Closed ball]
\[
\overline{B}_\varepsilon(x_0)
:=
\{x\in X : d(x,x_0)\le\varepsilon\}.
\]
\end{definition}

\begin{definition}[Open set]
A set $U\subseteq X$ is open if
\[
\forall x\in U\;\exists \varepsilon>0
\quad\text{such that}\quad
B_\varepsilon(x)\subseteq U.
\]
\end{definition}

\begin{definition}[Closed set]
A set $F\subseteq X$ is closed if $X\setminus F$ is open.
\end{definition}

\begin{definition}[Open cover]
A family $\{U_\alpha\}_{\alpha\in I}$ of open sets is an open cover of $A\subseteq X$ if
\[
A \subseteq \bigcup_{\alpha\in I} U_\alpha.
\]
\end{definition}

\begin{definition}[Closure]
\[
\overline{A}
=
\bigcap\{F\subseteq X : F \text{ closed and } A\subseteq F\}.
\]
\end{definition}

\begin{definition}[Interior]
\[
A^\circ
=
\bigcup\{U\subseteq X : U \text{ open and } U\subseteq A\}.
\]
\end{definition}

\begin{definition}[Limit point]
\[
\forall \varepsilon>0,\quad
(B_\varepsilon(x)\setminus\{x\})\cap A \neq \varnothing.
\]
\end{definition}

\begin{definition}[Compact set]
Every open cover admits a finite subcover.
\end{definition}

\begin{definition}[Sequential compactness]
Every sequence in $K$ has a convergent subsequence whose limit lies in $K$.
\end{definition}

\begin{definition}[Bounded set]
\[
A \subseteq B_R(x_0)
\quad\text{for some } x_0,R.
\]
\end{definition}

% =========================================================
\subsubsection{Main Theorems}
% =========================================================

\begin{theorem}[Neighborhood = ball in $\mathbb{R}$]
With $d(x,y)=|x-y|$,
\[
N_\varepsilon(x_0)=B_\varepsilon(x_0).
\]
\end{theorem}

\begin{theorem}[Ball characterization of closure]
\[
x\in \overline{A}
\iff
\forall \varepsilon>0,\;
B_\varepsilon(x)\cap A\neq\varnothing.
\]
\end{theorem}

\begin{theorem}[Ball characterization of interior]
\[
x\in A^\circ
\iff
\exists \varepsilon>0,\;
B_\varepsilon(x)\subseteq A.
\]
\end{theorem}

\begin{theorem}[Heine--Borel Theorem on $\mathbb{R}^n$]
For $K\subseteq\mathbb{R}^n$, the following are equivalent:

\begin{enumerate}
\item $K$ compact
\item $K$ sequentially compact
\item Every infinite subset has a limit point in $K$
\item $K$ closed and bounded
\end{enumerate}

Implication cycle:
\[
(1)\Rightarrow(2)\Rightarrow(3)\Rightarrow(4)\Rightarrow(1).
\]
\end{theorem}

\begin{proof}[Proof sketch]
(1)$\Rightarrow$(2) compact $\Rightarrow$ sequentially compact in metric spaces.

(2)$\Rightarrow$(3) subsequence convergence gives limit point.

(3)$\Rightarrow$(4) rules out unboundedness and non-closedness.

(4)$\Rightarrow$(1) classical Heine--Borel theorem in $\mathbb{R}^n$.
\end{proof}

\begin{remark}
Closed and bounded implies compact only in finite-dimensional spaces.
\end{remark}

% =========================================================
\subsubsection{Geometric Illustration}
% =========================================================

\begin{center}
\begin{tikzpicture}[scale=1.05, line cap=round, line join=round]

\def\rBig{3.0}
\def\rSmall{1.4}

\coordinate (a)  at (0,0);
\coordinate (x0) at (-1.4,0);

\fill[blue!25] (a) circle (\rBig);
\fill[red!30]  (x0) circle (\rSmall);

\draw[blue!70!black, dashed, thick] (a) circle (\rBig);
\draw[red!70!black,  dashed, thick] (x0) circle (\rSmall);

\fill (x0) circle (1.3pt);
\fill (a)  circle (1.3pt);

\node[above left] at (x0) {$x_0$};
\node[above right] at (a) {$a$};

\coordinate (epspt) at ($(x0)+(1.1,0)$);
\draw[thick] (x0) -- (epspt);
\node[above] at ($(x0)!0.55!(epspt)$) {$\varepsilon$};

\node at ($(x0)+(0.2,-0.8)$) {$B(x_0,\varepsilon)$};
\node at ($(a)+(1.6,-1.9)$) {$B(a,r)$};

\end{tikzpicture}
\end{center}

% =========================================================
\subsubsection{Consequences}
% =========================================================

\begin{remark}[Logical Structure]
\[
\text{Metric}
\Rightarrow
\text{Open Balls}
\Rightarrow
\text{Open Sets}
\Rightarrow
\text{Closure / Interior}
\Rightarrow
\text{Compactness}
\]
\end{remark}

\begin{remark}
In $\mathbb{R}^n$, compactness is equivalent to closed and bounded.
This depends critically on completeness.
\end{remark}


% =========================================================
% Future Sections (To Be Developed)
% =========================================================

% Uncomment as content expands:

% \input{07-topology/notes-closure-interior}
% % =========================================================
% Compactness Theorem
% =========================================================

\subsection{Compactness Theorem}

% ---------------------------------------------------------
% TOOLKIT
% ---------------------------------------------------------
\begin{tcolorbox}[colback=gray!6, colframe=gray!40, arc=2pt,
  left=6pt, right=6pt, top=4pt, bottom=4pt,
  title={\small\textbf{Compactness — Quick Reference}},
  fonttitle=\small\bfseries]
\small
\begin{tabular}{l l l}
\toprule
\textbf{Concept} & \textbf{Key fact} & \textbf{Detail} \\
\midrule
Compactness theorem & $\Gamma$ sat. $\Longleftrightarrow$ every finite $\Gamma_0\subseteq\Gamma$ sat. & \hyperref[thm:compactness]{↓ Thm} \\
Consequence form    & $\Gamma\models\varphi \Rightarrow$ some finite $\Gamma_0\models\varphi$ & \hyperref[cor:consequence-compact]{↓ Cor} \\
Graph coloring app. & Infinite graph $k$-colorable iff all finite subgraphs are & \hyperref[ex:graph-color]{↓ Ex} \\
Proof methods       & Via completeness; direct construction; ultraproducts & \hyperref[rem:compact-proof]{↓} \\
\bottomrule
\end{tabular}
\end{tcolorbox}

\vspace{1em}

% ---------------------------------------------------------
% Compactness Theorem
% ---------------------------------------------------------
\begin{tcolorbox}[colback=thmbox, colframe=thmborder, arc=2pt,
  left=6pt, right=6pt, top=4pt, bottom=4pt,
  title={\small\textbf{Theorem (Compactness of Propositional Logic)}},
  fonttitle=\small\bfseries]
\label{thm:compactness}
A set of formulas $\Gamma$ is satisfiable if and only if every finite subset
of $\Gamma$ is satisfiable.

Equivalently: if $\Gamma$ is unsatisfiable, some \emph{finite} subset of
$\Gamma$ is already unsatisfiable.
\end{tcolorbox}

\begin{remark}[English reading]
Compactness says that satisfiability is a \emph{finitary} property: an infinite
set of premises can only be contradictory if some finite fragment is already
contradictory. You can never need infinitely many premises to derive a
contradiction; finitely many always suffice.
\end{remark}

\begin{remark}[Fully quantified form]
$\forall \Gamma \subseteq \mathsf{WFF}:\;
\bigl(\exists v,\; v \models \Gamma\bigr)
\;\Longleftrightarrow\;
\bigl(\forall \Gamma_0 \subseteq_{\mathrm{fin}} \Gamma,\;
\exists v,\; v \models \Gamma_0\bigr)$.
\end{remark}

\begin{remark}[Which direction is trivial]
($\Rightarrow$): If $v$ satisfies $\Gamma$, it satisfies every subset. Trivial.

($\Leftarrow$): The substantive direction. If every finite subset is satisfiable,
is the whole set? Intuition says ``yes'' but it is not obvious — an infinite set
might encode constraints that collectively rule out every assignment even though
no finite fragment does.
\end{remark}

\begin{remark}[Why this is a \emph{meta}theorem]
Compactness is a statement about the entire logic, not about any particular
formula. It belongs to metatheory — it tells us something about the structure
of propositional logic itself.
\end{remark}

\begin{remark}[Proof methods]\label{rem:compact-proof}
Three standard approaches:
\begin{enumerate}
  \item \textbf{Via completeness:} If every finite subset of $\Gamma$ is
        satisfiable, no finite subset derives $\bot$. By completeness, $\Gamma$
        does not derive $\bot$. By soundness-completeness, $\Gamma$ is consistent,
        hence satisfiable. (Circular-looking but rigorous with careful bookkeeping.)
  \item \textbf{Direct construction (König's lemma):} Build a satisfying
        assignment by extending partial assignments level by level through all
        formulas in $\Gamma$, using compactness of $\{T,F\}^\omega$ (Tychonoff).
  \item \textbf{Ultraproducts:} Take an ultraproduct of the satisfying assignments
        for all finite subsets.
\end{enumerate}
\end{remark}

% ---------------------------------------------------------
% Consequence form
% ---------------------------------------------------------
\begin{corollary}[Compactness of Logical Consequence]\label{cor:consequence-compact}
If $\Gamma \models \varphi$, then there exists a \emph{finite} subset
$\Gamma_0 \subseteq \Gamma$ with $\Gamma_0 \models \varphi$.

Equivalently: logical consequence from an infinite premise set always depends
on only finitely many premises.
\end{corollary}



\begin{remark}[Practical implication]
In any propositional proof, one uses only finitely many formulas from $\Gamma$.
Compactness tells us this is not a limitation — we never \emph{need} infinitely
many premises for a single conclusion.
\end{remark}

% ---------------------------------------------------------
% Applications
% ---------------------------------------------------------
\begin{example}[Graph Coloring]\label{ex:graph-color}
\textbf{Claim:} An infinite graph $G$ is $k$-colorable if and only if every
finite subgraph of $G$ is $k$-colorable.

\textbf{Encoding:} For each vertex $v$ and color $i \in \{1,\ldots,k\}$,
introduce propositional variable $C_{v,i}$ (``vertex $v$ has color $i$'').
Add formulas:
\begin{itemize}
  \item For each $v$: $C_{v,1} \vee \cdots \vee C_{v,k}$ (every vertex has some color).
  \item For each $v$, distinct $i,j$: $\neg(C_{v,i} \wedge C_{v,j})$ (at most one color).
  \item For each edge $(u,v)$, each $i$: $\neg(C_{u,i} \wedge C_{v,i})$ (adjacent vertices differ).
\end{itemize}
Each finite subset of this formula set involves only finitely many vertices —
a finite subgraph. If every finite subgraph is $k$-colorable, every finite
subset is satisfiable. By compactness, the full set is satisfiable, giving a
$k$-coloring of $G$.
\end{example}

\begin{remark}[Why compactness is surprising here]
The graph coloring argument works even for uncountably infinite graphs.
Compactness is doing real work: it converts local, finite satisfiability into
global satisfiability for a system with infinitely many variables.
\end{remark}

% \input{07-topology/notes-subspace}
% \input{07-topology/notes-product-topology}

% =========================================================
% Logical Positioning
% =========================================================

\begin{remark}[Structural Position]
Topology is introduced after metric spaces
to separate the idea of neighborhood structure
from the notion of distance.

This abstraction allows compactness,
continuity, and convergence to be studied
in their most general setting.
\end{remark}
