% =========================================================
% Set Algebras — Structural Roadmap
% File: set-algebras/set-algebras.tex
% =========================================================

\subsection{Set Algebras}

% =========================================================
% Structural Roadmap
% =========================================================
\subsubsection*{Structural Roadmap}

The study of set algebras develops a structural theory
of families of subsets of a fixed space.
Rather than focusing on individual elements,
we investigate the \emph{stability properties}
of collections of sets under recombination.

The central question is:

\begin{center}
\emph{Under which operations is a family of sets closed?}
\end{center}

Different closure assumptions give rise to
distinct algebraic structures on $2^X$.

Each major topic follows the architecture:
\begin{center}
\textbf{Definitions $\longrightarrow$ Structural Consequences $\longrightarrow$ Hierarchy of Systems}
\end{center}

The global progression is:

\begin{enumerate}
  \item Families of sets and closure operations
  \item Finite closure structures:
        rings, algebras (fields), Boolean algebras
  \item Countable closure structures:
        $\sigma$-rings and $\sigma$-algebras
  \item Generated systems and minimal closure
  \item Auxiliary systems (e.g.\ $\pi$-systems, Dynkin systems)
  \item Structural comparison of closure regimes
\end{enumerate}

\vspace{1em}

\begin{remark}
Set algebras form a theory of
\emph{closure under operations}.
They classify families of sets according to how
stable they are under union, intersection,
complement, difference, and limit-type constructions.
\end{remark}

\begin{remark}
Finite closure produces algebraic stability;
countable closure prepares compatibility with
limit processes.
These structures will later appear naturally in
topology, measure theory, probability,
and Boolean algebra,
but they are studied here independently
as algebraic systems of subsets.
\end{remark}

% =========================================================
% I. Systems of Sets
% =========================================================

% =========================================================
% Systems of Sets
% =========================================================

\subsection{Systems of Sets}

% ---------------------------------------------------------
% TOOLKIT
% ---------------------------------------------------------
\begin{tcolorbox}[colback=gray!6, colframe=gray!40, arc=2pt,
  left=6pt, right=6pt, top=4pt, bottom=4pt,
  title={\small\textbf{Systems of Sets --- Quick Reference}},
  fonttitle=\small\bfseries]
\small
\begin{tabular}{l l l}
\toprule
\textbf{Structure} & \textbf{Closure properties} & \textbf{Detail} \\
\midrule
Family of sets   & Any $\mathcal{F} \subseteq 2^X$                                          & \hyperref[def:family-sets]{↓ Def} \\
Closed under $*$ & Inputs in $\mathcal{F}$ implies output in $\mathcal{F}$                   & \hyperref[def:closure-op]{↓ Def} \\
Ring of sets     & Closed under $\cup$ and $\setminus$                                      & \hyperref[def:ring-sets]{↓ Def} \\
Algebra of sets  & Contains $X$; closed under $(\cdot)^c$ and $\cup$                        & \hyperref[def:algebra-sets]{↓ Def} \\
$\sigma$-ring    & Ring closed under countable $\cup$                                        & \hyperref[def:sigma-ring]{↓ Def} \\
$\sigma$-algebra & Algebra closed under countable $\cup$                                     & \hyperref[def:sigma-algebra]{↓ Def} \\
\midrule
\multicolumn{3}{l}{\textit{Key results:}} \\
\midrule
Algebra = ring + $X$ & Equivalent characterisation                                          & \hyperref[prop:algebra-ring]{↓ Prop} \\
$\sigma$-algebra closure & Also closed under countable $\cap$, $\setminus$, finite Boolean ops & \hyperref[prop:sigma-closure]{↓ Prop} \\
\bottomrule
\end{tabular}
\end{tcolorbox}

\vspace{1em}

% ---------------------------------------------------------
% Families and closure
% ---------------------------------------------------------

Let $X$ be a set. A \emph{family of sets on $X$} is a collection
$\mathcal{F} \subseteq 2^X$. The objects of study are not elements of $X$
but collections of subsets, classified by their stability under
set-theoretic operations.

\begin{tcolorbox}[colback=propbox, colframe=propborder, arc=2pt,
  left=6pt, right=6pt, top=4pt, bottom=4pt,
  title={\small\textbf{Definition (Family of Sets)}},
  fonttitle=\small\bfseries]
\label{def:family-sets}
Let $X$ be a set. A \emph{family of sets on $X$} is any collection
$\mathcal{F} \subseteq 2^X$.
\end{tcolorbox}

\begin{tcolorbox}[colback=propbox, colframe=propborder, arc=2pt,
  left=6pt, right=6pt, top=4pt, bottom=4pt,
  title={\small\textbf{Definition (Closure Under an Operation)}},
  fonttitle=\small\bfseries]
\label{def:closure-op}
Let $\mathcal{F} \subseteq 2^X$ and let $*$ be an operation on subsets of
$X$. We say $\mathcal{F}$ is \emph{closed under $*$} if whenever the inputs
to $*$ belong to $\mathcal{F}$, the output also belongs to $\mathcal{F}$.
\end{tcolorbox}

\begin{remark}[Typical operations]
The operations of interest are $\cup$, $\cap$, $(\cdot)^c$, $\setminus$,
and $\triangle$ (symmetric difference). Different combinations of closure
under these operations produce the distinct algebraic structures defined
below.
\end{remark}

% ---------------------------------------------------------
% Finite closure structures
% ---------------------------------------------------------

\begin{tcolorbox}[colback=propbox, colframe=propborder, arc=2pt,
  left=6pt, right=6pt, top=4pt, bottom=4pt,
  title={\small\textbf{Definition (Ring of Sets)}},
  fonttitle=\small\bfseries]
\label{def:ring-sets}
A nonempty collection $\mathcal{R} \subseteq 2^X$ is a \emph{ring of sets}
if it is closed under finite union and set difference:
\begin{enumerate}[label=(\roman*)]
  \item $A, B \in \mathcal{R} \;\Rightarrow\; A \cup B \in \mathcal{R}$,
  \item $A, B \in \mathcal{R} \;\Rightarrow\; A \setminus B \in \mathcal{R}$.
\end{enumerate}
\end{tcolorbox}

\begin{remark}[Intersection and symmetric difference]
A ring of sets is automatically closed under $\cap$ and $\triangle$,
since $A \cap B = A \setminus (A \setminus B)$ and
$A \triangle B = (A \setminus B) \cup (B \setminus A)$.
Under symmetric difference and intersection, a ring of sets is an abelian
group under $\triangle$ and a commutative monoid under $\cap$, making it
a Boolean ring in the algebraic sense.
\end{remark}

\begin{tcolorbox}[colback=propbox, colframe=propborder, arc=2pt,
  left=6pt, right=6pt, top=4pt, bottom=4pt,
  title={\small\textbf{Definition (Algebra of Sets)}},
  fonttitle=\small\bfseries]
\label{def:algebra-sets}
A collection $\mathcal{A} \subseteq 2^X$ is an \emph{algebra of sets}
(also called a \emph{field of sets}) if:
\begin{enumerate}[label=(\roman*)]
  \item $X \in \mathcal{A}$,
  \item $A \in \mathcal{A} \;\Rightarrow\; A^c \in \mathcal{A}$,
  \item $A, B \in \mathcal{A} \;\Rightarrow\; A \cup B \in \mathcal{A}$.
\end{enumerate}
\end{tcolorbox}

\begin{proposition}[Algebra = Ring + $X$]\label{prop:algebra-ring}
An algebra of sets is precisely a ring of sets that contains $X$.
\end{proposition}



\begin{remark}[Finite Boolean stability]
An algebra of sets is closed under all finite Boolean combinations:
finite unions, finite intersections, complements, differences, and
symmetric differences. The qualifier ``finite'' is essential; the next
step is to add closure under countably infinite operations.
\end{remark}

% ---------------------------------------------------------
% Countable closure structures
% ---------------------------------------------------------

\begin{tcolorbox}[colback=propbox, colframe=propborder, arc=2pt,
  left=6pt, right=6pt, top=4pt, bottom=4pt,
  title={\small\textbf{Definition ($\sigma$-Ring)}},
  fonttitle=\small\bfseries]
\label{def:sigma-ring}
A ring of sets $\mathcal{R}$ is a \emph{$\sigma$-ring} if it is also
closed under countable unions:
\[
A_1, A_2, \dots \in \mathcal{R}
\;\Rightarrow\;
\bigcup_{n=1}^{\infty} A_n \in \mathcal{R}.
\]
\end{tcolorbox}

\begin{tcolorbox}[colback=propbox, colframe=propborder, arc=2pt,
  left=6pt, right=6pt, top=4pt, bottom=4pt,
  title={\small\textbf{Definition ($\sigma$-Algebra)}},
  fonttitle=\small\bfseries]
\label{def:sigma-algebra}
A collection $\mathcal{F} \subseteq 2^X$ is a \emph{$\sigma$-algebra} if:
\begin{enumerate}[label=(\roman*)]
  \item $X \in \mathcal{F}$,
  \item $A \in \mathcal{F} \;\Rightarrow\; A^c \in \mathcal{F}$,
  \item $A_1, A_2, \dots \in \mathcal{F} \;\Rightarrow\;
        \displaystyle\bigcup_{n=1}^{\infty} A_n \in \mathcal{F}$.
\end{enumerate}
\end{tcolorbox}

\begin{proposition}[Closure consequences of a $\sigma$-algebra]\label{prop:sigma-closure}
Every $\sigma$-algebra $\mathcal{F}$ is also closed under countable
intersections, set differences, and all finite Boolean combinations.
\end{proposition}



\begin{remark}[Finite vs.\ countable closure]
The passage from an algebra to a $\sigma$-algebra is precisely the step
from finite to countable closure. This is what makes $\sigma$-algebras
compatible with limit processes: if measurable sets are approximated by
sequences of other measurable sets, the limit set remains measurable.
\end{remark}

% ---------------------------------------------------------
% Hierarchy
% ---------------------------------------------------------

\begin{center}
\renewcommand{\arraystretch}{1.3}
\begin{tabular}{lccc}
\toprule
\textbf{Structure} & \textbf{Contains $X$} & \textbf{Closed under $(\cdot)^c$} & \textbf{Closed under $\bigcup$} \\
\midrule
Ring of sets   & No  & No  & Finite only \\
Algebra        & Yes & Yes & Finite only \\
$\sigma$-Ring  & No  & No  & Countable \\
$\sigma$-Algebra & Yes & Yes & Countable \\
\bottomrule
\end{tabular}
\end{center}

\begin{remark}[Reading the hierarchy]
Each row is strictly stronger than the one above it: every algebra is a
ring, and every $\sigma$-algebra is both an algebra and a $\sigma$-ring.
The top of the hierarchy, the $\sigma$-algebra, is the natural domain for
measure theory because it combines the Boolean stability of an algebra with
the limit-compatibility of countable closure.
\end{remark}
% =========================================================
% Power Set and Characteristic Functions
% File: set-algebras/notes/notes-power-set-and-characteristic-functions.tex
% =========================================================

\subsubsection{The Power Set as a Function Space}

Let $X$ be a set.

\begin{definition}[Power Set]
The \emph{power set} of $X$ is
\[
2^X := \{ A \mid A \subseteq X \}.
\]
\end{definition}

\begin{remark}
The notation $2^X$ reflects the fact that
\[
|2^X| = 2^{|X|}.
\]
Each element of $X$ has two possibilities
with respect to a subset: inclusion or exclusion.
\end{remark}

% =========================================================
% Characteristic Functions
% =========================================================

\subsubsection{Characteristic Functions}

\begin{definition}[Characteristic Function]
Let $A \subseteq X$.
The \emph{characteristic function} (or indicator function)
of $A$ is the function
\[
\chi_A : X \to \{0,1\}
\]
defined by
\[
\chi_A(x) =
\begin{cases}
1 & \text{if } x \in A, \\
0 & \text{if } x \notin A.
\end{cases}
\]
\end{definition}

\vspace{0.5em}

\begin{theorem}[Identification of Subsets with Functions]
The map
\[
A \longmapsto \chi_A
\]
is a bijection between $2^X$ and the function space
\[
\{0,1\}^X.
\]

Consequently,
\[
2^X = \{0,1\}^X.
\]
\end{theorem}

\begin{remark}
Subsets of $X$ may therefore be viewed as
functions from $X$ into a two-element set.
This functional interpretation is often more
structural than the set-theoretic one.
\end{remark}

% =========================================================
% Algebraic Structure of the Power Set
% =========================================================

\subsubsection{Algebraic Structure on $2^X$}

Under set-theoretic operations,
$2^X$ acquires rich algebraic structure.

\begin{itemize}
    \item Union corresponds to pointwise maximum.
    \item Intersection corresponds to pointwise minimum.
    \item Complement corresponds to $1 - \chi_A$.
\end{itemize}

\vspace{0.5em}

\begin{remark}
With operations
\[
A \triangle B
\quad \text{(symmetric difference)}
\]
and
\[
A \cap B,
\]
the power set $2^X$ forms a Boolean ring.
\end{remark}

\vspace{0.5em}

\begin{remark}
If addition is defined by symmetric difference,
then $2^X$ behaves like a vector space
over the two-element field $\mathbb{F}_2$.
This provides a bridge between set theory
and linear algebra.
\end{remark}

% =========================================================
% Structural Perspective
% =========================================================

\subsubsection{Structural Perspective}

Families of sets
\[
\mathcal{F} \subseteq 2^X
\]
may therefore be viewed as substructures
of the function space $\{0,1\}^X$.

Different closure assumptions determine whether
$\mathcal{F}$ forms:

\begin{itemize}
    \item a ring of sets,
    \item an algebra of sets,
    \item a $\sigma$-algebra,
    \item or other closure systems.
\end{itemize}

\vspace{0.5em}

\begin{remark}
This interpretation explains why set algebras
share structural similarities with
Boolean algebras, rings, and vector spaces.
\end{remark}
% =========================================================
% Future Sections (Expandable Architecture)
% =========================================================

% Uncomment as content expands:

% \input{intro-set-algebras/notes-finite-structures}
% \input{intro-set-algebras/notes-boolean-algebras}
% \input{intro-set-algebras/notes-sigma-structures}
% \input{intro-set-algebras/notes-generated-systems}
% \input{intro-set-algebras/notes-auxiliary-systems}
% \input{intro-set-algebras/notes-applications}

% =========================================================
% Logical Positioning
% =========================================================

\begin{remark}[Structural Position]
This section isolates the algebraic behavior
of systems of sets without committing
to any particular application.

Topology will later arise from arbitrary unions
and finite intersections;
measure theory will arise from
$\sigma$-algebras equipped with additive functionals.
Both are specializations of the general closure
framework developed here.
\end{remark}