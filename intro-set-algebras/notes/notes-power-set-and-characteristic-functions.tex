% =========================================================
% Power Set and Characteristic Functions
% File: set-algebras/notes/notes-power-set-and-characteristic-functions.tex
% =========================================================

\subsubsection{The Power Set as a Function Space}

Let $X$ be a set.

\begin{definition}[Power Set]
The \emph{power set} of $X$ is
\[
2^X := \{ A \mid A \subseteq X \}.
\]
\end{definition}

\begin{remark}
The notation $2^X$ reflects the fact that
\[
|2^X| = 2^{|X|}.
\]
Each element of $X$ has two possibilities
with respect to a subset: inclusion or exclusion.
\end{remark}

% =========================================================
% Characteristic Functions
% =========================================================

\subsubsection{Characteristic Functions}

\begin{definition}[Characteristic Function]
Let $A \subseteq X$.
The \emph{characteristic function} (or indicator function)
of $A$ is the function
\[
\chi_A : X \to \{0,1\}
\]
defined by
\[
\chi_A(x) =
\begin{cases}
1 & \text{if } x \in A, \\
0 & \text{if } x \notin A.
\end{cases}
\]
\end{definition}

\vspace{0.5em}

\begin{theorem}[Identification of Subsets with Functions]
The map
\[
A \longmapsto \chi_A
\]
is a bijection between $2^X$ and the function space
\[
\{0,1\}^X.
\]

Consequently,
\[
2^X = \{0,1\}^X.
\]
\end{theorem}

\begin{remark}
Subsets of $X$ may therefore be viewed as
functions from $X$ into a two-element set.
This functional interpretation is often more
structural than the set-theoretic one.
\end{remark}

% =========================================================
% Algebraic Structure of the Power Set
% =========================================================

\subsubsection{Algebraic Structure on $2^X$}

Under set-theoretic operations,
$2^X$ acquires rich algebraic structure.

\begin{itemize}
    \item Union corresponds to pointwise maximum.
    \item Intersection corresponds to pointwise minimum.
    \item Complement corresponds to $1 - \chi_A$.
\end{itemize}

\vspace{0.5em}

\begin{remark}
With operations
\[
A \triangle B
\quad \text{(symmetric difference)}
\]
and
\[
A \cap B,
\]
the power set $2^X$ forms a Boolean ring.
\end{remark}

\vspace{0.5em}

\begin{remark}
If addition is defined by symmetric difference,
then $2^X$ behaves like a vector space
over the two-element field $\mathbb{F}_2$.
This provides a bridge between set theory
and linear algebra.
\end{remark}

% =========================================================
% Structural Perspective
% =========================================================

\subsubsection{Structural Perspective}

Families of sets
\[
\mathcal{F} \subseteq 2^X
\]
may therefore be viewed as substructures
of the function space $\{0,1\}^X$.

Different closure assumptions determine whether
$\mathcal{F}$ forms:

\begin{itemize}
    \item a ring of sets,
    \item an algebra of sets,
    \item a $\sigma$-algebra,
    \item or other closure systems.
\end{itemize}

\vspace{0.5em}

\begin{remark}
This interpretation explains why set algebras
share structural similarities with
Boolean algebras, rings, and vector spaces.
\end{remark}