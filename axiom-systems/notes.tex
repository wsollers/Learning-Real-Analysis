% notes.tex

\subsection{Axiom Systems and the Natural Numbers}
\label{sec:axioms-naturals}

% ---------------------------------------------------------
% 0. Notes to Self (optional)
% ---------------------------------------------------------
% Goal:
%   - Record what an axiom system is (operationally).
%   - Present Peano axioms as the logical characterization of N.
%   - Present Zermelo and von Neumann as set-theoretic models/encodings.
%   - Build a home for induction + recursion + core proof drills.
%
% Style:
%   - Keep this section usable as a reference.
%   - Avoid deep ZFC digressions; only what is needed for the constructions.

% ---------------------------------------------------------
% 1. Axiom Systems (lightweight, operational)
% ---------------------------------------------------------
\subsection{What Is an Axiom System?}
\label{subsec:axiom-system}

\begin{definition}[Axiomatic theory (informal)]
An \emph{axiomatic theory} consists of:
\begin{itemize}
  \item a \emph{language} (symbols for objects, functions, relations), and
  \item a collection of \emph{axioms} (sentences in the language)
\end{itemize}
intended to describe a class of mathematical structures (\emph{models}).
\end{definition}

\begin{definition}[Model (informal)]
A \emph{model} of an axiom system is a structure in which all axioms are true
under the intended interpretation of the symbols.
\end{definition}

\begin{remark}[Why this matters here]
We want a precise description of the natural numbers $\mathbb{N}$ that supports
\emph{induction} and \emph{recursion} without circularity.
\end{remark}


% ---------------------------------------------------------
% 2. The Peano Axioms (logical characterization)
% ---------------------------------------------------------
\subsection{The Peano Axioms}
\label{subsec:peano}

\subsubsection{Language}
We work in a language with:
\begin{itemize}
  \item a constant symbol $0$,
  \item a unary function symbol $S$ (successor),
  \item and (optionally) equality $=$ as a logical symbol.
\end{itemize}

\begin{definition}[Peano system]
A \emph{Peano system} is a triple $(N,0,S)$ satisfying the Peano axioms below.
\end{definition}

\subsubsection{Axioms}
\begin{axiom}[Peano 1: $0$ is not a successor]\label{ax:peano1}
$\forall n\in N,\; S(n)\neq 0$.
\end{axiom}

\begin{axiom}[Peano 2: successor is injective]\label{ax:peano2}
$\forall m,n\in N,\; S(m)=S(n)\Rightarrow m=n$.
\end{axiom}

\begin{axiom}[Peano 3: $0\in N$]\label{ax:peano3}
$0\in N$.
\end{axiom}

\begin{axiom}[Peano 4: closure under successor]\label{ax:peano4}
$\forall n\in N,\; S(n)\in N$.
\end{axiom}

\begin{axiom}[Peano 5: Induction]\label{ax:peano5}
For every subset $A\subseteq N$, if
\[
0\in A \quad\text{and}\quad (\forall n\in N)\,(n\in A \Rightarrow S(n)\in A),
\]
then $A=N$.
\end{axiom}

\begin{remark}[Second-order nature]
Axiom~\ref{ax:peano5} quantifies over \emph{subsets} $A\subseteq N$ (i.e., over
predicates). This is why it is often described as \emph{second-order induction}.
\end{remark}

\begin{remark}[Uniqueness up to isomorphism]
Any two systems satisfying the Peano axioms are isomorphic.
Thus, the natural numbers are unique up to relabeling of elements.
\end{remark}



% ---------------------------------------------------------
% 7. Set-Theoretic Models of N (Zermelo, von Neumann)
% ---------------------------------------------------------
\subsection{Set-Theoretic Models of the Natural Numbers}
\label{subsec:set-theoretic-models}

\subsubsection{Zermelo numerals}
\begin{definition}[Zermelo construction]\label{def:zermelo}
Define
\[
0 := \varnothing,\qquad n^{+} := \{n\}.
\]
The \emph{Zermelo numeral} $n$ is obtained by iterating the successor operation
starting from $0$.
\end{definition}

\begin{remark}
This realizes each natural number as a set, but membership does not directly encode
order in the same convenient way as von Neumann ordinals do.
\end{remark}



Recall that in Zermelo's construction,
\[
0 := \varnothing,
\qquad
S(n) := \{n\}.
\]

The first few natural numbers are therefore:

\[
\begin{aligned}
0 &= \varnothing, \\[0.5em]
1 &= S(0) = \{0\} = \{\varnothing\}, \\[0.5em]
2 &= S(1) = \{1\} = \bigl\{\{\varnothing\}\bigr\}, \\[0.5em]
3 &= S(2) = \{2\} = \bigl\{\{\{\varnothing\}\}\bigr\}.
\end{aligned}
\]

Each natural number is a singleton whose sole element is the previous numeral.





\subsubsection{von Neumann ordinals}
\begin{definition}[von Neumann construction]\label{def:von-neumann}
Define
\[
0 := \varnothing,\qquad n^{+} := n\cup\{n\}.
\]
Equivalently, $n$ is the set of all previous numerals:
\[
n = \{0,1,2,\dots,n-1\}.
\]
\end{definition}


Recall that in the von Neumann construction,
\[
0 := \varnothing,
\qquad
S(n) := n \cup \{n\}.
\]

Equivalently, each natural number is defined as the set of all smaller natural
numbers.

The first few von Neumann numerals are:

\[
\begin{aligned}
0 &= \varnothing, \\[0.5em]
1 &= S(0) = 0 \cup \{0\} = \{\varnothing\}, \\[0.5em]
2 &= S(1) = 1 \cup \{1\} = \{\varnothing,\{\varnothing\}\}, \\[0.5em]
3 &= S(2) = 2 \cup \{2\}
   = \{\varnothing,\{\varnothing\},\{\varnothing,\{\varnothing\}\}\}.
\end{aligned}
\]

In this construction, membership encodes order: for natural numbers $m,n$,
\[
m \in n \quad \Longleftrightarrow \quad m < n.
\]

\begin{proposition}[Membership encodes order for von Neumann numerals]
Let $m,n\in\mathbb{N}$ be natural numbers represented as von Neumann numerals.
Then
\[
m \in n \quad \Longleftrightarrow \quad m < n.
\]
\end{proposition}

\begin{proof}
Recall that in the von Neumann construction,
\[
n := \{0,1,2,\dots,n-1\}.
\]

(\emph{$\Rightarrow$})  
If $m \in n$, then by definition $m$ is one of the elements of the set
$\{0,1,2,\dots,n-1\}$. Hence $m<n$.

(\emph{$\Leftarrow$})  
If $m<n$, then $m$ is one of the natural numbers listed in
$\{0,1,2,\dots,n-1\}$. Therefore $m\in n$.

Thus, membership and order coincide for von Neumann numerals.
\end{proof}

\begin{remark}
When working with von Neumann numerals, the order relation $<$ may be identified
with set membership $\in$.
\end{remark}



\begin{remark}
In the von Neumann construction, each natural number $n$ is literally the set of
all smaller natural numbers. As a result, the order relation on $\mathbb{N}$ is
not imposed separately; it is already built into the membership relation $\in$.
\end{remark}








\begin{remark}[Notation warning]
Until addition is defined, the successor $S(n)$ should not be interpreted
as $n+1$. Any appearance of $n-1$ or $n+1$ here is purely descriptive.
\end{remark}

\begin{remark}
The Peano axioms characterize the natural numbers by their behavior, not by
their internal structure. Zermelo and von Neumann constructions provide
concrete models, not definitions.
\end{remark}



\begin{definition}[Inductive set]
A set $A$ is called \emph{inductive} if
\[
0 \in A \quad \text{and} \quad (\forall n \in A)\, S(n) \in A.
\]
\end{definition}

\begin{proposition}[Minimality of $\mathbb{N}$]
The natural numbers form the smallest inductive set; that is, if $A$ is any
inductive set, then $\mathbb{N} \subseteq A$.
\end{proposition}

\begin{proof}
Apply Axiom~\ref{ax:peano5} with $A$ as given.
\end{proof}


% ---------------------------------------------------------
% Induction on the Natural Numbers
% ---------------------------------------------------------
\subsection{Induction on the Natural Numbers}
\label{subsec:induction}

Induction is not a proof technique that happens to work on $\mathbb{N}$; it is a
logical consequence of the axiomatic structure of the natural numbers. In
particular, the induction principle follows directly from the Peano axioms.

\begin{remark}
Induction on $\mathbb{N}$ is a special case of \emph{structural induction}, which
applies to recursively defined objects in general.
\end{remark}

% ---------------------------------------------------------
% Weak Induction
% ---------------------------------------------------------
\subsubsection{Weak Induction}

\begin{theorem}[Weak induction]\label{thm:weak-induction}
Let $(\mathbb{N},0,S)$ be a Peano system, and let $P(n)$ be a statement depending
on $n\in\mathbb{N}$. Suppose that:
\begin{enumerate}
  \item $P(0)$ is true, and
  \item for all $n\in\mathbb{N}$, $P(n)\Rightarrow P(S(n))$.
\end{enumerate}
Then $P(n)$ is true for all $n\in\mathbb{N}$.
\end{theorem}

\begin{proof}
Define
\[
A := \{ n\in\mathbb{N} : P(n) \text{ is true} \}.
\]
By assumption, $0\in A$, and if $n\in A$ then $S(n)\in A$. Hence $A$ is an
inductive subset of $\mathbb{N}$. By Axiom~\ref{ax:peano5}, $A=\mathbb{N}$.
\end{proof}

\begin{remark}
Weak induction corresponds directly to reasoning along the successor chain
\[
0,\ S(0),\ S(S(0)),\ \dots
\]
and is particularly well suited to statements defined recursively.
\end{remark}

% ---------------------------------------------------------
% Strong Induction
% ---------------------------------------------------------
\subsubsection{Strong Induction}

\begin{theorem}[Strong induction]\label{thm:strong-induction}
Let $(\mathbb{N},0,S)$ be a Peano system equipped with its usual order relation.
Let $P(n)$ be a statement depending on $n\in\mathbb{N}$. Suppose that for every
$n\in\mathbb{N}$,
\[
\bigl[(\forall k<n)\,P(k)\bigr] \;\Rightarrow\; P(n).
\]
Then $P(n)$ is true for all $n\in\mathbb{N}$.
\end{theorem}

\begin{proof}
Define
\[
A := \{ n\in\mathbb{N} : (\forall k<n)\,P(k) \}.
\]
By vacuity, $0\in A$. Suppose $n\in A$; then $P(k)$ holds for all $k<n$, and hence
$P(n)$ holds by hypothesis. Therefore $P(k)$ holds for all $k<S(n)$, which implies
$S(n)\in A$. Thus $A$ is inductive, and by Axiom~\ref{ax:peano5}, $A=\mathbb{N}$.
\end{proof}

\begin{remark}
Strong induction is logically equivalent to weak induction, but it is often more
convenient when the statement $P(n)$ depends on multiple earlier values.
\end{remark}

% ---------------------------------------------------------
% Example: Induction Proof on the Natural Numbers
% ---------------------------------------------------------
\subsubsection{Example: A Proof by Induction}

\begin{proposition}\label{prop:zero-is-minimal}
There is no natural number $n\in\mathbb{N}$ such that $S(n)=0$.
\end{proposition}

\begin{proof}
We prove the statement by induction on $n$.

Let $P(n)$ denote the statement $S(n)\neq 0$.

\textbf{Base case.}  
By Axiom~\ref{ax:peano1}, $S(0)\neq 0$, so $P(0)$ holds.

\textbf{Inductive step.}  
Assume $P(n)$ holds for some $n\in\mathbb{N}$. Then $S(n)\neq 0$. Applying the
successor function, we have $S(S(n))\neq 0$ as well, since $0$ is not a successor
of any element by Axiom~\ref{ax:peano1}. Hence $P(S(n))$ holds.

By Theorem~\ref{thm:weak-induction}, $P(n)$ holds for all $n\in\mathbb{N}$.
\end{proof}

\begin{remark}
The statement $(\forall k<0)\,P(k)$ is vacuously true, since there are no natural
numbers less than $0$.
\end{remark}

\begin{remark}
This proposition illustrates the typical structure of an induction proof:
a base case verifying the property at $0$, followed by a successor step showing
that the property propagates along the successor chain.
\end{remark}

\begin{proposition}[Equivalence of induction principles]
Weak induction and strong induction are logically equivalent.
\end{proposition}

\begin{proof}
Strong induction implies weak induction trivially. Conversely, strong induction
follows from weak induction by applying it to the statement
\[
Q(n) := (\forall k<n)\,P(k).
\]
\end{proof}




% ---------------------------------------------------------
% Recursion on the Natural Numbers
% ---------------------------------------------------------
\subsection{Recursion on the Natural Numbers}
\label{subsec:recursion}

Induction allows us to prove statements about all natural numbers.
Recursion allows us to \emph{define} objects indexed by the natural numbers.
Both principles arise from the same axiomatic structure.

% ---------------------------------------------------------
% Recursive Definitions (Motivation)
% ---------------------------------------------------------
\subsubsection{Recursive Definitions}

Many mathematical constructions on $\mathbb{N}$ are specified by:
\begin{itemize}
  \item an initial value at $0$, and
  \item a rule describing how to pass from $n$ to $S(n)$.
\end{itemize}

The recursion theorem guarantees that such specifications determine a
\emph{unique} function on $\mathbb{N}$.

% ---------------------------------------------------------
% Recursion Theorem
% ---------------------------------------------------------
\subsubsection{The Recursion Theorem}

\begin{theorem}[Recursion on $\mathbb{N}$]\label{thm:recursion}
Let $(\mathbb{N},0,S)$ be a Peano system.
Let $X$ be a set, let $x_0 \in X$, and let $F : X \to X$ be a function.
Then there exists a unique function
\[
f : \mathbb{N} \to X
\]
such that
\[
f(0) = x_0
\quad\text{and}\quad
f(S(n)) = F(f(n)) \text{ for all } n \in \mathbb{N}.
\]
\end{theorem}

\begin{remark}
The conditions above specify $f$ completely: the value at $0$ is fixed, and
each successor value depends only on the previous one.
\end{remark}

% ---------------------------------------------------------
% Proof of Uniqueness
% ---------------------------------------------------------
\subsubsection{Uniqueness of Recursive Definitions}

\begin{proposition}[Uniqueness]\label{prop:recursion-unique}
If $f,g:\mathbb{N}\to X$ both satisfy
\[
f(0)=g(0)=x_0
\quad\text{and}\quad
f(S(n))=F(f(n)),\; g(S(n))=F(g(n))
\]
for all $n\in\mathbb{N}$, then $f=g$.
\end{proposition}

\begin{proof}
Define the set
\[
A := \{ n\in\mathbb{N} : f(n)=g(n) \}.
\]
We have $0\in A$ by assumption. If $n\in A$, then
\[
f(S(n)) = F(f(n)) = F(g(n)) = g(S(n)),
\]
so $S(n)\in A$. Thus $A$ is inductive, and by
Axiom~\ref{ax:peano5}, $A=\mathbb{N}$. Hence $f=g$.
\end{proof}

\begin{remark}
Uniqueness relies only on induction; no set-theoretic construction is required.
\end{remark}

% ---------------------------------------------------------
% Existence (Outline)
% ---------------------------------------------------------
\subsubsection{Existence (Outline)}

\begin{remark}
The proof of existence of $f$ is more subtle than uniqueness.
One approach is to define $f(n)$ as the unique value obtained by iterating
$F$ exactly $n$ times starting from $x_0$, and then to verify inductively
that this assignment is well-defined.
\end{remark}

\begin{remark}
A fully formal existence proof can be carried out using either:
\begin{itemize}
  \item induction on initial segments of $\mathbb{N}$, or
  \item the von Neumann construction of $\mathbb{N}$ together with set-theoretic recursion.
\end{itemize}
We will not pursue these details here.
\end{remark}

% ---------------------------------------------------------
% Connection to Induction
% ---------------------------------------------------------
\subsubsection{Recursion and Induction}

\begin{remark}
Induction proves that a property holds for all natural numbers.
Recursion constructs an object indexed by the natural numbers.
Both rely on the fact that $\mathbb{N}$ is the smallest inductive set.
\end{remark}

\begin{remark}
In practice, recursive definitions are justified by the recursion theorem,
while properties of recursively defined objects are established using induction.
\end{remark}
