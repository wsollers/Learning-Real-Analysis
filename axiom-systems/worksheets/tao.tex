% =========================================================
% Worksheet: Tao — Natural Numbers / Induction / Arithmetic
% File: axiom-systems/worksheets/tao.tex
% =========================================================

\subsection{Tao}

\noindent\textbf{Source.} Terence Tao, \textit{Analysis I}.

\vspace{0.75em}
\begin{center}
\begin{tabular}{|p{5.0cm}|p{9.0cm}|}
\hline
\textbf{Problem ID} & \textbf{Exercise (descriptor)} \\
\hline

% -------------------------
% Chapter 2, \S2.2 — Induction
% -------------------------

\phantomsection
\hypertarget{ws-RA-TAO-C02-S2.2-E01}{}
\hyperlink{proof-RA-TAO-C02-S2.2-E01}{\texttt{RA-TAO-C02-S2.2-E01}}
&
Tao, Chapter 2, \S2.2, Ex.~2.2.1 — Prove Proposition~2.2.5 (fix two variables; induct on the third).
\\ \hline

\phantomsection
\hypertarget{ws-RA-TAO-C02-S2.2-E02}{}
\hyperlink{proof-RA-TAO-C02-S2.2-E02}{\texttt{RA-TAO-C02-S2.2-E02}}
&
Tao, Chapter 2, \S2.2, Ex.~2.2.2 — Prove Lemma~2.2.10 (use induction).
\\ \hline

\phantomsection
\hypertarget{ws-RA-TAO-C02-S2.2-E03}{}
\hyperlink{proof-RA-TAO-C02-S2.2-E03}{\texttt{RA-TAO-C02-S2.2-E03}}
&
Tao, Chapter 2, \S2.2, Ex.~2.2.3 — Prove Proposition~2.2.12 (use preceding results).
\\ \hline

\phantomsection
\hypertarget{ws-RA-TAO-C02-S2.2-E04}{}
\hyperlink{proof-RA-TAO-C02-S2.2-E04}{\texttt{RA-TAO-C02-S2.2-E04}}
&
Tao, Chapter 2, \S2.2, Ex.~2.2.4 — Justify the three statements marked \emph{(why?)} in the proof of Proposition~2.2.13.
\\ \hline

\phantomsection
\hypertarget{ws-RA-TAO-C02-S2.2-E05}{}
\hyperlink{proof-RA-TAO-C02-S2.2-E05}{\texttt{RA-TAO-C02-S2.2-E05}}
&
Tao, Chapter 2, \S2.2, Ex.~2.2.5 — Prove Proposition~2.2.14 (hint: define $Q(n)$ via $m_0 \le m < n$; vacuity when $n<m_0$).
\\ \hline

\phantomsection
\hypertarget{ws-RA-TAO-C02-S2.2-E06}{}
\hyperlink{proof-RA-TAO-C02-S2.2-E06}{\texttt{RA-TAO-C02-S2.2-E06}}
&
Tao, Chapter 2, \S2.2, Ex.~2.2.6 — Prove the principle of backwards induction (hint: induct on $n$).
\\ \hline


% -------------------------
% Chapter 2, \S2.3 — Addition and Multiplication of Natural Numbers
% -------------------------

\phantomsection
\hypertarget{ws-RA-TAO-C02-S2.3-E01}{}
\hyperlink{proof-RA-TAO-C02-S2.3-E01}{\texttt{RA-TAO-C02-S2.3-E01}}
&
Tao, Chapter 2, \S2.3, Ex.~2.3.1 — Prove Lemma~2.3.2 (modify proofs of Lemmas~2.2.2, 2.2.3 and Proposition~2.2.4).
\\ \hline

\phantomsection
\hypertarget{ws-RA-TAO-C02-S2.3-E02}{}
\hyperlink{proof-RA-TAO-C02-S2.3-E02}{\texttt{RA-TAO-C02-S2.3-E02}}
&
Tao, Chapter 2, \S2.3, Ex.~2.3.2 — Prove Lemma~2.3.3 (hint: prove the second statement first).
\\ \hline

\phantomsection
\hypertarget{ws-RA-TAO-C02-S2.3-E03}{}
\hyperlink{proof-RA-TAO-C02-S2.3-E03}{\texttt{RA-TAO-C02-S2.3-E03}}
&
Tao, Chapter 2, \S2.3, Ex.~2.3.3 — Prove Proposition~2.3.5 (modify Proposition~2.2.5; use distributive law).
\\ \hline

\phantomsection
\hypertarget{ws-RA-TAO-C02-S2.3-E04}{}
\hyperlink{proof-RA-TAO-C02-S2.3-E04}{\texttt{RA-TAO-C02-S2.3-E04}}
&
Tao, Chapter 2, \S2.3, Ex.~2.3.4 — Prove $(a+b)^2 = a^2 + 2ab + b^2$ for all $a,b\in\mathbb{N}$.
\\ \hline

\phantomsection
\hypertarget{ws-RA-TAO-C02-S2.3-E05}{}
\hyperlink{proof-RA-TAO-C02-S2.3-E05}{\texttt{RA-TAO-C02-S2.3-E05}}
&
Tao, Chapter 2, \S2.3, Ex.~2.3.5 — Prove Proposition~2.3.9 (fix $q$ and induct on $n$).
\\ \hline

\end{tabular}
\end{center}
