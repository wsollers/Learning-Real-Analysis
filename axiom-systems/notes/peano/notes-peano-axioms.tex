% =========================================================
% The Peano Axioms (Tao §2.1)
% =========================================================

\begin{tcolorbox}[colback=propbox, colframe=propborder, arc=2pt,
  left=6pt, right=6pt, top=4pt, bottom=4pt,
  title={\small\textbf{The Peano Axioms}},
  fonttitle=\small\bfseries]
\begin{enumerate}
  \item[\textbf{P1.}] $0 \in \mathbb{N}$ \quad (zero is a natural number)
  \item[\textbf{P2.}] $n \in \mathbb{N} \Rightarrow n\pp \in \mathbb{N}$
    \quad (closure under successor)
  \item[\textbf{P3.}] $n\pp \neq 0$ for all $n \in \mathbb{N}$
    \quad (zero is not a successor)
  \item[\textbf{P4.}] $n\pp = m\pp \Rightarrow n = m$
    \quad (successor is injective)
  \item[\textbf{P5.}] \emph{Induction}: if $P(0)$ is true and
    $P(n) \Rightarrow P(n\pp)$ for all $n \in \mathbb{N}$,
    then $P(n)$ is true for all $n \in \mathbb{N}$.
\end{enumerate}
\end{tcolorbox}

\begin{remark}[Why five axioms?]
P1--P2 build $\mathbb{N}$ upward from $0$.
P3 prevents wrap-around (e.g.\ $3\pp = 0$).
P4 prevents collapse (e.g.\ $4\pp = 2\pp$ with $4 \neq 2$).
P5 excludes rogue elements (e.g.\ $0.5 \in \mathbb{N}$) by
asserting $\mathbb{N}$ is the \emph{smallest} set satisfying P1--P2.
\end{remark}

\begin{remark}[P5 is second-order]
P5 quantifies over \emph{properties} $P$, not just elements.
This makes it an axiom schema (one axiom per property $P$),
not a single first-order axiom.
\end{remark}

\begin{remark}[Axioms define structure, not objects]
The Peano axioms do not say what $0$ or $n\pp$ \emph{are} ---
they say what properties they \emph{have}.
$0$ could be the empty set $\varnothing$, or the singleton
$\{\varnothing\}$, or an abstract symbol with no set-theoretic
content. None of this matters. What matters is that the five
axioms are satisfied. Any two systems satisfying P1--P5 are
isomorphic: there is a unique structure-preserving bijection
between them. This is why we speak of \emph{the} natural numbers
rather than \emph{a} natural number system --- all valid models
are interchangeable for mathematical purposes.
This is what it means for an axiom system to be
\emph{categorical} (in second-order logic).
In first-order logic the situation is different: first-order
Peano arithmetic admits nonstandard models containing
``infinite'' natural numbers not reachable from $0$ by
successors. Second-order induction (P5 as stated) rules
these out.
\end{remark}

\begin{remark}[Tao's notation]
Tao writes $n\pp$ for the successor of $n$ (increment), following
computer-language convention. We adopt this notation throughout.
$n\pp$ and $S(n)$ refer to the same object.
\end{remark}