% --------------------------------------
% Proof: RA-TAO-C02-S2.3-E05
% --------------------------------------
\clearpage
\phantomsection
\hypertarget{proof-RA-TAO-C02-S2.3-E05}{}

\subsubsection[Euclidean Algorithm (P2.3.9)]{Proof --- RA-TAO-C02-S2.3-E05}

\noindent
\textbf{Source.}
\srccite{TaoAnalysis1}{Chapter~2, \S2.3, Exercise~2.3.5}.

\vspace{0.75em}

\noindent
\textbf{Goal.}
Prove Proposition~2.3.9 (Euclidean algorithm):

\medskip
\noindent
\textbf{Proposition~2.3.9.}
Let $n$ be a natural number, and let $q$ be a positive number. Then there
exist natural numbers $m$ and $r$ such that $0 \leq r < q$ and $n = mq + r$.

\vspace{0.75em}

\noindent
% =========================================================
% Proof: Proposition 2.3.9 — Euclidean Algorithm
% =========================================================

\begin{proposition}[Euclidean algorithm]
\label{prop:euclidean}
Let $n$ be a natural number and let $q$ be a positive natural
number. Then there exist natural numbers $m, r$ such that
$0 \leq r < q$ and $n = mq + r$.
\end{proposition}

\begin{proof}
The goal is a universal existence claim:
$\forall n \in \mathbb{N},\ \forall q \in \mathbb{N}^+,\
\exists\, m, r \in \mathbb{N}$ such that $0 \leq r < q$
and $n = mq + r$.

Fix $q$ as an arbitrary positive natural number.
We prove by induction on $n$ the statement:
\[
P(n) :\equiv \exists\, m, r \in \mathbb{N}
\text{ such that } 0 \leq r < q
\text{ and } n = mq + r.
\]

% ---------------------------------------------------------
\medskip
\noindent\textbf{Base case} ($n = 0$).

\noindent
\begin{tabular}{T S J}
\toprule
\textbf{Tag} &
\multicolumn{1}{p{0.44\textwidth}}{\textbf{Step}} &
\textbf{Justification} \\
\midrule
\addlinespace[4pt]

\tagDU
&
\text{Take } m := 0,\ r := 0
&
Exhibit witnesses. Both $0, 0 \in \mathbb{N}$.
\\[10pt]

\tagAM
&
0 = 0 \cdot q + 0
&
$0 \times q = 0$ by M1, and $0 + 0 = 0$ by A1.
\\[10pt]

\tagTA
&
0 \leq 0 < q
&
$0 \leq 0$ by reflexivity; $0 < q$ since $q$ is
positive. \hfill$\blacksquare_{P(0)}$
\\[6pt]

\bottomrule
\end{tabular}

% ---------------------------------------------------------
\medskip
\noindent\textbf{Inductive step.}
Assume $P(n)$ holds: there exist $m, r \in \mathbb{N}$
with $0 \leq r < q$ and $n = mq + r$.
We construct fresh witnesses for $P(n\pp)$.

\medskip
\noindent
Since $n = mq + r$, we have $n\pp = mq + r + 1$.
Since $r < q$ by the inductive hypothesis, $r + 1 \leq q$.
The two cases $r + 1 < q$ and $r + 1 = q$ are exhaustive.

\medskip
\noindent\textbf{Case 1:} $r + 1 < q$.

\noindent
\begin{tabular}{T S J}
\toprule
\textbf{Tag} &
\multicolumn{1}{p{0.44\textwidth}}{\textbf{Step}} &
\textbf{Justification} \\
\midrule
\addlinespace[4pt]

\tagDU
&
\text{Take } m' := m,\ r' := r + 1
&
Exhibit fresh witnesses for $P(n\pp)$.
\\[10pt]

\tagIH
&
n\pp = mq + (r + 1) = m'q + r'
&
Substituting $n = mq + r$ gives
$n\pp = mq + r + 1 = m'q + r'$.
\\[10pt]

\tagCS
&
0 \leq r' < q
&
$r' = r+1 > 0 \geq 0$; $r' = r+1 < q$
by assumption of Case~1.
\hfill$\blacksquare_{\text{Case 1}}$
\\[6pt]

\bottomrule
\end{tabular}

\medskip
\noindent\textbf{Case 2:} $r + 1 = q$.

\noindent
\begin{tabular}{T S J}
\toprule
\textbf{Tag} &
\multicolumn{1}{p{0.44\textwidth}}{\textbf{Step}} &
\textbf{Justification} \\
\midrule
\addlinespace[4pt]

\tagDU
&
\text{Take } m' := m + 1,\ r' := 0
&
Exhibit fresh witnesses for $P(n\pp)$.
\\[10pt]

\tagIH
&
n\pp = mq + q = (m+1)q
&
Substituting $r + 1 = q$:
$n\pp = mq + (r+1) = mq + q$.
Factor using distributivity P2.3.4.
\\[10pt]

\tagAM
&
(m+1)q + 0 = m'q + r'
&
$A1$: adding $0$ does not change the value.
\\[10pt]

\tagCS
&
0 \leq r' < q
&
$r' = 0 \geq 0$; $0 < q$ since $q$ is positive.
\hfill$\blacksquare_{\text{Case 2}}$
\\[6pt]

\bottomrule
\end{tabular}

\medskip
\noindent
Both cases are exhaustive ($r + 1 \leq q$ from IH, splitting
into $r+1 < q$ and $r+1 = q$). In each case fresh witnesses
$m', r'$ for $P(n\pp)$ were constructed. By induction (P5),
$P(n)$ holds for all $n \in \mathbb{N}$. Since $q$ was an
arbitrary positive natural number, the result holds for all
such $q$. \hfill$\blacksquare$

\end{proof}

\begin{remark}[Fresh witnesses per inductive step]
The witnesses $m, r$ from the inductive hypothesis are valid
only for $P(n)$. Proving $P(n\pp)$ requires constructing new
witnesses $m', r'$ from scratch, using $m$ and $r$ as raw
material. Each application of the inductive step opens a new
existential context — analogous to a new stack frame in which
local variables are freshly instantiated.
\end{remark}

\begin{remark}[Case split is forced by the remainder bound]
The IH gives $r < q$, so $r + 1 \leq q$. The case split on
whether $r + 1 < q$ or $r + 1 = q$ is exhaustive by
trichotomy (Proposition~2.2.13). Case~1 increments the
remainder; Case~2 resets the remainder to $0$ and carries
into the quotient. This is exactly the arithmetic of division
with remainder.
\end{remark}

\begin{remark}[This proof seeds number theory]
The Euclidean algorithm is the foundation of divisibility
theory, the gcd, and Bézout's identity. Every result in
elementary number theory that involves remainders ultimately
rests on this proposition.
\end{remark}

\vspace{1em}
