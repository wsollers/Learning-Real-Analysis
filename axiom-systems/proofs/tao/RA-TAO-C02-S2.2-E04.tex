% --------------------------------------
% Proof: RA-TAO-C02-S2.2-E04
% --------------------------------------
\clearpage
\phantomsection
\hypertarget{proof-RA-TAO-C02-S2.2-E04}{}

\section*{Proof --- RA-TAO-C02-S2.2-E04}

\noindent
\hyperlink{ws-RA-TAO-C02-S2.2-E04}{\textbf{← Back to worksheet}}

\noindent
\textbf{Source.}
\srccite{TaoAnalysis1}{Chapter~2, \S2.2, Exercise~2.2.4}.

\vspace{0.75em}

\noindent
\textbf{Goal.}
Justify each of the three statements marked \emph{(why?)} in Tao's
sketch proof of Proposition~2.2.13 (Trichotomy of order for natural
numbers). The three gaps are:

\medskip
\begin{enumerate}
  \item[\textbf{(i)}] \textbf{Base case} ($a = 0$, $b$ fixed).
    Tao claims: $0 \leq b$ for all $b \in \mathbb{N}$. \emph{(why?)}

  \item[\textbf{(ii)}] \textbf{Inductive step, case $a > b$.}
    Tao claims: if $a > b$, then $a\pp > b$. \emph{(why?)}

  \item[\textbf{(iii)}] \textbf{Inductive step, case $a = b$.}
    Tao claims: if $a = b$, then $a\pp > b$. \emph{(why?)}
\end{enumerate}

\medskip
\noindent
For each item give a complete justification using only
Definition~2.2.11, Proposition~2.2.12, and the Peano axioms.
Do not appeal to trichotomy itself.

\vspace{0.75em}

\noindent
\textbf{Proof.}

\vspace{1em}
\noindent
\hyperlink{ws-RA-TAO-C02-S2.2-E04}{\textbf{← Back to worksheet}}
