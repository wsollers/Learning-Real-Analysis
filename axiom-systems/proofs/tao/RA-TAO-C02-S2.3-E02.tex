% --------------------------------------
% Proof: RA-TAO-C02-S2.3-E02
% --------------------------------------
\clearpage
\phantomsection
\hypertarget{proof-RA-TAO-C02-S2.3-E02}{}

\section*{Proof --- RA-TAO-C02-S2.3-E02}

\noindent
\hyperlink{ws-RA-TAO-C02-S2.3-E02}{\textbf{← Back to worksheet}}

\noindent
\textbf{Source.}
\srccite{TaoAnalysis1}{Chapter~2, \S2.3, Exercise~2.3.2}.

\vspace{0.75em}

\noindent
\textbf{Goal.}
Prove Lemma~2.3.3 (Positive natural numbers have no zero divisors):

\medskip
\noindent
\textbf{Lemma~2.3.3.}
Let $n, m$ be natural numbers. Then $n \times m = 0$ if and only if at
least one of $n$, $m$ is equal to zero. In particular, if $n$ and $m$ are
both positive, then $nm$ is also positive.

\vspace{0.75em}

\noindent
% =========================================================
% Proof: Lemma 2.3.3 — No Zero Divisors
% =========================================================

\begin{lemma}[No zero divisors]
\label{lem:no-zero-divisors}
Let $n, m$ be natural numbers. Then $n \times m = 0$ if and only if
at least one of $n, m$ is equal to zero. In particular, if $n$ and $m$
are both positive, then $nm$ is also positive.
\end{lemma}

\begin{proof}
We prove both directions of the biconditional separately.
Per Tao's hint, we prove the harder direction $(\Rightarrow)$ first,
via its contrapositive.

% ---------------------------------------------------------
\medskip
\noindent\textbf{Part 1: $(\Rightarrow)$ direction (via contrapositive).}

We prove by induction on $n$ the statement:
\[
P(n) :\equiv \text{if } n > 0 \text{ and } m > 0
\text{ then } n \times m > 0.
\]

\medskip
\noindent\textbf{Base case} ($n = 0$).
$P(0)$ states: if $0 > 0$ and $m > 0$ then $0 \times m > 0$.
The hypothesis $0 > 0$ is false, so $P(0)$ is vacuously true.

\medskip
\noindent\textbf{Inductive step.}
Assume $P(n)$ holds. We show $P(n\pp)$.

$P(n\pp)$ states: if $n\pp > 0$ and $m > 0$ then $(n\pp) \times m > 0$.

\noindent
\begin{tabular}{T S J}
\toprule
\textbf{Tag} &
\multicolumn{1}{p{0.44\textwidth}}{\textbf{Step}} &
\textbf{Justification} \\
\midrule
\addlinespace[4pt]

\tagTA
&
n\pp > 0
&
By P3: $n\pp \neq 0$ for all $n \in \mathbb{N}$, and in
$\mathbb{N}$ nonzero means positive. The first hypothesis
of $P(n\pp)$ is automatically satisfied; the live hypothesis
is $m > 0$.
\\[10pt]

\tagDU
&
(n\pp) \times m = (n \times m) + m
&
Definition of multiplication M2:
$(n\pp) \times m := (n \times m) + m$.
\\[10pt]

\tagIH
&
n \times m \geq 0
&
$n \times m \in \mathbb{N}$, so $n \times m \geq 0$.
(The inductive hypothesis $P(n)$ gives $nm > 0$ when
$n > 0$; when $n = 0$, $nm = 0$. Either way $nm \geq 0$.)
\\[10pt]

\tagTA
&
(n \times m) + m > 0
&
Proposition~2.2.8: if $b > 0$ and $a \in \mathbb{N}$
then $a + b > 0$. Apply with $a := nm \geq 0$
and $b := m > 0$.
\\[10pt]

\tagAM
&
(n\pp) \times m > 0
&
Substituting M2. \hfill$\blacksquare_{P(n\pp)}$
\\[6pt]

\bottomrule
\end{tabular}

\medskip
\noindent
By induction (P5), $P(n)$ holds for all $n \in \mathbb{N}$.
Thus if $n > 0$ and $m > 0$ then $nm > 0$, i.e., $nm \neq 0$.
Taking the contrapositive: if $nm = 0$ then $n = 0$ or $m = 0$.
\hfill$\blacksquare_{(\Rightarrow)}$

% ---------------------------------------------------------
\medskip
\noindent\textbf{Part 2: $(\Leftarrow)$ direction.}

We prove: if $n = 0$ or $m = 0$, then $nm = 0$.
We split into two cases.

\noindent
\begin{tabular}{T S J}
\toprule
\textbf{Tag} &
\multicolumn{1}{p{0.44\textwidth}}{\textbf{Step}} &
\textbf{Justification} \\
\midrule
\addlinespace[4pt]

\tagDU
&
\textbf{Case 1}: \text{assume } n = 0
&
First disjunct of the hypothesis.
\\[10pt]

\tagDU
&
0 \times m = 0
&
Definition of multiplication M1: $0 \times m := 0$.
\\[10pt]

\tagDU
&
\textbf{Case 2}: \text{assume } m = 0
&
Second disjunct of the hypothesis.
\\[10pt]

\tagTA
&
n \times 0 = 0 \times n = 0
&
Commutativity (Lemma~2.3.2), then M1.
\hfill$\blacksquare_{(\Leftarrow)}$
\\[6pt]

\bottomrule
\end{tabular}

\medskip
\noindent
Both directions established. Therefore
$nm = 0 \iff n = 0 \text{ or } m = 0$.

\medskip
\noindent
The ``in particular'' clause is immediate: if $n > 0$ and $m > 0$
then neither is zero, so by $(\Rightarrow)$ we have $nm \neq 0$,
i.e., $nm > 0$.

\end{proof}

\begin{remark}[Contrapositive as the proof engine]
The direction $nm = 0 \Rightarrow n = 0$ or $m = 0$ is proved via
its contrapositive: if $n > 0$ and $m > 0$ then $nm > 0$.
This is the natural choice because the negation of the conclusion
is more tractable than the hypothesis $nm = 0$ alone.
\end{remark}

\begin{remark}[Vacuous base case]
The base case $P(0)$ is vacuously true because the hypothesis
$n > 0$ fails at $n = 0$. This is the same pattern as Lemma~2.2.10.
A vacuous base case is logically valid and requires no further argument;
it appears whenever the claim carries a positivity precondition.
\end{remark}

\begin{remark}[Proposition 2.2.8 closes the inductive step]
The key tool in the inductive step is not the inductive hypothesis
directly but Proposition~2.2.8: any natural number plus a positive
natural number is positive. This fires because $m > 0$ is in scope
throughout the induction. The inductive hypothesis contributes
$nm \geq 0$, which is all that is needed.
\end{remark}

\begin{remark}[The ``in particular'' clause is free]
The positivity statement is not proved separately --- it is exactly
what Part~1 establishes. Tao attaches it as a named consequence
to make it available by citation in later results, notably
Proposition~2.3.6 (multiplication preserves order).
\end{remark}
\vspace{1em}
\noindent
\hyperlink{ws-RA-TAO-C02-S2.3-E02}{\textbf{← Back to worksheet}}
