% --------------------------------------
% Proof: RA-TAO-C02-S2.3-E04
% --------------------------------------
\clearpage
\phantomsection
\hypertarget{proof-RA-TAO-C02-S2.3-E04}{}

\section*{Proof --- RA-TAO-C02-S2.3-E04}

\noindent
\hyperlink{ws-RA-TAO-C02-S2.3-E04}{\textbf{← Back to worksheet}}

\noindent
\textbf{Source.}
\srccite{TaoAnalysis1}{Chapter~2, \S2.3, Exercise~2.3.4}.

\vspace{0.75em}

\noindent
\textbf{Goal.}
Prove the following identity for all $a, b \in \mathbb{N}$:
\[
(a + b)^2 \;=\; a^2 + 2ab + b^2.
\]
Here exponentiation is as defined in Definition~2.3.11, and $2ab$ denotes
$ab + ab$. You may use all results from \S2.2 and \S2.3 without further
justification.

\vspace{0.75em}

\noindent
% =========================================================
% Proof: Exercise 2.3.4 — Binomial Square Identity
% =========================================================

\begin{proposition}[Binomial square identity]
\label{prop:binomial-square}
For all natural numbers $a, b$:
\[
(a + b)^2 = a^2 + 2ab + b^2.
\]
\end{proposition}

\begin{proof}
The goal is a universal equality. All tools from Chapter~2 are
available: Definitions 2.3.1 (multiplication), 2.3.11
(exponentiation), commutativity (L2.3.2), distributivity
(P2.3.4), and associativity (P2.3.5).
No induction is required; the proof is pure algebraic
manipulation.

\medskip
\noindent
\begin{tabular}{T S J}
\toprule
\textbf{Tag} &
\multicolumn{1}{p{0.44\textwidth}}{\textbf{Step}} &
\textbf{Justification} \\
\midrule
\addlinespace[4pt]

\tagDU
&
(a + b)^2 = (a + b) \times (a + b)
&
Definition~2.3.11: $m^2 = m^1 \times m = m^0 \times m \times m
= 1 \times m \times m = m \times m$.
Apply with $m := a + b$.
\\[10pt]

\tagTA
&
(a + b) \times (a + b)
= a \times (a + b) + b \times (a + b)
&
Distributivity P2.3.4: $(\alpha + \beta) \times \gamma
= \alpha \times \gamma + \beta \times \gamma$.
Apply with $\alpha := a$, $\beta := b$,
$\gamma := (a + b)$.
\\[10pt]

\tagTA
&
a \times (a + b) = a \times a + a \times b
&
Distributivity P2.3.4: $\alpha \times (\beta + \gamma)
= \alpha \times \beta + \alpha \times \gamma$.
Apply with $\alpha := a$, $\beta := a$, $\gamma := b$.
\\[10pt]

\tagTA
&
b \times (a + b) = b \times a + b \times b
&
Distributivity P2.3.4, apply with $\alpha := b$,
$\beta := a$, $\gamma := b$.
\\[10pt]

\tagAM
&
(a+b)^2
= a \times a + a \times b + b \times a + b \times b
&
Combining the two distributivity expansions.
\\[10pt]

\tagTA
&
b \times a = a \times b
&
Commutativity L2.3.2.
\\[10pt]

\tagAM
&
(a+b)^2
= a \times a + a \times b + a \times b + b \times b
&
Substituting $b \times a = a \times b$.
\\[10pt]

\tagDU
&
a \times b + a \times b = 2 \times (a \times b)
&
Definition~2.3.1: $2 \times m := (1\pp) \times m
= (1 \times m) + m = m + m$.
Apply with $m := a \times b$.
\\[10pt]

\tagDU
&
a \times a = a^2, \quad b \times b = b^2
&
Definition~2.3.11: $m^2 = m \times m$.
\\[10pt]

\tagAM
&
(a + b)^2 = a^2 + 2ab + b^2
&
Substituting all three simplifications.
\hfill$\blacksquare$
\\[6pt]

\bottomrule
\end{tabular}

\end{proof}

\begin{remark}[No induction required]
Unlike the preceding results in Chapter~2, this proof requires
no induction. All tools are already available as proved lemmas
and definitions. The proof reduces entirely to definition
unpacking followed by repeated application of distributivity
and commutativity.
When all operations involved have known algebraic laws, reach
for direct manipulation before reaching for induction.
\end{remark}

\begin{remark}[$2$ is a defined constant, not assumed]
In Tao's system, $2 := 1\pp = 0\pp\pp$. The identity
$ab + ab = 2 \times ab$ is not assumed from arithmetic
intuition but follows directly from Definition~2.3.1:
$2 \times m = (1\pp) \times m = (1 \times m) + m = m + m$.
Every numerical constant must be grounded in the Peano
construction until the full arithmetic toolkit is available.
\end{remark}

\begin{remark}[Distributivity fires twice]
The core move is applying distributivity twice: first to
split $(a+b)(a+b)$ into two terms, then to expand each term.
This double application of P2.3.4 is the standard engine
for expanding products of sums in $\mathbb{N}$.
\end{remark}

\vspace{1em}
\noindent
\hyperlink{ws-RA-TAO-C02-S2.3-E04}{\textbf{← Back to worksheet}}
