% =========================================================
% Proof Sheets — Tao (Analysis I) — Induction / Natural Numbers
% Sections 2.2 and 2.3
% =========================================================

% --------------------------------------
% Proof: RA-TAO-C02-S2.2-E01
% --------------------------------------
\clearpage
\phantomsection
\hypertarget{proof-RA-TAO-C02-S2.2-E01}{}

\section*{Proof — RA-TAO-C02-S2.2-E01}

\noindent
\hyperlink{ws-RA-TAO-C02-S2.2-E01}{\textbf{← Back to worksheet}}

\noindent
\textbf{Source.}
\srccite{TaoAnalysis1}{Chapter~2, \S2.2, Exercise~2.2.1}.

\vspace{0.75em}

\noindent
\textbf{Goal.}
Prove Tao's Proposition~2.2.5 (Addition is associative), namely: for all
$a,b,c\in\mathbb{N}$,
\[
(a+b)+c \;=\; a+(b+c).
\]
\emph{(Hint: use induction.)}

\vspace{0.75em}

\noindent
% =========================================================
% Proof: Addition is Associative (Tao Prop. 2.2.5)
% =========================================================

\begin{proposition*}[Addition is associative]
\label{prop:add-assoc}
For all $a, b, c \in \mathbb{N}$,
\[
(a + b) + c \;=\; a + (b + c).
\]
\end{proposition*}

\begin{proof}
Let $b, c \in \mathbb{N}$ be arbitrary and fixed throughout.
We induct on $a$.

Let $P(a)$ denote the statement $(a + b) + c = a + (b + c)$.

\medskip
\noindent\textbf{Base case} ($a = 0$). We show $P(0)$:
$(0 + b) + c = 0 + (b + c)$.

\noindent
\begin{tabular}{T S J}
\toprule
\textbf{Tag} & \multicolumn{1}{p{0.44\textwidth}}{\textbf{Step}} & \textbf{Justification} \\
\midrule
\addlinespace[4pt]

\tagDU
&
(0 + b) + c = b + c
&
Apply A1 to the inner sum: $0 + b = b$.
\\[10pt]

\tagDU
&
0 + (b + c) = b + c
&
Apply A1 to the outer sum: $0 + (b+c) = b+c$.
\\[10pt]

\tagAM
&
(0 + b) + c \;=\; 0 + (b + c)
&
Both sides equal $b + c$. Hence $P(0)$ holds.
\hfill$\blacksquare_0$
\\[6pt]

\bottomrule
\end{tabular}

\medskip
\noindent\textbf{Inductive step.}
Assume $P(a)$: $(a + b) + c = a + (b + c)$. \quad
\textit{(Inductive hypothesis.)}

We show $P(a\pp)$: $(a\pp + b) + c = a\pp + (b + c)$.

\noindent
\begin{tabular}{T S J}
\toprule
\textbf{Tag} & \multicolumn{1}{p{0.44\textwidth}}{\textbf{Step}} & \textbf{Justification} \\
\midrule
\addlinespace[4pt]

\tagDU
&
a\pp + b = (a + b)\pp
&
Apply A2 to the left argument $a\pp$:
$(a\pp) + b := (a + b)\pp$.
\\[10pt]

\tagAM
&
(a\pp + b) + c = (a+b)\pp + c
&
Substituting the result above into the left side of $P(a\pp)$.
\\[10pt]

\tagDU
&
(a+b)\pp + c = ((a+b) + c)\pp
&
Apply A2 with left argument $(a+b)\pp$:
$((a+b)\pp) + c := ((a+b)+c)\pp$.
\\[10pt]

\tagTA
&
((a+b)+c)\pp = (a+(b+c))\pp
&
Apply inductive hypothesis $P(a)$:
$(a+b)+c = a+(b+c)$.
Substitute inside the successor.
\\[10pt]

\tagDU
&
a\pp + (b+c) = (a + (b+c))\pp
&
Apply A2 with left argument $a\pp$:
$(a\pp) + (b+c) := (a+(b+c))\pp$.
\\[10pt]

\tagAM
&
(a\pp + b) + c \;=\; a\pp + (b+c)
&
Left side $= (a+(b+c))\pp$ (steps 1--4).
Right side $= (a+(b+c))\pp$ (step 5).
Both sides equal. Hence $P(a\pp)$ holds.
\hfill$\blacksquare$
\\[6pt]

\bottomrule
\end{tabular}

\medskip
By induction (P5), $P(a)$ holds for all $a \in \mathbb{N}$.
\end{proof}

\begin{remark}[Strategy: induction on the left argument]
Addition in $\mathbb{N}$ is defined by recursion on the
\emph{left} argument (Rule A2). When proving an identity
involving addition, always ask which variable sits in the
left position of the outermost addition on both sides.
That variable is the one to induct on, because A2 will
fire cleanly on both sides simultaneously.

Here, both $(a+b)+c$ and $a+(b+c)$ have $a$ (or $a\pp$)
in the left position of their outermost or inner sums,
making $a$ the correct induction variable.
Inducting on $b$ or $c$ would leave A2 unable to simplify
one or both sides.
\end{remark}

\begin{remark}[Strategy: inductive hypothesis as a cite]
The inductive hypothesis $P(a)$ is not used to simplify
either side directly --- it is used as a \emph{theorem citation}
at the moment when both sides have been reduced to successors
of expressions that differ only in grouping.
This is a common pattern: reduce both sides independently
using definitions (A1, A2), then close the gap with the
inductive hypothesis tagged \tagTA.
\end{remark}

\vspace{1em}
\noindent
\hyperlink{ws-RA-TAO-C02-S2.2-E01}{\textbf{← Back to worksheet}}

