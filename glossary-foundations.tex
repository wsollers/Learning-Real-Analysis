\chapter*{Glossary of Foundational Terms}
\addcontentsline{toc}{chapter}{Glossary of Foundational Terms}

% ======================================
% Core Logical Concepts
% ======================================

\begin{definition}[Logic]
\emph{Logic} is the formal study of valid reasoning, including the structure
of statements and the rules by which conclusions may be derived from
assumptions.
\end{definition}

\begin{definition}[Deduction]
\emph{Deduction} is the process of deriving conclusions from given assumptions
by the application of accepted rules of reasoning.
\end{definition}

\begin{definition}[Inference]
An \emph{inference} is a single application of a rule of reasoning that produces
a new statement from one or more given statements.
\end{definition}

\begin{definition}[Mathematical Axiom]
A \emph{mathematical axiom} is a statement assumed to be true without proof
and used as a foundational starting point for logical deduction within a
given mathematical theory.
\end{definition}

\begin{definition}[Proof]
A \emph{proof} is a finite sequence of statements, each of which is either a
premise, an assumption, or follows from earlier statements by a permitted
rule of inference, such that the final statement is the desired conclusion.
\end{definition}

\begin{definition}[Derivation]
A \emph{derivation} is an ordered sequence of formulas constructed according
to the rules of a formal system, where each formula is either a premise, an
assumption, or is obtained from previous formulas by a rule of inference.
\end{definition}

\begin{definition}[Assumption]
An \emph{assumption} is a statement temporarily introduced in a proof for the
purpose of deriving further statements. Assumptions must be properly
discharged before a proof is completed.
\end{definition}

% ======================================
% Propositions and Truth
% ======================================

\begin{definition}[Proposition]
A \emph{proposition} is a declarative statement that has a definite truth value,
namely true or false.
\end{definition}

\begin{definition}[Truth Value]
The \emph{truth value} of a proposition is the value \emph{true} or \emph{false}
assigned to it under a given interpretation.
\end{definition}

\begin{definition}[Truth-Functional]
A connective is \emph{truth-functional} if the truth value of the compound
proposition it forms depends only on the truth values of its components.
\end{definition}

\begin{definition}[Tautology]
A \emph{tautology} is a proposition that is true under every possible
interpretation of its sentential variables.
\end{definition}

% ======================================
% Syntax and Formulas
% ======================================

\begin{definition}[Formula]
A \emph{formula} is a finite symbolic expression constructed from atomic
formulas using logical connectives according to the formation rules of a
logical language.
\end{definition}

\begin{definition}[Well-Formed Formula]
A \emph{well-formed formula} (WFF) is a formula constructed in strict accordance
with the formation rules of a logical language.
\end{definition}

\begin{definition}[Sentential Variable]
A \emph{sentential variable} is a symbol that stands for an unspecified
proposition.
\end{definition}

\begin{definition}[Sentential Form]
A \emph{sentential form} is a symbolic expression composed of sentential
variables and logical connectives.
\end{definition}

% ======================================
% Logical Connectives
% ======================================

\begin{definition}[Connective]
A \emph{connective} is a logical operator that combines propositions to form
compound propositions.
\end{definition}

\begin{definition}[Negation]
A \emph{negation} is a connective that reverses the truth value of a
proposition.
\end{definition}

\begin{definition}[Disjunction]
A \emph{disjunction} is a connective that forms a proposition true if at least
one of its component propositions is true.
\end{definition}

\begin{definition}[Conditional Proposition]
A \emph{conditional proposition} is a compound proposition of the form
``if $P$, then $Q$''.
\end{definition}

\begin{definition}[Material Implication]
\emph{Material implication} is the truth-functional connective corresponding
to conditional propositions, false only when the antecedent is true and the
consequent is false.
\end{definition}

\begin{definition}[Antecedent]
The \emph{antecedent} is the proposition following ``if'' in a conditional.
\end{definition}

\begin{definition}[Consequent]
The \emph{consequent} is the proposition following ``then'' in a conditional.
\end{definition}

\begin{definition}[Converse]
The \emph{converse} of $P \rightarrow Q$ is $Q \rightarrow P$.
\end{definition}

\begin{definition}[Inverse]
The \emph{inverse} of $P \rightarrow Q$ is $\neg P \rightarrow \neg Q$.
\end{definition}

\begin{definition}[Contrapositive]
The \emph{contrapositive} of $P \rightarrow Q$ is $\neg Q \rightarrow \neg P$,
and is logically equivalent to the original conditional.
\end{definition}

\begin{definition}[Biconditional Proposition]
A \emph{biconditional proposition} asserts that two propositions have the same
truth value.
\end{definition}

\begin{definition}[Material Equivalence]
\emph{Material equivalence} is the truth-functional connective corresponding
to biconditional propositions.
\end{definition}

\begin{definition}[Equivalent Propositions]
Two propositions are \emph{equivalent} if they have the same truth value under
all interpretations.
\end{definition}

% ======================================
% Arguments and Validity
% ======================================

\begin{definition}[Argument]
An \emph{argument} consists of premises together with a conclusion claimed to
follow from them.
\end{definition}

\begin{definition}[Valid Argument]
An argument is \emph{valid} if there is no interpretation under which all
premises are true and the conclusion is false.
\end{definition}

\begin{definition}[Formal Derivation]
A \emph{formal derivation} is a syntactic sequence of formulas produced by
rules of inference from premises and assumptions.
\end{definition}

% ======================================
% Rules of Inference
% ======================================

\begin{definition}[Modus Ponens]
(Modus Ponens; also called the \emph{Law of Detachment})  
From $P \rightarrow Q$ and $P$, infer $Q$.
\end{definition}

\begin{definition}[Modus Tollens]
From $P \rightarrow Q$ and $\neg Q$, infer $\neg P$.
\end{definition}

\begin{definition}[Hypothetical Syllogism]
From $P \rightarrow Q$ and $Q \rightarrow R$, infer $P \rightarrow R$.
\end{definition}

\begin{definition}[Disjunctive Syllogism]
(Also called \emph{modus tollendo ponens})  
From $P \vee Q$ and $\neg P$, infer $Q$ (and symmetrically).
\end{definition}

\begin{definition}[Addition]
(Disjunction Introduction)  
From $P$, infer $P \vee Q$.
\end{definition}

\begin{definition}[Simplification]
(Conjunction Elimination)  
From $P \wedge Q$, infer $P$ (or $Q$).
\end{definition}

\begin{definition}[Adjunction]
(Conjunction Introduction)  
From $P$ and $Q$, infer $P \wedge Q$.
\end{definition}

\begin{definition}[Double Negation]
From $\neg\neg P$, infer $P$, and conversely.
\end{definition}

\begin{definition}[Disjunction Elimination]
(Proof by Cases)  
If $P \vee Q$ and $R$ follows from each of $P$ and $Q$, infer $R$.
\end{definition}

\begin{definition}[Conditional Proof]
If assuming $P$ leads to $Q$, infer $P \rightarrow Q$.
\end{definition}

\begin{definition}[Indirect Proof]
(Reductio ad Absurdum)  
If assuming $\neg P$ leads to a contradiction, infer $P$.
\end{definition}
