\chapter*{Glossary of Foundational Terms}
\addcontentsline{toc}{chapter}{Glossary of Foundational Terms}

\begin{definition}[Logic]
\emph{Logic} is the formal study of valid reasoning, including the structure
of statements and the rules by which conclusions may be derived from
assumptions.
\end{definition}

\begin{definition}[Deduction]
\emph{Deduction} is the process of deriving conclusions from given assumptions
by the application of accepted rules of reasoning.
\end{definition}

\begin{definition}[Inference]
An \emph{inference} is a single application of a rule of reasoning that produces
a new statement from one or more given statements.
\end{definition}

\begin{definition}[Mathematical Axiom]
A \emph{mathematical axiom} is a statement assumed to be true without proof
and used as a foundational starting point for logical deduction within a
given mathematical theory.
\end{definition}

\begin{definition}[Proof]
A \emph{proof} is a finite sequence of logically valid inferences that establishes
the truth of a mathematical statement from axioms, definitions, and previously
proved results.
\end{definition}

\begin{definition}[Theorem]
A \emph{theorem} is a mathematical statement whose truth is established by
proof within a given axiomatic system.
\end{definition}

\begin{definition}[Proposition]
A \emph{proposition} is a declarative statement that has a definite truth value,
namely true or false.
\end{definition}

\begin{definition}[Truth Value]
The \emph{truth value} of a proposition is the value \emph{true} or \emph{false}
assigned to it within a given interpretation.
\end{definition}

\begin{definition}[Truth-Functional]
An operator or connective is \emph{truth-functional} if the truth value of the
compound proposition it forms depends only on the truth values of its
component propositions.
\end{definition}

\begin{definition}[Connective]
A \emph{connective} is a logical operator that combines propositions to form
a compound proposition, typically determining its truth value from those of
its components.
\end{definition}

\begin{definition}[Disjunction]
A \emph{disjunction} is a connective that forms a compound proposition which
is true if at least one of its component propositions is true.
\end{definition}

\begin{definition}[Disjunction]
A \emph{disjunction} is a connective that forms a compound proposition which
is true if at least one of its component propositions is true.
\end{definition}

\begin{definition}[Negation]
A \emph{negation} is a connective that forms a proposition whose truth value
is the opposite of the truth value of the original proposition.
\end{definition}

\begin{definition}[Statement Form]
A \emph{statement form} is an expression built from sentential variables and
logical connectives that becomes a proposition when its variables are replaced
by propositions.
\end{definition}

\begin{definition}[Sentential Form]
A \emph{sentential form} is a symbolic expression composed of sentential
variables and logical connectives, representing the structure of a proposition.
\end{definition}

\begin{definition}[Sentential Variable]
A \emph{sentential variable} is a symbol that stands for an unspecified
proposition and may take either truth value.
\end{definition}

\begin{definition}[Conditional Proposition]
A \emph{conditional proposition} is a compound proposition of the form
``if $P$, then $Q$,'' expressing that the truth of $Q$ is asserted whenever
$P$ is true.
\end{definition}

\begin{definition}[Material Implication]
\emph{Material implication} is the truth-functional connective that forms a
conditional proposition which is false only when the antecedent is true and
the consequent is false.
\end{definition}

\begin{definition}[Antecedent]
The \emph{antecedent} of a conditional proposition is the proposition that
follows ``if'' and precedes ``then.''
\end{definition}

\begin{definition}[Hypothesis]
A \emph{hypothesis} is an assumed proposition from which a conclusion is
derived, often appearing as the antecedent of a conditional statement.
\end{definition}

\begin{definition}[Consequent]
The \emph{consequent} of a conditional proposition is the proposition that
follows ``then.''
\end{definition}

\begin{definition}[Conclusion]
A \emph{conclusion} is a proposition that is logically derived from given
assumptions or hypotheses.
\end{definition}

\begin{definition}[Necessary]
A condition $P$ is \emph{necessary} for a condition $Q$ if $Q$ cannot be true
unless $P$ is true.
\end{definition}

\begin{definition}[Sufficient]
A condition $P$ is \emph{sufficient} for a condition $Q$ if the truth of $P$
guarantees the truth of $Q$.
\end{definition}

\begin{definition}[Necessary Condition]
A \emph{necessary condition} for a statement $Q$ is a condition that must hold
whenever $Q$ is true.
\end{definition}

\begin{definition}[Sufficient Condition]
A \emph{sufficient condition} for a statement $Q$ is a condition whose truth
ensures that $Q$ is true.
\end{definition}

\begin{definition}[Converse]
The \emph{converse} of a conditional proposition ``if $P$, then $Q$'' is the
conditional proposition ``if $Q$, then $P$.''
\end{definition}

\begin{definition}[Biconditional Proposition]
A \emph{biconditional proposition} is a compound proposition of the form
``$P$ if and only if $Q$,'' asserting that $P$ and $Q$ have the same truth value.
\end{definition}

\begin{definition}[Material Equivalence]
\emph{Material equivalence} is the truth-functional connective that forms a
biconditional proposition which is true when both component propositions have
the same truth value.
\end{definition}

\begin{definition}[Equivalent Propositions]
Two propositions are \emph{equivalent} if they have the same truth value under
every possible interpretation.
\end{definition}

\begin{definition}[Tautology]
A \emph{tautology} is a proposition that is true under every possible
interpretation of its sentential variables.
\end{definition}

\begin{definition}[Implication]
\emph{Implication} is the logical relationship expressed by a conditional
statement of the form ``if $P$, then $Q$,'' indicating that the truth of $Q$
is claimed to follow from the truth of $P$.
\end{definition}

\begin{definition}[Formula]
A \emph{formula} is a finite symbolic expression constructed from atomic
formulas using logical connectives according to the formation rules of a
given logical language.
\end{definition}

\begin{definition}[Well-Formed Formula]
A \emph{well-formed formula} (WFF) is a formula that is constructed in strict
accordance with the formation rules of a given logical language, so that its
syntactic structure is unambiguous and grammatically correct.
\end{definition}

\begin{definition}{Modus Ponens}
Modus Ponens is a rule of inference stating that from a conditional proposition
$P \rightarrow Q$ and its antecedent $P$, one may validly infer the consequent $Q$.
Symbolically:
\[
P \rightarrow Q,\quad P \;\vdash\; Q.
\]
Modus ponens is truth-preserving and is a fundamental rule in propositional logic,
predicate logic, and formal proof systems.
\end{definition}


\begin{definition}{Modus Tollens}
Modus Tollens is a rule of inference stating that from a conditional proposition
$P \rightarrow Q$ and the negation of its consequent $\neg Q$, one may validly infer
the negation of the antecedent $\neg P$.
Symbolically:
\[
\begin{array}{ll}
\text{Premise 1.} & P \rightarrow Q \\
\text{Premise 2.} & \neg Q \\
\hline
\text{Conclusion.} & \neg P
\end{array}
\]
\end{definition}

\begin{definition}{Law of Premises}
The Law of Premises is the principle that any statement explicitly assumed
as a premise in a proof may be asserted as true at any subsequent step of that proof,
provided it has not been discharged. Premises serve as the foundational assumptions
from which conclusions are derived.
\end{definition}

\begin{definition}{Law of Detachment}
The Law of Detachment is a rule of inference stating that if a conditional
proposition $P \rightarrow Q$ is true and its antecedent $P$ is true, then the
consequent $Q$ may be validly inferred. The law of detachment is equivalent to
modus ponens.
\end{definition}







