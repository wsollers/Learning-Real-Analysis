% =========================================================
% When Are $\varepsilon$-Arguments Required for Suprema?
% =========================================================

\subsection{When Are $\varepsilon$-Arguments Required?}

A recurring source of confusion in real analysis is determining when the
$\varepsilon$-characterization of supremum or infimum is logically necessary,
and when a proof can be completed using order and bounds alone.
This section provides a structural guide for making that decision.

\subsubsection{Two Equivalent Definitions of Supremum}

Let $S \subseteq \mathbb{R}$ be nonempty and bounded above.
A real number $\alpha$ is the supremum of $S$ if and only if one of the
following equivalent characterizations holds.

\paragraph{Order-theoretic definition.}
$\alpha = \sup S$ if:
\begin{enumerate}
  \item $\alpha$ is an upper bound of $S$, and
  \item for every upper bound $u$ of $S$, $\alpha \le u$.
\end{enumerate}

\paragraph{$\varepsilon$-approximation definition.}
$\alpha = \sup S$ if:
\begin{enumerate}
  \item $\alpha$ is an upper bound of $S$, and
  \item for every $\varepsilon > 0$, there exists $s \in S$ such that
  $\alpha - \varepsilon < s$.
\end{enumerate}

Both definitions are equivalent in $\mathbb{R}$, but they are not
equally convenient in all proofs.

\subsubsection{The Central Distinction}

The need for an $\varepsilon$-argument depends on whether
\emph{leastness of the bound can be established by direct comparison}
or whether it must be established by \emph{approximation from within the set}.

\begin{quote}
\textbf{Key Principle.}
$\varepsilon$-arguments are required precisely when leastness of a bound
cannot be proved by comparing it directly to all other bounds.
\end{quote}

\subsubsection{When $\varepsilon$ Is \emph{Not} Required}

An $\varepsilon$-argument is \emph{not} required when the situation allows
direct comparison of bounds.
This occurs when the operation under consideration induces a
\emph{direct correspondence between bounds}.

Typical cases include:
\begin{itemize}
  \item Translation: $f(x) = x + t$
  \item Positive scaling: $f(x) = cx$ with $c > 0$
  \item Reflection: $f(x) = -x$
\end{itemize}

In these cases:
\begin{itemize}
  \item Upper (or lower) bounds of $S$ correspond exactly to bounds of $f(S)$.
  \item Leastness can be shown by comparing arbitrary bounds via the inverse map.
  \item The order-theoretic definition of supremum or infimum suffices.
\end{itemize}

\subsubsection{When $\varepsilon$ \emph{Is} Required}

An $\varepsilon$-argument \emph{is} required when:
\begin{itemize}
  \item There is no direct correspondence between bounds, or
  \item The operation combines elements in a way that obscures bound comparison.
\end{itemize}

Typical cases include:
\begin{itemize}
  \item $\sup(A+B) = \sup A + \sup B$
  \item $\sup(S^2) = (\sup S)^2$
  \item $\sup(f(S)) = f(\sup S)$ for nonlinear $f$
\end{itemize}

In these situations:
\begin{itemize}
  \item One can usually show the candidate value is an upper bound.
  \item Leastness must be shown by ruling out smaller bounds.
  \item This requires producing elements of the set arbitrarily close to the candidate,
        which is exactly the role of the $\varepsilon$-definition.
\end{itemize}

\subsubsection{Decision Checklist}

Before beginning a supremum or infimum proof, ask:

\begin{enumerate}
  \item Does the operation define a single-valued function on $\mathbb{R}$?
  \item Is the function bijective onto its image?
  \item Does it preserve or reverse order?
  \item Can every bound of the image be pulled back to a bound of the original set?
\end{enumerate}

\begin{itemize}
  \item If \emph{yes} to all: use order and bounds (no $\varepsilon$ needed).
  \item If \emph{any no}: an $\varepsilon$-argument is required.
\end{itemize}

\subsubsection{Summary}

\begin{quote}
\textbf{Bounds arguments establish existence.  
$\varepsilon$-arguments establish sharpness.}
\end{quote}

Understanding which role is needed in a given proof prevents unnecessary
$\varepsilon$-work while ensuring logical completeness when approximation
is unavoidable.


% =========================================================
% Worked Micro-Examples: When Is $\varepsilon$ Required?
% =========================================================

\subsubsection{Worked Micro-Examples}

The following short examples illustrate how the decision rule applies in
practice.  Each example is intentionally minimal and annotated according
to whether an $\varepsilon$-argument is logically required.

\paragraph{Example 1: Translation (No $\varepsilon$ Needed).}

Let $S \subseteq \mathbb{R}$ be nonempty and bounded above, and let $t \in \mathbb{R}$.
Define $S+t = \{s+t : s \in S\}$.

\[
\sup(S+t) = \sup S + t.
\]

\emph{Reason.}
The map $f(x)=x+t$ is an order-preserving bijection with order-preserving inverse.
Upper bounds of $S$ correspond exactly to upper bounds of $S+t$ under translation.
Leastness is established by direct comparison of bounds.

\begin{center}
\textbf{$\varepsilon$ not needed}
\end{center}

\paragraph{Example 2: Positive Scaling (No $\varepsilon$ Needed).}

Let $S \subseteq \mathbb{R}$ be nonempty and bounded above, and let $c>0$.
Define $cS=\{cs:s\in S\}$.

\[
\sup(cS)=c\,\sup S.
\]

\emph{Reason.}
The map $f(x)=cx$ is an order-preserving bijection with inverse $x\mapsto x/c$.
Upper bounds correspond under multiplication, and minimality follows by
transporting arbitrary bounds through the inverse.

\begin{center}
\textbf{$\varepsilon$ not needed}
\end{center}

\paragraph{Example 3: Reflection (No $\varepsilon$ Needed).}

Let $S \subseteq \mathbb{R}$ be nonempty and bounded.
Define $-S=\{-s:s\in S\}$.

\[
\sup(-S)=-\inf S, \qquad \inf(-S)=-\sup S.
\]

\emph{Reason.}
The map $f(x)=-x$ is an order-reversing bijection.
Upper bounds of $-S$ correspond to lower bounds of $S$, and vice versa.
Leastness and greatestness are obtained by direct comparison.

\begin{center}
\textbf{$\varepsilon$ not needed}
\end{center}

\paragraph{Example 4: Sum of Sets ($\varepsilon$ Needed).}

Let $A,B \subseteq \mathbb{R}$ be nonempty and bounded above.
Define $A+B=\{a+b:a\in A,b\in B\}$.

\[
\sup(A+B)=\sup A+\sup B.
\]

\emph{Reason.}
Although $\sup A+\sup B$ is easily shown to be an upper bound of $A+B$,
there is no direct correspondence between upper bounds of $A$, $B$, and $A+B$.
To show leastness, one must produce elements $a\in A$, $b\in B$ arbitrarily
close to their respective suprema.

\begin{center}
\textbf{$\varepsilon$ needed}
\end{center}

\paragraph{Example 5: Squaring ($\varepsilon$ Needed).}

Let $S \subseteq \mathbb{R}$ be nonempty and bounded above, and define
$T=\{s^2:s\in S\}$.

\[
\sup T = (\sup S)^2.
\]

\emph{Reason.}
The squaring function is not order-preserving on $\mathbb{R}$ and does not
establish a correspondence between bounds.
Leastness of $(\sup S)^2$ must be established by approximation using elements
of $S$ arbitrarily close to $\sup S$.

\begin{center}
\textbf{$\varepsilon$ needed}
\end{center}

\paragraph{Example 6: Cubing ($\varepsilon$ Needed).}

Let $S \subseteq \mathbb{R}$ be nonempty and bounded above, and define $f(x)=x^3$.

\[
\sup f(S) = f(\sup S).
\]

\emph{Reason.}
Although $f$ is order-preserving and bijective, the least-upper-bound property
cannot be transported by direct comparison of bounds alone.
To exclude smaller upper bounds, elements of $S$ must be shown to approximate
$\sup S$, requiring the $\varepsilon$-definition.

\begin{center}
\textbf{$\varepsilon$ needed}
\end{center}

\subsubsection{Meta-Observation}

\begin{quote}
Order-preserving structure determines whether bounds move correctly;  
$\varepsilon$-arguments are required only when that structure is insufficient
to establish sharpness.
\end{quote}




