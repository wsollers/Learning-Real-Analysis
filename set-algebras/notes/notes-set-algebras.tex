% =========================================================
% Systems of Sets
% File: set-algebras/notes/notes-set-algebras.tex
% =========================================================

\subsubsection{Families of Sets and Closure Operations}

Let $X$ be a set.
A \emph{family of sets on $X$} is a collection
\[
\mathcal{F} \subseteq 2^X.
\]

The objects of study are not elements of $X$,
but collections of subsets and their stability
under set-theoretic operations.

\vspace{0.5em}

\begin{definition}[Closure Under an Operation]
Let $\mathcal{F} \subseteq 2^X$ and let $\ast$
be an operation on subsets of $X$.
We say that $\mathcal{F}$ is \emph{closed under $\ast$}
if whenever the inputs belong to $\mathcal{F}$,
the result also belongs to $\mathcal{F}$.
\end{definition}

\vspace{0.5em}

Typical operations include:
\[
\cup, \quad \cap, \quad (\cdot)^c, \quad \setminus,
\quad \triangle.
\]

\vspace{0.5em}

\begin{remark}
Different combinations of closure properties
produce distinct algebraic structures.
The strength of a system of sets is determined
by which operations it is stable under.
\end{remark}

% =========================================================
% Finite Closure Structures
% =========================================================

\subsubsection{Finite Closure Structures}

\begin{definition}[Ring of Sets]
A nonempty collection $\mathcal{R} \subseteq 2^X$
is a \emph{ring of sets} if:

\begin{enumerate}
  \item $A,B \in \mathcal{R} \Rightarrow A \cup B \in \mathcal{R}$,
  \item $A,B \in \mathcal{R} \Rightarrow A \setminus B \in \mathcal{R}$.
\end{enumerate}
\end{definition}

\vspace{0.5em}

\begin{definition}[Algebra of Sets (Field of Sets)]
A collection $\mathcal{A} \subseteq 2^X$
is an \emph{algebra of sets} if:

\begin{enumerate}
  \item $X \in \mathcal{A}$,
  \item $A \in \mathcal{A} \Rightarrow A^c \in \mathcal{A}$,
  \item $A,B \in \mathcal{A} \Rightarrow A \cup B \in \mathcal{A}$.
\end{enumerate}
\end{definition}

\vspace{0.5em}

\begin{theorem}[Equivalent Characterization]
An algebra of sets is precisely
a ring of sets that contains $X$.
\end{theorem}

\vspace{0.5em}

\begin{remark}
Finite closure yields stability under all
finite Boolean combinations:
finite unions, finite intersections,
differences, complements,
and symmetric differences.
\end{remark}

\vspace{0.5em}

\begin{remark}
With symmetric difference
\[
A \triangle B := (A \setminus B) \cup (B \setminus A),
\]
a ring of sets forms an abelian group
under the operation $\triangle$.
This connects set algebras with Boolean rings.
\end{remark}

% =========================================================
% Countable Closure Structures
% =========================================================

\subsubsection{Countable Closure Structures}

\begin{definition}[$\sigma$-Ring]
A ring of sets $\mathcal{R}$ is a \emph{$\sigma$-ring}
if it is closed under countable unions:
\[
A_1, A_2, \dots \in \mathcal{R}
\Rightarrow
\bigcup_{n=1}^{\infty} A_n \in \mathcal{R}.
\]
\end{definition}

\vspace{0.5em}

\begin{definition}[$\sigma$-Algebra]
A collection $\mathcal{F} \subseteq 2^X$
is a \emph{$\sigma$-algebra} if:

\begin{enumerate}
  \item $X \in \mathcal{F}$,
  \item $A \in \mathcal{F} \Rightarrow A^c \in \mathcal{F}$,
  \item $A_1, A_2, \dots \in \mathcal{F}
        \Rightarrow
        \bigcup_{n=1}^{\infty} A_n \in \mathcal{F}$.
\end{enumerate}
\end{definition}

\vspace{0.5em}

\begin{theorem}[Closure Consequences]
A $\sigma$-algebra is closed under:
\begin{enumerate}
  \item Countable intersections,
  \item Differences,
  \item Finite Boolean combinations.
\end{enumerate}
\end{theorem}

\vspace{0.5em}

\begin{remark}
The transition from finite closure to countable closure
marks the passage from purely algebraic stability
to compatibility with limit processes.
\end{remark}

% =========================================================
% Structural Hierarchy
% =========================================================

\subsubsection{Hierarchy of Closure Regimes}

\begin{center}
\begin{tabular}{|c|c|c|c|}
\hline
Structure & Closed Under Complements & Finite Unions & Countable Unions \\
\hline
Ring & No & Yes & No \\
Algebra & Yes & Yes & No \\
$\sigma$-Algebra & Yes & Yes & Yes \\
\hline
\end{tabular}
\end{center}

\vspace{0.5em}

\begin{remark}[Structural Perspective]
Systems of sets are classified by the strength
of their closure properties.

Finite closure produces algebraic systems.
Countable closure produces $\sigma$-systems.
Further extensions (e.g.\ arbitrary unions)
lead to topological structures.
\end{remark}