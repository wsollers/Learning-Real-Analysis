% =========================================================
% Case Analysis
% =========================================================
\paragraph{Proof by Cases.}

If the hypotheses or conclusion naturally divide into exhaustive
and mutually exclusive alternatives, handle each case separately.

\begin{tcolorbox}[colback=propbox, colframe=propborder, arc=2pt,
  left=6pt, right=6pt, top=4pt, bottom=4pt,
  title={\small\textbf{Template: Cases}}, fonttitle=\small\bfseries]
We consider all cases.\\[4pt]
\textbf{Case 1:} [first alternative]. \quad $\vdots$ \quad [conclude $Q$]\\
\textbf{Case 2:} [second alternative]. \quad $\vdots$ \quad [conclude $Q$]\\[4pt]
In all cases, $Q$ holds. \qed
\end{tcolorbox}

\begin{remark}[The exhaustiveness obligation]
A case proof is only valid if the cases together cover every
possibility. Always verify this. Common exhaustive splits:
\begin{itemize}
  \item $n$ even or $n$ odd
  \item $a > 0$, $a = 0$, or $a < 0$
  \item $a \mid b$ or $a \nmid b$
  \item $x \in A$ or $x \notin A$
\end{itemize}
\end{remark}

\begin{remark}[Cases can reduce to earlier cases]
In a two-case proof, if Case 2 is symmetric to Case 1, it is
acceptable to write: ``Case 2 is symmetric.'' But only when
the symmetry is genuinely complete --- when the argument for
Case 2 is obtained from Case 1 by renaming variables.
\end{remark}

\begin{example}[Parity of $n(n+1)$]
\textbf{Claim.} For every $n \in \mathbb{Z}$,
$n(n+1)$ is even.

\medskip
\noindent\textit{Proof.}
\textbf{Case 1:} $n$ is even. Write $n = 2k$. Then
$n(n+1) = 2k(n+1)$, which is even.

\textbf{Case 2:} $n$ is odd. Write $n+1 = 2k$. Then
$n(n+1) = n \cdot 2k$, which is even.

In both cases $n(n+1)$ is even. Since every integer is even
or odd, the proof is complete. \qed
\end{example}
