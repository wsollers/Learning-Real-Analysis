% =========================================================
% Existence and Uniqueness Patterns
% =========================================================
\paragraph{Existence and Uniqueness.}

The statement $\exists!\, x\; P(x)$ has two parts:
existence ($\exists x\; P(x)$) and uniqueness ($\exists!\, x$).
They are proved separately.

\begin{tcolorbox}[colback=propbox, colframe=propborder, arc=2pt,
  left=6pt, right=6pt, top=4pt, bottom=4pt,
  title={\small\textbf{Template: Existence + Uniqueness}},
  fonttitle=\small\bfseries]
\textbf{Existence.}
Define $x := [\text{explicit construction}]$.
Verify that $P(x)$ holds.

\medskip
\textbf{Uniqueness.}
Let $x$ and $y$ both satisfy $P$.
Show $x = y$.
[This is the assume-two-and-compare pattern.]
\end{tcolorbox}

\begin{remark}[Existence must come first]
The uniqueness argument assumes that at least one object satisfying
$P$ exists; otherwise ``$x$ and $y$ both satisfy $P$'' has no
content. Always prove existence before uniqueness.
\end{remark}

\begin{remark}[The key move in assume-two-and-compare]
The proof that $x = y$ almost always uses $x$ and $y$ against
each other: apply the definition of $x$ as an identity (or inverse,
or fixed point) with $y$ as the input, and simultaneously apply
the definition of $y$ with $x$ as the input.
The two applications collapse to give $x = y$.

This move appears in:
uniqueness of group identity, uniqueness of inverses,
uniqueness of the zero vector, uniqueness of limits,
uniqueness of the supremum.
\end{remark}

\begin{example}[Uniqueness of group identity]
\textit{Proof.}
Suppose $e$ and $e'$ both satisfy the identity axiom in $G$.
Then:
\[
e = e \cdot e' = e'.
\]
The first equality uses $e'$ as an identity; the second uses
$e$ as an identity. Hence $e = e'$. \qed
\end{example}

\begin{remark}[Uniqueness creates a tool]
Every uniqueness theorem, once proved, activates the
satisfy-and-cite tactic (Section~\ref{sec:satisfy-cite}).
Proving uniqueness is not just an end in itself ---
it is the creation of a new proof tool that applies to all
subsequent arguments in the same structure.
\end{remark}
