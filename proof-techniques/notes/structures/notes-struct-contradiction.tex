% =========================================================
% Proof by Contradiction
% =========================================================
\paragraph{Proof by Contradiction.}

To prove $P$, assume $\neg P$ and derive a statement that
contradicts a known truth: a hypothesis, an axiom, or a
previously proved theorem.

\begin{tcolorbox}[colback=propbox, colframe=propborder, arc=2pt,
  left=6pt, right=6pt, top=4pt, bottom=4pt,
  title={\small\textbf{Template: Contradiction}}, fonttitle=\small\bfseries]
\textbf{Goal.} $P$.\\[4pt]
Suppose for contradiction that $\neg P$.\\
$\vdots$ \quad [derive a contradiction]\\
This contradicts [name the violated fact].\\
Therefore $P$. \qed
\end{tcolorbox}

\begin{remark}[When to use contradiction]
Contradiction is best reserved for claims whose negation has
strong consequences. It is the natural tool for:
\begin{itemize}
  \item Irrationality proofs ($\neg P$ gives a fraction in lowest
    terms and leads to a parity contradiction).
  \item Infinitude of primes ($\neg P$ gives a finite list;
    constructing a number not on the list contradicts completeness).
  \item Statements involving nonexistence.
\end{itemize}
Avoid using contradiction as a default. When a direct proof or
contrapositive works, it is cleaner.
\end{remark}

\begin{remark}[Name the contradiction explicitly]
A well-written contradiction proof always names what is contradicted.
``This contradicts [G2: identity axiom]'' is better than
``contradiction.'' Naming the violated fact confirms that the
contradiction is genuine and locates the logical break.
\end{remark}

\begin{example}[$\sqrt{2}$ is irrational]
\textit{Proof.}
Suppose for contradiction that $\sqrt{2} \in \mathbb{Q}$.
Write $\sqrt{2} = p/q$ with $p, q \in \mathbb{Z}$, $q \neq 0$,
and $\gcd(p,q) = 1$.
Then $2q^2 = p^2$, so $p^2$ is even, so $p$ is even (write $p = 2k$).
Then $2q^2 = 4k^2$, so $q^2 = 2k^2$, so $q$ is even.
But then $\gcd(p,q) \geq 2$, contradicting $\gcd(p,q) = 1$. \qed
\end{example}
