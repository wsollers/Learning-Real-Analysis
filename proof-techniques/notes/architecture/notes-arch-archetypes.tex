% =========================================================
% The Five Archetypes
% =========================================================
\paragraph{The Five Proof Archetypes.}

At the highest level of compression, every mathematical proof
reduces to one of five archetypes. Everything else is refinement
of these five.

\begin{tcolorbox}[colback=propbox, colframe=propborder, arc=2pt,
  left=6pt, right=6pt, top=4pt, bottom=4pt,
  title={\small\textbf{The Five Archetypes}}, fonttitle=\small\bfseries]
\begin{enumerate}
  \item \textbf{Construct something.}
    Exhibit a concrete object and verify it has the required property.
    Used for all existence claims.

  \item \textbf{Assume two and compare.}
    Assume two objects both satisfy a definition; derive that they are equal.
    Used for all uniqueness claims.

  \item \textbf{Take a minimal element.}
    Apply the well-ordering principle to a nonempty set;
    exploit the minimality of the least element.
    Used in number theory, induction, and contradiction arguments.

  \item \textbf{Induct.}
    Verify a base case; show the property propagates to successors.
    Used whenever the claim concerns all natural numbers or a
    recursively defined structure.

  \item \textbf{Chase an arbitrary element.}
    Take an arbitrary element of one set or satisfying one property;
    track it through definitions to land in the target.
    Used for set inclusions, function properties, and structural claims. Also may be used for direct proofs when the logic flows in a straight line.
\end{enumerate}
\end{tcolorbox}

\begin{remark}[Using the archetypes]
When starting a proof, identify which archetype applies before
consulting the detailed strategy map. The archetype tells you the
global shape of the argument. The statement map tells you
the specific opening move. The construction algorithm
(Section~\ref{sec:construction}) tells you how to fill in each line.
\end{remark}

\begin{remark}[Archetypes can nest]
A single proof may use more than one archetype. A common pattern
in algebra is to construct a witness (archetype 1) and then
invoke a uniqueness theorem (archetype 2) to conclude that the
constructed object equals a previously named one.
This nesting is the \emph{satisfy-and-cite} tactic developed
in Section~\ref{sec:satisfy-cite}.
\end{remark}
