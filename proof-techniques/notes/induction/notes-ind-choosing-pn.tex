% =========================================================
% Choosing P(n): The Hard Part
% =========================================================
\paragraph{Choosing $P(n)$: The Hard Part.}
\label{sec:choosing-pn}

The statement you induct on is a choice, not given data.
For textbook sum formulas, $P(n)$ is handed to you.
In original proofs, you must \emph{invent} it.
Sometimes the obvious choice does not work and must be strengthened.

\begin{remark}[The strengthening principle]
If the inductive step fails with $P(n)$, the fix is usually to
\emph{add more to $P(n)$}, not to try a different proof strategy.
This is counterintuitive: the stronger claim $P'(n)$
(which implies $P(n)$) is easier to induct on because the
inductive hypothesis $P'(n)$ gives you more to work with.
\end{remark}

\begin{example}[Strengthening is necessary]
\textbf{Goal.}
Show that the Fibonacci sequence $F_n$ defined by $F_1 = F_2 = 1$,
$F_{n+2} = F_{n+1} + F_n$ satisfies
$F_n < 2^n$ for all $n \geq 1$.

\medskip
\noindent\textbf{Naive attempt.}
Let $P(n)$: ``$F_n < 2^n$''.

Inductive step: assume $F_n < 2^n$; show $F_{n+1} < 2^{n+1}$.

We have $F_{n+1} = F_n + F_{n-1}$.
We know $F_n < 2^n$ by hypothesis. But what is $F_{n-1}$?
We have no bound on $F_{n-1}$ from $P(n)$ alone.
The step is stuck.

\medskip
\noindent\textbf{Fix: use strong induction.}
Assume $F_k < 2^k$ for all $k \leq n$.
Then:
\[
F_{n+1} = F_n + F_{n-1} < 2^n + 2^{n-1} < 2^n + 2^n = 2^{n+1}.
\]
The step goes through because the stronger hypothesis provides
$F_{n-1} < 2^{n-1}$.
This is the canonical signal that strong induction is needed.
\end{example}

\begin{example}[Strengthening the claim itself]
\textbf{Goal.}
Prove that for all $n \geq 1$, the sum
$\displaystyle S_n = \sum_{k=1}^n \frac{1}{k(k+1)}$
equals $\displaystyle 1 - \frac{1}{n+1}$.

\medskip
\noindent\textbf{What $P(n)$ to choose?}
The statement is handed to us: $P(n)$: $S_n = 1 - \frac{1}{n+1}$.
Here the step is: $S_{n+1} = S_n + \frac{1}{(n+1)(n+2)}$.
Substitute $S_n = 1 - \frac{1}{n+1}$ and simplify.
The inductive step works because $P(n)$ gives the exact
closed form needed.

\medskip
\noindent\textbf{Lesson.}
When the claim is a closed-form identity, $P(n)$ is the identity
itself. The step is always: split off the last term, apply the
inductive hypothesis, simplify algebraically.
\end{example}

\begin{remark}[Diagnostic questions for choosing $P(n)$]
When stuck choosing $P(n)$:
\begin{enumerate}
  \item What exactly do I need to prove for $n = 0, 1, 2, 3$?
    Write it out. The pattern of $P(n)$ will appear.
  \item Does the inductive step require knowing $P(n-1)$ as well
    as $P(n)$? If yes, switch to strong induction.
  \item Does the inductive step produce a bound or equation that
    is slightly weaker than $P(n+1)$? If yes, strengthen $P(n)$.
\end{enumerate}
\end{remark}
