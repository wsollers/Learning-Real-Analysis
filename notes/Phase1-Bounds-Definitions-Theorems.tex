% ============================================================
% Phase 1: Bounds, Supremum/Infimum, and Core Preliminaries
% INCLUDE FILE — NO PREAMBLE COMMANDS
% ============================================================

\section{Number Systems and Basic Structure}

\begin{definition}[Natural numbers, integers, rationals]
\[
\begin{aligned}
\N &= \{1,2,3,\dots\},\\
\Z &= \{\dots,-2,-1,0,1,2,\dots\},\\
\Q &= \left\{\frac{m}{n} : m\in\Z,\ n\in\N\right\}.
\end{aligned}
\]
\end{definition}

\begin{remark}
Some authors define $\N=\{0,1,2,\dots\}$.  
In this document we follow the convention $\N=\{1,2,3,\dots\}$ unless stated otherwise.
\end{remark}

% ============================================================
\section{Peano Axioms and Induction}
% ============================================================

\begin{axiom}[Peano axioms]
There exists a set $\N_0$, an element $0\in\N_0$, and a successor function
$S:\N_0\to\N_0$ such that:
\begin{enumerate}
  \item $0\in\N_0$.
  \item If $n\in\N_0$, then $S(n)\in\N_0$.
  \item There is no $n\in\N_0$ such that $S(n)=0$.
  \item If $S(n)=S(m)$, then $n=m$.
  \item (Induction axiom)  
  If $A\subseteq\N_0$ satisfies $0\in A$ and
  $n\in A\Rightarrow S(n)\in A$, then $A=\N_0$.
\end{enumerate}
\end{axiom}

\begin{theorem}[Principle of mathematical induction]
Let $P(n)$ be a statement about $n\in\N$.  
If $P(1)$ is true and $P(k)\Rightarrow P(k+1)$ for all $k\in\N$,  
then $P(n)$ holds for all $n\in\N$.
\end{theorem}

\begin{theorem}[Well-ordering principle]
Every nonempty subset of $\N$ has a least element.
\end{theorem}

\begin{remark}
The well-ordering principle and mathematical induction are equivalent.
\end{remark}

% ============================================================
\section{Sets, Functions, and Injectivity}
% ============================================================

\begin{definition}[Function]
A function $f:A\to B$ assigns to each $a\in A$ a unique element $f(a)\in B$.
\end{definition}

\begin{definition}[Injective (one-to-one)]
A function $f:A\to B$ is injective if
\[
f(x_1)=f(x_2)\;\Rightarrow\;x_1=x_2.
\]
\end{definition}

% ============================================================
\section{Cardinality and Countability}
% ============================================================

\begin{definition}[Cardinality]
Two sets $A$ and $B$ have the same cardinality if there exists a bijection $A\to B$.
\end{definition}

\begin{definition}[Countable]
A set is countable if it is finite or admits a bijection with $\N$.
\end{definition}

\begin{theorem}[Schr\"oder--Bernstein]
If there exist injections $A\to B$ and $B\to A$, then $|A|=|B|$.
\end{theorem}

\begin{theorem}
The sets $\Z$ and $\Q$ are countable, while $\R$ is uncountable.
\end{theorem}

% ============================================================
\section[Absolute Value and Distance]{Absolute Value and Distance on $\R$}
% ============================================================

\begin{definition}[Absolute value]
\[
|x|=\begin{cases}
x,&x\ge0,\\
-x,&x<0.
\end{cases}
\]
\end{definition}

\begin{theorem}[Triangle inequality]
For all $x,y\in\R$,
\[
|x+y|\le|x|+|y|.
\]
\end{theorem}

\begin{definition}[Distance on $\R$]
\[
d(x,y)=|x-y|.
\]
\end{definition}

\begin{theorem}[$\varepsilon$-characterization of equality]
For $a,b\in\R$,
\[
a=b \iff (\forall\varepsilon>0)\ |a-b|<\varepsilon.
\]
\end{theorem}

\begin{noteenv}
Add a proof of this characterization to a later problem set.
\end{noteenv}

% ============================================================
\section[Bounds and Completeness]{Bounds, Suprema, Infima, and Completeness}
% ============================================================

\begin{definition}[Upper and lower bounds]
Let $A\subseteq\R$.
\begin{itemize}
  \item $U$ is an upper bound if $a\le U$ for all $a\in A$.
  \item $L$ is a lower bound if $L\le a$ for all $a\in A$.
\end{itemize}
\end{definition}

\begin{definition}[Supremum]
$\alpha=\sup A$ if $\alpha$ is an upper bound of $A$ and
for every $\varepsilon>0$ there exists $a\in A$ with $\alpha-\varepsilon<a$.
\end{definition}

\begin{definition}[Infimum]
$\beta=\inf A$ if $\beta$ is a lower bound of $A$ and
for every $\varepsilon>0$ there exists $a\in A$ with $a<\beta+\varepsilon$.
\end{definition}

\begin{axiom}[Completeness]
Every nonempty subset of $\R$ that is bounded above has a supremum in $\R$.
\end{axiom}

% ============================================================
\section{Standard Consequences}
% ============================================================

\begin{theorem}[Archimedean property]
For every $x\in\R$ there exists $n\in\N$ such that $n>x$.
\end{theorem}

\begin{theorem}[Density of $\Q$ in $\R$]
For any $a<b$ in $\R$ there exists $q\in\Q$ with $a<q<b$.
\end{theorem}

\begin{theorem}[Nested interval property]
If $[a_n,b_n]$ is a nested sequence of closed intervals, then
$\bigcap_{n=1}^\infty [a_n,b_n]\neq\varnothing$.
\end{theorem}
