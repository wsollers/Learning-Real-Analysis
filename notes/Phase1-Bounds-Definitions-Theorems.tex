% ============================================================
% Phase 1: Bounds, Supremum/Infimum, and Core Preliminaries
% INCLUDE FILE: meant to be used with \input{...}
% ============================================================
% Improve spacing around theorem environments
\usepackage{mathtools}
\usepackage{setspace}
\theoremstyle{definition}
\newtheorem{definition}{Definition}[section]

\theoremstyle{plain}
\newtheorem{theorem}[definition]{Theorem}
\newtheorem{axiom}[definition]{Axiom}

\theoremstyle{remark}
\newtheorem{remark}[definition]{Remark}
\newtheorem{noteenv}[definition]{Note}

\setlength{\parskip}{0.6em}
\setlength{\parindent}{0pt}

\section{Number Systems and Basic Structure}

\begin{definition}[Natural numbers, integers, rationals]
\[
\begin{aligned}
\N &= \{1,2,3,\dots\},\\
\Z &= \{\dots,-2,-1,0,1,2,\dots\},\\
\Q &= \left\{\frac{m}{n} : m\in\Z,\ n\in\N\right\}.
\end{aligned}
\]


\end{definition}

\begin{remark}
Some authors define $\N=\{0,1,2,\dots\}$. In this document we follow the convention
$\N=\{1,2,3,\dots\}$ unless otherwise stated.
\end{remark}

% ============================================================
\section{Peano Axioms and Induction}
% ============================================================

\begin{axiom}[Peano axioms (one common formulation)]
There is a set $\N_0$ of natural numbers containing a distinguished element $0$ and a successor function
$S:\N_0\to\N_0$ satisfying:
\begin{enumerate}
  \item $0\in\N_0$.
  \item For every $n\in\N_0$, $S(n)\in\N_0$.
  \item There is no $n\in\N_0$ such that $S(n)=0$.
  \item If $S(n)=S(m)$ then $n=m$ (injectivity of successor).
  \item (Induction axiom) If $A\subseteq \N_0$ contains $0$ and satisfies $n\in A\Rightarrow S(n)\in A$,
  then $A=\N_0$.
\end{enumerate}
\end{axiom}

\begin{theorem}[Principle of mathematical induction]
Let $P(n)$ be a statement about $n\in\N$. If:
\begin{enumerate}
  \item $P(1)$ is true, and
  \item for all $k\in\N$, $P(k)\Rightarrow P(k+1)$,
\end{enumerate}
then $P(n)$ is true for all $n\in\N$.
\end{theorem}

\begin{theorem}[Well-ordering principle]
Every nonempty subset of $\N$ has a least element.
\end{theorem}

\begin{remark}
The well-ordering principle and the principle of induction are equivalent (over standard foundational systems).
\end{remark}

% ============================================================
\section{Sets, Functions, and Injectivity}
% ============================================================

\begin{definition}[Function]
A function $f:A\to B$ assigns to each $a\in A$ a unique element $f(a)\in B$.
\end{definition}

\begin{definition}[Injective (one-to-one)]
A function $f:A\to B$ is \emph{injective} if for all $x_1,x_2\in A$,
\[
f(x_1)=f(x_2)\;\Rightarrow\; x_1=x_2.
\]
\end{definition}

% ============================================================
\section{Cardinality and Countability}
% ============================================================

\begin{definition}[Cardinality]
Two sets $A$ and $B$ have the same \emph{cardinality} if there exists a bijection $f:A\to B$.
The cardinality of $A$ is denoted $|A|$.
\end{definition}

\begin{definition}[Countable]
A set $S$ is \emph{countable} if it is finite or if there exists a bijection $S\to\N$.
\end{definition}

\begin{theorem}[Schr\"oder--Bernstein (Cantor--Bernstein)]
Let $A$ and $B$ be sets. If there exist injections $f:A\to B$ and $g:B\to A$,
then there exists a bijection $h:A\to B$. In particular, $|A|=|B|$.
\end{theorem}

\begin{theorem}[Countability and uncountability]
The sets $\Z$ and $\Q$ are countable. The set $\R$ is uncountable.
\end{theorem}

% ============================================================
\section{Absolute Value and Distance on \texorpdfstring{$\R$}{R}}
% ============================================================

\begin{definition}[Absolute value]
For $x\in\R$,
\[
|x|=\begin{cases}
x, & x\ge 0,\\
-x,& x<0.
\end{cases}
\]
\end{definition}

\begin{theorem}[Basic properties of absolute value]
For all $x,y\in\R$:
\begin{enumerate}
  \item (Nonnegativity) $|x|\ge 0$, and $|x|=0\iff x=0$.
  \item (Multiplicativity) $|xy|=|x||y|$.
  \item (Triangle inequality) $|x+y|\le |x|+|y|$.
\end{enumerate}
\end{theorem}

\begin{definition}[Distance on the real line]
Define $d:\R\times\R\to\R$ by
\[
d(x,y)=|x-y|.
\]
\end{definition}

\begin{theorem}[Metric properties on $\R$]
For all $x,y,z\in\R$, the function $d(x,y)=|x-y|$ satisfies:
\begin{enumerate}
  \item (Nonnegativity) $d(x,y)\ge 0$.
  \item (Identity of indiscernibles) $d(x,y)=0 \iff x=y$.
  \item (Symmetry) $d(x,y)=d(y,x)$.
  \item (Triangle inequality) $d(x,z)\le d(x,y)+d(y,z)$.
\end{enumerate}
\end{theorem}

\begin{theorem}[$\varepsilon$-characterization of equality]
Let $a,b\in\R$. Then $a=b$ if and only if
\[
\forall \varepsilon>0,\quad |a-b|<\varepsilon.
\]
\end{theorem}

\begin{noteenv}[Future problem]
Prove the $\varepsilon$-characterization of equality and add it to a problem set later.
\end{noteenv}

% ============================================================
\section{Ordered Field Framework (Field Axioms + Order Properties)}
% ============================================================

\subsection{Field axioms (algebraic structure)}
\begin{axiom}[Field axioms for $\R$]
The set $\R$ is equipped with operations $+$ and $\cdot$ such that:
\begin{enumerate}
  \item (Additive associativity) $(a+b)+c=a+(b+c)$.
  \item (Additive commutativity) $a+b=b+a$.
  \item (Additive identity) $\exists 0\in\R$ such that $a+0=a$.
  \item (Additive inverses) $\forall a\in\R,\exists (-a)\in\R$ with $a+(-a)=0$.
  \item (Multiplicative associativity) $(ab)c=a(bc)$.
  \item (Multiplicative commutativity) $ab=ba$.
  \item (Multiplicative identity) $\exists 1\in\R$, $1\ne 0$, such that $a\cdot 1=a$.
  \item (Multiplicative inverses) $\forall a\in\R\setminus\{0\},\exists a^{-1}\in\R$ with $aa^{-1}=1$.
  \item (Distributivity) $a(b+c)=ab+ac$.
\end{enumerate}
\end{axiom}

\subsection{Order axioms (order structure)}
\begin{axiom}[Order axioms / ordered field properties (standard list)]
The set $\R$ has a total order $<$ compatible with addition and multiplication:
\begin{enumerate}
  \item (Trichotomy) For $a,b\in\R$, exactly one of $a<b$, $a=b$, $a>b$ holds.
  \item (Transitivity) If $a<b$ and $b<c$, then $a<c$.
  \item (Additive monotonicity) If $a<b$, then $a+c<b+c$ for all $c\in\R$.
  \item (Multiplicative monotonicity) If $a<b$ and $c>0$, then $ac<bc$.
\end{enumerate}
\end{axiom}

% ============================================================
\section{Bounds, Suprema, Infima, and Completeness}
% ============================================================

\subsection{Bounds and unboundedness}

\begin{definition}[Upper bound / lower bound]
Let $A\subseteq\R$.
\begin{itemize}
  \item A number $U\in\R$ is an \emph{upper bound} of $A$ if $\forall a\in A,\; a\le U$.
  \item A number $L\in\R$ is a \emph{lower bound} of $A$ if $\forall a\in A,\; L\le a$.
\end{itemize}
\end{definition}

\begin{definition}[Unbounded]
A set $A\subseteq\R$ is \emph{unbounded above} if it has no upper bound, i.e.
\[
\forall M\in\R,\;\exists a\in A \text{ such that } a>M.
\]
Similarly, $A$ is \emph{unbounded below} if $\forall m\in\R,\exists a\in A$ such that $a<m$.
If both occur, $A$ is \emph{unbounded}.
\end{definition}

\begin{definition}[$\varepsilon$-upper bound formulation (buffered bound)]
Let $A\subseteq\R$. A number $U\in\R$ is an upper bound of $A$ if and only if
\[
\forall \varepsilon>0,\;\forall a\in A,\quad a\le U+\varepsilon.
\]
Similarly, $L\in\R$ is a lower bound of $A$ if and only if
\[
\forall \varepsilon>0,\;\forall a\in A,\quad L-\varepsilon\le a.
\]
\end{definition}

\begin{noteenv}[Future problem]
Prove the $\varepsilon$-upper/lower bound formulations and add them to a problem set later.
\end{noteenv}

\subsection{Visualizing bounds and $\varepsilon$-definitions}

% ---------- Figure A: epsilon approximation for sup/inf ----------
\begin{figure}[h]
\centering
\begin{tikzpicture}[x=1cm,y=1cm,>=Latex]

\def\xL{0}
\def\xR{14}
\def\infA{4}
\def\supA{10.5}
\def\eps{1.5}
\def\aSup{9.8}
\def\aInf{4.6}

\draw[thick] (\xL,0) -- (\xR,0);

\draw[line width=3pt] (\infA+0.3,0) -- (\supA-0.9,0);
\foreach \x in {9.3,9.6,9.85,10.1} {
  \fill (\x,0) circle (2pt);
}
\node[below] at ({(\infA+\supA)/2},-0.4) {$A$};

\draw[decorate,decoration={brace,amplitude=6pt}]
(\xL,-0.8) -- (\infA,-0.8);
\node[below] at ({(\xL+\infA)/2},-1.2) {lower bounds};

\draw[decorate,decoration={brace,amplitude=6pt}]
(\supA,-0.8) -- (\xR,-0.8);
\node[below] at ({(\supA+\xR)/2},-1.2) {upper bounds};

\draw[thick] (\infA,0) -- (\infA,0.55);
\node[above] at (\infA,0.55) {$\inf A$};

\draw[thick] (\supA,0) -- (\supA,0.55);
\node[above] at (\supA,0.55) {$\sup A$};

\draw[thick] (\supA-\eps,0) -- (\supA,0);
\draw[<->] (\supA-\eps,0.9) -- (\supA,0.9);
\node[above] at (\supA-\eps/2,0.9) {$\varepsilon$};

\fill (\aSup,0) circle (2.5pt);
\node[above right] at (\aSup,0.05) {$a\in A$};

\node[align=left] at (12.4,1.7)
{$\forall \varepsilon>0\ \exists a\in A$\\
$\sup A-\varepsilon<a\le \sup A$};

\draw[thick] (\infA,0) -- (\infA+\eps,0);
\draw[<->] (\infA,0.9) -- (\infA+\eps,0.9);
\node[above] at (\infA+\eps/2,0.9) {$\varepsilon$};

\fill (\aInf,0) circle (2.5pt);
\node[above left] at (\aInf,0.05) {$a\in A$};

\node[align=left] at (2.2,1.7)
{$\forall \varepsilon>0\ \exists a\in A$\\
$\inf A\le a<\inf A+\varepsilon$};

\end{tikzpicture}
\caption{Supremum and infimum via the $\varepsilon$-approximation property.}
\end{figure}

% ---------- Figure B: epsilon-buffered upper/lower bounds ----------
\begin{figure}[h]
\centering
\begin{tikzpicture}[x=1cm,y=1cm,>=Latex]

\def\xL{0}
\def\xR{14}
\def\infA{4}
\def\supA{10.5}
\def\eps{1.4}

\draw[thick] (\xL,0) -- (\xR,0);

\draw[line width=3pt] (\infA+0.3,0) -- (\supA-0.9,0);
\foreach \x in {9.3,9.6,9.9,10.15} {
  \fill (\x,0) circle (2pt);
}
\node[below] at ({(\infA+\supA)/2},-0.4) {$A$};

\draw[thick] (\supA,0) -- (\supA,0.55);
\node[above] at (\supA,0.55) {$U=\sup A$};

\draw[dashed,thick] (\supA+\eps,0) -- (\supA+\eps,0.55);
\node[above] at (\supA+\eps,0.55) {$U+\varepsilon$};

\draw[decorate,decoration={brace,amplitude=6pt}]
(\supA,0.9) -- (\supA+\eps,0.9);
\node[above] at (\supA+\eps/2,0.9) {$\varepsilon$};

\node[align=left] at (12.2,1.9)
{$\forall a\in A$\\
$a\le U+\varepsilon$};

\draw[thick] (\infA,0) -- (\infA,0.55);
\node[above] at (\infA,0.55) {$L=\inf A$};

\draw[dashed,thick] (\infA-\eps,0) -- (\infA-\eps,0.55);
\node[above] at (\infA-\eps,0.55) {$L-\varepsilon$};

\draw[decorate,decoration={brace,amplitude=6pt}]
(\infA-\eps,0.9) -- (\infA,0.9);
\node[above] at (\infA-\eps/2,0.9) {$\varepsilon$};

\node[align=left] at (2.3,1.9)
{$\forall a\in A$\\
$L-\varepsilon\le a$};

\end{tikzpicture}
\caption{Upper and lower bounds with $\varepsilon$-buffers.}
\end{figure}

\subsection{Supremum, infimum, maximum, minimum}

\begin{definition}[Supremum (least upper bound)]
Let $A\subseteq\R$ be nonempty and bounded above. A number $\alpha\in\R$ is the \emph{supremum} of $A$,
written $\alpha=\sup A$, if:
\begin{enumerate}
  \item $\alpha$ is an upper bound of $A$, and
  \item for every $\varepsilon>0$ there exists $a\in A$ such that $\alpha-\varepsilon<a$.
\end{enumerate}
\end{definition}

\begin{definition}[Infimum (greatest lower bound)]
Let $A\subseteq\R$ be nonempty and bounded below. A number $\beta\in\R$ is the \emph{infimum} of $A$,
written $\beta=\inf A$, if:
\begin{enumerate}
  \item $\beta$ is a lower bound of $A$, and
  \item for every $\varepsilon>0$ there exists $a\in A$ such that $a<\beta+\varepsilon$.
\end{enumerate}
\end{definition}

\begin{definition}[Maximum and minimum]
Let $A\subseteq\R$.
\begin{itemize}
  \item $M\in A$ is a \emph{maximum} of $A$ if $\forall a\in A,\ a\le M$.
  \item $m\in A$ is a \emph{minimum} of $A$ if $\forall a\in A,\ m\le a$.
\end{itemize}
\end{definition}

\begin{remark}
If $\max A$ exists, then $\max A=\sup A$. If $\min A$ exists, then $\min A=\inf A$.
The key distinction is: maxima/minima must belong to the set.
\end{remark}

\subsection{Completeness}

\begin{axiom}[Axiom of completeness (least upper bound property)]
Every nonempty subset of $\R$ that is bounded above has a least upper bound (a supremum) in $\R$.
\end{axiom}

% ============================================================
\section{Core Consequences and Standard Theorems}
% ============================================================

\begin{theorem}[Archimedean property]
For every real number $x\in\R$ there exists $n\in\N$ such that $n>x$.
Equivalently, for every $y>0$ there exists $m\in\N$ such that $\frac{1}{m}<y$.
\end{theorem}

\begin{theorem}[Density of the rationals in the reals]
For any $a,b\in\R$ with $a<b$, there exists $q\in\Q$ such that
\[
a<q<b.
\]
\end{theorem}

\begin{remark}
Trivially, $\R$ is dense in $\R$ (between any two distinct reals lies another real).
The density statement of real analysis interest is the density of $\Q$ in $\R$.
\end{remark}

\begin{theorem}[Nested interval property]
Let $\{[a_n,b_n]\}_{n=1}^\infty$ be a sequence of closed intervals such that
\[
[a_1,b_1]\supseteq [a_2,b_2]\supseteq [a_3,b_3]\supseteq \cdots.
\]
Then
\[
\bigcap_{n=1}^\infty [a_n,b_n]\neq\varnothing.
\]
Moreover, if $b_n-a_n\to 0$, then the intersection consists of exactly one point.
\end{theorem}

% ============================================================
\section{Notes for Future Problem Sets (Proof Targets)}
% ============================================================

\begin{noteenv}[Transformation rules for bounds/sup/inf]
Write proofs for how bounds, supremum, and infimum behave under:
\begin{itemize}
  \item translations $A\mapsto A+r=\{a+r:a\in A\}$,
  \item scalings $A\mapsto cA=\{ca:a\in A\}$ (separate cases $c>0$ and $c<0$),
\end{itemize}
and do this for:
\[
\text{upper/lower bounds, }\sup,\inf,\max,\min.
\]
\end{noteenv}

\begin{noteenv}[Union/intersection rules for sup/inf and bounds]
Write proofs for relationships involving unions and intersections, e.g.
\[
\sup(A\cup B),\ \inf(A\cup B),\ \sup(A\cap B),\ \inf(A\cap B),
\]
and corresponding statements about upper/lower bounds (under appropriate hypotheses, e.g. nonempty intersections).
\end{noteenv}

\begin{noteenv}[Exercise: translation of supremum/infimum]
Let $A\subseteq\R$ and define $B=\{x+r:\ x\in A\}$ for a fixed real $r$.
Prove the relationship between $\sup A$ and $\sup B$ (and similarly for $\inf$).
\end{noteenv}
