% ============================================================
% Phase 1: Foundations for Real Analysis
% Core Definitions, Axioms, and Theorems
% INCLUDE FILE — to be used with \input{...}
% ============================================================

\section{Number Systems and Basic Structure}

\begin{definition}[Natural numbers, integers, rationals]
\[
\begin{aligned}
\N &= \{1,2,3,\dots\},\\
\Z &= \{\dots,-2,-1,0,1,2,\dots\},\\
\Q &= \left\{\frac{m}{n} : m\in\Z,\ n\in\N\right\}.
\end{aligned}
\]
\end{definition}

\begin{remark}
Some authors define $\N=\{0,1,2,\dots\}$.  
In this document we follow the convention $\N=\{1,2,3,\dots\}$ unless otherwise stated.
\end{remark}

% ============================================================
\section{Peano Axioms and Induction}
% ============================================================

\begin{axiom}[Peano axioms]
There exists a set $\N_0$, a distinguished element $0\in\N_0$, and a successor function
$S:\N_0\to\N_0$ satisfying:
\begin{enumerate}
  \item $0\in\N_0$.
  \item If $n\in\N_0$, then $S(n)\in\N_0$.
  \item There is no $n\in\N_0$ such that $S(n)=0$.
  \item If $S(n)=S(m)$, then $n=m$.
  \item (Induction axiom)  
  If $A\subseteq\N_0$ contains $0$ and satisfies  
  $n\in A \Rightarrow S(n)\in A$, then $A=\N_0$.
\end{enumerate}
\end{axiom}

\begin{theorem}[Principle of mathematical induction]
Let $P(n)$ be a statement about $n\in\N$.  
If $P(1)$ is true and $P(k)\Rightarrow P(k+1)$ for all $k\in\N$,  
then $P(n)$ holds for all $n\in\N$.
\end{theorem}

\begin{theorem}[Well-ordering principle]
Every nonempty subset of $\N$ has a least element.
\end{theorem}

\begin{remark}
The well-ordering principle and the principle of induction are equivalent.
\end{remark}

% ============================================================
\section{Sets, Functions, and Injectivity}
% ============================================================

\begin{definition}[Function]
A function $f:A\to B$ assigns to each $a\in A$ a unique element $f(a)\in B$.
\end{definition}

\begin{definition}[Injective (one-to-one)]
A function $f:A\to B$ is injective if
\[
f(x_1)=f(x_2)\Rightarrow x_1=x_2.
\]
\end{definition}

% ============================================================
\section{Cardinality and Countability}
% ============================================================

\begin{definition}[Cardinality]
Two sets $A$ and $B$ have the same cardinality if there exists a bijection $A\to B$.
\end{definition}

\begin{definition}[Countable]
A set is countable if it is finite or admits a bijection with $\N$.
\end{definition}

\begin{theorem}[Schr\"oder--Bernstein]
If there exist injections $A\to B$ and $B\to A$, then $|A|=|B|$.
\end{theorem}

\begin{theorem}
The sets $\Z$ and $\Q$ are countable, while $\R$ is uncountable.
\end{theorem}

% ============================================================
\section{Absolute Value and Distance on $\R$}
% ============================================================

\begin{definition}[Absolute value]
\[
|x|=
\begin{cases}
x, & x\ge 0,\\
-x, & x<0.
\end{cases}
\]
\end{definition}

\begin{theorem}[Triangle inequality]
For all $x,y\in\R$,
\[
|x+y|\le |x|+|y|.
\]
\end{theorem}

\begin{definition}[Distance on the real line]
\[
d(x,y)=|x-y|.
\]
\end{definition}

\begin{theorem}[$\varepsilon$-characterization of equality]
For $a,b\in\R$,
\[
a=b \iff (\forall \varepsilon>0)\ |a-b|<\varepsilon.
\]
\end{theorem}

\begin{noteenv}
Add a proof of this characterization to a future problem set.
\end{noteenv}

% ============================================================
\section{Bounds, Suprema, Infima, and Completeness}
% ============================================================

\subsection{Visualizing bounds and $\varepsilon$-definitions}

\begin{figure}[h]
\centering
\begin{tikzpicture}[x=1cm,y=1cm,>=Latex]
\def\xL{0}\def\xR{14}\def\infA{4}\def\supA{10.5}\def\eps{1.5}
\draw[thick] (\xL,0)--(\xR,0);
\draw[line width=3pt] (\infA+0.3,0)--(\supA-0.9,0);
\node[below] at ({(\infA+\supA)/2},-0.4) {$A$};
\draw[thick] (\infA,0)--(\infA,0.5);
\draw[thick] (\supA,0)--(\supA,0.5);
\node[above] at (\infA,0.5) {$\inf A$};
\node[above] at (\supA,0.5) {$\sup A$};
\end{tikzpicture}
\caption{Location of bounds, supremum, and infimum.}
\end{figure}

\begin{figure}[h]
\centering
\begin{tikzpicture}[x=1cm,y=1cm,>=Latex]
\def\xL{0}\def\xR{14}\def\infA{4}\def\supA{10.5}\def\eps{1.2}
\draw[thick] (\xL,0)--(\xR,0);
\draw[line width=3pt] (\infA+0.3,0)--(\supA-0.9,0);
\draw[dashed] (\supA+\eps,0)--(\supA+\eps,0.5);
\node[above] at (\supA+\eps,0.5) {$\sup A+\varepsilon$};
\draw[dashed] (\infA-\eps,0)--(\infA-\eps,0.5);
\node[above] at (\infA-\eps,0.5) {$\inf A-\varepsilon$};
\end{tikzpicture}
\caption{$\varepsilon$-buffer interpretation of bounds.}
\end{figure}

\subsection{Definitions}

\begin{definition}[Upper and lower bounds]
Let $A\subseteq\R$.
\begin{itemize}
  \item $U$ is an upper bound of $A$ if $a\le U$ for all $a\in A$.
  \item $L$ is a lower bound of $A$ if $L\le a$ for all $a\in A$.
\end{itemize}
\end{definition}

\begin{definition}[Supremum]
Let $A\subseteq\R$ be nonempty and bounded above.  
A number $\alpha\in\R$ is the supremum of $A$, written $\alpha=\sup A$, if:
\begin{enumerate}
  \item $\alpha$ is an upper bound of $A$, and
  \item for every $\varepsilon>0$ there exists $a\in A$ such that
  \[
  \alpha-\varepsilon<a\le\alpha.
  \]
\end{enumerate}
\end{definition}

\begin{definition}[Infimum]
Let $A\subseteq\R$ be nonempty and bounded below.  
A number $\beta\in\R$ is the infimum of $A$, written $\beta=\inf A$, if:
\begin{enumerate}
  \item $\beta$ is a lower bound of $A$, and
  \item for every $\varepsilon>0$ there exists $a\in A$ such that
  \[
  \beta\le a<\beta+\varepsilon.
  \]
\end{enumerate}
\end{definition}

\begin{axiom}[Axiom of completeness]
Every nonempty subset of $\R$ that is bounded above has a supremum in $\R$.
\end{axiom}
