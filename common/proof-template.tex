% =========================
% Proof Template (Modern Style)
% =========================

% ---------- Packages (typical) ----------
% \usepackage{amsmath, amssymb, amsthm}
% \usepackage{hyperref}

% ---------- Theorem/Proof environments ----------
% (If you already have these, keep yours.)
\newtheorem{theorem}{Theorem}[section]
\newtheorem{lemma}[theorem]{Lemma}
\newtheorem{proposition}[theorem]{Proposition}
\newtheorem{corollary}[theorem]{Corollary}
\theoremstyle{definition}
\newtheorem{definition}[theorem]{Definition}

% ---------- House-style macros (optional but useful) ----------
\newcommand{\Given}{\textbf{Given.}\ }
\newcommand{\Goal}{\textbf{Goal.}\ }
\newcommand{\Strategy}{\textbf{Strategy.}\ }
\newcommand{\ProofPlan}{\textbf{Plan.}\ }
\newcommand{\Recall}{\textbf{Recall.}\ }
\newcommand{\Claim}{\textbf{Claim.}\ }
\newcommand{\Wlog}{\textit{Without loss of generality}}
\newcommand{\Fix}{\textbf{Fix.}\ }
\newcommand{\Let}{\textbf{Let.}\ }
\newcommand{\Choose}{\textbf{Choose.}\ }
\newcommand{\Take}{\textbf{Take.}\ }
\newcommand{\Note}{\textbf{Note.}\ }
\newcommand{\ByDef}{\textbf{By definition,}\ }
\newcommand{\ByAss}{\textbf{By assumption,}\ }
\newcommand{\ByThm}[1]{\textbf{By #1,}\ }
\newcommand{\Thus}{\textbf{Thus,}\ }
\newcommand{\Hence}{\textbf{Hence,}\ }
\newcommand{\Therefore}{\textbf{Therefore,}\ }
\newcommand{\AsReq}{\textbf{as required.}}

% ============================================================
% TEMPLATE: A theorem/proposition with a professional proof
% ============================================================

\begin{theorem}[TITLE GOES HERE]\label{thm:YOUR_LABEL}
% Statement:
Let $X$ be \dots\ . Assume \dots\ . Then \dots\ .
\end{theorem}

\begin{proof}
% --- (0) Context / Setup (1–3 sentences) ---
\Given (Write the hypotheses in words if helpful; do not restate everything verbatim unless needed.)

\Goal (State the target conclusion in one sentence: “Show that …”.)

% --- (1) Strategy (optional but recommended for longer proofs) ---
\Strategy (One sentence describing the approach: direct / contrapositive / contradiction / induction / construction.)

% --- (2) Definitions / key facts (optional, but good when a definition drives the proof) ---
\Recall (Insert the exact definition(s) or lemma(s) you will use, especially if quantifiers matter.)

% --- (3) Main argument ---
% Use explicit quantifier discipline: "Let ε>0 be arbitrary. Choose δ>0 ..." etc.

\Let (Introduce arbitrary elements with correct dependency, e.g. “Let $\varepsilon>0$ be arbitrary.”)

\Choose (Choose objects only after existence is justified, e.g. “Since $A\neq\varnothing$, choose $a\in A$.”)

\ByDef (Unpack the definition at the critical step.)

\ByThm{Lemma~\ref{lem:YOUR_LEMMA_LABEL}} (Cite prior results explicitly when used.)

\Therefore Conclude an intermediate result that clearly follows.

% --- (4) Closing (explicitly match the goal statement) ---
\Hence (Restate the conclusion in the same form as the theorem statement.)
\AsReq
\end{proof}

% ============================================================
% VARIANTS: Common proof shapes
% ============================================================

% ---------- (A) Proof by Contrapositive ----------
% \begin{proof}
% \Strategy We prove the contrapositive.
% Assume $\neg Q$. We show $\neg P$.
% ...
% Hence $\neg Q \Rightarrow \neg P$, so $P \Rightarrow Q$. \AsReq
% \end{proof}

% ---------- (B) Proof by Contradiction ----------
% \begin{proof}
% \Strategy Suppose, toward a contradiction, that the conclusion is false.
% ...
% This contradicts (STATE WHAT IT CONTRADICTS).
% Therefore the conclusion holds. \AsReq
% \end{proof}

% ---------- (C) Induction ----------
% \begin{proof}
% \Strategy We proceed by induction on $n\in\mathbb{N}$.
% \textbf{Base case.} (Prove for $n=1$ or $n=0$.)
% \textbf{Inductive hypothesis.} Assume the statement holds for some fixed $n$.
% \textbf{Inductive step.} Prove it holds for $n+1$ using the hypothesis.
% Therefore the statement holds for all $n\in\mathbb{N}$. \AsReq
% \end{proof}

% ---------- (D) Existence / Construction ----------
% \begin{proof}
% \Strategy We explicitly construct an object $x$ with the required property.
% Define $x := \dots$.
% \Note First check $x$ is well-defined.
% Then verify the required properties:
% (i) ...
% (ii) ...
% Hence such an $x$ exists. \AsReq
% \end{proof}

% ---------- (E) Uniqueness (often paired with existence) ----------
% \begin{proof}
% \Strategy Suppose $x$ and $y$ both satisfy the defining properties; show $x=y$.
% ...
% Therefore the object is unique. \AsReq
% \end{proof}
