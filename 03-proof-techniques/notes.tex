% =========================================================
% Proof Techniques
% =========================================================

\subsection{A General Procedure for Writing Proofs}

Mathematical proofs follow precise logical rules, but they are not written
randomly. Most proofs can be constructed systematically by following a small
number of disciplined steps. This section records a general \emph{proof-writing
procedure} intended to be applied consciously until it becomes internalized.

\subsection{The Proof Construction Algorithm}

When asked to prove a mathematical statement, proceed as follows.

\subsubsection*{Step 0: Classify the Statement}

Before writing the proof, determine the logical form of the claim. Common forms
include:
\begin{itemize}
\item Universal statements: $\forall x\, P(x)$
\item Conditional statements: $P \rightarrow Q$
\item Biconditional statements: $P \leftrightarrow Q$
\item Existential statements: $\exists x\, P(x)$
\item Equality statements: $X = Y$
\item Set equality: $A = B$
\item Structural assertions: ``$R$ is an equivalence relation'', ``$f$ is injective''
\end{itemize}

The logical form determines the overall structure of the proof.

\subsubsection*{Step 1: Restate the Givens}

Explicitly record what is given. Introduce all sets, relations, and functions,
and note any structural properties that are assumed.

For example:
\begin{quote}
Let $A$ be a set and let $R$ be an equivalence relation on $A$.
\end{quote}

This licenses the later use of reflexivity, symmetry, and transitivity without
reintroducing them each time.

\subsubsection*{Step 2: Identify the Objects and Their Types}

Before reasoning begins, identify the \emph{type} of each object involved:
\begin{itemize}
\item Is it an element, a set, a function, or a relation?
\item What is its ambient universe?
\end{itemize}

For example:
\[
x \in A,
\quad
f : A \to B,
\quad
R \subseteq A \times A.
\]

Many logical errors arise from confusing equality, membership, and inclusion.
Every object used in a proof must have a declared type.

\subsubsection*{Step 3: Introduce Arbitrary Elements}

If the claim is universal, immediately introduce arbitrary elements.

For example:
\[
\text{Let } a,b \in A \text{ be arbitrary.}
\]

This step enables general reasoning and avoids illegal specialization.
No conclusions may be drawn from an arbitrary element other than those justified
by the hypotheses.

\subsubsection*{Step 4: Expand the Goal}

Rewrite the conclusion using definitions rather than named concepts.

Examples:
\begin{itemize}
\item To prove $f$ is injective, expand the definition of injectivity.
\item To prove $A = B$, prove $A \subseteq B$ and $B \subseteq A$.
\item To prove $x \in A \cup B$, rewrite it as $x \in A \lor x \in B$.
\end{itemize}

One should never attempt to prove a named concept directly without first
expanding its definition.

\subsubsection*{Step 5: Introduce Helper Objects}

Introduce any auxiliary objects needed for the argument, such as witnesses,
intermediate elements, blocks of a partition, bounds, or images under functions.

Examples:
\[
\text{Let } x \in A,
\quad
\text{let } P_a \text{ be the block containing } a,
\quad
\text{let } b := f(a).
\]

No object should appear in a proof without being explicitly introduced.

\subsubsection*{Step 6: Apply One Definition or Property at a Time}

Each logical step should follow from exactly one of:
\begin{itemize}
\item a definition,
\item a hypothesis,
\item a previously proved theorem,
\item a basic logical rule.
\end{itemize}

If progress stalls, identify which definition has not yet been unpacked.
Avoid combining multiple logical moves into a single unexplained step.

\subsubsection*{Step 7: Use Forward and Backward Reasoning}

Proofs often alternate between two modes:
\begin{itemize}
\item \emph{Forward reasoning}: deducing consequences from the hypotheses.
\item \emph{Backward reasoning}: rewriting the goal to determine what would
suffice to prove it.
\end{itemize}

Backward reasoning is particularly effective when the conclusion is complex
or involves nested definitions.

\subsubsection*{Step 8: Handle Cases Explicitly}

If a statement naturally splits into cases, enumerate them clearly and exhaust
all possibilities.

For example:
\[
\text{Either } x \in A \text{ or } x \notin A.
\]

Each case must be treated separately, and together the cases must cover all
possibilities.

\subsubsection*{Step 9: Close the Argument}

Once the desired conclusion has been reached, explicitly state it and explain
why it completes the proof.

Examples:
\begin{itemize}
\item ``Thus $x \in B$, so $A \subseteq B$.''  
\item ``Hence $f$ is injective.''  
\item ``Therefore the two sets are equal.''
\end{itemize}

\subsubsection*{Step 10: Signal Completion}

Conclude cleanly with a standard closing such as
\[
\qed
\]
or an explicit statement that the proof is complete.

\subsection{Legal Moves in Proofs}

The following rules should be enforced throughout every proof:
\begin{itemize}
\item An existential witness may not be chosen before existence is proved.
\item An arbitrary element may only be introduced under universal scope.
\item The conclusion may never be assumed.
\item Definitions must be applied in full, not partially.
\end{itemize}

Violations of these rules invalidate the argument, even if the conclusion is
correct.

\subsection{Line-by-Line Proof Discipline}

For each line of a proof, the following questions should be silently checked:
\begin{enumerate}[label=(\roman*)]
\item Has every object appearing in the line been defined?
\item Which definition, hypothesis, or theorem justifies this step?
\item Does this step move the argument closer to the stated goal?
\end{enumerate}

If any answer is unclear, the line should be rewritten.

\subsection{Specialized Proof Templates}

The general algorithm specializes naturally to common proof types.

\paragraph{Implication ($P \rightarrow Q$).}
Assume $P$, expand its definition, deduce consequences, and conclude $Q$.

\paragraph{Biconditional ($P \leftrightarrow Q$).}
Prove $P \Rightarrow Q$ and $Q \Rightarrow P$ separately.

\paragraph{Set Equality ($A = B$).}
Prove $A \subseteq B$ and $B \subseteq A$.

\paragraph{Existence ($\exists x\, P(x)$).}
Explicitly define a candidate object and verify that it satisfies $P$.

\paragraph{Structural Properties.}
Statements such as ``$R$ is an equivalence relation'' or ``$f$ is a partial order''
should be expanded into their component properties and used selectively as needed.

\subsection{Stop Conditions}

A proof is complete when:
\begin{itemize}
\item a universal claim has been shown for an arbitrary element;
\item an existential claim has produced a valid witness;
\item a set equality has established both inclusions;
\item each direction of a biconditional has been proved.
\end{itemize}

Knowing when to stop is as important as knowing how to begin.

\subsection{Remark on Practice}

This procedure is intentionally explicit and mechanical.
With practice, these steps become internalized and are applied subconsciously.
Experienced mathematicians follow the same process, but omit intermediate steps
once correctness is assured.

Until fluency is achieved, it is preferable to write proofs that are slightly
longer but logically complete rather than shorter proofs with implicit gaps.
