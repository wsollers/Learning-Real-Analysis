% --------------------------------------
% Proof: AA-GAL-C00-E07
% --------------------------------------
\clearpage
\phantomsection
\hypertarget{proof-AA-GAL-C00-E07}{}

\section*{Proof — AA-GAL-C00-E07}

\noindent
\hyperlink{ws-AA-GAL-C00-E07}{\textbf{← Back to worksheet}}

\noindent
\textbf{Source.}
\srccite{GallianCAA7}{Chapter~0, Exercise~7}.

\vspace{0.75em}

\noindent
\textbf{Goal.}
Show that if $a$ and $b$ are positive integers, then
\[
ab = \operatorname{lcm}(a,b)\,\gcd(a,b).
\]

\vspace{0.75em}

\noindent



\begin{proof}
Let $a$ and $b$ be positive integers. Then
\[
  ab = \operatorname{lcm}(a,b)\,\gcd(a,b).
\]


Let $d = \gcd(a,b)$. Since $d \mid a$ and $d \mid b$, we may write
\[
  a = d a_1 \qquad \text{and} \qquad b = d b_1
\]
for positive integers $a_1$ and $b_1$.

We claim that $\gcd(a_1, b_1) = 1$. Suppose, for contradiction, that
$\gcd(a_1, b_1) = c$ for some integer $c > 1$. Then $c \mid a_1$ and
$c \mid b_1$, so $dc \mid da_1 = a$ and $dc \mid db_1 = b$. Thus $dc$ is
a common divisor of $a$ and $b$ with $dc > d$, contradicting the fact that
$d = \gcd(a,b)$ is the greatest common divisor. Hence $\gcd(a_1, b_1) = 1$.

We now show that $\operatorname{lcm}(a,b) = da_1 b_1$. First, $da_1 b_1 = ab_1$
is a multiple of $a$, and $da_1 b_1 = a_1 b$ is a multiple of $b$, so $da_1 b_1$
is a common multiple of $a$ and $b$. Second, let $c$ be any common multiple of
$a$ and $b$, and write $c = au = da_1 u$ and $c = bv = db_1 v$ for integers
$u$ and $v$. Then $a_1 u = b_1 v$, so $b_1 \mid a_1 u$. Since $\gcd(a_1, b_1) = 1$,
it follows that $b_1 \mid u$, say $u = b_1 w$ for some integer $w$. Then
\[
  c = au = da_1 b_1 w,
\]
so $da_1 b_1 \mid c$. Therefore $da_1 b_1$ is the least common multiple, that is,
$\operatorname{lcm}(a,b) = da_1 b_1$.

Finally,
\[
  \operatorname{lcm}(a,b)\,\gcd(a,b)
  = (da_1 b_1) \cdot d
  = d^2 a_1 b_1
  = (da_1)(db_1)
  = ab. \qedhere
\]
\end{proof}



\end{proof}

\vspace{2cm}
\noindent
\hyperlink{ws-AA-GAL-C00-E07}{\textbf{← Back to worksheet}}
