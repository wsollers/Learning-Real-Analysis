% =========================================================
% Sequences
% =========================================================

\subsection{Sequences}

% ---------------------------------------------------------
\subsubsection{Basic Definitions}

\begin{definition}[Sequence]
A \emph{sequence} in a set $X$ is a function
\[
x : \mathbb{N} \to X.
\]
For each $n\in\mathbb{N}$, the value $x(n)$ is denoted
\[
x_n := x(n),
\]
and the sequence is written
\[
(x_n)_{n\in\mathbb{N}}
\quad\text{or simply}\quad
(x_n).
\]
\end{definition}

\begin{remark}[Logical Form]
\[
\exists x\;
\bigl(
x:\mathbb{N}\to X
\ \wedge\
\forall n\in\mathbb{N},\ x_n = x(n)
\bigr).
\]
\end{remark}

\begin{remark}[Common Notation]
Equivalent notations:
\[
(x_n)_{n=1}^{\infty},
\qquad
(x_n),
\qquad
\{x_n\}_{n=1}^{\infty},
\qquad
n \mapsto x_n.
\]
All represent the same underlying function.
\end{remark}

\begin{remark}
When $X=\mathbb{R}$, we speak of a \emph{real sequence}.
When $X=\mathbb{R}^m$, we speak of a \emph{vector-valued sequence}.
\end{remark}

\begin{definition}[Subsequence]
Let $(x_n)$ be a sequence in $X$.
Let $(n_k)$ be a strictly increasing sequence in $\mathbb{N}$.
Then the sequence $(x_{n_k})$ is called a \emph{subsequence} of $(x_n)$.
\end{definition}

\begin{remark}
The strict monotonicity of $(n_k)$ ensures order preservation.
This guarantees that subsequences do not reorder terms but only delete some.
\end{remark}

\begin{remark}[Sources]
This functional definition of a sequence appears in
\cite{AbbottUnderstandingAnalysis},
\cite{TaoAnalysis1},
\cite{RossElementaryAnalysis},
\cite{Rudin1976},
and \cite{LeblBasicAnalysisI}.
\end{remark}

% ---------------------------------------------------------
\subsubsection{Main Theorems}

\begin{theorem}[A Sequence Is Determined by Its Values]
Let $x,y : \mathbb{N}\to X$ be sequences.
If
\[
\forall n\in\mathbb{N},\quad x_n = y_n,
\]
then $x=y$ as functions.
\end{theorem}

\begin{proof}
If $x(n)=y(n)$ for all $n\in\mathbb{N}$,
then the functions agree on their entire domain.
By the definition of equality of functions,
$x=y$.
\qed
\end{proof}

\begin{remark}
This theorem relies only on the definition of equality of functions.
It implies that sequence equality is pointwise equality.
\end{remark}

\begin{theorem}[Every Subsequence Is a Sequence]
If $(x_n)$ is a sequence in $X$ and $(n_k)$ is strictly increasing in $\mathbb{N}$,
then $(x_{n_k})$ is a sequence in $X$.
\end{theorem}

\begin{proof}
Define $y:\mathbb{N}\to X$ by $y(k)=x_{n_k}$.
Since $x$ maps $\mathbb{N}$ into $X$ and $n_k\in\mathbb{N}$,
the composition is well-defined.
Thus $(x_{n_k})$ is a function from $\mathbb{N}$ to $X$,
hence a sequence.
\qed
\end{proof}

\begin{remark}
This theorem depends on function composition.
It implies that subsequences remain within the same ambient space.
Later, this becomes crucial in Bolzano–Weierstrass and compactness arguments.
\end{remark}

% ---------------------------------------------------------
\subsubsection{Consequences}

The logical implication of this section is:

A sequence is not merely an ordered list,
but a function with domain $\mathbb{N}$.
Therefore:

\[
\text{Sequence theory} 
=
\text{Function theory on } \mathbb{N}.
\]

This structural viewpoint allows us to:
\begin{itemize}
\item Define limits using quantifiers over $\mathbb{N}$.
\item Treat subsequences as compositions.
\item Apply functional equality rigorously.
\end{itemize}

\begin{remark}[Logical Structure]
The foundational flow is:

\[
\text{Definition of Sequence}
\Rightarrow
\text{Equality Theorem}
\Rightarrow
\text{Subsequence Definition}
\Rightarrow
\text{Subsequence Theorem}.
\]

Thus all later limit theory rests on the functional interpretation.
\end{remark}


% ---------------------------------------------------------
\subsubsection{Canonical Examples}

\begin{example}[Constant sequence]
Fix $c\in\mathbb{R}$. Define
\[
x_n := c
\quad\text{for all } n\in\mathbb{N}.
\]
This is the simplest example of a sequence.
\end{example}

\begin{example}[Arithmetic sequence]
Given $a,d\in\mathbb{R}$, define
\[
x_n := a + (n-1)d.
\]
Each term differs from the previous one by the fixed increment $d$.
\end{example}

\begin{example}[Geometric sequence]
Given $a,r\in\mathbb{R}$, define
\[
x_n := a r^{\,n-1}.
\]
Each term is obtained by multiplying the previous one by the ratio $r$.
\end{example}

\begin{example}[Harmonic sequence]
\[
x_n := \frac{1}{n}.
\]
This sequence decreases to $0$.
\end{example}

\begin{example}[Alternating sequence]
\[
x_n := (-1)^n.
\]
This sequence oscillates between $1$ and $-1$.
\end{example}

\begin{example}[Polynomial growth sequence]
Let $k\in\mathbb{N}$. Define
\[
x_n := n^k.
\]
This sequence grows without bound.
\end{example}

\begin{example}[Exponential decay sequence]
\[
x_n := \frac{1}{2^n}.
\]
This sequence decreases rapidly toward $0$.
\end{example}

\begin{remark}[Purpose of Canonical Examples]
These examples serve as test cases for:
\begin{itemize}
\item boundedness,
\item monotonicity,
\item convergence and divergence,
\item oscillation,
\item growth rates.
\end{itemize}
Nearly every theorem about sequences can be sanity-checked against this list.
\end{remark}

% ---------------------------------------------------------
\subsubsection{Logical Classification Table}

\begin{center}
\renewcommand{\arraystretch}{1.2}
\begin{tabular}{|l|c|c|c|c|}
\hline
\textbf{Sequence} 
& \textbf{Bounded} 
& \textbf{Monotone} 
& \textbf{Convergent} 
& \textbf{Divergent} \\
\hline

Constant $x_n = c$ 
& Yes 
& Yes 
& Yes 
& No \\

Arithmetic $x_n = a+(n-1)d$ 
& If $d=0$ 
& If $d\ge0$ or $d\le0$ 
& If $d=0$ 
& If $d\ne0$ \\

Geometric $x_n = ar^{n-1}$ 
& If $|r|\le1$ 
& If $r\ge0$ 
& If $|r|<1$ 
& If $|r|>1$ \\

Harmonic $x_n=\frac1n$ 
& Yes 
& Decreasing 
& Yes 
& No \\

Alternating $x_n=(-1)^n$ 
& Yes 
& No 
& No 
& Yes \\

Polynomial $x_n=n^k$ 
& No 
& Increasing 
& No 
& Yes \\

Exponential decay $x_n=\frac1{2^n}$ 
& Yes 
& Decreasing 
& Yes 
& No \\

\hline
\end{tabular}
\end{center}

\begin{remark}[Logical Dependencies]
The classifications rely on later results:

\begin{itemize}
\item Monotone sequences converge if and only if they are bounded
      (Monotone Convergence Theorem).
\item Convergent sequences are bounded.
\item Divergence may occur via unbounded growth or oscillation.
\end{itemize}

Thus this table previews the structure of future theorems.
\end{remark}


