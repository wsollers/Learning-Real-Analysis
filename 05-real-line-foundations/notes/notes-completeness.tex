% =========================================================
% Completeness and Its Consequences
% =========================================================

\subsection{Completeness of the Real Numbers}

% ---------------------------------------------------------
\subsubsection{Completeness Axiom}
% ---------------------------------------------------------

\begin{axiom}[Completeness Axiom (Least Upper Bound Property)]
Every nonempty subset of $\mathbb{R}$ that is bounded above has a supremum in $\mathbb{R}$.

Equivalently: if $S\subseteq\mathbb{R}$ is nonempty and bounded above,
then $\sup S$ exists as a real number.
\end{axiom}

\begin{remark}[Logical form]
\[
\forall S\;\Bigl(
(S\subseteq\mathbb{R}\wedge S\neq\varnothing\wedge 
\exists M\in\mathbb{R}\ \forall x\in S\ (x\le M))
\rightarrow
\exists s\in\mathbb{R}\ (s=\sup S)
\Bigr).
\]
\end{remark}

\begin{remark}[Equivalent formulations]
Over the ordered-field axioms, completeness is equivalent to:
\begin{itemize}
\item Every nonempty set bounded below has an infimum.
\item Every Cauchy sequence in $\mathbb{R}$ converges.
\item (Nested interval property) Every nested sequence of nonempty closed intervals with lengths $\to 0$ has nonempty intersection.
\end{itemize}
\end{remark}

% =========================================================
% Nested Interval Property
% =========================================================

\subsubsection{Nested Interval Property}

\begin{theorem}[Nested Interval Property]
Let $\{[a_n,b_n]\}_{n\in\mathbb{N}}$ be nonempty closed intervals such that
\[
[a_{n+1},b_{n+1}] \subseteq [a_n,b_n]
\quad\text{for all } n.
\]
Then
\[
\bigcap_{n=1}^\infty [a_n,b_n] \neq \varnothing.
\]
If additionally $b_n-a_n \to 0$, the intersection consists of exactly one point.
\end{theorem}

\begin{proof}
From the nesting we have
\[
a_n \le a_{n+1}, \qquad b_{n+1} \le b_n.
\]
Thus $(a_n)$ is increasing and bounded above by $b_1$.
By completeness,
\[
a := \sup\{a_n\}
\]
exists.

Since $a_n \le a$ for all $n$, and for each fixed $n$ every $a_k$ with $k\ge n$
lies in $[a_n,b_n]$, we have $a \le b_n$.
Thus $a_n \le a \le b_n$, so
\[
a \in [a_n,b_n]
\quad \text{for all } n.
\]

Hence the intersection is nonempty.

If $b_n-a_n \to 0$ and $x,y$ lie in the intersection,
then
\[
|x-y| \le b_n-a_n \to 0,
\]
so $x=y$.
\end{proof}

% =========================================================
% Archimedean Property
% =========================================================

\subsubsection{Archimedean Property}

\begin{definition}
$\mathbb{R}$ satisfies the Archimedean property if
\[
\forall x\in\mathbb{R}\;\exists n\in\mathbb{N}\;(n>x).
\]
Equivalently:
\[
\forall x>0\,\forall y\in\mathbb{R}\;\exists n\in\mathbb{N}\;(nx>y).
\]
\end{definition}

\begin{remark}[Logical form]
\[
\forall x\;\exists n\;(n>x).
\]
\[
\forall x>0\,\forall y\;\exists n\;(nx>y).
\]
\end{remark}

\begin{theorem}
$\mathbb{R}$ satisfies the Archimedean property.
\end{theorem}

\begin{proof}
If not, then $\mathbb{N}$ is bounded above.
By completeness $s:=\sup\mathbb{N}$ exists.
Then $s-1$ is not an upper bound, so there exists $m\in\mathbb{N}$ with $m>s-1$.
Thus $m+1>s$, contradicting that $s$ is an upper bound.
\end{proof}

\begin{corollary}
If $x>0$ and $y\in\mathbb{R}$, then $\exists n\in\mathbb{N}$ such that $nx>y$.
\end{corollary}

\begin{proof}
Apply the Archimedean property to $y/x$.
\end{proof}

% =========================================================
% Integer Part
% =========================================================

\subsubsection{Integer Part}

\begin{lemma}[Floor lemma]
For every $x\in\mathbb{R}$ there exists $m\in\mathbb{Z}$ such that
\[
m \le x < m+1.
\]
\end{lemma}

% =========================================================
% Density
% =========================================================

\subsubsection{Density}

\begin{definition}
A subset $A$ of a linearly ordered set $X$ is \emph{dense in $X$} if
\[
\forall a<b \;\exists c\in A\;(a<c<b).
\]
\end{definition}

\begin{remark}[Logical form for $\mathbb{Q}$ dense in $\mathbb{R}$]
\[
\forall a<b \;\exists q\in\mathbb{Q}\;(a<q<b).
\]
\end{remark}

\begin{theorem}[Density of $\mathbb{Q}$]
$\mathbb{Q}$ is dense in $\mathbb{R}$.
\end{theorem}

\begin{proof}
Let $a<b$. Choose $n$ with $1/n<b-a$.
Let $m\le na < m+1$ and set $q=(m+1)/n$.
Then $a<q<b$.
\end{proof}

\begin{corollary}
Between any two rationals lies an irrational.
\end{corollary}

\begin{corollary}
The irrationals are dense in $\mathbb{R}$.
\end{corollary}

% =========================================================
% Square Roots
% =========================================================

\subsubsection{Existence of Square Roots}

\begin{theorem}
For every $a\ge 0$ there exists a unique $x\ge 0$ such that $x^2=a$.
\end{theorem}

\begin{remark}[Logical form]
\[
\forall a\ge 0\;\exists!\,x\ge 0\;(x^2=a).
\]
\end{remark}

\begin{proof}[Existence]
Let
\[
S=\{x\ge 0 : x^2\le a\}.
\]
$S$ is nonempty and bounded above, so $s=\sup S$ exists.

If $s^2>a$, then $s-\varepsilon$ is an upper bound for small $\varepsilon$.
If $s^2<a$, then $s+\delta\in S$ for small $\delta$.
Both contradict the definition of supremum.
Hence $s^2=a$.
\end{proof}

\begin{proof}[Uniqueness]
If $x^2=y^2$ and $x,y\ge 0$, then $(x-y)(x+y)=0$.
Since $x+y>0$, $x=y$.
\end{proof}

% =========================================================
% Structural Summary
% =========================================================

\subsubsection*{Structural Summary}

\[
\text{Field Axioms}
\Rightarrow
\text{Order Axioms}
\Rightarrow
\text{Completeness}
\]

Completeness yields:

\[
\text{Nested Intervals}
\Rightarrow
\text{Archimedean Property}
\Rightarrow
\text{Density}
\Rightarrow
\text{Existence of }\sqrt{a}
\]

Completeness is the property that prevents “holes” in $\mathbb{R}$.
It is equivalent to the Cauchy Criterion and underlies all major limit theorems.
