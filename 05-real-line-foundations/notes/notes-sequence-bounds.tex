% =========================================================
% Bounded Sequences and Types of Bounds
% =========================================================

\subsection{Bounded Sequences and Types of Bounds}

\subsubsection{Bounded Sequences}

% ---------------------------------------------------------
\subsubsection{Basic Definitions}

\begin{definition}[Bounded sequence]
A sequence $(x_n)$ in $\mathbb{R}$ is called \emph{bounded} if there exists a real
number $M > 0$ such that
\[
|x_n| \le M
\quad\text{for all } n \in \mathbb{N}.
\]
\end{definition}

\begin{remark}[Equivalent form]
A sequence $(x_n)$ is bounded if and only if there exist real numbers
$m$ and $M$ such that
\[
m \le x_n \le M
\quad\text{for all } n \in \mathbb{N}.
\]
\end{remark}

\begin{remark}[Logical form]
\[
\exists M > 0\;
\forall n \in \mathbb{N}\;
\bigl(|x_n| \le M\bigr).
\]
\[
\exists m, M \in \mathbb{R}\;
\forall n \in \mathbb{N}\;
\bigl(m \le x_n \le M\bigr).
\]
\end{remark}

\begin{remark}[Three structural viewpoints of boundedness]
A sequence may be understood as bounded at three distinct but equivalent levels:
\begin{enumerate}
  \item \textbf{Function viewpoint.}
        A sequence $\{x_n\}_{n=1}^\infty$ is a function
        $x : \mathbb{N} \to \mathbb{R}$, $x(n) = x_n$.
        Boundedness is the condition that this function is bounded on
        its domain.

  \item \textbf{Set (range) viewpoint.}
        The values form the subset
        $\{x_n : n \in \mathbb{N}\} \subseteq \mathbb{R}$.
        The sequence is bounded if and only if this range is a bounded
        subset of $\mathbb{R}$.

  \item \textbf{Order-theoretic viewpoint.}
        Boundedness is not a property of the index set $\mathbb{N}$
        but of the range in $\mathbb{R}$. The sequence bounds (upper and
        lower) are exactly the set-theoretic bounds of the range, as
        defined in Section~\ref{sec:bounds}.
\end{enumerate}
These perspectives are logically equivalent but conceptually distinct,
and each becomes useful in different parts of analysis.
\end{remark}

% ---------------------------------------------------------
% Types of Bounds for Sequences
% ---------------------------------------------------------

\begin{definition}[Upper bound for a sequence]
A real number $M$ is an \emph{upper bound} for the sequence $(x_n)$ if
\[
x_n \le M
\quad\text{for all } n \in \mathbb{N}.
\]
\end{definition}

\begin{definition}[Lower bound for a sequence]
A real number $m$ is a \emph{lower bound} for the sequence $(x_n)$ if
\[
m \le x_n
\quad\text{for all } n \in \mathbb{N}.
\]
\end{definition}

\begin{definition}[Bounded above / below]
A sequence $(x_n)$ is said to be
\begin{itemize}
  \item \emph{bounded above} if it has at least one upper bound;
  \item \emph{bounded below} if it has at least one lower bound;
  \item \emph{bounded} if it is both bounded above and bounded below.
\end{itemize}
\end{definition}

\begin{remark}
These are special cases of the set-theoretic definitions in
Section~\ref{sec:bounds}, applied to the range
$\{x_n : n \in \mathbb{N}\}$.
\end{remark}

% ---------------------------------------------------------
% Least Upper and Greatest Lower Bounds of a Sequence
% ---------------------------------------------------------

\begin{definition}[Supremum of a sequence]
If the set $\{x_n : n \in \mathbb{N}\}$ is bounded above, its least upper
bound is called the \emph{supremum} of the sequence, denoted
\[
\sup\{x_n\}
\quad\text{or}\quad
\sup_{n \in \mathbb{N}} x_n.
\]
\end{definition}

\begin{definition}[Infimum of a sequence]
If the set $\{x_n : n \in \mathbb{N}\}$ is bounded below, its greatest
lower bound is called the \emph{infimum} of the sequence, denoted
\[
\inf\{x_n\}
\quad\text{or}\quad
\inf_{n \in \mathbb{N}} x_n.
\]
\end{definition}

\begin{definition}[Maximum and minimum of a sequence]
A sequence $(x_n)$ has a \emph{maximum} if there exists $N \in \mathbb{N}$
such that
\[
x_N \ge x_n
\quad\text{for all } n \in \mathbb{N}.
\]
A sequence $(x_n)$ has a \emph{minimum} if there exists $N \in \mathbb{N}$
such that
\[
x_N \le x_n
\quad\text{for all } n \in \mathbb{N}.
\]
\end{definition}

\begin{remark}
If a maximum exists, then $\max\{x_n\} = \sup\{x_n\}$,
and similarly for a minimum and the infimum.
The supremum or infimum of a sequence need not be attained by any term.
\end{remark}

% ---------------------------------------------------------
\subsubsection{Main Theorems}

% This section is definitional in nature.
% The main theorems involving bounded sequences
% (e.g.\ every convergent sequence is bounded)
% appear in the convergence section.

% ---------------------------------------------------------
\subsubsection{Consequences}

\[
\text{Bounded sequence}
\Longleftrightarrow
\text{Range } \{x_n\} \text{ is bounded in } \mathbb{R}.
\]

Thus sequence boundedness reduces to set boundedness, and all
results about suprema and infima of sets apply directly to sequences.

\begin{remark}[Logical Structure]
\[
\text{Upper/Lower bounds}
\Rightarrow
\text{Boundedness}
\Rightarrow
\text{Supremum/Infimum}
\Rightarrow
\text{Maximum/Minimum}.
\]
All later extremal and limit arguments depend on this hierarchy.
\end{remark}