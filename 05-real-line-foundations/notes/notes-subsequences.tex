% =========================================================
% Subsequences
% File: notes-subsequences.tex
% =========================================================

\subsection{Subsequences}

% (Original local heading preserved)
\subsubsection{Subsequence}

% ---------------------------------------------------------
\subsubsection{Basic Definitions}

\begin{definition}[Subsequence]
Let $(a_n)$ be a sequence in $\mathbb{R}$.
A sequence $(a_{n_k})$ is called a \emph{subsequence} of $(a_n)$ if
there exists a function
\[
k \mapsto n_k
\]
from $\mathbb{N}$ to $\mathbb{N}$ such that
\begin{enumerate}
    \item $n_k < n_{k+1}$ for all $k \in \mathbb{N}$, and
    \item the $k$th term of the new sequence is $a_{n_k}$.
\end{enumerate}
\end{definition}

\begin{remark}
A subsequence is obtained by selecting terms from $(a_n)$ without duplication
while preserving their original order.
\end{remark}

% ---------------------------------------------------------
\subsubsection{Main Theorems}

\begin{theorem}[Subsequences inherit the same limit]
Let $(a_n)$ be a sequence of real numbers and let $(a_{n_k})$ be any subsequence,
where $(n_k)$ is strictly increasing.
If $a_n \to L$, then $a_{n_k} \to L$.
\end{theorem}

\begin{proof}
Assume $a_n \to L$. Let $\varepsilon > 0$ be arbitrary.
By definition of convergence, there exists $N \in \mathbb{N}$ such that
\[
n \ge N \implies |a_n - L| < \varepsilon.
\]

Since $(n_k)$ is strictly increasing and unbounded in $\mathbb{N}$,
there exists $K$ such that
\[
k \ge K \implies n_k \ge N.
\]

Thus for all $k \ge K$,
\[
|a_{n_k} - L| < \varepsilon.
\]

Hence $a_{n_k} \to L$.
\end{proof}

\begin{remark}[Uniqueness of limits]
This theorem implies that limits of sequences are unique.
If a sequence had two distinct limits, then every subsequence
would converge to both limits, which is impossible.
\end{remark}

\begin{remark}[Detecting nonconvergence]
If a sequence admits two subsequences converging to different limits,
then the original sequence cannot converge.
\end{remark}

% ---------------------------------------------------------

\begin{theorem}[Bolzano--Weierstrass]
Every bounded sequence of real numbers has a convergent subsequence.
Equivalently: if $(a_n)$ is bounded in $\mathbb{R}$, then there exist
$L \in \mathbb{R}$ and a strictly increasing sequence $(n_k)$ in $\mathbb{N}$
such that $a_{n_k} \to L$.
\end{theorem}

\begin{proof}
Let $(a_n)$ be bounded. Then there exist real numbers $m<M$ such that
$a_n \in [m,M]$ for all $n \in \mathbb{N}$.

We construct nested closed intervals
\[
I_k = [\alpha_k,\beta_k] \subseteq [m,M]
\]
such that:

\begin{enumerate}
\item $I_{k+1} \subseteq I_k$,
\item $\beta_k-\alpha_k = \dfrac{M-m}{2^{k-1}}$,
\item infinitely many terms lie in each $I_k$.
\end{enumerate}

Start with $I_1=[m,M]$.
Given $I_k=[\alpha_k,\beta_k]$, let $c_k=\dfrac{\alpha_k+\beta_k}{2}$ and split

\[
I_k = [\alpha_k,c_k]\ \cup\ [c_k,\beta_k].
\]

At least one half contains infinitely many terms.
Define $I_{k+1}$ to be such a half.

Because the intervals are nested and their lengths tend to $0$,
their intersection contains exactly one point.
Let
\[
L = \bigcap_{k=1}^\infty I_k.
\]

Now construct $n_1<n_2<\cdots$ with $a_{n_k}\in I_k$.
This is possible because each $I_k$ contains infinitely many terms.

Let $\varepsilon>0$.
Choose $K$ such that $\beta_K-\alpha_K<\varepsilon$.
For $k\ge K$,
\[
a_{n_k}, L \in I_K
\]
so
\[
|a_{n_k}-L|\le \beta_K-\alpha_K < \varepsilon.
\]

Thus $a_{n_k}\to L$.
\end{proof}

\begin{corollary}[Existence of subsequential limits]
Every bounded sequence admits at least one subsequential limit.
\end{corollary}

\begin{corollary}[Sequential compactness of closed intervals]
Every sequence contained in a closed bounded interval
$[a,b]$ has a convergent subsequence whose limit lies in $[a,b]$.
\end{corollary}

\begin{proof}
If $(a_n)\subseteq [a,b]$, then it is bounded.
By Bolzano--Weierstrass it has a convergent subsequence.
Since $[a,b]$ is closed, the limit lies in $[a,b]$.
\end{proof}

% ---------------------------------------------------------
\subsubsection{Consequences}

This section establishes two structural principles.

\medskip

\textbf{(1) Stability of Limits.}

\[
a_n \to L
\Rightarrow
a_{n_k} \to L.
\]

Convergence is preserved under passage to subsequences.

\medskip

\textbf{(2) Emergence of Subsequence Limits.}

\[
\text{Bounded}
\Rightarrow
\text{Convergent subsequence}.
\]

Boundedness alone does not ensure convergence,
but it guarantees partial convergence.

\medskip

\begin{remark}[Logical Structure]

\[
\text{Convergent}
\Rightarrow
\text{All subsequences converge to same limit},
\]

\[
\text{Bounded}
\Rightarrow
\text{Bolzano--Weierstrass}
\Rightarrow
\text{Existence of subsequential limits}.
\]

These results form the bridge between basic convergence
theory and completeness theory.

Bolzano--Weierstrass is equivalent (in an ordered field)
to:
\begin{itemize}
\item The Monotone Convergence Theorem,
\item The Cauchy Criterion,
\item The Nested Interval Property.
\end{itemize}

\end{remark}
