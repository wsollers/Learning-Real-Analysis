% =========================================================
% Sequences toolkit
% File: notes-toolkit.tex
% =========================================================

\subsection{Sequences Toolkit}

% =========================================================
% Subsequence Decomposition Lemma
% =========================================================

\begin{lemma}[Subsequence Decomposition Lemma]
Let $(a_n)$ be a sequence in $\mathbb{R}$ and let $k \in \mathbb{N}$ with $k \ge 1$.
For each $r \in \{0,1,\dots,k-1\}$, define the subsequence
\[
(a_{kn+r})_{n=0}^{\infty}.
\]
Then:

\begin{enumerate}
\item The index set $\mathbb{N}$ decomposes as
\[
\mathbb{N}
=
\bigcup_{r=0}^{k-1}
\{\, kn+r : n \in \mathbb{N} \,\}.
\]

\item The sets
\[
\{\, kn+r : n \in \mathbb{N} \,\}
\quad\text{and}\quad
\{\, kn+s : n \in \mathbb{N} \,\}
\]
are disjoint whenever $r \ne s$.
\end{enumerate}

Consequently, every term $a_m$ of the sequence belongs to exactly one of the subsequences
\[
(a_{kn}), (a_{kn+1}), \dots, (a_{kn+k-1}).
\]
\end{lemma}

\begin{theorem}[Residue-Class Convergence Criterion]
Let $(a_n)$ be a sequence in $\mathbb{R}$ and let $k \ge 1$.
If for each $r \in \{0,1,\dots,k-1\}$ the subsequence
\[
(a_{kn+r})
\]
converges to the same limit $L$, then $(a_n)$ converges to $L$.
\end{theorem}

% =========================================================
% Residue-Class Convergence Theorem
% =========================================================

\begin{theorem}[Residue-Class Convergence Theorem]
Let $(a_n)$ be a sequence in $\mathbb{R}$ and let $k \in \mathbb{N}$ with $k \ge 1$.
For each $r \in \{0,1,\dots,k-1\}$, consider the subsequence
\[
(a_{kn+r})_{n=0}^{\infty}.
\]

If for every $r \in \{0,1,\dots,k-1\}$ the subsequence $(a_{kn+r})$
converges to the same limit $L$, then the full sequence $(a_n)$
converges to $L$.
\end{theorem}

\begin{proof}
Let $\varepsilon > 0$ be arbitrary.

For each $r \in \{0,1,\dots,k-1\}$, since
\[
(a_{kn+r}) \to L,
\]
there exists $N_r \in \mathbb{N}$ such that for all $n \ge N_r$,
\[
|a_{kn+r} - L| < \varepsilon.
\]

Define
\[
N := \max_{0 \le r \le k-1} \{ kN_r + r \}.
\]

Now let $m \in \mathbb{N}$ with $m \ge N$.
By the Division Algorithm, there exist unique integers $n \in \mathbb{N}$
and $r \in \{0,1,\dots,k-1\}$ such that
\[
m = kn + r.
\]

Because $m \ge kN_r + r$, it follows that $n \ge N_r$.
Hence,
\[
|a_m - L|
=
|a_{kn+r} - L|
<
\varepsilon.
\]

Since $\varepsilon>0$ was arbitrary, we conclude that
\[
(a_n) \to L.
\]
\end{proof}

% =========================================================
% Finite Partition Convergence Principle
% =========================================================

\begin{theorem}[Finite Partition Convergence Principle]
Let $(a_n)$ be a sequence in a metric space $(X,d)$.
Suppose the index set $\mathbb{N}$ admits a finite partition
\[
\mathbb{N} = E_1 \cup E_2 \cup \cdots \cup E_k,
\]
where:

\begin{enumerate}
\item $E_i \cap E_j = \varnothing$ whenever $i \ne j$,
\item each $E_i$ is infinite.
\end{enumerate}

For each $i \in \{1,\dots,k\}$, let $(a_n)_{n \in E_i}$ denote the
subsequence indexed by $E_i$.

If each subsequence $(a_n)_{n \in E_i}$ converges to the same limit $L \in X$,
then the full sequence $(a_n)$ converges to $L$.
\end{theorem}

\begin{proof}
Let $\varepsilon > 0$ be arbitrary.

For each $i \in \{1,\dots,k\}$, since $(a_n)_{n \in E_i} \to L$,
there exists $N_i \in \mathbb{N}$ such that for all $n \in E_i$ with $n \ge N_i$,
\[
d(a_n, L) < \varepsilon.
\]

Define
\[
N := \max\{N_1, \dots, N_k\}.
\]

Now let $m \ge N$.
Since $\mathbb{N} = \bigcup_{i=1}^k E_i$, there exists a unique
$i$ such that $m \in E_i$.

Because $m \ge N \ge N_i$ and $m \in E_i$, we have
\[
d(a_m, L) < \varepsilon.
\]

Since $\varepsilon > 0$ was arbitrary,
\[
(a_n) \to L.
\]
\end{proof}


% =========================================================
% Structural Transfer of Sequence Properties
% =========================================================

\begin{center}
\begin{tabular}{|c|c|c|c|}
\hline
\textbf{Property} 
& \textbf{Inherited by Subsequences?} 
& \textbf{Reflected by One Subsequence?}
& \textbf{Structural Type} \\
\hline
Convergent 
& Yes 
& No 
& Tail Property \\
\hline
Cauchy 
& Yes 
& No 
& Tail Property \\
\hline
Bounded 
& Yes 
& No 
& Tail Property \\
\hline
Monotone 
& Yes 
& No 
& Global Structural Property \\
\hline
Eventually monotone 
& Yes 
& No 
& Tail Property \\
\hline
Every subsequence convergent 
& --- 
& Yes 
& Universal Subsequence Property \\
\hline
Every subsequence Cauchy 
& --- 
& Yes 
& Universal Subsequence Property \\
\hline
\end{tabular}
\end{center}

% =========================================================
% Structural Classes of Sequence Properties
% =========================================================

\begin{definition}[Tail Property]
A property $\mathcal{P}$ of sequences is called a \emph{tail property}
if whenever two sequences $(a_n)$ and $(b_n)$ satisfy
\[
a_n = b_n \quad \text{for all } n \ge N
\]
for some $N \in \mathbb{N}$, then
\[
(a_n) \text{ satisfies } \mathcal{P}
\quad \Longleftrightarrow \quad
(b_n) \text{ satisfies } \mathcal{P}.
\]
In other words, altering finitely many initial terms does not affect
whether the sequence has property $\mathcal{P}$.
\end{definition}

\begin{definition}[Universal Subsequence Property]
A property $\mathcal{P}$ of sequences is called a
\emph{universal subsequence property}
if for every sequence $(a_n)$,
\[
(a_n) \text{ satisfies } \mathcal{P}
\quad \Longleftrightarrow \quad
\text{every subsequence of } (a_n) \text{ satisfies } \mathcal{P}.
\]
\end{definition}






% =========================================================
% Residue Classes Modulo k
% =========================================================

\begin{definition}[Residue classes modulo $k$]
Let $k \in \mathbb{N}$ with $k \ge 1$. 
For each $r \in \{0,1,\dots,k-1\}$, define the \emph{residue class modulo $k$} by
\[
[r]_k 
:=
\{\, m \in \mathbb{Z} : m \equiv r \pmod{k} \,\}.
\]
\end{definition}

\begin{proposition}[Partition property]
Let $k \in \mathbb{N}$, $k \ge 1$. Then the collection
\[
\{[0]_k, [1]_k, \dots, [k-1]_k\}
\]
forms a partition of $\mathbb{Z}$. That is:

\begin{enumerate}
\item \textbf{Union property:}
\[
\mathbb{Z}
=
\bigcup_{r=0}^{k-1} [r]_k.
\]

\item \textbf{Disjointness property:}
For $r \ne s$,
\[
[r]_k \cap [s]_k = \varnothing.
\]
\end{enumerate}
\end{proposition}

\begin{example}[Residue classes modulo $3$]
The three residue classes modulo $3$ are:
\[
[0]_3 = \{ 0,3,6,9,\dots \},
\]
\[
[1]_3 = \{ 1,4,7,10,\dots \},
\]
\[
[2]_3 = \{ 2,5,8,11,\dots \}.
\]

They satisfy
\[
\mathbb{Z} = [0]_3 \cup [1]_3 \cup [2]_3
\]
and
\[
[0]_3 \cap [1]_3
=
[1]_3 \cap [2]_3
=
[0]_3 \cap [2]_3
=
\varnothing.
\]
\end{example}





