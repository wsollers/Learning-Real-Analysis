% --------------------------------------
% Proof: RA-ABB-C01-S1.3-E03A
% --------------------------------------
\clearpage
\phantomsection
\hypertarget{proof-RA-ABB-C01-S1.3-E03A}{}

\section*{Proof — RA-ABB-C01-S1.3-E03A}

\noindent
\hyperlink{ws-RA-ABB-C01-S1.3-E03A}{\textbf{← Back to worksheet}}

\noindent
\textbf{Source.}
\srccite{AbbottUnderstandingAnalysis}{Chapter~1, \S1.3, Exercise~1.3.3(a)}.

\vspace{0.75em}

\noindent
\textbf{Goal.}
Let $A$ be nonempty and bounded below, and define
\[
B=\{\, b\in\mathbb{R} : b \text{ is a lower bound for } A \,\}.
\]
Show that $\sup B = \inf A$.

\vspace{0.75em}

\noindent
\textbf{Proof.}

\begin{proof}
Let $A \subset \mathbb{R}$ be nonempty and bounded below, and define
\[
B := \{\, b \in \mathbb{R} : b \text{ is a lower bound for } A \,\}.
\]

First observe that if $b \in B$ and $a \in A$, then $b \le a$ by definition of
lower bound. Consequently, every $a \in A$ is an upper bound for $B$.

Let
\[
s := \sup B \quad \text{and} \quad i := \inf A.
\]

Since every element of $A$ is an upper bound for $B$, we have $s \le a$ for all
$a \in A$. Hence $s$ is a lower bound for $A$, so $s \in B$.

Next, let $b \in B$ be arbitrary. Since $b$ is a lower bound for $A$, we have
$b \le a$ for all $a \in A$, and in particular $b \le i = \inf A$.
Thus $i$ is an upper bound for $B$.

Because $s = \sup B$ is the least upper bound of $B$, it follows that
\[
s \le i.
\]
On the other hand, since $s \in B$ and $i$ is an upper bound for $B$, we have
\[
i \le s.
\]
Therefore $s = i$, and hence
\[
\sup B = \inf A.
\]
\end{proof}


\vspace{1em}
\noindent
\hyperlink{ws-RA-ABB-C01-S1.3-E03A}{\textbf{← Back to worksheet}}
