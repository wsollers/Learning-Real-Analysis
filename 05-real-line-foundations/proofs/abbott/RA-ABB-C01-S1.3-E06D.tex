
% --------------------------------------
% Proof: RA-ABB-C01-S1.3-E06D
% --------------------------------------
\clearpage
\phantomsection
\hypertarget{proof-RA-ABB-C01-S1.3-E06D}{}

\section*{Proof — RA-ABB-C01-S1.3-E06D}

\noindent
\hyperlink{ws-RA-ABB-C01-S1.3-E06D}{\textbf{← Back to worksheet}}

\noindent
\textbf{Source.}
\srccite{AbbottUnderstandingAnalysis}{Chapter~1, \S1.3, Exercise~1.3.6(d)}.

\vspace{0.75em}

\noindent
\textbf{Goal.} Construct another proof of $\sup(A+B)=\sup A+\sup B$ using Lemma~1.3.8.

\vspace{0.75em}

\noindent
\textbf{Proof (stub).}
\begin{itemize}
\item Write down what Lemma~1.3.8 says (in your own words or symbolically).
\item Apply it to $s=\sup A$ and $t=\sup B$ to produce elements $a\in A$ and $b\in B$ close to $s$ and $t$.
\item Use these to show elements of $A+B$ get arbitrarily close to $s+t$ from below.
\item Combine with “$s+t$ is an upper bound” to conclude it is the supremum.
\end{itemize}

\vspace{1em}
\noindent
\hyperlink{ws-RA-ABB-C01-S1.3-E06D}{\textbf{← Back to worksheet}}