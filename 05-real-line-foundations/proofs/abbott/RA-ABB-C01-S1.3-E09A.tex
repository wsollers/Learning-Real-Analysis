% --------------------------------------
% Proof: RA-ABB-C01-S1.3-E09A
% --------------------------------------
\clearpage
\phantomsection
\hypertarget{proof-RA-ABB-C01-S1.3-E09A}{}

\section*{Proof — RA-ABB-C01-S1.3-E09A}

\noindent
\hyperlink{ws-RA-ABB-C01-S1.3-E09A}{\textbf{← Back to worksheet}}

\noindent
\textbf{Source.}
\srccite{AbbottUnderstandingAnalysis}{Chapter~1, \S1.3, Exercise~1.3.9(a)}.

\vspace{0.75em}

\noindent
\textbf{Goal.}
If $\sup A < \sup B$, show that there exists an element $b\in B$ that is an upper bound for $A$.

\vspace{0.75em}

\noindent
\textbf{Proof.}

\begin{definition}[Supremum — $\varepsilon$ Definition]
Let $A \subseteq \mathbb{R}$ be nonempty and bounded above, and let $s \in \mathbb{R}$.  
Then $s = \sup A$ if and only if:
\begin{enumerate}
\item $s$ is an upper bound for $A$, and
\item for every $\varepsilon > 0$, there exists $a \in A$ such that
\[
s - \varepsilon < a.
\]
\end{enumerate}
\end{definition}

\vspace{1em}
\noindent
\hyperlink{ws-RA-ABB-C01-S1.3-E09A}{\textbf{← Back to worksheet}}
