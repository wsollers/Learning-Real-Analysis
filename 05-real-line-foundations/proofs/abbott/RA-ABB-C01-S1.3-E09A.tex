% --------------------------------------
% Proof: RA-ABB-C01-S1.3-E09A
% --------------------------------------
\clearpage
\phantomsection
\hypertarget{proof-RA-ABB-C01-S1.3-E09A}{}

\section*{Proof — RA-ABB-C01-S1.3-E09A}

\noindent
\hyperlink{ws-RA-ABB-C01-S1.3-E09A}{\textbf{← Back to worksheet}}

\noindent
\textbf{Source.}
\srccite{AbbottUnderstandingAnalysis}{Chapter~1, \S1.3, Exercise~1.3.9(a)}.

\vspace{0.75em}

\noindent
\textbf{Goal.}
If $\sup A < \sup B$, show that there exists an element $b\in B$ that is an upper bound for $A$.

\vspace{0.75em}

\noindent
\textbf{Proof (stub).}
\begin{itemize}
\item Let $s=\sup A$ and $t=\sup B$ with $s<t$.
\item Use the defining property of supremum for $B$ to find $b\in B$ with $t-\varepsilon < b \le t$ for a well-chosen $\varepsilon$.
\item Choose $\varepsilon$ in terms of $t-s$ so that $b>s$.
\item Conclude $b$ is an upper bound for $A$ because $a\le s < b$ for all $a\in A$.
\end{itemize}

\vspace{1em}
\noindent
\hyperlink{ws-RA-ABB-C01-S1.3-E09A}{\textbf{← Back to worksheet}}
