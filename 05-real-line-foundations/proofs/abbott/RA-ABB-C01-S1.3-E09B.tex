% --------------------------------------
% Proof: RA-ABB-C01-S1.3-E09B
% --------------------------------------
\clearpage
\phantomsection
\hypertarget{proof-RA-ABB-C01-S1.3-E09B}{}

\section*{Proof — RA-ABB-C01-S1.3-E09B}

\noindent
\hyperlink{ws-RA-ABB-C01-S1.3-E09B}{\textbf{← Back to worksheet}}

\noindent
\textbf{Source.}
\srccite{AbbottUnderstandingAnalysis}{Chapter~1, \S1.3, Exercise~1.3.9(b)}.

\vspace{0.75em}

\noindent
\textbf{Goal.} Give an example to show that the conclusion in (a) is not always the case if we only assume $\sup A \le \sup B$.

\vspace{0.75em}

\noindent
\textbf{Proof (stub).}
\begin{itemize}
\item Seek $A,B$ with $\sup A=\sup B$ but such that \emph{no} element of $B$ is an upper bound for $A$.
\item Strategy: make $A$ “cluster up to” its supremum, and make $B$ lie strictly below that supremum while still having the same supremum (so $B$ does not attain it).
\item Verify: compute $\sup A$ and $\sup B$, check $\sup A\le \sup B$, then prove: $\forall b\in B,\ \exists a\in A\text{ with }a>b$.
\end{itemize}

\vspace{1em}
\noindent
\hyperlink{ws-RA-ABB-C01-S1.3-E09B}{\textbf{← Back to worksheet}}
