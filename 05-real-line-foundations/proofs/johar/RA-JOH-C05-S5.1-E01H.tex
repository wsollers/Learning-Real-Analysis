% --------------------------------------
% Proof: RA-JOH-C05-S5.1-E01H
% --------------------------------------
\clearpage
\phantomsection
\hypertarget{proof-RA-JOH-C05-S5.1-E01H}{}

\section*{Proof — RA-JOH-C05-S5.1-E01H}

\noindent
\hyperlink{ws-RA-JOH-C05-S5.1-E01H}{\textbf{← Back to worksheet}}

\noindent
\textbf{Source.}
\srccite{JoharBigBookRealAnalysis}{Chapter~5, Exercise~5.1(h)}.

\vspace{0.75em}

\noindent
\textbf{Goal.}
Let $(a_n)$ be defined by
\[
a_n = n-\sqrt{n^2+\sqrt{n}}.
\]
Show that $a_n \to 0$. In $\varepsilon$--$N$ form: for a fixed $\varepsilon>0$, find
$N(\varepsilon)\in\mathbb{N}$ such that $|a_n|<\varepsilon$ for all $n\ge N(\varepsilon)$.

\vspace{0.75em}

\begin{proof}
\end{proof}

\vspace{1em}
\noindent
\hyperlink{ws-RA-JOH-C05-S5.1-E01H}{\textbf{← Back to worksheet}}