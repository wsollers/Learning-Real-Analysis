% --------------------------------------
% Proof: RA-JOH-C05-S5-7
% --------------------------------------
\clearpage
\phantomsection
\hypertarget{proof-RA-JOH-C05-S5-7}{}

\section*{Proof — RA-JOH-C05-S5-7}

\noindent
\hyperlink{ws-RA-JOH-C05-S5-7}{\textbf{← Back to worksheet}}

\noindent
\textbf{Source.}
\srccite{JoharBigBookRealAnalysis}{Chapter~5, Exercise~5.7}.

\vspace{0.75em}

\noindent
\textbf{Goal.}
Let $(a_n)$ be a real sequence.
\begin{enumerate}[label=(\alph*)]
\item Suppose that the even-indexed subsequence $(a_{2n})$ and the odd-indexed subsequence $(a_{2n+1})$
both converge to the same limit. Prove that the whole sequence $(a_n)$ also converges to the same limit.
\item More generally, assume that the subsequences $(a_{3n})$, $(a_{3n+1})$, and $(a_{3n+2})$
all converge to the same limit. Show that the whole sequence $(a_n)$ also converges to the same limit.
\end{enumerate}

\vspace{0.75em}


\begin{proof}
\[
\]
(a) \\
Assume $a_{2n} \to L$ and $a_{2n+1} \to L$.

Let $\varepsilon > 0$ be given.

Since $a_{2n} \to L$, there exists $N_1 \in \mathbb{N}$ such that for all $n \geq N_1$,
\[
|a_{2n} - L| < \varepsilon.
\]

Since $a_{2n+1} \to L$, there exists $N_2 \in \mathbb{N}$ such that for all $n \geq N_2$,
\[
|a_{2n+1} - L| < \varepsilon.
\]

Let $N = \max\{2N_1, 2N_2 + 1\}$.

Let $n \geq N$ be arbitrary. Either $n$ is even or $n$ is odd.

\textbf{Case 1:} $n$ is even. Then $n = 2k$ for some $k \in \mathbb{N}$. 
Since $n \geq 2N_1$, we have $k \geq N_1$, so
\[
|a_n - L| = |a_{2k} - L| < \varepsilon.
\]

\textbf{Case 2:} $n$ is odd. Then $n = 2k + 1$ for some $k \in \mathbb{N}$. 
Since $n \geq 2N_2 + 1$, we have $k \geq N_2$, so
\[
|a_n - L| = |a_{2k+1} - L| < \varepsilon.
\]

In both cases, $|a_n - L| < \varepsilon$. Since $\varepsilon > 0$ was arbitrary, 
$a_n \to L$. \\
(b) \\
Assume that $(a_{3n}) \to a$, $(a_{3n+1}) \to a$, and $(a_{3n+2}) \to a$.

Let $\varepsilon>0$ be arbitrary.

Since $(a_{3n}) \to a$, there exists $N_0 \in \mathbb{N}$ such that for all $n \ge N_0$,
\[
|a_{3n}-a|<\varepsilon.
\]
Since $(a_{3n+1}) \to a$, there exists $N_1 \in \mathbb{N}$ such that for all $n \ge N_1$,
\[
|a_{3n+1}-a|<\varepsilon.
\]
Since $(a_{3n+2}) \to a$, there exists $N_2 \in \mathbb{N}$ such that for all $n \ge N_2$,
\[
|a_{3n+2}-a|<\varepsilon.
\]

Let
\[
N := \max\{3N_0,\; 3N_1+1,\; 3N_2+2\}.
\]

Now let $m \in \mathbb{N}$ with $m \ge N$. We show that $|a_m-a|<\varepsilon$.
There are three cases:

\textbf{Case 1:} $m \equiv 0 \pmod 3$. Then $m=3n$ for some $n\in\mathbb{N}$, and from $m \ge 3N_0$ we get $n \ge N_0$.
Hence $|a_m-a| = |a_{3n}-a|<\varepsilon$.

\textbf{Case 2:} $m \equiv 1 \pmod 3$. Then $m=3n+1$ for some $n\in\mathbb{N}$, and from $m \ge 3N_1+1$ we get $n \ge N_1$.
Hence $|a_m-a| = |a_{3n+1}-a|<\varepsilon$.

\textbf{Case 3:} $m \equiv 2 \pmod 3$. Then $m=3n+2$ for some $n\in\mathbb{N}$, and from $m \ge 3N_2+2$ we get $n \ge N_2$.
Hence $|a_m-a| = |a_{3n+2}-a|<\varepsilon$.

In all cases, $|a_m-a|<\varepsilon$ for all $m \ge N$. Therefore $(a_n) \to a$.
\end{proof}

\vspace{1em}
\noindent
\hyperlink{ws-RA-JOH-C05-S5-7}{\textbf{← Back to worksheet}}
