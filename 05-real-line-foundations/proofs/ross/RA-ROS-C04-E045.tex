% =========================================================
% Ross — Elementary Analysis (2e), Chapter 4
% Proof Pages: Exercises 4.5–4.12
% One page per (sub)problem. NO proof stubs included.
% =========================================================

% --------------------------------------
% Proof: RA-ROS-C04-E045
% --------------------------------------
\clearpage
\phantomsection
\hypertarget{proof-RA-ROS-C04-E045}{}

\section*{Proof — RA-ROS-C04-E045}

\noindent
\hyperlink{ws-RA-ROS-C04-E045}{\textbf{← Back to worksheet}}

\noindent
\textbf{Source.}
\srccite{RossElementaryAnalysis}{Chapter~4, Exercise~4.5}.

\vspace{0.75em}

\noindent
\textbf{Goal.}
Let $S$ be a nonempty subset of $\mathbb{R}$ that is bounded above.
Prove: if $\sup S \in S$, then $\sup S=\max S$.

\vspace{0.75em}


\noindent
\textbf{Proof.}

Let $S \subseteq \mathbb{R}$ be nonempty and bounded above, and assume that
$\sup S \in S$.
Define
\[
s_0 := \sup S.
\]

By definition of the supremum, $s_0$ is an upper bound for $S$, so
\[
s \le s_0 \quad \text{for all } s \in S.
\]
Since $s_0 \in S$ by assumption, it follows that $s_0$ is an element of $S$
that is greater than or equal to every element of $S$.

By definition of a maximum, this shows that $s_0 = \max S$.
Therefore,
\[
\sup S = \max S.
\]
\qed






\vspace{6cm}

\noindent
\hyperlink{ws-RA-ROS-C04-E045}{\textbf{← Back to worksheet}}

