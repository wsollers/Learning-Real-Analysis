% reading-list.tex

% =========================================================
% REAL ANALYSIS — MULTI-AUTHOR, MULTI-PERSPECTIVE SYLLABUS
% Depth-Oriented, Rudin-Aligned
% =========================================================



\noindent
This syllabus is designed for deep mastery of real analysis.
Each topic is studied through multiple authors to expose distinct
motivations, proof styles, and problem sets.
The plan is aligned to \textit{Rudin, Principles of Mathematical Analysis}
(3rd ed.), but aims for understanding beyond procedural fluency.

Each topic is allocated approximately three days of focused study.

% ---------------------------------------------------------
\section*{Topic 1: Order Structure, Bounds, Suprema, and Infima}
\addcontentsline{toc}{section}{Topic 1: Order Structure, Bounds, Suprema, and Infima}

\textbf{Core ideas:}
ordered fields, upper/lower bounds, least upper bound property.

\begin{itemize}
  \item \textbf{Abbott} — Ch.~1 §§1.3--1.4  
  (Careful intuition-building; student-friendly proofs)
  
  \item \textbf{Ross, Elementary Analysis} — Ch.~3  
  (Clean axiomatic development; many worked examples)
  
  \item \textbf{Bruckner--Thomson} — Ch.~1 §§1--3  
  (Technical precision; early exposure to pathological sets)
  
  \item \textbf{Kolmogorov--Fomin} — Ch.~1 §§2--3  
  (Structural, classical approach; minimal hand-holding)
  
  \item \textbf{Johar} — Ch.~1 (Order \& Bounds sections)  
  (Breadth of examples; exploratory problems)
  
  \item \textbf{Mendelson, Number Systems} — Ch.~3  
  (Order-theoretic foundations; abstraction-first)
  
  \item \textbf{Lebl} — Ch.~1 (Bounds discussion)  
  (Modern exposition; excellent exercises)
\end{itemize}

\textbf{Goal:}
Internalize the distinction between bounds, extrema, and least/greatest bounds.

% ---------------------------------------------------------
\section*{Topic 2: Completeness of \(\mathbb{R}\)}
\addcontentsline{toc}{section}{Topic 2: Completeness of \(\mathbb{R}\)}

\textbf{Core ideas:}
completeness axiom, Dedekind cuts, failure of completeness in \(\mathbb{Q}\).

\begin{itemize}
  \item \textbf{Abbott} — Ch.~1 §1.5  
  (Conceptual motivation; supremum-based approach)
  
  \item \textbf{Ross, Elementary Analysis} — Ch.~4  
  (Order-theoretic clarity)
  
  \item \textbf{Bruckner--Thomson} — Ch.~1 §4  
  (Tight proofs; early applications)
  
  \item \textbf{Mendelson} — Ch.~4  
  (Construction of \(\mathbb{R}\); logical foundations)
  
  \item \textbf{Tao} — Ch.~1 §§1.6--1.7  
  (Axiomatic minimalism; sharp lemmas)
  
  \item \textbf{Belding} — Ch.~3  
  (Abstract framing; proof discipline)
\end{itemize}

\textbf{Goal:}
Understand completeness as the defining analytic property of \(\mathbb{R}\).

% ---------------------------------------------------------
\section*{Topic 3: Sequences and Limits}
\addcontentsline{toc}{section}{Topic 3: Sequences and Limits}

\textbf{Core ideas:}
convergence, subsequences, uniqueness of limits.

\begin{itemize}
  \item \textbf{Abbott} — Ch.~2 §§2.1--2.3  
  (Pedagogical clarity; strong intuition)
  
  \item \textbf{Ross, Elementary Analysis} — Ch.~5  
  (Formal epsilon--$N$ control)
  
  \item \textbf{Bruckner--Thomson} — Ch.~2 §§1--3  
  (Counterexamples and edge cases)
  
  \item \textbf{Kolmogorov--Fomin} — Ch.~2 §§1--2  
  (Classical exposition)
  
  \item \textbf{Tao} — Ch.~2 §§2.1--2.2  
  (Concise, lemma-driven approach)
  
  \item \textbf{Lebl} — Ch.~2  
  (Modern structure; extensive problem sets)
\end{itemize}

\textbf{Goal:}
Be able to prove convergence rigorously and recognize false intuitions.

% ---------------------------------------------------------
\section*{Topic 4: Cauchy Sequences and Completeness Revisited}
\addcontentsline{toc}{section}{Topic 4: Cauchy Sequences and Completeness Revisited}

\textbf{Core ideas:}
Cauchy criterion, internal vs external convergence.

\begin{itemize}
  \item \textbf{Abbott} — Ch.~2 §§2.4--2.6  
  (Bridging convergence and completeness)
  
  \item \textbf{Ross, Elementary Analysis} — Chs.~6--7  
  (Systematic development)
  
  \item \textbf{Bruckner--Thomson} — Ch.~2 §§4--6  
  (Pathologies and subtleties)
  
  \item \textbf{Tao} — Ch.~2 §§2.3--2.4  
  (Abstract framing)
  
  \item \textbf{Magnus} — Ch.~4  
  (Metric-space generalization)
\end{itemize}

\textbf{Goal:}
Understand completeness independently of limits.

% ---------------------------------------------------------
\section*{Topic 5: Open and Closed Sets in \(\mathbb{R}\)}
\addcontentsline{toc}{section}{Topic 5: Open and Closed Sets in \(\mathbb{R}\)}

\textbf{Core ideas:}
topology of the real line, neighborhoods, limit points.

\begin{itemize}
  \item \textbf{Abbott} — Ch.~3 §3.1  
  (Concrete intuition)
  
  \item \textbf{Ross, Elementary Analysis} — Ch.~8  
  (Formal topological language)
  
  \item \textbf{Magnus} — Ch.~2  
  (Metric-space perspective)
  
  \item \textbf{Bruckner--Thomson} — Ch.~3  
  (Technical detail)
  
  \item \textbf{Mendelson, Topology} — Ch.~2  
  (Abstract topological grounding)
  
  \item \textbf{Lebl} — Ch.~3  
  (Clear synthesis)
\end{itemize}

\textbf{Goal:}
Switch effortlessly between metric, topological, and sequential viewpoints.

% ---------------------------------------------------------
\section*{Topic 6: Compactness}
\addcontentsline{toc}{section}{Topic 6: Compactness}

\textbf{Core ideas:}
Heine--Borel, sequential compactness, finite subcovers.

\begin{itemize}
  \item \textbf{Abbott} — Ch.~3 §3.2  
  (Motivation and examples)
  
  \item \textbf{Ross, Elementary Analysis} — Ch.~9  
  (Systematic proofs)
  
  \item \textbf{Magnus} — Ch.~5  
  (Metric-space generalization)
  
  \item \textbf{Bruckner--Thomson} — Ch.~4  
  (Subtle examples)
  
  \item \textbf{Tao} — Ch.~3  
  (Elegant equivalences)
  
  \item \textbf{Lebl} — Ch.~4  
  (Excellent exercises)
\end{itemize}

\textbf{Goal:}
Recognize compactness as a unifying principle across analysis.

% ---------------------------------------------------------
\section*{Topic 7: Continuity and Uniform Continuity}
\addcontentsline{toc}{section}{Topic 7: Continuity and Uniform Continuity}

\textbf{Core ideas:}
$\varepsilon$--$\delta$ continuity, uniform behavior, compactness effects.

\begin{itemize}
  \item \textbf{Abbott} — Ch.~4  
  (Multiple equivalent definitions)
  
  \item \textbf{Ross, Elementary Analysis} — Chs.~11--13  
  (Logical structure)
  
  \item \textbf{Magnus} — Ch.~6  
  (Metric-space continuity)
  
  \item \textbf{Tao} — Ch.~4  
  (Sharp abstraction)
  
  \item \textbf{Bruckner--Thomson} — Ch.~5  
  (Counterexamples)
  
  \item \textbf{Lebl} — Ch.~5  
  (Exercises emphasizing technique)
\end{itemize}

\textbf{Goal:}
Control continuity arguments in all equivalent forms.

% ---------------------------------------------------------
\section*{Topic 8: Differentiation}
\addcontentsline{toc}{section}{Topic 8: Differentiation}

\textbf{Core ideas:}
derivative definition, mean value theorem, consequences.

\begin{itemize}
  \item \textbf{Abbott} — Ch.~5  
  (Conceptual grounding)
  
  \item \textbf{Ross, Elementary Analysis} — Chs.~14--15  
  (Formal proofs)
  
  \item \textbf{Bruckner--Thomson} — Ch.~6  
  (Advanced examples)
  
  \item \textbf{Tao} — Ch.~5  
  (Structural insight)
  
  \item \textbf{Johar} — Ch.~7  
  (Breadth of problems)
\end{itemize}

\textbf{Goal:}
Understand differentiation as a limit process, not a computational rule.

% ---------------------------------------------------------
\section*{Topic 9: Integration and the Fundamental Theorem}
\addcontentsline{toc}{section}{Topic 9: Integration and the Fundamental Theorem}

\textbf{Core ideas:}
Riemann integration, integrability criteria, FTC.

\begin{itemize}
  \item \textbf{Abbott} — Ch.~6  
  (Clear motivation)
  
  \item \textbf{Ross, Elementary Analysis} — Chs.~16--17  
  (Rigorous construction)
  
  \item \textbf{Bruckner--Thomson} — Ch.~7  
  (Technical depth)
  
  \item \textbf{Tao} — Ch.~6  
  (Conceptual framing)
  
  \item \textbf{Johar} — Ch.~8  
  (Extended examples)
\end{itemize}

\textbf{Goal:}
See integration as a structural inverse to differentiation.

% ---------------------------------------------------------
\section*{Topic 10: Series and Uniform Convergence}
\addcontentsline{toc}{section}{Topic 10: Series and Uniform Convergence}

\textbf{Core ideas:}
pointwise vs uniform convergence, interchange of limits.

\begin{itemize}
  \item \textbf{Abbott} — Ch.~7  
  (Motivation and examples)
  
  \item \textbf{Ross, Elementary Analysis} — Chs.~18--19  
  (Formal criteria)
  
  \item \textbf{Bruckner--Thomson} — Ch.~8  
  (Pathological constructions)
  
  \item \textbf{Tao} — Ch.~7  
  (Structural understanding)
  
  \item \textbf{Lebl} — Ch.~7  
  (Problem-driven mastery)
\end{itemize}

\textbf{Goal:}
Control limits of functions rigorously.

\bigskip
\noindent
\textbf{Outcome:}
Completion of this syllabus provides deep, multi-perspective mastery of real
analysis and prepares the reader to engage confidently with Rudin's text and
its exercises.
