% =========================================================
% Worksheet: Johar — Sequences and Limits (Exercise 5.1)
% File: 05-real-line-foundations/worksheets/johar.tex
% =========================================================

\subsection{Johar}

\noindent\textbf{Source.} Syafiq Johar, \textit{The Big Book of Real Analysis}.

\vspace{0.75em}
\begin{center}
\begin{tabular}{|p{5.0cm}|p{9.0cm}|}
\hline
\textbf{Problem ID} & \textbf{Exercise (descriptor)} \\
\hline

% -------------------------
% Chapter 5, Exercise 5.1 — $\varepsilon$--$N$ proofs that $a_n \to 0$
% -------------------------

\phantomsection
\hypertarget{ws-RA-JOH-C05-S5.1-E01A}{}
\hyperlink{proof-RA-JOH-C05-S5.1-E01A}{\texttt{RA-JOH-C05-S5.1-E01A}}
&
Johar, Chapter~5, Ex.~5.1(a) — Show $a_n\to 0$ for $a_n=\dfrac{1}{n^2+3}$ (find $N(\varepsilon)$).
\\ \hline

\phantomsection
\hypertarget{ws-RA-JOH-C05-S5.1-E01B}{}
\hyperlink{proof-RA-JOH-C05-S5.1-E01B}{\texttt{RA-JOH-C05-S5.1-E01B}}
&
Johar, Chapter~5, Ex.~5.1(b) — Show $a_n\to 0$ for $a_n=\dfrac{1}{n-\frac{5}{2}}$ (find $N(\varepsilon)$).
\\ \hline

\phantomsection
\hypertarget{ws-RA-JOH-C05-S5.1-E01C}{}
\hyperlink{proof-RA-JOH-C05-S5.1-E01C}{\texttt{RA-JOH-C05-S5.1-E01C}}
&
Johar, Chapter~5, Ex.~5.1(c) — Show $a_n\to 0$ for $a_n=\dfrac{1}{n\left(n-\frac{1}{2}\right)}$ (find $N(\varepsilon)$).
\\ \hline

\phantomsection
\hypertarget{ws-RA-JOH-C05-S5.1-E01D}{}
\hyperlink{proof-RA-JOH-C05-S5.1-E01D}{\texttt{RA-JOH-C05-S5.1-E01D}}
&
Johar, Chapter~5, Ex.~5.1(d) — Show $a_n\to 0$ for $a_n=\dfrac{1}{\sqrt{5n}-1}$ (find $N(\varepsilon)$).
\\ \hline

\phantomsection
\hypertarget{ws-RA-JOH-C05-S5.1-E01E}{}
\hyperlink{proof-RA-JOH-C05-S5.1-E01E}{\texttt{RA-JOH-C05-S5.1-E01E}}
&
Johar, Chapter~5, Ex.~5.1(e) — Show $a_n\to 0$ for $a_n=\dfrac{\sin(n)}{n}$ (find $N(\varepsilon)$).
\\ \hline

\phantomsection
\hypertarget{ws-RA-JOH-C05-S5.1-E01F}{}
\hyperlink{proof-RA-JOH-C05-S5.1-E01F}{\texttt{RA-JOH-C05-S5.1-E01F}}
&
Johar, Chapter~5, Ex.~5.1(f) — Show $a_n\to 0$ for
$
a_n=
\begin{cases}
\dfrac{1}{2^n}, & \text{if $n$ is prime or $n=1$,}\\
\dfrac{1}{3^n}, & \text{if $n$ is not prime.}
\end{cases}
$
(find $N(\varepsilon)$).
\\ \hline

\phantomsection
\hypertarget{ws-RA-JOH-C05-S5.1-E01G}{}
\hyperlink{proof-RA-JOH-C05-S5.1-E01G}{\texttt{RA-JOH-C05-S5.1-E01G}}
&
Johar, Chapter~5, Ex.~5.1(g) — Show $a_n\to 0$ for $a_n=\sqrt{n+1}-\sqrt{n}$ (find $N(\varepsilon)$).
\\ \hline

\phantomsection
\hypertarget{ws-RA-JOH-C05-S5.1-E01H}{}
\hyperlink{proof-RA-JOH-C05-S5.1-E01H}{\texttt{RA-JOH-C05-S5.1-E01H}}
&
Johar, Chapter~5, Ex.~5.1(h) — Show $a_n\to 0$ for $a_n=n-\sqrt{n^2+\sqrt{n}}$ (find $N(\varepsilon)$).
\\ \hline

\end{tabular}
\end{center}
