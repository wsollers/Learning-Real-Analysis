% Fragment (no \documentclass)

\hypertarget{TB-P1-C01-004}{}
\subsubsection*{TB-P1-C01-004: Characterization of the Supremum}

\noindent\textbf{Navigation: }
\hyperlink{TB-P1-C01-WS-001}{Back to worksheet}
\quad|\quad
\hyperlink{P1-TB-C01}{Back to chapter}

\medskip

\noindent\textbf{Claim.}
Let $S \subseteq \mathbb{R}$ be nonempty and bounded above.
Prove that $\alpha = \sup S$ if and only if:
\begin{enumerate}[label=(\roman*)]
\item $\alpha$ is an upper bound of $S$, and
\item for every $\varepsilon > 0$ there exists $s \in S$ such that $\alpha - \varepsilon < s$.
\end{enumerate}

\medskip

\noindent\textbf{Proof.}

($\implies$)

Assume $\alpha = \sup S$.

By the definition of supremum, $\alpha$ is an upper bound of $S$.

Let $\varepsilon > 0$ be arbitrary.
If $\alpha - \varepsilon \ge s$ for all $s \in S$, then $\alpha - \varepsilon$ is an upper bound of $S$ with
$\alpha - \varepsilon < \alpha$.
This contradicts the minimality of $\alpha$ as an upper bound.

\medskip

($\Longleftarrow$)

Assume that $\alpha$ is an upper bound of $S$, and that for every $\varepsilon > 0$ there exists
$s \in S$ such that $\alpha - \varepsilon < s$.

Assume, for contradiction, that there exists an upper bound $\beta$ of $S$ such that $\beta < \alpha$.
Define $\varepsilon := \alpha - \beta > 0$.
By the assumption above, there exists $s \in S$ such that
\[
\alpha - \varepsilon < s.
\]
Substituting $\varepsilon = \alpha - \beta$ gives $\beta < s$, contradicting the assumption that
$\beta$ is an upper bound of $S$.

Therefore, no upper bound of $S$ is strictly less than $\alpha$, and hence $\alpha = \sup S$.

\qed
