% Fragment (no \documentclass)

\hypertarget{TB-P1-C01-002}{}\n\subsubsection*{TB-P1-C01-002: Set difference distributes over union}\n\n\noindent\textbf{Navigation: }
\hyperlink{TB-P1-C01-WS-001}{Back to worksheet}\n\quad|\quad
\hyperlink{P1-TB-C01}{Back to chapter}\n\n\textbf

{Claim.} Prove $A\setminus (B\cup C) = (A\setminus B)\cap (A\setminus C)$.\n\n\textbf

{Proof.}

\medskip
\noindent\textbf{($\subseteq$)}

Consider sets A, B, C.

Let $x \in A\setminus (B\cup C)$.

By definition of set difference, $x \in A \land x \notin (B \cup C)$.

By definition of union, $x \notin (B \cup C) \implies x \notin B \land x \notin C$.

Thus $(x \in A \land x \notin B) \land (x \in A \land x \notin C)$.

Therefore $x \in (A\setminus B) \cap (A\setminus C)$.

\medskip
\noindent\textbf{($\supseteq$)}

Let $x \in (A\setminus B) \cap (A\setminus C)$.

By definition of intersection, $x \in (A \setminus B) \land x \in (A \setminus C)$.

From $x \in A \setminus B$ conclude that  $x \in$ A and $x \notin B$.

From $x \in A \setminus C$ conclude that  $x \in$ A and $x \notin C$.


By definition of union $x \notin B \land x \notin C \implies x \notin (B \cup C)$.

Thus $x \in A \land x \notin (B \cup C).$

Therefore $x \in A\setminus (B\cup C)$. 
\qed
