% Fragment (no \documentclass)

\hypertarget{TB-P1-C01-001}{}\n\subsubsection*{TB-P1-C01-001: De Morgan's law}\n\n\noindent\textbf{Navigation: }
\hyperlink{TB-P1-C01-WS-001}{Back to worksheet}\n\quad|\quad
\hyperlink{P1-TB-C01}{Back to chapter}\n\n\textbf

{Claim.} Prove $(A\cup B)^c = A^c \cap B^c$ (complements relative to a fixed universe).\n\n\textbf

{Proof.}



\begin{proof}
Let $U$ be the fixed universe and let $A,B\subseteq U$. We prove equality by showing mutual inclusion.

\medskip
\noindent\textbf{($\subseteq$)} Let $x\in (A\cup B)^c$. By definition of complement, this means $x\notin A\cup B$.
By definition of union, $x\notin A\cup B$ implies $(x\notin A)\ \wedge\ (x\notin B)$.
Equivalently, $x\in A^c$ and $x\in B^c$. Hence $x\in A^c\cap B^c$.

\medskip
\noindent\textbf{($\supseteq$)} Let $x\in A^c\cap B^c$. Then $x\in A^c$ and $x\in B^c$, so $x\notin A$ and $x\notin B$.
Therefore $x\notin A\cup B$ (since membership in a union would require membership in at least one of the sets).
Thus $x\in (A\cup B)^c$.

\medskip
Since $(A\cup B)^c\subseteq A^c\cap B^c$ and $A^c\cap B^c\subseteq (A\cup B)^c$, we conclude
\[
(A\cup B)^c = A^c\cap B^c.\qedhere
\]
\end{proof}




\qed
