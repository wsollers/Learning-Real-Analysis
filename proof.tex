Let $S \subset \mathbb{R}$ be a nonempty set that is bounded above.

We define the set
\[
T := \{\, x \in \mathbb{R} \mid x \le s \text{ for all } s \in S \,\}.
\]

Part (a): Lower Bounds and Nonemptiness

Prove that 
��
T is nonempty and bounded above.

Goal: Demonstrate that you understand what it means to exhibit a lower bound without guessing one, and how boundedness transfers between sets.

Assume S to be a non-empty subset of R bounded below.

By definition of a lower bound there exists a point p that is less than or equal to every member of S.

Every element in T is less than or equal to the point p. Therefore p is an upper bound of T.


, and let s be an arbitrary element of S.

By definition, there exists an element t in T that is <= s. Therefore, T is non-empty.

Every element in T is less than or equal to every member of S, therefore T is bounded above.

Assume S to be a non-empty subset of R bounded above.
By definition of upper bound, there exists a point p that is greater than every element of S.
To prove that M is non-empty, it suffices to show that p is a member of M.
Since p is an upper bound of S, p is a member of M.
Therefore M is non-empty.
To be bounded below, we must show there is a point less than or equal to every element in M.
p is less than every member of M therefore M is bounded below.

infM=supS.