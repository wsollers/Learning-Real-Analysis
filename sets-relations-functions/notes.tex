% notes.tex

\subsection{Sets and Membership}

\begin{definition}[Primitive Notions]
In axiomatic set theory, the notions of \emph{set} and \emph{membership} are
\emph{primitive}.

\begin{itemize}
  \item A \emph{set} is an object.
  \item \emph{Membership} is a binary relation, denoted by $\in$, between objects.
\end{itemize}

If $x$ is an object and $A$ is a set, the statement
\[
x \in A
\]
is read as ``$x$ is an element of $A$.''

No definition of ``set'' or ``$\in$'' is given in more basic terms.
Their meaning is determined entirely by the axioms governing them.
\end{definition}

\subsection{Zermelo--Fraenkel Axioms with Choice (ZFC)}

\begin{axiom}[Axiom of Extensionality]
Two sets are equal if and only if they have the same elements.

Formally,
\[
\forall A \, \forall B \,
\Bigl( A = B \iff \forall x \, (x \in A \leftrightarrow x \in B) \Bigr).
\]
\end{axiom}

\begin{axiom}[Axiom of Empty Set]
There exists a set with no elements.

Formally,
\[
\exists A \, \forall x \, (x \notin A).
\]
\end{axiom}

\begin{axiom}[Axiom of Pairing]
For any two sets, there exists a set containing exactly those two sets.

Formally,
\[
\forall A \, \forall B \, \exists C \,
\forall x \, (x \in C \leftrightarrow (x = A \lor x = B)).
\]
\end{axiom}

\begin{axiom}[Axiom of Union]
For any set of sets, there exists a set containing exactly the elements of those sets.

Formally,
\[
\forall A \, \exists U \,
\forall x \,
\bigl( x \in U \leftrightarrow \exists B \, (B \in A \land x \in B) \bigr).
\]
\end{axiom}

\begin{axiom}[Axiom of Power Set]
For any set, there exists a set of all its subsets.

Formally,
\[
\forall A \, \exists P \,
\forall x \, (x \in P \leftrightarrow x \subseteq A).
\]
\end{axiom}

\begin{axiom}[Axiom of Infinity]
There exists an infinite set.

Formally,
\[
\exists A \,
\Bigl( \varnothing \in A \land
\forall x \, (x \in A \rightarrow x \cup \{x\} \in A) \Bigr).
\]
\end{axiom}

\begin{axiom}[Axiom Schema of Separation]
Given a set and a property, there exists a subset containing exactly the elements
satisfying that property.

Formally, for any formula $\varphi(x)$,
\[
\forall A \, \exists B \,
\forall x \, (x \in B \leftrightarrow (x \in A \land \varphi(x))).
\]
\end{axiom}

\begin{axiom}[Axiom Schema of Replacement]
For any formula $\varphi(x,y)$, if $\varphi$ defines a functional relation on a set $A$,
then the image of $A$ under $\varphi$ is a set.

Formally, for any formula $\varphi(x,y)$,
\[
\forall A\,
\Bigl(
\bigl(\forall x \in A\, \exists! y\, \varphi(x,y)\bigr)
\rightarrow
\exists B\, \forall y\,
\bigl(y \in B \leftrightarrow \exists x \in A\, \varphi(x,y)\bigr)
\Bigr).
\]
\end{axiom}


\begin{axiom}[Axiom of Foundation]
Every nonempty set has an $\in$-minimal element.

Formally,
\[
\forall A \,
\Bigl( A \neq \varnothing \rightarrow
\exists x \in A \, (x \cap A = \varnothing) \Bigr).
\]
\end{axiom}

\begin{axiom}[Axiom of Choice]
For any family of nonempty sets, there exists a function selecting one element
from each set.

Formally,
\[
\forall A \,
\Bigl(
(\forall B \in A \, (B \neq \varnothing))
\rightarrow
\exists f \,
\forall B \in A \, (f(B) \in B)
\Bigr).
\]
\end{axiom}

\begin{remark}[Role of the Axioms]
The axioms listed above constitute the foundational assumptions of formal set theory.
They are not statements to be proved, but rules that specify how the membership
relation $\in$ behaves and which sets are permitted to exist.

All subsequent notions—such as subsets, ordered pairs, relations, functions,
number systems, and topological or metric structures—are \emph{definitions}
introduced within this axiomatic framework. Every result proved later is a
\emph{theorem} derived logically from these axioms together with the rules of
inference.

From this point onward, set theory functions as the underlying language of
mathematics: we will no longer appeal to the axioms explicitly, but all reasoning
about mathematical objects is ultimately grounded in them.
\end{remark}




\subsection{Derived Set-Theoretic Constructions}

\begin{definition}[Empty Set]
The unique set with no elements is called the \emph{empty set} and is denoted
by $\varnothing$.
\end{definition}


\begin{remark}[The Empty Set and Vacuous Truth]
Statements involving universal quantification over the empty set are
\emph{vacuously true}.

In particular, if $A = \varnothing$, then any statement of the form
\[
\forall x \in A \, P(x)
\]
is true, since there exists no element $x \in A$ for which the statement $P(x)$
could fail.

This convention is not a logical curiosity but a necessary feature of the formal
system: it ensures that general definitions and theorems apply uniformly to all
sets, including the empty set, without requiring exceptional cases.
\end{remark}


\begin{definition}[Subset]
Let $A$ and $B$ be sets.
We say that $A$ is a \emph{subset} of $B$, and write $A \subseteq B$, if every
element of $A$ is also an element of $B$.

Formally,
\[
A \subseteq B \;\;\Longleftrightarrow\;\; \forall x \, (x \in A \rightarrow x \in B).
\]
\end{definition}

\begin{definition}[Proper Subset]
Let $A$ and $B$ be sets.
We say that $A$ is a \emph{proper subset} of $B$, and write $A \subsetneq B$, if
$A$ is a subset of $B$ but $A$ is not equal to $B$.

Formally,
\[
A \subsetneq B \;\;\Longleftrightarrow\;\; (A \subseteq B \land A \neq B).
\]
\end{definition}

\begin{remark}
Some authors write $A \subset B$ to denote a proper subset, while others use
$A \subset B$ to mean $A \subseteq B$.
In this text, we reserve $\subseteq$ for subsets and $\subsetneq$ for proper subsets.
\end{remark}




\begin{definition}[Set Equality]
Let $A$ and $B$ be sets.
We say that $A$ and $B$ are \emph{equal}, and write $A = B$, if they have the same
elements.

Formally,
\[
A = B
\;\;\Longleftrightarrow\;\;
\forall x \, (x \in A \leftrightarrow x \in B).
\]

Equivalently, two sets are equal if and only if each is a subset of the other:
\[
A = B
\;\;\Longleftrightarrow\;\;
(A \subseteq B \;\land\; B \subseteq A).
\]
\end{definition}

\subsection{Basic Set Operations}

\begin{definition}[Union]
Let $A$ and $B$ be sets.
The \emph{union} of $A$ and $B$, denoted by $A \cup B$, is the set of all elements
that belong to at least one of the sets.

Formally,
\[
A \cup B \;=\; \{\, x \mid x \in A \lor x \in B \,\}.
\]
\end{definition}


\begin{center}
\begin{tikzpicture}[scale=1]
  \draw (-1,0) circle (1.5);
  \draw ( 1,0) circle (1.5);

  \fill[pattern=north east lines] (-1,0) circle (1.5);
  \fill[pattern=north east lines] ( 1,0) circle (1.5);

  \node at (-2.2,-1.6) {$A$};
  \node at ( 2.2,-1.6) {$B$};
\end{tikzpicture}
\end{center}













\begin{definition}[Intersection]
Let $A$ and $B$ be sets.
The \emph{intersection} of $A$ and $B$, denoted by $A \cap B$, is the set of all
elements common to both sets.

Formally,
\[
A \cap B \;=\; \{\, x \mid x \in A \land x \in B \,\}.
\]
\end{definition}

\begin{center}
\begin{tikzpicture}[scale=1]
  \begin{scope}
    \clip (-1,0) circle (1.5);
    \fill[pattern=north east lines] (1,0) circle (1.5);
  \end{scope}

  \draw (-1,0) circle (1.5);
  \draw ( 1,0) circle (1.5);

  \node at (-2.2,-1.6) {$A$};
  \node at ( 2.2,-1.6) {$B$};
\end{tikzpicture}
\end{center}


\begin{definition}[Set Difference]
Let $A$ and $B$ be sets.
The \emph{difference} of $A$ and $B$, denoted by $A \setminus B$, is the set of
elements that belong to $A$ but not to $B$.

Formally,
\[
A \setminus B \;=\; \{\, x \mid x \in A \land x \notin B \,\}.
\]
\end{definition}

\begin{center}
\begin{tikzpicture}[scale=1]
  % Shade all of A
  \fill[pattern=north east lines] (-1,0) circle (1.5);

  % Remove (unshade) the overlap A ∩ B
  \begin{scope}
    \clip (1,0) circle (1.5);    % restrict to B
    \fill[white] (-1,0) circle (1.5); % whiten the part of A inside B
  \end{scope}

  % Draw outlines last for crisp printing
  \draw (-1,0) circle (1.5);
  \draw ( 1,0) circle (1.5);

  \node at (-2.2,-1.6) {$A$};
  \node at ( 2.2,-1.6) {$B$};
\end{tikzpicture}
\end{center}


\begin{definition}[Symmetric Difference]
Let $A$ and $B$ be sets.
The \emph{symmetric difference} of $A$ and $B$, denoted by $A \triangle B$, is the
set of elements that belong to exactly one of the sets.

Formally,
\[
A \triangle B
\;=\;
(A \setminus B) \cup (B \setminus A)
\;=\;
\{\, x \mid (x \in A \lor x \in B) \land x \notin A \cap B \,\}.
\]
\end{definition}

\begin{center}
\begin{tikzpicture}[scale=1]
  \fill[pattern=north east lines] (-1,0) circle (1.5);
  \fill[pattern=north east lines] ( 1,0) circle (1.5);

  \begin{scope}
    \clip (-1,0) circle (1.5);
    \fill[white] (1,0) circle (1.5);
  \end{scope}

  \draw (-1,0) circle (1.5);
  \draw ( 1,0) circle (1.5);

  \node at (-2.2,-1.6) {$A$};
  \node at ( 2.2,-1.6) {$B$};
\end{tikzpicture}
\end{center}

\begin{remark}
The operations $\cup$, $\cap$, and $\setminus$ correspond respectively to the
logical connectives ``or,'' ``and,'' and ``and not'' at the level of membership
statements.
\end{remark}

\begin{definition}[Complement]
Let $U$ be a set and let $A \subseteq U$.
The \emph{complement} of $A$ (relative to $U$), denoted $A^c$, is the set of
elements of $U$ not in $A$.

Formally,
\[
A^c = U \setminus A.
\]
\end{definition}

\begin{definition}[Cartesian Product]
Let $A$ and $B$ be sets.
The \emph{Cartesian product} of $A$ and $B$ is the set
\[
A \times B = \{\, (a,b) \mid a \in A \text{ and } b \in B \,\}.
\]
\end{definition}

\begin{remark}
Ordered pairs will be defined explicitly in the next section.
For now, $(a,b)$ is treated as a formal symbol representing an ordered pair.
\end{remark}

\begin{definition}[Set Duality]
Let $U$ be a fixed universe and let $\mathcal{P}(U)$ denote its power set.
Two set-theoretic expressions are said to be \emph{dual} if one is obtained from the
other by simultaneously replacing
\[
\cup \longleftrightarrow \cap
\quad\text{and}\quad
\varnothing \longleftrightarrow U,
\]
while leaving complements unchanged.

The resulting expression is called the \emph{dual} of the original expression.
\end{definition}

\begin{theorem}[De Morgan's Laws]
Let $U$ be a universe and let $A,B \subseteq U$.
Then
\[
(A \cup B)^c = A^c \cap B^c
\quad\text{and}\quad
(A \cap B)^c = A^c \cup B^c.
\]
\end{theorem}

\begin{proof}
We prove the first identity; the second follows by symmetry.

Let $x \in U$. Then
\[
\begin{aligned}
x \in (A \cup B)^c
&\iff x \notin A \cup B \\
&\iff (x \notin A \land x \notin B) \\
&\iff (x \in A^c \land x \in B^c) \\
&\iff x \in A^c \cap B^c.
\end{aligned}
\]
Since membership is equivalent for all $x \in U$, the sets are equal.
\end{proof}

\begin{corollary}[Principle of Set Duality]
Any identity involving the operations $\cup$, $\cap$, $\varnothing$, and $U$ that
holds for all subsets of a universe remains valid when each operation and constant
is replaced by its dual.
\end{corollary}

\begin{remark}
Set duality reflects the logical duality between conjunction and disjunction.
Under the correspondence
\[
\cap \leftrightarrow \land,
\qquad
\cup \leftrightarrow \lor,
\qquad
A^c \leftrightarrow \neg A,
\]
De Morgan's laws are the set-theoretic analogues of logical negation rules.
\end{remark}

\begin{example}[Using Set Duality]
Let $U$ be a universe and let $A,B,C \subseteq U$.
We prove the distributive law
\[
A \cap (B \cup C) = (A \cap B) \cup (A \cap C).
\]
\end{example}

\begin{proof}
Let $x \in U$. Then
\[
\begin{aligned}
x \in A \cap (B \cup C)
&\iff (x \in A \land (x \in B \lor x \in C)) \\
&\iff ((x \in A \land x \in B) \lor (x \in A \land x \in C)) \\
&\iff x \in (A \cap B) \cup (A \cap C).
\end{aligned}
\]
Since this holds for all $x \in U$, the sets are equal.
\end{proof}

\begin{corollary}[Dual Distributive Law]
For all $A,B,C \subseteq U$,
\[
A \cup (B \cap C) = (A \cup B) \cap (A \cup C).
\]
\end{corollary}

\begin{proof}
This identity is the dual of the previous distributive law.
It follows immediately from the principle of set duality by interchanging
$\cap$ and $\cup$.
\end{proof}


\begin{remark}[Using Duality in Proofs]
To prove a statement involving unions and intersections, it often suffices to
prove only one version. The dual statement follows automatically by replacing
$\cup$ with $\cap$ and vice versa.
\end{remark}

\begin{remark}[Transition to Relations]
The constructions introduced above allow us to form new sets from existing ones.
In particular, the Cartesian product provides a way to encode ordered information.

Relations and functions will be defined as special subsets of Cartesian products.
Thus, relations are not new foundational objects, but structured sets built using
the operations developed in this section.
\end{remark}

\subsection{Algebraic Laws of Set Operations}

\begin{remark}[Algebraic Structure of Set Operations]
The basic set operations $\cup$, $\cap$, and $\setminus$ satisfy algebraic laws
analogous to those of logical connectives and arithmetic operations.
These laws allow set expressions to be rearranged and simplified without changing
their meaning and will be used implicitly throughout later proofs.
\end{remark}

\begin{theorem}[Associativity of Union and Intersection]
Let $A,B,C$ be sets. Then
\[
(A \cup B) \cup C = A \cup (B \cup C)
\quad\text{and}\quad
(A \cap B) \cap C = A \cap (B \cap C).
\]
\end{theorem}

\begin{proof}
We prove associativity of union; the proof for intersection is analogous.

Let $x$ be arbitrary. Then
\[
\begin{aligned}
x \in (A \cup B) \cup C
&\iff (x \in A \cup B) \lor (x \in C) \\
&\iff (x \in A \lor x \in B) \lor x \in C \\
&\iff x \in A \lor (x \in B \lor x \in C) \\
&\iff x \in A \cup (B \cup C).
\end{aligned}
\]
Thus the sets are equal.
\end{proof}

\begin{theorem}[Commutativity of Union and Intersection]
Let $A,B$ be sets. Then
\[
A \cup B = B \cup A
\quad\text{and}\quad
A \cap B = B \cap A.
\]
\end{theorem}

\begin{proof}
We prove commutativity of union; intersection is analogous.

Let $x$ be arbitrary. Then
\[
x \in A \cup B \iff (x \in A \lor x \in B)
\iff (x \in B \lor x \in A)
\iff x \in B \cup A.
\]
\end{proof}

\begin{theorem}[Distributive Laws]
Let $A,B,C$ be sets. Then
\[
A \cap (B \cup C) = (A \cap B) \cup (A \cap C),
\]
\[
A \cup (B \cap C) = (A \cup B) \cap (A \cup C).
\]
\end{theorem}

\begin{theorem}[Identity and Absorption Laws]
Let $A,B$ be sets. Then
\[
A \cup \varnothing = A,
\qquad
A \cap \varnothing = \varnothing,
\]
\[
A \cup (A \cap B) = A,
\qquad
A \cap (A \cup B) = A.
\]
\end{theorem}


\begin{remark}[Non-Commutative and Non-Associative Operations]
The set difference operation $\setminus$ is neither commutative nor associative.
In general,
\[
A \setminus B \neq B \setminus A,
\qquad
(A \setminus B) \setminus C \neq A \setminus (B \setminus C).
\]
Thus, expressions involving set difference must be handled with care.
\end{remark}


\paragraph{Set Difference $\setminus$.}
The set difference operation is neither commutative nor associative:
\[
A \setminus B \neq B \setminus A,
\qquad
(A \setminus B) \setminus C \neq A \setminus (B \setminus C).
\]
It does not satisfy a simple distributive law, but it interacts with union and
intersection via the identities
\[
A \setminus (B \cup C) = (A \setminus B) \cap (A \setminus C),
\qquad
A \setminus (B \cap C) = (A \setminus B) \cup (A \setminus C),
\]
and
\[
(A \cup B) \setminus C = (A \setminus C) \cup (B \setminus C),
\qquad
(A \cap B) \setminus C = (A \setminus C) \cap (B \setminus C).
\]
These identities follow from the relation $A \setminus B = A \cap B^c$ together
with De Morgan’s laws.

\paragraph{Symmetric Difference $\triangle$.}
The symmetric difference is commutative and associative:
\[
A \triangle B = B \triangle A,
\qquad
(A \triangle B) \triangle C = A \triangle (B \triangle C).
\]
It satisfies
\[
A \triangle \varnothing = A,
\qquad
A \triangle A = \varnothing.
\]
Under $\triangle$, the power set $\mathcal{P}(U)$ forms an abelian group.

\paragraph{Complement $(\cdot)^c$.}
Complementation is a unary operation and therefore neither commutative nor
associative.
It satisfies the involution property
\[
(A^c)^c = A,
\]
and interacts with union and intersection through De Morgan’s laws:
\[
(A \cup B)^c = A^c \cap B^c,
\qquad
(A \cap B)^c = A^c \cup B^c.
\]

\paragraph{Cartesian Product $\times$.}
The Cartesian product is not commutative:
\[
A \times B \neq B \times A
\quad\text{in general}.
\]
It is associative up to canonical isomorphism:
\[
(A \times B) \times C \cong A \times (B \times C).
\]
It distributes over union in each coordinate:
\[
A \times (B \cup C) = (A \times B) \cup (A \times C),
\qquad
(A \cup B) \times C = (A \times C) \cup (B \times C).
\]

\begin{remark}
Among the basic set operations, only $\cup$ and $\cap$ exhibit full commutative,
associative, and distributive behavior.
The failure of these laws for $\setminus$ and $\times$ explains why expressions
involving these operations must be handled with explicit attention to order and
grouping.
\end{remark}









\begin{remark}
Together, the operations $\cup$, $\cap$, and complement endow $\mathcal{P}(U)$
with the structure of a Boolean algebra.
These algebraic laws justify the manipulation of set expressions in later sections,
particularly in proofs involving equivalence classes, partitions, and functions.
\end{remark}











\subsection{Relations on Sets}

\begin{definition}[Ordered Pair]
Let $a$ and $b$ be sets.
The \emph{ordered pair} $(a,b)$ is defined by
\[
(a,b) \;:=\; \{\{a\}, \{a,b\}\}.
\]
\end{definition}

\begin{theorem}[Uniqueness of Ordered Pairs]
For any sets $a,b,c,d$,
\[
(a,b) = (c,d)
\;\;\Longleftrightarrow\;\;
(a = c \land b = d).
\]
\end{theorem}

\begin{proof}
Suppose $(a,b) = (c,d)$.
Then $\{a\} \in (a,b)$, so $\{a\} \in (c,d)$.
Hence $\{a\} = \{c\}$ or $\{a\} = \{c,d\}$.

If $\{a\} = \{c\}$, then $a = c$.
If $\{a\} = \{c,d\}$, then $c = d = a$.
In either case, $a = c$.

Similarly, $\{a,b\} \in (a,b)$ implies $\{a,b\} \in (c,d)$, which forces $b = d$.
Thus $a = c$ and $b = d$.

The converse direction is immediate.
\end{proof}

\begin{definition}[Cartesian Product]
Let $A$ and $B$ be sets.
The \emph{Cartesian product} of $A$ and $B$ is the set
\[
A \times B
\;:=\;
\{\, (a,b) \mid a \in A \text{ and } b \in B \,\}.
\]
\end{definition}

\begin{definition}[Relation]
Let $A$ and $B$ be sets.
A \emph{relation} from $A$ to $B$ is any subset of the Cartesian product
$A \times B$.

That is, a relation $R$ from $A$ to $B$ satisfies
\[
R \subseteq A \times B.
\]
\end{definition}

\begin{remark}
If $(a,b) \in R$, we often write $a \, R \, b$ and say that
``$a$ is related to $b$ by $R$.''

When $A = B$, we say that $R$ is a \emph{relation on $A$}.
\end{remark}

\begin{remark}[Relations as Structured Sets]
Relations introduce no new foundational objects.
They are simply sets of ordered pairs, constructed using the axioms and
operations developed earlier.

This perspective allows properties of relations to be studied using ordinary
set-theoretic reasoning.
\end{remark}

\subsection{Properties of Relations on Sets}

\begin{definition}[Reflexive Relation]
The relation $R$ is said to be \emph{reflexive} if every element of $A$ is related
to itself.

Formally,
\[
\forall a \in A, \; (a,a) \in R.
\]
\end{definition}

\begin{definition}[Irreflexive Relation]
The relation $R$ is said to be \emph{irreflexive} if no element of $A$ is related
to itself.

Formally,
\[
\forall a \in A, \; (a,a) \notin R.
\]
\end{definition}

\begin{definition}[Symmetric Relation]
The relation $R$ is said to be \emph{symmetric} if whenever an element $a$ is
related to $b$, then $b$ is related to $a$.

Formally,
\[
\forall a,b \in A, \; (a,b) \in R \rightarrow (b,a) \in R.
\]
\end{definition}

\begin{definition}[Antisymmetric Relation]
The relation $R$ is said to be \emph{antisymmetric} if the only way two elements
can be mutually related is for them to be equal.

Formally,
\[
\forall a,b \in A, \;
\bigl( (a,b) \in R \land (b,a) \in R \bigr) \rightarrow a = b.
\]
\end{definition}

\begin{definition}[Asymmetric Relation]
The relation $R$ is said to be \emph{asymmetric} if whenever $a$ is related to $b$,
then $b$ is not related to $a$.

Formally,
\[
\forall a,b \in A, \;
(a,b) \in R \rightarrow (b,a) \notin R.
\]
\end{definition}

\begin{definition}[Transitive Relation]
The relation $R$ is said to be \emph{transitive} if whenever $a$ is related to $b$
and $b$ is related to $c$, then $a$ is related to $c$.

Formally,
\[
\forall a,b,c \in A, \;
\bigl( (a,b) \in R \land (b,c) \in R \bigr) \rightarrow (a,c) \in R.
\]
\end{definition}







\begin{definition}[Total (Connex) Relation]
The relation $R$ is said to be \emph{total} (or \emph{connex}) if every pair of
elements of $A$ is comparable under $R$.

Formally,
\[
\forall a,b \in A, \;
(a,b) \in R \lor (b,a) \in R.
\]
\end{definition}

\begin{center}
\renewcommand{\arraystretch}{1.3}
\begin{tabular}{|l|l|l|}
\hline
\textbf{Relational Property}
& \textbf{Formal Definition}
& \textbf{Canonical Example} \\
\hline
Reflexive
& $\forall a \in A,\; (a,a) \in R$
& Equality $(=)$ on any set \\
\hline
Irreflexive
& $\forall a \in A,\; (a,a) \notin R$
& $<$ on $\mathbb{N}$ \\
\hline
Symmetric
& $(a,b) \in R \Rightarrow (b,a) \in R$
& “Has the same age as” \\
\hline
Antisymmetric
& $(a,b),(b,a) \in R \Rightarrow a=b$
& $\leq$ on $\mathbb{N}$ \\
\hline
Asymmetric
& $(a,b) \in R \Rightarrow (b,a) \notin R$
& $<$ on $\mathbb{N}$ \\
\hline
Transitive
& $(a,b),(b,c) \in R \Rightarrow (a,c) \in R$
& $\leq$ on $\mathbb{N}$ \\
\hline
Total (Connex)
& $(a,b)\in R \lor (b,a)\in R$
& $\leq$ on $\mathbb{R}$ \\
\hline
\end{tabular}
\end{center}


\begin{definition}[Equivalence Relation]
The relation $R$ is an \emph{equivalence relation} if it is
reflexive, symmetric, and transitive.
\end{definition}

\begin{definition}[Preorder]
The relation $R$ is a \emph{preorder} if it is reflexive and transitive.
\end{definition}

\begin{definition}[Partial Order]
The relation $R$ is a \emph{partial order} if it is reflexive, antisymmetric,
and transitive.
\end{definition}

\begin{definition}[Total Order]
The relation $R$ is a \emph{total order} if it is a partial order and is total.
\end{definition}

\begin{remark}
The properties listed above are logically independent unless otherwise stated.
For example, a relation may be transitive without being reflexive, or symmetric
without being transitive.

Certain combinations of properties give rise to important mathematical structures,
such as equivalence relations and order relations, which will be studied in detail
in subsequent sections.
\end{remark}

\begin{center}
\renewcommand{\arraystretch}{1.3}
\begin{tabular}{|l|l|l|}
\hline
\textbf{Structural Class}
& \textbf{Defining Properties}
& \textbf{Canonical Example} \\
\hline
Equivalence Relation
& Reflexive, Symmetric, Transitive
& Equality $(=)$ on any set \\
\hline
Preorder
& Reflexive, Transitive
& $\leq$ on $\mathbb{N}$ (ignoring antisymmetry) \\
\hline
Partial Order
& Reflexive, Antisymmetric, Transitive
& $\subseteq$ on $\mathcal{P}(A)$ \\
\hline
Total Order
& Partial Order, Total (Connex)
& $\leq$ on $\mathbb{R}$ \\
\hline
Strict Partial Order
& Irreflexive, Transitive
& $<$ on $\mathbb{N}$ \\
\hline
Strict Total Order
& Strict Partial Order, Total (Connex)
& $<$ on $\mathbb{R}$ \\
\hline
\end{tabular}
\end{center}

\begin{remark}
Structural classes of relations are defined by logical conjunctions of basic
relational properties.
Each statement of the form ``$R$ is a \emph{[structural class]}'' is shorthand for
a conjunction of predicates imposed on $R$.

\end{remark}



\begin{remark}[Using Structural Properties in Proofs]
When a relation is said to have a structural property—such as being an equivalence
relation or a partial order—this assertion serves as shorthand for a conjunction
of basic relational properties.
In proofs, it is common to \emph{expand} such a statement inline by invoking only
the specific component property that is needed at that moment.

For example, rather than restating all defining properties of a structural class,
one may write ``since $R$ is a partial order, antisymmetry implies \dots'' or
``since $R$ is an equivalence relation, transitivity yields \dots''.
This practice keeps proofs concise while preserving logical precision.
\end{remark}

\begin{example}[Using a Structural Property in a Proof]
Let $A$ be a set and let $R$ be an equivalence relation on $A$.
Show that for all $a,b \in A$,
\[
(a,b) \in R \;\Longrightarrow\; [a] = [b],
\]
where $[a] = \{\, x \in A \mid (a,x) \in R \,\}$ denotes the equivalence class of $a$.
\end{example}

\begin{proof}
Assume $(a,b) \in R$.

Since $R$ is an equivalence relation, it is reflexive, symmetric, and transitive.
In particular, symmetry implies $(b,a) \in R$, and transitivity implies that
whenever $(a,x) \in R$, we have $(b,x) \in R$.

First, let $x \in [a]$. Then $(a,x) \in R$.
Since $(b,a) \in R$ and $R$ is transitive, it follows that $(b,x) \in R$.
Hence $x \in [b]$, so $[a] \subseteq [b]$.

Conversely, let $x \in [b]$. Then $(b,x) \in R$.
Since $(a,b) \in R$ and $R$ is transitive, it follows that $(a,x) \in R$.
Hence $x \in [a]$, so $[b] \subseteq [a]$.

Therefore $[a] = [b]$.
\end{proof}

\begin{remark}
Using a structural property in a proof always begins by expanding it into its
defining relational properties and applying only those required for the argument.
\end{remark}


\subsection{Indexed Families of Sets}

Many constructions in set theory involve collections of sets indexed by elements
of another set, rather than finite lists.

\begin{definition}[Indexed Family of Sets]
Let $I$ be a set, called the \emph{index set}.
An \emph{indexed family of sets} is a function
\[
F : I \to \mathcal{P}(U),
\]
where $U$ is some universe.

For each $i \in I$, the value $F(i)$ is a set, often denoted by
\[
F(i) = A_i.
\]

Thus an indexed family is commonly written as
\[
\{A_i\}_{i \in I}.
\]
\end{definition}

\begin{remark}
An indexed family is formally a function, not a set of sets.
Different indices may correspond to the same set:
\[
i \neq j \quad\text{does not imply}\quad A_i \neq A_j.
\]
This distinction becomes important in constructions involving equivalence classes
and quotient sets.
\end{remark}

\begin{definition}[Union of an Indexed Family]
Let $\{A_i\}_{i \in I}$ be an indexed family of sets.
The \emph{union} of the family is defined by
\[
\bigcup_{i \in I} A_i
\;:=\;
\{\, x \mid \exists i \in I \text{ such that } x \in A_i \,\}.
\]
\end{definition}

\begin{definition}[Intersection of an Indexed Family]
Let $\{A_i\}_{i \in I}$ be an indexed family of sets, with $I \neq \varnothing$.
The \emph{intersection} of the family is defined by
\[
\bigcap_{i \in I} A_i
\;:=\;
\{\, x \mid \forall i \in I,\; x \in A_i \,\}.
\]
\end{definition}

\begin{remark}
The requirement $I \neq \varnothing$ ensures that intersections are not defined
vacuously without further convention.
In particular,
\[
\bigcap_{i \in \varnothing} A_i
\]
is usually left undefined unless a universal set is specified.
\end{remark}

\begin{definition}[Disjoint Indexed Family]
An indexed family $\{A_i\}_{i \in I}$ is said to be \emph{pairwise disjoint} if
\[
\forall i,j \in I,\;
(i \neq j \rightarrow A_i \cap A_j = \varnothing).
\]
\end{definition}

\begin{remark}[Purpose of Indexed Families]
Indexed families provide the formal language for describing constructions such as:
\begin{itemize}
\item partitions of a set,
\item collections of equivalence classes,
\item unions and intersections over infinitely many sets.
\end{itemize}

In the next section, equivalence classes will naturally form an indexed family
whose union recovers the original set.
\end{remark}

\subsection{Arbitrary Cartesian Products}

Finite Cartesian products generalize naturally to products indexed by an arbitrary
set.

\begin{definition}[Arbitrary Cartesian Product]
Let $I$ be a set and let $\{A_i\}_{i \in I}$ be an indexed family of sets.
The \emph{Cartesian product} of the family is defined by
\[
\prod_{i \in I} A_i
\;:=\;
\Bigl\{
f : I \to \bigcup_{i \in I} A_i
\;\Bigm|\;
\forall i \in I,\; f(i) \in A_i
\Bigr\}.
\]
\end{definition}

\begin{remark}
An element of $\prod_{i \in I} A_i$ is a \emph{choice function} that assigns to each
index $i \in I$ an element of the corresponding set $A_i$.

Thus, an element of the product is not a tuple in the finite sense, but a function
whose domain is the index set $I$.
\end{remark}

\begin{remark}[Connection with Finite Products]
If $I = \{1,2,\dots,n\}$, then an element of $\prod_{i=1}^n A_i$ may be identified
with an $n$-tuple
\[
(a_1,a_2,\dots,a_n),
\quad\text{where } a_i \in A_i.
\]
For infinite index sets, this tuple notation is no longer adequate, and the
functional definition is essential.
\end{remark}

\begin{remark}[Nonemptiness and the Axiom of Choice]
The Cartesian product $\prod_{i \in I} A_i$ is nonempty if and only if there exists
a function $f$ such that $f(i) \in A_i$ for all $i \in I$.

When $I$ is infinite, the existence of such a function is not guaranteed by the
other axioms of set theory and is equivalent to the Axiom of Choice.
\end{remark}

























\subsection{Equivalence Classes and Partitions}



\begin{definition}[Set Cover]
Let $A$ be a set and let $\mathcal{C}$ be a collection of subsets of $A$
(that is, $\mathcal{C} \subseteq \mathcal{P}(A)$).

We say that $\mathcal{C}$ is a \emph{cover} of $A$ if every element of $A$
belongs to at least one set in $\mathcal{C}$.

Formally,
\[
\mathcal{C} \text{ is a cover of } A
\;\;\Longleftrightarrow\;\;
\forall x \in A,\; \exists C \in \mathcal{C} \text{ such that } x \in C.
\]

Equivalently,
\[
\bigcup_{C \in \mathcal{C}} C = A.
\]
\end{definition}





\begin{definition}[Equivalence Class]
Let $A$ be a set and let $R$ be an equivalence relation on $A$.
For $a \in A$, the \emph{equivalence class} of $a$ (with respect to $R$) is the set
\[
[a]_R \;:=\; \{\, x \in A \mid (a,x) \in R \,\}.
\]
When the relation $R$ is clear from context, we write $[a]$ instead of $[a]_R$.
\end{definition}

\begin{definition}[Index of an Equivalence Relation]
Let $A$ be a set and let $R$ be an equivalence relation on $A$.
The \emph{index} of $R$ (or of $A$ modulo $R$) is the cardinality of the quotient
set $A/R$ and is denoted by
\[
|A/R|.
\]
Equivalently, the index of $R$ is the number of equivalence classes of $R$.
\end{definition}

\begin{remark}
The index $|A/R|$ may be finite or infinite.
In particular, even when $A$ is infinite, the quotient $A/R$ may have finite index,
countably infinite index, or uncountable index, depending on the structure of $R$.
\end{remark}

\begin{definition}[Quotient Set]
Let $A$ be a set and let $R$ be an equivalence relation on $A$.
The \emph{quotient set} of $A$ by $R$, denoted $A / R$, is the set of all
equivalence classes of $R$:
\[
A / R \;:=\; \{\, [a]_R \mid a \in A \,\}.
\]
\end{definition}

\begin{definition}[Canonical Surjection]
Let $A$ be a set and let $R$ be an equivalence relation on $A$.
The function
\[
\pi : A \to A/R,
\qquad
\pi(a) := [a],
\]
is called the \emph{canonical surjection} (or \emph{quotient map}) associated with $R$.
\end{definition}

\begin{remark}
Quotient sets and canonical surjections reappear throughout mathematics,
notably in quotient groups, quotient rings, and quotient topological spaces.
\end{remark}








\begin{remark}
The map $\pi$ is surjective by construction, since every equivalence class
$[a] \in A/R$ is the image of any of its elements.
Elements $a,b \in A$ satisfy $\pi(a) = \pi(b)$ if and only if $(a,b) \in R$.
\end{remark}

\begin{remark}[Universality of the Canonical Surjection]
The canonical surjection $\pi : A \to A/R$ is the prototype for all quotient
constructions.
In later contexts, it appears as the quotient homomorphism in algebra and the
quotient map in topology.
\end{remark}


\begin{definition}[Partition]
Let $A$ be a set.
A \emph{partition} of $A$ is a collection $\mathcal{P}$ of subsets of $A$ such that:
\begin{enumerate}[label=(\roman*)]
\item every block is nonempty: $\forall P \in \mathcal{P},\; P \neq \varnothing$;
\item distinct blocks are disjoint:
\[
\forall P,Q \in \mathcal{P},\; (P \neq Q \rightarrow P \cap Q = \varnothing);
\]
\item the blocks cover $A$:
\[
\bigcup_{P \in \mathcal{P}} P \;=\; A.
\]
\end{enumerate}
The sets $P \in \mathcal{P}$ are called the \emph{blocks} (or \emph{parts}) of the partition.
\end{definition}



\begin{remark}[Equivalence Relations and Partitions]
Equivalence relations and partitions encode the same structural information in
two different forms.
An equivalence relation on a set $A$ \emph{induces} a partition of $A$ by grouping
elements into their equivalence classes.
Conversely, a partition of $A$ \emph{induces} an equivalence relation by declaring
two elements equivalent precisely when they belong to the same block of the
partition.

Thus, passing from an equivalence relation to its induced partition of
equivalence classes, and passing from a partition to its induced equivalence
relation, are inverse constructions.
The correspondence between equivalence relations and partitions will be made
precise and proved in the following theorem.
\end{remark}




\begin{theorem}[Equivalence Relations and Partitions]
Let $A$ be a set.
\begin{enumerate}[label=(\roman*)]
\item If $R$ is an equivalence relation on $A$, then the family of equivalence classes
\[
A/R \;:=\; \{\, [a]_R \mid a \in A \,\}
\]
is a partition of $A$.

\item If $\mathcal{P}$ is a partition of $A$, define a relation $R_{\mathcal{P}}$ on $A$ by
\[
(a,b)\in R_{\mathcal{P}}
\;\;\Longleftrightarrow\;\;
\exists P\in \mathcal{P}\ \text{such that}\ a\in P\ \text{and}\ b\in P.
\]
Then $R_{\mathcal{P}}$ is an equivalence relation on $A$.

\item These constructions are inverse to each other: starting from $R$, forming the
induced partition $A/R$, and then forming the induced equivalence relation
$R_{A/R}$ yields $R$; starting from $\mathcal{P}$, forming $R_{\mathcal{P}}$, and then
forming the induced partition $A/R_{\mathcal{P}}$ yields $\mathcal{P}$.
\end{enumerate}
\end{theorem}

\begin{proof}
\textbf{(i)} Assume $R$ is an equivalence relation on $A$.

We show that $A/R$ is a partition of $A$.

\emph{Nonempty blocks.}
Let $a\in A$. Since $R$ is reflexive, $(a,a)\in R$, hence $a\in [a]_R$.
Therefore $[a]_R\neq \varnothing$.

\emph{Covering.}
Let $x\in A$. By reflexivity, $(x,x)\in R$, so $x\in [x]_R$.
Thus $x \in \bigcup_{a\in A} [a]_R$, and hence
\[
\bigcup_{C\in A/R} C \;=\; A.
\]

\emph{Pairwise disjointness.}
Let $[a]_R,[b]_R \in A/R$ and assume $[a]_R \cap [b]_R \neq \varnothing$.
Choose $x \in [a]_R \cap [b]_R$. Then $(a,x)\in R$ and $(b,x)\in R$.
Since $R$ is symmetric, $(x,b)\in R$. By transitivity, from $(a,x)\in R$ and
$(x,b)\in R$ we obtain $(a,b)\in R$.

We claim $[a]_R = [b]_R$. Let $y\in [a]_R$. Then $(a,y)\in R$.
From $(b,a)\in R$ (by symmetry of $(a,b)\in R$) and $(a,y)\in R$, transitivity
gives $(b,y)\in R$, hence $y\in [b]_R$. So $[a]_R \subseteq [b]_R$.
By symmetry of the argument, $[b]_R \subseteq [a]_R$, and thus $[a]_R=[b]_R$.

Therefore, if $[a]_R\neq [b]_R$, we must have $[a]_R\cap [b]_R=\varnothing$.
So $A/R$ is a partition of $A$.

\medskip
\textbf{(ii)} Let $\mathcal{P}$ be a partition of $A$, and define $R_{\mathcal{P}}$ as above.

We prove that $R_{\mathcal{P}}$ is reflexive, symmetric, and transitive.

\emph{Reflexive.}
Let $a\in A$. Since $\mathcal{P}$ covers $A$, there exists $P\in \mathcal{P}$ with $a\in P$.
Then $a$ and $a$ belong to the same block $P$, so $(a,a)\in R_{\mathcal{P}}$.

\emph{Symmetric.}
Assume $(a,b)\in R_{\mathcal{P}}$. Then there exists $P\in \mathcal{P}$ with $a\in P$ and $b\in P$.
The same $P$ witnesses that $(b,a)\in R_{\mathcal{P}}$.

\emph{Transitive.}
Assume $(a,b)\in R_{\mathcal{P}}$ and $(b,c)\in R_{\mathcal{P}}$.
Then there exist $P,Q\in \mathcal{P}$ such that $a,b\in P$ and $b,c\in Q$.
Hence $b\in P\cap Q$. Since $\mathcal{P}$ is a partition, distinct blocks are disjoint,
so $P\cap Q\neq\varnothing$ implies $P=Q$. Therefore $a,c\in P$, and thus $(a,c)\in R_{\mathcal{P}}$.

So $R_{\mathcal{P}}$ is an equivalence relation.

\medskip
\textbf{(iii)} We show the constructions are inverse.

\emph{From $R$ to $A/R$ to $R_{A/R}$.}
Let $R$ be an equivalence relation on $A$ and consider the induced partition $A/R$.
For $a,b\in A$,
\[
(a,b)\in R_{A/R}
\iff \exists C\in A/R\ \text{with}\ a\in C\ \text{and}\ b\in C.
\]
But $C\in A/R$ means $C=[x]_R$ for some $x\in A$.
Then $a\in [x]_R$ and $b\in [x]_R$ imply $(x,a)\in R$ and $(x,b)\in R$.
By symmetry, $(a,x)\in R$, and by transitivity, from $(a,x)\in R$ and $(x,b)\in R$
we get $(a,b)\in R$. Hence $R_{A/R}\subseteq R$.

Conversely, if $(a,b)\in R$, then $b\in [a]_R$, so $a\in [a]_R$ and $b\in [a]_R$,
which shows $(a,b)\in R_{A/R}$. Thus $R\subseteq R_{A/R}$, and therefore
$R_{A/R}=R$.

\emph{From $\mathcal{P}$ to $R_{\mathcal{P}}$ to $A/R_{\mathcal{P}}$.}
Let $\mathcal{P}$ be a partition of $A$ and form $R_{\mathcal{P}}$.
Fix $a\in A$ and let $P_a$ be the unique block of $\mathcal{P}$ containing $a$
(existence by covering; uniqueness by disjointness).
We claim
\[
[a]_{R_{\mathcal{P}}} = P_a.
\]
If $x\in [a]_{R_{\mathcal{P}}}$, then $(a,x)\in R_{\mathcal{P}}$, so $a$ and $x$
lie in a common block $P\in \mathcal{P}$. Since $a\in P_a$ and also $a\in P$,
we have $P\cap P_a\neq\varnothing$, hence $P=P_a$, so $x\in P_a$.
Thus $[a]_{R_{\mathcal{P}}}\subseteq P_a$.

Conversely, if $x\in P_a$, then $a\in P_a$ and $x\in P_a$, so by definition
$(a,x)\in R_{\mathcal{P}}$, hence $x\in [a]_{R_{\mathcal{P}}}$.
Thus $P_a\subseteq [a]_{R_{\mathcal{P}}}$, proving equality.

Therefore the collection of equivalence classes $A/R_{\mathcal{P}}$ is exactly the
collection of blocks of $\mathcal{P}$, i.e.\ $A/R_{\mathcal{P}}=\mathcal{P}$.

This completes the proof.
\end{proof}


\begin{remark}[Covers, Partitions, and Equivalence Classes]
Let $A$ be a set.

Every partition of $A$ is a cover of $A$ whose members are nonempty and pairwise
disjoint.
Equivalently, partitions are precisely those covers that satisfy a disjointness
condition.

An equivalence relation on $A$ induces a partition of $A$ via its equivalence
classes, and this partition is therefore a special kind of cover.
Conversely, a partition of $A$ induces an equivalence relation by declaring two
elements equivalent exactly when they lie in the same block of the partition.

Thus, equivalence relations correspond bijectively to partitions of $A$, and
partitions form a distinguished subclass of covers characterized by disjointness.
\end{remark}





\begin{example}[Extremal Equivalence Relations]
Let $A$ be a set.

\begin{enumerate}
\item \textbf{Equality relation.}
Let $R$ be equality on $A$, that is,
\[
(a,b) \in R \;\Longleftrightarrow\; a=b.
\]
Then every equivalence class is a singleton:
\[
[a] = \{a\}.
\]
Thus the quotient set $A/R$ may be identified with $A$ itself, and the induced
partition is the \emph{finest partition} of $A$, consisting of all singletons.

\item \textbf{Universal relation.}
Let $R = A \times A$, so that every pair of elements of $A$ is related.
Then there is exactly one equivalence class:
\[
[a] = A \quad \text{for all } a \in A.
\]
Thus the quotient set $A/R$ has a single element, and the induced partition is
the \emph{coarsest partition} of $A$, consisting of one block.
\end{enumerate}
\end{example}

\begin{remark}
These two examples represent the extremes of equivalence relations on a set:
equality yields the finest possible partition, while $A \times A$ yields the
coarsest.
All other equivalence relations lie between these extremes.
\end{remark}


\begin{lemma}[Representative Independence Lemma]
Let $A$ be a set and let $R$ be an equivalence relation on $A$.
For any $a,b \in A$,
\[
[a] = [b]
\;\;\Longleftrightarrow\;\;
(a,b) \in R.
\]
\end{lemma}

\begin{proof}
($\Rightarrow$)
Assume $[a] = [b]$.
Since $a \in [a]$, we have $a \in [b]$.
By definition of equivalence class, this means $(b,a) \in R$.
Since $R$ is symmetric, $(a,b) \in R$.

($\Leftarrow$)
Assume $(a,b) \in R$.
We show $[a] = [b]$ by mutual inclusion.

Let $x \in [a]$. Then $(a,x) \in R$.
Since $(a,b) \in R$, symmetry gives $(b,a) \in R$.
By transitivity, $(b,x) \in R$, hence $x \in [b]$.
Thus $[a] \subseteq [b]$.

The reverse inclusion follows by symmetry of the argument.
Therefore $[a] = [b]$.
\end{proof}

\begin{remark}
This lemma shows that equivalence classes do not depend on the choice of
representative.
It is the key fact underlying the well-definedness of functions defined on
quotient sets.
\end{remark}


















\subsection{Functions}

\begin{definition}[Function]
Let $A$ and $B$ be sets.
A \emph{function} from $A$ to $B$ is a relation $f \subseteq A \times B$ such that:
\begin{enumerate}[label=(\roman*)]
\item (\emph{Existence}) for every $a \in A$, there exists $b \in B$ with $(a,b) \in f$;
\item (\emph{Uniqueness}) for all $a \in A$ and $b_1,b_2 \in B$,
\[
(a,b_1)\in f \;\land\; (a,b_2)\in f \;\Longrightarrow\; b_1=b_2.
\]
\end{enumerate}
If $(a,b)\in f$, we write $f(a)=b$.
\end{definition}

\begin{remark}
A function is a special kind of relation: it is a relation that is
\emph{left-total} (every element of the domain appears) and
\emph{right-unique} (each element of the domain is related to exactly one element
of the codomain).

Thus, every function is a relation, but not every relation is a function.
\end{remark}

\begin{definition}[Domain and Codomain]
If $f$ is a function from $A$ to $B$, we write
\[
f : A \to B,
\]
where $A$ is called the \emph{domain} of $f$, denoted $\dom(f)$, and $B$ the
\emph{codomain}, denoted $\cod(f)$.
\end{definition}

\begin{definition}[Image of a Function]
Let $f : A \to B$ be a function.
The \emph{image} (or \emph{range}) of $f$ is the set
\[
\im(f) \;:=\; \{\, b \in B \mid \exists a \in A,\ f(a)=b \,\}.
\]
The image may be a proper subset of the codomain.
\end{definition}

\begin{definition}[Image of a Set]
Let $f : A \to B$ be a function, and let $S \subseteq A$.
The \emph{image} of $S$ under $f$ is the set
\[
f(S) \;:=\; \{\, f(a) \mid a \in S \,\} \;=\; \{\, b \in B \mid \exists a \in S,\ f(a)=b \,\}.
\]
\end{definition}

\begin{remark}
The image of the entire domain equals the image of the function:
$f(A) = \im(f)$.
\end{remark}

\begin{definition}[Preimage]
Let $f : A \to B$ be a function, and let $T \subseteq B$.
The \emph{preimage} (or \emph{inverse image}) of $T$ under $f$ is the set
\[
f^{-1}(T) \;:=\; \{\, a \in A \mid f(a) \in T \,\}.
\]
\end{definition}

\begin{remark}
The notation $f^{-1}(T)$ does \emph{not} denote an inverse function.
It is defined for every function $f$, regardless of whether $f$ is injective or
bijective.
\end{remark}

\begin{definition}[Fiber]
Let $f : A \to B$ be a function, and let $b \in B$.
The \emph{fiber} of $f$ over $b$ is the preimage of the singleton $\{b\}$:
\[
f^{-1}(\{b\}) \;=\; \{\, a \in A \mid f(a) = b \,\}.
\]
\end{definition}

\begin{remark}
The fibers of a function $f : A \to B$ partition the domain $A$.
More precisely, the collection
\[
\{\, f^{-1}(\{b\}) \mid b \in \im(f) \,\}
\]
is a partition of $A$, and two elements $a_1, a_2 \in A$ lie in the same fiber
if and only if $f(a_1) = f(a_2)$.

Thus, every function induces an equivalence relation on its domain:
\[
a_1 \sim_f a_2 \;\Longleftrightarrow\; f(a_1) = f(a_2),
\]
and the equivalence classes are precisely the fibers of $f$.
\end{remark}

\begin{theorem}[Preimages Preserve Set Operations]
Let $f : A \to B$ be a function.
For any subsets $S, T \subseteq B$,
\begin{enumerate}[label=(\roman*)]
\item $f^{-1}(S \cup T) = f^{-1}(S) \cup f^{-1}(T)$,
\item $f^{-1}(S \cap T) = f^{-1}(S) \cap f^{-1}(T)$,
\item $f^{-1}(S \setminus T) = f^{-1}(S) \setminus f^{-1}(T)$,
\item $f^{-1}(S^c) = (f^{-1}(S))^c$.
\end{enumerate}
\end{theorem}

\begin{remark}
Preimages preserve all set-theoretic operations.
This property makes preimages central in topology, measure theory, and analysis.
\end{remark}

\begin{theorem}[Images and Set Operations]
Let $f : A \to B$ be a function.
For any subsets $S, T \subseteq A$,
\begin{enumerate}[label=(\roman*)]
\item $f(S \cup T) = f(S) \cup f(T)$,
\item $f(S \cap T) \subseteq f(S) \cap f(T)$,
\item $f(S \setminus T) \supseteq f(S) \setminus f(T)$.
\end{enumerate}
Equality holds in (ii) and (iii) for all $S, T$ if and only if $f$ is injective.
\end{theorem}

\begin{remark}
Forward images preserve unions but generally do not preserve intersections or
complements.
This asymmetry between images and preimages is fundamental.
\end{remark}

\begin{definition}[Graph of a Function]
Let $f : A \to B$ be a function.
The \emph{graph} of $f$ is the subset of the Cartesian product $A \times B$ defined by
\[
\operatorname{Graph}(f)
\;:=\;
\{\, (a,b) \in A \times B \mid b = f(a) \,\}.
\]
\end{definition}

\begin{remark}
The graph of a function is a relation that uniquely determines the function.
Conversely, a relation $G \subseteq A \times B$ is the graph of a function
$f : A \to B$ if and only if $G$ is left-total and right-unique.

Thus, functions may be identified with their graphs, and a function is nothing
more than a special kind of relation.
\end{remark}

\begin{definition}[Injective Function]
Let $f : A \to B$ be a function.
We say that $f$ is \emph{injective} (or \emph{one-to-one}) if distinct elements of
the domain have distinct images.

Formally,
\[
\forall a_1,a_2 \in A,\;
f(a_1)=f(a_2) \;\Longrightarrow\; a_1=a_2.
\]
\end{definition}

\begin{definition}[Surjective Function]
Let $f : A \to B$ be a function.
We say that $f$ is \emph{surjective} (or \emph{onto}) if every element of the
codomain is the image of some element of the domain.

Formally,
\[
\forall b \in B,\; \exists a \in A \text{ such that } f(a)=b.
\]
Equivalently, $\im(f)=B$.
\end{definition}

\begin{definition}[Bijective Function]
Let $f : A \to B$ be a function.
We say that $f$ is \emph{bijective} if it is both injective and surjective.
\end{definition}

\begin{remark}
Injectivity concerns uniqueness of preimages: each fiber has at most one element.
Surjectivity concerns completeness of images: every fiber is nonempty.
Bijectivity combines both: every fiber has exactly one element.
\end{remark}

\begin{definition}[Identity Function]
Let $A$ be a set.
The \emph{identity function} on $A$ is the function
\[
\id_A : A \to A,
\qquad
\id_A(a)=a \text{ for all } a \in A.
\]
\end{definition}

\begin{definition}[Inclusion Map]
Let $A \subseteq B$ be sets.
The \emph{inclusion map} (or \emph{canonical injection}) from $A$ to $B$ is the function
\[
\iota : A \hookrightarrow B,
\qquad
\iota(a) = a \text{ for all } a \in A.
\]
\end{definition}

\begin{remark}
The inclusion map is injective.
It differs from the identity function in that the codomain may be strictly larger
than the domain.
\end{remark}

\begin{definition}[Composition of Functions]
Let
\[
f : A \to B
\quad\text{and}\quad
g : B \to C
\]
be functions.
The \emph{composition} of $g$ with $f$ is the function
\[
g \circ f : A \to C
\]
defined by
\[
(g \circ f)(a) := g\bigl(f(a)\bigr)
\quad\text{for all } a \in A.
\]
\end{definition}

\begin{theorem}[Associativity of Composition]
Let $f : A \to B$, $g : B \to C$, and $h : C \to D$ be functions.
Then
\[
h \circ (g \circ f) = (h \circ g) \circ f.
\]
\end{theorem}

\begin{remark}
Composition is generally not commutative: $g \circ f \neq f \circ g$ in general,
even when both compositions are defined.
\end{remark}

\begin{theorem}[Identity and Composition]
For any function $f : A \to B$,
\[
f \circ \id_A = f
\quad\text{and}\quad
\id_B \circ f = f.
\]
Thus, identity functions act as neutral elements for composition.
\end{theorem}

\begin{theorem}[Injectivity and Surjectivity Under Composition]
Let $f : A \to B$ and $g : B \to C$ be functions.
\begin{enumerate}[label=(\roman*)]
\item If $f$ and $g$ are injective, then $g \circ f$ is injective.
\item If $f$ and $g$ are surjective, then $g \circ f$ is surjective.
\item If $g \circ f$ is injective, then $f$ is injective.
\item If $g \circ f$ is surjective, then $g$ is surjective.
\end{enumerate}
\end{theorem}

\begin{remark}
In (iii), the function $g$ need not be injective.
In (iv), the function $f$ need not be surjective.
\end{remark}

\begin{definition}[Inverse Function]
Let $f : A \to B$ be a bijective function.
The \emph{inverse function} of $f$ is the function
\[
f^{-1} : B \to A
\]
defined by the rule
\[
f^{-1}(b) = a \quad\text{if and only if}\quad f(a)=b.
\]
\end{definition}

\begin{theorem}[Characterization of Inverse Functions]
Let $f : A \to B$ be a bijective function.
Then
\[
f^{-1} \circ f = \id_A
\quad\text{and}\quad
f \circ f^{-1} = \id_B.
\]
Conversely, a function admits an inverse if and only if it is bijective.
\end{theorem}

\begin{theorem}[Inverse of a Composition]
Let $f : A \to B$ and $g : B \to C$ be bijective functions.
Then $g \circ f$ is bijective and
\[
(g \circ f)^{-1} = f^{-1} \circ g^{-1}.
\]
\end{theorem}

\begin{definition}[Left Inverse]
Let $f : A \to B$ be a function.
A function $g : B \to A$ is called a \emph{left inverse} of $f$ if
\[
g \circ f = \id_A,
\]
that is, $g(f(a)) = a$ for all $a \in A$.
\end{definition}

\begin{definition}[Right Inverse]
Let $f : A \to B$ be a function.
A function $h : B \to A$ is called a \emph{right inverse} (or \emph{section}) of $f$ if
\[
f \circ h = \id_B,
\]
that is, $f(h(b)) = b$ for all $b \in B$.
\end{definition}

\begin{theorem}[One-Sided Inverses and Function Properties]
Let $f : A \to B$ be a function.
\begin{enumerate}[label=(\roman*)]
\item $f$ has a left inverse if and only if $f$ is injective (and $A$ is nonempty
or $A = B = \varnothing$).
\item $f$ has a right inverse if and only if $f$ is surjective.
\item If $f$ has both a left inverse $g$ and a right inverse $h$, then $g = h$,
and $f$ is bijective.
\end{enumerate}
\end{theorem}

\begin{remark}
The existence of a right inverse for every surjective function is equivalent to
the Axiom of Choice.
\end{remark}

\begin{definition}[Constant Function]
Let $f : A \to B$ be a function.
We say that $f$ is \emph{constant} if there exists $b_0 \in B$ such that
\[
\forall a \in A,\; f(a)=b_0.
\]
\end{definition}

\begin{definition}[Restriction of a Function]
Let $f : A \to B$ be a function and let $C \subseteq A$.
The \emph{restriction} of $f$ to $C$ is the function
\[
f|_C : C \to B
\]
defined by
\[
f|_C(a)=f(a) \quad \text{for all } a \in C.
\]
\end{definition}

\begin{definition}[Extension of a Function]
Let $A \subseteq A'$, and let $f : A \to B$ and $g : A' \to B$ be functions.
We say that $g$ is an \emph{extension} of $f$ (and that $f$ is a \emph{restriction}
of $g$) if
\[
g|_A = f,
\]
that is, $g(a) = f(a)$ for all $a \in A$.
\end{definition}

\begin{remark}
An extension of a function agrees with the original function on its entire domain
but may be defined on a larger set.
Extensions are generally not unique unless additional constraints are imposed.
\end{remark}

\begin{remark}
Questions about extending functions under additional conditions
(such as continuity, linearity, or boundedness)
lead to important results such as the Tietze Extension Theorem
and the Hahn–Banach Theorem.
\end{remark}


\subsection{Ordered Sets}

\begin{definition}[Ordered Set]
An \emph{ordered set} is a pair $(A,\leq)$ consisting of a set $A$ together with
a binary relation $\leq$ on $A$ that is a partial order.

That is, $\leq$ satisfies:
\begin{enumerate}[label=(\roman*)]
\item \emph{Reflexivity:} $\forall a \in A,\; a \leq a$;
\item \emph{Antisymmetry:} $\forall a,b \in A,\;
(a \leq b \land b \leq a) \Rightarrow a=b$;
\item \emph{Transitivity:} $\forall a,b,c \in A,\;
(a \leq b \land b \leq c) \Rightarrow a \leq c$.
\end{enumerate}
\end{definition}

\begin{remark}
An ordered set is therefore a set equipped with a specified partial order.
Different orders on the same underlying set generally produce different ordered
sets.
\end{remark}

\begin{definition}[Strict Order]
Let $(A,\leq)$ be an ordered set.
The associated \emph{strict order} $<$ on $A$ is defined by
\[
a < b
\;\;\Longleftrightarrow\;\;
(a \leq b \land a \neq b).
\]
\end{definition}

\begin{remark}
The relation $<$ is irreflexive and transitive.
Conversely, given a strict partial order $<$ on $A$, one may define a non-strict
order $\leq$ by
\[
a \leq b \;\Longleftrightarrow\; (a < b \lor a=b).
\]
Thus strict and non-strict orders carry the same information.
\end{remark}

\begin{definition}[Comparable Elements]
Let $(A,\leq)$ be an ordered set.
Two elements $a,b \in A$ are said to be \emph{comparable} if either
\[
a \leq b \quad\text{or}\quad b \leq a.
\]
\end{definition}

\begin{definition}[Noncomparable Elements]
Let $(A,\leq)$ be an ordered set.
Two elements $a,b \in A$ are said to be \emph{noncomparable} (or \emph{incomparable})
if neither precedes the other under the order.

Formally,
\[
a \nleq b \;\;\text{and}\;\; b \nleq a.
\]
Equivalently,
\[
\neg(a \leq b) \land \neg(b \leq a).
\]
\end{definition}

\begin{remark}
Noncomparable elements can occur only in partially ordered sets.
In a totally ordered set, every pair of elements is comparable, so
noncomparable elements do not exist.
\end{remark}


\begin{definition}[Total (Linear) Order]
An ordered set $(A,\leq)$ is called a \emph{total order} (or \emph{linear order})
if every pair of elements is comparable:
\[
\forall a,b \in A,\; a \leq b \lor b \leq a.
\]
\end{definition}

\begin{remark}
Total orders generalize the familiar ordering of the real numbers.
Partial orders allow incomparable elements and arise naturally in algebra,
topology, and logic.
\end{remark}

\begin{definition}[Upper and Lower Bounds]
Let $(A,\leq)$ be an ordered set and let $S \subseteq A$.

\begin{itemize}
\item An element $u \in A$ is an \emph{upper bound} of $S$ if
\[
\forall s \in S,\; s \leq u.
\]
\item An element $\ell \in A$ is a \emph{lower bound} of $S$ if
\[
\forall s \in S,\; \ell \leq s.
\]
\end{itemize}
\end{definition}

\begin{remark}
The notions of upper and lower bounds depend on the chosen order.
They will play a central role in the study of completeness, supremum and infimum,
and ordered topological and metric structures.
\end{remark}

\begin{definition}[Minimal and Maximal Elements]
Let $(A,\leq)$ be an ordered set and let $S \subseteq A$.

\begin{itemize}
\item An element $m \in S$ is a \emph{minimal element} of $S$ if there is no
$s \in S$ such that $s < m$.
\item An element $M \in S$ is a \emph{maximal element} of $S$ if there is no
$s \in S$ such that $M < s$.
\end{itemize}
\end{definition}

\begin{remark}
Minimal and maximal elements need not be unique, and need not exist.
They should not be confused with least or greatest elements, which are stronger
notions.
\end{remark}

\begin{definition}[Least and Greatest Elements]
Let $(A,\leq)$ be an ordered set and let $S \subseteq A$.

\begin{itemize}
\item An element $\ell \in S$ is the \emph{least element} of $S$ if
\[
\forall s \in S,\; \ell \leq s.
\]
\item An element $g \in S$ is the \emph{greatest element} of $S$ if
\[
\forall s \in S,\; s \leq g.
\]
\end{itemize}
\end{definition}

\begin{remark}
If a least or greatest element exists, it is unique.
Every least element is minimal, and every greatest element is maximal, but the
converses need not hold.
\end{remark}

\begin{remark}[Transition]
Ordered sets provide the abstract framework for order structures on the real
numbers, function spaces, and metric spaces.
In later sections, order will interact with topology and analysis through
intervals, monotone functions, and completeness properties.
\end{remark}

\begin{definition}[Order-Preserving Mapping]
Let $(M,\leq)$ and $(M',\leq')$ be partially ordered sets, and let
\[
f : M \to M'
\]
be a function.
We say that $f$ is \emph{order-preserving} (or \emph{monotone}) if
\[
\forall a,b \in M,\quad a \leq b \;\Longrightarrow\; f(a) \leq' f(b).
\]
\end{definition}

\begin{definition}[Order Isomorphism]
Let $(M,\leq)$ and $(M',\leq')$ be partially ordered sets.
A function
\[
f : M \to M'
\]
is called an \emph{order isomorphism} if it is bijective and satisfies
\[
\forall a,b \in M,\quad
a \leq b \;\Longleftrightarrow\; f(a) \leq' f(b).
\]
\end{definition}

\begin{remark}
An order isomorphism preserves and reflects the order structure exactly.
In particular, it preserves all order-theoretic properties such as
comparability, minimal and maximal elements, bounds, and chains.

Two partially ordered sets are said to be \emph{order-isomorphic} if there
exists an order isomorphism between them.
In this case, the two ordered sets are considered structurally identical
from the standpoint of order theory.
\end{remark}


\begin{definition}[Well-Ordered Set]
An ordered set $(A,<)$ is said to be \emph{well-ordered} if every nonempty subset
$S \subseteq A$ has a least element.

That is,
\[
\forall S \subseteq A,\;
\bigl(S \neq \varnothing \;\Rightarrow\; \exists m \in S \text{ such that }
m \leq s \text{ for all } s \in S \bigr).
\]
\end{definition}

\begin{remark}
Every well-ordered set is a totally ordered set.
In particular, any two elements of a well-ordered set are comparable.

The defining property is stronger than totality: it requires the existence of
a least element in \emph{every} nonempty subset, not just the whole set.
\end{remark}

\begin{example}
\begin{enumerate}
\item $(\mathbb{N}, \leq)$ is well-ordered.
\item $(\mathbb{Z}, \leq)$ is \emph{not} well-ordered, since the subset
$\{\dots,-3,-2,-1\}$ has no least element.
\item $(\mathbb{R}, \leq)$ is not well-ordered; for example, the interval $(0,1)$
has no least element.
\end{enumerate}
\end{example}

\begin{remark}[Connection to Ordinal Numbers]
Well-ordered sets admit a powerful classification.
Two well-ordered sets are said to be \emph{order-isomorphic} if there exists an
order-preserving bijection between them.

An \emph{ordinal number} is defined (in set theory) as an equivalence class of
well-ordered sets under order isomorphism.
Thus, ordinal numbers measure the \emph{order type} of well-ordered sets rather
than their size.

Ordinal theory will be developed later; at this stage, well-ordering is the
fundamental structural concept.
\end{remark}


\begin{definition}[Chain]
A subset $C \subseteq A$ of a partially ordered set $(A,\le)$ is called a
\emph{chain} if every pair of elements in $C$ is comparable.
\end{definition}

\begin{definition}[Antichain]
A subset $S \subseteq A$ is called an \emph{antichain} if no two distinct elements
of $S$ are comparable.
\end{definition}

\begin{definition}[Initial Segment]
Let $(A,\le)$ be an ordered set.
A subset $I \subseteq A$ is an \emph{initial segment} if
\[
a \in I \text{ and } b \le a \;\Rightarrow\; b \in I.
\]
\end{definition}




























