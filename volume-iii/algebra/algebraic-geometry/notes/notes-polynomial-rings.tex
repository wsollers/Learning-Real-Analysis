% =========================================================
% Polynomial Rings
% =========================================================

% ---------------------------------------------------------
\subsubsection{Preliminary Definitions}
% ---------------------------------------------------------

\begin{definition}[Ring]
A set $R$ with two binary operations $+ : R \times R \to R$
and $\cdot : R \times R \to R$ is a \emph{ring} if:

\begin{enumerate}
\item \textbf{Additive structure:}
$(R, +)$ is an abelian group.

\item \textbf{Associativity of multiplication:}
\[
(a \cdot b) \cdot c = a \cdot (b \cdot c)
\quad \text{for all } a,b,c \in R.
\]

\item \textbf{Multiplicative identity:}
There exists $1 \in R$ such that
\[
1 \cdot a = a \cdot 1 = a
\quad \text{for all } a \in R.
\]

\item \textbf{Distributivity:}
\[
a \cdot (b + c) = a \cdot b + a \cdot c
\quad \text{and} \quad
(a + b) \cdot c = a \cdot c + b \cdot c
\quad \text{for all } a,b,c \in R.
\]
\end{enumerate}
\end{definition}

\begin{remark}
A ring generalizes a field by dropping the requirement that
nonzero elements have multiplicative inverses,
and by not requiring multiplication to be commutative.
When multiplication is commutative, we call $R$ a
\emph{commutative ring}.
\end{remark}

\begin{definition}[Commutative Ring]
A ring $R$ is \emph{commutative} if
\[
a \cdot b = b \cdot a
\quad \text{for all } a, b \in R.
\]
\end{definition}

\begin{remark}[Structural Summary]
The hierarchy of algebraic structures encountered so far is:

\begin{center}
\begin{tabular}{l|c|c}
Structure & Additive Group & Multiplicative Structure \\
\hline
Ring            & Abelian & Associative, with identity \\
Commutative Ring & Abelian & Associative, commutative, with identity \\
Field           & Abelian & Abelian on nonzero elements
\end{tabular}
\end{center}

Every field is a commutative ring, but not every commutative ring is a field.
\end{remark}

% ---------------------------------------------------------
\subsubsection{Polynomial Rings}
% ---------------------------------------------------------

\begin{definition}[Polynomial]
Let $R$ be a commutative ring.
A \emph{polynomial} in one variable over $R$ is a formal expression
\[
f = a_n x^n + a_{n-1} x^{n-1} + \cdots + a_1 x + a_0,
\]
where $n \in \mathbb{N}$, the \emph{coefficients} $a_0, a_1, \dots, a_n \in R$,
and $x$ is a formal symbol called an \emph{indeterminate}.
\end{definition}

\begin{remark}
The indeterminate $x$ is not a variable ranging over values in $R$.
It is a formal placeholder that encodes the coefficient sequence.
Two polynomials are equal if and only if all their coefficients are equal.
\end{remark}

\begin{definition}[Degree]
Let $f = a_n x^n + \cdots + a_0$ be a polynomial over $R$.
If $a_n \neq 0$, then the \emph{degree} of $f$ is
\[
\deg(f) := n.
\]
The coefficient $a_n$ is called the \emph{leading coefficient} of $f$.
A polynomial with leading coefficient $1$ is called \emph{monic}.
\end{definition}

\begin{remark}
The zero polynomial $f = 0$ has no leading coefficient.
Its degree is left undefined, or assigned $-\infty$ by convention
to preserve the identity $\deg(fg) = \deg(f) + \deg(g)$.
\end{remark}

\begin{definition}[Polynomial Ring]
Let $R$ be a commutative ring.
The \emph{polynomial ring} over $R$ in one indeterminate $x$,
denoted $R[x]$, is the set of all polynomials in $x$ with
coefficients in $R$, equipped with addition and multiplication
defined by:
\[
\left(\sum_{i} a_i x^i\right) + \left(\sum_{i} b_i x^i\right)
:= \sum_{i} (a_i + b_i) x^i,
\]
\[
\left(\sum_{i} a_i x^i\right) \cdot \left(\sum_{j} b_j x^j\right)
:= \sum_{k} \left(\sum_{i+j=k} a_i b_j\right) x^k.
\]
\end{definition}

\begin{example}
Let $R = \mathbb{Z}$ and consider
\[
f = 2x^2 + 3x + 1, \qquad g = x + 4 \quad \in \mathbb{Z}[x].
\]
Then:
\[
f + g = 2x^2 + 4x + 5,
\]
\[
f \cdot g = 2x^3 + 11x^2 + 13x + 4.
\]
\end{example}

\begin{remark}[Structural Summary]
With these operations, $R[x]$ is itself a commutative ring.
The original ring $R$ embeds into $R[x]$ as the constant polynomials.

\begin{center}
\begin{tabular}{l|c}
Property & Holds in $R[x]$? \\
\hline
Commutative ring & Always \\
Field & Only in degenerate cases \\
$R$ embeds in $R[x]$ & Always
\end{tabular}
\end{center}
\end{remark}

% ---------------------------------------------------------
\subsubsection{Polynomial Rings in Several Variables}
% ---------------------------------------------------------

\begin{definition}[Polynomial Ring in $n$ Variables]
Let $R$ be a commutative ring and let $n \in \mathbb{N}$.
The \emph{polynomial ring in $n$ variables} over $R$, denoted
\[
R[x_1, x_2, \dots, x_n],
\]
is defined inductively by
\[
R[x_1, \dots, x_n] := R[x_1, \dots, x_{n-1}][x_n].
\]
Its elements are finite sums of the form
\[
f = \sum_{\alpha} a_\alpha \, x^\alpha,
\]
where the sum ranges over multi-indices
$\alpha = (\alpha_1, \dots, \alpha_n) \in \mathbb{N}^n$,
the coefficients $a_\alpha \in R$, and
$x^\alpha := x_1^{\alpha_1} \cdots x_n^{\alpha_n}$.
\end{definition}

\begin{definition}[Monomial]
A \emph{monomial} in $R[x_1, \dots, x_n]$ is a polynomial
of the form
\[
x^\alpha = x_1^{\alpha_1} \cdots x_n^{\alpha_n}
\]
for some multi-index $\alpha \in \mathbb{N}^n$.
\end{definition}

\begin{definition}[Total Degree]
The \emph{total degree} of the monomial $x^\alpha$ is
\[
|\alpha| := \alpha_1 + \alpha_2 + \cdots + \alpha_n.
\]
The degree of a polynomial $f \in R[x_1,\dots,x_n]$ is
the maximum total degree among all monomials with nonzero coefficient.
\end{definition}

\begin{example}
In $\mathbb{R}[x,y,z]$, the polynomial
\[
f = 3x^2 y + xy^2 z - 5z^3
\]
has three terms with total degrees $3$, $4$, and $3$ respectively.
Hence $\deg(f) = 4$.
\end{example}

\begin{remark}[Relevance to Algebraic Geometry]
Polynomial rings in several variables are the foundational
algebraic object of algebraic geometry.
The geometric objects studied in subsequent sections ---
affine varieties, ideals, coordinate rings ---
are all defined in terms of $k[x_1, \dots, x_n]$
for a field $k$.
The interplay between the algebra of $k[x_1,\dots,x_n]$
and the geometry of its zero sets is the central theme
of Clader--Ross.
\end{remark}

% ---------------------------------------------------------
\subsubsection{Ideals}
% ---------------------------------------------------------

\begin{definition}[Ideal]
Let $R$ be a commutative ring.
A subset $I \subseteq R$ is an \emph{ideal} of $R$ if:
\begin{enumerate}
\item $0 \in I$,
\item $a + b \in I$ for all $a, b \in I$,
\item $r \cdot a \in I$ for all $r \in R$ and $a \in I$.
\end{enumerate}
\end{definition}

\begin{definition}[Generated Ideal]
Let $R$ be a commutative ring and let $f_1, \dots, f_m \in R$.
The \emph{ideal generated by} $f_1, \dots, f_m$ is
\[
\langle f_1, \dots, f_m \rangle
:= \left\{ \sum_{i=1}^m r_i f_i : r_i \in R \right\}.
\]
An ideal of this form is called \emph{finitely generated}.
\end{definition}

\begin{definition}[Finitely Generated Ideal]
An ideal $I \subseteq R$ is \emph{finitely generated} if there exist
$f_1, \dots, f_m \in R$ such that $I = \langle f_1, \dots, f_m \rangle$.
\end{definition}

\begin{definition}[Noetherian Ring]
A commutative ring $R$ is \emph{Noetherian} if every ideal of $R$
is finitely generated.
\end{definition}

\begin{theorem}[Hilbert Basis Theorem]
If $R$ is a Noetherian ring, then the polynomial ring $R[x]$
is also Noetherian.

In particular, $k[x_1, \dots, x_n]$ is Noetherian for any field $k$.
\end{theorem}

\begin{remark}
The Hilbert Basis Theorem guarantees that every ideal in
$k[x_1, \dots, x_n]$ is finitely generated.
This is a foundational finiteness result:
it means every algebraic variety can be cut out by
finitely many polynomial equations.
\end{remark}

% ---------------------------------------------------------
\subsubsection{Quotient Rings}
% ---------------------------------------------------------

\begin{definition}[Quotient Ring]
Let $R$ be a commutative ring and let $I \subseteq R$ be an ideal.
The \emph{quotient ring} $R/I$ is the set of cosets
\[
R/I := \{ a + I : a \in R \},
\]
equipped with addition and multiplication defined by
\[
(a + I) + (b + I) := (a + b) + I,
\]
\[
(a + I) \cdot (b + I) := (a \cdot b) + I.
\]
\end{definition}

\begin{remark}
The quotient ring $R/I$ is a commutative ring.
Elements of $R/I$ are equivalence classes under the relation
$a \sim b \iff a - b \in I$.
\end{remark}

\begin{example}
In $\mathbb{R}[x]$, consider the ideal $I = \langle x^2 + 1 \rangle$.
The quotient ring
\[
\mathbb{R}[x] / \langle x^2 + 1 \rangle \;\cong\; \mathbb{C},
\]
since in the quotient, $x$ satisfies $x^2 = -1$,
which is precisely the defining relation of $i \in \mathbb{C}$.
\end{example}

\begin{remark}[Structural Position]
Quotient rings of polynomial rings are the coordinate rings
of affine varieties, which are studied in depth in the next chapter.
The passage
\[
k[x_1,\dots,x_n] \;\longrightarrow\; k[x_1,\dots,x_n]/I
\]
is the algebraic encoding of restricting from all of $\mathbb{A}^n$
to the variety $V(I)$.
\end{remark}