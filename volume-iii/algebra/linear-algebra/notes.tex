% =========================================================
% Breadcrumb
% =========================================================
\begin{tcolorbox}[
  colback=gray!6,
  colframe=gray!40,
  arc=2pt,
  left=8pt, right=8pt, top=6pt, bottom=6pt,
  title={\small\textbf{Where You Are in the Journey}},
  fonttitle=\small\bfseries
]
\begin{center}
\small
Propositional Logic
$\;\to\;$ Predicate Calculus
$\;\to\;$ Sets \& Functions
$\;\to\;$ Proof Techniques
$\;\to\;$ $\mathbb{R}$
$\;\to\;$ Algebraic Structures
$\;\to\;$ \textbf{Linear Algebra}
$\;\to\;$ Topology
$\;\to\;$ $\cdots$
\end{center}

\medskip
\noindent\textbf{How we got here.}
Algebraic structures gave us the hierarchy of groups, rings, and fields.
A field is the scalar system: a set of numbers with addition and
multiplication. An abelian group is the additive backbone.
Linear algebra asks: what happens when you combine an abelian group
of ``vectors'' with scalar multiplication from a field?

\medskip
\noindent\textbf{What this chapter builds.}
We develop vector spaces from the axioms, then build the theory of
subspaces, linear combinations, span, linear independence, bases,
and dimension. Linear maps are the central objects of study ---
not matrices, which are only representations relative to a basis.

\medskip
\noindent\textbf{Where this leads.}
Every proof about eigenvalues, the spectral theorem, and inner product
spaces relies directly on the axioms established here.
Functional analysis generalises vector spaces to infinite dimensions;
topology provides the analytic tools for that generalisation.
\end{tcolorbox}
\vspace{1em}

% =========================================================
% Linear Algebra — Structural Roadmap
% Driver Text: Sheldon Axler, Linear Algebra Done Right
% =========================================================

\subsection{Linear Algebra}

% =========================================================
% Structural Roadmap
% =========================================================


\subsubsection*{Structural Roadmap}

The development of linear algebra in this project follows
the definition--theorem--structure architecture used
throughout the analysis volumes.

The primary driver is \textit{Linear Algebra Done Right}
by Sheldon Axler. The emphasis is on linear maps,
invariant structure, and conceptual clarity
rather than computational techniques.

Each major topic is organized as:
\begin{center}
\textbf{Definitions $\longrightarrow$ Main Theorems $\longrightarrow$ Consequences and Structural Insight}
\end{center}

The global progression is:

\begin{enumerate}
  \item Vector spaces and subspaces
  \item Linear combinations, span, and linear independence
  \item Bases and dimension
  \item Linear maps
  \item Null space and range
  \item Matrix representations of linear maps
  \item Eigenvalues and eigenvectors
  \item Invariant subspaces
  \item Triangularization and diagonalization
  \item Inner product spaces
  \item Orthogonality and projections
  \item Self-adjoint and normal operators
  \item Spectral theorem (finite-dimensional case)
\end{enumerate}

\vspace{1em}

\begin{remark}
This treatment prioritizes structural understanding over
algorithmic manipulation. Determinants are introduced only
after linear maps and eigenvalues have been conceptually
understood.
\end{remark}

\begin{remark}
The central object of study is not matrices,
but linear maps between vector spaces.
Matrices serve only as representations relative to a basis.
\end{remark}

\begin{remark}[Structural Position]
Linear algebra is developed independently of analysis.
Its structural viewpoint will later support operator theory,
Hilbert spaces, and functional analysis.
\end{remark}

% =========================================================
% I. Core Linear Structure
% =========================================================

% =========================================================
% Vector Spaces — Main File
% =========================================================
% Input order follows logical dependency:
%   prerequisites → definitions → examples → propositions

% =========================================================
% Abelian Groups and Fields
% =========================================================

% ---------------------------------------------------------
\subsubsection{Preliminary Definitions}
% ---------------------------------------------------------

\begin{definition}[Abelian group]
A set $G$ with a binary operation $\ast : G \times G \to G$
is an \emph{abelian group} if:

\begin{enumerate}
\item \textbf{Associativity:}
\[
(a \ast b) \ast c = a \ast (b \ast c)
\quad\text{for all } a,b,c \in G.
\]

\item \textbf{Identity element:}
There exists $e \in G$ such that
\[
a \ast e = e \ast a = a
\quad\text{for all } a \in G.
\]

\item \textbf{Inverse element:}
For each $a \in G$ there exists $b \in G$ such that
\[
a \ast b = b \ast a = e.
\]

\item \textbf{Commutativity:}
\[
a \ast b = b \ast a
\quad\text{for all } a,b \in G.
\]
\end{enumerate}
\end{definition}

\begin{remark}
An abelian group is simply a group whose operation is commutative.
\end{remark}

\begin{definition}[Field --- structural form]
A \emph{field} is a set $\mathbb{F}$ equipped with two operations
$+$ and $\cdot$ such that:

\begin{itemize}
\item $(\mathbb{F}, +)$ is an abelian group,
\item $(\mathbb{F}\setminus\{0\}, \cdot)$ is an abelian group,
\item Multiplication distributes over addition.
\end{itemize}
\end{definition}

\begin{remark}[Expanded Axioms]
Expanding the abelian group axioms yields the familiar list:

\begin{itemize}
\item Associativity, commutativity, identity, and inverse for addition.
\item Associativity, commutativity, identity, and inverse (for nonzero elements)
      for multiplication.
\item The distributive law.
\end{itemize}

Thus the usual field axioms are not independent assumptions,
but consequences of the two abelian group structures together with distributivity.
\end{remark}

\begin{remark}[Structural Summary]
A field consists of two compatible abelian group structures:

\begin{center}
\begin{tabular}{l|c|c}
Structure & Additive Group & Multiplicative Structure \\
\hline
Field & Abelian & Abelian (on nonzero elements)
\end{tabular}
\end{center}

Distributivity links the two operations.
\end{remark}
% =========================================================
% The Complex Numbers
% =========================================================

\subsubsection{The Complex Numbers}

\begin{definition}[Complex Numbers]
The set of \emph{complex numbers}, denoted $\mathbb{C}$, is defined as
\[
\mathbb{C} := \{(a,b) : a,b \in \mathbb{R}\},
\]
with operations defined by
\[
(a,b) + (c,d) := (a+c,\; b+d),
\]
\[
(a,b)\cdot(c,d) := (ac - bd,\; ad + bc).
\]
\end{definition}

\begin{remark}
We identify $(a,b)$ with the symbol
\[
a + bi,
\]
where $i := (0,1)$ satisfies $i^2 = -1$.
\end{remark}

\begin{remark}[Field structure of $\mathbb{C}$]
With the operations defined above, $\mathbb{C}$ satisfies the field axioms:

\begin{itemize}
\item $(\mathbb{C}, +)$ is an abelian group,
\item $(\mathbb{C}\setminus\{0\}, \cdot)$ is an abelian group,
\item multiplication distributes over addition.
\end{itemize}

Hence $\mathbb{C}$ is a field containing $\mathbb{R}$ as the subset
$\{(a,0): a \in \mathbb{R}\}$.
\end{remark}
% =========================================================
% Vector Space Definition
% =========================================================

\subsubsection{Basic Definitions}

\begin{definition}[Vector Space --- informal description]
A vector space is a set equipped with addition and scalar multiplication
that behave like familiar vector arithmetic.
\end{definition}

\begin{remark}
The informal description suggests structure, but precision requires axioms.
\end{remark}

\begin{definition}[Vector Space]
Let $\mathbb{F}$ be a field.
A \textbf{vector space} over $\mathbb{F}$ is a triple $(V, +, \cdot)$ where $V$ is a set,
$+: V \times V \to V$ is a binary operation called \textbf{vector addition}, and
$\cdot : \mathbb{F} \times V \to V$ is a binary operation called \textbf{scalar multiplication},
such that:

\begin{itemize}
    \item $(V, +)$ is an abelian group,
    \item $\alpha (u + v) = \alpha u + \alpha v$
          for all $\alpha \in \mathbb{F}$ and $u, v \in V$,
    \item $(\alpha + \beta) v = \alpha v + \beta v$
          for all $\alpha, \beta \in \mathbb{F}$ and $v \in V$,
    \item $(\alpha \beta) v = \alpha (\beta v)$
          for all $\alpha, \beta \in \mathbb{F}$ and $v \in V$,
    \item $1_{\mathbb{F}} v = v$ for all $v \in V$.
\end{itemize}

The elements of $V$ are called \textbf{vectors} and the elements of $\mathbb{F}$
are called \textbf{scalars}.
\end{definition}

\begin{remark}[Expanded Axioms]
Expanding the abelian group structure of $(V,+)$ yields the full list of vector space axioms:

\begin{itemize}
\item \textbf{Associativity of addition:}
      $(u + v) + w = u + (v + w)$ for all $u,v,w \in V$.
\item \textbf{Commutativity of addition:}
      $u + v = v + u$ for all $u,v \in V$.
\item \textbf{Additive identity:}
      There exists $\mathbf{0} \in V$ such that $v + \mathbf{0} = v$ for all $v \in V$.
\item \textbf{Additive inverse:}
      For each $v \in V$ there exists $-v \in V$ such that $v + (-v) = \mathbf{0}$.
\item \textbf{Distributivity over vector addition:}
      $\alpha(u + v) = \alpha u + \alpha v$
      for all $\alpha \in \mathbb{F}$, $u,v \in V$.
\item \textbf{Distributivity over scalar addition:}
      $(\alpha + \beta)v = \alpha v + \beta v$
      for all $\alpha,\beta \in \mathbb{F}$, $v \in V$.
\item \textbf{Compatibility:}
      $(\alpha\beta)v = \alpha(\beta v)$
      for all $\alpha,\beta \in \mathbb{F}$, $v \in V$.
\item \textbf{Unit law:} $1_{\mathbb{F}} v = v$ for all $v \in V$.
\end{itemize}

Thus the structural definition compresses eight axioms into one abelian group condition
plus four scalar multiplication laws.
\end{remark}

\begin{remark}[Structural Summary]
A vector space combines an abelian group with a compatible scalar action:

\begin{center}
\begin{tabular}{l|c|c}
Structure & Additive Group & Scalar Action \\
\hline
Vector Space & Abelian & $\mathbb{F}$-linear
\end{tabular}
\end{center}

The scalar multiplication axioms express compatibility between the field $\mathbb{F}$
and the group $(V,+)$.
\end{remark}

\begin{definition}[Zero vector]
Let $V$ be a vector space over a field $\mathbb{F}$.
The \emph{zero vector} of $V$, denoted $\mathbf{0}$, is the additive identity
of the abelian group $(V,+)$; that is, the unique element $\mathbf{0} \in V$
such that
\[
v + \mathbf{0} = v
\quad \text{for all } v \in V.
\]
\end{definition}
% =========================================================
% Tuples and Lists
% =========================================================

\subsubsection{Tuples and Lists}

\begin{definition}[$n$-tuple]
Let $n \in \mathbb{N}$ and let $X$ be a set.
An \emph{$n$-tuple} of elements of $X$ is a function
\[
f : \{1,2,\dots,n\} \to X.
\]
We write such a function as $(x_1,\dots,x_n)$, where $x_i := f(i)$ for each $i$.
\end{definition}

\begin{definition}[Finite list]
Let $X$ be a set.
A \emph{finite list} of elements of $X$ is a function
$f : \{1,2,\dots,n\} \to X$ for some $n \in \mathbb{N}$.
\end{definition}

\begin{definition}[Set]
A \emph{set} is a collection of distinct elements with no inherent ordering.
\end{definition}

\begin{example}
Consider the elements $1$ and $2$.
The set $\{1,2,2\}$ equals $\{1,2\}$ because sets do not record duplicates.
However, the list $(1,2,2)$ differs from $(1,2)$ because lists record both
order and repetition.
\end{example}

\begin{definition}[Coordinate]
Let $\mathbb{F}$ be a field and let $v = (v_1,\dots,v_n) \in \mathbb{F}^n$.
The element $v_i$ is called the \emph{$i$th coordinate} of $v$.
\end{definition}
% =========================================================
% Function Spaces
% =========================================================

\subsubsection{The Function-Space Viewpoint}

\begin{definition}[Set Exponentiation]
Let $X$ and $Y$ be sets. Define
\[
Y^X := \{ f : X \to Y \},
\]
the set of all functions from $X$ to $Y$.
\end{definition}

\begin{remark}
If $|X| = n$ and $|Y| = m$ are finite, then $|Y^X| = m^n$.
Thus the notation $Y^X$ agrees with the rules of cardinal arithmetic.
\end{remark}

\begin{definition}[$\mathbb{F}^S$ --- the function space over a set]
Let $\mathbb{F}$ be a field and let $S$ be a set. Define
\[
\mathbb{F}^S := \{ f : S \to \mathbb{F} \},
\]
the set of all functions from $S$ to $\mathbb{F}$, with \textbf{pointwise} operations:
\[
(f + g)(x) := f(x) + g(x),
\qquad
(\lambda f)(x) := \lambda f(x),
\]
for all $f,g \in \mathbb{F}^S$, $\lambda \in \mathbb{F}$, and $x \in S$.
\end{definition}

\begin{example}[$\mathbb{F}^S$ is a vector space]
With the pointwise operations above, $\mathbb{F}^S$ is a vector space over $\mathbb{F}$.
The zero vector is $\mathbf{0}(x) = 0$ for all $x \in S$,
and the additive inverse of $f$ is $(-f)(x) = -f(x)$ for all $x \in S$.
All vector space axioms follow from the field axioms applied pointwise; for example,
\[
((f+g)+h)(x) = f(x)+g(x)+h(x) = (f+(g+h))(x),
\]
so associativity holds because addition in $\mathbb{F}$ is associative.
Every other axiom follows similarly.
\end{example}

\begin{remark}[Zero vector in $\mathbb{F}^S$]
In the function space $\mathbb{F}^S$, the zero vector is the function
\[
\mathbf{0} : S \to \mathbb{F}
\quad \text{defined by} \quad
\mathbf{0}(x) = 0 \text{ for all } x \in S.
\]
\end{remark}

% ---------------------------------------------------------
\subsubsection{Coordinate Spaces as Function Spaces}

\begin{definition}[Coordinate space as a function space]
Let $\mathbb{F}$ be a field and let $[n] := \{1,\dots,n\}$. Then
\[
\mathbb{F}^n \;=\; \mathbb{F}^{[n]} \;=\; \{ f : [n] \to \mathbb{F} \}.
\]
An element $v \in \mathbb{F}^n$ is a function $v : [n] \to \mathbb{F}$,
written $v = (v_1,\dots,v_n)$ with $v_i := v(i)$.
Thus an $n$-tuple is simply a function from a finite index set,
and $\mathbb{F}^n$ inherits its vector space structure as a special case of $\mathbb{F}^S$.
\end{definition}

\begin{remark}
With the componentwise operations defined above, $\mathbb{F}^n$ is a vector space
over $\mathbb{F}$ for any field $\mathbb{F}$ and any $n \in \mathbb{N}$.
\end{remark}

% ---------------------------------------------------------
\subsubsection{The Space $\mathbb{F}^{\infty}$}

\begin{definition}[$\mathbb{F}^{\infty}$ --- the space of sequences]
Let $\mathbb{F}$ be a field. Define
\[
\mathbb{F}^{\infty}
:= \mathbb{F}^{\mathbb{N}}
= \{ f : \mathbb{N} \to \mathbb{F} \}
= \{ (x_1,x_2,\dots) : x_k \in \mathbb{F} \text{ for all } k \in \mathbb{N} \}.
\]
An element of $\mathbb{F}^{\infty}$ is called a \emph{sequence} of elements of $\mathbb{F}$.
With pointwise operations inherited from $\mathbb{F}^S$,
$\mathbb{F}^{\infty}$ is a vector space over $\mathbb{F}$.
\end{definition}

\begin{remark}[Conceptual Unification]
The passage
\[
\mathbb{F}^n
\quad\longrightarrow\quad
\mathbb{F}^{\mathbb{N}}
\quad\longrightarrow\quad
\mathbb{F}^S
\]
shows that finite tuples, sequences, and general function spaces are all instances
of the same construction. Finite-dimensional linear algebra is the special case
where the index set is finite, and this viewpoint connects directly to
infinite-dimensional spaces and functional analysis.
\end{remark}
% =========================================================
% Basic Propositions of Vector Spaces
% =========================================================

\subsubsection{Basic Propositions}

\begin{proposition}[Uniqueness of the Additive Identity]
Let $V$ be a vector space over a field $\mathbb{F}$.
The additive identity of $V$ is unique.
\end{proposition}
\begin{proof}
Suppose $\mathbf{0}$ and $\mathbf{0}'$ are both additive identities in $V$. Then
\[
\mathbf{0} = \mathbf{0} + \mathbf{0}' = \mathbf{0}',
\]
where the first equality holds because $\mathbf{0}'$ is an additive identity,
and the second because $\mathbf{0}$ is an additive identity.
Hence $\mathbf{0} = \mathbf{0}'$.
\end{proof}

% ---------------------------------------------------------

\begin{proposition}[Uniqueness of the Additive Inverse]
Let $V$ be a vector space over a field $\mathbb{F}$.
For each $v \in V$, the additive inverse of $v$ is unique.
\end{proposition}
\begin{proof}
Let $v \in V$ and suppose $w, w' \in V$ both satisfy
\[
v + w = \mathbf{0}
\qquad \text{and} \qquad
v + w' = \mathbf{0}.
\]
Then
\[
w
= w + \mathbf{0}
= w + (v + w')
= (w + v) + w'
= \mathbf{0} + w'
= w',
\]
where we used the additive identity, the assumption on $w'$,
associativity, the assumption on $w$, and the additive identity again.
Hence $w = w'$.
\end{proof}

% ---------------------------------------------------------

\begin{proposition}[Scalar Multiplication by Zero and Negation]
Let $V$ be a vector space over a field $\mathbb{F}$,
let $v \in V$, and let $\alpha \in \mathbb{F}$. Then:
\begin{enumerate}
    \item $0_{\mathbb{F}} v = \mathbf{0}$,
    \item $\alpha \mathbf{0} = \mathbf{0}$,
    \item $(-1_{\mathbb{F}}) v = -v$.
\end{enumerate}
\end{proposition}
% Proof: Scalar Multiplication by Zero and Negation
% Three-column format: Tag / Step / Justification
%
% Column widths (must sum to <= \textwidth, leaving room for inter-column padding):
%   Tag:           0.07\textwidth  — short label
%   Step:          0.36\textwidth  — math, centered via >{\centering\arraybackslash}
%   Justification: 0.50\textwidth  — wrapped prose

\clearpage
\begin{center}
  {\large\bfseries Proof: Scalar Multiplication by Zero and Negation}\\[4pt]
  {\small\itshape (Three-column format: tag / step / justification)}\\[6pt]
  {\small
    \tagDU\ = Definition Unpacked \quad
    \tagTA\ = Theorem Applied \quad
    \tagAM\ = Algebraic Manipulation
  }
\end{center}

\bigskip

% ------------------------------------------------------------------
\noindent\textit{Part 1: $0_{\mathbb{F}} v = \mathbf{0}$.}
\medskip

\noindent
\begin{tabular}{p{0.07\textwidth}
                >{\centering\arraybackslash}p{0.36\textwidth}
                p{0.50\textwidth}}
\toprule
\textbf{Tag} & \textbf{Step} & \textbf{Justification} \\
\midrule
\addlinespace[4pt]

\tagAM
&
$0_{\mathbb{F}} v = (0_{\mathbb{F}} + 0_{\mathbb{F}}) v
= 0_{\mathbb{F}} v + 0_{\mathbb{F}} v$
&
$0_{\mathbb{F}} + 0_{\mathbb{F}} = 0_{\mathbb{F}}$ in $\mathbb{F}$;
distributivity over scalar addition.
\\[10pt]

\tagTA
&
$\mathbf{0} = 0_{\mathbb{F}} v$
&
Add $-(0_{\mathbb{F}} v)$ to both sides; uniqueness of the
additive inverse collapses the right side.
\hfill$\square$
\\[6pt]

\bottomrule
\end{tabular}

\bigskip

% ------------------------------------------------------------------
\noindent\textit{Part 2: $\alpha \mathbf{0} = \mathbf{0}$.}
\medskip

\noindent
\begin{tabular}{p{0.07\textwidth}
                >{\centering\arraybackslash}p{0.36\textwidth}
                p{0.50\textwidth}}
\toprule
\textbf{Tag} & \textbf{Step} & \textbf{Justification} \\
\midrule
\addlinespace[4pt]

\tagAM
&
$\alpha \mathbf{0} = \alpha(\mathbf{0} + \mathbf{0})
= \alpha \mathbf{0} + \alpha \mathbf{0}$
&
$\mathbf{0} + \mathbf{0} = \mathbf{0}$ in $V$;
distributivity over vector addition.
\\[10pt]

\tagTA
&
$\mathbf{0} = \alpha \mathbf{0}$
&
Add $-(\alpha \mathbf{0})$ to both sides; uniqueness of the
additive inverse collapses the right side.
\hfill$\square$
\\[6pt]

\bottomrule
\end{tabular}

\bigskip

% ------------------------------------------------------------------
\noindent\textit{Part 3: $(-1_{\mathbb{F}}) v = -v$.}
\medskip

\noindent
\begin{tabular}{p{0.07\textwidth}
                >{\centering\arraybackslash}p{0.36\textwidth}
                p{0.50\textwidth}}
\toprule
\textbf{Tag} & \textbf{Step} & \textbf{Justification} \\
\midrule
\addlinespace[4pt]

\tagDU
&
$\text{Show } v + (-1_{\mathbb{F}})v = \mathbf{0}$
&
By uniqueness of the additive inverse, it suffices to
verify this characterizing property.
\\[10pt]

\tagAM
&
$v + (-1_{\mathbb{F}})v
= 1_{\mathbb{F}} v + (-1_{\mathbb{F}}) v
= (1_{\mathbb{F}} + (-1_{\mathbb{F}})) v$
&
Unit law; distributivity over scalar addition.
\\[10pt]

\tagTA
&
$(1_{\mathbb{F}} + (-1_{\mathbb{F}})) v = 0_{\mathbb{F}} v = \mathbf{0}$
&
$1_{\mathbb{F}} + (-1_{\mathbb{F}}) = 0_{\mathbb{F}}$ in $\mathbb{F}$;
Part~1.
\\[10pt]

\tagDU
&
$\therefore\; (-1_{\mathbb{F}}) v = -v$
&
Uniqueness of the additive inverse (Proposition above).
\hfill$\blacksquare$
\\[6pt]

\bottomrule
\end{tabular}

\bigskip\bigskip
\noindent\hrule
\medskip
\noindent\textbf{Study Notes.}
\medskip

\noindent
\begin{tabular}{p{0.22\textwidth} p{0.72\textwidth}}
\toprule
\textbf{Tool} & \textbf{Role in this proof} \\
\midrule
\addlinespace[4pt]
Self-cancellation trick
&
Parts 1 and 2 use the same strategy: write zero as a sum of two zeros,
distribute, then cancel. The argument runs in two directions ---
once through scalar addition, once through vector addition.
\\[8pt]
Characterization by property
&
Part 3 avoids direct manipulation by instead verifying the
defining property of $-v$, then invoking uniqueness.
This is cleaner and more structurally transparent.
\\[8pt]
Uniqueness results
&
Both the uniqueness of the additive identity and the uniqueness of the
additive inverse are essential; without them the cancellation steps
in Parts 1--2 and the conclusion of Part 3 would not be justified.
\\[6pt]
\bottomrule
\end{tabular}

% =========================================================
% II. Linear Maps
% =========================================================

% \input{volume-iii/algebra/linear-algebra/notes/notes-linear-independence}
% \input{volume-iii/algebra/linear-algebra/notes/notes-basis-dimension}
% \input{volume-iii/algebra/linear-algebra/notes/notes-linear-maps}
% \input{volume-iii/algebra/linear-algebra/notes/notes-null-range}
% \input{volume-iii/algebra/linear-algebra/notes/notes-matrix-representations}

% =========================================================
% III. Eigenstructure
% =========================================================

% \input{volume-iii/algebra/linear-algebra/notes/notes-eigenvalues}
% \input{volume-iii/algebra/linear-algebra/notes/notes-invariant-subspaces}
% \input{volume-iii/algebra/linear-algebra/notes/notes-diagonalization}

% =========================================================
% IV. Inner Product Structure
% =========================================================

% \input{volume-iii/algebra/linear-algebra/notes/notes-inner-products}
% \input{volume-iii/algebra/linear-algebra/notes/notes-orthogonality}
% \input{volume-iii/algebra/linear-algebra/notes/notes-spectral-theorem}

% =========================================================
% Future Sections (To Be Developed)
% =========================================================

% \input{volume-iii/algebra/linear-algebra/notes/notes-determinants}
% \input{volume-iii/algebra/linear-algebra/notes/notes-jordan-form}
% \input{volume-iii/algebra/linear-algebra/notes/notes-advanced-operators}

% =========================================================
% Logical Positioning
% =========================================================

