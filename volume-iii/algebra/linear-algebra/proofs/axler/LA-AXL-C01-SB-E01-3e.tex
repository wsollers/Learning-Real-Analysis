% =========================================================
% Proof Sheets — Axler (Linear Algebra Done Right)
% Chapter 1 — Vector Spaces
% =========================================================
% --------------------------------------
% Proof: LA-AXL-C01-1B-E01
% --------------------------------------
\clearpage
\phantomsection
\hypertarget{proof-LA-AXL-C01-SB-E01-3e}{}
\section*{Proof — LA-AXL-C01-SB-E01-3e}
\noindent
\hyperlink{ws-LA-AXL-C01-SB-E01}{\textbf{← Back to worksheet}}
\noindent
\textbf{Source.}
\srccite{AxlerLADR3e}{Chapter~1, Section~B, Question~1}.
\vspace{0.75em}
\noindent
\textbf{Goal.}
Let $\mathbb{F}$ be a field and let $S$ be a nonempty set.
Prove that $\mathbb{F}^S$, with addition and scalar multiplication defined
pointwise by
\[
(f+g)(x) := f(x) + g(x),
\qquad
(\lambda f)(x) := \lambda f(x),
\]
for all $f,g \in \mathbb{F}^S$, $\lambda \in \mathbb{F}$,
and $x \in S$, is a vector space over $\mathbb{F}$.
\vspace{0.75em}
\noindent
\textbf{Proof.}
\begin{proof}
We verify the vector space axioms.

\paragraph*{(A1) Associativity of addition.}
Let $u, v, w \in \mathbb{F}^S$ and let $x \in S$. Then
\begin{align*}
((u + v) + w)(x)
&= (u + v)(x) + w(x)       \\
&= (u(x) + v(x)) + w(x)    \\
&= u(x) + (v(x) + w(x))    \\
&= u(x) + (v + w)(x)       \\
&= (u + (v + w))(x).
\end{align*}
Since $x \in S$ was arbitrary, $(u + v) + w = u + (v + w)$.

\paragraph*{(A2) Commutativity of addition.}
Let $u, v \in \mathbb{F}^S$ and let $x \in S$. Then
\begin{align*}
(u + v)(x)
&= u(x) + v(x)  \\
&= v(x) + u(x)  \\
&= (v + u)(x).
\end{align*}
Since $x \in S$ was arbitrary, $u + v = v + u$.

\paragraph*{(A3) Additive identity.}
Define $\mathbf{0} : S \to \mathbb{F}$ by $\mathbf{0}(x) = 0$ for all $x \in S$.
Let $f \in \mathbb{F}^S$ and let $x \in S$. Then
\begin{align*}
(f + \mathbf{0})(x)
&= f(x) + \mathbf{0}(x)  \\
&= f(x) + 0              \\
&= f(x).
\end{align*}
Since $x \in S$ was arbitrary, $f + \mathbf{0} = f$.

\paragraph*{(A4) Additive inverses.}
Define $-f : S \to \mathbb{F}$ by $(-f)(x) := -f(x)$ for all $x \in S$.
Let $f \in \mathbb{F}^S$ and let $x \in S$. Then
\begin{align*}
(f + (-f))(x)
&= f(x) + (-f)(x)  \\
&= f(x) + (-f(x))  \\
&= 0               \\
&= \mathbf{0}(x).
\end{align*}
Since $x \in S$ was arbitrary, $f + (-f) = \mathbf{0}$.

\paragraph*{(A5) Multiplicative identity.}
Let $f \in \mathbb{F}^S$ and let $x \in S$. Then
\begin{align*}
(1 \cdot f)(x)
&= 1 \cdot f(x)  \\
&= f(x).
\end{align*}
Since $x \in S$ was arbitrary, $1 \cdot f = f$.

\paragraph*{(A6) Distributivity over vector addition.}
Let $\lambda \in \mathbb{F}$, $f, g \in \mathbb{F}^S$, and $x \in S$. Then
\begin{align*}
(\lambda(f + g))(x)
&= \lambda \cdot (f + g)(x)      \\
&= \lambda \cdot (f(x) + g(x))   \\
&= \lambda f(x) + \lambda g(x)   \\
&= (\lambda f)(x) + (\lambda g)(x) \\
&= (\lambda f + \lambda g)(x).
\end{align*}
Since $x \in S$ was arbitrary, $\lambda(f + g) = \lambda f + \lambda g$.

\paragraph*{(A7) Distributivity over scalar addition.}
Let $\lambda, \mu \in \mathbb{F}$, $f \in \mathbb{F}^S$, and $x \in S$. Then
\begin{align*}
((\lambda + \mu) f)(x)
&= (\lambda + \mu) \cdot f(x)        \\
&= \lambda f(x) + \mu f(x)           \\
&= (\lambda f)(x) + (\mu f)(x)       \\
&= (\lambda f + \mu f)(x).
\end{align*}
Since $x \in S$ was arbitrary, $(\lambda + \mu)f = \lambda f + \mu f$.

\paragraph*{(A8) Compatibility of scalar multiplication.}
Let $\lambda, \mu \in \mathbb{F}$, $f \in \mathbb{F}^S$, and $x \in S$. Then
\begin{align*}
((\lambda \mu) f)(x)
&= (\lambda \mu) \cdot f(x)      \\
&= \lambda \cdot (\mu \cdot f(x))  \\
&= \lambda \cdot (\mu f)(x)        \\
&= (\lambda(\mu f))(x).
\end{align*}
Since $x \in S$ was arbitrary, $(\lambda \mu)f = \lambda(\mu f)$.

\vspace{0.5em}
All eight vector space axioms hold, so $\mathbb{F}^S$ is a vector space
over $\mathbb{F}$.
\end{proof}

Thus $\mathbb{F}^S$ is a vector space.
\end{proof}




\vspace{2cm}

\noindent
\hyperlink{ws-LA-AXL-C01-SB-E01}{\textbf{← Back to worksheet}}
