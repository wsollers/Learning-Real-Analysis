% =========================================================
% Proof Sheets — Axler (Linear Algebra Done Right)
% Chapter 1 — Vector Spaces
% Section 1C — Subspaces
% =========================================================

% --------------------------------------
% Proof: LA-AXL-C01-1C-X1.35
% --------------------------------------
\clearpage
\phantomsection
\hypertarget{proof-LA-AXL-C01-1C-X1.35}{}
\section*{Proof — LA-AXL-C01-1C-X1.35}

\noindent
\hyperlink{ws-LA-AXL-C01-1C-X1.35}{\textbf{← Back to worksheet}}

\vspace{0.5em}
\noindent
\textbf{Source.}
\srccite{AxlerLinearAlgebraDoneRight4e}{Chapter~1, \S1C, Example~1.35}.

\vspace{0.75em}
\noindent
\textbf{Goal.}
Verify the following five subspace claims.

\vspace{0.5em}
\noindent
\textbf{Recall (Subspace Test).}
A nonempty subset $U$ of a vector space $V$ over $\mathbf{F}$ is a subspace
iff it satisfies three conditions:
\begin{enumerate}
  \item[(S1)] \emph{Additive identity:} $\mathbf{0} \in U$.
  \item[(S2)] \emph{Closed under addition:} $u, w \in U \Rightarrow u + w \in U$.
  \item[(S3)] \emph{Closed under scalar multiplication:} $a \in \mathbf{F},\;
    u \in U \Rightarrow au \in U$.
\end{enumerate}

\vspace{0.75em}
\hrule
\vspace{0.75em}

% ----------------------------------------------------------
% (a)
% ----------------------------------------------------------
\noindent
\textbf{(a)} Let $b \in \mathbf{F}$. Show that
\[
  U = \{(x_1, x_2, x_3, x_4) \in \mathbf{F}^4 : x_3 = 5x_4 + b\}
\]
is a subspace of $\mathbf{F}^4$ if and only if $b = 0$.

\begin{proof}
$(\Leftarrow)$ Suppose $b = 0$, so $U = \{(x_1,x_2,x_3,x_4) : x_3 = 5x_4\}$.

\textit{(S1)} $(0,0,0,0)$ satisfies $0 = 5 \cdot 0$. \checkmark

\textit{(S2)} If $x_3 = 5x_4$ and $y_3 = 5y_4$, then $x_3 + y_3 = 5(x_4 + y_4)$. \checkmark

\textit{(S3)} If $x_3 = 5x_4$ and $a \in \mathbf{F}$, then $ax_3 = 5(ax_4)$. \checkmark

Hence $U$ is a subspace when $b = 0$.

$(\Rightarrow)$ Suppose $U$ is a subspace. Then $\mathbf{0} \in U$, so the
zero vector satisfies $0 = 5 \cdot 0 + b$, giving $b = 0$.
\end{proof}

\vspace{0.75em}
\hrule
\vspace{0.75em}

% ----------------------------------------------------------
% (b)
% ----------------------------------------------------------
\noindent
\textbf{(b)} Show that the set of continuous real-valued functions on $[0,1]$
is a subspace of $\mathbf{R}^{[0,1]}$.

\begin{proof}
Let $U = \mathcal{C}([0,1]) = \{f : [0,1] \to \mathbf{R} : f \text{ continuous}\}$.

\textit{(S1)} The zero function $f \equiv 0$ is continuous. \checkmark

\textit{(S2)} The sum of two continuous functions is continuous (a standard
result from analysis). \checkmark

\textit{(S3)} A scalar multiple of a continuous function is continuous. \checkmark

Hence $\mathcal{C}([0,1])$ is a subspace of $\mathbf{R}^{[0,1]}$.
\end{proof}

\vspace{0.75em}
\hrule
\vspace{0.75em}

% ----------------------------------------------------------
% (c)
% ----------------------------------------------------------
\noindent
\textbf{(c)} Show that the set of differentiable real-valued functions on
$\mathbf{R}$ is a subspace of $\mathbf{R}^{\mathbf{R}}$.

\begin{proof}
Let $U = \{f : \mathbf{R} \to \mathbf{R} : f \text{ differentiable}\}$.

\textit{(S1)} $f \equiv 0$ is differentiable, with $f' \equiv 0$. \checkmark

\textit{(S2)} $(f + g)' = f' + g'$, so the sum of differentiable functions
is differentiable. \checkmark

\textit{(S3)} $(af)' = af'$, so a scalar multiple of a differentiable function
is differentiable. \checkmark

Hence $U$ is a subspace of $\mathbf{R}^{\mathbf{R}}$.
\end{proof}

\vspace{0.75em}
\hrule
\vspace{0.75em}

% ----------------------------------------------------------
% (d)
% ----------------------------------------------------------
\noindent
\textbf{(d)} Let $b \in \mathbf{R}$. Show that
\[
  U = \{f \in \mathbf{R}^{(0,3)} : f \text{ differentiable and } f'(2) = b\}
\]
is a subspace of $\mathbf{R}^{(0,3)}$ if and only if $b = 0$.

\begin{proof}
$(\Leftarrow)$ Suppose $b = 0$, so $U = \{f \in \mathbf{R}^{(0,3)} : f
\text{ differentiable}, f'(2) = 0\}$.

\textit{(S1)} The zero function satisfies $0'(2) = 0$. \checkmark

\textit{(S2)} If $f'(2) = 0$ and $g'(2) = 0$, then $(f+g)'(2) = 0 + 0 = 0$. \checkmark

\textit{(S3)} If $f'(2) = 0$ and $a \in \mathbf{R}$, then $(af)'(2) = af'(2) = 0$. \checkmark

Hence $U$ is a subspace when $b = 0$.

$(\Rightarrow)$ Suppose $U$ is a subspace. Then $\mathbf{0} \in U$, so
$0'(2) = b$, giving $b = 0$.
\end{proof}

\vspace{0.75em}
\hrule
\vspace{0.75em}

% ----------------------------------------------------------
% (e)
% ----------------------------------------------------------
\noindent
\textbf{(e)} Show that the set of sequences of complex numbers with limit $0$
is a subspace of $\mathbf{C}^\infty$.

\begin{proof}
Let $U = \{(z_1, z_2, \ldots) \in \mathbf{C}^\infty : \lim_{n \to \infty}
z_n = 0\}$.

\textit{(S1)} The zero sequence $(0, 0, \ldots)$ has limit $0$. \checkmark

\textit{(S2)} If $\lim z_n = 0$ and $\lim w_n = 0$, then $\lim(z_n + w_n) =
0 + 0 = 0$ (limit of sum is sum of limits). \checkmark

\textit{(S3)} If $\lim z_n = 0$ and $a \in \mathbf{C}$, then $\lim(az_n) =
a \cdot 0 = 0$ (scalar multiple passes through limit). \checkmark

Hence $U$ is a subspace of $\mathbf{C}^\infty$.
\end{proof}

\vspace{2cm}
\noindent
\hyperlink{ws-LA-AXL-C01-1C-X1.35}{\textbf{← Back to worksheet}}
