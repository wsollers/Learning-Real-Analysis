% =========================================================
% Worksheet: Axler — Vector Spaces
% File: linear-algebra/worksheets/axler.tex
% =========================================================

\subsection{Axler}

\noindent\textbf{Source.} Sheldon Axler, \textit{Linear Algebra Done Right}.

\vspace{0.75em}
\begin{center}
\begin{tabular}{|p{5.0cm}|p{9.0cm}|}
\hline
\textbf{Problem ID} & \textbf{Exercise (descriptor)} \\
\hline

% -------------------------
% Chapter 1, Section B — Vector Spaces
% -------------------------

\phantomsection
\hypertarget{ws-LA-AXL-C01-SB-E01}{}
\hyperlink{proof-LA-AXL-C01-SB-E01-3e}{\texttt{LA-AXL-C01-SB-E01-3e}}
&
Axler, Chapter~1, Section~B, Q.~1 — Verify that $\mathbb{F}^S$ (functions $S\to\mathbb{F}$) is a vector space with pointwise operations; identify $\mathbf{0}$ and $-f$.
\\ \hline

% -------------------------
% Chapter 1, Section C — Subspaces
% -------------------------

\phantomsection
\hypertarget{ws-LA-AXL-C01-1C-X1.35}{}
\hyperlink{proof-LA-AXL-C01-1C-X1.35}{\texttt{LA-AXL-C01-1C-X1.35}}
&
Ch.~1, \S1C, Example~1.35 --- Five subspace verifications:
(a)~$x_3 = 5x_4 + b$ iff $b=0$;
(b)~continuous functions on $[0,1]$;
(c)~differentiable functions on $\mathbf{R}$;
(d)~$f'(2)=b$ iff $b=0$;
(e)~sequences with limit~$0$.
\\ \hline

\phantomsection
\hypertarget{ws-LA-AXL-C01-1C-E10}{}
\hyperlink{proof-LA-AXL-C01-1C-E10}{\texttt{LA-AXL-C01-1C-E10}}
&
Ch.~1, \S1C, Ex.~10 --- Prove that $V_1 \cap V_2$ is a subspace of $V$.
\\ \hline

\phantomsection
\hypertarget{ws-LA-AXL-C01-1C-E11}{}
\hyperlink{proof-LA-AXL-C01-1C-E11}{\texttt{LA-AXL-C01-1C-E11}}
&
Ch.~1, \S1C, Ex.~11 --- Prove that the intersection of every collection of subspaces of $V$ is a subspace of $V$.
\\ \hline

\phantomsection
\hypertarget{ws-LA-AXL-C01-1C-E12}{}
\hyperlink{proof-LA-AXL-C01-1C-E12}{\texttt{LA-AXL-C01-1C-E12}}
&
Ch.~1, \S1C, Ex.~12 --- Prove that the union of two subspaces of $V$ is a subspace iff one is contained in the other.
\\ \hline

\phantomsection
\hypertarget{ws-LA-AXL-C01-1C-E24}{}
\hyperlink{proof-LA-AXL-C01-1C-E24}{\texttt{LA-AXL-C01-1C-E24}}
&
Ch.~1, \S1C, Ex.~24 --- Show that $\mathbf{R}^{\mathbf{R}} = V_e \oplus V_o$ (even/odd decomposition).
\\ \hline

\phantomsection
\hypertarget{ws-LA-AXL-C01-1C-E23}{}
\hyperlink{proof-LA-AXL-C01-1C-E23}{\texttt{LA-AXL-C01-1C-E23}}
&
Ch.~1, \S1C, Ex.~23 --- Prove or give a counterexample: $V = V_1 \oplus U$ and $V = V_2 \oplus U$ implies $V_1 = V_2$.
\\ \hline

\phantomsection
\hypertarget{ws-LA-AXL-C01-1C-E19}{}
\hyperlink{proof-LA-AXL-C01-1C-E19}{\texttt{LA-AXL-C01-1C-E19}}
&
Ch.~1, \S1C, Ex.~19 --- Prove or give a counterexample: $V_1 + U = V_2 + U$ implies $V_1 = V_2$.
\\ \hline

\end{tabular}
\end{center}
