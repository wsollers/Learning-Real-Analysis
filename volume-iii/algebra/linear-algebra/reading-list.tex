% linear-algebra-reading-list.tex

% =========================================================
% LINEAR ALGEBRA — VOLUME I
% Structural, Axler-Driven
% =========================================================

\noindent
This syllabus is designed for deep structural mastery of finite-dimensional
linear algebra.

Primary driver text:
\textit{Sheldon Axler, Linear Algebra Done Right} (3rd ed.).

Supplementary reinforcement:
\begin{itemize}
  \item \textbf{A. L. Gorodentsev}, \textit{Linear Algebra and Geometry}
  \item \textbf{Joseph Gallian}, \textit{Contemporary Abstract Algebra}
\end{itemize}

The focus is linear maps, invariant structure, and operator theory —
not computational matrix mechanics.

Each topic is allocated approximately three days of focused study.

% ---------------------------------------------------------
\section*{Topic 1: Vector Spaces and Subspaces}
\addcontentsline{toc}{section}{Topic 1: Vector Spaces and Subspaces}

\textbf{Core ideas:}
vector space axioms, subspaces, span.

\begin{itemize}
  \item \textbf{Axler} — Ch.~1
  \item \textbf{Gorodentsev} — Ch.~1 (Vector spaces and linear structure)
  \item \textbf{Gallian} — Ch.~6 (Vector spaces over fields)
\end{itemize}

\textbf{Goal:}
Understand vector spaces abstractly, independent of coordinates.

% ---------------------------------------------------------
\section*{Topic 2: Linear Combinations and Linear Independence}
\addcontentsline{toc}{section}{Topic 2: Linear Independence}

\textbf{Core ideas:}
span, linear independence, minimal generating sets.

\begin{itemize}
  \item \textbf{Axler} — Ch.~2
  \item \textbf{Gorodentsev} — Ch.~1 §§1.2--1.4
  \item \textbf{Gallian} — Ch.~6 (Independence and generating sets)
\end{itemize}

\textbf{Goal:}
Master dependence arguments and proof techniques.

% ---------------------------------------------------------
\section*{Topic 3: Bases and Dimension}
\addcontentsline{toc}{section}{Topic 3: Bases and Dimension}

\textbf{Core ideas:}
existence of bases, dimension invariance.

\begin{itemize}
  \item \textbf{Axler} — Ch.~2
  \item \textbf{Gorodentsev} — Ch.~1 §1.5 (Dimension theory)
  \item \textbf{Gallian} — Ch.~6 (Basis and dimension parallels)
\end{itemize}

\textbf{Goal:}
Internalize dimension as a structural invariant.

% ---------------------------------------------------------
\section*{Topic 4: Linear Maps}
\addcontentsline{toc}{section}{Topic 4: Linear Maps}

\textbf{Core ideas:}
definition of linear map, composition, isomorphisms.

\begin{itemize}
  \item \textbf{Axler} — Ch.~3
  \item \textbf{Gorodentsev} — Ch.~2 §§2.1--2.2 (Linear transformations)
  \item \textbf{Gallian} — Ch.~7 (Homomorphisms; structural parallels)
\end{itemize}

\textbf{Goal:}
Shift viewpoint from vectors to transformations.

% ---------------------------------------------------------
\section*{Topic 5: Null Space, Range, and Rank-Nullity}
\addcontentsline{toc}{section}{Topic 5: Rank-Nullity}

\textbf{Core ideas:}
kernel, image, rank-nullity theorem.

\begin{itemize}
  \item \textbf{Axler} — Ch.~3 §3.C
  \item \textbf{Gorodentsev} — Ch.~2 §2.3 (Exact sequences viewpoint)
  \item \textbf{Gallian} — Ch.~7 (Kernel and image in algebraic context)
\end{itemize}

\textbf{Goal:}
Understand rank-nullity as structural decomposition.

% ---------------------------------------------------------
\section*{Topic 6: Eigenvalues and Eigenvectors}
\addcontentsline{toc}{section}{Topic 6: Eigenvalues and Eigenvectors}

\textbf{Core ideas:}
eigenvalues, eigenspaces, invariant subspaces.

\begin{itemize}
  \item \textbf{Axler} — Ch.~5
  \item \textbf{Gorodentsev} — Ch.~3 §§3.1--3.2 (Eigenstructure)
  \item \textbf{Gallian} — Ch.~6 (Polynomial structure reinforcement)
\end{itemize}

\textbf{Goal:}
See eigenvalues as structural decompositions of operators.

% ---------------------------------------------------------
\section*{Topic 7: Diagonalization and Triangularization}
\addcontentsline{toc}{section}{Topic 7: Diagonalization}

\textbf{Core ideas:}
diagonalizable operators, invariant decompositions.

\begin{itemize}
  \item \textbf{Axler} — Ch.~5--6
  \item \textbf{Gorodentsev} — Ch.~3 §3.3
\end{itemize}

\textbf{Goal:}
Understand when and why operators decompose.

% ---------------------------------------------------------
\section*{Topic 8: Inner Product Spaces}
\addcontentsline{toc}{section}{Topic 8: Inner Product Spaces}

\textbf{Core ideas:}
inner products, orthogonality, Gram-Schmidt.

\begin{itemize}
  \item \textbf{Axler} — Ch.~6
  \item \textbf{Gorodentsev} — Ch.~4 §§4.1--4.2
\end{itemize}

\textbf{Goal:}
Integrate geometric structure with algebraic structure.

% ---------------------------------------------------------
\section*{Topic 9: Spectral Theorem (Finite-Dimensional)}
\addcontentsline{toc}{section}{Topic 9: Spectral Theorem}

\textbf{Core ideas:}
normal operators, orthogonal diagonalization.

\begin{itemize}
  \item \textbf{Axler} — Ch.~7
  \item \textbf{Gorodentsev} — Ch.~4 §4.3
\end{itemize}

\textbf{Goal:}
See the spectral theorem as the culmination of finite-dimensional operator theory.

\bigskip
\noindent
\textbf{Outcome:}
Completion of this syllabus provides structural mastery of
finite-dimensional linear algebra centered on linear maps,
invariant subspaces, eigenstructure, and operator theory.
