% ============================================================
%  Algebraic Structures — Complete Reference
%  Foundations + all 10 Structure Blueprints
%
%  Required in main.tex preamble:
%    \usepackage{tcolorbox}
%    \tcbuselibrary{skins}
%    \usepackage{amsthm,amsmath,amssymb,xcolor}
%    \newtheorem{definition}{Definition}[chapter]
%    \newtheorem{remark}{Remark}[chapter]
%    and all \definecolor{cSet}...{cVS} + nodetext/sigtext/lbltext
%
%  Usage:
%    % ============================================================
%  Blueprint shared macros — % ============================================================
%  Blueprint shared macros — % ============================================================
%  Blueprint shared macros — \input{blueprint-macros} once
%  after \begin{document} before any blueprint \input calls.
%
%  Required in main.tex preamble:
%    \usepackage{tcolorbox}
%    \tcbuselibrary{skins}
%    \usepackage{amsthm,amsmath,amssymb,xcolor}
%    all \definecolor{cSet}...{cVS} definitions
%    \newtheorem{definition}{Definition}[chapter]  (or similar)
%    \newtheorem{remark}{Remark}[chapter]
% ============================================================

% ── Section boxes ─────────────────────────────────────────────
% NEW: strong frame in structure colour
\newtcolorbox{bpnew}[2]{
  enhanced,
  colback   = #1!14!white,
  colframe  = #1!80!black,
  coltitle  = white,
  fonttitle = \bfseries\scshape\small,
  title     = #2,
  boxrule   = 1.0pt, arc=2pt,
  left=7pt, right=7pt, top=4pt, bottom=5pt,
  toptitle=3pt, bottomtitle=3pt,
  before skip=0pt, after skip=7pt,
}

% INHERITED: faded frame in ancestor colour
\newtcolorbox{bpinh}[2]{
  enhanced,
  colback   = #1!7!white,
  colframe  = #1!38!white,
  coltitle  = #1!70!black,
  fonttitle = \bfseries\scshape\small,
  title     = #2,
  boxrule   = 0.6pt, arc=2pt,
  left=7pt, right=7pt, top=4pt, bottom=5pt,
  toptitle=3pt, bottomtitle=3pt,
  before skip=0pt, after skip=7pt,
}

% ── Entry line ────────────────────────────────────────────────
% \bpitem{symbol}{math}{gloss}   (gloss may be empty)
\newcommand{\bpitem}[3]{%
  \noindent
  {\small\bfseries #1}\hspace{6pt}%
  $\displaystyle #2$%
  \ifx\relax#3\relax\else
    \hfill{\small\rmfamily\itshape\color{lbltext!75}\,[#3]}%
  \fi
  \par\vspace{3pt}%
}

% ── Page header ───────────────────────────────────────────────
% \bpheader{colour}{Name}{signature}
\newcommand{\bpheader}[3]{%
  \noindent\vbox{\offinterlineskip
    \hbox{\colorbox{#1!85!black}{%
      \parbox[c][2.4em]{\dimexpr\linewidth-2\fboxsep\relax}{%
        \color{nodetext}%
        \hspace{8pt}{\large\bfseries #2}%
        \enspace{\normalsize\ttfamily\color{sigtext} #3}%
        \hfill{\footnotesize\scshape\bfseries Structure Blueprint\hspace{8pt}}%
      }%
    }}%
  }\par\vspace{8pt}%
}

% ── Footer ────────────────────────────────────────────────────
% \bpfooter{left}{right}
\newcommand{\bpfooter}[2]{%
  \vfill
  \noindent\hrule\vspace{4pt}%
  \noindent
  \begin{minipage}[t]{0.49\linewidth}\scriptsize #1\end{minipage}%
  \hfill
  \begin{minipage}[t]{0.49\linewidth}\scriptsize #2\end{minipage}%
} once
%  after \begin{document} before any blueprint \input calls.
%
%  Required in main.tex preamble:
%    \usepackage{tcolorbox}
%    \tcbuselibrary{skins}
%    \usepackage{amsthm,amsmath,amssymb,xcolor}
%    all \definecolor{cSet}...{cVS} definitions
%    \newtheorem{definition}{Definition}[chapter]  (or similar)
%    \newtheorem{remark}{Remark}[chapter]
% ============================================================

% ── Section boxes ─────────────────────────────────────────────
% NEW: strong frame in structure colour
\newtcolorbox{bpnew}[2]{
  enhanced,
  colback   = #1!14!white,
  colframe  = #1!80!black,
  coltitle  = white,
  fonttitle = \bfseries\scshape\small,
  title     = #2,
  boxrule   = 1.0pt, arc=2pt,
  left=7pt, right=7pt, top=4pt, bottom=5pt,
  toptitle=3pt, bottomtitle=3pt,
  before skip=0pt, after skip=7pt,
}

% INHERITED: faded frame in ancestor colour
\newtcolorbox{bpinh}[2]{
  enhanced,
  colback   = #1!7!white,
  colframe  = #1!38!white,
  coltitle  = #1!70!black,
  fonttitle = \bfseries\scshape\small,
  title     = #2,
  boxrule   = 0.6pt, arc=2pt,
  left=7pt, right=7pt, top=4pt, bottom=5pt,
  toptitle=3pt, bottomtitle=3pt,
  before skip=0pt, after skip=7pt,
}

% ── Entry line ────────────────────────────────────────────────
% \bpitem{symbol}{math}{gloss}   (gloss may be empty)
\newcommand{\bpitem}[3]{%
  \noindent
  {\small\bfseries #1}\hspace{6pt}%
  $\displaystyle #2$%
  \ifx\relax#3\relax\else
    \hfill{\small\rmfamily\itshape\color{lbltext!75}\,[#3]}%
  \fi
  \par\vspace{3pt}%
}

% ── Page header ───────────────────────────────────────────────
% \bpheader{colour}{Name}{signature}
\newcommand{\bpheader}[3]{%
  \noindent\vbox{\offinterlineskip
    \hbox{\colorbox{#1!85!black}{%
      \parbox[c][2.4em]{\dimexpr\linewidth-2\fboxsep\relax}{%
        \color{nodetext}%
        \hspace{8pt}{\large\bfseries #2}%
        \enspace{\normalsize\ttfamily\color{sigtext} #3}%
        \hfill{\footnotesize\scshape\bfseries Structure Blueprint\hspace{8pt}}%
      }%
    }}%
  }\par\vspace{8pt}%
}

% ── Footer ────────────────────────────────────────────────────
% \bpfooter{left}{right}
\newcommand{\bpfooter}[2]{%
  \vfill
  \noindent\hrule\vspace{4pt}%
  \noindent
  \begin{minipage}[t]{0.49\linewidth}\scriptsize #1\end{minipage}%
  \hfill
  \begin{minipage}[t]{0.49\linewidth}\scriptsize #2\end{minipage}%
} once
%  after \begin{document} before any blueprint \input calls.
%
%  Required in main.tex preamble:
%    \usepackage{tcolorbox}
%    \tcbuselibrary{skins}
%    \usepackage{amsthm,amsmath,amssymb,xcolor}
%    all \definecolor{cSet}...{cVS} definitions
%    \newtheorem{definition}{Definition}[chapter]  (or similar)
%    \newtheorem{remark}{Remark}[chapter]
% ============================================================

% ── Section boxes ─────────────────────────────────────────────
% NEW: strong frame in structure colour
\newtcolorbox{bpnew}[2]{
  enhanced,
  colback   = #1!14!white,
  colframe  = #1!80!black,
  coltitle  = white,
  fonttitle = \bfseries\scshape\small,
  title     = #2,
  boxrule   = 1.0pt, arc=2pt,
  left=7pt, right=7pt, top=4pt, bottom=5pt,
  toptitle=3pt, bottomtitle=3pt,
  before skip=0pt, after skip=7pt,
}

% INHERITED: faded frame in ancestor colour
\newtcolorbox{bpinh}[2]{
  enhanced,
  colback   = #1!7!white,
  colframe  = #1!38!white,
  coltitle  = #1!70!black,
  fonttitle = \bfseries\scshape\small,
  title     = #2,
  boxrule   = 0.6pt, arc=2pt,
  left=7pt, right=7pt, top=4pt, bottom=5pt,
  toptitle=3pt, bottomtitle=3pt,
  before skip=0pt, after skip=7pt,
}

% ── Entry line ────────────────────────────────────────────────
% \bpitem{symbol}{math}{gloss}   (gloss may be empty)
\newcommand{\bpitem}[3]{%
  \noindent
  {\small\bfseries #1}\hspace{6pt}%
  $\displaystyle #2$%
  \ifx\relax#3\relax\else
    \hfill{\small\rmfamily\itshape\color{lbltext!75}\,[#3]}%
  \fi
  \par\vspace{3pt}%
}

% ── Page header ───────────────────────────────────────────────
% \bpheader{colour}{Name}{signature}
\newcommand{\bpheader}[3]{%
  \noindent\vbox{\offinterlineskip
    \hbox{\colorbox{#1!85!black}{%
      \parbox[c][2.4em]{\dimexpr\linewidth-2\fboxsep\relax}{%
        \color{nodetext}%
        \hspace{8pt}{\large\bfseries #2}%
        \enspace{\normalsize\ttfamily\color{sigtext} #3}%
        \hfill{\footnotesize\scshape\bfseries Structure Blueprint\hspace{8pt}}%
      }%
    }}%
  }\par\vspace{8pt}%
}

% ── Footer ────────────────────────────────────────────────────
% \bpfooter{left}{right}
\newcommand{\bpfooter}[2]{%
  \vfill
  \noindent\hrule\vspace{4pt}%
  \noindent
  \begin{minipage}[t]{0.49\linewidth}\scriptsize #1\end{minipage}%
  \hfill
  \begin{minipage}[t]{0.49\linewidth}\scriptsize #2\end{minipage}%
}          % shared box/header macros
%    \input{algebraic-structures-complete}
% ============================================================

% =============================================================
%  FOUNDATIONS
%  Sets · Functions · Model Theory · Unary Operation
% =============================================================
\clearpage
\thispagestyle{plain}

% ── Page heading ─────────────────────────────────────────────
\noindent\vbox{\offinterlineskip
  \hbox{\colorbox{lbltext!85!black}{%
    \parbox[c][2.4em]{\dimexpr\linewidth-2\fboxsep\relax}{%
      \color{nodetext}%
      \hspace{8pt}{\large\bfseries Foundations}%
      \hfill{\footnotesize\scshape\bfseries
        Sets · Functions · Model Theory\hspace{8pt}}%
    }%
  }}%
}\par\vspace{8pt}

% =============================================================
%  §1  Model-Theoretic Foundations
% =============================================================
\subsubsection*{Model-Theoretic Foundations}

\noindent
Model theory provides a precise language for describing algebraic
structures: a \emph{signature} names the operations and constants a
structure carries, and a \emph{model} (or \emph{structure}) interprets
those names on a concrete set.  Every algebraic structure in these
blueprints is a model of some first-order signature.

\begin{definition}[First-Order Signature]
  A \textbf{first-order signature} $\mathcal{L}$ consists of:
  \begin{itemize}
    \item \textbf{Constant symbols} $c_1, c_2, \ldots$ — naming distinguished
          elements,
    \item \textbf{Function symbols} $f_1, f_2, \ldots$, each equipped with a
          specified \textbf{arity} $n \in \mathbb{N}$ — naming $n$-ary
          operations,
    \item \textbf{Relation symbols} $R_1, R_2, \ldots$ (optional), each with a
          specified arity — naming predicates.
  \end{itemize}
\end{definition}

\begin{definition}[$\mathcal{L}$-Structure]
  Given a signature $\mathcal{L}$, an \textbf{$\mathcal{L}$-structure}
  $\mathcal{A}$ consists of:
  \begin{itemize}
    \item A nonempty set $A$, called the \textbf{universe} (or
          \emph{carrier set}),
    \item For each constant symbol $c$: an element $c^{\mathcal{A}} \in A$,
    \item For each $n$-ary function symbol $f$: a function
          $f^{\mathcal{A}} : A^n \to A$,
    \item For each $n$-ary relation symbol $R$: a subset
          $R^{\mathcal{A}} \subseteq A^n$.
  \end{itemize}
  We write $\mathcal{A} = (A, \ldots)$, listing the universe and
  interpretations.
\end{definition}

\begin{remark}
  The signature is the \emph{type}; the structure is the \emph{instance}.
  A group, for example, has signature
  $\mathcal{L}_{\mathrm{grp}} = \{*, e, {}^{-1}\}$
  where $*$ has arity 2, ${}^{-1}$ has arity 1, and $e$ is a constant.
  Any group $(G, *, e, {}^{-1})$ is then an $\mathcal{L}_{\mathrm{grp}}$-structure
  satisfying the group axioms.  The axioms themselves are
  \emph{first-order sentences} in $\mathcal{L}_{\mathrm{grp}}$, and a
  \textbf{model} of those sentences is precisely a group.
  This framework unifies all algebraic structures: rings, fields, vector
  spaces, and so on are models of their respective signatures.
\end{remark}

\medskip

% =============================================================
%  §2  Functions
% =============================================================
\subsubsection*{Functions}

\noindent
A function is the primitive notion underlying every algebraic operation.
Before studying operations on sets, we record the full structure of
functions and their properties.

\begin{definition}[Function]
  Let $A$ and $B$ be sets. A \textbf{function} $f : A \to B$ is a rule
  such that
  \[
    (\forall\, a \in A)\;(\exists!\, b \in B)\;(f(a) = b).
  \]
  The components of $f$ are:
  \begin{itemize}
    \item \textbf{Domain}: $\operatorname{dom}(f) = A$,
    \item \textbf{Codomain}: $\operatorname{cod}(f) = B$,
    \item \textbf{Graph}: $\Gamma_f = \{(a,\, f(a)) : a \in A\}
          \subseteq A \times B$.
  \end{itemize}
  Two functions are equal iff they have the same domain, codomain, and graph.
\end{definition}

\begin{definition}[Image and Preimage]
  Let $f : A \to B$, $S \subseteq A$, $T \subseteq B$.
  \begin{itemize}
    \item \textbf{Image of $f$}:
      $\displaystyle\operatorname{Im}(f) = f(A) = \{\,f(a) : a \in A\,\}
       \subseteq B$.
    \item \textbf{Image of a subset} $S$:
      $\displaystyle f(S) = \{\,f(s) : s \in S\,\} \subseteq B$.
    \item \textbf{Preimage of a subset} $T$:
      $\displaystyle f^{-1}(T) = \{\,a \in A : f(a) \in T\,\} \subseteq A$.
  \end{itemize}
  Note: $f^{-1}(T)$ is defined for every function $f$; it does \emph{not}
  require $f$ to be invertible.
\end{definition}

\begin{definition}[Injective, Surjective, Bijective]
  Let $f : A \to B$.
  \begin{itemize}
    \item \textbf{Injective} (one-to-one):
      \[
        (\forall\, a_1, a_2 \in A)\;
        \bigl(f(a_1) = f(a_2) \;\Rightarrow\; a_1 = a_2\bigr).
      \]
      Equivalently, distinct inputs give distinct outputs.
    \item \textbf{Surjective} (onto):
      \[
        (\forall\, b \in B)\;(\exists\, a \in A)\;(f(a) = b).
      \]
      Equivalently, $\operatorname{Im}(f) = B$.
    \item \textbf{Bijective}: $f$ is both injective and surjective.
      Equivalently, $f$ establishes a one-to-one correspondence between
      $A$ and $B$.
  \end{itemize}
\end{definition}

\begin{definition}[Inverse Function]
  If $f : A \to B$ is bijective, then its \textbf{inverse}
  $f^{-1} : B \to A$ is the unique function satisfying
  \[
    (\forall\, a \in A)\quad f^{-1}(f(a)) = a,
    \qquad
    (\forall\, b \in B)\quad f(f^{-1}(b)) = b.
  \]
  That is, $f^{-1} \circ f = \mathrm{id}_A$ and
  $f \circ f^{-1} = \mathrm{id}_B$.
  A function has an inverse if and only if it is bijective.
\end{definition}

\begin{remark}
  Injectivity, surjectivity, and bijectivity measure how faithfully
  $f$ maps $A$ into $B$.  Injective: no collisions.  Surjective: full
  coverage.  Bijective: perfect pairing.  In algebra, the morphisms that
  preserve structure and are bijective (isomorphisms) are the ones that
  identify structures as \emph{the same up to relabelling}.
  Note also the distinction between $\operatorname{Im}(f) \subseteq B$
  (a subset) and $f^{-1}(T) \subseteq A$ (a preimage): the former goes
  forward, the latter pulls back, and neither requires the function to be
  invertible.
\end{remark}

\medskip

% =============================================================
%  §3  Unary Operation
% =============================================================
\subsubsection*{Unary (Singular) Operation}

\noindent
Before defining binary operations (arity 2), we fix the simpler case of
arity 1.  Unary operations appear throughout algebra: negation $a \mapsto -a$,
multiplicative inverse $a \mapsto a^{-1}$, complex conjugation
$z \mapsto \bar{z}$, and set complementation.

\begin{definition}[Unary Operation]
  Let $S$ be a set. A \textbf{unary operation} on $S$ is a function
  \[
    f : S \to S.
  \]
  Components:
  \begin{itemize}
    \item \textbf{Underlying set}: $S$ (domain and codomain coincide),
    \item \textbf{Function}: $f$,
    \item \textbf{Closure}: $(\forall\, x \in S)\;(\exists!\, y \in S)\;(y = f(x))$.
  \end{itemize}
  Model-theoretic signature:
  \[
    \mathcal{L}_{\mathrm{unary}} = \{\,f\,\},
    \qquad f \text{ a function symbol of arity } 1.
  \]
  The structure $(S, f)$ is then an $\mathcal{L}_{\mathrm{unary}}$-structure
  with no axioms beyond closure.
\end{definition}

\begin{remark}
  A unary operation is to a binary operation what a button is to a lever:
  it takes one input and produces one output, always within the same set.
  In group theory the inverse map ${}^{-1} : G \to G$ is a unary operation
  built into the signature; the axiom $a \cdot a^{-1} = e$ then constrains
  it.  Similarly, in a ring, negation $- : A \to A$ is unary.  Naming
  unary operations explicitly in the signature (rather than deriving them)
  is the model-theoretic convention that keeps axioms first-order.
\end{remark}

\clearpage


% =============================================================
%  BLUEPRINT 1 — Set
% =============================================================
\clearpage\thispagestyle{plain}
\bpheader{cSet}{Set}{S}

\begin{bpnew}{cSet}{New at Set}
  \bpitem{$\exists$}{a,\,b,\,c \in S}{\relax}
\end{bpnew}

\begin{definition}[Set]
  A \textbf{set} $S$ is a well-defined collection of distinct objects called
  \textbf{elements}.  We write $a \in S$ to denote that $a$ is an element of
  $S$, and $a \notin S$ otherwise.  Two sets are equal if and only if they
  have the same elements:
  \[
    S = T \iff \forall\,x:\; (x \in S \iff x \in T).
  \]
\end{definition}

\begin{remark}
  A set carries no additional structure — no operations, no ordering, no notion
  of distance.  It is the blank slate from which all algebraic structures are
  built by adding operations and axioms.  When we define a Magma $(A, *)$,
  the $A$ is just a set; the operation $*$ is layered on top.
\end{remark}

\bpfooter{%
  \textbf{Extended by:} Binary Operation, Magma \textit{(add a closed map)}
}{%
  \textbf{Note:} No operations or axioms at this level.
  The carrier set is the only ingredient.
}


% =============================================================
%  BLUEPRINT 2 — Binary Operation
% =============================================================
\clearpage\thispagestyle{plain}
\bpheader{cBinop}{Binary Operation}{$* : A \times A \to A$}

\begin{bpnew}{cBinop}{New at Binary Operation}
  \bpitem{$\mathrm{f}$}{* : A \times A \to A}{the binary map}
  \bpitem{$\forall$}{\forall\,a,b \in A:\; a * b \in A}{closure}
\end{bpnew}

\begin{bpinh}{cSet}{Inherited from Set}
  \bpitem{$\exists$}{a,\,b,\,c \in A}{elements of the carrier set}
\end{bpinh}

\begin{definition}[Binary Operation]
  Let $A$ be a set.  A \textbf{binary operation} on $A$ is a function
  \[
    * : A \times A \to A
  \]
  that assigns to each ordered pair $(a, b) \in A \times A$ a unique element
  $a * b \in A$.  The defining property is \textbf{closure}:
  $\forall\,a, b \in A,\; a * b \in A$, i.e.\ the operation never escapes the
  set.

  \medskip\noindent
  Model-theoretic signature:
  \[
    \mathcal{L}_{\mathrm{bin}} = \{\,*\,\},
    \qquad * \text{ a function symbol of arity } 2.
  \]
\end{definition}

\begin{remark}
  A binary operation is a \emph{concept}, not a structure in its own right —
  it is the interface that all group-like structures implement.  Closure is the
  only requirement: there are no demands on how $*$ behaves (no associativity,
  no identity, no inverses).  In OOP terms, this is a pure abstract interface;
  Magma is the first concrete class that implements it.
\end{remark}

\bpfooter{%
  \textbf{IS-A:} Set \textit{(carrier $A$)}\\
  \textbf{HAS-A:} implemented by Magma and all descendants
}{%
  \textbf{Note:} Addition $+$ on $\mathbb{Z}$, multiplication $\cdot$ on
  $\mathbb{R}$, and function composition are all binary operations.
}


% =============================================================
%  BLUEPRINT 3 — Magma
% =============================================================
\clearpage\thispagestyle{plain}
\bpheader{cMagma}{Magma}{(A,\,*)}

\begin{bpnew}{cMagma}{New at Magma}
  \bpitem{$\mathrm{f}$}{* : A \times A \to A}{closed binary operation}
  \bpitem{$\forall$}{\forall\,a,b \in A:\; a * b \in A}{closure}
\end{bpnew}

\begin{bpinh}{cSet}{Inherited from Set}
  \bpitem{$\exists$}{a,\,b,\,c \in A}{elements of the carrier set}
\end{bpinh}

\begin{definition}[Magma]
  A \textbf{magma} is a pair $(A, *)$ where $A$ is a nonempty set and
  $* : A \times A \to A$ is a binary operation satisfying:
  \[
    \forall\,a, b \in A:\; a * b \in A. \tag{Closure}
  \]
  No further axioms are imposed on $*$.

  \medskip\noindent
  Model-theoretic signature:
  \[
    \mathcal{L}_{\mathrm{magma}} = \{\,*\,\},
    \qquad * \text{ a function symbol of arity } 2.
  \]
\end{definition}

\begin{remark}
  A magma is the weakest non-trivial algebraic structure — the bare minimum
  for calling something an algebra.  The single axiom, closure, merely says
  that applying $*$ to elements of $A$ stays inside $A$.  This rules out
  partial operations but imposes no coherence (no associativity, no identity,
  no inverses).  Most naturally occurring operations satisfy much more, which
  is why the Magma is usually the invisible foundation rather than the object
  of study itself.
\end{remark}

\bpfooter{%
  \textbf{IS-A:} Set \quad \textbf{HAS-A:} Binary Operation\\
  \textbf{Extended by:} Semigroup \textit{(adds associativity)}
}{%
  \textbf{Example:} $(\mathbb{N},\,\text{rock-paper-scissors})$ is a magma —
  closed, but not associative.
}


% =============================================================
%  BLUEPRINT 4 — Semigroup
% =============================================================
\clearpage\thispagestyle{plain}
\bpheader{cSemi}{Semigroup}{(A,\,*)}

\begin{bpnew}{cSemi}{New at Semigroup}
  \bpitem{$\forall$}{\forall\,a,b,c \in A:\; (a*b)*c = a*(b*c)}{associativity}
\end{bpnew}

\begin{bpinh}{cMagma}{Inherited from Magma}
  \bpitem{$\mathrm{f}$}{* : A \times A \to A}{closed binary operation}
  \bpitem{$\forall$}{\forall\,a,b \in A:\; a * b \in A}{closure}
\end{bpinh}

\begin{bpinh}{cSet}{Inherited from Set}
  \bpitem{$\exists$}{a,\,b,\,c \in A}{elements of the carrier set}
\end{bpinh}

\begin{definition}[Semigroup]
  A \textbf{semigroup} is a pair $(A, *)$ where $A$ is a nonempty set and
  $* : A \times A \to A$ is a binary operation satisfying:
  \begin{align*}
    \forall\,a,b \in A:&\; a * b \in A \tag{Closure}\\
    \forall\,a,b,c \in A:&\; (a * b) * c = a * (b * c) \tag{Associativity}
  \end{align*}

  \noindent
  Model-theoretic signature:
  \[
    \mathcal{L}_{\mathrm{semi}} = \{\,*\,\},
    \qquad * \text{ a function symbol of arity } 2.
  \]
\end{definition}

\begin{remark}
  Associativity is what allows us to write $a * b * c$ without parentheses —
  the result is the same regardless of how we bracket a chain of operations.
  This seemingly minor property has profound consequences: it enables induction
  arguments over products, underpins the theory of free monoids (strings), and
  is what makes the integers under addition well-behaved.  The semigroup is the
  first structure where the operation itself has meaningful algebraic content.
\end{remark}

\bpfooter{%
  \textbf{IS-A:} Magma \quad \textbf{HAS-A:} Binary Operation\\
  \textbf{Extended by:} Monoid \textit{(adds identity element)}
}{%
  \textbf{Examples:} $(\mathbb{Z}^+, +)$, $(\mathbb{N}, \cdot)$,
  non-empty strings under concatenation.\\
  \textbf{Non-example:} $(\mathbb{Z}, -)$ — subtraction is not associative.
}


% =============================================================
%  BLUEPRINT 5 — Monoid
% =============================================================
\clearpage\thispagestyle{plain}
\bpheader{cMonoid}{Monoid}{(A,\,*,\,e)}

\begin{bpnew}{cMonoid}{New at Monoid}
  \bpitem{$\exists$}{e \in A}{identity element}
  \bpitem{$\forall$}{\forall\,a \in A:\; a * e = e * a = a}{identity law}
\end{bpnew}

\begin{bpinh}{cSemi}{Inherited from Semigroup}
  \bpitem{$\forall$}{\forall\,a,b,c \in A:\; (a*b)*c = a*(b*c)}{associativity}
\end{bpinh}

\begin{bpinh}{cMagma}{Inherited from Magma}
  \bpitem{$\mathrm{f}$}{* : A \times A \to A}{closed binary operation}
  \bpitem{$\forall$}{\forall\,a,b \in A:\; a * b \in A}{closure}
\end{bpinh}

\begin{definition}[Monoid]
  A \textbf{monoid} is a triple $(A, *, e)$ where $A$ is a nonempty set,
  $* : A \times A \to A$ is a binary operation, and $e \in A$, satisfying:
  \begin{align*}
    \forall\,a,b \in A:&\; a * b \in A \tag{Closure}\\
    \forall\,a,b,c \in A:&\; (a*b)*c = a*(b*c) \tag{Associativity}\\
    \forall\,a \in A:&\; a * e = e * a = a \tag{Identity}
  \end{align*}
  The element $e$ is called the \textbf{identity} (or \textbf{neutral element})
  of $*$.  It is unique: if $e'$ also satisfies the identity law, then
  $e = e * e' = e'$.

  \medskip\noindent
  Model-theoretic signature:
  \[
    \mathcal{L}_{\mathrm{mon}} = \{\,*,\, e\,\},
    \quad * \text{ arity } 2,\quad e \text{ a constant symbol.}
  \]
\end{definition}

\begin{remark}
  The identity element gives the monoid a ``do nothing'' operation — a baseline
  from which all elements are measured.  This is what makes monoids so prevalent
  in computing: the empty string is the identity under concatenation, $0$ is the
  identity under addition, $1$ under multiplication, and the empty list under
  append.  Monoids are the algebraic abstraction of \emph{combining things with
  a neutral default}.
\end{remark}

\bpfooter{%
  \textbf{IS-A:} Semigroup \quad \textbf{HAS-A:} Binary Operation\\
  \textbf{Extended by:} Group \textit{(adds inverses)};\ Ring
  \textit{(HAS-A Monoid under $\cdot$)}
}{%
  \textbf{Examples:} $(\mathbb{N}, +, 0)$, $(\mathbb{N}, \cdot, 1)$,
  $(\Sigma^*, \mathbin{\|}, \varepsilon)$ strings under concatenation.\\
  \textbf{Identity is unique}: if $a*e'=e'*a=a$ for all $a$, then $e'=e$.
}


% =============================================================
%  BLUEPRINT 6 — Group
% =============================================================
\clearpage\thispagestyle{plain}
\bpheader{cGroup}{Group}{$(A,*,e,a^{-1})$}

\begin{bpnew}{cGroup}{New at Group}
  \bpitem{$\exists$}{\forall\,a \in A,\;\exists\,a^{-1} \in A}{inverse element}
  \bpitem{$\forall$}{\forall\,a \in A:\; a * a^{-1} = a^{-1} * a = e}{inverse law}
\end{bpnew}

\begin{bpinh}{cMonoid}{Inherited from Monoid}
  \bpitem{$\exists$}{e \in A}{identity element}
  \bpitem{$\forall$}{\forall\,a \in A:\; a * e = e * a = a}{identity law}
\end{bpinh}

\begin{bpinh}{cSemi}{Inherited from Semigroup}
  \bpitem{$\forall$}{\forall\,a,b,c \in A:\; (a*b)*c = a*(b*c)}{associativity}
\end{bpinh}

\begin{bpinh}{cMagma}{Inherited from Magma}
  \bpitem{$\mathrm{f}$}{* : A \times A \to A}{closed binary operation}
  \bpitem{$\forall$}{\forall\,a,b \in A:\; a * b \in A}{closure}
\end{bpinh}

\begin{definition}[Group]
  A \textbf{group} is a quadruple $(A, *, e, {}^{-1})$ where $A$ is a nonempty
  set, $* : A \times A \to A$, $e \in A$, and ${}^{-1} : A \to A$, satisfying:
  \begin{align*}
    \forall\,a,b \in A:&\; a * b \in A \tag{Closure}\\
    \forall\,a,b,c \in A:&\; (a*b)*c = a*(b*c) \tag{Associativity}\\
    \forall\,a \in A:&\; a * e = e * a = a \tag{Identity}\\
    \forall\,a \in A:&\; a * a^{-1} = a^{-1} * a = e \tag{Inverses}
  \end{align*}
  The inverse $a^{-1}$ is unique for each $a$, and $(a^{-1})^{-1} = a$.

  \medskip\noindent
  Model-theoretic signature:
  \[
    \mathcal{L}_{\mathrm{grp}} = \{\,*,\, e,\, {}^{-1}\,\},
    \quad * \text{ arity } 2,\quad {}^{-1} \text{ arity } 1,\quad
    e \text{ a constant.}
  \]
\end{definition}

\begin{remark}
  Inverses are what turn a monoid into a group, and the conceptual leap is
  significant: every element now has an ``undo.''  This is the algebraic
  abstraction of \emph{reversible processes} — rotations of a shape,
  permutations of a set, invertible linear maps.  Group theory is arguably the
  most central structure in all of algebra, connecting symmetry in geometry,
  number theory (e.g.\ $(\mathbb{Z}/n\mathbb{Z})^*$), and the classification of
  elementary particles in physics.  Note that commutativity is \emph{not}
  required: $a * b$ need not equal $b * a$.
\end{remark}

\bpfooter{%
  \textbf{IS-A:} Monoid \quad \textbf{HAS-A:} Binary Operation\\
  \textbf{Extended by:} Abelian Group \textit{(adds commutativity)}
}{%
  \textbf{Examples:} $(\mathbb{Z}, +, 0)$, $(S_n, \circ, \mathrm{id})$
  permutations, $(GL_n(\mathbb{R}), \cdot, I)$ invertible matrices.\\
  \textbf{Non-example:} $(\mathbb{N}, +)$ — no additive inverses.
}


% =============================================================
%  BLUEPRINT 7 — Abelian Group
% =============================================================
\clearpage\thispagestyle{plain}
\bpheader{cAbelian}{Abelian Group}{$(A,+,0,-a)$}

\begin{bpnew}{cAbelian}{New at Abelian Group}
  \bpitem{$\forall$}{\forall\,a,b \in A:\; a + b = b + a}{commutativity}
  \bpitem{$\exists$}{\text{rename: } e \to 0,\quad a^{-1} \to {-a},\quad * \to +}{notational convention}
\end{bpnew}

\begin{bpinh}{cGroup}{Inherited from Group}
  \bpitem{$\exists$}{\forall\,a \in A,\;\exists\,{-a} \in A}{additive inverse}
  \bpitem{$\forall$}{\forall\,a \in A:\; a + ({-a}) = ({-a}) + a = 0}{inverse law}
  \bpitem{$\exists$}{0 \in A}{additive identity}
  \bpitem{$\forall$}{\forall\,a \in A:\; a + 0 = 0 + a = a}{identity law}
\end{bpinh}

\begin{bpinh}{cSemi}{Inherited from Semigroup}
  \bpitem{$\forall$}{\forall\,a,b,c \in A:\; (a+b)+c = a+(b+c)}{associativity}
\end{bpinh}

\begin{definition}[Abelian Group]
  A \textbf{group} $(A, +, 0, -)$ is called \textbf{abelian} (or
  \textbf{commutative}) if it additionally satisfies:
  \[
    \forall\,a, b \in A:\; a + b = b + a. \tag{Commutativity}
  \]
  By convention, the operation is written $+$, the identity as $0$, and the
  inverse of $a$ as $-a$.  An abelian group is thus a quintuple $(A, +, 0, -)$
  satisfying closure, associativity, identity, inverses, and commutativity.

  \medskip\noindent
  Model-theoretic signature:
  \[
    \mathcal{L}_{\mathrm{ab}} = \{\,+,\, 0,\, -\,\},
    \quad + \text{ arity } 2,\quad - \text{ arity } 1,\quad
    0 \text{ a constant.}
  \]
\end{definition}

\begin{remark}
  Commutativity is one axiom, but it changes the character of the structure
  profoundly.  Abelian groups are far more tractable than general groups: their
  subgroup structure is simpler, they admit a full classification theorem (every
  finitely generated abelian group is a product of cyclic groups), and they form
  the additive backbone of every ring and field.  The additive notation $+$
  signals commutativity by convention throughout algebra.
\end{remark}

\bpfooter{%
  \textbf{IS-A:} Group \quad \textbf{HAS-A:} Binary Operation\\
  \textbf{Extended by:} Ring \textit{(HAS-A Ab.\ Group under $+$)};\ Vector
  Space \textit{(HAS-A Ab.\ Group on vectors)}
}{%
  \textbf{Examples:} $(\mathbb{Z}, +, 0)$, $(\mathbb{R}, +, 0)$,
  $(\mathbb{Z}/n\mathbb{Z}, +, \bar{0})$.\\
  \textbf{Non-example:} $(S_3, \circ)$ — non-abelian group of order 6.
}


% =============================================================
%  BLUEPRINT 8 — Ring
% =============================================================
\clearpage\thispagestyle{plain}
\bpheader{cRing}{Ring}{$(A,+,\cdot,0,1)$}

\begin{bpnew}{cRing}{New at Ring}
  \bpitem{$\mathrm{f}$}{\cdot : A \times A \to A}{multiplicative binary operation}
  \bpitem{$\exists$}{1 \in A}{multiplicative identity}
  \bpitem{$\forall$}{\forall\,a \in A:\; a \cdot 1 = 1 \cdot a = a}{identity law for $\cdot$}
  \bpitem{$\forall$}{\forall\,a,b,c \in A:\; (a \cdot b) \cdot c = a \cdot (b \cdot c)}{associativity of $\cdot$}
  \bpitem{$\forall$}{\forall\,a,b,c \in A:\; a\cdot(b+c) = a\cdot b + a\cdot c}{left distributivity}
  \bpitem{$\forall$}{\forall\,a,b,c \in A:\; (a+b)\cdot c = a\cdot c + b\cdot c}{right distributivity}
\end{bpnew}

\noindent
\begin{minipage}[t]{0.485\linewidth}
  \begin{bpinh}{cAbelian}{HAS-A: Abelian Group under addition}
    \bpitem{$\mathrm{f}$}{+ : A \times A \to A}{addition}
    \bpitem{$\exists$}{0 \in A}{additive identity}
    \bpitem{$\exists$}{\forall\,a,\;\exists\,{-a} \in A}{additive inverse}
    \bpitem{$\forall$}{\forall\,a,b:\; a+b = b+a}{commutativity of $+$}
    \bpitem{$\forall$}{\forall\,a,b,c:\; (a+b)+c = a+(b+c)}{associativity of $+$}
  \end{bpinh}
\end{minipage}%
\hfill
\begin{minipage}[t]{0.485\linewidth}
  \begin{bpinh}{cMonoid}{HAS-A: Monoid under multiplication}
    \bpitem{$\mathrm{f}$}{\cdot : A \times A \to A}{multiplication}
    \bpitem{$\exists$}{1 \in A}{multiplicative identity}
    \bpitem{$\forall$}{\forall\,a,b,c:\; (a\cdot b)\cdot c = a\cdot(b\cdot c)}{associativity of $\cdot$}
  \end{bpinh}
\end{minipage}

\begin{definition}[Ring]
  A \textbf{ring} is a tuple $(A, +, \cdot, 0, 1)$ where $A$ is a nonempty set
  satisfying:
  \begin{align*}
    (A,+,0,-)\  &\text{is an abelian group} \tag{Ab.\ Group under $+$}\\
    (A,\cdot,1)\ &\text{is a monoid} \tag{Monoid under $\cdot$}\\
    \forall\,a,b,c \in A:\ &a\cdot(b+c) = a\cdot b + a\cdot c \tag{Left Distributivity}\\
    \forall\,a,b,c \in A:\ &(a+b)\cdot c = a\cdot c + b\cdot c \tag{Right Distributivity}
  \end{align*}
  Note: $\cdot$ need not be commutative, and multiplicative inverses need not
  exist.

  \medskip\noindent
  Model-theoretic signature:
  \[
    \mathcal{L}_{\mathrm{ring}} = \{\,+,\,\cdot,\,0,\,1,\,-\,\},
    \quad +,\cdot \text{ arity } 2,\quad - \text{ arity } 1,\quad
    0,1 \text{ constants.}
  \]
\end{definition}

\begin{remark}
  A ring is the algebraic formalization of \emph{arithmetic}: two operations
  linked by distributivity.  The integers $\mathbb{Z}$ are the canonical
  example.  Distributivity is the bridge between addition and multiplication —
  it is what makes the two operations interact coherently rather than live as
  independent structures on the same set.  Rings with commutative multiplication
  are called \emph{commutative rings}; those where every nonzero element has a
  multiplicative inverse are \emph{division rings}; both together give a
  \emph{field}.
\end{remark}

\bpfooter{%
  \textbf{HAS-A:} Abelian Group $(A,+,0,-)$;\ Monoid $(A,\cdot,1)$\\
  \textbf{Extended by:} Field \textit{(adds mult.\ inverses + commutativity)}
}{%
  \textbf{Examples:} $(\mathbb{Z},+,\cdot)$, $M_n(\mathbb{R})$ matrices,
  $\mathbb{Z}[x]$ polynomials.\\
  \textbf{Non-example:} $(\mathbb{N},+,\cdot)$ — no additive inverses.
}


% =============================================================
%  BLUEPRINT 9 — Field
% =============================================================
\clearpage\thispagestyle{plain}
\bpheader{cField}{Field}{$(A,+,\cdot,0,1,a^{-1})$}

\begin{bpnew}{cField}{New at Field}
  \bpitem{$\exists$}{\forall\,a \in A\setminus\{0\},\;\exists\,a^{-1} \in A}{multiplicative inverse}
  \bpitem{$\forall$}{\forall\,a \in A\setminus\{0\}:\; a \cdot a^{-1} = a^{-1} \cdot a = 1}{inverse law}
  \bpitem{$\forall$}{\forall\,a,b \in A:\; a \cdot b = b \cdot a}{commutativity of $\cdot$}
\end{bpnew}

\begin{bpinh}{cRing}{Inherited from Ring}
  \bpitem{$\forall$}{\forall\,a,b,c \in A:\; a\cdot(b+c)=a\cdot b+a\cdot c}{left distributivity}
  \bpitem{$\forall$}{\forall\,a,b,c \in A:\; (a+b)\cdot c=a\cdot c+b\cdot c}{right distributivity}
  \bpitem{$\exists$}{1 \in A}{multiplicative identity}
  \bpitem{$\forall$}{\forall\,a,b,c \in A:\; (a\cdot b)\cdot c = a\cdot(b\cdot c)}{associativity of $\cdot$}
\end{bpinh}

\begin{bpinh}{cAbelian}{Inherited from Abelian Group (under addition)}
  \bpitem{$\mathrm{f}$}{+ : A \times A \to A}{addition}
  \bpitem{$\exists$}{0 \in A}{additive identity}
  \bpitem{$\exists$}{\forall\,a,\;\exists\,{-a} \in A}{additive inverse}
  \bpitem{$\forall$}{\forall\,a,b \in A:\; a+b = b+a}{commutativity of $+$}
  \bpitem{$\forall$}{\forall\,a,b,c \in A:\; (a+b)+c = a+(b+c)}{associativity of $+$}
\end{bpinh}

\begin{definition}[Field]
  A \textbf{field} is a tuple $(A, +, \cdot, 0, 1)$ where $A$ is a set with
  $|A| \geq 2$, satisfying all ring axioms plus:
  \begin{align*}
    \forall\,a,b \in A:\ & a \cdot b = b \cdot a \tag{Commutativity of $\cdot$}\\
    \forall\,a \in A\setminus\{0\},\ \exists\,a^{-1} \in A:\ &
      a \cdot a^{-1} = a^{-1} \cdot a = 1 \tag{Multiplicative Inverses}
  \end{align*}
  Equivalently, $(A\setminus\{0\}, \cdot, 1, {}^{-1})$ is an abelian group.
  The condition $0 \ne 1$ (i.e.\ $|A|\ge 2$) excludes the trivial ring.

  \medskip\noindent
  Model-theoretic signature:
  \[
    \mathcal{L}_{\mathrm{fld}} = \{\,+,\,\cdot,\,0,\,1,\,-,\,{}^{-1}\,\},
    \quad +,\cdot \text{ arity } 2,\quad -,{}^{-1} \text{ arity } 1,\quad
    0,1 \text{ constants.}
  \]
\end{definition}

\begin{remark}
  A field is a ring where multiplication is as strong as addition: every nonzero
  element can be divided by.  This is the algebraic setting for linear algebra
  ($\mathbb{R}$, $\mathbb{C}$, $\mathbb{Q}$) and number theory (finite fields
  $\mathbb{F}_p$).  Division — the operation conspicuously absent from rings —
  is what makes solving linear equations ($ax = b \Rightarrow x = a^{-1}b$)
  always possible for $a \ne 0$.  The insistence on commutativity of $\cdot$
  distinguishes fields from \emph{division rings} (skew fields), where inverses
  exist but $ab \ne ba$ generally.
\end{remark}

\bpfooter{%
  \textbf{IS-A:} Ring\\
  \textbf{Used by:} Vector Space \textit{(USES-A Field for scalars)}
}{%
  \textbf{Examples:} $\mathbb{Q}$, $\mathbb{R}$, $\mathbb{C}$,
  $\mathbb{F}_p = \mathbb{Z}/p\mathbb{Z}$ ($p$ prime).\\
  \textbf{Non-example:} $\mathbb{Z}$ — no multiplicative inverse for $2$.
}


% =============================================================
%  BLUEPRINT 10 — Vector Space
% =============================================================
\clearpage\thispagestyle{plain}
\bpheader{cVS}{Vector Space}{$(V,+,0,-v,F,\cdot)$}

\begin{bpnew}{cVS}{New at Vector Space}
  \bpitem{$\mathrm{f}$}{\cdot : F \times V \to V}{scalar multiplication}
  \bpitem{$\forall$}{\forall\,a \in F,\,\mathbf{u},\mathbf{v} \in V:\;
    a\cdot(\mathbf{u}+\mathbf{v}) = a\cdot\mathbf{u} + a\cdot\mathbf{v}}{distributivity over vector addition}
  \bpitem{$\forall$}{\forall\,a,b \in F,\,\mathbf{v} \in V:\;
    (a+b)\cdot\mathbf{v} = a\cdot\mathbf{v} + b\cdot\mathbf{v}}{distributivity over scalar addition}
  \bpitem{$\forall$}{\forall\,a,b \in F,\,\mathbf{v} \in V:\;
    (ab)\cdot\mathbf{v} = a\cdot(b\cdot\mathbf{v})}{compatibility of scalar mult.}
  \bpitem{$\forall$}{\forall\,\mathbf{v} \in V:\; 1_F\cdot\mathbf{v} = \mathbf{v}}{identity scalar}
\end{bpnew}

\noindent
\begin{minipage}[t]{0.485\linewidth}
  \begin{bpinh}{cAbelian}{HAS-A: Abelian Group on $V$}
    \bpitem{$\mathrm{f}$}{+ : V \times V \to V}{vector addition}
    \bpitem{$\exists$}{\mathbf{0} \in V}{zero vector}
    \bpitem{$\exists$}{\forall\,\mathbf{v},\;\exists\,{-\mathbf{v}} \in V}{additive inverse}
    \bpitem{$\forall$}{\forall\,\mathbf{u},\mathbf{v} \in V:\;
      \mathbf{u}+\mathbf{v}=\mathbf{v}+\mathbf{u}}{commutativity}
    \bpitem{$\forall$}{\forall\,\mathbf{u},\mathbf{v},\mathbf{w}:\;
      (\mathbf{u}+\mathbf{v})+\mathbf{w}=\mathbf{u}+(\mathbf{v}+\mathbf{w})}{associativity}
  \end{bpinh}
\end{minipage}%
\hfill
\begin{minipage}[t]{0.485\linewidth}
  \begin{bpinh}{cField}{USES-A: Field $F$ for scalars}
    \bpitem{$\mathrm{f}$}{+,\,\cdot : F \times F \to F}{field operations}
    \bpitem{$\exists$}{0_F,\, 1_F \in F}{additive \& mult.\ identity}
    \bpitem{$\exists$}{\forall\,a \in F\setminus\{0\},\;\exists\,a^{-1}}{mult.\ inverse}
    \bpitem{$\forall$}{\forall\,a,b \in F:\; ab = ba}{commutativity of $\cdot$}
  \end{bpinh}
\end{minipage}

\begin{definition}[Vector Space]
  Let $F$ be a field.  A \textbf{vector space} over $F$ is a tuple
  $(V, +, \mathbf{0}, -, F, \cdot)$ where $(V, +, \mathbf{0}, -)$ is an
  abelian group and $\cdot : F \times V \to V$ satisfies, for all $a, b \in F$
  and $\mathbf{u}, \mathbf{v} \in V$:
  \begin{align*}
    a \cdot (\mathbf{u} + \mathbf{v}) &= a\cdot\mathbf{u} + a\cdot\mathbf{v}
      \tag{Distributivity over vector $+$}\\
    (a + b)\cdot\mathbf{v} &= a\cdot\mathbf{v} + b\cdot\mathbf{v}
      \tag{Distributivity over scalar $+$}\\
    (ab)\cdot\mathbf{v} &= a\cdot(b\cdot\mathbf{v})
      \tag{Compatibility}\\
    1_F\cdot\mathbf{v} &= \mathbf{v}
      \tag{Identity scalar}
  \end{align*}
  Elements of $V$ are called \textbf{vectors}; elements of $F$ are called
  \textbf{scalars}.

  \medskip\noindent
  Model-theoretic signature:
  \[
    \mathcal{L}_{\mathrm{VS}} = \{\,+_V,\,\mathbf{0},\,-_V,\,\cdot\,\}
    \cup \mathcal{L}_{\mathrm{fld}},
  \]
  where $\cdot$ has arity~2 (one scalar, one vector) and
  $\mathcal{L}_{\mathrm{fld}}$ names the field operations on $F$.
\end{definition}

\begin{remark}
  A vector space decouples two roles: vectors are the objects being combined
  (via $+$), and scalars are the coefficients that stretch or shrink them
  (via $\cdot$).  The scalars must form a field so that scalar division is
  always possible — this is what makes Gaussian elimination work.  The four
  scalar-multiplication axioms together say that scaling behaves consistently
  with both kinds of addition and with the field's own multiplication.  Linear
  algebra — spanning sets, bases, dimension, linear maps — is entirely the
  theory of vector spaces over a field.
\end{remark}

\bpfooter{%
  \textbf{HAS-A:} Abelian Group $(V,+,\mathbf{0})$ on vectors\\
  \textbf{USES-A:} Field $F$ for scalars (external dependency)
}{%
  \textbf{Examples:} $\mathbb{R}^n$ over $\mathbb{R}$,\;
  $\mathbb{C}^n$ over $\mathbb{C}$,\; polynomials $F[x]$ over $F$.\\
  \textbf{Note:} Changing the field changes the space:
  $\mathbb{C}$ is dim-2 over $\mathbb{R}$, dim-1 over $\mathbb{C}$.
}