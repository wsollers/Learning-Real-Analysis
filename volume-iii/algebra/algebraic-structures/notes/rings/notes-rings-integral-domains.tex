% =========================================================
% Integral Domains
% =========================================================
\paragraph{Integral Domains}

\begin{remark}[Motivation]
In $\mathbb{Z}$, if $ab = 0$ then $a = 0$ or $b = 0$.
This feels obvious but it is not an axiom of rings --- it fails in
$\mathbb{Z}/6\mathbb{Z}$, where $[2] \cdot [3] = [0]$ even though
neither $[2]$ nor $[3]$ is zero. Elements that behave like $[2]$
and $[3]$ are called zero divisors, and rings without them are
integral domains.

This property matters because it is exactly what is needed for
multiplicative cancellation --- and for the zero product argument
that appears throughout linear algebra.
\end{remark}

\begin{definition}[Zero Divisor]
Let $R$ be a commutative ring with unity.
A nonzero element $a \in R$ is a \emph{zero divisor} if there exists
a nonzero $b \in R$ such that $a \cdot b = 0$.
\end{definition}

\begin{definition}[Integral Domain]
A commutative ring with unity $R$ is an \emph{integral domain} if
$R$ has no zero divisors. Equivalently,
\[
a \cdot b = 0 \;\Longrightarrow\; a = 0 \;\text{ or }\; b = 0
\quad \text{for all } a, b \in R.
\]
\end{definition}

\begin{remark}[Connection to Linear Algebra]
The zero product property is the theorem cited in the vector space proof
that $av = \mathbf{0} \Rightarrow a = 0$ or $v = \mathbf{0}$.
Specifically, the case $a \neq 0$ uses the fact that the scalar field
$\mathbb{F}$ has no zero divisors --- which holds because every field
is an integral domain (proved in the next section).

The proof is not about vectors at all. It is about the scalar field.
\end{remark}

\begin{proposition}[Cancellation in Integral Domains]
\label{prop:domain-cancellation}
Let $R$ be an integral domain and $a, b, c \in R$ with $a \neq 0$.
Then
\[
ab = ac \;\Longrightarrow\; b = c.
\]
\end{proposition}


\begin{example}[Integral Domains and Non-Examples]
\begin{enumerate}[label=(\roman*)]
  \item $\mathbb{Z}$, $\mathbb{Q}$, $\mathbb{R}$, $\mathbb{C}$:
        all integral domains.
  \item $\mathbb{Z}[x]$: integral domain.
  \item $\mathbb{Z}/p\mathbb{Z}$ for prime $p$: integral domain
        (in fact a field, as shown in the next section).
  \item $\mathbb{Z}/6\mathbb{Z}$: \emph{not} an integral domain,
        since $[2][3] = [0]$ with $[2],[3] \neq [0]$.
  \item $M_2(\mathbb{R})$: \emph{not} an integral domain ---
        not commutative, and admits zero divisors.
\end{enumerate}
\end{example}
