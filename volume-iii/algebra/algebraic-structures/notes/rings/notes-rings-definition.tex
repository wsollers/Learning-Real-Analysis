% =========================================================
% Ring Definition and Axioms
% =========================================================
\paragraph{Definition of a Ring}

\begin{remark}[Motivation]
A group has one binary operation. A ring has two: addition and
multiplication. The addition makes the ring an abelian group.
Multiplication is layered on top, connected to addition through
the distributive laws.

Rings are the natural algebraic home of arithmetic. The integers
$\mathbb{Z}$, polynomials $\mathbb{Z}[x]$, and square matrices
$M_n(\mathbb{R})$ are all rings. What they share is not the
specific objects but the axiom structure --- and every theorem
proved here holds simultaneously in all of them.
\end{remark}

\begin{definition}[Ring]
A \emph{ring} is a triple $(R, +, \cdot)$ where $R$ is a set and
$+$ and $\cdot$ are binary operations on $R$ satisfying:

\begin{enumerate}[label=\textbf{R\arabic*.}]
  \item \textbf{Additive abelian group.}
        $(R, +)$ is an abelian group:
        \begin{itemize}
          \item $(a + b) + c = a + (b + c)$ for all $a,b,c \in R$.
          \item $a + b = b + a$ for all $a, b \in R$.
          \item There exists $0 \in R$ such that $a + 0 = a$ for all $a \in R$.
          \item For each $a \in R$, there exists $-a \in R$ such that
                $a + (-a) = 0$.
        \end{itemize}

  \item \textbf{Multiplicative associativity.}
        $(a \cdot b) \cdot c = a \cdot (b \cdot c)$
        for all $a, b, c \in R$.

  \item \textbf{Distributivity.}
        \begin{itemize}
          \item $a \cdot (b + c) = a \cdot b + a \cdot c$
                for all $a,b,c \in R$.
          \item $(a + b) \cdot c = a \cdot c + b \cdot c$
                for all $a,b,c \in R$.
        \end{itemize}
\end{enumerate}
\end{definition}

\begin{remark}[What a Ring Does and Does Not Require]
A ring does \emph{not} require:
\begin{itemize}
  \item commutativity of multiplication ($ab = ba$ need not hold),
  \item a multiplicative identity ($1$ need not exist),
  \item multiplicative inverses ($a^{-1}$ need not exist).
\end{itemize}
Each additional requirement produces a richer structure.
A ring with a multiplicative identity is a \emph{ring with unity}.
A commutative ring with unity where every nonzero element has a
multiplicative inverse is a \emph{field} --- covered in the next section.

Intuitively: a ring is the minimal structure needed for addition,
subtraction, and multiplication to coexist and interact sensibly.
Division is not guaranteed.
\end{remark}

\begin{definition}[Commutative Ring]
A ring $(R, +, \cdot)$ is \emph{commutative} if
\[
a \cdot b = b \cdot a \quad \text{for all } a, b \in R.
\]
\end{definition}

\begin{definition}[Ring with Unity]
A ring $(R, +, \cdot)$ is a \emph{ring with unity} if there exists
$1 \in R$ such that
\[
1 \cdot a = a \cdot 1 = a \quad \text{for all } a \in R.
\]
The element $1$ is called the \emph{multiplicative identity} or \emph{unity}.
When it exists, it is unique (proof identical to
Proposition~\ref{prop:group-identity-unique} applied to $(R, \cdot)$).
\end{definition}

\begin{example}[Canonical Examples of Rings]
\begin{enumerate}[label=(\roman*)]
  \item $(\mathbb{Z}, +, \cdot)$: integers.
        Commutative ring with unity $1$.
        No multiplicative inverses for $|n| \neq 1$.

  \item $(\mathbb{Z}/n\mathbb{Z}, +, \cdot)$: integers modulo $n$.
        Commutative ring with unity $[1]$.

  \item $(M_n(\mathbb{R}), +, \cdot)$: $n \times n$ real matrices.
        Ring with unity $I_n$. \emph{Not} commutative for $n \geq 2$.

  \item $(\mathbb{Z}[x], +, \cdot)$: polynomials with integer coefficients.
        Commutative ring with unity $1$.

  \item The trivial ring $\{0\}$, where $0 = 1$, is the only ring
        in which the additive and multiplicative identities coincide.
\end{enumerate}
\end{example}
