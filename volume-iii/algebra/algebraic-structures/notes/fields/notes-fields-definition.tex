% =========================================================
% Field Definition and Axioms
% =========================================================
\paragraph{Definition of a Field}

\begin{remark}[Motivation]
A ring allows addition, subtraction, and multiplication.
A field adds division: every nonzero element has a multiplicative inverse.
This is the structure of $\mathbb{Q}$, $\mathbb{R}$, and $\mathbb{C}$ ---
the number systems where you can always solve $ax = b$ for $a \neq 0$.

Fields are the scalars of linear algebra. Axler's $\mathbb{F}$ denotes
an arbitrary field, meaning every theorem in linear algebra holds
simultaneously over $\mathbb{R}$, $\mathbb{C}$, $\mathbb{Q}$,
and any other field --- because the proofs use only field axioms,
not properties specific to real or complex numbers.
\end{remark}

\begin{definition}[Field]
A \emph{field} is a commutative ring with unity $(F, +, \cdot)$
satisfying:

\begin{enumerate}[label=\textbf{F\arabic*.}]
  \item \textbf{Additive abelian group.}
        $(F, +)$ is an abelian group with identity $0$.

  \item \textbf{Multiplicative abelian group on nonzero elements.}
        $(F \setminus \{0\}, \cdot)$ is an abelian group with identity $1$.
        Explicitly: for each $a \in F$ with $a \neq 0$, there exists
        $a^{-1} \in F$ such that $a \cdot a^{-1} = 1$.

  \item \textbf{Distributivity.}
        $a \cdot (b + c) = a \cdot b + a \cdot c$
        for all $a, b, c \in F$.

  \item \textbf{Non-triviality.}
        $0 \neq 1$.
\end{enumerate}
\end{definition}

\begin{remark}[Unpacking the Definition]
A field is simultaneously:
\begin{itemize}
  \item $(F, +)$: an abelian group (additive structure),
  \item $(F \setminus \{0\}, \cdot)$: an abelian group (multiplicative structure),
  \item connected by distributivity.
\end{itemize}
The non-triviality axiom $0 \neq 1$ rules out the trivial ring $\{0\}$,
which would otherwise technically satisfy the other axioms.
\end{remark}

\begin{example}[Fields]
\begin{enumerate}[label=(\roman*)]
  \item $\mathbb{Q}$, $\mathbb{R}$, $\mathbb{C}$: the standard fields.
  \item $\mathbb{Z}/p\mathbb{Z}$ for any prime $p$: a finite field
        with $p$ elements, denoted $\mathbb{F}_p$.
  \item $\mathbb{Q}(\sqrt{2}) = \{a + b\sqrt{2} : a, b \in \mathbb{Q}\}$:
        a field extending $\mathbb{Q}$.
\end{enumerate}
\end{example}

\begin{example}[Non-Fields]
\begin{enumerate}[label=(\roman*)]
  \item $\mathbb{Z}$: not a field. The element $2$ has no multiplicative
        inverse in $\mathbb{Z}$.
  \item $\mathbb{Z}/6\mathbb{Z}$: not a field. $[2]$ has no inverse
        since $\gcd(2,6) \neq 1$. (Also not an integral domain.)
  \item $M_n(\mathbb{R})$ for $n \geq 2$: not a field.
        Not commutative, and singular matrices have no inverse.
\end{enumerate}
\end{example}
