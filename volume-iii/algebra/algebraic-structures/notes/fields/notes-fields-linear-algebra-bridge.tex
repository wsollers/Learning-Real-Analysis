% =========================================================
% Bridge to Linear Algebra
% =========================================================
\paragraph{Bridge to Linear Algebra}

\begin{remark}[Why This Chapter Exists]
Every proof in Axler's Chapter 1 draws on exactly two sources:
properties of the field $\mathbb{F}$, and properties of the
vector space $V$. This chapter is the permanent home of the
field side. The table below maps each type of linear algebra
proof step to the theorem it cites from this chapter.
\end{remark}

\begin{center}
\renewcommand{\arraystretch}{1.4}
\begin{tabular}{p{0.36\textwidth} p{0.26\textwidth} p{0.28\textwidth}}
\toprule
\textbf{Proof step in linear algebra} & \textbf{Structure used} & \textbf{Theorem} \\
\midrule
``since $a \neq 0$, $a^{-1}$ exists'' &
Field $\mathbb{F}$ &
Prop.~\ref{prop:field-inverse-exists} \\

``$ab = 0$ and $a \neq 0$ implies $b = 0$'' &
Field $\mathbb{F}$ &
Prop.~\ref{prop:field-zero-product} \\

``$(-1)v = -v$'' &
Ring theorem on $\mathbb{F}$ &
Prop.~\ref{prop:ring-mult-neg} \\

``$0 \cdot v = \mathbf{0}$'' &
Ring theorem on $\mathbb{F}$ acting on $V$ &
Prop.~\ref{prop:ring-mult-zero} \\

``the additive identity is unique'' &
Abelian group $(V, +)$ &
Prop.~\ref{prop:group-identity-unique} \\

``additive inverses are unique'' &
Abelian group $(V, +)$ &
Prop.~\ref{prop:group-inverse-unique} \\

``$-(-v) = v$'' &
Abelian group $(V, +)$ &
Prop.~\ref{prop:group-inverse-unique} \\
\bottomrule
\end{tabular}
\end{center}

\begin{remark}[The Interaction Layer]
The table above separates into two kinds of steps: those that use
the structure of the field $\mathbb{F}$ alone (rows 1--3), and those
that use the abelian group structure of $V$ alone (rows 5--7).
Row 4 is the interaction: it uses a field theorem applied across
scalar multiplication to a vector.

This separation is the heart of the vector space structure.
The field $\mathbb{F}$ and the abelian group $V$ are independent objects
connected by the scalar multiplication axioms. When a proof crosses
that connection, it is using the interaction layer. When it stays
on one side, it is using either field theory or group theory alone.
Knowing which side you are on at each step is what the DU/TA/AM
annotation system makes explicit.
\end{remark}
