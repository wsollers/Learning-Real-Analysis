\begin{proof}
Let $\mathbb{F}$ be a field. We show $\mathbb{F}$ has no zero divisors.

Suppose $a, b \in \mathbb{F}$ with $ab = 0$. We show $a = 0$ or $b = 0$.

\noindent
\begin{tabular}{T S J}
\toprule
\textbf{Tag} & \multicolumn{1}{p{0.44\textwidth}}{\textbf{Step}} & \textbf{Justification} \\
\midrule
\addlinespace[4pt]

\tagAM
&
\text{Assume } a \neq 0.
&
We show this forces $b = 0$.
\\[10pt]

\tagDU
&
a^{-1} \in \mathbb{F} \text{ exists}
&
$\mathbb{F}$ is a field and $a \neq 0$:
every nonzero element has a multiplicative inverse (Axiom F2).
\\[10pt]

\tagAM
&
a^{-1}(ab) = a^{-1} \cdot 0
&
Multiply both sides of $ab = 0$ on the left by $a^{-1}$.
\\[10pt]

\tagTA
&
(a^{-1}a)b = a^{-1} \cdot 0
&
Multiplicative associativity (Axiom R2).
\\[10pt]

\tagDU
&
1 \cdot b = a^{-1} \cdot 0
&
$a^{-1}a = 1$ by definition of multiplicative inverse (Axiom F2).
\\[10pt]

\tagTA
&
b = 0
&
Left side: $1 \cdot b = b$ by multiplicative identity.
Right side: $a^{-1} \cdot 0 = 0$ by
Proposition~\ref{prop:ring-mult-zero}.
\hfill$\blacksquare$
\\[6pt]

\bottomrule
\end{tabular}

\medskip
\noindent Since $a = 0$ or ($a \neq 0$ implies $b = 0$),
we have $a = 0$ or $b = 0$ in all cases.
\end{proof}
