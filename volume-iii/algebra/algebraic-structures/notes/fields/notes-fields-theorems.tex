% =========================================================
% Basic Theorems of Fields
% =========================================================
\paragraph{Basic Theorems}

\begin{remark}[Why These Theorems Matter]
These are the theorems cited in every vector space proof that crosses
from the scalar side to the vector side. They live here, in the field,
and are cited by number when needed in linear algebra.
\end{remark}

\begin{proposition}[Every Field is an Integral Domain]
\label{prop:field-is-domain}
Every field is an integral domain.
\end{proposition}


\begin{remark}
This is the theorem behind the linear algebra argument:
if $av = \mathbf{0}$ and $a \neq 0$, then $v = \mathbf{0}$.
The scalar $a$ lives in a field $\mathbb{F}$, and the key step
is that $\mathbb{F}$ has no zero divisors.
\end{remark}

\begin{proposition}[Zero Product Property in a Field]
\label{prop:field-zero-product}
Let $\mathbb{F}$ be a field and $a, b \in \mathbb{F}$. Then
\[
ab = 0 \;\Longrightarrow\; a = 0 \;\text{ or }\; b = 0.
\]
\end{proposition}

\begin{proof}
Immediate from Proposition~\ref{prop:field-is-domain},
since every field is an integral domain.
\end{proof}

\begin{proposition}[Nonzero Scalars Have Inverses]
\label{prop:field-inverse-exists}
Let $\mathbb{F}$ be a field and $a \in \mathbb{F}$ with $a \neq 0$.
Then there exists a unique $a^{-1} \in \mathbb{F}$ such that
$a \cdot a^{-1} = 1$.
\end{proposition}

\begin{proof}
Existence is Axiom F2. Uniqueness follows from
Proposition~\ref{prop:group-inverse-unique} applied to the
multiplicative group $(F \setminus \{0\}, \cdot)$.
\end{proof}

\begin{remark}
This is the theorem cited in the vector space proof that
$av = \mathbf{0}$ with $a \neq 0$ implies $v = \mathbf{0}$:
the step ``multiply both sides by $a^{-1}$'' is valid
precisely because $a^{-1}$ exists and is unique.
\end{remark}

\begin{proposition}[Characteristic of a Field]
\label{prop:field-characteristic}
Let $\mathbb{F}$ be a field. The \emph{characteristic} of $\mathbb{F}$
is the smallest positive integer $n$ such that
\[
\underbrace{1 + 1 + \cdots + 1}_{n} = 0,
\]
or $0$ if no such $n$ exists.
The characteristic of a field is either $0$ or a prime $p$.
\end{proposition}

\begin{remark}
$\mathbb{Q}$, $\mathbb{R}$, and $\mathbb{C}$ all have characteristic $0$.
$\mathbb{Z}/p\mathbb{Z}$ has characteristic $p$.

In Axler, $\mathbb{F}$ denotes $\mathbb{R}$ or $\mathbb{C}$,
both of characteristic $0$. Some results (such as the existence
of eigenvalues) require characteristic $0$ or algebraic closure
and would fail over $\mathbb{F}_p$.
\end{remark}
