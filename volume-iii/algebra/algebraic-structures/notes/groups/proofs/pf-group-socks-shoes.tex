\begin{proof}
Let $a, b \in G$. We show $b^{-1}a^{-1}$ satisfies the definition
of the inverse of $ab$, then invoke uniqueness.

\noindent
\begin{tabular}{T S J}
\toprule
\textbf{Tag} & \multicolumn{1}{p{0.44\textwidth}}{\textbf{Step}} & \textbf{Justification} \\
\midrule
\addlinespace[4pt]

\tagAM
&
(ab)(b^{-1}a^{-1}) = a(bb^{-1})a^{-1}
&
Associativity (Axiom G1), regrouping.
\\[10pt]

\tagDU
&
a(bb^{-1})a^{-1} = aea^{-1}
&
$bb^{-1} = e$ by definition of inverse (Axiom G3).
\\[10pt]

\tagDU
&
aea^{-1} = aa^{-1}
&
$ae = a$ by definition of identity (Axiom G2).
\\[10pt]

\tagDU
&
aa^{-1} = e
&
Definition of inverse (Axiom G3). So $(ab)(b^{-1}a^{-1}) = e$.
\\[10pt]

\tagAM
&
(b^{-1}a^{-1})(ab) = b^{-1}(a^{-1}a)b
&
Associativity (Axiom G1), regrouping.
\\[10pt]

\tagDU
&
b^{-1}(a^{-1}a)b = b^{-1}eb
&
$a^{-1}a = e$ by definition of inverse (Axiom G3).
\\[10pt]

\tagDU
&
b^{-1}eb = b^{-1}b = e
&
Identity axiom then inverse axiom (Axioms G2, G3).
So $(b^{-1}a^{-1})(ab) = e$.
\\[10pt]

\tagTA
&
(ab)^{-1} = b^{-1}a^{-1}
&
$b^{-1}a^{-1}$ satisfies both sides of the inverse definition.
By Proposition~\ref{prop:group-inverse-unique}, inverses are unique.
\hfill$\blacksquare$
\\[6pt]

\bottomrule
\end{tabular}
\end{proof}