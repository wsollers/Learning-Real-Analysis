\begin{proof}
Let $a \in G$ and suppose $b$ and $c$ are both inverses of $a$.

\noindent
\begin{tabular}{T S J}
\toprule
\textbf{Tag} & \multicolumn{1}{p{0.44\textwidth}}{\textbf{Step}} & \textbf{Justification} \\
\midrule
\addlinespace[4pt]

\tagDU
&
b \star a = e
&
$b$ is an inverse of $a$: definition of inverse (Axiom G3).
\\[10pt]

\tagDU
&
a \star c = e
&
$c$ is an inverse of $a$: definition of inverse (Axiom G3).
\\[10pt]

\tagDU
&
b = b \star e
&
$e$ is the identity: $b \star e = b$ (Axiom G2).
\\[10pt]

\tagAM
&
b \star e = b \star (a \star c)
&
Substituting $e = a \star c$.
\\[10pt]

\tagTA
&
b \star (a \star c) = (b \star a) \star c
&
Associativity (Axiom G1).
\\[10pt]

\tagAM
&
(b \star a) \star c = e \star c
&
Substituting $b \star a = e$.
\\[10pt]

\tagDU
&
e \star c = c
&
$e$ is the identity: $e \star c = c$ (Axiom G2). \hfill$\blacksquare$
\\[6pt]

\bottomrule
\end{tabular}

\medskip
\noindent Therefore $b = b \star e = b \star (a \star c) = (b \star a) \star c = e \star c = c$,
so the inverse of $a$ is unique.
\end{proof}