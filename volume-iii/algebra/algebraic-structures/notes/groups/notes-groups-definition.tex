% =========================================================
% Group Definition and Axioms
% =========================================================
\paragraph{Definition of a Group}

\begin{definition}[Group]
A \emph{group} is a pair $(G, \star)$ where $G$ is a set and
$\star$ is a binary operation on $G$ satisfying the following axioms:

\begin{enumerate}[label=\textbf{G\arabic*.}]
  \item \textbf{Associativity.}
        $(a \star b) \star c = a \star (b \star c)$
        for all $a, b, c \in G$.

  \item \textbf{Identity.}
        There exists an element $e \in G$ such that
        \[
          e \star a = a \star e = a
          \quad \text{for all } a \in G.
        \]
        The element $e$ is called the \emph{identity element} of $G$.

  \item \textbf{Inverses.}
        For each $a \in G$, there exists an element $a^{-1} \in G$ such that
        \[
          a \star a^{-1} = a^{-1} \star a = e.
        \]
        The element $a^{-1}$ is called the \emph{inverse} of $a$.
\end{enumerate}
\end{definition}

\begin{remark}
Closure is not listed as a separate axiom because it is already
encoded in the requirement that $\star : G \times G \to G$ is a
binary operation --- the codomain forces the result to stay in $G$.

Intuitively: a group is a set where you can combine elements,
undo combinations, and the order of grouping never matters.
\end{remark}

\begin{remark}[Axiom Count]
Some treatments list four axioms (closure, associativity, identity, inverses).
Here closure is absorbed into the definition of binary operation,
leaving three axioms. Both presentations define the same object.
\end{remark}

\begin{remark}[Notation]
When the operation is understood from context, we write $ab$ instead
of $a \star b$, and call $G$ itself a group rather than the pair $(G, \star)$.
For groups whose operation is addition, we write $a + b$, use $0$ for the
identity, and $-a$ for the inverse of $a$.
\end{remark}

\begin{definition}[Order of a Group]
The \emph{order} of a group $G$, denoted $|G|$, is the cardinality of $G$
as a set. If $|G|$ is finite, $G$ is called a \emph{finite group};
otherwise it is an \emph{infinite group}.
\end{definition}

\begin{example}[Canonical Examples of Groups]
\begin{enumerate}[label=(\roman*)]
  \item $(\mathbb{Z}, +)$: the integers under addition.
        Identity: $0$. Inverse of $n$: $-n$. Infinite group.

  \item $(\mathbb{Q} \setminus \{0\}, \cdot)$: nonzero rationals under multiplication.
        Identity: $1$. Inverse of $q$: $1/q$. Infinite group.

  \item $(\mathbb{Z}/n\mathbb{Z}, +)$: integers modulo $n$ under addition.
        Identity: $[0]$. Inverse of $[k]$: $[n-k]$. Finite group of order $n$.

  \item $(GL_n(\mathbb{R}), \cdot)$: invertible $n \times n$ real matrices
        under multiplication.
        Identity: $I_n$. Inverse: matrix inverse. Infinite group.
\end{enumerate}
\end{example}

\begin{remark}
Note what fails to be a group:
$(\mathbb{Z}, \cdot)$ is not a group because $2$ has no multiplicative
inverse in $\mathbb{Z}$.
$(\mathbb{N}, +)$ is not a group because positive integers have no
additive inverse in $\mathbb{N}$.
These failures illustrate why each axiom is necessary.
\end{remark}