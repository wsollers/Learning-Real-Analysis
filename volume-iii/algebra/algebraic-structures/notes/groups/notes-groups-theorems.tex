% =========================================================
% Basic Theorems of Groups
% =========================================================
\paragraph{Basic Theorems}

\begin{remark}[Why These Theorems Matter]
The group axioms guarantee existence of an identity and inverses,
but say nothing about uniqueness. The following theorems establish
that both are unique. This is essential: without uniqueness, we
cannot speak of \emph{the} identity or \emph{the} inverse of an element,
and proofs that equate two objects via the identity or inverse would
be invalid.

These same uniqueness theorems reappear in vector space proofs,
where they are cited by name. They are proved once here.
\end{remark}

% ---------------------------------------------------------
\begin{proposition}[Uniqueness of the Identity]
\label{prop:group-identity-unique}
Let $G$ be a group. The identity element of $G$ is unique.
\end{proposition}


\begin{remark}
The proof strategy is standard for uniqueness arguments:
assume two identities exist, then show they must be equal.
This pattern recurs throughout algebra whenever a definition
asserts existence of a distinguished element.

Intuitively: if two elements both act as an identity,
applying one to the other forces them to coincide.
\end{remark}

% ---------------------------------------------------------
\begin{proposition}[Uniqueness of Inverses]
\label{prop:group-inverse-unique}
Let $G$ be a group. For each $a \in G$, the inverse of $a$ is unique.
\end{proposition}


\begin{remark}
Intuitively: if two elements both undo $a$, then they must be
the same element --- because each can be obtained from the other
by cancellation.
\end{remark}

% ---------------------------------------------------------
\begin{proposition}[Cancellation Laws]
\label{prop:group-cancellation}
Let $G$ be a group and let $a, b, c \in G$. Then:
\begin{enumerate}[label=(\roman*)]
  \item \textbf{Left cancellation:} $ab = ac \implies b = c$.
  \item \textbf{Right cancellation:} $ba = ca \implies b = c$.
\end{enumerate}
\end{proposition}


\begin{remark}
Cancellation is what makes group equations solvable.
It does \emph{not} hold in general for rings or monoids without inverses.

Intuitively: multiply both sides by $a^{-1}$ and the common factor disappears.
\end{remark}

% ---------------------------------------------------------
\begin{proposition}[Socks-Shoes Property]
\label{prop:group-socks-shoes}
Let $G$ be a group and let $a, b \in G$. Then
\[
(ab)^{-1} = b^{-1} a^{-1}.
\]
\end{proposition}


\begin{remark}
The name comes from the observation that to undo putting on socks
then shoes, you must first remove the shoes, then the socks ---
in reverse order.

This reversal of order is characteristic of non-abelian groups and
becomes important in the theory of group homomorphisms and
in matrix algebra, where $(AB)^{-1} = B^{-1}A^{-1}$.
\end{remark}