% =========================================================
% Abelian Groups
% =========================================================
\paragraph{Abelian Groups}

\begin{definition}[Abelian Group]
A group $(G, \star)$ is called \emph{abelian} (or \emph{commutative}) if
\[
a \star b = b \star a \quad \text{for all } a, b \in G.
\]
\end{definition}

\begin{remark}
An abelian group is a group with one additional axiom: commutativity
of the operation. Every abelian group is a group, but not every group
is abelian.

Intuitively: in an abelian group, the order in which you combine
elements is irrelevant. This is the familiar arithmetic of addition
on $\mathbb{Z}$, $\mathbb{Q}$, $\mathbb{R}$, and $\mathbb{C}$.
\end{remark}

\begin{remark}[Structural Position]
Abelian groups are the additive backbone of every vector space.
The four vector space axioms governing addition --- associativity,
commutativity, existence of zero, existence of additive inverses ---
are precisely the axioms that make $(V, +)$ an abelian group.

This is why the vector space definition can be stated compactly as:
\emph{a vector space over $\mathbb{F}$ is an abelian group $(V,+)$
equipped with a scalar multiplication by $\mathbb{F}$}.
The abelian group structure is not an analogy; it is the literal
algebraic content of the first four vector space axioms.
\end{remark}

\begin{example}[Abelian and Non-Abelian Groups]
\begin{enumerate}[label=(\roman*)]
  \item $(\mathbb{Z}, +)$, $(\mathbb{R}, +)$, $(\mathbb{C}, +)$:
        all abelian. These are the additive groups underlying the
        standard vector spaces.

  \item $(\mathbb{Z}/n\mathbb{Z}, +)$: abelian for all $n \geq 1$.

  \item $(GL_n(\mathbb{R}), \cdot)$ for $n \geq 2$: \emph{not} abelian,
        since matrix multiplication does not commute in general.

  \item $(S_n, \circ)$ for $n \geq 3$: the symmetric group on $n$ symbols
        under composition is not abelian.
\end{enumerate}
\end{example}

\begin{remark}[Additive Notation Convention]
For abelian groups, it is standard to write the operation as $+$,
the identity as $0$, and the inverse of $a$ as $-a$.
This additive notation is used throughout linear algebra, where
$(V, +)$ is always an abelian group.
\end{remark}