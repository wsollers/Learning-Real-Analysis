% =========================================================
% Systems of Sets
% =========================================================

\subsection{Systems of Sets}

% ---------------------------------------------------------
% TOOLKIT
% ---------------------------------------------------------
\begin{tcolorbox}[colback=gray!6, colframe=gray!40, arc=2pt,
  left=6pt, right=6pt, top=4pt, bottom=4pt,
  title={\small\textbf{Systems of Sets --- Quick Reference}},
  fonttitle=\small\bfseries]
\small
\begin{tabular}{l l l}
\toprule
\textbf{Structure} & \textbf{Closure properties} & \textbf{Detail} \\
\midrule
Family of sets   & Any $\mathcal{F} \subseteq 2^X$                                          & \hyperref[def:family-sets]{↓ Def} \\
Closed under $*$ & Inputs in $\mathcal{F}$ implies output in $\mathcal{F}$                   & \hyperref[def:closure-op]{↓ Def} \\
Ring of sets     & Closed under $\cup$ and $\setminus$                                      & \hyperref[def:ring-sets]{↓ Def} \\
Algebra of sets  & Contains $X$; closed under $(\cdot)^c$ and $\cup$                        & \hyperref[def:algebra-sets]{↓ Def} \\
$\sigma$-ring    & Ring closed under countable $\cup$                                        & \hyperref[def:sigma-ring]{↓ Def} \\
$\sigma$-algebra & Algebra closed under countable $\cup$                                     & \hyperref[def:sigma-algebra]{↓ Def} \\
\midrule
\multicolumn{3}{l}{\textit{Key results:}} \\
\midrule
Algebra = ring + $X$ & Equivalent characterisation                                          & \hyperref[prop:algebra-ring]{↓ Prop} \\
$\sigma$-algebra closure & Also closed under countable $\cap$, $\setminus$, finite Boolean ops & \hyperref[prop:sigma-closure]{↓ Prop} \\
\bottomrule
\end{tabular}
\end{tcolorbox}

\vspace{1em}

% ---------------------------------------------------------
% Families and closure
% ---------------------------------------------------------

Let $X$ be a set. A \emph{family of sets on $X$} is a collection
$\mathcal{F} \subseteq 2^X$. The objects of study are not elements of $X$
but collections of subsets, classified by their stability under
set-theoretic operations.

\begin{tcolorbox}[colback=propbox, colframe=propborder, arc=2pt,
  left=6pt, right=6pt, top=4pt, bottom=4pt,
  title={\small\textbf{Definition (Family of Sets)}},
  fonttitle=\small\bfseries]
\label{def:family-sets}
Let $X$ be a set. A \emph{family of sets on $X$} is any collection
$\mathcal{F} \subseteq 2^X$.
\end{tcolorbox}

\begin{tcolorbox}[colback=propbox, colframe=propborder, arc=2pt,
  left=6pt, right=6pt, top=4pt, bottom=4pt,
  title={\small\textbf{Definition (Closure Under an Operation)}},
  fonttitle=\small\bfseries]
\label{def:closure-op}
Let $\mathcal{F} \subseteq 2^X$ and let $*$ be an operation on subsets of
$X$. We say $\mathcal{F}$ is \emph{closed under $*$} if whenever the inputs
to $*$ belong to $\mathcal{F}$, the output also belongs to $\mathcal{F}$.
\end{tcolorbox}

\begin{remark}[Typical operations]
The operations of interest are $\cup$, $\cap$, $(\cdot)^c$, $\setminus$,
and $\triangle$ (symmetric difference). Different combinations of closure
under these operations produce the distinct algebraic structures defined
below.
\end{remark}

% ---------------------------------------------------------
% Finite closure structures
% ---------------------------------------------------------

\begin{tcolorbox}[colback=propbox, colframe=propborder, arc=2pt,
  left=6pt, right=6pt, top=4pt, bottom=4pt,
  title={\small\textbf{Definition (Ring of Sets)}},
  fonttitle=\small\bfseries]
\label{def:ring-sets}
A nonempty collection $\mathcal{R} \subseteq 2^X$ is a \emph{ring of sets}
if it is closed under finite union and set difference:
\begin{enumerate}[label=(\roman*)]
  \item $A, B \in \mathcal{R} \;\Rightarrow\; A \cup B \in \mathcal{R}$,
  \item $A, B \in \mathcal{R} \;\Rightarrow\; A \setminus B \in \mathcal{R}$.
\end{enumerate}
\end{tcolorbox}

\begin{remark}[Intersection and symmetric difference]
A ring of sets is automatically closed under $\cap$ and $\triangle$,
since $A \cap B = A \setminus (A \setminus B)$ and
$A \triangle B = (A \setminus B) \cup (B \setminus A)$.
Under symmetric difference and intersection, a ring of sets is an abelian
group under $\triangle$ and a commutative monoid under $\cap$, making it
a Boolean ring in the algebraic sense.
\end{remark}

\begin{tcolorbox}[colback=propbox, colframe=propborder, arc=2pt,
  left=6pt, right=6pt, top=4pt, bottom=4pt,
  title={\small\textbf{Definition (Algebra of Sets)}},
  fonttitle=\small\bfseries]
\label{def:algebra-sets}
A collection $\mathcal{A} \subseteq 2^X$ is an \emph{algebra of sets}
(also called a \emph{field of sets}) if:
\begin{enumerate}[label=(\roman*)]
  \item $X \in \mathcal{A}$,
  \item $A \in \mathcal{A} \;\Rightarrow\; A^c \in \mathcal{A}$,
  \item $A, B \in \mathcal{A} \;\Rightarrow\; A \cup B \in \mathcal{A}$.
\end{enumerate}
\end{tcolorbox}

\begin{proposition}[Algebra = Ring + $X$]\label{prop:algebra-ring}
An algebra of sets is precisely a ring of sets that contains $X$.
\end{proposition}



\begin{remark}[Finite Boolean stability]
An algebra of sets is closed under all finite Boolean combinations:
finite unions, finite intersections, complements, differences, and
symmetric differences. The qualifier ``finite'' is essential; the next
step is to add closure under countably infinite operations.
\end{remark}

% ---------------------------------------------------------
% Countable closure structures
% ---------------------------------------------------------

\begin{tcolorbox}[colback=propbox, colframe=propborder, arc=2pt,
  left=6pt, right=6pt, top=4pt, bottom=4pt,
  title={\small\textbf{Definition ($\sigma$-Ring)}},
  fonttitle=\small\bfseries]
\label{def:sigma-ring}
A ring of sets $\mathcal{R}$ is a \emph{$\sigma$-ring} if it is also
closed under countable unions:
\[
A_1, A_2, \dots \in \mathcal{R}
\;\Rightarrow\;
\bigcup_{n=1}^{\infty} A_n \in \mathcal{R}.
\]
\end{tcolorbox}

\begin{tcolorbox}[colback=propbox, colframe=propborder, arc=2pt,
  left=6pt, right=6pt, top=4pt, bottom=4pt,
  title={\small\textbf{Definition ($\sigma$-Algebra)}},
  fonttitle=\small\bfseries]
\label{def:sigma-algebra}
A collection $\mathcal{F} \subseteq 2^X$ is a \emph{$\sigma$-algebra} if:
\begin{enumerate}[label=(\roman*)]
  \item $X \in \mathcal{F}$,
  \item $A \in \mathcal{F} \;\Rightarrow\; A^c \in \mathcal{F}$,
  \item $A_1, A_2, \dots \in \mathcal{F} \;\Rightarrow\;
        \displaystyle\bigcup_{n=1}^{\infty} A_n \in \mathcal{F}$.
\end{enumerate}
\end{tcolorbox}

\begin{proposition}[Closure consequences of a $\sigma$-algebra]\label{prop:sigma-closure}
Every $\sigma$-algebra $\mathcal{F}$ is also closed under countable
intersections, set differences, and all finite Boolean combinations.
\end{proposition}



\begin{remark}[Finite vs.\ countable closure]
The passage from an algebra to a $\sigma$-algebra is precisely the step
from finite to countable closure. This is what makes $\sigma$-algebras
compatible with limit processes: if measurable sets are approximated by
sequences of other measurable sets, the limit set remains measurable.
\end{remark}

% ---------------------------------------------------------
% Hierarchy
% ---------------------------------------------------------

\begin{center}
\renewcommand{\arraystretch}{1.3}
\begin{tabular}{lccc}
\toprule
\textbf{Structure} & \textbf{Contains $X$} & \textbf{Closed under $(\cdot)^c$} & \textbf{Closed under $\bigcup$} \\
\midrule
Ring of sets   & No  & No  & Finite only \\
Algebra        & Yes & Yes & Finite only \\
$\sigma$-Ring  & No  & No  & Countable \\
$\sigma$-Algebra & Yes & Yes & Countable \\
\bottomrule
\end{tabular}
\end{center}

\begin{remark}[Reading the hierarchy]
Each row is strictly stronger than the one above it: every algebra is a
ring, and every $\sigma$-algebra is both an algebra and a $\sigma$-ring.
The top of the hierarchy, the $\sigma$-algebra, is the natural domain for
measure theory because it combines the Boolean stability of an algebra with
the limit-compatibility of countable closure.
\end{remark}