% =========================================================
% Breadcrumb
% =========================================================
\begin{tcolorbox}[
  colback=gray!6,
  colframe=gray!40,
  arc=2pt,
  left=8pt, right=8pt, top=6pt, bottom=6pt,
  title={\small\textbf{Where You Are in the Journey}},
  fonttitle=\small\bfseries
]
\begin{center}
\small
Propositional Logic
$\;\to\;$ Sets \& Functions
$\;\to\;$ $\mathbb{R}$
$\;\to\;$ Algebraic Structures
$\;\to\;$ \textbf{Set Algebras}
$\;\to\;$ Measure Theory
$\;\to\;$ $\cdots$
\end{center}

\medskip
\noindent\textbf{How we got here.}
Set theory gave us the basic operations --- union, intersection, complement,
and power set --- and the language of subsets of a fixed space $X$.
Algebraic structures showed us how to classify mathematical objects by
their closure properties under operations. Set algebras bring these two
threads together: we ask which \emph{collections} of subsets are closed
under which set-theoretic operations, and what algebraic laws govern them.

\medskip
\noindent\textbf{What this chapter builds.}
We develop the hierarchy of closure systems on $2^X$: rings of sets,
algebras (fields) of sets, $\sigma$-rings, and $\sigma$-algebras.
Alongside this, we study characteristic functions as a bridge between
subsets and arithmetic, revealing that $2^X$ carries Boolean ring and
$\mathbb{F}_2$-vector space structure. The power set itself is the
ambient $\sigma$-algebra and the universal example.

\medskip
\noindent\textbf{Where this leads.}
Measure theory requires a $\sigma$-algebra as its foundational input:
a measure is defined on a $\sigma$-algebra, and measurability is a
membership condition in one. The Boolean algebra structure of $2^X$
connects back to propositional logic. Characteristic functions
reappear as simple functions in Lebesgue integration theory.
\end{tcolorbox}
\vspace{1em}

% =========================================================
% Structural Roadmap
% =========================================================
\subsection*{Structural Roadmap}

Each major topic is organised as:
\begin{center}
\textbf{Definitions $\longrightarrow$ Main Theorems
$\longrightarrow$ Consequences and Structural Insight}
\end{center}

The global progression is:
\begin{enumerate}
  \item Families of sets and closure operations
  \item Finite closure structures: rings of sets and algebras of sets
  \item Countable closure structures: $\sigma$-rings and $\sigma$-algebras
  \item The power set as a Boolean ring and $\mathbb{F}_2$-vector space
  \item Characteristic functions and the identification $2^X \cong \{0,1\}^X$
\end{enumerate}

\begin{remark}[Primary sources]
The treatment follows Kolmogorov and Fomin \textit{Introductory Real Analysis},
Chapter~1, and standard accounts in measure theory (Rudin, Royden).
Characteristic functions and the Boolean ring structure appear in Halmos
\textit{Measure Theory}.
\end{remark}

% =========================================================
% Content
% =========================================================
% =========================================================
% Systems of Sets
% =========================================================

\subsection{Systems of Sets}

% ---------------------------------------------------------
% TOOLKIT
% ---------------------------------------------------------
\begin{tcolorbox}[colback=gray!6, colframe=gray!40, arc=2pt,
  left=6pt, right=6pt, top=4pt, bottom=4pt,
  title={\small\textbf{Systems of Sets --- Quick Reference}},
  fonttitle=\small\bfseries]
\small
\begin{tabular}{l l l}
\toprule
\textbf{Structure} & \textbf{Closure properties} & \textbf{Detail} \\
\midrule
Family of sets   & Any $\mathcal{F} \subseteq 2^X$                                          & \hyperref[def:family-sets]{↓ Def} \\
Closed under $*$ & Inputs in $\mathcal{F}$ implies output in $\mathcal{F}$                   & \hyperref[def:closure-op]{↓ Def} \\
Ring of sets     & Closed under $\cup$ and $\setminus$                                      & \hyperref[def:ring-sets]{↓ Def} \\
Algebra of sets  & Contains $X$; closed under $(\cdot)^c$ and $\cup$                        & \hyperref[def:algebra-sets]{↓ Def} \\
$\sigma$-ring    & Ring closed under countable $\cup$                                        & \hyperref[def:sigma-ring]{↓ Def} \\
$\sigma$-algebra & Algebra closed under countable $\cup$                                     & \hyperref[def:sigma-algebra]{↓ Def} \\
\midrule
\multicolumn{3}{l}{\textit{Key results:}} \\
\midrule
Algebra = ring + $X$ & Equivalent characterisation                                          & \hyperref[prop:algebra-ring]{↓ Prop} \\
$\sigma$-algebra closure & Also closed under countable $\cap$, $\setminus$, finite Boolean ops & \hyperref[prop:sigma-closure]{↓ Prop} \\
\bottomrule
\end{tabular}
\end{tcolorbox}

\vspace{1em}

% ---------------------------------------------------------
% Families and closure
% ---------------------------------------------------------

Let $X$ be a set. A \emph{family of sets on $X$} is a collection
$\mathcal{F} \subseteq 2^X$. The objects of study are not elements of $X$
but collections of subsets, classified by their stability under
set-theoretic operations.

\begin{tcolorbox}[colback=propbox, colframe=propborder, arc=2pt,
  left=6pt, right=6pt, top=4pt, bottom=4pt,
  title={\small\textbf{Definition (Family of Sets)}},
  fonttitle=\small\bfseries]
\label{def:family-sets}
Let $X$ be a set. A \emph{family of sets on $X$} is any collection
$\mathcal{F} \subseteq 2^X$.
\end{tcolorbox}

\begin{tcolorbox}[colback=propbox, colframe=propborder, arc=2pt,
  left=6pt, right=6pt, top=4pt, bottom=4pt,
  title={\small\textbf{Definition (Closure Under an Operation)}},
  fonttitle=\small\bfseries]
\label{def:closure-op}
Let $\mathcal{F} \subseteq 2^X$ and let $*$ be an operation on subsets of
$X$. We say $\mathcal{F}$ is \emph{closed under $*$} if whenever the inputs
to $*$ belong to $\mathcal{F}$, the output also belongs to $\mathcal{F}$.
\end{tcolorbox}

\begin{remark}[Typical operations]
The operations of interest are $\cup$, $\cap$, $(\cdot)^c$, $\setminus$,
and $\triangle$ (symmetric difference). Different combinations of closure
under these operations produce the distinct algebraic structures defined
below.
\end{remark}

% ---------------------------------------------------------
% Finite closure structures
% ---------------------------------------------------------

\begin{tcolorbox}[colback=propbox, colframe=propborder, arc=2pt,
  left=6pt, right=6pt, top=4pt, bottom=4pt,
  title={\small\textbf{Definition (Ring of Sets)}},
  fonttitle=\small\bfseries]
\label{def:ring-sets}
A nonempty collection $\mathcal{R} \subseteq 2^X$ is a \emph{ring of sets}
if it is closed under finite union and set difference:
\begin{enumerate}[label=(\roman*)]
  \item $A, B \in \mathcal{R} \;\Rightarrow\; A \cup B \in \mathcal{R}$,
  \item $A, B \in \mathcal{R} \;\Rightarrow\; A \setminus B \in \mathcal{R}$.
\end{enumerate}
\end{tcolorbox}

\begin{remark}[Intersection and symmetric difference]
A ring of sets is automatically closed under $\cap$ and $\triangle$,
since $A \cap B = A \setminus (A \setminus B)$ and
$A \triangle B = (A \setminus B) \cup (B \setminus A)$.
Under symmetric difference and intersection, a ring of sets is an abelian
group under $\triangle$ and a commutative monoid under $\cap$, making it
a Boolean ring in the algebraic sense.
\end{remark}

\begin{tcolorbox}[colback=propbox, colframe=propborder, arc=2pt,
  left=6pt, right=6pt, top=4pt, bottom=4pt,
  title={\small\textbf{Definition (Algebra of Sets)}},
  fonttitle=\small\bfseries]
\label{def:algebra-sets}
A collection $\mathcal{A} \subseteq 2^X$ is an \emph{algebra of sets}
(also called a \emph{field of sets}) if:
\begin{enumerate}[label=(\roman*)]
  \item $X \in \mathcal{A}$,
  \item $A \in \mathcal{A} \;\Rightarrow\; A^c \in \mathcal{A}$,
  \item $A, B \in \mathcal{A} \;\Rightarrow\; A \cup B \in \mathcal{A}$.
\end{enumerate}
\end{tcolorbox}

\begin{proposition}[Algebra = Ring + $X$]\label{prop:algebra-ring}
An algebra of sets is precisely a ring of sets that contains $X$.
\end{proposition}



\begin{remark}[Finite Boolean stability]
An algebra of sets is closed under all finite Boolean combinations:
finite unions, finite intersections, complements, differences, and
symmetric differences. The qualifier ``finite'' is essential; the next
step is to add closure under countably infinite operations.
\end{remark}

% ---------------------------------------------------------
% Countable closure structures
% ---------------------------------------------------------

\begin{tcolorbox}[colback=propbox, colframe=propborder, arc=2pt,
  left=6pt, right=6pt, top=4pt, bottom=4pt,
  title={\small\textbf{Definition ($\sigma$-Ring)}},
  fonttitle=\small\bfseries]
\label{def:sigma-ring}
A ring of sets $\mathcal{R}$ is a \emph{$\sigma$-ring} if it is also
closed under countable unions:
\[
A_1, A_2, \dots \in \mathcal{R}
\;\Rightarrow\;
\bigcup_{n=1}^{\infty} A_n \in \mathcal{R}.
\]
\end{tcolorbox}

\begin{tcolorbox}[colback=propbox, colframe=propborder, arc=2pt,
  left=6pt, right=6pt, top=4pt, bottom=4pt,
  title={\small\textbf{Definition ($\sigma$-Algebra)}},
  fonttitle=\small\bfseries]
\label{def:sigma-algebra}
A collection $\mathcal{F} \subseteq 2^X$ is a \emph{$\sigma$-algebra} if:
\begin{enumerate}[label=(\roman*)]
  \item $X \in \mathcal{F}$,
  \item $A \in \mathcal{F} \;\Rightarrow\; A^c \in \mathcal{F}$,
  \item $A_1, A_2, \dots \in \mathcal{F} \;\Rightarrow\;
        \displaystyle\bigcup_{n=1}^{\infty} A_n \in \mathcal{F}$.
\end{enumerate}
\end{tcolorbox}

\begin{proposition}[Closure consequences of a $\sigma$-algebra]\label{prop:sigma-closure}
Every $\sigma$-algebra $\mathcal{F}$ is also closed under countable
intersections, set differences, and all finite Boolean combinations.
\end{proposition}



\begin{remark}[Finite vs.\ countable closure]
The passage from an algebra to a $\sigma$-algebra is precisely the step
from finite to countable closure. This is what makes $\sigma$-algebras
compatible with limit processes: if measurable sets are approximated by
sequences of other measurable sets, the limit set remains measurable.
\end{remark}

% ---------------------------------------------------------
% Hierarchy
% ---------------------------------------------------------

\begin{center}
\renewcommand{\arraystretch}{1.3}
\begin{tabular}{lccc}
\toprule
\textbf{Structure} & \textbf{Contains $X$} & \textbf{Closed under $(\cdot)^c$} & \textbf{Closed under $\bigcup$} \\
\midrule
Ring of sets   & No  & No  & Finite only \\
Algebra        & Yes & Yes & Finite only \\
$\sigma$-Ring  & No  & No  & Countable \\
$\sigma$-Algebra & Yes & Yes & Countable \\
\bottomrule
\end{tabular}
\end{center}

\begin{remark}[Reading the hierarchy]
Each row is strictly stronger than the one above it: every algebra is a
ring, and every $\sigma$-algebra is both an algebra and a $\sigma$-ring.
The top of the hierarchy, the $\sigma$-algebra, is the natural domain for
measure theory because it combines the Boolean stability of an algebra with
the limit-compatibility of countable closure.
\end{remark}
% =========================================================
% The Power Set and Characteristic Functions
% =========================================================

\subsection{The Power Set and Characteristic Functions}

% ---------------------------------------------------------
% TOOLKIT
% ---------------------------------------------------------
\begin{tcolorbox}[colback=gray!6, colframe=gray!40, arc=2pt,
  left=6pt, right=6pt, top=4pt, bottom=4pt,
  title={\small\textbf{Power Set and Characteristic Functions --- Quick Reference}},
  fonttitle=\small\bfseries]
\small
\begin{tabular}{l l l}
\toprule
\textbf{Concept} & \textbf{Meaning} & \textbf{Detail} \\
\midrule
Power set $2^X$          & All subsets of $X$; $|2^X| = 2^{|X|}$                        & \hyperref[def:power-set-alg]{↓ Def} \\
Characteristic function  & $\chi_A : X \to \{0,1\}$; encodes membership in $A$          & \hyperref[def:char-fn]{↓ Def} \\
$2^X \cong \{0,1\}^X$   & Bijection $A \mapsto \chi_A$                                  & \hyperref[prop:power-set-bij]{↓ Prop} \\
Boolean ring structure   & $(2^X, \triangle, \cap)$ is a Boolean ring                    & \hyperref[prop:boolean-ring]{↓ Prop} \\
$\mathbb{F}_2$-vector space & $(2^X, \triangle)$ is a vector space over $\mathbb{F}_2$ & \hyperref[prop:f2-vecspace]{↓ Prop} \\
\bottomrule
\end{tabular}
\end{tcolorbox}

\vspace{1em}

% ---------------------------------------------------------
% Power set
% ---------------------------------------------------------

\begin{tcolorbox}[colback=propbox, colframe=propborder, arc=2pt,
  left=6pt, right=6pt, top=4pt, bottom=4pt,
  title={\small\textbf{Definition (Power Set)}},
  fonttitle=\small\bfseries]
\label{def:power-set-alg}
The \emph{power set} of $X$ is
\[
2^X \;:=\; \{\, A \mid A \subseteq X \,\}.
\]
\end{tcolorbox}

\begin{remark}[Cardinality]
The notation $2^X$ reflects the identity $|2^X| = 2^{|X|}$ for finite $X$:
each element of $X$ independently either belongs to a subset (1) or does not
(0), giving $2^{|X|}$ binary choices in total. For infinite $X$, the same
exponential notation is retained and $2^X$ denotes the cardinal power.
\end{remark}

% ---------------------------------------------------------
% Characteristic functions
% ---------------------------------------------------------

\begin{tcolorbox}[colback=propbox, colframe=propborder, arc=2pt,
  left=6pt, right=6pt, top=4pt, bottom=4pt,
  title={\small\textbf{Definition (Characteristic Function)}},
  fonttitle=\small\bfseries]
\label{def:char-fn}
Let $A \subseteq X$. The \emph{characteristic function} (or
\emph{indicator function}) of $A$ is the function
$\chi_A : X \to \{0,1\}$ defined by
\[
\chi_A(x) \;:=\;
\begin{cases}
1 & \text{if } x \in A, \\
0 & \text{if } x \notin A.
\end{cases}
\]
\end{tcolorbox}

\begin{proposition}[Identification of subsets with functions]\label{prop:power-set-bij}
The map $A \mapsto \chi_A$ is a bijection
\[
2^X \;\xrightarrow{\;\sim\;}\; \{0,1\}^X.
\]
Consequently, subsets of $X$ and functions from $X$ to $\{0,1\}$ are
in exact correspondence.
\end{proposition}

\begin{proof}
\emph{Injective:} If $\chi_A = \chi_B$, then for every $x \in X$,
$\chi_A(x) = \chi_B(x)$, which means $x \in A \iff x \in B$, so $A = B$.

\emph{Surjective:} Given any $f : X \to \{0,1\}$, the set
$A := f^{-1}(\{1\}) = \{x \in X : f(x) = 1\}$ satisfies $\chi_A = f$.
\end{proof}

\begin{remark}[Functional perspective]
This bijection means we can study subsets of $X$ as functions, and
vice versa. The set-theoretic operations then correspond to pointwise
arithmetic: $\chi_{A \cup B} = \max(\chi_A, \chi_B)$,
$\chi_{A \cap B} = \min(\chi_A, \chi_B) = \chi_A \cdot \chi_B$, and
$\chi_{A^c} = 1 - \chi_A$. The symmetric difference corresponds to
addition modulo 2: $\chi_{A \triangle B} = \chi_A + \chi_B \pmod{2}$.
\end{remark}

% ---------------------------------------------------------
% Algebraic structure
% ---------------------------------------------------------

\begin{proposition}[Boolean ring structure on $2^X$]\label{prop:boolean-ring}
The power set $(2^X, \triangle, \cap)$, with symmetric difference as
addition and intersection as multiplication, is a \emph{Boolean ring}:
a commutative ring in which every element is idempotent ($A \cap A = A$).
\end{proposition}

\begin{proof}
One verifies the ring axioms directly. Symmetric difference is associative
and commutative, $\varnothing$ is the additive identity, and every element
is its own additive inverse ($A \triangle A = \varnothing$). Intersection
distributes over $\triangle$ and is associative and commutative with
identity $X$. Idempotency $A \cap A = A$ holds by definition of intersection.
\end{proof}

\begin{proposition}[$\mathbb{F}_2$-vector space structure on $2^X$]\label{prop:f2-vecspace}
The power set $(2^X, \triangle)$, with scalar multiplication defined by
$0 \cdot A = \varnothing$ and $1 \cdot A = A$, is a vector space over
the two-element field $\mathbb{F}_2 = \{0,1\}$.
\end{proposition}

\begin{remark}[Consequence]
This vector space structure means any family of subsets closed under
$\triangle$ and containing $\varnothing$ is a subspace of $2^X$ over
$\mathbb{F}_2$. Characteristic functions are then the natural coordinate
vectors, and the Boolean operations become linear-algebraic operations.
This bridge between set theory and linear algebra reappears whenever
one needs to decompose or represent families of measurable sets.
\end{remark}

\begin{remark}[Transition]
The full power set $2^X$ is itself a $\sigma$-algebra --- the largest one
on $X$. Every $\sigma$-algebra on $X$ is a sub-$\sigma$-algebra of $2^X$,
just as every subspace of a vector space sits inside the ambient space.
This ambient structure makes $2^X$ the natural starting point for
constructing the Borel $\sigma$-algebra in measure theory.
\end{remark}

% Future sections (uncomment as content expands):
% \input{volume-iii/algebra/set-algebras/notes/notes-generated-systems}
% \input{volume-iii/algebra/set-algebras/notes/notes-auxiliary-systems}