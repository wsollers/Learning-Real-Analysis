% =========================================================
% Breadcrumb
% =========================================================
\begin{tcolorbox}[
  colback=gray!6,
  colframe=gray!40,
  arc=2pt,
  left=8pt, right=8pt, top=6pt, bottom=6pt,
  title={\small\textbf{Where You Are in the Journey}},
  fonttitle=\small\bfseries
]
\begin{center}
\small
Propositional Logic
$\;\to\;$ Sets \& Functions
$\;\to\;$ $\mathbb{Z}$, $\mathbb{R}$
$\;\to\;$ Algebraic Structures
$\;\to\;$ Linear Algebra
$\;\to\;$ \textbf{Abstract Algebra}
$\;\to\;$ Algebraic Geometry
$\;\to\;$ $\cdots$
\end{center}

\medskip
\noindent\textbf{How we got here.}
Algebraic structures introduced groups, rings, and fields via their axioms.
Abstract algebra deepens this study: it asks what properties follow
from the axioms alone, develops the theory of homomorphisms and
isomorphisms, and classifies structures up to structural equivalence.

\medskip
\noindent\textbf{What this chapter builds.}
We develop group theory (subgroups, cosets, Lagrange's theorem,
normal subgroups, quotient groups, homomorphisms, isomorphisms,
and the isomorphism theorems), ring theory (ideals, quotient rings,
polynomial rings, and factorisation), and field extensions.

\medskip
\noindent\textbf{Where this leads.}
Galois theory uses group theory to answer questions about polynomial
equations. Algebraic geometry studies polynomial ideals and their
geometric solutions. Number theory uses ring-theoretic tools throughout.
\end{tcolorbox}
\vspace{1em}

% =========================================================
% Abstract Algebra — Structural Roadmap
% Driver Text: Gallian, Contemporary Abstract Algebra (7th ed.)
% File: abstract-algebra/abstract-algebra.tex
% =========================================================

\subsection{Abstract Algebra}

% =========================================================
% Structural Roadmap
% =========================================================

\subsubsection*{Structural Roadmap}

The development of abstract algebra in this project follows
the definition--theorem--structure architecture used
throughout the analysis volumes.

The primary driver is \textit{Contemporary Abstract Algebra}
by Joseph A. Gallian (7th ed.).

The emphasis is on algebraic structure as an abstraction of
symmetry, arithmetic, and linear operations.
Each major topic is organized as:

\begin{center}
\textbf{Definitions $\longrightarrow$ Structural Theorems $\longrightarrow$ Classification and Applications}
\end{center}

The global progression follows Gallian in five structural stages:

\begin{enumerate}

% =========================================================
% I. Integers and Equivalence
% =========================================================

\item \textbf{Foundations: Integers and Equivalence Relations}

\begin{enumerate}
    \item Modular arithmetic
    \item Equivalence relations and partitions
    \item Functions and mappings
\end{enumerate}

\vspace{0.5em}
\textbf{Structural Theme:}
Congruence and equivalence encode algebraic structure via partitions.
This stage formalizes symmetry at the level of arithmetic.

% =========================================================
% II. Group Theory
% =========================================================

\item \textbf{Groups}

\begin{enumerate}
    \item Definition and basic properties
    \item Subgroups
    \item Cyclic groups
    \item Permutation groups
    \item Isomorphisms
    \item Cosets and Lagrange’s Theorem
    \item Normal subgroups and factor groups
    \item Homomorphisms
    \item Fundamental Theorem of Finite Abelian Groups
\end{enumerate}

\vspace{0.5em}
\textbf{Structural Theme:}
Groups formalize symmetry.
Cosets measure deviation from subgroup structure.
Normality permits quotient construction.
Homomorphisms reveal structural preservation.
Classification emerges in the finite abelian case.

% =========================================================
% III. Ring Theory
% =========================================================

\item \textbf{Rings}

\begin{enumerate}
    \item Definition and examples
    \item Integral domains
    \item Ideals and factor rings
    \item Ring homomorphisms
    \item Polynomial rings
    \item Unique factorization domains
    \item Euclidean domains
\end{enumerate}

\vspace{0.5em}
\textbf{Structural Theme:}
Rings generalize arithmetic.
Ideals control structure via quotients.
Factorization encodes algebraic rigidity.
Euclidean domains permit algorithmic structure.

% =========================================================
% IV. Field Theory
% =========================================================

\item \textbf{Fields and Extensions}

\begin{enumerate}
    \item Vector spaces
    \item Field extensions
    \item Algebraic extensions
    \item Finite fields
    \item Geometric constructions
\end{enumerate}

\vspace{0.5em}
\textbf{Structural Theme:}
Fields unify arithmetic and linear algebra.
Extensions enlarge solvability.
Finite fields exhibit deep classification.
Constructibility links algebra and geometry.

% =========================================================
% V. Advanced Topics
% =========================================================

\item \textbf{Special Topics}

\begin{enumerate}
    \item Sylow Theorems
    \item Finite simple groups
    \item Generators and relations
    \item Symmetry groups
    \item Group actions and counting (Burnside)
    \item Coding theory
    \item Galois theory
    \item Cyclotomic extensions
\end{enumerate}

\vspace{0.5em}
\textbf{Structural Theme:}
Group actions connect algebra to combinatorics.
Sylow theory controls finite structure.
Galois theory connects symmetry and solvability.
Cyclotomic extensions bridge algebra and number theory.

\end{enumerate}

% =========================================================
% Structural Remarks
% =========================================================

\vspace{1em}

\begin{remark}
Abstract algebra studies algebraic structures defined by operations
subject to axioms. Each structure (group, ring, field) is a
package consisting of:
\[
\text{Set} + \text{Operations} + \text{Axioms}.
\]
\end{remark}

\begin{remark}
The central construction principle is:
\[
\textbf{Substructure} \longrightarrow \textbf{Quotient Structure}.
\]
Normal subgroups and ideals make quotient constructions possible.
\end{remark}

\begin{remark}[Structural Position]
Abstract algebra in this project builds on:

\begin{itemize}
    \item Set theory and logic (Phase 0)
    \item Linear algebra (vector space structure)
    \item Real analysis (structural proof discipline)
\end{itemize}

Group theory forms the conceptual backbone.
Ring theory extends arithmetic.
Field theory culminates in symmetry of roots (Galois theory).
\end{remark}

% =========================================================
% Input Files (To Be Developed)
% =========================================================

% -----------------------
% Foundations
% -----------------------
% =========================================================
% NOTES TEMPLATE (Definitions → Theorems/Proofs → Consequences)
% File: notes-integers.tex
% =========================================================

\subsection{Integers}
\label{sec:integers}  


% ---------------------------------------------------------
\subsubsection{Basic Definitions and Theorems}

\begin{axiom}[Well Ordering Principle]
Every nonempty set of positive integers contains a smallest member.
\[
\forall A\,
\Bigl(
\bigl( A \subseteq \mathbb{N}
\land \exists a\,(a \in A) \bigr)
\Rightarrow
\exists m\,
\bigl(
m \in A
\land
\forall a\,(a \in A \Rightarrow m \le a)
\bigr)
\Bigr).
\]
\end{axiom}

\begin{remark}
The well-ordering principle asserts that every nonempty subset, $A$,
of $\mathbb{N}$ possesses a least element.
Its quantifier structure is:
\[
\forall A\;
(\exists a \in A \Rightarrow \exists m \in A\;\forall a \in A).
\]
This principle is equivalent to the principle of mathematical induction.
\end{remark}

\begin{remark}[Axiomatic Status of the Well-Ordering Principle]
The Well-Ordering Principle is not a theorem of elementary arithmetic.
It cannot be proved using only the usual algebraic laws of addition
and multiplication on $\mathbb{N}$.

Rather, it is taken as an axiom describing the order structure of
the natural numbers.  In standard foundations, it is equivalent
to the Principle of Mathematical Induction and the Least Element
Principle.  Thus one must assume one of these principles in order
to derive the others.

In particular, arithmetic identities alone do not imply that every
nonempty subset of $\mathbb{N}$ has a smallest element; this property
is part of the defining structure of the natural numbers.
\end{remark}

\begin{remark}[Intuition]
Intuitively: the Well-Ordering Principle says you can always find a
``smallest'' object in any nonempty collection of positive integers.
This gives us a foothold — once we have a minimal element,
we can derive divisibility and gcd properties by contradiction.
\end{remark}

\begin{definition}
Let $a, b \in \mathbb{Z}$ with $b \neq 0$. In the expression $\dfrac{a}{b}$,
we call $a$ the \textit{numerator} and $b$ the \textit{denominator}.
\end{definition}

\begin{definition}
Let $a, b \in \mathbb{Z}$. We say $a$ is a \textit{multiple} of $b$ if
$a = bk$ for some $k \in \mathbb{Z}$.
\end{definition}

\begin{definition}[Divisibility]
Let $a, b \in \mathbb{Z}$ with $b \neq 0$.
We say $b$ \emph{divides} $a$, written $b \mid a$, if there exists
an integer $k$ such that
\[
a = bk.
\]
If $b$ divides $a$, we call $b$ a \emph{divisor} (or \emph{factor}) of $a$,
and $a$ a \emph{multiple} of $b$.
If $b$ does not divide $a$, we write $b \nmid a$.
\end{definition}

\begin{remark}[Unpacking Divisibility]
The statement $b \mid a$ is a claim about the \emph{existence} of an integer
$k$ such that $a = bk$. It says nothing about a remainder --- divisibility means
the remainder is exactly zero.

For example:
\begin{itemize}
\item $3 \mid 12$ because $12 = 3 \cdot 4$.
\item $5 \nmid 13$ because $13 = 5 \cdot 2 + 3$ and the remainder $3 \neq 0$.
\end{itemize}

This is the foundation for gcd, lcm, and Bézout's Identity.
\end{remark}

\begin{remark}[Intuition]
Intuitively: $b \mid a$ means $b$ fits into $a$ a whole number of times,
with nothing left over. Division is exact.
\end{remark}

\begin{lemma}[Divisibility of Linear Combinations]
\label{lem:div-linear-combo}
Let $a, b \in \mathbb{Z}$. If $t \mid a$ and $t \mid b$, then $t$ divides
every integer linear combination of $a$ and $b$. That is,
\[
t \mid (ua + vb)
\quad \text{for all } u, v \in \mathbb{Z}.
\]
\end{lemma}

\begin{proof}
Since $t \mid a$ and $t \mid b$, there exist integers $x, y$ such that
\[
a = tx, \qquad b = ty.
\]
Then for any $u, v \in \mathbb{Z}$,
\[
ua + vb = u(tx) + v(ty) = t(ux + vy).
\]
Since $ux + vy \in \mathbb{Z}$, we have $t \mid (ua + vb)$.
\end{proof}

\begin{remark}
This lemma is used constantly in number theory. Its most important
instance is: since $\gcd(a,b)$ divides both $a$ and $b$, it divides
every linear combination $ua + vb$. In particular it divides
$d = as + bt$ from Bézout's Identity, which is how we verify
the greatest-ness of the gcd.
\end{remark}

\begin{remark}[Intuition]
Intuitively: if $t$ fits into both $a$ and $b$ individually with no remainder,
then $t$ must fit into any way you scale and combine them --- the remainders
simply cannot accumulate.
\end{remark}

\begin{theorem}[Division Algorithm]
Let $a$ and $b$ be integers with $b > 0$. Then there exist unique integers
$q$ and $r$ such that
\[
a = bq + r,
\qquad
0 \le r < b.
\]
\end{theorem}

\begin{remark}
In the Division Algorithm, $a = bq + r$, we say $a$ is the \textit{dividend}
and $b$ is the \textit{divisor}. This corresponds to the division $a / b$,
yielding quotient $q$ and remainder $r$. When $r = 0$, we write $b \mid a$
(``$b$ divides $a$''), meaning $a = bq$ for some integer $q$.
\end{remark}

\begin{proof}
We begin with the existence portion of the theorem. Consider the set
\[
S = \{\, a - bk \mid k \in \mathbb{Z},\; a - bk \ge 0 \,\}.
\]

If $0 \in S$, then $b$ divides $a$ and we obtain the desired result with
$q = a/b$ and $r = 0$.

Now assume $0 \notin S$. Since $S$ is nonempty
(if $a > 0$, then $a - b\cdot 0 = a \in S$;
if $a < 0$, then choosing $k$ sufficiently negative makes $a - bk \ge 0$),
we may apply the Well-Ordering Principle to conclude that $S$ has a smallest
member, say
\[
r = a - bq.
\]
Then
\[
a = bq + r
\quad \text{and} \quad r \ge 0.
\]
It remains to prove that $r < b$.

Suppose, to the contrary, that $r \ge b$. Then
\[
a - b(q+1)
=
a - bq - b
=
r - b
\ge 0,
\]
so that $a - b(q+1) \in S$. But
\[
a - b(q+1) < a - bq,
\]
contradicting the minimality of $r = a - bq$. Hence $r < b$.

\medskip

To establish uniqueness, suppose there exist integers $q, q', r, r'$
such that
\[
a = bq + r, \qquad 0 \le r < b,
\]
and
\[
a = bq' + r', \qquad 0 \le r' < b.
\]

Subtracting the two equations gives
\[
b(q - q') = r' - r.
\]

Thus $b$ divides $r' - r$. Without loss of generality, assume $r' \ge r$.
Then
\[
0 \le r' - r < b.
\]

The only multiple of $b$ in this interval is $0$. Therefore,
\[
r' - r = 0,
\]
so $r' = r$ and hence $q = q'$.

This completes the proof.
\end{proof}
\cite{GallianCAA7}

\begin{definition}[Quotient and Remainder]
Let $a,b \in \mathbb{Z}$ with $b>0$.
If integers $q$ and $r$ satisfy
\[
a = bq + r,
\qquad
0 \le r < b,
\]
then:
\begin{itemize}
\item $q$ is called the \emph{quotient} of $a$ divided by $b$,
\item $r$ is called the \emph{remainder} of $a$ divided by $b$.
\end{itemize}
\end{definition}

\begin{remark}
The conditions $0 \le r < b$ ensure uniqueness.
Without this bound on $r$, the representation
\[
a = bq + r
\]
would not be unique, since one could replace $(q,r)$ with
\[
(q+1, r-b), \quad (q-1, r+b), \quad \text{etc.}
\]
The remainder condition selects exactly one representative.
\end{remark}

\begin{remark}[Interpretation of quotient and remainder]
The quotient measures how many full copies of $b$ fit into $a$,
while the remainder measures the leftover part that is strictly
smaller than $b$.
\end{remark}

\begin{definition}
An integer $p > 1$ is called \textit{prime} if its only positive divisors
are $1$ and $p$.
\end{definition}

\begin{definition}[Greatest Common Divisor]
Let $a$ and $b$ be integers, not both zero. The greatest common divisor
of $a$ and $b$ is the largest of all common divisors of $a$ and $b$.

A positive integer $d$ is called the \emph{greatest common divisor}
of $a$ and $b$ if:
\begin{enumerate}
\item $d \mid a$ and $d \mid b$ (that is, $d$ divides both $a$ and $b$), and
\item Whenever $c$ is an integer such that $c \mid a$ and $c \mid b$,
then $c \mid d$.
\end{enumerate}

The greatest common divisor of $a$ and $b$ is denoted
\[
\gcd(a,b).
\]
\end{definition}

\begin{remark}[Intuition]
Intuitively: the gcd is the largest ruler that measures both $a$ and $b$
exactly, with no remainder. Any other common divisor must itself be measured
exactly by the gcd.
\end{remark}

\begin{definition}[Relatively Prime Integers]
Two integers $a$ and $b$ are said to be \emph{relatively prime}
(or \emph{coprime}) if
\[
\gcd(a,b) = 1.
\]
\end{definition}

\begin{remark}
If $\gcd(a,b) = 1$, then the only positive integer dividing both $a$ and $b$
is $1$. In this case, $a$ and $b$ share no common prime factors.
\end{remark}

\begin{theorem}[Bézout's Identity]
Let $a$ and $b$ be integers, not both zero.
Then there exist integers $s$ and $t$ such that
\[
\gcd(a,b) = as + bt.
\]

Moreover, $\gcd(a,b)$ is the smallest positive integer
of the form
\[
as + bt,
\quad s,t \in \mathbb{Z}.
\]
\end{theorem}

\begin{remark}
The set
\[
\{\, as + bt : s,t \in \mathbb{Z} \,\}
\]
is called the set of \emph{integer linear combinations} of $a$ and $b$.
Bézout's Identity asserts that the greatest common divisor
is the minimal positive element of this set.
\end{remark}

\begin{proof}
Consider the set
\[
S = \{\, am + bn \mid m,n \in \mathbb{Z},\; am + bn > 0 \,\}.
\]

Since $S$ is nonempty (if some choice of $m,n$ gives $am+bn<0$,
replace $m,n$ with $-m,-n$), the Well-Ordering Principle
implies that $S$ has a smallest element.
Denote this smallest element by
\[
d = as + bt
\]
for some integers $s,t$.

We claim that $d = \gcd(a,b)$.

\medskip
\noindent
\textbf{Step 1: $d$ divides $a$ and $b$.}

Apply the Division Algorithm to write
\[
a = dq + r,
\qquad
0 \le r < d.
\]

If $r > 0$, then
\[
r = a - dq
  = a - (as + bt)q
  = a - asq - btq
  = a(1 - sq) + b(-tq).
\]
Thus $r$ is of the form $am + bn$, and since $r>0$,
we have $r \in S$. But $r < d$, contradicting
the minimality of $d$. Hence $r = 0$, so $d \mid a$.

By symmetry, the same argument shows that $d \mid b$.
Thus $d$ is a common divisor of $a$ and $b$.

\medskip
\noindent
\textbf{Step 2: $d$ is the greatest common divisor.}

Let $d'$ be any other common divisor of $a$ and $b$.
Then there exist integers $h,k$ such that
\[
a = d'h,
\qquad
b = d'k.
\]

Substituting into $d = as + bt$ gives
\[
d = (d'h)s + (d'k)t
  = d'(hs + kt).
\]

Hence $d'$ divides $d$.
Therefore, every common divisor of $a$ and $b$
divides $d$, and so $d$ is the greatest common divisor.

\medskip

Thus,
\[
d = \gcd(a,b).
\]
\end{proof}

\begin{remark}[Structural Map of the Proof]
This proof has three distinct logical engines.

\begin{enumerate}
\item \textbf{Well-Ordering Principle (existence of a minimal positive combination).}
We define
\[
S=\{\,am+bn : m,n\in\mathbb{Z},\; am+bn>0\,\}.
\]
The Well-Ordering Principle is invoked \emph{exactly here} to guarantee that
$S$ has a smallest element $d$. This is the sole point where order-theoretic
structure on $\mathbb{N}$ enters.

\item \textbf{Division Algorithm (turning minimality into divisibility).}
After choosing the minimal $d\in S$, we apply the Division Algorithm to write
\[
a = dq + r,\qquad 0 \le r < d.
\]
The key move is: if $r>0$, then $r$ can be rewritten as another positive
integer linear combination of $a$ and $b$, hence $r\in S$, contradicting the
minimality of $d$. Therefore $r=0$ and $d\mid a$ (and similarly $d\mid b$).
So the Division Algorithm is used \emph{exactly here} to convert the order
statement ``$d$ is smallest'' into the algebraic statement ``$d$ divides.''

\item \textbf{Greatest-ness (universal property among common divisors).}
Finally, for any common divisor $d'$ of $a$ and $b$, writing $a=d'h$ and $b=d'k$
shows
\[
d = as+bt = d'(hs+kt),
\]
so $d' \mid d$. This establishes that $d$ is the \emph{greatest} common divisor
in the divisibility order.
\end{enumerate}

\medskip
\noindent
\textbf{Prototype Euclidean-domain argument.}
The overall pattern is the Euclidean-domain template:
\[
\text{(nonempty set of ``sizes'')} \xRightarrow{\text{well-ordering}}
\text{minimal element} \xRightarrow{\text{division algorithm}}
\text{divisibility + gcd characterization}.
\]
In $\mathbb{Z}$, the ``size'' is the usual order on positive integers; in a
general Euclidean domain, the ``size'' is a Euclidean function
$\delta:R\setminus\{0\}\to\mathbb{N}$, and the same minimality-plus-division
strategy produces gcds and Bézout-type identities.
\end{remark}

\begin{corollary}
Let $a$ and $b$ be relatively prime integers.
Then there exist integers $s$ and $t$ such that
\[
as + bt = 1.
\]
\end{corollary}

\begin{proof}
If $a$ and $b$ are relatively prime, then
\[
\gcd(a,b) = 1.
\]
By Bézout's Identity, there exist integers $s$ and $t$ such that
\[
\gcd(a,b) = as + bt.
\]
Substituting $\gcd(a,b)=1$ gives
\[
as + bt = 1.
\]
\end{proof}

\begin{remark}[Bézout in Both Directions]
Bézout's Identity is often applied in two distinct directions,
and it is important to keep them separate.

\medskip
\noindent
\textbf{Forward direction.}
If $\gcd(a,b) = d$, then there exist integers $s, t$ such that
\[
as + bt = d.
\]
This is the content of the theorem itself.

\medskip
\noindent
\textbf{Reverse direction.}
If there exist integers $s, t$ such that $as + bt = 1$, then
$\gcd(a,b) = 1$.

\medskip
\noindent
\emph{Proof of reverse direction.}
Let $d = \gcd(a,b)$. By the forward direction, $d \mid a$ and $d \mid b$,
so by Lemma~\ref{lem:div-linear-combo}, $d \mid (as + bt) = 1$.
Since $d > 0$ and $d \mid 1$, we conclude $d = 1$.

\medskip
More generally, if $as + bt = c$ for some integers $s, t$, then
$\gcd(a,b) \mid c$, since $\gcd(a,b)$ divides any linear combination
of $a$ and $b$.
\end{remark}

\begin{remark}[Intuition]
Intuitively: a linear combination equaling $1$ leaves no room for any
common factor greater than $1$ --- if $d > 1$ divided both $a$ and $b$,
it would divide the combination too, but nothing greater than $1$ divides $1$.
\end{remark}

\begin{theorem}[Equivalent Characterizations of Relatively Prime Integers]
For integers $a$ and $b$, the following are equivalent:

\begin{enumerate}
\item $\gcd(a,b)=1$.
\item There exist integers $s,t$ such that $as+bt=1$.
\item $a$ and $b$ share no common prime factor.
\end{enumerate}
\end{theorem}

\begin{theorem}[Euclid's Lemma]
Let $p$ be a prime integer.
If
\[
p \mid ab,
\]
then
\[
p \mid a \quad \text{or} \quad p \mid b.
\]
\end{theorem}

\begin{proof}
Suppose $p$ is prime and $p \mid ab$.
If $p \mid a$, we are done.
So assume that $p \nmid a$.
We must show that $p \mid b$.

Because $p$ is prime and $p \nmid a$, it follows that
\[
\gcd(p,a) = 1.
\]
By Bézout's Identity, there exist integers $s$ and $t$ such that
\[
1 = as + pt.
\]

Multiplying both sides by $b$ gives
\[
b = abs + ptb.
\]

Since $p \mid ab$, the term $abs$ is divisible by $p$.
The term $ptb$ is clearly divisible by $p$.
Therefore the right-hand side is divisible by $p$,
and hence $p \mid b$.

Thus, if $p \mid ab$, then $p \mid a$ or $p \mid b$.
\end{proof}

\begin{remark}[Intuition]
Intuitively: primes cannot be fooled by multiplication --- if a prime
divides a product, it must have divided one of the factors going in.
This rigidity is what makes unique prime factorization possible.
\end{remark}

\begin{definition}[Least Common Multiple]
Let $a$ and $b$ be nonzero integers.
The \emph{least common multiple} of $a$ and $b$
is the unique positive integer $m$ such that:

\begin{enumerate}
\item $a \mid m$ and $b \mid m$ (so $m$ is a common multiple), and
\item If $c$ is any positive integer with $a \mid c$ and $b \mid c$,
then $m \mid c$.
\end{enumerate}

The least common multiple of $a$ and $b$ is denoted
\[
\operatorname{lcm}(a,b).
\]
\end{definition}

\begin{remark}[Intuition]
Intuitively: the lcm is the smallest number that both $a$ and $b$ divide
into evenly. Any other common multiple must itself be a multiple of the lcm.
\end{remark}

\begin{remark}[Relation with the Greatest Common Divisor]
For nonzero integers $a$ and $b$,
\[
\gcd(a,b)\,\operatorname{lcm}(a,b) = |ab|.
\]

Thus the greatest common divisor measures the
\emph{shared divisibility} of $a$ and $b$,
while the least common multiple measures their
\emph{combined multiplicative content}.

In prime factorization terms,
$\gcd$ takes the minimum exponent of each prime,
whereas $\operatorname{lcm}$ takes the maximum exponent.
\end{remark}

\begin{theorem}[Fundamental Theorem of Arithmetic]
Every integer $n > 1$ is either prime or can be written as a product of primes.
Moreover, this factorization is unique up to the order of the factors.

That is, if
\[
n = p_1 p_2 \cdots p_r
\quad\text{and}\quad
n = q_1 q_2 \cdots q_s,
\]
where each $p_i$ and $q_j$ is prime, then
\[
r = s,
\]
and after a reordering of the $q_j$,
\[
p_i = q_i \quad \text{for all } i.
\]
\end{theorem}

\begin{proof}
\textbf{(Existence)}

We prove by strong induction on $n > 1$ that $n$ is a prime or a product
of primes.

\medskip
\emph{Base case:}
For $n=2$, the statement holds because $2$ is prime.

\medskip
\emph{Inductive step:}
Assume every integer $m$ with $2 \le m < n$ is prime or a product of primes.
If $n$ is prime, we are done.

If $n$ is composite, then
\[
n = ab
\quad\text{with}\quad
1 < a < n, \; 1 < b < n.
\]
By the induction hypothesis, both $a$ and $b$ are products of primes.
Therefore $n = ab$ is also a product of primes.

Thus every integer greater than $1$ is prime or a product of primes.

\medskip
\textbf{(Uniqueness)}

Suppose
\[
n = p_1 p_2 \cdots p_r
\quad\text{and}\quad
n = q_1 q_2 \cdots q_s,
\]
where all $p_i$ and $q_j$ are primes.

Since $p_1 \mid n$, we have
\[
p_1 \mid q_1 q_2 \cdots q_s.
\]
By Euclid's Lemma, $p_1$ divides some $q_j$.
Because $q_j$ is prime, it follows that
\[
p_1 = q_j.
\]

After reordering, we may assume $p_1 = q_1$.
Canceling this common factor gives
\[
p_2 \cdots p_r = q_2 \cdots q_s.
\]

Repeating this argument inductively, we conclude that
\[
r = s
\quad\text{and}\quad
p_i = q_i \text{ for all } i
\]
after a suitable reordering.

Therefore, the prime factorization of $n$ is unique up to order.
\end{proof}

\begin{theorem}[Structural Relation Between $\gcd$ and $\operatorname{lcm}$]
Let $a$ and $b$ be nonzero integers. Then
\[
\gcd(a,b)\,\operatorname{lcm}(a,b) = |ab|.
\]
\end{theorem}

\begin{proof}
Let
\[
d = \gcd(a,b).
\]
Then there exist integers $a_1$ and $b_1$ such that
\[
a = d a_1, \qquad b = d b_1,
\]
and $\gcd(a_1,b_1)=1$.

Now
\[
ab = d^2 a_1 b_1.
\]

We claim that $\operatorname{lcm}(a,b) = d a_1 b_1$.
To see this, note that $a \mid d a_1 b_1$ (since $d a_1 b_1 = a \cdot b_1$)
and $b \mid d a_1 b_1$ (since $d a_1 b_1 = b \cdot a_1$).
If $c$ is any common multiple of $a$ and $b$, write $c = a u = b v$
for integers $u, v$. Then $d a_1 u = d b_1 v$, so $a_1 u = b_1 v$.
Since $\gcd(a_1, b_1) = 1$, we have $b_1 \mid u$, say $u = b_1 w$,
giving $c = a u = d a_1 b_1 w$. Hence $d a_1 b_1 \mid c$, confirming
that $d a_1 b_1$ is the least common multiple.

Therefore,
\[
\gcd(a,b)\,\operatorname{lcm}(a,b)
= d \cdot (d a_1 b_1)
= d^2 a_1 b_1
= |ab|.
\]

Taking absolute value accounts for possible signs of $a$ and $b$.
\end{proof}

\begin{remark}[Intuition]
Intuitively: $a$ and $b$ together contain a fixed amount of prime material.
The gcd captures what they share; the lcm captures everything between them.
Together they account for the full prime content of $ab$.
\end{remark}

% ---------------------------------------------------------
\subsubsection{Consequences and Logical Implications}
% ---------------------------------------------------------

\begin{remark}[Logical Dependency Chain]
The development of the integers in this section follows the chain
\[
\begin{aligned}
\text{Well-Ordering Principle}
&\;\Rightarrow\;
\text{Division Algorithm}
\;\Rightarrow\;
\text{Bézout's Identity}
\\[4pt]
&\;\Rightarrow\;
\text{Euclid's Lemma}
\;\Rightarrow\;
\text{Fundamental Theorem of Arithmetic}.
\end{aligned}
\]

Thus the order structure of $\mathbb{N}$ ultimately governs
prime factorization in $\mathbb{Z}$.
\end{remark}

\begin{remark}[Equivalences]
The following principles are logically equivalent:

\begin{itemize}
\item The Well-Ordering Principle.
\item The Principle of Mathematical Induction.
\item The Least Element Principle.
\end{itemize}

Each encodes the same structural property:
the integers admit no infinite strictly descending chains.
\end{remark}

\begin{remark}[Divisibility Structure]
Bézout's Identity upgrades divisibility into a linear-combination statement:
\[
\gcd(a,b)
=
\min \{\, as+bt>0 : s,t\in\mathbb{Z} \,\}.
\]

Thus the gcd is characterized by a universal property:
it is the greatest element (under divisibility)
among common divisors of $a$ and $b$.
\end{remark}

\begin{remark}[Prime Structure]
Euclid's Lemma implies that primes behave rigidly under multiplication:
\[
p \mid ab \Rightarrow p \mid a \text{ or } p \mid b.
\]

This rigidity is exactly what makes unique prime factorization possible.
Without Euclid's Lemma, uniqueness would fail.
\end{remark}

\begin{remark}[Arithmetic Decomposition]
The Fundamental Theorem of Arithmetic shows that
every integer $n>1$ decomposes uniquely into prime powers.
Thus $\mathbb{Z}$ is a \emph{Unique Factorization Domain (UFD)}.
\end{remark}

\begin{remark}[Duality of $\gcd$ and $\operatorname{lcm}$]
The identity
\[
\gcd(a,b)\,\operatorname{lcm}(a,b)=|ab|
\]
reveals a structural duality:

\begin{itemize}
\item $\gcd$ measures shared prime factors (intersection).
\item $\operatorname{lcm}$ measures total prime coverage (union).
\end{itemize}

Together they partition the prime-power structure of $ab$.
\end{remark}

\begin{remark}[Structural Position]
This section establishes that:

\begin{itemize}
\item $\mathbb{Z}$ has a well-ordered positive part.
\item Divisibility can be analyzed via minimal elements.
\item Primes control multiplicative structure.
\item Linear combinations control common divisors.
\end{itemize}

These ideas generalize later to:

\begin{itemize}
\item Euclidean domains,
\item Principal ideal domains,
\item Unique factorization domains.
\end{itemize}
\end{remark}
% =========================================================
% NOTES TEMPLATE (Definitions → Theorems/Proofs → Consequences)
% File: notes-modular-arithmetic.tex
% =========================================================

\subsection{Modular Arithmetic}
\label{sec:modular-arithmetic}

% ---------------------------------------------------------
\subsubsection{Basic Definitions and Theorems}

\begin{definition}[Remainder (Modular Reduction)]
Let $a \in \mathbb{Z}$ and $n \in \mathbb{Z}^+$. We say $a \bmod n$ is the unique integer $r$ satisfying
\[
a = qn + r, \quad 0 \leq r < n,
\]
for some integer $q$. The value of $a \bmod n$ is $r$. We call $r$ the \textit{remainder}, $q$ the \textit{quotient}, and $n$ the \textit{modulus} of $a$ upon division by $n$.
\end{definition}

\begin{remark}[Computing $a \bmod n$]
To compute $a \bmod n$, apply the Division Algorithm:
divide $a$ by $n$, and the remainder $r$ is the value of $a \bmod n$.

\medskip
\noindent\textbf{Examples.}
\begin{itemize}
\item $17 \bmod 5 = 2$, \quad since $17 = 3 \cdot 5 + 2$.
\item $20 \bmod 4 = 0$, \quad since $20 = 5 \cdot 4 + 0$.
\item $3 \bmod 7 = 3$, \quad since $3 = 0 \cdot 7 + 3$.
\item $-1 \bmod 5 = 4$, \quad since $-1 = (-1) \cdot 5 + 4$ and $0 \leq 4 < 5$.
\end{itemize}

Note the last example carefully: for negative integers,
the remainder is still required to satisfy $0 \le r < n$,
so the quotient $q$ may be negative.

\medskip
Equivalently, using the floor function,
\[
a \bmod n = a - n\left\lfloor \frac{a}{n} \right\rfloor,
\]
where $\lfloor x \rfloor$ denotes the greatest integer less than or equal to $x$.
This formula computes $q = \lfloor a/n \rfloor$ and then recovers $r = a - qn$.
\end{remark}

\begin{remark}[Intuition]
Intuitively: $a \bmod n$ asks ``after fitting as many full copies of $n$
into $a$ as possible, what is left over?'' The remainder is always
in $[0, n)$ regardless of the sign of $a$.
\end{remark}

\begin{lemma}[Idempotence of $\bmod$]
For any $a \in \mathbb{Z}$ and $n \in \mathbb{Z}^+$,
\[
(a \bmod n) \bmod n = a \bmod n.
\]
\end{lemma}



\begin{remark}[Intuition]
Intuitively: once a value has been reduced modulo $n$,
it is already in the range $[0, n)$, so reducing again changes nothing.
\end{remark}

\begin{remark}[Applying $\bmod n$ to an Equation Term by Term]
A key technique in modular arithmetic is reducing an entire equation
modulo $n$ by applying $\bmod n$ to each term. This works because
multiples of $n$ vanish under $\bmod n$:
\[
kn \bmod n = 0 \quad \text{for any } k \in \mathbb{Z}.
\]

\medskip
\noindent\textbf{General procedure.}
Write each integer using the Division Algorithm:
\[
a = q_1 n + r_1, \qquad b = q_2 n + r_2.
\]
Then any expression built from $a$ and $b$ using addition and multiplication
can be reduced modulo $n$ by discarding all multiples of $n$
and retaining only the remainders.

\medskip
\noindent\textbf{Example: addition.}
\[
a + b = (q_1 + q_2)n + (r_1 + r_2),
\]
so
\[
(a + b) \bmod n = (r_1 + r_2) \bmod n = (a \bmod n + b \bmod n) \bmod n.
\]
The extra outer $\bmod n$ is needed because $r_1 + r_2$ may exceed $n$.

\medskip
\noindent\textbf{Example: multiplication.}
\[
ab = (q_1 n + r_1)(q_2 n + r_2)
   = q_1 q_2 n^2 + q_1 r_2 n + q_2 r_1 n + r_1 r_2.
\]
Every term except $r_1 r_2$ is a multiple of $n$, so
\[
(ab) \bmod n = (r_1 r_2) \bmod n = (a \bmod n)(b \bmod n) \bmod n.
\]

\medskip
\noindent\textbf{Concrete example.}
Compute $(17 \cdot 13) \bmod 5$.
\[
17 \bmod 5 = 2, \qquad 13 \bmod 5 = 3, \qquad 2 \cdot 3 = 6, \qquad 6 \bmod 5 = 1.
\]
Check: $17 \cdot 13 = 221$ and $221 = 44 \cdot 5 + 1$, so $221 \bmod 5 = 1$. \checkmark

\medskip
\noindent\textbf{Why this works.}
Multiples of $n$ contribute $0$ under $\bmod n$.
So when expanding any polynomial expression in $a$ and $b$,
all cross-terms involving $n$ vanish, leaving only the product of remainders.
This is formalized in the Compatibility with Arithmetic theorem below.
\end{remark}

\begin{remark}[Intuition]
Intuitively: reducing modulo $n$ is like working on a clock face with
$n$ positions --- you only ever care about where you land, not how many
full laps you made to get there. Addition and multiplication just move
you around the clock, and multiples of $n$ bring you back to $0$.
\end{remark}

\begin{definition}[Congruence Modulo $n$]
Let $n$ be a positive integer.
For integers $a$ and $b$, we say that
\[
a \equiv b \pmod{n}
\]
if $n$ divides $a-b$; that is,
\[
n \mid (a-b).
\]
This relation is called \emph{congruence modulo $n$}.
\end{definition}

\begin{remark}[Unpacking the Definition]
The statement
\[
a \equiv b \pmod{n}
\]
means that $a$ and $b$ leave the same remainder upon division by $n$.

Equivalently, there exists an integer $k$ such that
\[
a = b + kn.
\]

Thus two integers are congruent modulo $n$ precisely when they differ
by a multiple of $n$.
\end{remark}

\begin{remark}[Intuition]
Intuitively: two integers are congruent modulo $n$ if they land on the
same position when you wrap the number line around a circle of
circumference $n$. The difference between them is always an exact
number of full laps.
\end{remark}

\begin{definition}[Congruence Class]
The \emph{congruence class} of $a$ modulo $n$ is the set of all integers
congruent to $a$ modulo $n$:
\[
[a]_n = \{ a + kn : k \in \mathbb{Z} \}.
\]
\end{definition}

\begin{remark}
The remainder $r$ in the Division Algorithm $a = bq + r$ is also called the
\emph{residue} of $a$ modulo $b$, and $[a]_n$ is also called a \emph{residue class}.
The collection of all congruence classes modulo $n$ forms the set
\[
\mathbb{Z}/n\mathbb{Z} = \bigl\{\, [0]_n,\, [1]_n,\, \ldots,\, [n-1]_n \,\bigr\}.
\]
$\left|\mathbb{Z}/n\mathbb{Z}\right| = n$, meaning $\mathbb{Z}/n\mathbb{Z}$ has exactly $n$ elements,
corresponding to the $n$ possible remainders upon division by $n$.
\end{remark}

\begin{remark}[Intuition]
Intuitively: congruence classes partition all of $\mathbb{Z}$ into $n$
equally spaced infinite families. Every integer belongs to exactly one
class, determined by its remainder upon division by $n$.
\end{remark}

\begin{theorem}[Congruence is an Equivalence Relation]
Let $n$ be a positive integer. Congruence modulo $n$ is an equivalence
relation on $\mathbb{Z}$. That is, for all $a, b, c \in \mathbb{Z}$:

\begin{enumerate}
\item \textbf{Reflexivity:} $a \equiv a \pmod{n}$.
\item \textbf{Symmetry:} If $a \equiv b \pmod{n}$, then $b \equiv a \pmod{n}$.
\item \textbf{Transitivity:} If $a \equiv b \pmod{n}$ and
      $b \equiv c \pmod{n}$, then $a \equiv c \pmod{n}$.
\end{enumerate}
\end{theorem}



\begin{remark}[Intuition]
Intuitively: an equivalence relation is just a precise way of saying
``these things behave the same for our purposes.'' Congruence mod $n$
groups integers by their remainder --- same remainder, same class,
same behavior under arithmetic mod $n$.
\end{remark}

\begin{theorem}[Compatibility with Arithmetic]
Let $n$ be a positive integer and suppose
\[
a \equiv b \pmod{n}
\quad\text{and}\quad
c \equiv d \pmod{n}.
\]
Then:
\begin{enumerate}
\item $a + c \equiv b + d \pmod{n}$,
\item $a - c \equiv b - d \pmod{n}$,
\item $ac \equiv bd \pmod{n}$.
\end{enumerate}
\end{theorem}



\begin{remark}[Intuition]
Intuitively: you can swap any integer for a congruent one before doing
arithmetic, and the result will still be congruent to what you would
have gotten. This is what makes working with representatives of classes
--- rather than the classes themselves --- legitimate.
\end{remark}

\begin{remark}[Consequence for $\mathbb{Z}/n\mathbb{Z}$]
The compatibility theorem means that addition and multiplication
of congruence classes are well-defined:
\[
[a]_n + [b]_n := [a+b]_n,
\qquad
[a]_n \cdot [b]_n := [ab]_n.
\]
The choice of representative within a class does not affect the result.
Under these operations, $\mathbb{Z}/n\mathbb{Z}$ is a commutative ring.
\end{remark}


% ---------------------------------------------------------
\subsubsection{Consequences and Logical Implications}
% ---------------------------------------------------------

\begin{remark}[Connection to the Integers]
Congruence modulo $n$ is the first instance of a general construction:
forming a \emph{quotient} of $\mathbb{Z}$ by identifying elements that
differ by a multiple of $n$. The set $\mathbb{Z}/n\mathbb{Z}$ is the
quotient of $\mathbb{Z}$ by the subgroup $n\mathbb{Z} = \{\ldots, -2n, -n, 0, n, 2n, \ldots\}$.

This construction depends directly on the Division Algorithm:
every integer belongs to exactly one congruence class modulo $n$,
corresponding to its remainder upon division by $n$.
\end{remark}

\begin{remark}[When $\mathbb{Z}/n\mathbb{Z}$ is a Field]
The ring $\mathbb{Z}/n\mathbb{Z}$ is a field if and only if $n$ is prime.

When $n = p$ is prime, every nonzero element $[a]_p$ satisfies
$\gcd(a, p) = 1$, so by Bézout's Identity there exist integers $s, t$
with $as + pt = 1$, giving $[a]_p \cdot [s]_p = [1]_p$.
Thus every nonzero element has a multiplicative inverse.

When $n$ is composite, say $n = ab$ with $1 < a, b < n$,
the element $[a]_n$ is a zero divisor: $[a]_n \cdot [b]_n = [0]_n$
with neither factor zero, so $\mathbb{Z}/n\mathbb{Z}$ is not an integral domain
and hence not a field.
\end{remark}

\begin{remark}[Intuition]
Intuitively: a field requires every nonzero element to have a
multiplicative inverse. In $\mathbb{Z}/n\mathbb{Z}$, an element $[a]_n$
has an inverse exactly when $a$ and $n$ share no common factor ---
and when $n$ is prime, this is true for every nonzero element.
\end{remark}

\begin{remark}[Structural Position]
Modular arithmetic occupies the boundary between the integers
and abstract algebra:

\begin{itemize}
\item It depends on the Division Algorithm and gcd theory developed in
      the integers section.
\item It is the prototype for quotient ring constructions, which generalize
      to arbitrary ideals in commutative ring theory.
\item The fields $\mathbb{Z}/p\mathbb{Z}$ are the simplest finite fields,
      foundational in algebra, number theory, and cryptography.
\end{itemize}
\end{remark}
% =========================================================
% Induction on the integers
% =========================================================

\subsection{Induction}

% ---------------------------------------------------------
\subsubsection{Definitions and Theorems}
% ---------------------------------------------------------

\begin{definition}[Principle of Mathematical Induction]
Let $a \in \mathbb{Z}$ and let $S \subseteq \mathbb{Z}$ be a set of integers
such that $a \in S$.

Suppose $S$ has the property that whenever an integer $n \ge a$
belongs to $S$, then $n+1$ also belongs to $S$.
That is,
\[
n \in S \;\Rightarrow\; n+1 \in S
\quad \text{for all } n \ge a.
\]

Then $S$ contains every integer greater than or equal to $a$.
In other words,
\[
\{ n \in \mathbb{Z} : n \ge a \} \subseteq S.
\]
\end{definition}

\begin{definition}[Mathematical Induction — Logical Form]
Let $a \in \mathbb{Z}$ and let $P(n)$ be a predicate defined
for integers $n \ge a$.

Suppose:

\begin{enumerate}
\item \textbf{Base case:}
\[
P(a) \text{ is true.}
\]

\item \textbf{Inductive step:}
\[
\forall n \ge a \;
\bigl(P(n) \Rightarrow P(n+1)\bigr).
\]
\end{enumerate}

Then
\[
\forall n \ge a,\; P(n) \text{ is true.}
\]
\end{definition}

\begin{remark}[Equivalence with the Well-Ordering Principle]
The Principle of Mathematical Induction is logically equivalent
to the Well-Ordering Principle.

\medskip

\textbf{Equivalence idea:}

\begin{itemize}
\item \emph{Well-Ordering $\Rightarrow$ Induction:}
If a statement $P(n)$ were false for some $n \ge a$,
then the set of counterexamples would be nonempty.
By the Well-Ordering Principle, it would contain a smallest element.
This smallest counterexample contradicts the inductive hypothesis.

\item \emph{Induction $\Rightarrow$ Well-Ordering:}
If a nonempty set of integers had no smallest element,
one can use induction to show that no integer belongs to it,
yielding a contradiction.
\end{itemize}

Thus induction and well-ordering are two formulations
of the same fundamental structural property of the integers.
\end{remark}

\begin{remark}[Logical Structure of Induction]
The Principle of Mathematical Induction has the quantifier pattern
\[
\text{(initial truth)}
\;\wedge\;
\text{(truth propagates)}
\;\Rightarrow\;
\text{(universal truth)}.
\]

More precisely,
\[
\bigl(P(a)
\;\wedge\;
\forall n \ge a\, (P(n) \Rightarrow P(n+1))\bigr)
\;\Rightarrow\;
\forall n \ge a\, P(n).
\]

Thus induction is a mechanism for converting a \emph{local implication}
($P(n) \Rightarrow P(n+1)$) into a \emph{global conclusion}
($P(n)$ holds for all $n \ge a$).

It upgrades step-by-step propagation into a universal statement.
\end{remark}

\begin{theorem}[Second Principle of Mathematical Induction (Strong Induction)]
Let $a \in \mathbb{Z}$ and let $S \subseteq \mathbb{Z}$ be a set of integers
such that $a \in S$.

Suppose $S$ has the following property:
for every integer $n \ge a$,
\[
\bigl(\forall k \in \mathbb{Z}\; (a \le k < n \Rightarrow k \in S)\bigr)
\;\Rightarrow\;
n \in S.
\]
Then
\[
\{\, n \in \mathbb{Z} : n \ge a \,\} \subseteq S,
\]
i.e., $S$ contains every integer greater than or equal to $a$.
\end{theorem}

\begin{remark}[Weak vs.\ Strong Induction]
There are two common formulations of induction:

\medskip
\textbf{Weak (ordinary) induction:}
\[
P(a)
\;\wedge\;
\forall n \ge a\, (P(n) \Rightarrow P(n+1))
\;\Rightarrow\;
\forall n \ge a\, P(n).
\]

\textbf{Strong induction:}
\[
P(a)
\;\wedge\;
\forall n > a\,
\Bigl(\bigl(\forall k \ (a \le k < n \Rightarrow P(k))\bigr)
\Rightarrow P(n)\Bigr)
\;\Rightarrow\;
\forall n \ge a\, P(n).
\]

\medskip
\textbf{Difference:}
Weak induction assumes only the immediately preceding case $P(n)$
to prove $P(n+1)$.
Strong induction assumes all earlier cases $P(a),\dots,P(n-1)$
to prove $P(n)$.

\medskip
\textbf{Use cases:}
\begin{itemize}
\item Weak induction is natural when the proof of $P(n+1)$
depends only on $P(n)$.
\item Strong induction is appropriate when the proof of $P(n)$
requires information about several earlier values,
such as in factorization arguments (e.g., the Fundamental
Theorem of Arithmetic) or recursive definitions.
\end{itemize}

\medskip
Although they appear different, weak and strong induction
are logically equivalent; each can be derived from the other.
They are two formulations of the same structural property
of the integers.
\end{remark}




% ---------------------------------------------------------
\subsubsection{Consequences}
% ---------------------------------------------------------

\begin{remark}[Logical Equivalence]
The following principles are logically equivalent:

\begin{itemize}
\item The Well-Ordering Principle.
\item The Principle of Mathematical Induction.
\item The Second (Strong) Principle of Mathematical Induction.
\item The Least Element Principle.
\end{itemize}

Each encodes the same structural property of the integers:
every nonempty subset of $\mathbb{N}$ has a minimal element,
and there are no infinite strictly descending chains.
\end{remark}

\begin{remark}[Induction as Minimal Counterexample Argument]
Induction may be reformulated as a \emph{minimal counterexample} principle.

To prove a statement $P(n)$ for all $n \ge a$,
it suffices to assume that a counterexample exists,
choose the smallest such counterexample using well-ordering,
and derive a contradiction.

Thus induction and minimal-counterexample arguments
are two views of the same logical engine.
\end{remark}

\begin{remark}[Recursive Definitions]
Induction justifies recursive constructions.

If a function or object is defined by:
\begin{itemize}
\item specifying its value at $a$, and
\item specifying how to construct its value at $n+1$
from its value at $n$,
\end{itemize}
then induction guarantees that the definition
extends uniquely to all $n \ge a$.

Thus induction underlies the construction of:
\begin{itemize}
\item exponentiation,
\item factorials,
\item recursively defined sequences,
\item algorithms on the integers.
\end{itemize}
\end{remark}

\begin{remark}[Structural Position]
Induction is not merely a proof technique;
it characterizes the order structure of the integers.

It upgrades a local propagation rule
\[
P(n) \Rightarrow P(n+1)
\]
into a global universal conclusion
\[
\forall n \ge a\, P(n).
\]

This mechanism is the prototype for:
\begin{itemize}
\item Euclidean-domain arguments (minimal element methods),
\item termination proofs in algorithms,
\item structural recursion in algebra and logic.
\end{itemize}
\end{remark}

\begin{remark}[Failure Outside Well-Ordered Sets]
Induction depends essentially on the well-ordering of $\mathbb{N}$.
In sets that admit infinite descending chains
(such as $\mathbb{Z}$ under the usual order),
induction in this form fails.

Thus induction is a structural consequence
of well-ordering, not a purely algebraic property.
\end{remark}

\begin{remark}[Logical Flow]
\[
\text{Well-Ordering}
\;\Longleftrightarrow\;
\text{Induction}
\;\Longleftrightarrow\;
\text{Strong Induction}
\]

These are different formulations of the same foundational
order-theoretic principle governing the integers.
\end{remark}




































% ---------------------------------------------------------
\subsubsection{Consequences}
% ---------------------------------------------------------

\begin{corollary}
<Immediate consequence.>
\end{corollary}

\begin{proof}
<Short proof if needed.>
\end{proof}

\begin{remark}[Structural Insight]
<Explain what this section reveals about the structure.>
\end{remark}

\begin{remark}[Logical Structure]
\[
\text{Local Definitions + Theorems}
\;\Rightarrow\;
\text{Structural Consequences}.
\]
\end{remark}

% =========================================================
% Relations and Functions
% File: notes-relations-and-functions.tex
% =========================================================

\subsection{Relations and Functions}

% ---------------------------------------------------------
\subsubsection{Definitions and Theorems}
% ---------------------------------------------------------

% ---------------------------------------------------------
% Relations
% ---------------------------------------------------------

\begin{definition}[Binary Relation]
Let $A$ and $B$ be sets.
A \emph{binary relation} from $A$ to $B$ is a subset
\[
R \subseteq A \times B.
\]

If $(a,b) \in R$, we write
\[
a \,R\, b.
\]

If $A = B$, we call $R$ a \emph{relation on $A$}.
\end{definition}

\begin{definition}[Properties of Relations]
Let $R$ be a relation on a set $S$.

\begin{itemize}
\item \textbf{Reflexive:}
\[
\forall a \in S,\ (a,a) \in R.
\]

\item \textbf{Symmetric:}
\[
(a,b) \in R \Rightarrow (b,a) \in R.
\]

\item \textbf{Transitive:}
\[
(a,b) \in R \land (b,c) \in R
\Rightarrow (a,c) \in R.
\]

\item \textbf{Antisymmetric:}
\[
(a,b) \in R \land (b,a) \in R
\Rightarrow a=b.
\]
\end{itemize}
\end{definition}

\begin{definition}[Equivalence Relation]
A relation $R$ on $S$ is an \emph{equivalence relation}
if it is reflexive, symmetric, and transitive.
\end{definition}

\begin{definition}[Equivalence Class]
Let $R$ be an equivalence relation on $S$.
For $a \in S$, the \emph{equivalence class} of $a$ is
\[
[a] := \{\, x \in S : (x,a) \in R \}
= \{\, x \in S : x R a \,\}.
\]
\end{definition}

\begin{theorem}[Equivalence Relations and Partitions]
Let $R$ be an equivalence relation on a set $S$.
Then the set of equivalence classes of $R$
forms a partition of $S$.

Conversely, every partition of $S$
determines an equivalence relation on $S$.
\end{theorem}



% ---------------------------------------------------------
% Functions
% ---------------------------------------------------------

\begin{definition}[Function (Mapping)]
Let $A$ and $B$ be sets.
A \emph{function} from $A$ to $B$ is a relation
$f \subseteq A \times B$ such that:

\begin{enumerate}
\item For every $a \in A$, there exists $b \in B$
with $(a,b) \in f$.
\item If $(a,b_1) \in f$ and $(a,b_2) \in f$,
then $b_1 = b_2$.
\end{enumerate}

We write $f : A \to B$.
\end{definition}

\begin{definition}[Injective, Surjective, Bijective]
Let $f : A \to B$.

\begin{itemize}
\item Injective:
\[
f(a_1)=f(a_2) \Rightarrow a_1=a_2.
\]

\item Surjective:
\[
\forall b \in B,\ \exists a \in A \text{ such that } f(a)=b.
\]

\item Bijective: both injective and surjective.
\end{itemize}
\end{definition}

\begin{definition}[Image and Preimage]
Let $f : A \to B$.

\[
f(A)=\{f(a):a\in A\}.
\]

For $C \subseteq B$,
\[
f^{-1}(C)=\{a\in A : f(a)\in C\}.
\]
\end{definition}

\begin{definition}[Composition of Functions]
Let $f : A \to B$ and $g : B \to C$.
Define
\[
(g\circ f)(a):=g(f(a)).
\]
\end{definition}

\begin{theorem}[Basic Properties of Function Composition]
Let
\[
a:A\to B,\quad b:B\to C,\quad g:C\to D.
\]

\begin{enumerate}
\item Associativity:
\[
g\circ(b\circ a)=(g\circ b)\circ a.
\]

\item If $a$ and $b$ are injective, then $b\circ a$ is injective.

\item If $a$ and $b$ are surjective, then $b\circ a$ is surjective.

\item If $a$ is bijective, then there exists
$a^{-1}:B\to A$ such that
\[
a^{-1}\circ a=\operatorname{id}_A,
\quad
a\circ a^{-1}=\operatorname{id}_B.
\]
\end{enumerate}
\end{theorem}

% ---------------------------------------------------------
\subsubsection{Consequences}
% ---------------------------------------------------------

\begin{remark}[Structural Map]
\[
\text{Relation}
\supset
\text{Function}
\supset
\text{Bijective Function}.
\]
\end{remark}

\begin{remark}[Equivalence Relations Produce Quotients]
If $R$ is an equivalence relation on $S$, then
\[
S/R=\{[a]:a\in S\}
\]
is the associated quotient set.
\end{remark}

\begin{remark}[Partitions vs.\ Structure]
Equivalence relations encode structural indistinguishability.
Partitions encode decomposition.
These viewpoints are equivalent.
\end{remark}

\begin{remark}[Upgrade Path]
These concepts reappear as:
\begin{itemize}
\item Congruence modulo $n$,
\item Cosets and quotient groups,
\item Kernels of homomorphisms,
\item Quotient spaces in topology.
\end{itemize}
\end{remark}

\clearpage

% Column types: T = tag column, S = step column, J = justification column
\newcolumntype{T}{p{0.06\textwidth}}
\newcolumntype{S}{>{\everymath{\displaystyle}$\displaystyle}p{0.44\textwidth}<{$}}
\newcolumntype{J}{p{0.40\textwidth}}

\newmdenv[
  backgroundcolor=gray!10,
  linecolor=gray!40,
  innerleftmargin=8pt,
  innerrightmargin=8pt,
  innertopmargin=6pt,
  innerbottommargin=6pt
]{subproof}


\begin{center}
  {\large\bfseries Example Proof}\\[4pt]
  {\large $ab = \operatorname{lcm}(a,b)\,\gcd(a,b)$}\\[2pt]
  {\small\itshape (Three-column format: tag / step / justification)}\\[6pt]
  {\small
    \tagDU\ = Definition Unpacked \quad
    \tagTA\ = Theorem Applied \quad
    \tagAM\ = Algebraic Manipulation
  }
\end{center}

\begin{theorem}
Let $a$ and $b$ be positive integers. Then
\[
  ab \;=\; \operatorname{lcm}(a,b)\,\gcd(a,b).
\]
\end{theorem}

\medskip
\noindent\textbf{Proof.}
\bigskip

\noindent\textit{Part 1: Setup --- write $a$ and $b$ in terms of their gcd.}
\medskip

\noindent
\begin{tabular}{T S J}
\toprule
\textbf{Tag} & \multicolumn{1}{p{0.44\textwidth}}{\textbf{Step}} & \textbf{Justification} \\
\midrule
\addlinespace[4pt]

\tagDU
&
\text{Let } d = \gcd(a,b).
&
Definition of gcd; $d$ is the greatest common divisor of $a$ and $b$.
\\[10pt]

\tagDU
&
\begin{aligned}
  &\text{Write } a = d a_1 \text{ and } b = d b_1 \\
  &\text{for positive integers } a_1, b_1.
\end{aligned}
&
Since $d \mid a$ and $d \mid b$, we can factor $d$ out of each.
\\[10pt]

\tagTA
&
\gcd(a_1, b_1) = 1.
&
\textit{(Sub-argument by contradiction --- see box below.)}
\\[6pt]

\bottomrule
\end{tabular}

\medskip

\begin{subproof}
\noindent\textbf{Sub-argument (contradiction):}
Why must $\gcd(a_1, b_1) = 1$?

\smallskip
\noindent
\begin{tabular}{T S J}
\toprule
\textbf{Tag} & \multicolumn{1}{p{0.44\textwidth}}{\textbf{Step}} & \textbf{Justification} \\
\midrule
\addlinespace[4pt]

\tagDU
&
\text{Suppose } \gcd(a_1, b_1) = c > 1.
&
Assume for contradiction that $a_1$ and $b_1$ share a common factor.
\\[10pt]

\tagDU
&
\text{Then } c \mid a_1 \text{ and } c \mid b_1.
&
Definition of common divisor.
\\[10pt]

\tagAM
&
\text{So } dc \mid da_1 = a \text{ and } dc \mid db_1 = b.
&
Multiplying both sides of each divisibility by $d$.
\\[10pt]

\tagDU
&
\text{Thus } dc \text{ is a common divisor of } a \text{ and } b.
&
It divides both $a$ and $b$.
\\[10pt]

\tagAM
&
dc > d.
&
Since $c > 1$.
\\[10pt]

\tagDU
&
\text{This contradicts } d = \gcd(a,b).
&
$d$ is the \emph{greatest} common divisor, so no common divisor
can exceed $d$. Contradiction. $\square$
\\[6pt]

\bottomrule
\end{tabular}

\smallskip
\noindent Therefore $\gcd(a_1, b_1) = 1$; that is, $a_1$ and $b_1$ are coprime.
\end{subproof}

\bigskip

\noindent\textit{Part 2: Identify $\operatorname{lcm}(a,b)$.}
\medskip

\noindent
\begin{tabular}{T S J}
\toprule
\textbf{Tag} & \multicolumn{1}{p{0.44\textwidth}}{\textbf{Step}} & \textbf{Justification} \\
\midrule
\addlinespace[4pt]

\tagDU
&
\text{Claim: } \operatorname{lcm}(a,b) = d a_1 b_1.
&
We verify both conditions in the definition of lcm.
\\[10pt]

\tagDU
&
a \mid d a_1 b_1
\quad\text{since}\quad
d a_1 b_1 = a \cdot b_1.
&
$a = d a_1$, so $d a_1 b_1 = (d a_1) b_1 = a b_1$.
Hence $a \mid d a_1 b_1$.
\\[10pt]

\tagDU
&
b \mid d a_1 b_1
\quad\text{since}\quad
d a_1 b_1 = b \cdot a_1.
&
$b = d b_1$, so $d a_1 b_1 = a_1 (d b_1) = a_1 b$.
Hence $b \mid d a_1 b_1$.
\\[10pt]

\tagTA
&
\text{If } a \mid c \text{ and } b \mid c,
\text{ then } d a_1 b_1 \mid c.
&
Write $c = a u = d a_1 u$ and $c = b v = d b_1 v$.
Then $a_1 u = b_1 v$.
Since $\gcd(a_1, b_1)=1$, we get $b_1 \mid u$,
say $u = b_1 w$. Then $c = d a_1 b_1 w$,
so $d a_1 b_1 \mid c$.
\\[10pt]

\tagDU
&
\therefore\; \operatorname{lcm}(a,b) = d a_1 b_1.
&
Both conditions of the lcm definition are satisfied.
\\[6pt]

\bottomrule
\end{tabular}

\bigskip

\noindent\textit{Part 3: Compute $\operatorname{lcm}(a,b)\,\gcd(a,b)$.}
\medskip

\noindent
\begin{tabular}{T S J}
\toprule
\textbf{Tag} & \multicolumn{1}{p{0.44\textwidth}}{\textbf{Step}} & \textbf{Justification} \\
\midrule
\addlinespace[4pt]

\tagAM
&
\operatorname{lcm}(a,b)\,\gcd(a,b) = (d a_1 b_1) \cdot d
&
Substituting $\operatorname{lcm}(a,b) = d a_1 b_1$ and $\gcd(a,b)=d$.
\\[10pt]

\tagAM
&
= d^2 a_1 b_1
&
Collecting the two factors of $d$.
\\[10pt]

\tagAM
&
= (d a_1)(d b_1)
&
Regrouping.
\\[10pt]

\tagAM
&
= ab
&
Since $a = d a_1$ and $b = d b_1$.
\\[6pt]

\bottomrule
\end{tabular}

\medskip
\noindent Therefore $ab = \operatorname{lcm}(a,b)\,\gcd(a,b)$. \hfill$\blacksquare$

\bigskip\bigskip

\noindent\hrule
\medskip
\noindent\textbf{Study Notes.}
\medskip
\noindent\textit{Where did each tool appear?}
\medskip

\noindent
\begin{tabular}{p{0.25\textwidth} p{0.65\textwidth}}
\toprule
\textbf{Tool} & \textbf{Role in this proof} \\
\midrule
\addlinespace[4pt]
Definition of gcd
&
Gave us $d \mid a$ and $d \mid b$, allowing us to write $a = da_1$, $b = db_1$.
\\[8pt]
Proof by contradiction
&
Embedded sub-argument showing $\gcd(a_1, b_1) = 1$.
The key: if they shared a factor $c>1$, then $dc > d$ would be
a common divisor, contradicting $d$ being \emph{greatest}.
\\[8pt]
Definition of lcm (universal property)
&
Used to \emph{verify} the candidate $da_1b_1$
rather than just assert it. Required checking both
(i) it is a common multiple, and (ii) it divides every common multiple.
\\[8pt]
Coprimality of $a_1, b_1$
&
The essential ingredient in Part 2 that makes the
minimality argument work: $\gcd(a_1,b_1)=1$ forced $b_1 \mid u$.
\\[8pt]
Algebraic regrouping
&
The final computation is just $d^2 a_1 b_1 = (da_1)(db_1)$.
All the work was in setting up the right objects.
\\[6pt]
\bottomrule
\end{tabular}
\clearpage
\begin{center}
  {\large\bfseries Proof GCD is a linear combination}\\[4pt]
  {\large $\gcd(a,b) = as + bt$ for some $s,t \in \mathbb{Z}$}\\[2pt]
  {\small\itshape (Three-column format: tag / step / justification)}\\[6pt]
  {\small
    \tagDU\ = Definition Unpacked \quad
    \tagTA\ = Theorem Applied \quad
    \tagAM\ = Algebraic Manipulation
  }
\end{center}

\begin{theorem}[GCD as Smallest Positive Linear Combination]
Let $a$ and $b$ be positive integers. Then $\gcd(a,b)$ is the smallest positive element of the set
\[
  S = \{ am + bn \mid m,n \in \mathbb{Z},\; am+bn > 0 \}.
\]
\end{theorem}

\medskip
\noindent\textbf{Proof.}
\bigskip

\noindent\textit{Part 1: S is nonempty and has a smallest member.}
\medskip

\noindent
\begin{tabular}{T S J}
\toprule
\textbf{Tag} & \multicolumn{1}{p{0.44\textwidth}}{\textbf{Step}} & \textbf{Justification} \\
\midrule
\addlinespace[4pt]

\tagDU
&
S = \{ am + bn \mid m,n \in \mathbb{Z},\; am+bn > 0 \}.
&
Define $S$ as the set of all positive integer linear combinations of $a$ and $b$.
\\[10pt]

\tagDU
&
S \neq \emptyset.
&
If some $m,n$ give $am+bn < 0$, replace $m,n$ with $-m,-n$ to get a positive value.
\\[10pt]

\tagTA
&
\text{Let } d = as + bt \text{ be the smallest member of } S.
&
By the Well-Ordering Principle, every nonempty set of positive integers has a least element.
\\[6pt]

\bottomrule
\end{tabular}

\bigskip

\noindent\textit{Part 2: $d$ divides both $a$ and $b$.}
\medskip

\noindent
\begin{tabular}{T S J}
\toprule
\textbf{Tag} & \multicolumn{1}{p{0.44\textwidth}}{\textbf{Step}} & \textbf{Justification} \\
\midrule
\addlinespace[4pt]

\tagTA
&
\text{Write } a = dq + r, \quad 0 \leq r < d.
&
Division Algorithm applied to $a$ and $d$.
\\[10pt]

\tagAM
&
r = a - dq = a - (as+bt)q = a(1-sq) + b(-tq).
&
Substituting $d = as+bt$ and expanding.
\\[10pt]

\tagDU
&
\text{If } r > 0, \text{ then } r \in S.
&
$r$ is a positive integer linear combination of $a$ and $b$.
\\[10pt]

\tagTA
&
r < d, \text{ contradicting } d = \min S.
&
But $r < d$ by the Division Algorithm, contradicting minimality of $d$.
\\[10pt]

\tagDU
&
\therefore\; r = 0, \text{ so } d \mid a.
&
The remainder must be zero; hence $d$ divides $a$.
\\[10pt]

\tagDU
&
d \mid b.
&
By the same argument applied symmetrically to $b$.
\\[10pt]

\tagDU
&
\therefore\; d \text{ is a common divisor of } a \text{ and } b.
&
$d$ divides both $a$ and $b$.
\\[6pt]

\bottomrule
\end{tabular}

\bigskip

\noindent\textit{Part 3: $d$ is the \emph{greatest} common divisor.}
\medskip

\noindent
\begin{tabular}{T S J}
\toprule
\textbf{Tag} & \multicolumn{1}{p{0.44\textwidth}}{\textbf{Step}} & \textbf{Justification} \\
\midrule
\addlinespace[4pt]

\tagDU
&
\text{Let } d' \text{ be any common divisor of } a \text{ and } b.
&
Suppose $d'$ is an arbitrary common divisor; write $a = d'h$ and $b = d'k$.
\\[10pt]

\tagAM
&
d = as + bt = (d'h)s + (d'k)t = d'(hs + kt).
&
Substituting $a = d'h$ and $b = d'k$ into $d = as+bt$.
\\[10pt]

\tagDU
&
\therefore\; d' \mid d.
&
$d$ is an integer multiple of $d'$, so $d' \leq d$.
\\[10pt]

\tagDU
&
\therefore\; d = \gcd(a,b).
&
$d$ is a common divisor of $a$ and $b$, and every other common divisor
divides $d$, so $d$ is greatest. \hfill$\blacksquare$
\\[6pt]

\bottomrule
\end{tabular}

\bigskip\bigskip

\noindent\hrule
\medskip
\noindent\textbf{Study Notes.}
\medskip
\noindent\textit{Where did each tool appear?}
\medskip

\noindent
\begin{tabular}{p{0.25\textwidth} p{0.65\textwidth}}
\toprule
\textbf{Tool} & \textbf{Role in this proof} \\
\midrule
\addlinespace[4pt]
Well-Ordering Principle
&
Guaranteed that $S$, being a nonempty set of positive integers, has a least element $d$.
\\[8pt]
Division Algorithm
&
Used to write $a = dq + r$ and derive that $r \in S$ if $r > 0$, forcing $r = 0$.
\\[8pt]
Proof by contradiction
&
If $r > 0$ then $r \in S$ with $r < d$, contradicting minimality of $d$.
\\[8pt]
Linear combination structure
&
The form $d = as + bt$ was essential in showing every common divisor $d'$ satisfies $d' \mid d$.
\\[8pt]
Symmetry argument
&
After showing $d \mid a$, the identical argument applies to $b$ without repeating the full proof.
\\[6pt]
\bottomrule
\end{tabular}
% -----------------------
% Groups
% -----------------------
% \input{volume-iii/algebra/abstract-algebra/notes/notes-groups}
% \input{volume-iii/algebra/abstract-algebra/notes/notes-subgroups}
% \input{volume-iii/algebra/abstract-algebra/notes/notes-cyclic-groups}
% \input{volume-iii/algebra/abstract-algebra/notes/notes-permutation-groups}
% \input{volume-iii/algebra/abstract-algebra/notes/notes-isomorphisms}
% \input{volume-iii/algebra/abstract-algebra/notes/notes-cosets}
% \input{volume-iii/algebra/abstract-algebra/notes/notes-normal-subgroups}
% \input{volume-iii/algebra/abstract-algebra/notes/notes-homomorphisms}
% \input{volume-iii/algebra/abstract-algebra/notes/notes-finite-abelian-groups}

% -----------------------
% Rings
% -----------------------
% \input{volume-iii/algebra/abstract-algebra/notes/notes-rings}
% \input{volume-iii/algebra/abstract-algebra/notes/notes-ideals}
% % =========================================================
% Polynomial Rings
% =========================================================

% ---------------------------------------------------------
\subsubsection{Preliminary Definitions}
% ---------------------------------------------------------

\begin{definition}[Ring]
A set $R$ with two binary operations $+ : R \times R \to R$
and $\cdot : R \times R \to R$ is a \emph{ring} if:

\begin{enumerate}
\item \textbf{Additive structure:}
$(R, +)$ is an abelian group.

\item \textbf{Associativity of multiplication:}
\[
(a \cdot b) \cdot c = a \cdot (b \cdot c)
\quad \text{for all } a,b,c \in R.
\]

\item \textbf{Multiplicative identity:}
There exists $1 \in R$ such that
\[
1 \cdot a = a \cdot 1 = a
\quad \text{for all } a \in R.
\]

\item \textbf{Distributivity:}
\[
a \cdot (b + c) = a \cdot b + a \cdot c
\quad \text{and} \quad
(a + b) \cdot c = a \cdot c + b \cdot c
\quad \text{for all } a,b,c \in R.
\]
\end{enumerate}
\end{definition}

\begin{remark}
A ring generalizes a field by dropping the requirement that
nonzero elements have multiplicative inverses,
and by not requiring multiplication to be commutative.
When multiplication is commutative, we call $R$ a
\emph{commutative ring}.
\end{remark}

\begin{definition}[Commutative Ring]
A ring $R$ is \emph{commutative} if
\[
a \cdot b = b \cdot a
\quad \text{for all } a, b \in R.
\]
\end{definition}

\begin{remark}[Structural Summary]
The hierarchy of algebraic structures encountered so far is:

\begin{center}
\begin{tabular}{l|c|c}
Structure & Additive Group & Multiplicative Structure \\
\hline
Ring            & Abelian & Associative, with identity \\
Commutative Ring & Abelian & Associative, commutative, with identity \\
Field           & Abelian & Abelian on nonzero elements
\end{tabular}
\end{center}

Every field is a commutative ring, but not every commutative ring is a field.
\end{remark}

% ---------------------------------------------------------
\subsubsection{Polynomial Rings}
% ---------------------------------------------------------

\begin{definition}[Polynomial]
Let $R$ be a commutative ring.
A \emph{polynomial} in one variable over $R$ is a formal expression
\[
f = a_n x^n + a_{n-1} x^{n-1} + \cdots + a_1 x + a_0,
\]
where $n \in \mathbb{N}$, the \emph{coefficients} $a_0, a_1, \dots, a_n \in R$,
and $x$ is a formal symbol called an \emph{indeterminate}.
\end{definition}

\begin{remark}
The indeterminate $x$ is not a variable ranging over values in $R$.
It is a formal placeholder that encodes the coefficient sequence.
Two polynomials are equal if and only if all their coefficients are equal.
\end{remark}

\begin{definition}[Degree]
Let $f = a_n x^n + \cdots + a_0$ be a polynomial over $R$.
If $a_n \neq 0$, then the \emph{degree} of $f$ is
\[
\deg(f) := n.
\]
The coefficient $a_n$ is called the \emph{leading coefficient} of $f$.
A polynomial with leading coefficient $1$ is called \emph{monic}.
\end{definition}

\begin{remark}
The zero polynomial $f = 0$ has no leading coefficient.
Its degree is left undefined, or assigned $-\infty$ by convention
to preserve the identity $\deg(fg) = \deg(f) + \deg(g)$.
\end{remark}

\begin{definition}[Polynomial Ring]
Let $R$ be a commutative ring.
The \emph{polynomial ring} over $R$ in one indeterminate $x$,
denoted $R[x]$, is the set of all polynomials in $x$ with
coefficients in $R$, equipped with addition and multiplication
defined by:
\[
\left(\sum_{i} a_i x^i\right) + \left(\sum_{i} b_i x^i\right)
:= \sum_{i} (a_i + b_i) x^i,
\]
\[
\left(\sum_{i} a_i x^i\right) \cdot \left(\sum_{j} b_j x^j\right)
:= \sum_{k} \left(\sum_{i+j=k} a_i b_j\right) x^k.
\]
\end{definition}

\begin{example}
Let $R = \mathbb{Z}$ and consider
\[
f = 2x^2 + 3x + 1, \qquad g = x + 4 \quad \in \mathbb{Z}[x].
\]
Then:
\[
f + g = 2x^2 + 4x + 5,
\]
\[
f \cdot g = 2x^3 + 11x^2 + 13x + 4.
\]
\end{example}

\begin{remark}[Structural Summary]
With these operations, $R[x]$ is itself a commutative ring.
The original ring $R$ embeds into $R[x]$ as the constant polynomials.

\begin{center}
\begin{tabular}{l|c}
Property & Holds in $R[x]$? \\
\hline
Commutative ring & Always \\
Field & Only in degenerate cases \\
$R$ embeds in $R[x]$ & Always
\end{tabular}
\end{center}
\end{remark}

% ---------------------------------------------------------
\subsubsection{Polynomial Rings in Several Variables}
% ---------------------------------------------------------

\begin{definition}[Polynomial Ring in $n$ Variables]
Let $R$ be a commutative ring and let $n \in \mathbb{N}$.
The \emph{polynomial ring in $n$ variables} over $R$, denoted
\[
R[x_1, x_2, \dots, x_n],
\]
is defined inductively by
\[
R[x_1, \dots, x_n] := R[x_1, \dots, x_{n-1}][x_n].
\]
Its elements are finite sums of the form
\[
f = \sum_{\alpha} a_\alpha \, x^\alpha,
\]
where the sum ranges over multi-indices
$\alpha = (\alpha_1, \dots, \alpha_n) \in \mathbb{N}^n$,
the coefficients $a_\alpha \in R$, and
$x^\alpha := x_1^{\alpha_1} \cdots x_n^{\alpha_n}$.
\end{definition}

\begin{definition}[Monomial]
A \emph{monomial} in $R[x_1, \dots, x_n]$ is a polynomial
of the form
\[
x^\alpha = x_1^{\alpha_1} \cdots x_n^{\alpha_n}
\]
for some multi-index $\alpha \in \mathbb{N}^n$.
\end{definition}

\begin{definition}[Total Degree]
The \emph{total degree} of the monomial $x^\alpha$ is
\[
|\alpha| := \alpha_1 + \alpha_2 + \cdots + \alpha_n.
\]
The degree of a polynomial $f \in R[x_1,\dots,x_n]$ is
the maximum total degree among all monomials with nonzero coefficient.
\end{definition}

\begin{example}
In $\mathbb{R}[x,y,z]$, the polynomial
\[
f = 3x^2 y + xy^2 z - 5z^3
\]
has three terms with total degrees $3$, $4$, and $3$ respectively.
Hence $\deg(f) = 4$.
\end{example}

\begin{remark}[Relevance to Algebraic Geometry]
Polynomial rings in several variables are the foundational
algebraic object of algebraic geometry.
The geometric objects studied in subsequent sections ---
affine varieties, ideals, coordinate rings ---
are all defined in terms of $k[x_1, \dots, x_n]$
for a field $k$.
The interplay between the algebra of $k[x_1,\dots,x_n]$
and the geometry of its zero sets is the central theme
of Clader--Ross.
\end{remark}

% ---------------------------------------------------------
\subsubsection{Ideals}
% ---------------------------------------------------------

\begin{definition}[Ideal]
Let $R$ be a commutative ring.
A subset $I \subseteq R$ is an \emph{ideal} of $R$ if:
\begin{enumerate}
\item $0 \in I$,
\item $a + b \in I$ for all $a, b \in I$,
\item $r \cdot a \in I$ for all $r \in R$ and $a \in I$.
\end{enumerate}
\end{definition}

\begin{definition}[Generated Ideal]
Let $R$ be a commutative ring and let $f_1, \dots, f_m \in R$.
The \emph{ideal generated by} $f_1, \dots, f_m$ is
\[
\langle f_1, \dots, f_m \rangle
:= \left\{ \sum_{i=1}^m r_i f_i : r_i \in R \right\}.
\]
An ideal of this form is called \emph{finitely generated}.
\end{definition}

\begin{definition}[Finitely Generated Ideal]
An ideal $I \subseteq R$ is \emph{finitely generated} if there exist
$f_1, \dots, f_m \in R$ such that $I = \langle f_1, \dots, f_m \rangle$.
\end{definition}

\begin{definition}[Noetherian Ring]
A commutative ring $R$ is \emph{Noetherian} if every ideal of $R$
is finitely generated.
\end{definition}

\begin{theorem}[Hilbert Basis Theorem]
If $R$ is a Noetherian ring, then the polynomial ring $R[x]$
is also Noetherian.

In particular, $k[x_1, \dots, x_n]$ is Noetherian for any field $k$.
\end{theorem}

\begin{remark}
The Hilbert Basis Theorem guarantees that every ideal in
$k[x_1, \dots, x_n]$ is finitely generated.
This is a foundational finiteness result:
it means every algebraic variety can be cut out by
finitely many polynomial equations.
\end{remark}

% ---------------------------------------------------------
\subsubsection{Quotient Rings}
% ---------------------------------------------------------

\begin{definition}[Quotient Ring]
Let $R$ be a commutative ring and let $I \subseteq R$ be an ideal.
The \emph{quotient ring} $R/I$ is the set of cosets
\[
R/I := \{ a + I : a \in R \},
\]
equipped with addition and multiplication defined by
\[
(a + I) + (b + I) := (a + b) + I,
\]
\[
(a + I) \cdot (b + I) := (a \cdot b) + I.
\]
\end{definition}

\begin{remark}
The quotient ring $R/I$ is a commutative ring.
Elements of $R/I$ are equivalence classes under the relation
$a \sim b \iff a - b \in I$.
\end{remark}

\begin{example}
In $\mathbb{R}[x]$, consider the ideal $I = \langle x^2 + 1 \rangle$.
The quotient ring
\[
\mathbb{R}[x] / \langle x^2 + 1 \rangle \;\cong\; \mathbb{C},
\]
since in the quotient, $x$ satisfies $x^2 = -1$,
which is precisely the defining relation of $i \in \mathbb{C}$.
\end{example}

\begin{remark}[Structural Position]
Quotient rings of polynomial rings are the coordinate rings
of affine varieties, which are studied in depth in the next chapter.
The passage
\[
k[x_1,\dots,x_n] \;\longrightarrow\; k[x_1,\dots,x_n]/I
\]
is the algebraic encoding of restricting from all of $\mathbb{A}^n$
to the variety $V(I)$.
\end{remark}
% \input{volume-iii/algebra/abstract-algebra/notes/notes-ufd-euclidean}

% -----------------------
% Fields
% -----------------------
% \input{volume-iii/algebra/abstract-algebra/notes/notes-field-extensions}
% \input{volume-iii/algebra/abstract-algebra/notes/notes-finite-fields}
% \input{volume-iii/algebra/abstract-algebra/notes/notes-galois-theory}

% -----------------------
% Special Topics
% -----------------------
% \input{volume-iii/algebra/abstract-algebra/notes/notes-sylow}
% \input{volume-iii/algebra/abstract-algebra/notes/notes-group-actions}
% \input{volume-iii/algebra/abstract-algebra/notes/notes-coding-theory}
