% =========================================================
% NOTES TEMPLATE (Definitions → Theorems/Proofs → Consequences)
% File: notes-integers.tex
% =========================================================

\subsection{Integers}
\label{sec:integers}  


% ---------------------------------------------------------
\subsubsection{Basic Definitions and Theorems}

\begin{axiom}[Well Ordering Principle]
Every nonempty set of positive integers contains a smallest member.
\[
\forall A\,
\Bigl(
\bigl( A \subseteq \mathbb{N}
\land \exists a\,(a \in A) \bigr)
\Rightarrow
\exists m\,
\bigl(
m \in A
\land
\forall a\,(a \in A \Rightarrow m \le a)
\bigr)
\Bigr).
\]
\end{axiom}

\begin{remark}
The well-ordering principle asserts that every nonempty subset, $A$,
of $\mathbb{N}$ possesses a least element.
Its quantifier structure is:
\[
\forall A\;
(\exists a \in A \Rightarrow \exists m \in A\;\forall a \in A).
\]
This principle is equivalent to the principle of mathematical induction.
\end{remark}

\begin{remark}[Axiomatic Status of the Well-Ordering Principle]
The Well-Ordering Principle is not a theorem of elementary arithmetic.
It cannot be proved using only the usual algebraic laws of addition
and multiplication on $\mathbb{N}$.

Rather, it is taken as an axiom describing the order structure of
the natural numbers.  In standard foundations, it is equivalent
to the Principle of Mathematical Induction and the Least Element
Principle.  Thus one must assume one of these principles in order
to derive the others.

In particular, arithmetic identities alone do not imply that every
nonempty subset of $\mathbb{N}$ has a smallest element; this property
is part of the defining structure of the natural numbers.
\end{remark}

\begin{remark}[Intuition]
Intuitively: the Well-Ordering Principle says you can always find a
``smallest'' object in any nonempty collection of positive integers.
This gives us a foothold — once we have a minimal element,
we can derive divisibility and gcd properties by contradiction.
\end{remark}

\begin{definition}
Let $a, b \in \mathbb{Z}$ with $b \neq 0$. In the expression $\dfrac{a}{b}$,
we call $a$ the \textit{numerator} and $b$ the \textit{denominator}.
\end{definition}

\begin{definition}
Let $a, b \in \mathbb{Z}$. We say $a$ is a \textit{multiple} of $b$ if
$a = bk$ for some $k \in \mathbb{Z}$.
\end{definition}

\begin{definition}[Divisibility]
Let $a, b \in \mathbb{Z}$ with $b \neq 0$.
We say $b$ \emph{divides} $a$, written $b \mid a$, if there exists
an integer $k$ such that
\[
a = bk.
\]
If $b$ divides $a$, we call $b$ a \emph{divisor} (or \emph{factor}) of $a$,
and $a$ a \emph{multiple} of $b$.
If $b$ does not divide $a$, we write $b \nmid a$.
\end{definition}

\begin{remark}[Unpacking Divisibility]
The statement $b \mid a$ is a claim about the \emph{existence} of an integer
$k$ such that $a = bk$. It says nothing about a remainder --- divisibility means
the remainder is exactly zero.

For example:
\begin{itemize}
\item $3 \mid 12$ because $12 = 3 \cdot 4$.
\item $5 \nmid 13$ because $13 = 5 \cdot 2 + 3$ and the remainder $3 \neq 0$.
\end{itemize}

This is the foundation for gcd, lcm, and Bézout's Identity.
\end{remark}

\begin{remark}[Intuition]
Intuitively: $b \mid a$ means $b$ fits into $a$ a whole number of times,
with nothing left over. Division is exact.
\end{remark}

\begin{lemma}[Divisibility of Linear Combinations]
\label{lem:div-linear-combo}
Let $a, b \in \mathbb{Z}$. If $t \mid a$ and $t \mid b$, then $t$ divides
every integer linear combination of $a$ and $b$. That is,
\[
t \mid (ua + vb)
\quad \text{for all } u, v \in \mathbb{Z}.
\]
\end{lemma}

\begin{proof}
Since $t \mid a$ and $t \mid b$, there exist integers $x, y$ such that
\[
a = tx, \qquad b = ty.
\]
Then for any $u, v \in \mathbb{Z}$,
\[
ua + vb = u(tx) + v(ty) = t(ux + vy).
\]
Since $ux + vy \in \mathbb{Z}$, we have $t \mid (ua + vb)$.
\end{proof}

\begin{remark}
This lemma is used constantly in number theory. Its most important
instance is: since $\gcd(a,b)$ divides both $a$ and $b$, it divides
every linear combination $ua + vb$. In particular it divides
$d = as + bt$ from Bézout's Identity, which is how we verify
the greatest-ness of the gcd.
\end{remark}

\begin{remark}[Intuition]
Intuitively: if $t$ fits into both $a$ and $b$ individually with no remainder,
then $t$ must fit into any way you scale and combine them --- the remainders
simply cannot accumulate.
\end{remark}

\begin{theorem}[Division Algorithm]
Let $a$ and $b$ be integers with $b > 0$. Then there exist unique integers
$q$ and $r$ such that
\[
a = bq + r,
\qquad
0 \le r < b.
\]
\end{theorem}

\begin{remark}
In the Division Algorithm, $a = bq + r$, we say $a$ is the \textit{dividend}
and $b$ is the \textit{divisor}. This corresponds to the division $a / b$,
yielding quotient $q$ and remainder $r$. When $r = 0$, we write $b \mid a$
(``$b$ divides $a$''), meaning $a = bq$ for some integer $q$.
\end{remark}

\begin{proof}
We begin with the existence portion of the theorem. Consider the set
\[
S = \{\, a - bk \mid k \in \mathbb{Z},\; a - bk \ge 0 \,\}.
\]

If $0 \in S$, then $b$ divides $a$ and we obtain the desired result with
$q = a/b$ and $r = 0$.

Now assume $0 \notin S$. Since $S$ is nonempty
(if $a > 0$, then $a - b\cdot 0 = a \in S$;
if $a < 0$, then choosing $k$ sufficiently negative makes $a - bk \ge 0$),
we may apply the Well-Ordering Principle to conclude that $S$ has a smallest
member, say
\[
r = a - bq.
\]
Then
\[
a = bq + r
\quad \text{and} \quad r \ge 0.
\]
It remains to prove that $r < b$.

Suppose, to the contrary, that $r \ge b$. Then
\[
a - b(q+1)
=
a - bq - b
=
r - b
\ge 0,
\]
so that $a - b(q+1) \in S$. But
\[
a - b(q+1) < a - bq,
\]
contradicting the minimality of $r = a - bq$. Hence $r < b$.

\medskip

To establish uniqueness, suppose there exist integers $q, q', r, r'$
such that
\[
a = bq + r, \qquad 0 \le r < b,
\]
and
\[
a = bq' + r', \qquad 0 \le r' < b.
\]

Subtracting the two equations gives
\[
b(q - q') = r' - r.
\]

Thus $b$ divides $r' - r$. Without loss of generality, assume $r' \ge r$.
Then
\[
0 \le r' - r < b.
\]

The only multiple of $b$ in this interval is $0$. Therefore,
\[
r' - r = 0,
\]
so $r' = r$ and hence $q = q'$.

This completes the proof.
\end{proof}
\cite{GallianCAA7}

\begin{definition}[Quotient and Remainder]
Let $a,b \in \mathbb{Z}$ with $b>0$.
If integers $q$ and $r$ satisfy
\[
a = bq + r,
\qquad
0 \le r < b,
\]
then:
\begin{itemize}
\item $q$ is called the \emph{quotient} of $a$ divided by $b$,
\item $r$ is called the \emph{remainder} of $a$ divided by $b$.
\end{itemize}
\end{definition}

\begin{remark}
The conditions $0 \le r < b$ ensure uniqueness.
Without this bound on $r$, the representation
\[
a = bq + r
\]
would not be unique, since one could replace $(q,r)$ with
\[
(q+1, r-b), \quad (q-1, r+b), \quad \text{etc.}
\]
The remainder condition selects exactly one representative.
\end{remark}

\begin{remark}[Interpretation of quotient and remainder]
The quotient measures how many full copies of $b$ fit into $a$,
while the remainder measures the leftover part that is strictly
smaller than $b$.
\end{remark}

\begin{definition}
An integer $p > 1$ is called \textit{prime} if its only positive divisors
are $1$ and $p$.
\end{definition}

\begin{definition}[Greatest Common Divisor]
Let $a$ and $b$ be integers, not both zero. The greatest common divisor
of $a$ and $b$ is the largest of all common divisors of $a$ and $b$.

A positive integer $d$ is called the \emph{greatest common divisor}
of $a$ and $b$ if:
\begin{enumerate}
\item $d \mid a$ and $d \mid b$ (that is, $d$ divides both $a$ and $b$), and
\item Whenever $c$ is an integer such that $c \mid a$ and $c \mid b$,
then $c \mid d$.
\end{enumerate}

The greatest common divisor of $a$ and $b$ is denoted
\[
\gcd(a,b).
\]
\end{definition}

\begin{remark}[Intuition]
Intuitively: the gcd is the largest ruler that measures both $a$ and $b$
exactly, with no remainder. Any other common divisor must itself be measured
exactly by the gcd.
\end{remark}

\begin{definition}[Relatively Prime Integers]
Two integers $a$ and $b$ are said to be \emph{relatively prime}
(or \emph{coprime}) if
\[
\gcd(a,b) = 1.
\]
\end{definition}

\begin{remark}
If $\gcd(a,b) = 1$, then the only positive integer dividing both $a$ and $b$
is $1$. In this case, $a$ and $b$ share no common prime factors.
\end{remark}

\begin{theorem}[Bézout's Identity]
Let $a$ and $b$ be integers, not both zero.
Then there exist integers $s$ and $t$ such that
\[
\gcd(a,b) = as + bt.
\]

Moreover, $\gcd(a,b)$ is the smallest positive integer
of the form
\[
as + bt,
\quad s,t \in \mathbb{Z}.
\]
\end{theorem}

\begin{remark}
The set
\[
\{\, as + bt : s,t \in \mathbb{Z} \,\}
\]
is called the set of \emph{integer linear combinations} of $a$ and $b$.
Bézout's Identity asserts that the greatest common divisor
is the minimal positive element of this set.
\end{remark}

\begin{proof}
Consider the set
\[
S = \{\, am + bn \mid m,n \in \mathbb{Z},\; am + bn > 0 \,\}.
\]

Since $S$ is nonempty (if some choice of $m,n$ gives $am+bn<0$,
replace $m,n$ with $-m,-n$), the Well-Ordering Principle
implies that $S$ has a smallest element.
Denote this smallest element by
\[
d = as + bt
\]
for some integers $s,t$.

We claim that $d = \gcd(a,b)$.

\medskip
\noindent
\textbf{Step 1: $d$ divides $a$ and $b$.}

Apply the Division Algorithm to write
\[
a = dq + r,
\qquad
0 \le r < d.
\]

If $r > 0$, then
\[
r = a - dq
  = a - (as + bt)q
  = a - asq - btq
  = a(1 - sq) + b(-tq).
\]
Thus $r$ is of the form $am + bn$, and since $r>0$,
we have $r \in S$. But $r < d$, contradicting
the minimality of $d$. Hence $r = 0$, so $d \mid a$.

By symmetry, the same argument shows that $d \mid b$.
Thus $d$ is a common divisor of $a$ and $b$.

\medskip
\noindent
\textbf{Step 2: $d$ is the greatest common divisor.}

Let $d'$ be any other common divisor of $a$ and $b$.
Then there exist integers $h,k$ such that
\[
a = d'h,
\qquad
b = d'k.
\]

Substituting into $d = as + bt$ gives
\[
d = (d'h)s + (d'k)t
  = d'(hs + kt).
\]

Hence $d'$ divides $d$.
Therefore, every common divisor of $a$ and $b$
divides $d$, and so $d$ is the greatest common divisor.

\medskip

Thus,
\[
d = \gcd(a,b).
\]
\end{proof}

\begin{remark}[Structural Map of the Proof]
This proof has three distinct logical engines.

\begin{enumerate}
\item \textbf{Well-Ordering Principle (existence of a minimal positive combination).}
We define
\[
S=\{\,am+bn : m,n\in\mathbb{Z},\; am+bn>0\,\}.
\]
The Well-Ordering Principle is invoked \emph{exactly here} to guarantee that
$S$ has a smallest element $d$. This is the sole point where order-theoretic
structure on $\mathbb{N}$ enters.

\item \textbf{Division Algorithm (turning minimality into divisibility).}
After choosing the minimal $d\in S$, we apply the Division Algorithm to write
\[
a = dq + r,\qquad 0 \le r < d.
\]
The key move is: if $r>0$, then $r$ can be rewritten as another positive
integer linear combination of $a$ and $b$, hence $r\in S$, contradicting the
minimality of $d$. Therefore $r=0$ and $d\mid a$ (and similarly $d\mid b$).
So the Division Algorithm is used \emph{exactly here} to convert the order
statement ``$d$ is smallest'' into the algebraic statement ``$d$ divides.''

\item \textbf{Greatest-ness (universal property among common divisors).}
Finally, for any common divisor $d'$ of $a$ and $b$, writing $a=d'h$ and $b=d'k$
shows
\[
d = as+bt = d'(hs+kt),
\]
so $d' \mid d$. This establishes that $d$ is the \emph{greatest} common divisor
in the divisibility order.
\end{enumerate}

\medskip
\noindent
\textbf{Prototype Euclidean-domain argument.}
The overall pattern is the Euclidean-domain template:
\[
\text{(nonempty set of ``sizes'')} \xRightarrow{\text{well-ordering}}
\text{minimal element} \xRightarrow{\text{division algorithm}}
\text{divisibility + gcd characterization}.
\]
In $\mathbb{Z}$, the ``size'' is the usual order on positive integers; in a
general Euclidean domain, the ``size'' is a Euclidean function
$\delta:R\setminus\{0\}\to\mathbb{N}$, and the same minimality-plus-division
strategy produces gcds and Bézout-type identities.
\end{remark}

\begin{corollary}
Let $a$ and $b$ be relatively prime integers.
Then there exist integers $s$ and $t$ such that
\[
as + bt = 1.
\]
\end{corollary}

\begin{proof}
If $a$ and $b$ are relatively prime, then
\[
\gcd(a,b) = 1.
\]
By Bézout's Identity, there exist integers $s$ and $t$ such that
\[
\gcd(a,b) = as + bt.
\]
Substituting $\gcd(a,b)=1$ gives
\[
as + bt = 1.
\]
\end{proof}

\begin{remark}[Bézout in Both Directions]
Bézout's Identity is often applied in two distinct directions,
and it is important to keep them separate.

\medskip
\noindent
\textbf{Forward direction.}
If $\gcd(a,b) = d$, then there exist integers $s, t$ such that
\[
as + bt = d.
\]
This is the content of the theorem itself.

\medskip
\noindent
\textbf{Reverse direction.}
If there exist integers $s, t$ such that $as + bt = 1$, then
$\gcd(a,b) = 1$.

\medskip
\noindent
\emph{Proof of reverse direction.}
Let $d = \gcd(a,b)$. By the forward direction, $d \mid a$ and $d \mid b$,
so by Lemma~\ref{lem:div-linear-combo}, $d \mid (as + bt) = 1$.
Since $d > 0$ and $d \mid 1$, we conclude $d = 1$.

\medskip
More generally, if $as + bt = c$ for some integers $s, t$, then
$\gcd(a,b) \mid c$, since $\gcd(a,b)$ divides any linear combination
of $a$ and $b$.
\end{remark}

\begin{remark}[Intuition]
Intuitively: a linear combination equaling $1$ leaves no room for any
common factor greater than $1$ --- if $d > 1$ divided both $a$ and $b$,
it would divide the combination too, but nothing greater than $1$ divides $1$.
\end{remark}

\begin{theorem}[Equivalent Characterizations of Relatively Prime Integers]
For integers $a$ and $b$, the following are equivalent:

\begin{enumerate}
\item $\gcd(a,b)=1$.
\item There exist integers $s,t$ such that $as+bt=1$.
\item $a$ and $b$ share no common prime factor.
\end{enumerate}
\end{theorem}

\begin{theorem}[Euclid's Lemma]
Let $p$ be a prime integer.
If
\[
p \mid ab,
\]
then
\[
p \mid a \quad \text{or} \quad p \mid b.
\]
\end{theorem}

\begin{proof}
Suppose $p$ is prime and $p \mid ab$.
If $p \mid a$, we are done.
So assume that $p \nmid a$.
We must show that $p \mid b$.

Because $p$ is prime and $p \nmid a$, it follows that
\[
\gcd(p,a) = 1.
\]
By Bézout's Identity, there exist integers $s$ and $t$ such that
\[
1 = as + pt.
\]

Multiplying both sides by $b$ gives
\[
b = abs + ptb.
\]

Since $p \mid ab$, the term $abs$ is divisible by $p$.
The term $ptb$ is clearly divisible by $p$.
Therefore the right-hand side is divisible by $p$,
and hence $p \mid b$.

Thus, if $p \mid ab$, then $p \mid a$ or $p \mid b$.
\end{proof}

\begin{remark}[Intuition]
Intuitively: primes cannot be fooled by multiplication --- if a prime
divides a product, it must have divided one of the factors going in.
This rigidity is what makes unique prime factorization possible.
\end{remark}

\begin{definition}[Least Common Multiple]
Let $a$ and $b$ be nonzero integers.
The \emph{least common multiple} of $a$ and $b$
is the unique positive integer $m$ such that:

\begin{enumerate}
\item $a \mid m$ and $b \mid m$ (so $m$ is a common multiple), and
\item If $c$ is any positive integer with $a \mid c$ and $b \mid c$,
then $m \mid c$.
\end{enumerate}

The least common multiple of $a$ and $b$ is denoted
\[
\operatorname{lcm}(a,b).
\]
\end{definition}

\begin{remark}[Intuition]
Intuitively: the lcm is the smallest number that both $a$ and $b$ divide
into evenly. Any other common multiple must itself be a multiple of the lcm.
\end{remark}

\begin{remark}[Relation with the Greatest Common Divisor]
For nonzero integers $a$ and $b$,
\[
\gcd(a,b)\,\operatorname{lcm}(a,b) = |ab|.
\]

Thus the greatest common divisor measures the
\emph{shared divisibility} of $a$ and $b$,
while the least common multiple measures their
\emph{combined multiplicative content}.

In prime factorization terms,
$\gcd$ takes the minimum exponent of each prime,
whereas $\operatorname{lcm}$ takes the maximum exponent.
\end{remark}

\begin{theorem}[Fundamental Theorem of Arithmetic]
Every integer $n > 1$ is either prime or can be written as a product of primes.
Moreover, this factorization is unique up to the order of the factors.

That is, if
\[
n = p_1 p_2 \cdots p_r
\quad\text{and}\quad
n = q_1 q_2 \cdots q_s,
\]
where each $p_i$ and $q_j$ is prime, then
\[
r = s,
\]
and after a reordering of the $q_j$,
\[
p_i = q_i \quad \text{for all } i.
\]
\end{theorem}

\begin{proof}
\textbf{(Existence)}

We prove by strong induction on $n > 1$ that $n$ is a prime or a product
of primes.

\medskip
\emph{Base case:}
For $n=2$, the statement holds because $2$ is prime.

\medskip
\emph{Inductive step:}
Assume every integer $m$ with $2 \le m < n$ is prime or a product of primes.
If $n$ is prime, we are done.

If $n$ is composite, then
\[
n = ab
\quad\text{with}\quad
1 < a < n, \; 1 < b < n.
\]
By the induction hypothesis, both $a$ and $b$ are products of primes.
Therefore $n = ab$ is also a product of primes.

Thus every integer greater than $1$ is prime or a product of primes.

\medskip
\textbf{(Uniqueness)}

Suppose
\[
n = p_1 p_2 \cdots p_r
\quad\text{and}\quad
n = q_1 q_2 \cdots q_s,
\]
where all $p_i$ and $q_j$ are primes.

Since $p_1 \mid n$, we have
\[
p_1 \mid q_1 q_2 \cdots q_s.
\]
By Euclid's Lemma, $p_1$ divides some $q_j$.
Because $q_j$ is prime, it follows that
\[
p_1 = q_j.
\]

After reordering, we may assume $p_1 = q_1$.
Canceling this common factor gives
\[
p_2 \cdots p_r = q_2 \cdots q_s.
\]

Repeating this argument inductively, we conclude that
\[
r = s
\quad\text{and}\quad
p_i = q_i \text{ for all } i
\]
after a suitable reordering.

Therefore, the prime factorization of $n$ is unique up to order.
\end{proof}

\begin{theorem}[Structural Relation Between $\gcd$ and $\operatorname{lcm}$]
Let $a$ and $b$ be nonzero integers. Then
\[
\gcd(a,b)\,\operatorname{lcm}(a,b) = |ab|.
\]
\end{theorem}

\begin{proof}
Let
\[
d = \gcd(a,b).
\]
Then there exist integers $a_1$ and $b_1$ such that
\[
a = d a_1, \qquad b = d b_1,
\]
and $\gcd(a_1,b_1)=1$.

Now
\[
ab = d^2 a_1 b_1.
\]

We claim that $\operatorname{lcm}(a,b) = d a_1 b_1$.
To see this, note that $a \mid d a_1 b_1$ (since $d a_1 b_1 = a \cdot b_1$)
and $b \mid d a_1 b_1$ (since $d a_1 b_1 = b \cdot a_1$).
If $c$ is any common multiple of $a$ and $b$, write $c = a u = b v$
for integers $u, v$. Then $d a_1 u = d b_1 v$, so $a_1 u = b_1 v$.
Since $\gcd(a_1, b_1) = 1$, we have $b_1 \mid u$, say $u = b_1 w$,
giving $c = a u = d a_1 b_1 w$. Hence $d a_1 b_1 \mid c$, confirming
that $d a_1 b_1$ is the least common multiple.

Therefore,
\[
\gcd(a,b)\,\operatorname{lcm}(a,b)
= d \cdot (d a_1 b_1)
= d^2 a_1 b_1
= |ab|.
\]

Taking absolute value accounts for possible signs of $a$ and $b$.
\end{proof}

\begin{remark}[Intuition]
Intuitively: $a$ and $b$ together contain a fixed amount of prime material.
The gcd captures what they share; the lcm captures everything between them.
Together they account for the full prime content of $ab$.
\end{remark}

% ---------------------------------------------------------
\subsubsection{Consequences and Logical Implications}
% ---------------------------------------------------------

\begin{remark}[Logical Dependency Chain]
The development of the integers in this section follows the chain
\[
\begin{aligned}
\text{Well-Ordering Principle}
&\;\Rightarrow\;
\text{Division Algorithm}
\;\Rightarrow\;
\text{Bézout's Identity}
\\[4pt]
&\;\Rightarrow\;
\text{Euclid's Lemma}
\;\Rightarrow\;
\text{Fundamental Theorem of Arithmetic}.
\end{aligned}
\]

Thus the order structure of $\mathbb{N}$ ultimately governs
prime factorization in $\mathbb{Z}$.
\end{remark}

\begin{remark}[Equivalences]
The following principles are logically equivalent:

\begin{itemize}
\item The Well-Ordering Principle.
\item The Principle of Mathematical Induction.
\item The Least Element Principle.
\end{itemize}

Each encodes the same structural property:
the integers admit no infinite strictly descending chains.
\end{remark}

\begin{remark}[Divisibility Structure]
Bézout's Identity upgrades divisibility into a linear-combination statement:
\[
\gcd(a,b)
=
\min \{\, as+bt>0 : s,t\in\mathbb{Z} \,\}.
\]

Thus the gcd is characterized by a universal property:
it is the greatest element (under divisibility)
among common divisors of $a$ and $b$.
\end{remark}

\begin{remark}[Prime Structure]
Euclid's Lemma implies that primes behave rigidly under multiplication:
\[
p \mid ab \Rightarrow p \mid a \text{ or } p \mid b.
\]

This rigidity is exactly what makes unique prime factorization possible.
Without Euclid's Lemma, uniqueness would fail.
\end{remark}

\begin{remark}[Arithmetic Decomposition]
The Fundamental Theorem of Arithmetic shows that
every integer $n>1$ decomposes uniquely into prime powers.
Thus $\mathbb{Z}$ is a \emph{Unique Factorization Domain (UFD)}.
\end{remark}

\begin{remark}[Duality of $\gcd$ and $\operatorname{lcm}$]
The identity
\[
\gcd(a,b)\,\operatorname{lcm}(a,b)=|ab|
\]
reveals a structural duality:

\begin{itemize}
\item $\gcd$ measures shared prime factors (intersection).
\item $\operatorname{lcm}$ measures total prime coverage (union).
\end{itemize}

Together they partition the prime-power structure of $ab$.
\end{remark}

\begin{remark}[Structural Position]
This section establishes that:

\begin{itemize}
\item $\mathbb{Z}$ has a well-ordered positive part.
\item Divisibility can be analyzed via minimal elements.
\item Primes control multiplicative structure.
\item Linear combinations control common divisors.
\end{itemize}

These ideas generalize later to:

\begin{itemize}
\item Euclidean domains,
\item Principal ideal domains,
\item Unique factorization domains.
\end{itemize}
\end{remark}