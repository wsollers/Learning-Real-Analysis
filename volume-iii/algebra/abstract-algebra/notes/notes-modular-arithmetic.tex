% =========================================================
% NOTES TEMPLATE (Definitions → Theorems/Proofs → Consequences)
% File: notes-modular-arithmetic.tex
% =========================================================

\subsection{Modular Arithmetic}
\label{sec:modular-arithmetic}

% ---------------------------------------------------------
\subsubsection{Basic Definitions and Theorems}

\begin{definition}[Remainder (Modular Reduction)]
Let $a \in \mathbb{Z}$ and $n \in \mathbb{Z}^+$. We say $a \bmod n$ is the unique integer $r$ satisfying
\[
a = qn + r, \quad 0 \leq r < n,
\]
for some integer $q$. The value of $a \bmod n$ is $r$. We call $r$ the \textit{remainder}, $q$ the \textit{quotient}, and $n$ the \textit{modulus} of $a$ upon division by $n$.
\end{definition}

\begin{remark}[Computing $a \bmod n$]
To compute $a \bmod n$, apply the Division Algorithm:
divide $a$ by $n$, and the remainder $r$ is the value of $a \bmod n$.

\medskip
\noindent\textbf{Examples.}
\begin{itemize}
\item $17 \bmod 5 = 2$, \quad since $17 = 3 \cdot 5 + 2$.
\item $20 \bmod 4 = 0$, \quad since $20 = 5 \cdot 4 + 0$.
\item $3 \bmod 7 = 3$, \quad since $3 = 0 \cdot 7 + 3$.
\item $-1 \bmod 5 = 4$, \quad since $-1 = (-1) \cdot 5 + 4$ and $0 \leq 4 < 5$.
\end{itemize}

Note the last example carefully: for negative integers,
the remainder is still required to satisfy $0 \le r < n$,
so the quotient $q$ may be negative.

\medskip
Equivalently, using the floor function,
\[
a \bmod n = a - n\left\lfloor \frac{a}{n} \right\rfloor,
\]
where $\lfloor x \rfloor$ denotes the greatest integer less than or equal to $x$.
This formula computes $q = \lfloor a/n \rfloor$ and then recovers $r = a - qn$.
\end{remark}

\begin{remark}[Intuition]
Intuitively: $a \bmod n$ asks ``after fitting as many full copies of $n$
into $a$ as possible, what is left over?'' The remainder is always
in $[0, n)$ regardless of the sign of $a$.
\end{remark}

\begin{lemma}[Idempotence of $\bmod$]
For any $a \in \mathbb{Z}$ and $n \in \mathbb{Z}^+$,
\[
(a \bmod n) \bmod n = a \bmod n.
\]
\end{lemma}



\begin{remark}[Intuition]
Intuitively: once a value has been reduced modulo $n$,
it is already in the range $[0, n)$, so reducing again changes nothing.
\end{remark}

\begin{remark}[Applying $\bmod n$ to an Equation Term by Term]
A key technique in modular arithmetic is reducing an entire equation
modulo $n$ by applying $\bmod n$ to each term. This works because
multiples of $n$ vanish under $\bmod n$:
\[
kn \bmod n = 0 \quad \text{for any } k \in \mathbb{Z}.
\]

\medskip
\noindent\textbf{General procedure.}
Write each integer using the Division Algorithm:
\[
a = q_1 n + r_1, \qquad b = q_2 n + r_2.
\]
Then any expression built from $a$ and $b$ using addition and multiplication
can be reduced modulo $n$ by discarding all multiples of $n$
and retaining only the remainders.

\medskip
\noindent\textbf{Example: addition.}
\[
a + b = (q_1 + q_2)n + (r_1 + r_2),
\]
so
\[
(a + b) \bmod n = (r_1 + r_2) \bmod n = (a \bmod n + b \bmod n) \bmod n.
\]
The extra outer $\bmod n$ is needed because $r_1 + r_2$ may exceed $n$.

\medskip
\noindent\textbf{Example: multiplication.}
\[
ab = (q_1 n + r_1)(q_2 n + r_2)
   = q_1 q_2 n^2 + q_1 r_2 n + q_2 r_1 n + r_1 r_2.
\]
Every term except $r_1 r_2$ is a multiple of $n$, so
\[
(ab) \bmod n = (r_1 r_2) \bmod n = (a \bmod n)(b \bmod n) \bmod n.
\]

\medskip
\noindent\textbf{Concrete example.}
Compute $(17 \cdot 13) \bmod 5$.
\[
17 \bmod 5 = 2, \qquad 13 \bmod 5 = 3, \qquad 2 \cdot 3 = 6, \qquad 6 \bmod 5 = 1.
\]
Check: $17 \cdot 13 = 221$ and $221 = 44 \cdot 5 + 1$, so $221 \bmod 5 = 1$. \checkmark

\medskip
\noindent\textbf{Why this works.}
Multiples of $n$ contribute $0$ under $\bmod n$.
So when expanding any polynomial expression in $a$ and $b$,
all cross-terms involving $n$ vanish, leaving only the product of remainders.
This is formalized in the Compatibility with Arithmetic theorem below.
\end{remark}

\begin{remark}[Intuition]
Intuitively: reducing modulo $n$ is like working on a clock face with
$n$ positions --- you only ever care about where you land, not how many
full laps you made to get there. Addition and multiplication just move
you around the clock, and multiples of $n$ bring you back to $0$.
\end{remark}

\begin{definition}[Congruence Modulo $n$]
Let $n$ be a positive integer.
For integers $a$ and $b$, we say that
\[
a \equiv b \pmod{n}
\]
if $n$ divides $a-b$; that is,
\[
n \mid (a-b).
\]
This relation is called \emph{congruence modulo $n$}.
\end{definition}

\begin{remark}[Unpacking the Definition]
The statement
\[
a \equiv b \pmod{n}
\]
means that $a$ and $b$ leave the same remainder upon division by $n$.

Equivalently, there exists an integer $k$ such that
\[
a = b + kn.
\]

Thus two integers are congruent modulo $n$ precisely when they differ
by a multiple of $n$.
\end{remark}

\begin{remark}[Intuition]
Intuitively: two integers are congruent modulo $n$ if they land on the
same position when you wrap the number line around a circle of
circumference $n$. The difference between them is always an exact
number of full laps.
\end{remark}

\begin{definition}[Congruence Class]
The \emph{congruence class} of $a$ modulo $n$ is the set of all integers
congruent to $a$ modulo $n$:
\[
[a]_n = \{ a + kn : k \in \mathbb{Z} \}.
\]
\end{definition}

\begin{remark}
The remainder $r$ in the Division Algorithm $a = bq + r$ is also called the
\emph{residue} of $a$ modulo $b$, and $[a]_n$ is also called a \emph{residue class}.
The collection of all congruence classes modulo $n$ forms the set
\[
\mathbb{Z}/n\mathbb{Z} = \bigl\{\, [0]_n,\, [1]_n,\, \ldots,\, [n-1]_n \,\bigr\}.
\]
$\left|\mathbb{Z}/n\mathbb{Z}\right| = n$, meaning $\mathbb{Z}/n\mathbb{Z}$ has exactly $n$ elements,
corresponding to the $n$ possible remainders upon division by $n$.
\end{remark}

\begin{remark}[Intuition]
Intuitively: congruence classes partition all of $\mathbb{Z}$ into $n$
equally spaced infinite families. Every integer belongs to exactly one
class, determined by its remainder upon division by $n$.
\end{remark}

\begin{theorem}[Congruence is an Equivalence Relation]
Let $n$ be a positive integer. Congruence modulo $n$ is an equivalence
relation on $\mathbb{Z}$. That is, for all $a, b, c \in \mathbb{Z}$:

\begin{enumerate}
\item \textbf{Reflexivity:} $a \equiv a \pmod{n}$.
\item \textbf{Symmetry:} If $a \equiv b \pmod{n}$, then $b \equiv a \pmod{n}$.
\item \textbf{Transitivity:} If $a \equiv b \pmod{n}$ and
      $b \equiv c \pmod{n}$, then $a \equiv c \pmod{n}$.
\end{enumerate}
\end{theorem}



\begin{remark}[Intuition]
Intuitively: an equivalence relation is just a precise way of saying
``these things behave the same for our purposes.'' Congruence mod $n$
groups integers by their remainder --- same remainder, same class,
same behavior under arithmetic mod $n$.
\end{remark}

\begin{theorem}[Compatibility with Arithmetic]
Let $n$ be a positive integer and suppose
\[
a \equiv b \pmod{n}
\quad\text{and}\quad
c \equiv d \pmod{n}.
\]
Then:
\begin{enumerate}
\item $a + c \equiv b + d \pmod{n}$,
\item $a - c \equiv b - d \pmod{n}$,
\item $ac \equiv bd \pmod{n}$.
\end{enumerate}
\end{theorem}



\begin{remark}[Intuition]
Intuitively: you can swap any integer for a congruent one before doing
arithmetic, and the result will still be congruent to what you would
have gotten. This is what makes working with representatives of classes
--- rather than the classes themselves --- legitimate.
\end{remark}

\begin{remark}[Consequence for $\mathbb{Z}/n\mathbb{Z}$]
The compatibility theorem means that addition and multiplication
of congruence classes are well-defined:
\[
[a]_n + [b]_n := [a+b]_n,
\qquad
[a]_n \cdot [b]_n := [ab]_n.
\]
The choice of representative within a class does not affect the result.
Under these operations, $\mathbb{Z}/n\mathbb{Z}$ is a commutative ring.
\end{remark}


% ---------------------------------------------------------
\subsubsection{Consequences and Logical Implications}
% ---------------------------------------------------------

\begin{remark}[Connection to the Integers]
Congruence modulo $n$ is the first instance of a general construction:
forming a \emph{quotient} of $\mathbb{Z}$ by identifying elements that
differ by a multiple of $n$. The set $\mathbb{Z}/n\mathbb{Z}$ is the
quotient of $\mathbb{Z}$ by the subgroup $n\mathbb{Z} = \{\ldots, -2n, -n, 0, n, 2n, \ldots\}$.

This construction depends directly on the Division Algorithm:
every integer belongs to exactly one congruence class modulo $n$,
corresponding to its remainder upon division by $n$.
\end{remark}

\begin{remark}[When $\mathbb{Z}/n\mathbb{Z}$ is a Field]
The ring $\mathbb{Z}/n\mathbb{Z}$ is a field if and only if $n$ is prime.

When $n = p$ is prime, every nonzero element $[a]_p$ satisfies
$\gcd(a, p) = 1$, so by Bézout's Identity there exist integers $s, t$
with $as + pt = 1$, giving $[a]_p \cdot [s]_p = [1]_p$.
Thus every nonzero element has a multiplicative inverse.

When $n$ is composite, say $n = ab$ with $1 < a, b < n$,
the element $[a]_n$ is a zero divisor: $[a]_n \cdot [b]_n = [0]_n$
with neither factor zero, so $\mathbb{Z}/n\mathbb{Z}$ is not an integral domain
and hence not a field.
\end{remark}

\begin{remark}[Intuition]
Intuitively: a field requires every nonzero element to have a
multiplicative inverse. In $\mathbb{Z}/n\mathbb{Z}$, an element $[a]_n$
has an inverse exactly when $a$ and $n$ share no common factor ---
and when $n$ is prime, this is true for every nonzero element.
\end{remark}

\begin{remark}[Structural Position]
Modular arithmetic occupies the boundary between the integers
and abstract algebra:

\begin{itemize}
\item It depends on the Division Algorithm and gcd theory developed in
      the integers section.
\item It is the prototype for quotient ring constructions, which generalize
      to arbitrary ideals in commutative ring theory.
\item The fields $\mathbb{Z}/p\mathbb{Z}$ are the simplest finite fields,
      foundational in algebra, number theory, and cryptography.
\end{itemize}
\end{remark}