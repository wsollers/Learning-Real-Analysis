% =========================================================
% Relations and Functions
% File: notes-relations-and-functions.tex
% =========================================================

\subsection{Relations and Functions}

% ---------------------------------------------------------
\subsubsection{Definitions and Theorems}
% ---------------------------------------------------------

% ---------------------------------------------------------
% Relations
% ---------------------------------------------------------

\begin{definition}[Binary Relation]
Let $A$ and $B$ be sets.
A \emph{binary relation} from $A$ to $B$ is a subset
\[
R \subseteq A \times B.
\]

If $(a,b) \in R$, we write
\[
a \,R\, b.
\]

If $A = B$, we call $R$ a \emph{relation on $A$}.
\end{definition}

\begin{definition}[Properties of Relations]
Let $R$ be a relation on a set $S$.

\begin{itemize}
\item \textbf{Reflexive:}
\[
\forall a \in S,\ (a,a) \in R.
\]

\item \textbf{Symmetric:}
\[
(a,b) \in R \Rightarrow (b,a) \in R.
\]

\item \textbf{Transitive:}
\[
(a,b) \in R \land (b,c) \in R
\Rightarrow (a,c) \in R.
\]

\item \textbf{Antisymmetric:}
\[
(a,b) \in R \land (b,a) \in R
\Rightarrow a=b.
\]
\end{itemize}
\end{definition}

\begin{definition}[Equivalence Relation]
A relation $R$ on $S$ is an \emph{equivalence relation}
if it is reflexive, symmetric, and transitive.
\end{definition}

\begin{definition}[Equivalence Class]
Let $R$ be an equivalence relation on $S$.
For $a \in S$, the \emph{equivalence class} of $a$ is
\[
[a] := \{\, x \in S : (x,a) \in R \}
= \{\, x \in S : x R a \,\}.
\]
\end{definition}

\begin{theorem}[Equivalence Relations and Partitions]
Let $R$ be an equivalence relation on a set $S$.
Then the set of equivalence classes of $R$
forms a partition of $S$.

Conversely, every partition of $S$
determines an equivalence relation on $S$.
\end{theorem}



% ---------------------------------------------------------
% Functions
% ---------------------------------------------------------

\begin{definition}[Function (Mapping)]
Let $A$ and $B$ be sets.
A \emph{function} from $A$ to $B$ is a relation
$f \subseteq A \times B$ such that:

\begin{enumerate}
\item For every $a \in A$, there exists $b \in B$
with $(a,b) \in f$.
\item If $(a,b_1) \in f$ and $(a,b_2) \in f$,
then $b_1 = b_2$.
\end{enumerate}

We write $f : A \to B$.
\end{definition}

\begin{definition}[Injective, Surjective, Bijective]
Let $f : A \to B$.

\begin{itemize}
\item Injective:
\[
f(a_1)=f(a_2) \Rightarrow a_1=a_2.
\]

\item Surjective:
\[
\forall b \in B,\ \exists a \in A \text{ such that } f(a)=b.
\]

\item Bijective: both injective and surjective.
\end{itemize}
\end{definition}

\begin{definition}[Image and Preimage]
Let $f : A \to B$.

\[
f(A)=\{f(a):a\in A\}.
\]

For $C \subseteq B$,
\[
f^{-1}(C)=\{a\in A : f(a)\in C\}.
\]
\end{definition}

\begin{definition}[Composition of Functions]
Let $f : A \to B$ and $g : B \to C$.
Define
\[
(g\circ f)(a):=g(f(a)).
\]
\end{definition}

\begin{theorem}[Basic Properties of Function Composition]
Let
\[
a:A\to B,\quad b:B\to C,\quad g:C\to D.
\]

\begin{enumerate}
\item Associativity:
\[
g\circ(b\circ a)=(g\circ b)\circ a.
\]

\item If $a$ and $b$ are injective, then $b\circ a$ is injective.

\item If $a$ and $b$ are surjective, then $b\circ a$ is surjective.

\item If $a$ is bijective, then there exists
$a^{-1}:B\to A$ such that
\[
a^{-1}\circ a=\operatorname{id}_A,
\quad
a\circ a^{-1}=\operatorname{id}_B.
\]
\end{enumerate}
\end{theorem}

% ---------------------------------------------------------
\subsubsection{Consequences}
% ---------------------------------------------------------

\begin{remark}[Structural Map]
\[
\text{Relation}
\supset
\text{Function}
\supset
\text{Bijective Function}.
\]
\end{remark}

\begin{remark}[Equivalence Relations Produce Quotients]
If $R$ is an equivalence relation on $S$, then
\[
S/R=\{[a]:a\in S\}
\]
is the associated quotient set.
\end{remark}

\begin{remark}[Partitions vs.\ Structure]
Equivalence relations encode structural indistinguishability.
Partitions encode decomposition.
These viewpoints are equivalent.
\end{remark}

\begin{remark}[Upgrade Path]
These concepts reappear as:
\begin{itemize}
\item Congruence modulo $n$,
\item Cosets and quotient groups,
\item Kernels of homomorphisms,
\item Quotient spaces in topology.
\end{itemize}
\end{remark}
