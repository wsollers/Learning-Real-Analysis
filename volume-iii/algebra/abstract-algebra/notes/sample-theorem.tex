\clearpage

% Column types: T = tag column, S = step column, J = justification column
\newcolumntype{T}{p{0.06\textwidth}}
\newcolumntype{S}{>{\everymath{\displaystyle}$\displaystyle}p{0.44\textwidth}<{$}}
\newcolumntype{J}{p{0.40\textwidth}}

\newmdenv[
  backgroundcolor=gray!10,
  linecolor=gray!40,
  innerleftmargin=8pt,
  innerrightmargin=8pt,
  innertopmargin=6pt,
  innerbottommargin=6pt
]{subproof}


\begin{center}
  {\large\bfseries Example Proof}\\[4pt]
  {\large $ab = \operatorname{lcm}(a,b)\,\gcd(a,b)$}\\[2pt]
  {\small\itshape (Three-column format: tag / step / justification)}\\[6pt]
  {\small
    \tagDU\ = Definition Unpacked \quad
    \tagTA\ = Theorem Applied \quad
    \tagAM\ = Algebraic Manipulation
  }
\end{center}

\begin{theorem}
Let $a$ and $b$ be positive integers. Then
\[
  ab \;=\; \operatorname{lcm}(a,b)\,\gcd(a,b).
\]
\end{theorem}

\medskip
\noindent\textbf{Proof.}
\bigskip

\noindent\textit{Part 1: Setup --- write $a$ and $b$ in terms of their gcd.}
\medskip

\noindent
\begin{tabular}{T S J}
\toprule
\textbf{Tag} & \multicolumn{1}{p{0.44\textwidth}}{\textbf{Step}} & \textbf{Justification} \\
\midrule
\addlinespace[4pt]

\tagDU
&
\text{Let } d = \gcd(a,b).
&
Definition of gcd; $d$ is the greatest common divisor of $a$ and $b$.
\\[10pt]

\tagDU
&
\begin{aligned}
  &\text{Write } a = d a_1 \text{ and } b = d b_1 \\
  &\text{for positive integers } a_1, b_1.
\end{aligned}
&
Since $d \mid a$ and $d \mid b$, we can factor $d$ out of each.
\\[10pt]

\tagTA
&
\gcd(a_1, b_1) = 1.
&
\textit{(Sub-argument by contradiction --- see box below.)}
\\[6pt]

\bottomrule
\end{tabular}

\medskip

\begin{subproof}
\noindent\textbf{Sub-argument (contradiction):}
Why must $\gcd(a_1, b_1) = 1$?

\smallskip
\noindent
\begin{tabular}{T S J}
\toprule
\textbf{Tag} & \multicolumn{1}{p{0.44\textwidth}}{\textbf{Step}} & \textbf{Justification} \\
\midrule
\addlinespace[4pt]

\tagDU
&
\text{Suppose } \gcd(a_1, b_1) = c > 1.
&
Assume for contradiction that $a_1$ and $b_1$ share a common factor.
\\[10pt]

\tagDU
&
\text{Then } c \mid a_1 \text{ and } c \mid b_1.
&
Definition of common divisor.
\\[10pt]

\tagAM
&
\text{So } dc \mid da_1 = a \text{ and } dc \mid db_1 = b.
&
Multiplying both sides of each divisibility by $d$.
\\[10pt]

\tagDU
&
\text{Thus } dc \text{ is a common divisor of } a \text{ and } b.
&
It divides both $a$ and $b$.
\\[10pt]

\tagAM
&
dc > d.
&
Since $c > 1$.
\\[10pt]

\tagDU
&
\text{This contradicts } d = \gcd(a,b).
&
$d$ is the \emph{greatest} common divisor, so no common divisor
can exceed $d$. Contradiction. $\square$
\\[6pt]

\bottomrule
\end{tabular}

\smallskip
\noindent Therefore $\gcd(a_1, b_1) = 1$; that is, $a_1$ and $b_1$ are coprime.
\end{subproof}

\bigskip

\noindent\textit{Part 2: Identify $\operatorname{lcm}(a,b)$.}
\medskip

\noindent
\begin{tabular}{T S J}
\toprule
\textbf{Tag} & \multicolumn{1}{p{0.44\textwidth}}{\textbf{Step}} & \textbf{Justification} \\
\midrule
\addlinespace[4pt]

\tagDU
&
\text{Claim: } \operatorname{lcm}(a,b) = d a_1 b_1.
&
We verify both conditions in the definition of lcm.
\\[10pt]

\tagDU
&
a \mid d a_1 b_1
\quad\text{since}\quad
d a_1 b_1 = a \cdot b_1.
&
$a = d a_1$, so $d a_1 b_1 = (d a_1) b_1 = a b_1$.
Hence $a \mid d a_1 b_1$.
\\[10pt]

\tagDU
&
b \mid d a_1 b_1
\quad\text{since}\quad
d a_1 b_1 = b \cdot a_1.
&
$b = d b_1$, so $d a_1 b_1 = a_1 (d b_1) = a_1 b$.
Hence $b \mid d a_1 b_1$.
\\[10pt]

\tagTA
&
\text{If } a \mid c \text{ and } b \mid c,
\text{ then } d a_1 b_1 \mid c.
&
Write $c = a u = d a_1 u$ and $c = b v = d b_1 v$.
Then $a_1 u = b_1 v$.
Since $\gcd(a_1, b_1)=1$, we get $b_1 \mid u$,
say $u = b_1 w$. Then $c = d a_1 b_1 w$,
so $d a_1 b_1 \mid c$.
\\[10pt]

\tagDU
&
\therefore\; \operatorname{lcm}(a,b) = d a_1 b_1.
&
Both conditions of the lcm definition are satisfied.
\\[6pt]

\bottomrule
\end{tabular}

\bigskip

\noindent\textit{Part 3: Compute $\operatorname{lcm}(a,b)\,\gcd(a,b)$.}
\medskip

\noindent
\begin{tabular}{T S J}
\toprule
\textbf{Tag} & \multicolumn{1}{p{0.44\textwidth}}{\textbf{Step}} & \textbf{Justification} \\
\midrule
\addlinespace[4pt]

\tagAM
&
\operatorname{lcm}(a,b)\,\gcd(a,b) = (d a_1 b_1) \cdot d
&
Substituting $\operatorname{lcm}(a,b) = d a_1 b_1$ and $\gcd(a,b)=d$.
\\[10pt]

\tagAM
&
= d^2 a_1 b_1
&
Collecting the two factors of $d$.
\\[10pt]

\tagAM
&
= (d a_1)(d b_1)
&
Regrouping.
\\[10pt]

\tagAM
&
= ab
&
Since $a = d a_1$ and $b = d b_1$.
\\[6pt]

\bottomrule
\end{tabular}

\medskip
\noindent Therefore $ab = \operatorname{lcm}(a,b)\,\gcd(a,b)$. \hfill$\blacksquare$

\bigskip\bigskip

\noindent\hrule
\medskip
\noindent\textbf{Study Notes.}
\medskip
\noindent\textit{Where did each tool appear?}
\medskip

\noindent
\begin{tabular}{p{0.25\textwidth} p{0.65\textwidth}}
\toprule
\textbf{Tool} & \textbf{Role in this proof} \\
\midrule
\addlinespace[4pt]
Definition of gcd
&
Gave us $d \mid a$ and $d \mid b$, allowing us to write $a = da_1$, $b = db_1$.
\\[8pt]
Proof by contradiction
&
Embedded sub-argument showing $\gcd(a_1, b_1) = 1$.
The key: if they shared a factor $c>1$, then $dc > d$ would be
a common divisor, contradicting $d$ being \emph{greatest}.
\\[8pt]
Definition of lcm (universal property)
&
Used to \emph{verify} the candidate $da_1b_1$
rather than just assert it. Required checking both
(i) it is a common multiple, and (ii) it divides every common multiple.
\\[8pt]
Coprimality of $a_1, b_1$
&
The essential ingredient in Part 2 that makes the
minimality argument work: $\gcd(a_1,b_1)=1$ forced $b_1 \mid u$.
\\[8pt]
Algebraic regrouping
&
The final computation is just $d^2 a_1 b_1 = (da_1)(db_1)$.
All the work was in setting up the right objects.
\\[6pt]
\bottomrule
\end{tabular}