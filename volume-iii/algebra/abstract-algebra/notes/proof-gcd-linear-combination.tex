\clearpage
\begin{center}
  {\large\bfseries Proof GCD is a linear combination}\\[4pt]
  {\large $\gcd(a,b) = as + bt$ for some $s,t \in \mathbb{Z}$}\\[2pt]
  {\small\itshape (Three-column format: tag / step / justification)}\\[6pt]
  {\small
    \tagDU\ = Definition Unpacked \quad
    \tagTA\ = Theorem Applied \quad
    \tagAM\ = Algebraic Manipulation
  }
\end{center}

\begin{theorem}[GCD as Smallest Positive Linear Combination]
Let $a$ and $b$ be positive integers. Then $\gcd(a,b)$ is the smallest positive element of the set
\[
  S = \{ am + bn \mid m,n \in \mathbb{Z},\; am+bn > 0 \}.
\]
\end{theorem}

\medskip
\noindent\textbf{Proof.}
\bigskip

\noindent\textit{Part 1: S is nonempty and has a smallest member.}
\medskip

\noindent
\begin{tabular}{T S J}
\toprule
\textbf{Tag} & \multicolumn{1}{p{0.44\textwidth}}{\textbf{Step}} & \textbf{Justification} \\
\midrule
\addlinespace[4pt]

\tagDU
&
S = \{ am + bn \mid m,n \in \mathbb{Z},\; am+bn > 0 \}.
&
Define $S$ as the set of all positive integer linear combinations of $a$ and $b$.
\\[10pt]

\tagDU
&
S \neq \emptyset.
&
If some $m,n$ give $am+bn < 0$, replace $m,n$ with $-m,-n$ to get a positive value.
\\[10pt]

\tagTA
&
\text{Let } d = as + bt \text{ be the smallest member of } S.
&
By the Well-Ordering Principle, every nonempty set of positive integers has a least element.
\\[6pt]

\bottomrule
\end{tabular}

\bigskip

\noindent\textit{Part 2: $d$ divides both $a$ and $b$.}
\medskip

\noindent
\begin{tabular}{T S J}
\toprule
\textbf{Tag} & \multicolumn{1}{p{0.44\textwidth}}{\textbf{Step}} & \textbf{Justification} \\
\midrule
\addlinespace[4pt]

\tagTA
&
\text{Write } a = dq + r, \quad 0 \leq r < d.
&
Division Algorithm applied to $a$ and $d$.
\\[10pt]

\tagAM
&
r = a - dq = a - (as+bt)q = a(1-sq) + b(-tq).
&
Substituting $d = as+bt$ and expanding.
\\[10pt]

\tagDU
&
\text{If } r > 0, \text{ then } r \in S.
&
$r$ is a positive integer linear combination of $a$ and $b$.
\\[10pt]

\tagTA
&
r < d, \text{ contradicting } d = \min S.
&
But $r < d$ by the Division Algorithm, contradicting minimality of $d$.
\\[10pt]

\tagDU
&
\therefore\; r = 0, \text{ so } d \mid a.
&
The remainder must be zero; hence $d$ divides $a$.
\\[10pt]

\tagDU
&
d \mid b.
&
By the same argument applied symmetrically to $b$.
\\[10pt]

\tagDU
&
\therefore\; d \text{ is a common divisor of } a \text{ and } b.
&
$d$ divides both $a$ and $b$.
\\[6pt]

\bottomrule
\end{tabular}

\bigskip

\noindent\textit{Part 3: $d$ is the \emph{greatest} common divisor.}
\medskip

\noindent
\begin{tabular}{T S J}
\toprule
\textbf{Tag} & \multicolumn{1}{p{0.44\textwidth}}{\textbf{Step}} & \textbf{Justification} \\
\midrule
\addlinespace[4pt]

\tagDU
&
\text{Let } d' \text{ be any common divisor of } a \text{ and } b.
&
Suppose $d'$ is an arbitrary common divisor; write $a = d'h$ and $b = d'k$.
\\[10pt]

\tagAM
&
d = as + bt = (d'h)s + (d'k)t = d'(hs + kt).
&
Substituting $a = d'h$ and $b = d'k$ into $d = as+bt$.
\\[10pt]

\tagDU
&
\therefore\; d' \mid d.
&
$d$ is an integer multiple of $d'$, so $d' \leq d$.
\\[10pt]

\tagDU
&
\therefore\; d = \gcd(a,b).
&
$d$ is a common divisor of $a$ and $b$, and every other common divisor
divides $d$, so $d$ is greatest. \hfill$\blacksquare$
\\[6pt]

\bottomrule
\end{tabular}

\bigskip\bigskip

\noindent\hrule
\medskip
\noindent\textbf{Study Notes.}
\medskip
\noindent\textit{Where did each tool appear?}
\medskip

\noindent
\begin{tabular}{p{0.25\textwidth} p{0.65\textwidth}}
\toprule
\textbf{Tool} & \textbf{Role in this proof} \\
\midrule
\addlinespace[4pt]
Well-Ordering Principle
&
Guaranteed that $S$, being a nonempty set of positive integers, has a least element $d$.
\\[8pt]
Division Algorithm
&
Used to write $a = dq + r$ and derive that $r \in S$ if $r > 0$, forcing $r = 0$.
\\[8pt]
Proof by contradiction
&
If $r > 0$ then $r \in S$ with $r < d$, contradicting minimality of $d$.
\\[8pt]
Linear combination structure
&
The form $d = as + bt$ was essential in showing every common divisor $d'$ satisfies $d' \mid d$.
\\[8pt]
Symmetry argument
&
After showing $d \mid a$, the identical argument applies to $b$ without repeating the full proof.
\\[6pt]
\bottomrule
\end{tabular}