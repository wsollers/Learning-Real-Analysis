% --------------------------------------
% Proof: AA-GAL-C00-E53
% --------------------------------------
\clearpage
\phantomsection
\hypertarget{proof-AA-GAL-C00-E53}{}

\section*{Proof — AA-GAL-C00-E53}

\noindent
\hyperlink{ws-AA-GAL-C00-E53}{\textbf{← Back to worksheet}}

\noindent
\textbf{Source.}
\srccite{GallianCAA7}{Chapter~0, Exercise~53}.

\vspace{0.75em}

\noindent
\textbf{Goal.}
Let $S$ be the set of integers.
For $a,b \in S$, define $aRb$ if $ab \ge 0$.
Determine whether $R$ is an equivalence relation on $S$.

\vspace{0.75em}

\noindent
\begin{proof}
Let $S = \mathbb{Z}$ and define $aRb$ if and only if $ab \geq 0$.
We verify the three properties of an equivalence relation.

\medskip
\noindent\textbf{Case 1: Reflexivity.}
We must show that $aRa$ for all $a \in S$.
Let $a \in S$ be arbitrary. Then
\[
a \cdot a = a^2 \geq 0,
\]
since squares of integers are always nonnegative. Hence $aRa$ holds.

\medskip
\noindent\textbf{Case 2: Symmetry.}
We must show that if $aRb$ then $bRa$.
Let $a, b \in S$ be arbitrary and assume $aRb$, so $ab \geq 0$.
By commutativity of multiplication on $\mathbb{Z}$,
\[
b \cdot a = a \cdot b \geq 0.
\]
Hence $bRa$ holds.

\medskip
\noindent\textbf{Case 3: Transitivity.}
We must show that if $aRb$ and $bRc$ then $aRc$.
This fails. Take $a = 1$, $b = 0$, $c = -1$. Then
\begin{align*}
ab &= 1 \cdot 0 = 0 \geq 0, \quad \text{so } aRb \text{ holds,} \\
bc &= 0 \cdot (-1) = 0 \geq 0, \quad \text{so } bRc \text{ holds,} \\
ac &= 1 \cdot (-1) = -1 < 0, \quad \text{so } aRc \text{ fails.}
\end{align*}
Hence $R$ is not transitive.

\medskip
Since $R$ is not transitive, $R$ is not an equivalence relation on $S$. \qedhere
\end{proof}

\begin{remark}[How the counterexample was found]
The counterexample was motivated by sign analysis on the transitivity condition.
Suppose $aRb$ and $bRc$, so $ab \geq 0$ and $bc \geq 0$. There are two ways
these products can be nonnegative:

\medskip
\noindent\textbf{Case A: $b \neq 0$.}
If $b > 0$, then $ab \geq 0$ forces $a \geq 0$, and $bc \geq 0$ forces $c \geq 0$.
So $ac \geq 0$ and transitivity holds in this case.
If $b < 0$, then $ab \geq 0$ forces $a \leq 0$, and $bc \geq 0$ forces $c \leq 0$.
So again $ac \geq 0$ and transitivity holds.

\medskip
\noindent\textbf{Case B: $b = 0$.}
Then $ab = 0 \geq 0$ and $bc = 0 \geq 0$ regardless of the signs of $a$ and $c$.
This means $a$ and $c$ are completely unconstrained --- they can have opposite signs,
giving $ac < 0$. This is the case that breaks transitivity, and it motivated
choosing $a = 1$, $b = 0$, $c = -1$.
\end{remark}

\vspace{2cm}
\noindent
\hyperlink{ws-AA-GAL-C00-E53}{\textbf{← Back to worksheet}}
