% --------------------------------------
% Proof: AA-GAL-C00-E12
% --------------------------------------
\clearpage
\phantomsection
\hypertarget{proof-AA-GAL-C00-E12}{}

\section*{Proof — AA-GAL-C00-E12}

\noindent
\hyperlink{ws-AA-GAL-C00-E12}{\textbf{← Back to worksheet}}

\noindent
\textbf{Source.}
\srccite{GallianCAA7}{Chapter~0, Exercise~12}.

\vspace{0.75em}

\noindent
\textbf{Goal.}
Let $a$ and $b$ be positive integers and let
$d = \gcd(a,b)$ and $m = \operatorname{lcm}(a,b)$.
If $t$ divides both $a$ and $b$, prove that $t$ divides $d$.
If $s$ is a multiple of both $a$ and $b$, prove that $s$ is a multiple of $m$.

\vspace{0.75em}

\noindent
\textbf{Proof.}
\begin{proof}
Let $a$ and $b$ be positive integers and let $d = \gcd(a,b)$.

\textbf{Part 1.} Assume $t$ divides both $a$ and $b$. Write $a = tx$ and $b = ty$ for some $x, y \in \mathbb{Z}$. Then for any $u, v \in \mathbb{Z}$,
\[
ua + vb = u(tx) + v(ty) = t(ux + vy),
\]
so $t$ divides any linear combination of $a$ and $b$. By Bézout's theorem, $d$ is a linear combination of $a$ and $b$, hence $t \mid d$.

\textbf{Part 2.} [Proof that if $s$ is a multiple of both $a$ and $b$, then $m \mid s$.]
\end{proof>

\vspace{2cm}
\noindent
\hyperlink{ws-AA-GAL-C00-E12}{\textbf{← Back to worksheet}}
