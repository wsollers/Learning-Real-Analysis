% --------------------------------------
% Proof: AA-GAL-C00-E54
% --------------------------------------
\clearpage
\phantomsection
\hypertarget{proof-AA-GAL-C00-E54}{}

\section*{Proof — AA-GAL-C00-E54}

\noindent
\hyperlink{ws-AA-GAL-C00-E54}{\textbf{← Back to worksheet}}

\noindent
\textbf{Source.}
\srccite{GallianCAA7}{Chapter~0, Exercise~54}.

\vspace{0.75em}

\noindent
\textbf{Goal.}
Let $S$ be the set of integers.
For $a,b \in S$, define $aRb$ if $a+b$ is even.
Prove that $R$ is an equivalence relation and determine
the equivalence classes of $S$.

\vspace{0.75em}

\noindent
\begin{proof}
Let $S = \mathbb{Z}$ and define $aRb$ if and only if $a + b$ is even.
We verify the three properties of an equivalence relation.

\medskip
\noindent\textbf{Case 1: Reflexivity.}
We must show that $aRa$ for all $a \in S$.
Let $a \in S$ be arbitrary. Then
\[
a + a = 2a,
\]
which is even since it has the form $2m$. Hence $aRa$ holds.

\medskip
\noindent\textbf{Case 2: Symmetry.}
We must show that if $aRb$ then $bRa$.
Let $a, b \in S$ be arbitrary and assume $aRb$, so $a + b$ is even.
By commutativity of addition on $\mathbb{Z}$,
\[
b + a = a + b,
\]
which is even. Hence $bRa$ holds.

\medskip
\noindent\textbf{Case 3: Transitivity.}
We must show that if $aRb$ and $bRc$ then $aRc$.
Let $a, b, c \in S$ be arbitrary and assume $aRb$ and $bRc$.
Then there exist integers $n, m$ such that
\begin{align*}
a + b &= 2n \\
b + c &= 2m.
\end{align*}
Adding these equations and solving for $a + c$ gives
\[
a + c = 2n + 2m - 2b = 2(n + m - b).
\]
Since $n + m - b \in \mathbb{Z}$, $a + c$ is even. Hence $aRc$ holds.

\medskip
Since $R$ is reflexive, symmetric, and transitive, $R$ is an equivalence relation
on $S$. \qedhere
\end{proof}

\vspace{2cm}
\noindent
\hyperlink{ws-AA-GAL-C00-E54}{\textbf{← Back to worksheet}}
