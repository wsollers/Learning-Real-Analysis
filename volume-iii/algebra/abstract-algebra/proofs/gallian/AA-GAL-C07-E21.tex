% --------------------------------------
% Proof: AA-GAL-C07-E21
% --------------------------------------
\clearpage
\phantomsection
\hypertarget{proof-AA-GAL-C07-E21}{}

\section*{Proof — AA-GAL-C07-E21}

\noindent
\hyperlink{ws-AA-GAL-C07-E21}{\textbf{← Back to worksheet}}

\noindent
\textbf{Source.}
\srccite{GallianCAA7}{Chapter~7, Exercise~21}.

\vspace{0.75em}

\noindent
\textbf{Goal.}
Prove that there are infinitely many primes.

\vspace{0.75em}

\begin{proof}
Suppose for contradiction that there are only finitely many primes.
Then there exists a largest prime $p_n$, and we may list all primes as
$p_1, p_2, \ldots, p_n$. Construct
\[
N := p_1 p_2 \cdots p_n, \qquad Q := N + 1.
\]
Since each $p_i \mid N$, dividing $Q$ by any $p_i$ leaves remainder $1$.
Thus no prime in our list divides $Q$. There are two cases.

\medskip
\noindent\textbf{Case 1.} $Q$ is prime. Then $Q$ is a prime not in the list
$p_1, p_2, \ldots, p_n$, contradicting that $p_n$ is the largest prime.

\medskip
\noindent\textbf{Case 2.} $Q$ is composite. Then $Q$ has a prime factor $p$.
Since no $p_i$ divides $Q$, $p$ is not in the list $p_1, p_2, \ldots, p_n$,
again contradicting that $p_n$ is the largest prime.

\medskip
In both cases we reach a contradiction. Therefore there are infinitely many primes.
\end{proof}

\vspace{2cm}
\noindent
\hyperlink{ws-AA-GAL-C07-E21}{\textbf{← Back to worksheet}}

