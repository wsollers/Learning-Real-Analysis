% =========================================================
% Measure Theory — Planned
% =========================================================
\chapter{Measure Theory}

\begin{tcolorbox}[
  colback=gray!6,
  colframe=gray!40,
  arc=2pt,
  left=8pt, right=8pt, top=6pt, bottom=6pt,
  title={\small\textbf{Where You Are in the Journey}},
  fonttitle=\small\bfseries
]
\begin{center}
\small
$\mathbb{R}$, Real Analysis
$\;\to\;$ Set Algebras
$\;\to\;$ Metric Spaces
$\;\to\;$ Topology
$\;\to\;$ \textbf{Measure Theory}
$\;\to\;$ Functional Analysis
$\;\to\;$ $\cdots$
\end{center}

\medskip
\noindent\textbf{How we got here.}
Real analysis defined the Riemann integral for nice functions, but
many naturally arising functions are not Riemann integrable.
Measure theory builds a more powerful integration theory by replacing
the interval-based approach with an abstract notion of ``size''
(measure) on arbitrary sets.

\medskip
\noindent\textbf{What this chapter will build.}
$\sigma$-algebras, measures (including Lebesgue measure),
measurable functions, the Lebesgue integral, and the convergence
theorems (monotone convergence, dominated convergence, Fatou's lemma).

\medskip
\noindent\textbf{Where this leads.}
$L^p$ spaces are the natural setting for functional analysis.
Probability theory is measure theory applied to probability spaces.
\end{tcolorbox}

\vspace{1em}
\begin{tcolorbox}[
  colback=gray!6, colframe=gray!40, arc=2pt,
  left=8pt, right=8pt, top=6pt, bottom=6pt,
  title={\small\textbf{Status: Planned}},
  fonttitle=\small\bfseries
]
\begin{center}\Large\bfseries Coming Soon\end{center}
\vspace{6pt}
\noindent Notes, proofs, and exercises will appear here in a future revision.
\end{tcolorbox}

\section{Notes}
\textit{To be populated.}

\section{Proofs}
\textit{To be populated.}

\section{Capstone}
\textit{To be populated.}
