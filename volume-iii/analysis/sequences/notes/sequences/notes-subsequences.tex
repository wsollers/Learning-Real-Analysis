% =========================================================
% Subsequences
% File: notes-subsequences.tex
% =========================================================

\subsection{Subsequences}

% ---------------------------------------------------------
% Toolkit
% ---------------------------------------------------------
\begin{tcolorbox}[colback=gray!6, colframe=gray!40, arc=2pt,
  left=6pt, right=6pt, top=4pt, bottom=4pt,
  title={\small\textbf{Subsequences — Quick Reference}},
  fonttitle=\small\bfseries]
\begin{tabular}{@{}p{0.28\textwidth}p{0.68\textwidth}@{}}
\textbf{Core items} & Key definitions/results introduced in this file.\\
\textbf{How to use} & Read the boxed items first; proofs and consequences follow.\\
\textbf{Dependencies} & Refer back to earlier sections as needed.\\
\end{tabular}
\end{tcolorbox}


% ---------------------------------------------------------
\subsubsection{Basic Definitions}

\begin{tcolorbox}[colback=propbox, colframe=propborder, arc=2pt,
  left=6pt, right=6pt, top=4pt, bottom=4pt,
  title={\small\textbf{Definition (Subsequence)}},
  fonttitle=\small\bfseries]
Let $(a_n)$ be a sequence in $\mathbb{R}$.
A sequence $(a_{n_k})$ is called a \emph{subsequence} of $(a_n)$ if
there exists a function
\[
k \mapsto n_k
\]
from $\mathbb{N}$ to $\mathbb{N}$ such that
\begin{enumerate}
    \item $n_k < n_{k+1}$ for all $k \in \mathbb{N}$, and
    \item the $k$th term of the new sequence is $a_{n_k}$.
\end{enumerate}
\end{tcolorbox}

\begin{remark}
A subsequence is obtained by selecting terms from $(a_n)$ without duplication
while preserving their original order. The index sequence $(n_k)$ determines
which terms are selected.
\end{remark}

\begin{lemma}[Index Growth]
If $(n_k)$ is a strictly increasing sequence in $\mathbb{N}$, then $n_k \geq k$ for all $k \in \mathbb{N}$.
In particular, $n_k \to \infty$ as $k \to \infty$.
\end{lemma}



\begin{tcolorbox}[colback=propbox, colframe=propborder, arc=2pt,
  left=6pt, right=6pt, top=4pt, bottom=4pt,
  title={\small\textbf{Definition (Subsequential Limit)}},
  fonttitle=\small\bfseries]
Let $(a_n)$ be a sequence. A real number $L$ is called a \emph{subsequential limit}
of $(a_n)$ if there exists a subsequence $(a_{n_k})$ such that $a_{n_k} \to L$.

The set of all subsequential limits of $(a_n)$ is denoted $\mathcal{L}(a_n)$.
\end{tcolorbox}

% ---------------------------------------------------------
\subsubsection{Main Theorems}

\begin{theorem}[Subsequences Inherit Limits]
Let $(a_n)$ be a sequence of real numbers and let $(a_{n_k})$ be any subsequence.
If $a_n \to L$, then $a_{n_k} \to L$.
\end{theorem}



\begin{remark}[Consistency of Subsequential Limits]
If $(a_n) \to L$, then every subsequence converges to $L$.
Consequently, a convergent sequence has exactly one subsequential limit:
\[
a_n \to L \implies \mathcal{L}(a_n) = \{L\}.
\]
\end{remark}

\begin{remark}[Detecting Divergence]
Contrapositive: if two subsequences of $(a_n)$ converge to different limits,
then $(a_n)$ does not converge. This provides a practical test for divergence.
\end{remark}

\begin{theorem}[Convergence via Even and Odd Subsequences]
Let $(a_n)$ be a sequence. If the even-indexed subsequence $(a_{2n})$ and the
odd-indexed subsequence $(a_{2n+1})$ both converge to the same limit $L$,
then $(a_n) \to L$.
\end{theorem}



\begin{remark}
More generally, if $(a_n)$ can be partitioned into finitely many subsequences,
each converging to the same limit $L$, then $a_n \to L$.
\end{remark}

% ---------------------------------------------------------

\begin{theorem}[Bolzano--Weierstrass]
Every bounded sequence of real numbers has a convergent subsequence.
Equivalently: if $(a_n)$ is bounded in $\mathbb{R}$, then there exist
$L \in \mathbb{R}$ and a strictly increasing sequence $(n_k)$ in $\mathbb{N}$
such that $a_{n_k} \to L$.
\end{theorem}



\begin{corollary}[Existence of Subsequential Limits]
Every bounded sequence admits at least one subsequential limit.
That is, if $(a_n)$ is bounded, then $\mathcal{L}(a_n) \neq \varnothing$.
\end{corollary}

\begin{corollary}[Sequential Compactness of Closed Intervals]
Every sequence contained in a closed bounded interval
$[a, b]$ has a convergent subsequence whose limit lies in $[a, b]$.
\end{corollary}



% ---------------------------------------------------------

\begin{theorem}[Monotone Subsequence Theorem]
Every sequence in $\mathbb{R}$ has a monotone subsequence.
\end{theorem}



\begin{remark}[Alternative Proof of Bolzano--Weierstrass]
The Monotone Subsequence Theorem combined with the Monotone Convergence Theorem
provides an alternative proof of Bolzano--Weierstrass:
\begin{enumerate}
\item Every sequence has a monotone subsequence (Monotone Subsequence Theorem).
\item A bounded monotone sequence converges (Monotone Convergence Theorem).
\item Therefore, every bounded sequence has a convergent subsequence.
\end{enumerate}
\end{remark}

% ---------------------------------------------------------
\subsubsection{Divergence Criteria}

\begin{theorem}[Characterization of Divergence]
A sequence $(a_n)$ diverges if and only if at least one of the following holds:
\begin{enumerate}
\item $(a_n)$ is unbounded.
\item $(a_n)$ has two subsequences converging to different finite limits.
\item $(a_n)$ has a subsequence diverging to $+\infty$ or $-\infty$.
\end{enumerate}
\end{theorem}



% ---------------------------------------------------------
\subsubsection{Consequences and Structural Summary}

This section establishes three structural principles.

\medskip

\textbf{(1) Inheritance of Limits.}
\[
a_n \to L \implies a_{n_k} \to L.
\]
Convergence is preserved under passage to subsequences.

\medskip

\textbf{(2) Reconstruction from Subsequences.}
\[
(a_{2n} \to L) \land (a_{2n+1} \to L) \implies a_n \to L.
\]
Convergence can be established by verifying it on a finite partition.

\medskip

\textbf{(3) Emergence of Subsequential Limits.}
\[
\text{Bounded} \implies \text{Convergent subsequence exists}.
\]
Boundedness alone does not ensure convergence,
but it guarantees the existence of convergent subsequences.

\medskip

\begin{remark}[Connection to Limit Superior and Inferior]
For a bounded sequence $(a_n)$, the set $\mathcal{L}(a_n)$ of subsequential
limits is nonempty, closed, and bounded. Moreover,
\[
\limsup_{n \to \infty} a_n = \sup \mathcal{L}(a_n) = \max \mathcal{L}(a_n),
\]
\[
\liminf_{n \to \infty} a_n = \inf \mathcal{L}(a_n) = \min \mathcal{L}(a_n).
\]
Thus $\limsup$ and $\liminf$ are themselves subsequential limits,
and they are the largest and smallest such limits.
\end{remark}

\begin{remark}[Logical Structure]
\[
\text{Convergent}
\implies
\text{All subsequences converge to the same limit},
\]
\[
\text{Bounded}
\xRightarrow{\text{Bolzano--Weierstrass}}
\text{Existence of subsequential limits}.
\]

These results form the bridge between basic convergence
theory and completeness theory.

Bolzano--Weierstrass is equivalent (in an ordered field) to:
\begin{itemize}
\item The Monotone Convergence Theorem,
\item The Cauchy Criterion,
\item The Nested Interval Property,
\item The Least Upper Bound Property.
\end{itemize}
\end{remark}

% =========================================================
% Structural Transfer of Sequence Properties
% =========================================================

\begin{remark}[Inheritance and Reflection of Properties]
Different sequence properties behave differently with respect to subsequences.
\end{remark}

\begin{center}
\begin{tabular}{|l|c|c|l|}
\hline
\textbf{Property} 
& \makecell{\textbf{Inherited by} \\ \textbf{Subsequences?}} 
& \makecell{\textbf{Reflected by One} \\ \textbf{Subsequence?}}
& \textbf{Structural Type} \\
\hline
Convergent 
& Yes 
& No 
& Tail Property \\
\hline
Cauchy 
& Yes 
& No 
& Tail Property \\
\hline
Bounded 
& Yes 
& No 
& Global Property \\
\hline
Monotone 
& Yes 
& No 
& Global Structural \\
\hline
Eventually monotone 
& Yes 
& No 
& Tail Property \\
\hline
Every subseq.\ convergent 
& --- 
& Yes 
& Universal Property \\
\hline
Every subseq.\ Cauchy 
& --- 
& Yes 
& Universal Property \\
\hline
\end{tabular}
\end{center}

\begin{remark}
\emph{Inherited} means: if $(a_n)$ has property $P$, then every subsequence has $P$.

\emph{Reflected} means: if some subsequence has property $P$, then $(a_n)$ has $P$.

Most natural properties are inherited but not reflected. The sequence
$a_n = (-1)^n$ is bounded and has convergent subsequences, but does not converge.
\end{remark}