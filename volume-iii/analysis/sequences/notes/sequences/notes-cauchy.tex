% =========================================================
% Cauchy Sequences
% File: notes-cauchy-sequences.tex
% =========================================================

\subsection{Cauchy Sequences}

% ---------------------------------------------------------
% Toolkit
% ---------------------------------------------------------
\begin{tcolorbox}[colback=gray!6, colframe=gray!40, arc=2pt,
  left=6pt, right=6pt, top=4pt, bottom=4pt,
  title={\small\textbf{Cauchy — Quick Reference}},
  fonttitle=\small\bfseries]
\begin{tabular}{@{}p{0.28\textwidth}p{0.68\textwidth}@{}}
\textbf{Core items} & Key definitions/results introduced in this file.\\
\textbf{How to use} & Read the boxed items first; proofs and consequences follow.\\
\textbf{Dependencies} & Refer back to earlier sections as needed.\\
\end{tabular}
\end{tcolorbox}


% (Original local heading preserved)
\subsubsection{Cauchy Sequences}

% ---------------------------------------------------------
\subsubsection{Basic Definitions}

\begin{tcolorbox}[colback=propbox, colframe=propborder, arc=2pt,
  left=6pt, right=6pt, top=4pt, bottom=4pt,
  title={\small\textbf{Definition (Cauchy sequence)}},
  fonttitle=\small\bfseries]
A sequence $(a_n)$ in $\mathbb{R}$ is Cauchy if
\[
\forall \varepsilon > 0 \;\exists N \in \mathbb{N}
\;\forall m,n \in \mathbb{N},
\quad
(m \ge N \land n \ge N)
\Rightarrow
|a_n - a_m| < \varepsilon.
\]
\end{tcolorbox}

\begin{remark}
    A Cauchy sequence is one whose terms become arbitrarily close to each other —
without mentioning a limit.
\end{remark}

% ---------------------------------------------------------
\subsubsection{Main Theorems}

% NOTE: The following theorem + proof appears twice in the source extract.
% Both instances are preserved verbatim below to ensure no content is lost.

\begin{theorem}[Every convergent sequence is Cauchy]
If $(a_n)$ converges in $\mathbb{R}$, then $(a_n)$ is a Cauchy sequence.
\end{theorem}



\begin{corollary}[Every convergent series has Cauchy partial sums]
If $\sum_{n=1}^\infty a_n$ converges and $s_n:=\sum_{k=1}^n a_k$ denotes the sequence
of partial sums, then $(s_n)$ is a Cauchy sequence. Equivalently,
\[
\forall \varepsilon>0\ \exists N\ \text{s.t.}\ m>n\ge N \implies
\left|\sum_{k=n+1}^{m} a_k\right| < \varepsilon.
\]
\end{corollary}



\begin{theorem}[Every convergent sequence is Cauchy]
If $(a_n)$ converges in $\mathbb{R}$, then $(a_n)$ is a Cauchy sequence.
\end{theorem}



\begin{corollary}[Every convergent series has Cauchy partial sums]
If $\sum_{n=1}^\infty a_n$ converges and $s_n:=\sum_{k=1}^n a_k$ denotes the sequence
of partial sums, then $(s_n)$ is a Cauchy sequence. Equivalently,
\[
\forall \varepsilon>0\ \exists N\ \text{s.t.}\ m>n\ge N \implies
\left|\sum_{k=n+1}^{m} a_k\right| < \varepsilon.
\]
\end{corollary}



\begin{theorem}[Cauchy sequences are bounded]
If $(a_n)$ is a Cauchy sequence in $\mathbb{R}$, then $(a_n)$ is bounded.
\end{theorem}



% ---------------------------------------------------------
\begin{theorem}[Bolzano--Weierstrass Theorem]
Every bounded sequence in $\mathbb{R}$ has a convergent subsequence.
\end{theorem}



\begin{remark}[Logical Position]
Bolzano--Weierstrass is the bridge between boundedness
and convergence.  
A bounded sequence need not converge, but it cannot
avoid convergence entirely: some subsequence must
converge.
\end{remark}

\begin{remark}[Dependence on Completeness]
The proof relies on the Monotone Convergence Theorem,
which in turn depends on the Least Upper Bound Property
of $\mathbb{R}$.  
Thus Bolzano--Weierstrass is a manifestation of
completeness.
\end{remark}

\begin{corollary}[Sequential Compactness of Closed Intervals]
Every sequence in a closed bounded interval
$[a,b]\subset\mathbb{R}$ has a convergent subsequence
whose limit lies in $[a,b]$.
\end{corollary}



\begin{remark}[Structural Consequence]
Bolzano--Weierstrass supplies the critical step in the
Cauchy Criterion:
\[
\text{Cauchy}
\Rightarrow
\text{bounded}
\Rightarrow
\text{convergent subsequence}
\Rightarrow
\text{full convergence}.
\]
It is therefore one of the equivalent formulations of
completeness.
\end{remark}





\begin{theorem}[Cauchy Criterion in $\mathbb{R}$]
A sequence $(a_n)$ of real numbers converges if and only if it is Cauchy.
\end{theorem}



% ---------------------------------------------------------
\subsubsection{Consequences}

The logical implication of this entire section is:

\begin{remark}[Interdependence of the Major Theorems on Sequences]

At first glance, the principal results of this section may appear to be
independent facts:

\begin{itemize}
    \item Convergent $\Rightarrow$ Cauchy
    \item Cauchy $\Rightarrow$ bounded
    \item Bolzano--Weierstrass (bounded $\Rightarrow$ convergent subsequence)
    \item Cauchy Criterion (Cauchy $\Leftrightarrow$ convergent)
\end{itemize}

In reality, these theorems form a tightly interlocking structure whose
foundation is the completeness of $\mathbb{R}$.

\medskip

\textbf{Structural Dependencies.}

\begin{enumerate}
    \item \emph{Convergence $\Rightarrow$ Cauchy.}  
    This direction uses only the triangle inequality.
    It does not rely on completeness.

    \item \emph{Cauchy $\Rightarrow$ bounded.}  
    Once the terms eventually cluster tightly,
    the entire sequence must lie inside some finite interval.

    \item \emph{Bounded $\Rightarrow$ convergent subsequence} (Bolzano--Weierstrass).  
    Boundedness alone does not guarantee convergence,
    but it guarantees partial convergence.

    \item \emph{Cauchy $\Rightarrow$ Convergent} (Cauchy Criterion).  
    This direction synthesizes the previous results:
    \[
    \text{Cauchy}
    \Rightarrow
    \text{bounded}
    \Rightarrow
    \text{convergent subsequence}
    \Rightarrow
    \text{full convergence}.
    \]
\end{enumerate}

\medskip

\textbf{Logical Structure.}

\[
\text{Convergent}
\Rightarrow
\text{Cauchy}
\Rightarrow
\text{Bounded}
\Rightarrow
\text{Convergent subsequence}.
\]

Completeness of $\mathbb{R}$ upgrades the final step:
\[
\text{Cauchy}
\Longleftrightarrow
\text{Convergent}.
\]

Without completeness (for example in $\mathbb{Q}$),
the implication $\text{Cauchy} \Rightarrow \text{Convergent}$ fails.

\medskip
\medskip

\begin{remark}[Direct Consequences of Bolzano--Weierstrass]

The Bolzano--Weierstrass Theorem yields several immediate structural facts:

\begin{enumerate}
    \item \textbf{Bounded nonconvergent sequences oscillate.}  
    If a bounded sequence does not converge, then it must admit
    at least two subsequences converging to different limits.

    \item \textbf{Extremal subsequences exist.}  
    For every bounded sequence $(a_n)$,
    there exist subsequences converging to
    $\limsup a_n$ and to $\liminf a_n$.

    \item \textbf{Sequential compactness of closed intervals.}  
    Every sequence contained in a closed bounded interval
    $[a,b]$ admits a convergent subsequence
    whose limit lies in $[a,b]$.
\end{enumerate}

These facts show that Bolzano--Weierstrass
controls the \emph{long-term structure} of bounded sequences.
\end{remark}

\medskip

\textbf{Extended Logical Chain.}

Combining all major results of this section, we obtain:

\[
\text{Convergent}
\Rightarrow
\text{Cauchy}
\Rightarrow
\text{Bounded}
\Rightarrow
\text{Convergent subsequence}
\]

and, using completeness,

\[
\text{Cauchy}
\Longleftrightarrow
\text{Convergent}.
\]

Thus boundedness alone does not ensure convergence,
but it prevents total divergence.
Some limiting behavior must emerge.

\textbf{Conceptual Summary.}

Convergence is an external statement about approaching a number.  
Cauchy is an internal statement about self-consistency of the sequence.  

Completeness asserts that internal consistency is sufficient:
there are no ``holes'' in the real line.

Thus the Cauchy Criterion is not merely a technical tool —
it is an equivalent formulation of completeness.
\end{remark}

