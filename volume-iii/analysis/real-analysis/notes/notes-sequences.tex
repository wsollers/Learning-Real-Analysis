% =========================================================
% Sequences
% =========================================================

\subsection{Sequences}

% ---------------------------------------------------------
\subsubsection{Basic Definitions}

\begin{definition}[Sequence]
A \emph{sequence} in a set $X$ is a function
\[
x : \mathbb{N} \to X.
\]
For each $n\in\mathbb{N}$, the value $x(n)$ is denoted
\[
x_n := x(n),
\]
and the sequence is written
\[
(x_n)_{n\in\mathbb{N}}
\quad\text{or simply}\quad
(x_n).
\]
\end{definition}

\begin{remark}[Logical Form]
\[
\exists x\;
\bigl(
x:\mathbb{N}\to X
\ \wedge\
\forall n\in\mathbb{N},\ x_n = x(n)
\bigr).
\]
\end{remark}

\begin{remark}[Common Notation]
Equivalent notations:
\[
(x_n)_{n=1}^{\infty},
\qquad
(x_n),
\qquad
\{x_n\}_{n=1}^{\infty},
\qquad
n \mapsto x_n.
\]
All represent the same underlying function.
\end{remark}

\begin{remark}
When $X=\mathbb{R}$, we speak of a \emph{real sequence}.
When $X=\mathbb{R}^m$, we speak of a \emph{vector-valued sequence}.
\end{remark}

\begin{definition}[Subsequence]
Let $(x_n)_{n \in \mathbb{N}}$ be a sequence in $X$, i.e., a function
\[
x : \mathbb{N} \to X, \qquad n \mapsto x_n.
\]
A function $y : \mathbb{N} \to X$ is a \emph{subsequence} of $(x_n)$ if and
only if there exists a strictly increasing function
\[
\sigma : \mathbb{N} \to \mathbb{N},
\qquad
k < \ell \implies \sigma(k) < \sigma(\ell),
\]
such that $y = x \circ \sigma$. Writing $n_k := \sigma(k)$, the composed
function takes the form
\[
(x \circ \sigma)(k) = x(\sigma(k)) = x(n_k) =: x_{n_k},
\]
and the subsequence is written
\[
(x_{n_k})_{k \in \mathbb{N}}
\quad\text{or simply}\quad
(x_{n_k}).
\]
\end{definition}

\begin{remark}[Logical Form]
A function $y : \mathbb{N} \to X$ is a subsequence of $x$ if and only if
\[
\exists\, \sigma : \mathbb{N} \to \mathbb{N}
\;\Bigl(
  \forall\, k < \ell \in \mathbb{N},\ \sigma(k) < \sigma(\ell)
  \;\wedge\;
  y = x \circ \sigma
\Bigr).
\]
The quantifier $\exists\,\sigma$ asserts that a witnessing index function
can be exhibited. The subsequence is the composition $y = x \circ \sigma$,
not $\sigma$ itself.
\end{remark}

\begin{remark}[Function Composition]
The composition $x \circ \sigma$ makes the two-layer structure explicit:
\[
\mathbb{N}
\xrightarrow{\ \sigma\ }
\mathbb{N}
\xrightarrow{\ x\ }
X.
\]
$\sigma$ selects an infinite subset of indices in their natural order;
$x$ maps those indices to elements of $X$.
The subsequence is entirely determined by the choice of $\sigma$.
\end{remark}

\begin{remark}[Implications of Strict Monotonicity]
Since $\sigma$ is strictly increasing, the following hold.
\begin{itemize}
  \item \textbf{Injectivity.} Strict monotonicity implies injectivity:
        $k \neq \ell \implies \sigma(k) \neq \sigma(\ell)$,
        since either $k < \ell$, giving $\sigma(k) < \sigma(\ell)$,
        or $k > \ell$, giving $\sigma(k) > \sigma(\ell)$.

  \item \textbf{Index growth.} $\sigma(k) \geq k$ for all $k \in \mathbb{N}$.
        This follows directly from strict monotonicity and the discreteness
        of $\mathbb{N}$: $\sigma(1) \geq 1$, and $\sigma(k+1) \geq \sigma(k) + 1$,
        so $\sigma(k) \geq k$ at each step without appeal to a separate
        induction argument.

  \item \textbf{Divergence of indices.} Since $\sigma(k) \geq k$ for all $k$
        and $k \to \infty$, it follows immediately that $\sigma(k) \to \infty$.
        This is not an independent assumption — it is a consequence of strict
        monotonicity alone. In particular, $n_k \to \infty$, so every
        subsequence reaches arbitrarily far into the original sequence.
        This fact is used in convergence proofs where $n_k$ must be
        eventually large.

  \item \textbf{Order preservation.} $k < \ell \implies n_k < n_\ell$,
        so the relative order of selected terms is unchanged.

  \item \textbf{Deletion only.} Terms of $(x_n)$ may be omitted but
        never duplicated or reinserted out of order. A subsequence is
        a selection, not a rearrangement.
\end{itemize}
\end{remark}

\begin{remark}[Common Notation]
Equivalent notations for the subsequence $x \circ \sigma$:
\[
(x_{n_k})_{k=1}^{\infty},
\qquad
(x_{n_k}),
\qquad
k \mapsto x_{n_k}.
\]
The index $k$ tracks position within the subsequence;
$n_k = \sigma(k)$ tracks position within the original sequence.
These two indices must be kept distinct in proofs.
\end{remark}

\begin{remark}[Sources]
The functional definition of a sequence as a map $x : \mathbb{N} \to X$
appears in \srcciteauthor{AbbottUnderstandingAnalysis},
\srcciteauthor{TaoAnalysis1},
\srcciteauthor{RossElementaryAnalysis},
\srcciteauthor{Rudin1976},
and \srcciteauthor{LeblBasicAnalysisI}.
The compositional formulation $x \circ \sigma$ is made explicit in
\srccitesection{TaoAnalysis1}{Chapter~6}
and \srccitesection{Rudin1976}{Chapter~3}.
\end{remark}
\FloatBarrier
\begin{figure}[H]
\centering

\begin{tikzpicture}[
  every node/.style={font=\small},
  arr/.style={-{Stealth[length=7pt]}, thick},
  node distance=3.8cm
]

% ── Set boxes ─────────────────────────────────────────────────────────────────
\node[
  draw=indexblue, fill=lightblue, rounded corners=6pt,
  minimum width=2.6cm, minimum height=7.2cm,
  label={[font=\bfseries\normalsize, text=indexblue]above:Subsequence Index $k$},
  label={[font=\small\itshape, text=indexblue]below:$\mathbb{N}$
         \normalfont\scriptsize(domain of $x\circ\sigma$)}
] (boxK) at (0,0) {};

\node[
  draw=natblue, fill=lightblue!60, rounded corners=6pt,
  minimum width=2.6cm, minimum height=7.2cm,
  label={[font=\bfseries\normalsize, text=natblue]above:Sequence Index $n_k$},
  label={[font=\small\itshape, text=natblue]below:$\mathbb{N}$
         \normalfont\scriptsize(domain of $x$)}
] (boxN) at (4.8,0) {};

\node[
  draw=realorange, fill=lightorange, rounded corners=6pt,
  minimum width=2.6cm, minimum height=7.2cm,
  label={[font=\bfseries\normalsize, text=realorange]above:Value $x_{n_k}$},
  label={[font=\small\itshape, text=realorange]below:$\mathbb{R}$
         \normalfont\scriptsize(codomain)}
] (boxR) at (9.6,0) {};

% ── Element nodes ─────────────────────────────────────────────────────────────
\foreach \k/\y in {1/2.6, 2/1.5, 3/0.4, 4/-0.7, 5/-1.8}{
  \node[circle, fill=indexblue, inner sep=2.5pt,
        text=white, font=\footnotesize\bfseries]
    (k\k) at (0,\y) {$\k$};
}
\node[font=\footnotesize, text=indexblue] at (0,-2.8) {$\vdots$};

\foreach \n/\y/\lab in {
  n1/2.6/{$n_1=2$},
  n2/1.5/{$n_2=5$},
  n3/0.4/{$n_3=7$},
  n4/-0.7/{$n_4=11$},
  n5/-1.8/{$n_5=13$}
}{
  \node[circle, fill=natblue, inner sep=2.5pt,
        text=white, font=\footnotesize\bfseries] (\n) at (4.8,\y) {};
  \node[font=\footnotesize, text=natblue, right=2pt of \n] {\lab};
}
\node[font=\footnotesize, text=natblue] at (4.8,-2.8) {$\vdots$};

\foreach \v/\y/\lab in {
  v1/2.6/{$x_2$},
  v2/1.5/{$x_5$},
  v3/0.4/{$x_7$},
  v4/-0.7/{$x_{11}$},
  v5/-1.8/{$x_{13}$}
}{
  \node[circle, fill=realorange, inner sep=2.5pt,
        text=white, font=\footnotesize\bfseries] (\v) at (9.6,\y) {};
  \node[font=\footnotesize, text=realorange, right=3pt of \v] {\lab};
}
\node[font=\footnotesize, text=realorange] at (9.6,-2.8) {$\vdots$};

% ── Mapping arrows ────────────────────────────────────────────────────────────
\foreach \k/\n in {k1/n1, k2/n2, k3/n3, k4/n4, k5/n5}{
  \draw[arr, color=sigmagreen] (\k) -- (\n);
}
\foreach \n/\v in {n1/v1, n2/v2, n3/v3, n4/v4, n5/v5}{
  \draw[arr, color=realorange!80!black] (\n) -- (\v);
}

% ── Function labels ───────────────────────────────────────────────────────────
\draw[-{Stealth[length=8pt]}, line width=1.4pt, color=sigmagreen, bend left=15]
  ($(boxK.north east)+(0.1,0.1)$)
  to node[above, font=\bfseries\normalsize, text=sigmagreen]
     {$\sigma$ \normalfont\small (strictly increasing)}
  ($(boxN.north west)+(-0.1,0.1)$);

\draw[-{Stealth[length=8pt]}, line width=1.4pt,
      color=realorange!80!black, bend left=15]
  ($(boxN.north east)+(0.1,0.1)$)
  to node[above, font=\bfseries\normalsize, text=realorange!80!black]
     {$x$ \normalfont\small (parent sequence)}
  ($(boxR.north west)+(-0.1,0.1)$);

% ── Composition arrow ─────────────────────────────────────────────────────────
\draw[-{Stealth[length=9pt]}, line width=1.8pt,
      color=indexblue!70!black, rounded corners=8pt]
  ($(boxK.south)+(0,-0.55)$)
  -- node[below, font=\normalsize\bfseries, text=indexblue!80!black]
     {$x \circ \sigma \;:\; k \mapsto x_{n_k}$}
  ($(boxR.south)+(0,-0.55)$);

\end{tikzpicture}

\vspace{0.6cm}

% ── Properties panel ──────────────────────────────────────────────────────────
\begin{tikzpicture}[
  every node/.style={font=\small},
  propnode/.style={
    draw=propborder, fill=propbox, rounded corners=5pt,
    text width=3.4cm, align=left, inner sep=8pt,
    minimum height=2.8cm
  }
]

% Header
\node[font=\small\itshape, text=gray] (header) at (7.4, 1.6)
  {Four consequences of strict monotonicity:};

\node[propnode] (p1) at (0,0) {
  \textbf{\textcolor{sigmagreen}{Order Preservation}}\\[4pt]
  $k < \ell \Rightarrow \sigma(k) < \sigma(\ell)$\\[3pt]
  {\footnotesize Strict monotonicity. The defining property of $\sigma$.}
};
\node[propnode, right=0.35cm of p1] (p2) {
  \textbf{\textcolor{natblue}{Index Growth}}\\[4pt]
  $\sigma(k) \geq k \quad \forall k\in\mathbb{N}$\\[3pt]
  {\footnotesize From strict monotonicity and discreteness of $\mathbb{N}$.}
};
\node[propnode, right=0.35cm of p2] (p3) {
  \textbf{\textcolor{indexblue}{Injectivity}}\\[4pt]
  $k \neq \ell \Rightarrow \sigma(k) \neq \sigma(\ell)$\\[3pt]
  {\footnotesize Strict monotonicity implies injectivity directly.}
};
\node[propnode, right=0.35cm of p3, text width=3.8cm] (p4) {
  \textbf{\textcolor{realorange}{Unbounded Indices}}\\[4pt]
  $\sigma(k) \to \infty$ as $k\to\infty$\\[3pt]
  {\footnotesize Since $\sigma(k)\geq k\to\infty$. Follows from index growth.}
};
\end{tikzpicture}

\vspace{0.4cm}

% ── Key identity box ──────────────────────────────────────────────────────────
\begin{tikzpicture}
\node[
  draw=indexblue, fill=lightblue!40, rounded corners=6pt,
  inner sep=12pt, text width=11cm, align=center
]{
  \textbf{Key Identity}\quad
  $(x \circ \sigma)(k) \;=\; x\bigl(\sigma(k)\bigr)
   \;=\; x(n_k) \;=:\; x_{n_k}$\\[6pt]
{\small $k$ = position in subsequence}\\[2pt]
  {\small $n_k = \sigma(k)$ = position in original sequence}\\[4pt]
  {\small\itshape These two indices must be kept distinct in every proof.}
};
\end{tikzpicture}

\caption{Subsequence as function composition
  $\mathbb{N}\xrightarrow{\ \sigma\ }\mathbb{N}\xrightarrow{\ x\ }\mathbb{R}$.}
\label{fig:subsequence-composition}
\end{figure}
% ---------------------------------------------------------
\subsubsection{Main Theorems}

\begin{theorem}[A Sequence Is Determined by Its Values]
Let $x,y : \mathbb{N}\to X$ be sequences.
If
\[
\forall n\in\mathbb{N},\quad x_n = y_n,
\]
then $x=y$ as functions.
\end{theorem}

\begin{proof}
If $x(n)=y(n)$ for all $n\in\mathbb{N}$,
then the functions agree on their entire domain.
By the definition of equality of functions,
$x=y$.
\qed
\end{proof}

\begin{remark}
This theorem relies only on the definition of equality of functions.
It implies that sequence equality is pointwise equality.
\end{remark}

\begin{theorem}[Every Subsequence Is a Sequence]
If $(x_n)$ is a sequence in $X$ and $(n_k)$ is strictly increasing in $\mathbb{N}$,
then $(x_{n_k})$ is a sequence in $X$.
\end{theorem}

\begin{proof}
Define $y:\mathbb{N}\to X$ by $y(k)=x_{n_k}$.
Since $x$ maps $\mathbb{N}$ into $X$ and $n_k\in\mathbb{N}$,
the composition is well-defined.
Thus $(x_{n_k})$ is a function from $\mathbb{N}$ to $X$,
hence a sequence.
\qed
\end{proof}

\begin{remark}
This theorem depends on function composition.
It implies that subsequences remain within the same ambient space.
Later, this becomes crucial in Bolzano–Weierstrass and compactness arguments.
\end{remark}

% ---------------------------------------------------------
\subsubsection{Consequences}

The logical implication of this section is:

A sequence is not merely an ordered list,
but a function with domain $\mathbb{N}$.
Therefore:

\[
\text{Sequence theory} 
=
\text{Function theory on } \mathbb{N}.
\]

This structural viewpoint allows us to:
\begin{itemize}
\item Define limits using quantifiers over $\mathbb{N}$.
\item Treat subsequences as compositions.
\item Apply functional equality rigorously.
\end{itemize}

\begin{remark}[Logical Structure]
The foundational flow is:

\[
\text{Definition of Sequence}
\Rightarrow
\text{Equality Theorem}
\Rightarrow
\text{Subsequence Definition}
\Rightarrow
\text{Subsequence Theorem}.
\]

Thus all later limit theory rests on the functional interpretation.
\end{remark}


% ---------------------------------------------------------
\subsubsection{Canonical Examples}

\begin{example}[Constant sequence]
Fix $c\in\mathbb{R}$. Define
\[
x_n := c
\quad\text{for all } n\in\mathbb{N}.
\]
This is the simplest example of a sequence.
\end{example}

\begin{example}[Arithmetic sequence]
Given $a,d\in\mathbb{R}$, define
\[
x_n := a + (n-1)d.
\]
Each term differs from the previous one by the fixed increment $d$.
\end{example}

\begin{example}[Geometric sequence]
Given $a,r\in\mathbb{R}$, define
\[
x_n := a r^{\,n-1}.
\]
Each term is obtained by multiplying the previous one by the ratio $r$.
\end{example}

\begin{example}[Harmonic sequence]
\[
x_n := \frac{1}{n}.
\]
This sequence decreases to $0$.
\end{example}

\begin{example}[Alternating sequence]
\[
x_n := (-1)^n.
\]
This sequence oscillates between $1$ and $-1$.
\end{example}

\begin{example}[Polynomial growth sequence]
Let $k\in\mathbb{N}$. Define
\[
x_n := n^k.
\]
This sequence grows without bound.
\end{example}

\begin{example}[Exponential decay sequence]
\[
x_n := \frac{1}{2^n}.
\]
This sequence decreases rapidly toward $0$.
\end{example}

\begin{remark}[Purpose of Canonical Examples]
These examples serve as test cases for:
\begin{itemize}
\item boundedness,
\item monotonicity,
\item convergence and divergence,
\item oscillation,
\item growth rates.
\end{itemize}
Nearly every theorem about sequences can be sanity-checked against this list.
\end{remark}

% ---------------------------------------------------------
\subsubsection{Logical Classification Table}

\begin{center}
\renewcommand{\arraystretch}{1.2}
\begin{tabular}{|l|c|c|c|c|}
\hline
\textbf{Sequence} 
& \textbf{Bounded} 
& \textbf{Monotone} 
& \textbf{Convergent} 
& \textbf{Divergent} \\
\hline

Constant $x_n = c$ 
& Yes 
& Yes 
& Yes 
& No \\

Arithmetic $x_n = a+(n-1)d$ 
& If $d=0$ 
& If $d\ge0$ or $d\le0$ 
& If $d=0$ 
& If $d\ne0$ \\

Geometric $x_n = ar^{n-1}$ 
& If $|r|\le1$ 
& If $r\ge0$ 
& If $|r|<1$ 
& If $|r|>1$ \\

Harmonic $x_n=\frac1n$ 
& Yes 
& Decreasing 
& Yes 
& No \\

Alternating $x_n=(-1)^n$ 
& Yes 
& No 
& No 
& Yes \\

Polynomial $x_n=n^k$ 
& No 
& Increasing 
& No 
& Yes \\

Exponential decay $x_n=\frac1{2^n}$ 
& Yes 
& Decreasing 
& Yes 
& No \\

\hline
\end{tabular}
\end{center}

\begin{remark}[Logical Dependencies]
The classifications rely on later results:

\begin{itemize}
\item Monotone sequences converge if and only if they are bounded
      (Monotone Convergence Theorem).
\item Convergent sequences are bounded.
\item Divergence may occur via unbounded growth or oscillation.
\end{itemize}

Thus this table previews the structure of future theorems.
\end{remark}


