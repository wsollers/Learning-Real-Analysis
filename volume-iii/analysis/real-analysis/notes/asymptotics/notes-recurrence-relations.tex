\subsection{Recurrence Inequalities and Iterative Control}

% ---------------------------------------------------------
% Toolkit
% ---------------------------------------------------------
\begin{tcolorbox}[colback=gray!6, colframe=gray!40, arc=2pt,
  left=6pt, right=6pt, top=4pt, bottom=4pt,
  title={\small\textbf{Recurrence Relations — Quick Reference}},
  fonttitle=\small\bfseries]
\begin{tabular}{@{}p{0.28\textwidth}p{0.68\textwidth}@{}}
\textbf{Core items} & Key definitions/results introduced in this file.\\
\textbf{How to use} & Read the boxed items first; proofs and consequences follow.\\
\textbf{Dependencies} & Refer back to earlier sections as needed.\\
\end{tabular}
\end{tcolorbox}


Many arguments involving sequences and subsequences rely on
\emph{propagating an inequality across indices}. 
This section isolates that structure.

% --------------------------------------------------------
% General Recurrence Inequality
% --------------------------------------------------------

\begin{tcolorbox}[colback=propbox, colframe=propborder, arc=2pt,
  left=6pt, right=6pt, top=4pt, bottom=4pt,
  title={\small\textbf{Definition (Recurrence Inequality)}},
  fonttitle=\small\bfseries]
Let $(u_n)$ be a nonnegative sequence, i.e., $u_n \geq 0$ for all $n \geq 0$.
We say $(u_n)$ satisfies a \emph{recurrence inequality}
if there exists a function $T : [0,\infty) \to [0,\infty)$ such that
\[
u_{n+1} \le T(u_n)
\quad \text{for all } n \ge 0.
\]
\end{tcolorbox}

\begin{remark}
A recurrence inequality provides \emph{local control}:
each term is bounded in terms of the previous term.
The main technique is to \emph{iterate} the inequality
to obtain global control.
\end{remark}

\begin{tcolorbox}[colback=propbox, colframe=propborder, arc=2pt,
  left=6pt, right=6pt, top=4pt, bottom=4pt,
  title={\small\textbf{Definition (Iterates of a Function)}},
  fonttitle=\small\bfseries]
Let $T : [0,\infty) \to [0,\infty)$ be a function. The \emph{$n$-fold iterate} of $T$,
denoted $T^n$, is defined recursively by
\[
T^0(x) := x, \qquad T^{n+1}(x) := T(T^n(x)).
\]
\end{tcolorbox}

\begin{lemma}[Iterative Propagation]
Let $(u_n)$ be a nonnegative sequence satisfying
\[
u_{n+1} \le T(u_n)
\]
for all $n \ge 0$, where $T : [0,\infty) \to [0,\infty)$ is monotone increasing.
Then for all $n \ge 0$,
\[
u_n \le T^n(u_0).
\]
\end{lemma}



\begin{remark}[Structural Insight]
The method is:
\begin{enumerate}
\item Extract a recurrence inequality.
\item Iterate it (often by induction).
\item Obtain explicit asymptotic control.
\item Deduce convergence or Cauchy behavior.
\end{enumerate}
This pattern underlies contraction mapping arguments,
geometric decay estimates, and certain subadditivity arguments.
\end{remark}

% --------------------------------------------------------
% Specific Recurrence Types
% --------------------------------------------------------

\begin{tcolorbox}[colback=propbox, colframe=propborder, arc=2pt,
  left=6pt, right=6pt, top=4pt, bottom=4pt,
  title={\small\textbf{Definition (Multiplicative Recurrence)}},
  fonttitle=\small\bfseries]
A nonnegative sequence $(u_n)$ satisfies a \emph{multiplicative recurrence}
if
\[
u_{n+1} \le c u_n
\quad \text{for some } 0 < c < 1.
\]
\end{tcolorbox}

\begin{proposition}[Multiplicative Recurrence Bound]
If $(u_n)$ is nonnegative and satisfies $u_{n+1} \leq c u_n$ with $0 < c < 1$, then
\[
u_n \leq c^n u_0.
\]
In particular, $u_n \to 0$ as $n \to \infty$.
\end{proposition}



\begin{tcolorbox}[colback=propbox, colframe=propborder, arc=2pt,
  left=6pt, right=6pt, top=4pt, bottom=4pt,
  title={\small\textbf{Definition (Affine Recurrence)}},
  fonttitle=\small\bfseries]
A nonnegative sequence $(u_n)$ satisfies an \emph{affine recurrence}
if
\[
u_{n+1} \le c u_n + b
\quad \text{for some } 0 < c < 1 \text{ and } b \ge 0.
\]
\end{tcolorbox}

\begin{proposition}[Affine Recurrence Bound]
If $(u_n)$ is nonnegative and satisfies $u_{n+1} \le c u_n + b$ with $0 < c < 1$ and $b \ge 0$, then
\[
u_n \le c^n u_0 + \frac{b(1 - c^n)}{1 - c}.
\]
In particular,
\[
\limsup_{n \to \infty} u_n \le \frac{b}{1 - c}.
\]
\end{proposition}



% --------------------------------------------------------
% Discrete Grönwall Inequality
% --------------------------------------------------------

\begin{theorem}[Discrete Grönwall Inequality]
Let $(u_n)$ and $(\beta_n)$ be nonnegative sequences satisfying
\[
u_n \le \alpha + \sum_{k=0}^{n-1} \beta_k u_k
\quad \text{for all } n \ge 0,
\]
where $\alpha \ge 0$ is a constant. Then
\[
u_n \le \alpha \prod_{k=0}^{n-1} (1 + \beta_k).
\]
In particular,
\[
u_n \le \alpha \exp\left( \sum_{k=0}^{n-1} \beta_k \right).
\]
\end{theorem}



\begin{remark}[Connection to Continuous Grönwall Inequality]
The discrete Grönwall inequality is the analogue of the continuous 
\emph{Grönwall--Bellman inequality}: if $u : [a, b] \to \mathbb{R}$ satisfies
\[
u(t) \le \alpha + \int_a^t \beta(s)\, u(s)\, ds,
\]
with $\alpha \geq 0$ and $\beta(s) \geq 0$, then
\[
u(t) \le \alpha \exp\left( \int_a^t \beta(s)\, ds \right).
\]
This continuous version is fundamental in the theory of ordinary
differential equations, particularly for proving uniqueness and
continuous dependence on initial conditions.
\end{remark}

% --------------------------------------------------------
% Subsequence Recurrence
% --------------------------------------------------------

\begin{tcolorbox}[colback=propbox, colframe=propborder, arc=2pt,
  left=6pt, right=6pt, top=4pt, bottom=4pt,
  title={\small\textbf{Definition (Subsequence Recurrence)}},
  fonttitle=\small\bfseries]
Let $(a_n)$ be a sequence.
We say that a \emph{subsequence recurrence inequality}
holds along $(a_{n_k})$ if there exists a function $T$ such that
\[
|a_{n_{k+1}} - a_{n_k}| \le T(|a_{n_k} - a_{n_{k-1}}|)
\]
for all sufficiently large $k$.
\end{tcolorbox}

\begin{remark}[Induction on Subsequence Indices]
When verifying a property $\mathcal{P}(k)$ for all $k \in \mathbb{N}$,
standard mathematical induction applies to the subsequence index $k$,
regardless of the values of the original indices $n_k$.
\end{remark}

% --------------------------------------------------------
% Terminology
% --------------------------------------------------------

\begin{tcolorbox}[colback=propbox, colframe=propborder, arc=2pt,
  left=6pt, right=6pt, top=4pt, bottom=4pt,
  title={\small\textbf{Definition (Terminology)}},
  fonttitle=\small\bfseries]
\begin{itemize}
\item \textbf{Propagation:} Repeated application of a recurrence inequality.
\item \textbf{Iterative control:} Bounding $u_n$ via repeated composition $T^n$.
\item \textbf{Geometric decay:} Bounds of the form $u_n \le C c^n$ with $0 < c < 1$.
\item \textbf{Telescoping argument:} Summing successive differences to control total variation.
\item \textbf{Block decomposition:} Writing $n = qk + r$ to propagate additive bounds.
\end{itemize}
\end{tcolorbox}

% --------------------------------------------------------
% Subadditive and Superadditive Sequences
% --------------------------------------------------------

\begin{tcolorbox}[colback=propbox, colframe=propborder, arc=2pt,
  left=6pt, right=6pt, top=4pt, bottom=4pt,
  title={\small\textbf{Definition (Subadditive Sequence)}},
  fonttitle=\small\bfseries]
A sequence $(a_n)$ is \emph{subadditive} if
\[
a_{m+n} \le a_m + a_n
\quad \text{for all } m, n \in \mathbb{N}.
\]
\end{tcolorbox}

\begin{tcolorbox}[colback=propbox, colframe=propborder, arc=2pt,
  left=6pt, right=6pt, top=4pt, bottom=4pt,
  title={\small\textbf{Definition (Superadditive Sequence)}},
  fonttitle=\small\bfseries]
A sequence $(a_n)$ is \emph{superadditive} if
\[
a_{m+n} \ge a_m + a_n
\quad \text{for all } m, n \in \mathbb{N}.
\]
\end{tcolorbox}

\begin{remark}
A sequence $(a_n)$ is superadditive if and only if $(-a_n)$ is subadditive.
This duality allows results about subadditive sequences to be transferred
to superadditive sequences by negation.
\end{remark}

\begin{theorem}[Fekete's Lemma for Subadditive Sequences]
If $(a_n)$ is subadditive, then
\[
\lim_{n \to \infty} \frac{a_n}{n}
=
\inf_{k \ge 1} \frac{a_k}{k}.
\]
In particular, the limit exists (possibly equal to $-\infty$).
\end{theorem}



\begin{theorem}[Fekete's Lemma for Superadditive Sequences]
If $(a_n)$ is superadditive, then
\[
\lim_{n \to \infty} \frac{a_n}{n}
=
\sup_{k \ge 1} \frac{a_k}{k}.
\]
In particular, the limit exists (possibly equal to $+\infty$).
\end{theorem}



% --------------------------------------------------------
% Contraction Example
% --------------------------------------------------------

\begin{tcolorbox}[colback=propbox, colframe=propborder, arc=2pt,
  left=6pt, right=6pt, top=4pt, bottom=4pt,
  title={\small\textbf{Example (Contraction-Type Recurrence)}},
  fonttitle=\small\bfseries]
Suppose $(x_n)$ satisfies
\[
|x_{n+1} - x_n|
\le
c |x_n - x_{n-1}|
\quad \text{for all } n \ge 1,
\]
where $0 < c < 1$.
\end{tcolorbox}

\begin{proposition}
If $(x_n)$ satisfies a contraction-type recurrence with $0 < c < 1$,
then $(x_n)$ is Cauchy.
\end{proposition}



% --------------------------------------------------------
% Subsequence Contraction Example
% --------------------------------------------------------

\begin{tcolorbox}[colback=propbox, colframe=propborder, arc=2pt,
  left=6pt, right=6pt, top=4pt, bottom=4pt,
  title={\small\textbf{Example (Contraction Along a Subsequence)}},
  fonttitle=\small\bfseries]
Suppose $(a_{n_k})$ is a subsequence satisfying
\[
|a_{n_{k+1}} - a_{n_k}|
\le
c |a_{n_k} - a_{n_{k-1}}|
\quad \text{for all } k \ge 1,
\]
where $0 < c < 1$.
\end{tcolorbox}

\begin{proposition}
If $(a_{n_k})$ satisfies a contraction-type recurrence with $0 < c < 1$,
then $(a_{n_k})$ is Cauchy.
\end{proposition}



% --------------------------------------------------------
% Structural Summary
% --------------------------------------------------------

\begin{remark}[Unifying Structure]
The results in this section share a common logical structure:
\[
\text{Local inequality}
\;\Longrightarrow\;
\text{Iterated bound}
\;\Longrightarrow\;
\text{Explicit asymptotic control}
\;\Longrightarrow\;
\text{Tail property}.
\]
\end{remark}

\begin{remark}[Hierarchy of Recurrence Types]
The section develops the following hierarchy:
\begin{center}
\begin{tabular}{ll}
\textbf{Type} & \textbf{Key Result} \\
\hline
Multiplicative ($u_{n+1} \leq c u_n$) & Geometric decay: $u_n \leq c^n u_0$ \\
Affine ($u_{n+1} \leq c u_n + b$) & Bounded limit: $\limsup u_n \leq \frac{b}{1-c}$ \\
Subadditive ($a_{m+n} \leq a_m + a_n$) & Fekete: $\lim \frac{a_n}{n} = \inf \frac{a_k}{k}$ \\
Discrete integral ($u_n \leq \alpha + \sum \beta_k u_k$) & Grönwall: $u_n \leq \alpha \exp(\sum \beta_k)$ \\
\end{tabular}
\end{center}
\end{remark}