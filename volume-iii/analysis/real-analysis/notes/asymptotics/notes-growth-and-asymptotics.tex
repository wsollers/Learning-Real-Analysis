% =========================================================
% Growth and Asymptotics of Sequences
% (Place after notes-series-manipulations.tex)
% =========================================================

\subsection{Growth and Asymptotic Behavior of Sequences}

% ---------------------------------------------------------
% Toolkit
% ---------------------------------------------------------
\begin{tcolorbox}[title=Toolkit: Growth And Asymptotics]
\begin{tabular}{@{}p{0.28\textwidth}p{0.68\textwidth}@{}}
\textbf{Core items} & Key definitions/results introduced in this file.\\
\textbf{How to use} & Read the boxed items first; proofs and consequences follow.\\
\textbf{Dependencies} & Refer back to earlier sections as needed.\\
\end{tabular}
\end{tcolorbox}


% ---------------------------------------------------------
\subsubsection{Basic Definitions}

\begin{tcolorbox}[title=Definition (Ratio behavior)]
Let $(a_n)$ be a sequence with $a_n \neq 0$ for large $n$.
The \emph{ratio behavior} of $(a_n)$ is governed by the limit (if it exists)
\[
\lim_{n\to\infty} \frac{a_{n+1}}{a_n}.
\]
\end{tcolorbox}

\begin{tcolorbox}[title=Definition (Root behavior)]
Let $(a_n)$ be a sequence with $a_n \ge 0$ for large $n$.
The \emph{root behavior} of $(a_n)$ is governed by the limit (if it exists)
\[
\lim_{n\to\infty} \sqrt[n]{a_n}.
\]
\end{tcolorbox}

\begin{tcolorbox}[title=Definition (Big-O notation)]
Let $(a_n)$ and $(b_n)$ be sequences with $b_n \neq 0$ eventually.
We write
\[
a_n = O(b_n)
\]
if there exist constants $C>0$ and $N$ such that
\[
|a_n| \le C |b_n|
\quad \text{for all } n \ge N.
\]
\end{tcolorbox}

\begin{tcolorbox}[title=Definition (Asymptotic comparison)]
Let $(a_n)$ and $(b_n)$ be sequences with $b_n \neq 0$ eventually.

\[
a_n = o(b_n)
\quad\text{if}\quad
\lim_{n\to\infty} \frac{a_n}{b_n} = 0.
\]

\[
a_n \sim b_n
\quad\text{if}\quad
\lim_{n\to\infty} \frac{a_n}{b_n} = 1.
\]
\end{tcolorbox}

\begin{tcolorbox}[title=Definition (Polynomial, exponential, and factorial growth)]
A sequence exhibits:

\begin{itemize}
\item \emph{Polynomial growth} if $a_n \sim n^k$ for some $k>0$.
\item \emph{Exponential growth} if $a_n \sim c^n$ for some $c>1$.
\item \emph{Factorial growth} if $a_n \sim n!$.
\end{itemize}
\end{tcolorbox}

% ---------------------------------------------------------
\subsubsection{Main Theorems}

\begin{theorem}[Ratio theorem for sequences]
Let $(a_n)$ be positive and suppose
\[
\lim_{n\to\infty} \frac{a_{n+1}}{a_n} = L.
\]
\begin{enumerate}
\item If $L<1$, then $a_n \to 0$.
\item If $L>1$, then $a_n \to \infty$.
\end{enumerate}
\end{theorem}

\begin{proof}
Assume $L<1$. Choose $r$ such that $L<r<1$.
For large $n$,
\[
\frac{a_{n+1}}{a_n} < r.
\]
Thus
\[
a_{n+1} < r a_n.
\]
Iterating,
\[
a_n \le C r^n
\]
for some constant $C$, and since $r^n \to 0$, we have $a_n \to 0$.

If $L>1$, a similar argument shows $a_n$ grows at least geometrically,
hence diverges to infinity.
\end{proof}

\begin{theorem}[Root theorem for sequences]
Let $(a_n)$ be positive and suppose
\[
\lim_{n\to\infty} \sqrt[n]{a_n} = L.
\]
\begin{enumerate}
\item If $L<1$, then $a_n \to 0$.
\item If $L>1$, then $a_n \to \infty$.
\end{enumerate}
\end{theorem}

\begin{proof}
If $L<1$, choose $r$ with $L<r<1$.
Then for large $n$,
\[
\sqrt[n]{a_n} < r,
\]
so
\[
a_n < r^n.
\]
Since $r^n \to 0$, we conclude $a_n \to 0$.
The case $L>1$ is analogous.
\end{proof}

\begin{theorem}[Properties of asymptotic equivalence]
The relation $a_n \sim b_n$ is:
\begin{itemize}
\item reflexive,
\item symmetric,
\item transitive.
\end{itemize}
Hence asymptotic equivalence defines an equivalence relation on sequences
that are eventually nonzero.
\end{theorem}

\begin{theorem}[Stolz--Cesàro]
Let $(a_n)$ and $(b_n)$ be sequences with $(b_n)$ strictly increasing and
$b_n \to \infty$. If
\[
\lim_{n\to\infty} \frac{a_{n+1}-a_n}{b_{n+1}-b_n} = L,
\]
then
\[
\lim_{n\to\infty} \frac{a_n}{b_n} = L.
\]
\end{theorem}

\begin{proof}
Fix $\varepsilon>0$. For large $n$,
\[
\left|\frac{a_{n+1}-a_n}{b_{n+1}-b_n} - L\right| < \varepsilon.
\]
Summing telescopically yields bounds on
\[
\frac{a_n}{b_n},
\]
which converge to $L$ as $n\to\infty$.
\end{proof}

\begin{theorem}[Fundamental asymptotic limits]
\[
\lim_{n\to\infty} n^{1/n} = 1,
\qquad
\lim_{n\to\infty} \frac{\log n}{n^\alpha} = 0 \quad (\alpha>0),
\]
\[
\lim_{n\to\infty} \left(1+\frac{x}{n}\right)^n = e^x.
\]
\end{theorem}

% ---------------------------------------------------------
\subsubsection{Consequences}

The logical implication of this section is:

\[
\text{Local growth ratios}
\Rightarrow
\text{Global asymptotic classification}.
\]

\begin{remark}[Connection to series tests]
\[
\text{Ratio behavior}
\Rightarrow
\text{Ratio Test for series}.
\]

\[
\text{Root behavior}
\Rightarrow
\text{Root Test for series}.
\]

Thus asymptotic growth tools directly govern convergence of infinite series.
\end{remark}

\begin{remark}[Hierarchy of growth]
For large $n$:
\[
\log n
\ll
n^k
\ll
c^n
\ll
n!
\]
for any $k>0$ and $c>1$.
\end{remark}

\begin{remark}[Logical Structure]
\[
\text{Difference quotients}
\Rightarrow
\text{Stolz--Cesàro}
\Rightarrow
\text{Asymptotic comparison}
\Rightarrow
\text{Growth hierarchy}
\Rightarrow
\text{Series convergence behavior}.
\]
\end{remark}
