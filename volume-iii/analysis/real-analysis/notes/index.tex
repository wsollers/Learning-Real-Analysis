% =========================================================
% Breadcrumb
% =========================================================
\begin{tcolorbox}[
  colback=gray!6,
  colframe=gray!40,
  arc=2pt,
  left=8pt, right=8pt, top=6pt, bottom=6pt,
  title={\small\textbf{Where You Are in the Journey}},
  fonttitle=\small\bfseries
]
\begin{center}
\small
Propositional Logic
$\;\to\;$ Predicate Calculus
$\;\to\;$ Sets \& Functions
$\;\to\;$ Proof Techniques
$\;\to\;$ $\mathbb{N}$, $\mathbb{Z}$, $\mathbb{Q}$, $\mathbb{R}$
$\;\to\;$ \textbf{Real Analysis}
$\;\to\;$ Algebraic Structures
$\;\to\;$ Metric Spaces
$\;\to\;$ Topology
$\;\to\;$ $\cdots$
\end{center}

\medskip
\noindent\textbf{How we got here.}
Volume~II established the real numbers as a complete ordered field
and proved the foundational theorems about bounds, sequences, and series.
Real analysis takes these tools and asks deeper questions: when do
functions have limits? When are they continuous, differentiable, integrable?
What structure underlies convergence?

\medskip
\noindent\textbf{What this chapter builds.}
We develop rigorous $\varepsilon$-$N$ and $\varepsilon$-$\delta$ analysis
on $\mathbb{R}$: limits of sequences and series, continuity and uniform
continuity, differentiation, and Riemann integration.
The Bolzano--Weierstrass theorem, the Heine--Borel theorem, and the
completeness of $\mathbb{R}$ are the engines driving every major result.

\medskip
\noindent\textbf{Where this leads.}
Metric spaces generalise the distance structure of $\mathbb{R}$ to
arbitrary sets. Topology abstracts further, replacing distance with
open sets. Real analysis on $\mathbb{R}$ is the prototype for all of it.
\end{tcolorbox}
\vspace{1em}

% =========================================================
% Structural Roadmap
% =========================================================
\subsection*{Structural Roadmap}

The development follows the definition--theorem--structure architecture.
The foundations of $\mathbb{R}$ (axioms, bounds, completeness, sequences,
series) are developed in Volume~II, Chapter: Real Numbers.
This chapter focuses on the \emph{analytic} content:

\begin{center}
\textbf{Completeness $\longrightarrow$ Convergence
$\longrightarrow$ Continuity $\longrightarrow$ Differentiation
$\longrightarrow$ Integration}
\end{center}

The global progression is:
\begin{enumerate}
  \item Limits of functions and continuity
  \item Uniform continuity and the extreme value theorem
  \item Differentiation and the mean value theorem
  \item The Riemann integral
  \item The fundamental theorem of calculus
  \item Uniform convergence of function sequences
  \item Power series as functions
  \item Applications and worked examples
\end{enumerate}

\vspace{1em}

\begin{remark}[Cross-reference to Volume II]
The notes on sequences, series, bounds, and completeness live in
\textbf{Volume~II, Chapter: Real Numbers ($\mathbb{R}$)}.
That chapter is the direct prerequisite for everything here.
\end{remark}

\begin{remark}[Primary sources]
Abbott, \textit{Understanding Analysis};
Rudin, \textit{Principles of Mathematical Analysis};
Pons, \textit{Real Analysis for the Undergraduate};
Ross, \textit{Elementary Analysis}.
\end{remark}

% =========================================================
% Content notes pulled from Volume II foundations
% (sequence/series theory lives in volume-ii/reals/notes)
% =========================================================
% Future: add notes on continuity, differentiation, integration here
% For now, sequence and series theory is referenced via volume-ii
