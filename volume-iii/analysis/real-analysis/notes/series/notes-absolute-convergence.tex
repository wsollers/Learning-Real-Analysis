% =========================================================
% Absolute and Conditional Convergence
% File: notes-absolute-convergence.tex
% =========================================================

\subsection{Absolute and Conditional Convergence}

% ---------------------------------------------------------
% Toolkit
% ---------------------------------------------------------
\begin{tcolorbox}[colback=gray!6, colframe=gray!40, arc=2pt,
  left=6pt, right=6pt, top=4pt, bottom=4pt,
  title={\small\textbf{Absolute Convergence — Quick Reference}},
  fonttitle=\small\bfseries]
\begin{tabular}{@{}p{0.28\textwidth}p{0.68\textwidth}@{}}
\textbf{Core items} & Key definitions/results introduced in this file.\\
\textbf{How to use} & Read the boxed items first; proofs and consequences follow.\\
\textbf{Dependencies} & Refer back to earlier sections as needed.\\
\end{tabular}
\end{tcolorbox}


% ---------------------------------------------------------
\subsubsection{Basic Definitions}

\begin{tcolorbox}[colback=propbox, colframe=propborder, arc=2pt,
  left=6pt, right=6pt, top=4pt, bottom=4pt,
  title={\small\textbf{Definition (Absolute convergence)}},
  fonttitle=\small\bfseries]
A series
\[
\sum_{n=1}^{\infty} a_n
\]
is said to \emph{converge absolutely} if the series
\[
\sum_{n=1}^{\infty} |a_n|
\]
converges.
\end{tcolorbox}

\begin{tcolorbox}[colback=propbox, colframe=propborder, arc=2pt,
  left=6pt, right=6pt, top=4pt, bottom=4pt,
  title={\small\textbf{Definition (Conditional convergence)}},
  fonttitle=\small\bfseries]
A series
\[
\sum_{n=1}^{\infty} a_n
\]
is said to \emph{converge conditionally} if it converges, but does not converge absolutely.
\end{tcolorbox}

\begin{remark}[Logical form]
\[
\text{Absolute convergence}
\quad\Longleftrightarrow\quad
\sum |a_n| \text{ converges}.
\]

\[
\text{Conditional convergence}
\quad\Longleftrightarrow\quad
\sum a_n \text{ converges and } \sum |a_n| \text{ diverges}.
\]
\end{remark}

\begin{remark}
Absolute convergence is a stronger property than convergence.
It imposes global control on the total variation of the series.
\end{remark}

% ---------------------------------------------------------
\subsubsection{Main Theorems}

\begin{theorem}[Absolute convergence implies convergence]
If
\[
\sum_{n=1}^{\infty} |a_n|
\]
converges, then
\[
\sum_{n=1}^{\infty} a_n
\]
converges.
\end{theorem}



\begin{remark}
The proof uses only:
\begin{itemize}
\item the triangle inequality,
\item the Cauchy Criterion,
\item completeness of $\mathbb{R}$.
\end{itemize}
\end{remark}

\begin{theorem}[Comparison via absolute values]
If $|a_n| \le b_n$ for all $n$, where $b_n \ge 0$ and
\[
\sum b_n
\]
converges, then
\[
\sum a_n
\]
converges absolutely (and hence converges).
\end{theorem}



\begin{theorem}[Absolute convergence is rearrangement invariant]
If a series converges absolutely, then every rearrangement of the series
converges to the same sum.
\end{theorem}

\begin{remark}
The proof of this theorem requires additional combinatorial estimates
and will be developed later in the study of rearrangements.
The key idea is that absolute convergence prevents cancellation effects
from altering the limit.
\end{remark}

% ---------------------------------------------------------
\subsubsection{Canonical Examples}

\begin{tcolorbox}[colback=propbox, colframe=propborder, arc=2pt,
  left=6pt, right=6pt, top=4pt, bottom=4pt,
  title={\small\textbf{Example (Geometric series)}},
  fonttitle=\small\bfseries]
For $|r|<1$,
\[
\sum_{n=0}^{\infty} r^n
\]
converges absolutely since
\[
\sum |r|^n
\]
is geometric.
\end{tcolorbox}

\begin{tcolorbox}[colback=propbox, colframe=propborder, arc=2pt,
  left=6pt, right=6pt, top=4pt, bottom=4pt,
  title={\small\textbf{Example (Alternating harmonic series)}},
  fonttitle=\small\bfseries]
\[
\sum_{n=1}^{\infty} \frac{(-1)^{n+1}}{n}
\]
converges, but
\[
\sum_{n=1}^{\infty} \frac{1}{n}
\]
diverges.

Hence it is conditionally convergent.
\end{tcolorbox}

% ---------------------------------------------------------
\subsubsection{Consequences}

The logical implication of this section is:

\[
\text{Absolute convergence}
\;\Rightarrow\;
\text{Convergence}.
\]

However,

\[
\text{Convergence}
\;\not\Rightarrow\;
\text{Absolute convergence}.
\]

\begin{remark}[Logical Structure]
The major sequence theorems interlock as follows:

\[
\sum |a_n| \text{ converges}
\Rightarrow
\text{Cauchy partial sums}
\Rightarrow
\text{Convergent series}.
\]

Absolute convergence therefore sits structurally between:

\[
\text{Comparison tests}
\quad\text{and}\quad
\text{Rearrangement theory}.
\]
\end{remark}

\begin{remark}[Philosophical interpretation]
Absolute convergence eliminates the possibility that convergence
is caused merely by oscillatory cancellation.
It measures the total accumulated magnitude of the series.
\end{remark}
