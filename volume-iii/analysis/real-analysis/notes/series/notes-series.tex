% =========================================================
% Series
% File: notes-series.tex
% =========================================================

\subsection{Series}

% ---------------------------------------------------------
% Toolkit
% ---------------------------------------------------------
\begin{tcolorbox}[colback=gray!6, colframe=gray!40, arc=2pt,
  left=6pt, right=6pt, top=4pt, bottom=4pt,
  title={\small\textbf{Series — Quick Reference}},
  fonttitle=\small\bfseries]
\begin{tabular}{@{}p{0.28\textwidth}p{0.68\textwidth}@{}}
\textbf{Core items} & Key definitions/results introduced in this file.\\
\textbf{How to use} & Read the boxed items first; proofs and consequences follow.\\
\textbf{Dependencies} & Refer back to earlier sections as needed.\\
\end{tabular}
\end{tcolorbox}


% ---------------------------------------------------------
\subsubsection{Basic Definitions}

\begin{tcolorbox}[colback=propbox, colframe=propborder, arc=2pt,
  left=6pt, right=6pt, top=4pt, bottom=4pt,
  title={\small\textbf{Definition (Series)}},
  fonttitle=\small\bfseries]
Let $(a_n)$ be a sequence of real numbers.
The \emph{series} associated with $(a_n)$ is the formal expression
\[
\sum_{n=1}^\infty a_n.
\]
\end{tcolorbox}

\begin{remark}
A series is not itself a number, but a symbolic object whose meaning
is defined via its sequence of partial sums.
\end{remark}

\begin{tcolorbox}[colback=propbox, colframe=propborder, arc=2pt,
  left=6pt, right=6pt, top=4pt, bottom=4pt,
  title={\small\textbf{Definition (Partial sums)}},
  fonttitle=\small\bfseries]
Given a sequence $(a_n)$, define the sequence of \emph{partial sums}
$(s_N)$ by
\[
s_N := \sum_{n=1}^N a_n.
\]
\end{tcolorbox}

\begin{remark}
The series $\sum_{n=1}^\infty a_n$ is said to \emph{converge} if the sequence
of partial sums $(s_N)$ converges in $\mathbb{R}$.
\end{remark}

\begin{remark}[Logical structure]
\[
\sum_{n=1}^\infty a_n \text{ converges }
\iff
\exists L\in\mathbb{R}\;
\bigl(
\lim_{N\to\infty} s_N = L
\bigr).
\]
\end{remark}

% ---------------------------------------------------------
\subsubsection{Main Theorems}

\begin{theorem}[Cauchy Condensation Test]
Let $(a_n)$ be a nonincreasing sequence of nonnegative real numbers.
Then the series
\[
\sum_{n=1}^\infty a_n
\]
converges if and only if the condensed series
\[
\sum_{k=0}^\infty 2^k\, a_{2^k}
\]
converges.
\end{theorem}



% ---------------------------------------------------------
\subsubsection{Consequences}

The logical implication of this section is:

\begin{itemize}
\item A series is completely determined by its sequence of partial sums.
\item Convergence of a series is therefore a special case of sequence convergence.
\item Tests for series are methods for proving convergence of the associated
partial-sum sequence.
\item The Cauchy Condensation Test reduces certain monotone nonnegative series
to a sparser dyadic subsequence.
\end{itemize}

\begin{remark}[Structural Position]
\[
\text{Series convergence}
=
\text{Convergence of partial sums}.
\]

Thus all sequence results (Cauchy Criterion, Bolzano--Weierstrass,
Algebra of Limits, etc.) apply immediately to series
via the sequence $(s_N)$.
\end{remark}
