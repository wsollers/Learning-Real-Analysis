% =========================================================
% Power Series and Radius of Convergence
% File: notes-power-series.tex
% =========================================================

\subsection{Power Series and Radius of Convergence}

% ---------------------------------------------------------
% Toolkit
% ---------------------------------------------------------
\begin{tcolorbox}[colback=gray!6, colframe=gray!40, arc=2pt,
  left=6pt, right=6pt, top=4pt, bottom=4pt,
  title={\small\textbf{Power Series — Quick Reference}},
  fonttitle=\small\bfseries]
\begin{tabular}{@{}p{0.28\textwidth}p{0.68\textwidth}@{}}
\textbf{Core items} & Key definitions/results introduced in this file.\\
\textbf{How to use} & Read the boxed items first; proofs and consequences follow.\\
\textbf{Dependencies} & Refer back to earlier sections as needed.\\
\end{tabular}
\end{tcolorbox}


% ---------------------------------------------------------
\subsubsection{Basic Definitions}

\begin{tcolorbox}[colback=propbox, colframe=propborder, arc=2pt,
  left=6pt, right=6pt, top=4pt, bottom=4pt,
  title={\small\textbf{Definition (Power series)}},
  fonttitle=\small\bfseries]
Let $(a_n)$ be a sequence of real numbers and let $c \in \mathbb{R}$.
A \emph{power series centered at $c$} is a series of the form
\[
\sum_{n=0}^{\infty} a_n (x-c)^n.
\]
\end{tcolorbox}

\begin{tcolorbox}[colback=propbox, colframe=propborder, arc=2pt,
  left=6pt, right=6pt, top=4pt, bottom=4pt,
  title={\small\textbf{Definition (Radius of convergence)}},
  fonttitle=\small\bfseries]
The \emph{radius of convergence} of a power series
\[
\sum_{n=0}^{\infty} a_n (x-c)^n
\]
is the number $R \in [0,\infty]$ such that:

\begin{itemize}
\item the series converges absolutely whenever $|x-c| < R$,
\item the series diverges whenever $|x-c| > R$.
\end{itemize}
\end{tcolorbox}

\begin{tcolorbox}[colback=propbox, colframe=propborder, arc=2pt,
  left=6pt, right=6pt, top=4pt, bottom=4pt,
  title={\small\textbf{Definition (Interval of convergence)}},
  fonttitle=\small\bfseries]
The \emph{interval of convergence} is the set of $x$ for which the series converges.
It is of the form
\[
(c-R,c+R)
\]
possibly including one or both endpoints.
\end{tcolorbox}

% ---------------------------------------------------------
\subsubsection{Main Theorems}

\begin{theorem}[Radius of Convergence Theorem]
For every power series
\[
\sum_{n=0}^{\infty} a_n (x-c)^n,
\]
there exists $R \in [0,\infty]$ such that:

\begin{enumerate}
\item The series converges absolutely for all $|x-c|<R$.
\item The series diverges for all $|x-c|>R$.
\end{enumerate}
\end{theorem}



\begin{theorem}[Cauchy--Hadamard Formula]
Let
\[
\sum_{n=0}^{\infty} a_n (x-c)^n
\]
be a power series. Then the radius of convergence is
\[
R
=
\frac{1}{\limsup_{n\to\infty} |a_n|^{1/n}}.
\]
\end{theorem}



\begin{theorem}[Term-by-term differentiation]
Let
\[
f(x) = \sum_{n=0}^{\infty} a_n (x-c)^n
\]
have radius of convergence $R>0$.
Then for all $|x-c|<R$:

\begin{enumerate}
\item The series converges uniformly on every closed interval
\[
[c-r,c+r] \subset (c-R,c+R).
\]
\item The function $f$ is differentiable on $(c-R,c+R)$.
\item The derivative is obtained by term-by-term differentiation:
\[
f'(x)
=
\sum_{n=1}^{\infty} n a_n (x-c)^{n-1}.
\]
\item The differentiated series has the same radius of convergence $R$.
\end{enumerate}
\end{theorem}



% ---------------------------------------------------------
\subsubsection{Consequences and Logical Structure}

\begin{remark}[Structural Position]
Power series sit at the intersection of:

\[
\text{Sequences}
\rightarrow
\text{Series}
\rightarrow
\text{Absolute convergence}
\rightarrow
\text{Root test}
\rightarrow
\text{limsup}.
\]

The Cauchy--Hadamard formula is the culmination of the entire
limsup theory.
\end{remark}

\begin{remark}[Uniform convergence inside the radius]
On every compact subinterval of $(c-R,c+R)$, power series converge uniformly.
This makes them exceptionally well-behaved:
\[
\text{Inside } R:
\quad
\text{Uniform convergence}
\Rightarrow
\text{Continuous}
\Rightarrow
\text{Differentiable}
\Rightarrow
\text{Smooth}.
\]
\end{remark}

\begin{remark}[Completeness connection]
The existence of $R$ ultimately depends on:
\begin{itemize}
\item completeness of $\mathbb{R}$,
\item limsup existence,
\item root test,
\item absolute convergence theory.
\end{itemize}

Thus power series are a structural synthesis of the entire sequence
and series development.
\end{remark}


