% =========================================================
% Manipulation and Rearrangement of Series
% File: notes-series-manipulations.tex
% =========================================================

\subsection{Manipulation and Rearrangement of Series}

% ---------------------------------------------------------
% Toolkit
% ---------------------------------------------------------
\begin{tcolorbox}[colback=gray!6, colframe=gray!40, arc=2pt,
  left=6pt, right=6pt, top=4pt, bottom=4pt,
  title={\small\textbf{Series Rearrangements — Quick Reference}},
  fonttitle=\small\bfseries]
\begin{tabular}{@{}p{0.28\textwidth}p{0.68\textwidth}@{}}
\textbf{Core items} & Key definitions/results introduced in this file.\\
\textbf{How to use} & Read the boxed items first; proofs and consequences follow.\\
\textbf{Dependencies} & Refer back to earlier sections as needed.\\
\end{tabular}
\end{tcolorbox}


% ---------------------------------------------------------
\subsubsection{Basic Definitions}
% ---------------------------------------------------------

\begin{tcolorbox}[colback=propbox, colframe=propborder, arc=2pt,
  left=6pt, right=6pt, top=4pt, bottom=4pt,
  title={\small\textbf{Definition (Rearrangement of a series)}},
  fonttitle=\small\bfseries]
Let $\sum_{n=1}^\infty a_n$ be a series.
A \emph{rearrangement} of this series is any series of the form
\[
\sum_{n=1}^\infty a_{\sigma(n)},
\]
where $\sigma : \mathbb{N} \to \mathbb{N}$ is a bijection.
\end{tcolorbox}

\begin{remark}
A rearrangement preserves all terms of the series,
but possibly changes their order.
\end{remark}

\begin{tcolorbox}[colback=propbox, colframe=propborder, arc=2pt,
  left=6pt, right=6pt, top=4pt, bottom=4pt,
  title={\small\textbf{Definition (Regrouping)}},
  fonttitle=\small\bfseries]
A \emph{regrouping} of a series consists of inserting parentheses
into the series in such a way that finitely many consecutive terms
are summed together before taking limits.
\end{tcolorbox}

% ---------------------------------------------------------
\subsubsection{Main Theorems}
% ---------------------------------------------------------

% =========================================================
% Absolute convergence stability
% =========================================================

\begin{theorem}[Absolute convergence is stable under rearrangement]
If $\sum a_n$ converges absolutely, then every rearrangement
\[
\sum a_{\sigma(n)}
\]
converges and has the same sum.
\end{theorem}



% =========================================================
% Riemann Rearrangement Theorem
% =========================================================

\begin{theorem}[Riemann Rearrangement Theorem]
If $\sum a_n$ converges conditionally
(i.e.\ converges but not absolutely),
then for every $L \in \mathbb{R}$
there exists a rearrangement that converges to $L$.

Moreover, there exist rearrangements that diverge to
$+\infty$ or $-\infty$.
\end{theorem}



% =========================================================
% Regrouping stability
% =========================================================

\begin{theorem}[Regrouping and convergence]
If $\sum a_n$ converges,
then any regrouping of finitely many consecutive terms
converges to the same sum.
\end{theorem}



\begin{remark}
Regrouping preserves convergence.
Rearrangement does not — unless the series converges absolutely.
\end{remark}

% =========================================================
% Cauchy Product
% =========================================================

\begin{tcolorbox}[colback=propbox, colframe=propborder, arc=2pt,
  left=6pt, right=6pt, top=4pt, bottom=4pt,
  title={\small\textbf{Definition (Cauchy product)}},
  fonttitle=\small\bfseries]
Let $\sum a_n$ and $\sum b_n$ be two series.
The \emph{Cauchy product} is the series
\[
\sum_{n=0}^\infty c_n,
\quad
c_n := \sum_{k=0}^n a_k b_{n-k}.
\]
\end{tcolorbox}

\begin{theorem}[Cauchy Product Theorem]
If both $\sum a_n$ and $\sum b_n$ converge absolutely,
then the Cauchy product converges absolutely and
\[
\sum c_n
=
\left(\sum a_n\right)
\left(\sum b_n\right).
\]
\end{theorem}



% =========================================================
% Failure without absolute convergence
% =========================================================

\begin{remark}
If convergence is not absolute,
the Cauchy product may fail to converge.

Similarly, rearrangements can change the value.
\end{remark}

% ---------------------------------------------------------
\subsubsection{Consequences and Logical Structure}
% ---------------------------------------------------------

The hierarchy of stability is:

\[
\text{Absolute convergence}
\Rightarrow
\text{Rearrangement stability}
\Rightarrow
\text{Cauchy product stability}.
\]

Conditional convergence implies instability under rearrangement.

\begin{remark}[Philosophical Summary]
Absolute convergence behaves like finite sums.

Conditional convergence behaves like an infinite balancing act:
order matters.
\end{remark}
