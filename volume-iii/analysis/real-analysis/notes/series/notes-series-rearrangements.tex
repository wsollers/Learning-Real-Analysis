% =========================================================
% Manipulation and Rearrangement of Series
% File: notes-series-manipulations.tex
% =========================================================

\subsection{Manipulation and Rearrangement of Series}

% ---------------------------------------------------------
% Toolkit
% ---------------------------------------------------------
\begin{tcolorbox}[title=Toolkit: Series Rearrangements]
\begin{tabular}{@{}p{0.28\textwidth}p{0.68\textwidth}@{}}
\textbf{Core items} & Key definitions/results introduced in this file.\\
\textbf{How to use} & Read the boxed items first; proofs and consequences follow.\\
\textbf{Dependencies} & Refer back to earlier sections as needed.\\
\end{tabular}
\end{tcolorbox}


% ---------------------------------------------------------
\subsubsection{Basic Definitions}
% ---------------------------------------------------------

\begin{tcolorbox}[title=Definition (Rearrangement of a series)]
Let $\sum_{n=1}^\infty a_n$ be a series.
A \emph{rearrangement} of this series is any series of the form
\[
\sum_{n=1}^\infty a_{\sigma(n)},
\]
where $\sigma : \mathbb{N} \to \mathbb{N}$ is a bijection.
\end{tcolorbox}

\begin{remark}
A rearrangement preserves all terms of the series,
but possibly changes their order.
\end{remark}

\begin{tcolorbox}[title=Definition (Regrouping)]
A \emph{regrouping} of a series consists of inserting parentheses
into the series in such a way that finitely many consecutive terms
are summed together before taking limits.
\end{tcolorbox}

% ---------------------------------------------------------
\subsubsection{Main Theorems}
% ---------------------------------------------------------

% =========================================================
% Absolute convergence stability
% =========================================================

\begin{theorem}[Absolute convergence is stable under rearrangement]
If $\sum a_n$ converges absolutely, then every rearrangement
\[
\sum a_{\sigma(n)}
\]
converges and has the same sum.
\end{theorem}

\begin{proof}
Assume $\sum |a_n|$ converges.
Let $S = \sum a_n$.

Let $\sigma$ be any bijection.
Define partial sums of the rearranged series:
\[
S_N' = \sum_{n=1}^N a_{\sigma(n)}.
\]

Because $\sum |a_n|$ converges,
the tails of the original series can be made arbitrarily small:
\[
\forall \varepsilon>0 \;\exists M
\quad\text{s.t.}\quad
\sum_{n>M} |a_n| < \varepsilon.
\]

Choose $N$ large enough that
$\{\sigma(1),\dots,\sigma(N)\}$ contains all indices $\le M$.
Then
\[
S_N' - S
=
\sum_{\sigma(n)>M} a_{\sigma(n)}
-
\sum_{n>M} a_n.
\]

Both sums are subseries of the tail,
hence bounded in absolute value by
\[
\sum_{n>M} |a_n| < \varepsilon.
\]

Thus $|S_N' - S| < \varepsilon$.
Hence $S_N' \to S$.
\end{proof}

% =========================================================
% Riemann Rearrangement Theorem
% =========================================================

\begin{theorem}[Riemann Rearrangement Theorem]
If $\sum a_n$ converges conditionally
(i.e.\ converges but not absolutely),
then for every $L \in \mathbb{R}$
there exists a rearrangement that converges to $L$.

Moreover, there exist rearrangements that diverge to
$+\infty$ or $-\infty$.
\end{theorem}

\begin{proof}[Proof sketch]
Since the series is conditionally convergent:

\[
\sum a_n^+ = \infty,
\qquad
\sum a_n^- = \infty,
\]

where
\[
a_n^+ = \max(a_n,0),
\quad
a_n^- = -\min(a_n,0).
\]

To rearrange to a prescribed $L$:

\begin{enumerate}
\item Add positive terms until the partial sum exceeds $L$.
\item Add negative terms until the partial sum drops below $L$.
\item Repeat.
\end{enumerate}

Because positive and negative parts both diverge,
this process continues indefinitely.

The oscillations shrink to zero since $a_n \to 0$.
Thus the rearranged series converges to $L$.

For divergence to $\pm\infty$, simply add only enough of one sign.
\end{proof}

% =========================================================
% Regrouping stability
% =========================================================

\begin{theorem}[Regrouping and convergence]
If $\sum a_n$ converges,
then any regrouping of finitely many consecutive terms
converges to the same sum.
\end{theorem}

\begin{proof}
Let $S_N$ be partial sums.
A regrouping defines a subsequence of $(S_N)$.

Since $S_N \to S$,
every subsequence converges to $S$.
\end{proof}

\begin{remark}
Regrouping preserves convergence.
Rearrangement does not — unless the series converges absolutely.
\end{remark}

% =========================================================
% Cauchy Product
% =========================================================

\begin{tcolorbox}[title=Definition (Cauchy product)]
Let $\sum a_n$ and $\sum b_n$ be two series.
The \emph{Cauchy product} is the series
\[
\sum_{n=0}^\infty c_n,
\quad
c_n := \sum_{k=0}^n a_k b_{n-k}.
\]
\end{tcolorbox}

\begin{theorem}[Cauchy Product Theorem]
If both $\sum a_n$ and $\sum b_n$ converge absolutely,
then the Cauchy product converges absolutely and
\[
\sum c_n
=
\left(\sum a_n\right)
\left(\sum b_n\right).
\]
\end{theorem}

\begin{proof}
Assume absolute convergence.

Then
\[
\sum_{n=0}^\infty \sum_{k=0}^n |a_k||b_{n-k}|
=
\left(\sum |a_n|\right)
\left(\sum |b_n|\right),
\]
by Fubini-type rearrangement for nonnegative series.

Thus the product series converges absolutely.

One verifies directly that partial sums
approximate the product of partial sums,
and the limit follows.
\end{proof}

% =========================================================
% Failure without absolute convergence
% =========================================================

\begin{remark}
If convergence is not absolute,
the Cauchy product may fail to converge.

Similarly, rearrangements can change the value.
\end{remark}

% ---------------------------------------------------------
\subsubsection{Consequences and Logical Structure}
% ---------------------------------------------------------

The hierarchy of stability is:

\[
\text{Absolute convergence}
\Rightarrow
\text{Rearrangement stability}
\Rightarrow
\text{Cauchy product stability}.
\]

Conditional convergence implies instability under rearrangement.

\begin{remark}[Philosophical Summary]
Absolute convergence behaves like finite sums.

Conditional convergence behaves like an infinite balancing act:
order matters.
\end{remark}
