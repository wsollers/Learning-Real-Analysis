% =========================================================
% Tests for Series
% File: notes-series-tests.tex
% =========================================================

\subsection{Tests for Series}

% ---------------------------------------------------------
% Toolkit
% ---------------------------------------------------------
\begin{tcolorbox}[title=Toolkit: Series Tests]
\begin{tabular}{@{}p{0.28\textwidth}p{0.68\textwidth}@{}}
\textbf{Core items} & Key definitions/results introduced in this file.\\
\textbf{How to use} & Read the boxed items first; proofs and consequences follow.\\
\textbf{Dependencies} & Refer back to earlier sections as needed.\\
\end{tabular}
\end{tcolorbox}


% ---------------------------------------------------------
\subsubsection{Basic Logical Structure}

All convergence tests reduce to properties of the partial sums
\[
s_N = \sum_{n=1}^N a_n.
\]

Thus every test ultimately proves that $(s_N)$ is either:

\begin{itemize}
\item bounded and monotone, or
\item Cauchy.
\end{itemize}

% =========================================================
\subsubsection{Main Theorems}
% =========================================================

% =========================================================
% Direct Comparison Test
% =========================================================

\begin{theorem}[Direct Comparison Test]
Let $a_n,b_n \ge 0$.

\begin{enumerate}
\item If $a_n \le b_n$ eventually and $\sum b_n$ converges,
then $\sum a_n$ converges.
\item If $a_n \ge b_n$ eventually and $\sum b_n$ diverges,
then $\sum a_n$ diverges.
\end{enumerate}
\end{theorem}

\begin{proof}
Assume $a_n \le b_n$ for $n \ge N_0$.

Define partial sums:
\[
A_N = \sum_{n=1}^N a_n,
\qquad
B_N = \sum_{n=1}^N b_n.
\]

For $N \ge N_0$,
\[
A_N
=
A_{N_0-1}
+
\sum_{n=N_0}^N a_n
\le
A_{N_0-1}
+
\sum_{n=N_0}^N b_n
\le
A_{N_0-1}
+
B_N.
\]

If $\sum b_n$ converges, $(B_N)$ is bounded.
Hence $(A_N)$ is bounded and increasing,
so $\sum a_n$ converges.

The divergence part follows similarly.
\end{proof}

% =========================================================
% Limit Comparison Test
% =========================================================

\begin{theorem}[Limit Comparison Test]
Let $a_n,b_n > 0$ and suppose
\[
\lim_{n\to\infty} \frac{a_n}{b_n} = L,
\quad 0 < L < \infty.
\]
Then
\[
\sum a_n \text{ converges }
\Longleftrightarrow
\sum b_n \text{ converges}.
\]
\end{theorem}

\begin{proof}
Choose $\varepsilon = L/2$.
For sufficiently large $n$,
\[
\left|\frac{a_n}{b_n} - L\right| < \frac{L}{2},
\]
so
\[
\frac{L}{2} < \frac{a_n}{b_n} < \frac{3L}{2}.
\]

Thus for large $n$,
\[
\frac{L}{2} b_n \le a_n \le \frac{3L}{2} b_n.
\]

Apply the Direct Comparison Test.
\end{proof}

% =========================================================
% Ratio Test
% =========================================================

\begin{theorem}[Ratio Test]
Let
\[
L = \limsup_{n\to\infty}
\left|\frac{a_{n+1}}{a_n}\right|.
\]

\begin{enumerate}
\item If $L < 1$, the series converges absolutely.
\item If $L > 1$, the series diverges.
\end{enumerate}
\end{theorem}

\begin{proof}
Assume $L<1$.
Choose $r$ such that $L<r<1$.

Then eventually
\[
\left|\frac{a_{n+1}}{a_n}\right| \le r.
\]

Hence for $n \ge N$,
\[
|a_n|
\le
|a_N| r^{\,n-N}.
\]

Thus $|a_n|$ is bounded above by a geometric sequence.
Since $\sum r^n$ converges, comparison yields
absolute convergence.

If $L>1$, then for infinitely many $n$,
\[
|a_{n+1}| > |a_n|.
\]
Hence $a_n$ does not tend to zero,
so the series diverges.
\end{proof}

% =========================================================
% Root Test
% =========================================================

\begin{theorem}[Root Test]
Let
\[
L = \limsup_{n\to\infty} \sqrt[n]{|a_n|}.
\]

\begin{enumerate}
\item If $L<1$, the series converges absolutely.
\item If $L>1$, it diverges.
\end{enumerate}
\end{theorem}

\begin{proof}
Assume $L<1$.
Choose $r$ with $L<r<1$.

Then for sufficiently large $n$,
\[
\sqrt[n]{|a_n|} \le r,
\]
so
\[
|a_n| \le r^n.
\]

Since $\sum r^n$ converges,
the series converges absolutely by comparison.

If $L>1$, then $|a_n|^{1/n} > 1$ infinitely often,
so $a_n$ does not tend to zero,
and the series diverges.
\end{proof}

% =========================================================
% Integral Test
% =========================================================

\begin{theorem}[Integral Test]
Let $f$ be continuous, positive, decreasing on $[1,\infty)$.
Let $a_n = f(n)$.

Then
\[
\sum a_n
\text{ converges }
\Longleftrightarrow
\int_1^\infty f(x)\,dx
\text{ converges}.
\]
\end{theorem}

\begin{proof}
For $n\ge1$,
\[
\int_{n+1}^{n+2} f(x)\,dx
\le
f(n+1)
\le
\int_n^{n+1} f(x)\,dx.
\]

Summing these inequalities yields
\[
\int_1^{N+1} f(x)\,dx
\le
\sum_{n=1}^N f(n)
\le
f(1)+\int_1^N f(x)\,dx.
\]

Thus the series converges iff the improper integral converges.
\end{proof}

% =========================================================
% p-Series
% =========================================================

\begin{theorem}[p-Series]
\[
\sum_{n=1}^\infty \frac{1}{n^p}
\]
converges iff $p>1$.
\end{theorem}

\begin{proof}
Apply the Integral Test to $f(x)=x^{-p}$.
\[
\int_1^\infty x^{-p}\,dx
=
\begin{cases}
\frac{1}{p-1}, & p>1,\\
\infty, & p\le1.
\end{cases}
\]
\end{proof}

% =========================================================
% Alternating Series Test
% =========================================================

\begin{theorem}[Alternating Series Test]
If $b_n \ge 0$, decreasing, and $b_n\to0$, then
\[
\sum (-1)^{n+1} b_n
\]
converges.
\end{theorem}

\begin{proof}
Let $S_N$ be partial sums.
Even and odd partial sums form monotone bounded sequences.

One verifies:
\[
S_{2k} \le S_{2k+2},
\qquad
S_{2k+1} \ge S_{2k+3}.
\]

Both are bounded and converge.
Their limits coincide since $b_n\to0$.
\end{proof}

% =========================================================
% Dirichlet Test
% =========================================================

\begin{theorem}[Dirichlet Test]
If
\begin{itemize}
\item partial sums of $\sum a_n$ are bounded,
\item $b_n$ is monotone and $b_n\to0$,
\end{itemize}
then $\sum a_n b_n$ converges.
\end{theorem}

\begin{proof}
Use summation by parts:
\[
\sum_{n=1}^N a_n b_n
=
A_N b_{N+1}
+
\sum_{n=1}^N A_n (b_n - b_{n+1}).
\]

Since $A_n$ is bounded and $b_n\to0$,
each term remains controlled.
The second sum converges by comparison.
\end{proof}

% =========================================================
% Abel Test
% =========================================================

\begin{theorem}[Abel Test]
If $\sum a_n$ converges and $b_n$ is bounded monotone,
then $\sum a_n b_n$ converges.
\end{theorem}

\begin{proof}
Again use summation by parts.
Since $A_n\to A$ and $b_n$ bounded monotone,
each term is controlled and convergence follows.
\end{proof}

% ---------------------------------------------------------
\subsubsection{Consequences}

Hierarchy of strength:

\[
\text{Root}
\Rightarrow
\text{Ratio}
\Rightarrow
\text{Comparison}.
\]

Absolute convergence tests imply unconditional stability.

Alternating / Dirichlet / Abel capture cancellation-driven convergence.

\begin{remark}[Logical Core]
All tests ultimately rely on:
\begin{itemize}
\item comparison,
\item Cauchy criterion,
\item bounded monotone convergence.
\end{itemize}
\end{remark}
