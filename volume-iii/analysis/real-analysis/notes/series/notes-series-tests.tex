% =========================================================
% Tests for Series
% File: notes-series-tests.tex
% =========================================================

\subsection{Tests for Series}

% ---------------------------------------------------------
% Toolkit
% ---------------------------------------------------------
\begin{tcolorbox}[colback=gray!6, colframe=gray!40, arc=2pt,
  left=6pt, right=6pt, top=4pt, bottom=4pt,
  title={\small\textbf{Series Tests — Quick Reference}},
  fonttitle=\small\bfseries]
\begin{tabular}{@{}p{0.28\textwidth}p{0.68\textwidth}@{}}
\textbf{Core items} & Key definitions/results introduced in this file.\\
\textbf{How to use} & Read the boxed items first; proofs and consequences follow.\\
\textbf{Dependencies} & Refer back to earlier sections as needed.\\
\end{tabular}
\end{tcolorbox}


% ---------------------------------------------------------
\subsubsection{Basic Logical Structure}

All convergence tests reduce to properties of the partial sums
\[
s_N = \sum_{n=1}^N a_n.
\]

Thus every test ultimately proves that $(s_N)$ is either:

\begin{itemize}
\item bounded and monotone, or
\item Cauchy.
\end{itemize}

% =========================================================
\subsubsection{Main Theorems}
% =========================================================

% =========================================================
% Direct Comparison Test
% =========================================================

\begin{theorem}[Direct Comparison Test]
Let $a_n,b_n \ge 0$.

\begin{enumerate}
\item If $a_n \le b_n$ eventually and $\sum b_n$ converges,
then $\sum a_n$ converges.
\item If $a_n \ge b_n$ eventually and $\sum b_n$ diverges,
then $\sum a_n$ diverges.
\end{enumerate}
\end{theorem}



% =========================================================
% Limit Comparison Test
% =========================================================

\begin{theorem}[Limit Comparison Test]
Let $a_n,b_n > 0$ and suppose
\[
\lim_{n\to\infty} \frac{a_n}{b_n} = L,
\quad 0 < L < \infty.
\]
Then
\[
\sum a_n \text{ converges }
\Longleftrightarrow
\sum b_n \text{ converges}.
\]
\end{theorem}



% =========================================================
% Ratio Test
% =========================================================

\begin{theorem}[Ratio Test]
Let
\[
L = \limsup_{n\to\infty}
\left|\frac{a_{n+1}}{a_n}\right|.
\]

\begin{enumerate}
\item If $L < 1$, the series converges absolutely.
\item If $L > 1$, the series diverges.
\end{enumerate}
\end{theorem}



% =========================================================
% Root Test
% =========================================================

\begin{theorem}[Root Test]
Let
\[
L = \limsup_{n\to\infty} \sqrt[n]{|a_n|}.
\]

\begin{enumerate}
\item If $L<1$, the series converges absolutely.
\item If $L>1$, it diverges.
\end{enumerate}
\end{theorem}



% =========================================================
% Integral Test
% =========================================================

\begin{theorem}[Integral Test]
Let $f$ be continuous, positive, decreasing on $[1,\infty)$.
Let $a_n = f(n)$.

Then
\[
\sum a_n
\text{ converges }
\Longleftrightarrow
\int_1^\infty f(x)\,dx
\text{ converges}.
\]
\end{theorem}



% =========================================================
% p-Series
% =========================================================

\begin{theorem}[p-Series]
\[
\sum_{n=1}^\infty \frac{1}{n^p}
\]
converges iff $p>1$.
\end{theorem}



% =========================================================
% Alternating Series Test
% =========================================================

\begin{theorem}[Alternating Series Test]
If $b_n \ge 0$, decreasing, and $b_n\to0$, then
\[
\sum (-1)^{n+1} b_n
\]
converges.
\end{theorem}



% =========================================================
% Dirichlet Test
% =========================================================

\begin{theorem}[Dirichlet Test]
If
\begin{itemize}
\item partial sums of $\sum a_n$ are bounded,
\item $b_n$ is monotone and $b_n\to0$,
\end{itemize}
then $\sum a_n b_n$ converges.
\end{theorem}



% =========================================================
% Abel Test
% =========================================================

\begin{theorem}[Abel Test]
If $\sum a_n$ converges and $b_n$ is bounded monotone,
then $\sum a_n b_n$ converges.
\end{theorem}



% ---------------------------------------------------------
\subsubsection{Consequences}

Hierarchy of strength:

\[
\text{Root}
\Rightarrow
\text{Ratio}
\Rightarrow
\text{Comparison}.
\]

Absolute convergence tests imply unconditional stability.

Alternating / Dirichlet / Abel capture cancellation-driven convergence.

\begin{remark}[Logical Core]
All tests ultimately rely on:
\begin{itemize}
\item comparison,
\item Cauchy criterion,
\item bounded monotone convergence.
\end{itemize}
\end{remark}
