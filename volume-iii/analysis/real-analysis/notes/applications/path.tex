% =========================================================
% Real Analysis — Sequences Completion Checklist
% =========================================================


\newpage


\begin{center}
    {\LARGE \textbf{Real Analysis — Sequences Completion Checklist}}\\[1em]
\end{center}

% =========================================================
\section*{I. Real Number Foundations}

\subsection*{Field \& Order Structure}

% ---------------------------------------------------------
% Toolkit
% ---------------------------------------------------------
\begin{tcolorbox}[title=Toolkit: Path]
\begin{tabular}{@{}p{0.28\textwidth}p{0.68\textwidth}@{}}
\textbf{Core items} & Key definitions/results introduced in this file.\\
\textbf{How to use} & Read the boxed items first; proofs and consequences follow.\\
\textbf{Dependencies} & Refer back to earlier sections as needed.\\
\end{tabular}
\end{tcolorbox}


\begin{itemize}
\item \checkbox Field axioms of $\mathbb{R}$
\item \checkbox Order axioms of $\mathbb{R}$
\item \checkbox Trichotomy Law
\item \checkbox Compatibility of order with addition
\item \checkbox Compatibility of order with multiplication
\end{itemize}

\subsection*{Absolute Value}

\begin{itemize}
\item \checkbox Definition of absolute value
\item \checkbox $|x| \ge 0$ and $|x| = 0 \iff x = 0$
\item \checkbox $|xy| = |x||y|$
\item \checkbox Triangle inequality
\item \checkbox Reverse triangle inequality
\end{itemize}

\subsection*{Suprema \& Infima}

\begin{itemize}
\item \checkbox Definition of upper bound
\item \checkbox Definition of lower bound
\item \checkbox Definition of supremum
\item \checkbox Definition of infimum
\item \checkbox Supremum is unique
\item \checkbox Completeness (Least Upper Bound Property)
\end{itemize}

\subsection*{Archimedean Property}

\begin{itemize}
\item \checkbox Archimedean property statement
\item \checkbox Equivalent forms
\item \checkbox Proof that $\frac{1}{n} \to 0$
\end{itemize}

% =========================================================
\section*{II. Basic Sequence Theory}

\subsection*{Definitions}

\begin{itemize}
\item \checkbox Definition of sequence ($\mathbb{N} \to \mathbb{R}$)
\item \checkbox Definition of convergence ($\varepsilon$–$N$ form)
\item \checkbox Negation of convergence
\end{itemize}

\subsection*{Fundamental Theorems}

\begin{itemize}
\item \checkbox Uniqueness of limits
\item \checkbox Convergent $\Rightarrow$ bounded
\end{itemize}

\subsection*{Algebra of Limits}

\begin{itemize}
\item \checkbox Sum rule
\item \checkbox Scalar multiple rule
\item \checkbox Product rule
\item \checkbox Quotient rule (nonzero limit)
\end{itemize}

\subsection*{Order Limit Theorems}

\begin{itemize}
\item \checkbox Limit preserves inequalities
\item \checkbox Squeeze theorem
\end{itemize}

% =========================================================
\section*{III. Structural Sequence Theory}

\subsection*{Monotone Sequences}

\begin{itemize}
\item \checkbox Definition of monotone increasing
\item \checkbox Definition of monotone decreasing
\item \checkbox Monotone Convergence Theorem
\end{itemize}

\subsection*{Subsequences}

\begin{itemize}
\item \checkbox Definition of subsequence
\item \checkbox $n_k$ strictly increasing
\item \checkbox Subsequence of convergent sequence converges to same limit
\item \checkbox Subsequence of subsequence lemma
\item \checkbox Index growth fact: $n_k \ge k$
\end{itemize}

\subsection*{Bolzano--Weierstrass}

\begin{itemize}
\item \checkbox Every bounded sequence has a convergent subsequence
\end{itemize}

% =========================================================
\section*{IV. Cauchy Theory \& Completeness}

\subsection*{Cauchy Sequences}

\begin{itemize}
\item \checkbox Definition of Cauchy sequence
\item \checkbox Convergent $\Rightarrow$ Cauchy
\item \checkbox Cauchy $\Rightarrow$ bounded
\item \checkbox Cauchy $\Rightarrow$ convergent (Completeness of $\mathbb{R}$)
\end{itemize}

% =========================================================
\section*{V. Limit Superior / Limit Inferior}

\begin{itemize}
\item \checkbox Definition of tail set
\item \checkbox Definition of $s_n = \sup_{k \ge n} a_k$
\item \checkbox Definition of $\limsup a_n$
\item \checkbox Definition of $\liminf a_n$
\item \checkbox $\limsup$ always exists (possibly $\pm\infty$)
\item \checkbox $\liminf$ always exists
\item \checkbox $\liminf \le \limsup$
\item \checkbox Convergence $\iff \limsup = \liminf$
\item \checkbox $\limsup$ equals largest subsequential limit
\item \checkbox $\liminf$ equals smallest subsequential limit
\end{itemize}

% =========================================================
\section*{VI. Subsequence Toolkit (Advanced)}

\begin{itemize}
\item \checkbox Finite Partition Convergence Principle
\item \checkbox Residue class convergence
\item \checkbox Even/odd convergence principle
\item \checkbox Dense subsequence criterion
\item \checkbox Diagonal subsequence lemma
\item \checkbox Inherited vs.\ reflected properties
\item \checkbox Tail properties
\item \checkbox Universal subsequence properties
\end{itemize}

% =========================================================
\section*{Final Mastery Check}

\begin{itemize}
\item \checkbox Prove Monotone Convergence from completeness
\item \checkbox Prove Bolzano--Weierstrass
\item \checkbox Prove Cauchy $\iff$ Convergent in $\mathbb{R}$
\item \checkbox Extract convergent subsequences intentionally
\item \checkbox Compute $\limsup$ and $\liminf$ in nontrivial examples
\item \checkbox Characterize convergence via $\limsup/\liminf$
\end{itemize}


\newpage