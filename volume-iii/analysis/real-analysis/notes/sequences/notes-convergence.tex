% =========================================================
% Convergence of Sequences in \mathbb{R}
% =========================================================

\subsection{Convergence of Sequences in $\mathbb{R}$}

% ---------------------------------------------------------
% Toolkit
% ---------------------------------------------------------
\begin{tcolorbox}[colback=gray!6, colframe=gray!40, arc=2pt,
  left=6pt, right=6pt, top=4pt, bottom=4pt,
  title={\small\textbf{Convergence — Quick Reference}},
  fonttitle=\small\bfseries]
\begin{tabular}{@{}p{0.28\textwidth}p{0.68\textwidth}@{}}
\textbf{Core items} & Key definitions/results introduced in this file.\\
\textbf{How to use} & Read the boxed items first; proofs and consequences follow.\\
\textbf{Dependencies} & Refer back to earlier sections as needed.\\
\end{tabular}
\end{tcolorbox}


% (Original local heading preserved from source extract)
\subsubsection{Convergence of Sequences}

% ---------------------------------------------------------
\subsubsection{Basic Definitions}

\begin{tcolorbox}[colback=propbox, colframe=propborder, arc=2pt,
  left=6pt, right=6pt, top=4pt, bottom=4pt,
  title={\small\textbf{Definition (Convergence — informal description)}},
  fonttitle=\small\bfseries]
A sequence $(x_n)$ of real numbers is said to \emph{converge} to a real number $L$
if the terms $x_n$ can be made arbitrarily close to $L$ by taking $n$ sufficiently large.
\end{tcolorbox}

\begin{remark}
This description captures the intuition of convergence but is not logically precise.
A rigorous definition is given below using $\varepsilon$-bounds.
\end{remark}

\begin{tcolorbox}[colback=propbox, colframe=propborder, arc=2pt,
  left=6pt, right=6pt, top=4pt, bottom=4pt,
  title={\small\textbf{Definition (Convergence — $\varepsilon$ definition)}},
  fonttitle=\small\bfseries]
Let $(x_n)$ be a sequence in $\mathbb{R}$ and let $L\in\mathbb{R}$.
We say that $(x_n)$ \emph{converges to $L$}, and write
\[
x_n \to L
\quad\text{or}\quad
\lim_{n\to\infty} x_n = L,
\]
if
\[
\forall \varepsilon>0\;
\exists N\in\mathbb{N}\;
\forall n\in\mathbb{N}\;
\bigl(n\ge N \rightarrow |x_n - L| < \varepsilon\bigr).
\]
\end{tcolorbox}

\begin{remark}[Logical form]
\[
\forall \varepsilon>0\;
\exists N\in\mathbb{N}\;
\forall n\in\mathbb{N}\;
\bigl(
n\ge N \Rightarrow x_n \in (L-\varepsilon,\,L+\varepsilon)
\bigr).
\]
\end{remark}

\begin{remark}
The number $N$ may depend on $\varepsilon$, but must work for all $n\ge N$.
\end{remark}

\begin{remark}[Source alignment]
The $\varepsilon$--definition of convergence is standard in real analysis texts;
see, e.g., \cite{AbbottUnderstandingAnalysis, TaoAnalysis1, RossElementaryAnalysis, Rudin1976, LeblBasicAnalysisI}.
\end{remark}

\begin{tcolorbox}[colback=propbox, colframe=propborder, arc=2pt,
  left=6pt, right=6pt, top=4pt, bottom=4pt,
  title={\small\textbf{Definition (Neighborhood formulation of convergence in $\mathbb{R}$)}},
  fonttitle=\small\bfseries]
Let $(x_n)$ be a sequence in $\mathbb{R}$ and let $L\in\mathbb{R}$.
The sequence $(x_n)$ converges to $L$ if and only if
\[
\forall U \subseteq \mathbb{R}\;
\bigl(
U \text{ is an open neighborhood of } L
\rightarrow
\exists N\in\mathbb{N}\;
\forall n\ge N,\ x_n\in U
\bigr).
\]
\end{tcolorbox}

\begin{remark}
Equivalently, $(x_n)$ converges to $L$ if and only if
\[
\text{for every open set } U \text{ containing } L,
\text{ all but finitely many terms of the sequence lie in } U.
\]
\end{remark}

\begin{remark}[Source alignment]
Neighborhood formulations of convergence are standard in topology/metric space treatments;
see, e.g., \cite{MendelsonTopology, AnderssonBjornWimanMetricSpaces, Rudin1976}.
\end{remark}

% ---------------------------------------------------------
\subsubsection{Main Theorems}

\begin{theorem}[Every convergent sequence is bounded]
Let $(x_n)$ be a sequence in $\mathbb{R}$.  
If $(x_n)$ converges, then $(x_n)$ is bounded.
\end{theorem}

\begin{proof}
Suppose $(x_n)$ converges to $L\in\mathbb{R}$.
By the $\varepsilon$--definition of convergence, there exists $N\in\mathbb{N}$
such that
\[
|x_n - L| < 1
\quad\text{for all } n \ge N.
\]
Hence, for all $n \ge N$,
\[
|x_n| \le |x_n - L| + |L| < 1 + |L|.
\]

Define
\[
M := \max\bigl\{|x_1|, |x_2|, \dots, |x_{N-1}|,\, 1+|L|\bigr\}.
\]
Then $|x_n| \le M$ for all $n\in\mathbb{N}$, so $(x_n)$ is bounded.
\end{proof}

\begin{remark}
Boundedness is a necessary but not sufficient condition for convergence.
\end{remark}

\begin{remark}
    This theorem assumes the $\varepsilon$--definition of convergence (and standard inequalities for absolute value).

    This theorem implies: whenever you prove $x_n \to L$, you automatically obtain a bound $|x_n|\le M$ for all $n$.
\end{remark}

\begin{theorem}[Equivalence of convergence definitions in $\mathbb{R}$]
For a sequence $(x_n)$ in $\mathbb{R}$ and a point $L\in\mathbb{R}$, the following
are equivalent:
\begin{enumerate}
\item $(x_n)$ converges to $L$ in the $\varepsilon$-sense.
\item For every $\varepsilon>0$, eventually $x_n\in (L-\varepsilon,L+\varepsilon)$.
\item For every open neighborhood $U$ of $L$, eventually $x_n\in U$.
\end{enumerate}
\end{theorem}

\begin{remark}
In $\mathbb{R}$, the $\varepsilon$-definition and the topological definition of
convergence coincide because open neighborhoods are precisely unions of open
intervals.
\end{remark}

\begin{remark}
    This theorem assumes: the standard topology on $\mathbb{R}$ (equivalently, the usual metric).
    
    This theorem implies: you may freely switch between $\varepsilon$--language and neighborhood language when working in $\mathbb{R}$.
\end{remark}








% ---------------------------------------------------------
% Figure (kept from source extract; included here as a visual aid)
% ---------------------------------------------------------
% Requires:
% \usepackage{tikz}
% \usetikzlibrary{arrows.meta}

\begin{figure}[h]
\centering
\begin{tikzpicture}[x=1.05cm,y=1.0cm,>=Latex, line cap=round, line join=round]

% ----------------------------
% Parameters
% ----------------------------
\def\L{3.0}        % level of ell
\def\Eps{0.6}      % epsilon
\def\m{6}          % index m

% ----------------------------
% Axes
% ----------------------------
\draw[->,thick] (0,0) -- (12,0) node[below right] {$n$};
\draw[->,thick] (0,0) -- (0,6.3) node[above left] {$s_n$};

% x-axis ticks
\foreach \k in {1,2,3,4,5,6,7,8,9,10,11}{
  \draw (\k,0) -- (\k,-0.12);
}

\node[below] at (0,0) {$0$};
\node[below] at (1,-0.12) {$1$};
\node[below] at (2,-0.12) {$2$};
\node[below] at (3,-0.12) {$3$};
\node[below] at (\m,-0.12) {$m$};
\node[below] at ({\m+1},-0.12) {$m+1$};
\node[below] at ({\m+2},-0.12) {$m+2$};

% ----------------------------
% Horizontal reference lines
% ----------------------------
\draw[dashed,thick] (0.4,{\L+\Eps}) -- (11.6,{\L+\Eps});
\draw[thick]        (0.4,\L)        -- (11.6,\L);
\draw[dashed,thick] (0.4,{\L-\Eps}) -- (11.6,{\L-\Eps});

\node[right] at (11.7,{\L+\Eps}) {$\ell+\varepsilon$};
\node[right] at (11.7,\L) {$\ell$};
\node[right] at (11.7,{\L-\Eps}) {$\ell-\varepsilon$};

% ----------------------------
% Vertical line at m
% ----------------------------
\draw[thick] (\m,0) -- (\m,6.1);

% ----------------------------
% Sequence points (left of m)
% ----------------------------
\fill (1,1.6) circle (2.2pt) node[below left] {$s_1$};
\fill (2,5.1) circle (2.2pt) node[above] {$s_2$};
\fill (3,2.1) circle (2.2pt) node[above right] {$s_3$};
\fill (4,4.3) circle (2.2pt) node[above right] {$s_4$};
\fill (5,2.6) circle (2.2pt) node[above right] {$s_5$};

% Point at m
\fill (\m,{\L-\Eps+0.05}) circle (2.2pt) node[below right] {$s_m$};

% ----------------------------
% Points after m (inside epsilon band)
% ----------------------------
\fill ({\m+1}, {\L-0.15}) circle (2.2pt) node[above right] {$s_{m+1}$};
\fill ({\m+2}, {\L+0.18}) circle (2.2pt) node[above right] {$s_{m+2}$};
\fill ({\m+3}, {\L-0.28}) circle (2.2pt) node[above right] {$s_{m+3}$};

% ----------------------------
% Bracket for epsilon neighborhood
% ----------------------------
\draw[thin] (10.4,{\L-\Eps}) -- (10.8,{\L-\Eps});
\draw[thin] (10.4,{\L+\Eps}) -- (10.8,{\L+\Eps});
\draw[thin] (10.8,{\L-\Eps}) -- (10.8,{\L+\Eps});

\node[right] at (10.9,\L) {$\varepsilon$-neighborhood of $\ell$};

\end{tikzpicture}
\caption{Illustration of $\varepsilon$--convergence of a sequence $(s_n)$ to $\ell$.}
\end{figure}

% ---------------------------------------------------------
\subsubsection{Consequences}

The logical implication of this entire section is:

\begin{itemize}
\item The informal ``terms get closer to $L$'' intuition is made rigorous by a quantifier pattern
\[
\forall \varepsilon>0\ \exists N\ \forall n\ge N:\ |x_n-L|<\varepsilon.
\]
\item Convergence forces boundedness:
\[
x_n \to L \;\Rightarrow\; (x_n)\ \text{is bounded}.
\]
\item In $\mathbb{R}$, the $\varepsilon$--definition and the neighborhood definition are equivalent, so either language may be used depending on context.
\end{itemize}

\begin{remark}[Logical Structure]
The major convergence statements interlock as follows:
\[
\text{(Informal intuition)}
\Rightarrow
\text{($\varepsilon$--definition)}
\Rightarrow
\text{Boundedness Theorem}
\]
and in parallel
\[
\text{($\varepsilon$--definition)}
\Longleftrightarrow
\text{Neighborhood definition in $\mathbb{R}$}.
\]
\end{remark}

\begin{proposition}
The converse of the statement
\[
\text{convergent} \Rightarrow \text{bounded}
\]
is false.
\end{proposition}

\begin{proof}
Boundedness of a sequence $(x_n)$ means
\[
\exists M>0 \;
\forall n \in \mathbb{N} :
|x_n| \le M.
\]

Convergence to $L$ means
\[
\exists L \in \mathbb{R} \;
\forall \varepsilon > 0 \;
\exists N \in \mathbb{N} \;
\forall n \ge N :
|x_n - L| < \varepsilon.
\]

The quantifier structure of convergence is strictly stronger.

Consider the sequence $x_n = (-1)^n$. Then
\[
|x_n| \le 1 \quad \text{for all } n,
\]
so the sequence is bounded.

Suppose for contradiction that $(x_n)$ converges to some $L$.
Let $\varepsilon = \tfrac12$. Then there exists $N$ such that
\[
|(-1)^n - L| < \tfrac12
\quad\text{for all } n \ge N.
\]

But infinitely many terms equal $1$ and infinitely many equal $-1$,
so both
\[
|1 - L| < \tfrac12
\quad\text{and}\quad
|-1 - L| < \tfrac12
\]
must hold, which is impossible.
Therefore the sequence is bounded but not convergent.
\end{proof}



\begin{remark}[Dependence on the metric]
The definition of convergence depends on the metric of the space.

In $\mathbb{R}$, the metric is
\[
d(x,y) = |x-y|,
\]
so the familiar $\varepsilon$--definition of convergence uses absolute value.

However, in a general metric space $(X,d)$, convergence is defined by
\[
d(x_n,x) < \varepsilon.
\]

Thus absolute value is not an intrinsic feature of convergence;
it is simply the metric on $\mathbb{R}$.
\end{remark}
