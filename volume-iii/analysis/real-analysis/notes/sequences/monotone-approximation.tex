% =========================================================
% Monotone Approximation + Completeness Equivalences
% =========================================================

\subsection{Monotone Approximation and Completeness Equivalences}

% ---------------------------------------------------------
% Toolkit
% ---------------------------------------------------------
\begin{tcolorbox}[colback=gray!6, colframe=gray!40, arc=2pt,
  left=6pt, right=6pt, top=4pt, bottom=4pt,
  title={\small\textbf{Monotone Approximation — Quick Reference}},
  fonttitle=\small\bfseries]
\begin{tabular}{@{}p{0.28\textwidth}p{0.68\textwidth}@{}}
\textbf{Core items} & Key definitions/results introduced in this file.\\
\textbf{How to use} & Read the boxed items first; proofs and consequences follow.\\
\textbf{Dependencies} & Refer back to earlier sections as needed.\\
\end{tabular}
\end{tcolorbox}

\label{sec:monotone-approx-and-completeness}

% ---------------------------------------------------------
\subsubsection{Monotone Approximation of Suprema and Infima}

\begin{proposition}[Monotone Approximation of Bounds]
\label{prop:monotone-approx-bounds}
Let $S \subset \mathbb{R}$ be nonempty and bounded. Then there exist
monotone sequences $\{x_n\}_{n\in\mathbb{N}}$ and $\{y_n\}_{n\in\mathbb{N}}$
such that $x_n, y_n \in S$ for all $n$ and
\[
\sup S = \lim_{n\to\infty} x_n,
\qquad
\inf S = \lim_{n\to\infty} y_n.
\]
\end{proposition}

\begin{remark}[Source note]\label{rem:source-lebl-2.1.13}
\srccite{LeblBasicAnalysisI}{Proposition~2.1.13}.
\end{remark}


% ---------------------------------------------------------
\subsubsection{Supremum Case: Proof + Dissection}



\paragraph{Quantifier Dissection (Supremum engine).}
Let $\alpha := \sup S$. The core logical clause used to \emph{build} a sequence is
\[
(\forall \varepsilon>0)(\exists s\in S)\ \big(\alpha-\varepsilon<s\le \alpha\big).
\]
Choosing a \emph{specific} $\varepsilon$-schedule, e.g.\ $\varepsilon_n = 1/n$, yields
\[
(\forall n\in\mathbb{N})(\exists x_n\in S)\ \Big(\alpha-\frac1n<x_n\le \alpha\Big),
\]
and monotonicity is forced by the definitional operator
\[
X_n := \max\{x_1,\dots,x_n\}.
\]

\paragraph{Logical Skeleton (Supremum case).}
\[
\Big(S\neq\varnothing \ \land\ S \text{ bounded above}\Big)
\Rightarrow
\exists \alpha\in\mathbb{R}\ (\alpha=\sup S)
\Rightarrow
\exists (X_n)\subset S\ \Big(X_n\uparrow \alpha\Big).
\]

\paragraph{Logical Skeleton (Pure quantifier form; supremum part).}
\[
\forall S\subset\mathbb{R}\Big[
\big((\exists s_0\in S)\land(\exists M)(\forall s\in S)(s\le M)\big)
\rightarrow
\exists (X_n)_{n\in\mathbb{N}}
\]
\[
\Big(
(\forall n)(X_n\in S)\land(\forall n)(X_n\le X_{n+1})
\land(\forall \varepsilon>0)(\exists N)(\forall n\ge N)(|X_n-\sup S|<\varepsilon)
\Big)\Big].
\]

\paragraph{Construction Mechanism (Supremum).}
\begin{itemize}
\item \textbf{Approximate from inside:} for each $n$, pick $x_n\in S$ with $\sup S-\frac1n < x_n\le \sup S$.
\item \textbf{Monotonize:} set $X_n := \max\{x_1,\dots,x_n\}$.
\item \textbf{Conclude:} $X_n\uparrow \sup S$.
\end{itemize}

% ---------------------------------------------------------
\subsubsection{Infimum Case: Symmetric Proof + Dissection}



\paragraph{Quantifier Dissection (Infimum engine).}
Let $\beta := \inf S$. The core clause is
\[
(\forall \varepsilon>0)(\exists s\in S)\ \big(\beta\le s<\beta+\varepsilon\big).
\]
Choosing $\varepsilon_n = 1/n$ yields
\[
(\forall n\in\mathbb{N})(\exists y_n\in S)\ \Big(\beta\le y_n<\beta+\frac1n\Big),
\]
and monotonicity is enforced by
\[
Y_n := \min\{y_1,\dots,y_n\}.
\]

\paragraph{Logical Skeleton (Infimum case).}
\[
\Big(S\neq\varnothing \ \land\ S \text{ bounded below}\Big)
\Rightarrow
\exists \beta\in\mathbb{R}\ (\beta=\inf S)
\Rightarrow
\exists (Y_n)\subset S\ \Big(Y_n\downarrow \beta\Big).
\]

\paragraph{Logical Skeleton (Pure quantifier form; infimum part).}
\[
\forall S\subset\mathbb{R}\Big[
\big((\exists s_0\in S)\land(\exists m)(\forall s\in S)(m\le s)\big)
\rightarrow
\exists (Y_n)_{n\in\mathbb{N}}
\]
\[
\Big(
(\forall n)(Y_n\in S)\land(\forall n)(Y_{n+1}\le Y_n)
\land(\forall \varepsilon>0)(\exists N)(\forall n\ge N)(|Y_n-\inf S|<\varepsilon)
\Big)\Big].
\]

\paragraph{Construction Mechanism (Infimum).}
\begin{itemize}
\item \textbf{Approximate from inside:} for each $n$, pick $y_n\in S$ with $\inf S\le y_n< \inf S+\frac1n$.
\item \textbf{Monotonize:} set $Y_n := \min\{y_1,\dots,y_n\}$.
\item \textbf{Conclude:} $Y_n\downarrow \inf S$.
\end{itemize}

% ---------------------------------------------------------
\subsubsection{Least Upper Bound Property \texorpdfstring{$\Longleftrightarrow$}{<->} Monotone Convergence}

\begin{tcolorbox}[colback=propbox, colframe=propborder, arc=2pt,
  left=6pt, right=6pt, top=4pt, bottom=4pt,
  title={\small\textbf{Definition (Least Upper Bound Property (LUB))}},
  fonttitle=\small\bfseries]
\label{def:LUB}
We say $\mathbb{R}$ has the \emph{least upper bound property} if every nonempty set
$A\subset\mathbb{R}$ that is bounded above has a supremum in $\mathbb{R}$.
\end{tcolorbox}

\begin{theorem}[Monotone Convergence Theorem (MCT)]
\label{thm:MCT}
Every bounded monotone sequence of real numbers converges.
\end{theorem}

\begin{theorem}[Equivalence: LUB $\Longleftrightarrow$ MCT]
\label{thm:LUB-iff-MCT}
The following are equivalent:
\begin{enumerate}
\item $\mathbb{R}$ has the least upper bound property.
\item Every bounded monotone sequence of real numbers converges.
\end{enumerate}
\end{theorem}

% ---------------------------------------------------------
\subsubsection{(LUB $\Rightarrow$ MCT): Proof + Dissection}



\paragraph{Quantifier Dissection (LUB $\Rightarrow$ MCT).}
The convergence proof is the same quantifier engine as supremum approximation:
\[
\alpha=\sup A
\ \Rightarrow\
(\forall \varepsilon>0)\ \exists N\ (x_N>\alpha-\varepsilon),
\]
and monotonicity upgrades $\exists N$ to a tail statement:
\[
(\forall \varepsilon>0)(\exists N)(\forall n\ge N)\ (\alpha-\varepsilon<x_n\le \alpha).
\]

\paragraph{Logical Skeleton (LUB $\Rightarrow$ MCT).}
\[
\text{bounded increasing }(x_n)
\Rightarrow
A=\{x_n\}\text{ bounded above}
\Rightarrow
\exists \alpha=\sup A
\Rightarrow
x_n\to \alpha.
\]

\paragraph{Logical Skeleton (Pure quantifier form; LUB $\Rightarrow$ MCT).}
\[
\Big(
\forall A\subset\mathbb{R}\big((\exists a_0\in A)\land(\exists M)(\forall a\in A)(a\le M)\Rightarrow \exists \sup A\big)
\Big)
\Rightarrow
\]
\[
\forall (x_n)\Big(
\big((\forall n)(x_n\le x_{n+1})\land(\exists M)(\forall n)(x_n\le M)\big)
\Rightarrow
\exists L\ (\forall \varepsilon>0)(\exists N)(\forall n\ge N)(|x_n-L|<\varepsilon)
\Big).
\]

\paragraph{Construction Mechanism (LUB $\Rightarrow$ MCT).}
\begin{itemize}
\item Form the image set $A=\{x_n\}$.
\item Use LUB to produce $\alpha=\sup A$.
\item Use the $\varepsilon$-witness property of $\sup$ to get one index $N$ with $x_N>\alpha-\varepsilon$.
\item Use monotonicity to push that inequality to all $n\ge N$.
\end{itemize}

% ---------------------------------------------------------
\subsubsection{(MCT $\Rightarrow$ LUB): Proof + Dissection (Bisection Construction)}



\paragraph{Quantifier Dissection (MCT $\Rightarrow$ LUB).}
The bisection construction is a \emph{quantifier-control machine}:
\begin{itemize}
\item Maintain a \emph{universal} invariant: $(\forall a\in A)(a\le u_n)$ (upper-bound witness).
\item Maintain a \emph{failure} invariant: $l_n$ is not an upper bound, i.e.\
$(\exists a\in A)(a>l_n)$ (inside-point witness).
\item Force a shrinking bracket: $u_n-l_n\to 0$ (metric control).
\item Use MCT to produce limits of monotone bounded sequences and collapse the bracket.
\item Convert ``$l_N$ is not an upper bound'' into the required $\varepsilon$-witness for $\sup$:
\[
(\forall \varepsilon>0)(\exists a\in A)(s-\varepsilon<a).
\]
\end{itemize}

\paragraph{Logical Skeleton (MCT $\Rightarrow$ LUB).}
\[
A\neq\varnothing,\ A\text{ bounded above}
\Rightarrow
\exists (l_n)\uparrow,\ \exists (u_n)\downarrow,\ u_n-l_n\to 0
\Rightarrow
l_n\to s,\ u_n\to s
\Rightarrow
s \text{ is an upper bound and is least}
\Rightarrow
s=\sup A.
\]

\paragraph{Logical Skeleton (Pure quantifier form; MCT $\Rightarrow$ LUB).}
\[
\Big(
\forall (x_n)\big(
((\forall n)(x_n\le x_{n+1})\land(\exists M)(\forall n)(x_n\le M))
\Rightarrow
\exists L(\forall \varepsilon>0)(\exists N)(\forall n\ge N)(|x_n-L|<\varepsilon)
\big)
\Big)
\Rightarrow
\]
\[
\forall A\subset\mathbb{R}\Big(
((\exists a_0\in A)\land(\exists U)(\forall a\in A)(a\le U))
\Rightarrow
\exists s\in\mathbb{R}\ \big(s=\sup A\big)
\Big).
\]

\paragraph{Construction Mechanism (MCT $\Rightarrow$ LUB).}
\begin{itemize}
\item Pick $l_1\in A$ and an upper bound $u_1$.
\item Repeatedly bisect: $m_n=(l_n+u_n)/2$.
\item If $m_n$ is an upper bound, tighten from above: $u_{n+1}=m_n$.
\item If not, tighten from below: $l_{n+1}=m_n$.
\item The bracket width shrinks geometrically, and MCT provides limits.
\item Use invariants to prove the limit is the least upper bound.
\end{itemize}

% ---------------------------------------------------------
\subsubsection{Logical Implications for the Journey}

\begin{remark}[Why this matters structurally]
The chain of ideas in this subsection is a reusable meta-template:
\[
\text{(Order axiom / extremal property)}
\Longleftrightarrow
\text{(Monotone sequential convergence)}.
\]
It explains why so much of early analysis can be done with monotone sequences:
they are a \emph{computational interface} to completeness.
\end{remark}

\begin{remark}[Downstream reuse]
The quantifier pattern
\[
(\forall \varepsilon>0)(\exists \text{witness})
\Rightarrow
(\exists \text{sequence with controlled tail})
\]
reappears in:
\begin{itemize}
\item Bolzano--Weierstrass (extracting convergent subsequences),
\item $\limsup/\liminf$ (tail suprema as a decreasing sequence),
\item compactness (finite subcovers as ``finite witnesses''),
\item later approximation theorems (simple functions, step functions, etc.).
\end{itemize}
\end{remark}