% =========================================================
% Subsequences
% File: notes-subsequences.tex
% =========================================================

\subsection{Subsequences}

% ---------------------------------------------------------
% Toolkit
% ---------------------------------------------------------
\begin{tcolorbox}[colback=gray!6, colframe=gray!40, arc=2pt,
  left=6pt, right=6pt, top=4pt, bottom=4pt,
  title={\small\textbf{Subsequences — Quick Reference}},
  fonttitle=\small\bfseries]
\begin{tabular}{@{}p{0.28\textwidth}p{0.68\textwidth}@{}}
\textbf{Core items} & Key definitions/results introduced in this file.\\
\textbf{How to use} & Read the boxed items first; proofs and consequences follow.\\
\textbf{Dependencies} & Refer back to earlier sections as needed.\\
\end{tabular}
\end{tcolorbox}


% ---------------------------------------------------------
\subsubsection{Basic Definitions}

\begin{tcolorbox}[colback=propbox, colframe=propborder, arc=2pt,
  left=6pt, right=6pt, top=4pt, bottom=4pt,
  title={\small\textbf{Definition (Subsequence)}},
  fonttitle=\small\bfseries]
Let $(a_n)$ be a sequence in $\mathbb{R}$.
A sequence $(a_{n_k})$ is called a \emph{subsequence} of $(a_n)$ if
there exists a function
\[
k \mapsto n_k
\]
from $\mathbb{N}$ to $\mathbb{N}$ such that
\begin{enumerate}
    \item $n_k < n_{k+1}$ for all $k \in \mathbb{N}$, and
    \item the $k$th term of the new sequence is $a_{n_k}$.
\end{enumerate}
\end{tcolorbox}

\begin{remark}
A subsequence is obtained by selecting terms from $(a_n)$ without duplication
while preserving their original order. The index sequence $(n_k)$ determines
which terms are selected.
\end{remark}

\begin{lemma}[Index Growth]
\label{lem:index-growth}
If $(n_k)$ is a strictly increasing sequence in $\mathbb{N}$, then $n_k \geq k$ for all $k \in \mathbb{N}$.
In particular, $n_k \to \infty$ as $k \to \infty$.
\end{lemma}

\begin{remark}[Proof]
See \hyperref[prf:index-growth]{Proof $\to$ RA-ABB-L-IDX-GROWTH (Index Growth Lemma)}.
\end{remark}

\begin{proof}
We proceed by induction on $k$.

For $k = 1$: since $n_1 \in \mathbb{N}$, we have $n_1 \geq 1$.

Assume $n_k \geq k$. Since $(n_k)$ is strictly increasing,
\[
n_{k+1} > n_k \geq k.
\]
Since $n_{k+1} \in \mathbb{N}$ and $n_{k+1} > k$, we have $n_{k+1} \geq k + 1$.

By induction, $n_k \geq k$ for all $k$. Since $k \to \infty$, it follows that $n_k \to \infty$.
\end{proof}

\begin{tcolorbox}[colback=propbox, colframe=propborder, arc=2pt,
  left=6pt, right=6pt, top=4pt, bottom=4pt,
  title={\small\textbf{Definition (Subsequential Limit)}},
  fonttitle=\small\bfseries]
Let $(a_n)$ be a sequence. A real number $L$ is called a \emph{subsequential limit}
of $(a_n)$ if there exists a subsequence $(a_{n_k})$ such that $a_{n_k} \to L$.

The set of all subsequential limits of $(a_n)$ is denoted $\mathcal{L}(a_n)$.
\end{tcolorbox}

% ---------------------------------------------------------
\subsubsection{Main Theorems}

\begin{theorem}[Subsequences Inherit Limits]
Let $(a_n)$ be a sequence of real numbers and let $(a_{n_k})$ be any subsequence.
If $a_n \to L$, then $a_{n_k} \to L$.
\end{theorem}

\begin{proof}
Assume $a_n \to L$. Let $\varepsilon > 0$ be arbitrary.
By definition of convergence, there exists $N \in \mathbb{N}$ such that
\[
n \ge N \implies |a_n - L| < \varepsilon.
\]

By the Index Growth Lemma, $n_k \to \infty$, so there exists $K \in \mathbb{N}$ such that
\[
k \ge K \implies n_k \ge N.
\]

Thus for all $k \ge K$,
\[
|a_{n_k} - L| < \varepsilon.
\]

Hence $a_{n_k} \to L$.
\end{proof}

\begin{remark}[Consistency of Subsequential Limits]
If $(a_n) \to L$, then every subsequence converges to $L$.
Consequently, a convergent sequence has exactly one subsequential limit:
\[
a_n \to L \implies \mathcal{L}(a_n) = \{L\}.
\]
\end{remark}

\begin{remark}[Detecting Divergence]
Contrapositive: if two subsequences of $(a_n)$ converge to different limits,
then $(a_n)$ does not converge. This provides a practical test for divergence.
\end{remark}

\begin{theorem}[Convergence via Even and Odd Subsequences]
Let $(a_n)$ be a sequence. If the even-indexed subsequence $(a_{2n})$ and the
odd-indexed subsequence $(a_{2n+1})$ both converge to the same limit $L$,
then $(a_n) \to L$.
\end{theorem}

\begin{proof}
Assume $a_{2n} \to L$ and $a_{2n+1} \to L$.

Let $\varepsilon > 0$ be given.

Since $a_{2n} \to L$, there exists $N_1 \in \mathbb{N}$ such that for all $n \geq N_1$,
\[
|a_{2n} - L| < \varepsilon.
\]

Since $a_{2n+1} \to L$, there exists $N_2 \in \mathbb{N}$ such that for all $n \geq N_2$,
\[
|a_{2n+1} - L| < \varepsilon.
\]

Let $N = \max\{2N_1, 2N_2 + 1\}$.

Let $n \geq N$ be arbitrary. Either $n$ is even or $n$ is odd.

\textbf{Case 1:} $n$ is even. Then $n = 2k$ for some $k \in \mathbb{N}$.
Since $n \geq N \geq 2N_1$, we have $k \geq N_1$, so
\[
|a_n - L| = |a_{2k} - L| < \varepsilon.
\]

\textbf{Case 2:} $n$ is odd. Then $n = 2k + 1$ for some $k \in \mathbb{N}$.
Since $n \geq N \geq 2N_2 + 1$, we have $k \geq N_2$, so
\[
|a_n - L| = |a_{2k+1} - L| < \varepsilon.
\]

In both cases, $|a_n - L| < \varepsilon$. Since $\varepsilon > 0$ was arbitrary, $a_n \to L$.
\end{proof}

\begin{remark}
More generally, if $(a_n)$ can be partitioned into finitely many subsequences,
each converging to the same limit $L$, then $a_n \to L$.
\end{remark}

% ---------------------------------------------------------

\begin{theorem}[Bolzano--Weierstrass]
\label{thm:bolzano-weierstrass}
Every bounded sequence of real numbers has a convergent subsequence.
Equivalently: if $(a_n)$ is bounded in $\mathbb{R}$, then there exist
$L \in \mathbb{R}$ and a strictly increasing sequence $(n_k)$ in $\mathbb{N}$
such that $a_{n_k} \to L$.
\end{theorem}

\begin{remark}[Proof]
See \hyperref[prf:bolzano-weierstrass]{Proof $\to$ RA-ABB-T-BW (Bolzano--Weierstrass Theorem)}.
\end{remark}

\begin{proof}
Let $(a_n)$ be bounded. Then there exist real numbers $m < M$ such that
$a_n \in [m, M]$ for all $n \in \mathbb{N}$.

We construct nested closed intervals
\[
I_k = [\alpha_k, \beta_k] \subseteq [m, M]
\]
satisfying:
\begin{enumerate}
\item $I_{k+1} \subseteq I_k$,
\item $\beta_k - \alpha_k = \dfrac{M - m}{2^{k-1}}$,
\item infinitely many terms of $(a_n)$ lie in each $I_k$.
\end{enumerate}

\textbf{Construction.} Start with $I_1 = [m, M]$.
Given $I_k = [\alpha_k, \beta_k]$, let $c_k = \dfrac{\alpha_k + \beta_k}{2}$ and consider
\[
[\alpha_k, c_k] \quad \text{and} \quad [c_k, \beta_k].
\]
At least one half contains infinitely many terms of $(a_n)$.
Define $I_{k+1}$ to be such a half.

\textbf{Intersection.} The intervals $(I_k)$ are nested with lengths
$\beta_k - \alpha_k = \frac{M - m}{2^{k-1}} \to 0$.
By the Nested Interval Property, $\bigcap_{k=1}^\infty I_k$ contains exactly one point.
Let
\[
\{L\} = \bigcap_{k=1}^\infty I_k.
\]

\textbf{Subsequence construction.} We construct indices $n_1 < n_2 < \cdots$
with $a_{n_k} \in I_k$ inductively. Since each $I_k$ contains infinitely many
terms and $I_{k+1} \subseteq I_k$, we can always choose $n_{k+1} > n_k$
with $a_{n_{k+1}} \in I_{k+1}$.

\textbf{Convergence.} Let $\varepsilon > 0$.
Choose $K$ such that $\beta_K - \alpha_K < \varepsilon$.
For $k \geq K$, both $a_{n_k}$ and $L$ lie in $I_K$, so
\[
|a_{n_k} - L| \leq \beta_K - \alpha_K < \varepsilon.
\]

Thus $a_{n_k} \to L$.
\end{proof}

\begin{corollary}[Existence of Subsequential Limits]
Every bounded sequence admits at least one subsequential limit.
That is, if $(a_n)$ is bounded, then $\mathcal{L}(a_n) \neq \varnothing$.
\end{corollary}

\begin{corollary}[Sequential Compactness of Closed Intervals]
Every sequence contained in a closed bounded interval
$[a, b]$ has a convergent subsequence whose limit lies in $[a, b]$.
\end{corollary}

\begin{proof}
If $(a_n) \subseteq [a, b]$, then $(a_n)$ is bounded.
By Bolzano--Weierstrass, it has a convergent subsequence $(a_{n_k}) \to L$.
Since $a_{n_k} \in [a, b]$ for all $k$ and $[a, b]$ is closed,
the limit $L$ lies in $[a, b]$.
\end{proof}

% ---------------------------------------------------------

\begin{theorem}[Monotone Subsequence Theorem]
Every sequence in $\mathbb{R}$ has a monotone subsequence.
\end{theorem}

\begin{proof}
Let $(a_n)$ be a sequence. Call an index $n$ a \emph{peak} if
$a_n \geq a_m$ for all $m > n$.

\textbf{Case 1:} There are infinitely many peaks.

Let $n_1 < n_2 < n_3 < \cdots$ be the peaks in increasing order.
Since $n_1$ is a peak and $n_2 > n_1$, we have $a_{n_1} \geq a_{n_2}$.
Similarly, $a_{n_2} \geq a_{n_3}$, and so on.
Thus $(a_{n_k})$ is a decreasing subsequence.

\textbf{Case 2:} There are only finitely many peaks.

Let $N$ be larger than all peak indices (or $N = 1$ if there are no peaks).
Then no index $n \geq N$ is a peak.

We construct an increasing subsequence starting from $n_1 = N$.
Since $n_1$ is not a peak, there exists $n_2 > n_1$ with $a_{n_2} > a_{n_1}$.
Since $n_2$ is not a peak, there exists $n_3 > n_2$ with $a_{n_3} > a_{n_2}$.
Continuing inductively, we obtain a strictly increasing subsequence $(a_{n_k})$.

In either case, $(a_n)$ has a monotone subsequence.
\end{proof}

\begin{remark}[Alternative Proof of Bolzano--Weierstrass]
The Monotone Subsequence Theorem combined with the Monotone Convergence Theorem
provides an alternative proof of Bolzano--Weierstrass:
\begin{enumerate}
\item Every sequence has a monotone subsequence (Monotone Subsequence Theorem).
\item A bounded monotone sequence converges (Monotone Convergence Theorem).
\item Therefore, every bounded sequence has a convergent subsequence.
\end{enumerate}
\end{remark}

% ---------------------------------------------------------
\subsubsection{Divergence Criteria}

\begin{theorem}[Characterization of Divergence]
A sequence $(a_n)$ diverges if and only if at least one of the following holds:
\begin{enumerate}
\item $(a_n)$ is unbounded.
\item $(a_n)$ has two subsequences converging to different finite limits.
\item $(a_n)$ has a subsequence diverging to $+\infty$ or $-\infty$.
\end{enumerate}
\end{theorem}

\begin{proof}
$(\Leftarrow)$ Each condition implies divergence:
\begin{enumerate}
\item Convergent sequences are bounded, so unbounded sequences diverge.
\item If $a_n \to L$, all subsequences converge to $L$. Two different
      subsequential limits contradict convergence.
\item If $a_n \to L \in \mathbb{R}$, all subsequences converge to $L \in \mathbb{R}$,
      so no subsequence can diverge to $\pm\infty$.
\end{enumerate}

$(\Rightarrow)$ Suppose $(a_n)$ diverges.

If $(a_n)$ is unbounded, condition (1) holds.

If $(a_n)$ is bounded, then by Bolzano--Weierstrass, it has a convergent
subsequence $(a_{n_k}) \to L$. Since $(a_n)$ does not converge to $L$,
there exists $\varepsilon_0 > 0$ and infinitely many indices $m$ with
$|a_m - L| \geq \varepsilon_0$. These indices form a bounded subsequence,
which by Bolzano--Weierstrass has a further convergent subsequence
$(a_{m_j}) \to L'$. Since $|a_{m_j} - L| \geq \varepsilon_0$ for all $j$,
we have $|L' - L| \geq \varepsilon_0$, so $L' \neq L$.
Thus condition (2) holds.
\end{proof}

% ---------------------------------------------------------
\subsubsection{Consequences and Structural Summary}

This section establishes three structural principles.

\medskip

\textbf{(1) Inheritance of Limits.}
\[
a_n \to L \implies a_{n_k} \to L.
\]
Convergence is preserved under passage to subsequences.

\medskip

\textbf{(2) Reconstruction from Subsequences.}
\[
(a_{2n} \to L) \land (a_{2n+1} \to L) \implies a_n \to L.
\]
Convergence can be established by verifying it on a finite partition.

\medskip

\textbf{(3) Emergence of Subsequential Limits.}
\[
\text{Bounded} \implies \text{Convergent subsequence exists}.
\]
Boundedness alone does not ensure convergence,
but it guarantees the existence of convergent subsequences.

\medskip

\begin{remark}[Connection to Limit Superior and Inferior]
For a bounded sequence $(a_n)$, the set $\mathcal{L}(a_n)$ of subsequential
limits is nonempty, closed, and bounded. Moreover,
\[
\limsup_{n \to \infty} a_n = \sup \mathcal{L}(a_n) = \max \mathcal{L}(a_n),
\]
\[
\liminf_{n \to \infty} a_n = \inf \mathcal{L}(a_n) = \min \mathcal{L}(a_n).
\]
Thus $\limsup$ and $\liminf$ are themselves subsequential limits,
and they are the largest and smallest such limits.
\end{remark}

\begin{remark}[Logical Structure]
\[
\text{Convergent}
\implies
\text{All subsequences converge to the same limit},
\]
\[
\text{Bounded}
\xRightarrow{\text{Bolzano--Weierstrass}}
\text{Existence of subsequential limits}.
\]

These results form the bridge between basic convergence
theory and completeness theory.

Bolzano--Weierstrass is equivalent (in an ordered field) to:
\begin{itemize}
\item The Monotone Convergence Theorem,
\item The Cauchy Criterion,
\item The Nested Interval Property,
\item The Least Upper Bound Property.
\end{itemize}
\end{remark}

% =========================================================
% Structural Transfer of Sequence Properties
% =========================================================

\begin{remark}[Inheritance and Reflection of Properties]
Different sequence properties behave differently with respect to subsequences.
\end{remark}

\begin{center}
\begin{tabular}{|l|c|c|l|}
\hline
\textbf{Property} 
& \makecell{\textbf{Inherited by} \\ \textbf{Subsequences?}} 
& \makecell{\textbf{Reflected by One} \\ \textbf{Subsequence?}}
& \textbf{Structural Type} \\
\hline
Convergent 
& Yes 
& No 
& Tail Property \\
\hline
Cauchy 
& Yes 
& No 
& Tail Property \\
\hline
Bounded 
& Yes 
& No 
& Global Property \\
\hline
Monotone 
& Yes 
& No 
& Global Structural \\
\hline
Eventually monotone 
& Yes 
& No 
& Tail Property \\
\hline
Every subseq.\ convergent 
& --- 
& Yes 
& Universal Property \\
\hline
Every subseq.\ Cauchy 
& --- 
& Yes 
& Universal Property \\
\hline
\end{tabular}
\end{center}

\begin{remark}
\emph{Inherited} means: if $(a_n)$ has property $P$, then every subsequence has $P$.

\emph{Reflected} means: if some subsequence has property $P$, then $(a_n)$ has $P$.

Most natural properties are inherited but not reflected. The sequence
$a_n = (-1)^n$ is bounded and has convergent subsequences, but does not converge.
\end{remark}