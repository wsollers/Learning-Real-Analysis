% =========================================================
% Monotone Sequences and Monotone Convergence
% =========================================================

\subsection{Monotone Sequences and Monotone Convergence}

% ---------------------------------------------------------
% Toolkit
% ---------------------------------------------------------
\begin{tcolorbox}[title=Toolkit: Monotone Sequences]
\begin{tabular}{@{}p{0.28\textwidth}p{0.68\textwidth}@{}}
\textbf{Core items} & Key definitions/results introduced in this file.\\
\textbf{How to use} & Read the boxed items first; proofs and consequences follow.\\
\textbf{Dependencies} & Refer back to earlier sections as needed.\\
\end{tabular}
\end{tcolorbox}


% =========================================================
\subsubsection{Basic Definitions}
% =========================================================

\begin{tcolorbox}[title=Definition (Increasing and decreasing sequences)]
A sequence $(x_n)$ is called

\begin{itemize}
\item \emph{increasing} if $x_n \le x_{n+1}$ for all $n \in \mathbb{N}$,
\item \emph{strictly increasing} if $x_n < x_{n+1}$ for all $n$,
\item \emph{decreasing} if $x_{n+1} \le x_n$ for all $n$,
\item \emph{strictly decreasing} if $x_{n+1} < x_n$ for all $n$.
\end{itemize}
\end{tcolorbox}

\begin{tcolorbox}[title=Definition (Monotone sequence)]
A sequence is \emph{monotone} if it is either increasing or decreasing.
\end{tcolorbox}

% =========================================================
\subsubsection{Main Theorem}
% =========================================================

\begin{theorem}[Monotone Convergence Theorem]
Every bounded monotone sequence of real numbers converges.

More precisely:

\begin{enumerate}
\item If $(x_n)$ is increasing and bounded above, then
\[
\lim_{n\to\infty} x_n = \sup\{x_n : n\in\mathbb{N}\}.
\]

\item If $(x_n)$ is decreasing and bounded below, then
\[
\lim_{n\to\infty} x_n = \inf\{x_n : n\in\mathbb{N}\}.
\]
\end{enumerate}
\end{theorem}

\begin{proof}

(1) Suppose $(x_n)$ is increasing and bounded above.
Let
\[
s := \sup\{x_n : n\in\mathbb{N}\}.
\]

Fix $\varepsilon > 0$.
Then $s - \varepsilon$ is not an upper bound,
so there exists $N$ such that
\[
x_N > s - \varepsilon.
\]

Since $(x_n)$ is increasing,
for all $n \ge N$,
\[
x_n \ge x_N > s - \varepsilon.
\]

Also, by definition of supremum,
\[
x_n \le s \quad \text{for all } n.
\]

Thus for $n \ge N$,
\[
s - \varepsilon < x_n \le s,
\]
so
\[
|x_n - s| < \varepsilon.
\]

Hence $x_n \to s$.

\medskip

(2) Suppose $(x_n)$ is decreasing and bounded below.
Let
\[
i := \inf\{x_n : n\in\mathbb{N}\}.
\]

Fix $\varepsilon > 0$.
Then $i + \varepsilon$ is not a lower bound,
so there exists $N$ such that
\[
x_N < i + \varepsilon.
\]

Since $(x_n)$ is decreasing,
for all $n \ge N$,
\[
x_n \le x_N < i + \varepsilon.
\]

Also $i \le x_n$ for all $n$.

Thus
\[
i \le x_n < i + \varepsilon
\]
for $n \ge N$,
so $x_n \to i$.

\end{proof}

% =========================================================
\subsubsection{Consequences}
% =========================================================

The logical implication of this entire section is:

\[
\text{Monotone} + \text{Bounded}
\;\Rightarrow\;
\text{Convergent}.
\]

More precisely,

\[
\text{Increasing and bounded above}
\Rightarrow
\lim x_n = \sup\{x_n\},
\]
\[
\text{Decreasing and bounded below}
\Rightarrow
\lim x_n = \inf\{x_n\}.
\]

\begin{remark}[Extended real limits]
If $(x_n)$ is increasing but not bounded above, then
\[
x_n \to +\infty.
\]

If $(x_n)$ is decreasing but not bounded below, then
\[
x_n \to -\infty.
\]

Thus every monotone sequence has a limit in
\[
\mathbb{R} \cup \{\pm\infty\}.
\]
\end{remark}

\begin{remark}[Logical structure]
\[
\text{Increasing/Decreasing}
\Rightarrow
\text{Monotone}
\Rightarrow
\text{Monotone Convergence}.
\]

The Monotone Convergence Theorem relies on:

\begin{itemize}
\item Order structure,
\item Boundedness,
\item The Least Upper Bound Property (Completeness).
\end{itemize}

In fact, it is equivalent (in the presence of ordered field axioms) to:

\begin{itemize}
\item Bolzano--Weierstrass,
\item The Cauchy Criterion,
\item The Nested Interval Property.
\end{itemize}
\end{remark}
