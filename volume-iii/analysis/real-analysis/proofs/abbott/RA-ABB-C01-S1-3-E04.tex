% --------------------------------------
% Proof: RA-ABB-C01-S1-3-E04
% --------------------------------------
\clearpage
\phantomsection
\hypertarget{proof-RA-ABB-C01-S1-3-E04}{}

\section*{Proof — RA-ABB-C01-S1-3-E04}

\noindent
\hyperlink{ws-RA-ABB-C01-S1-3-E04}{\textbf{← Back to worksheet}}

\noindent
\textbf{Source.}
\srccite{AbbottUnderstandingAnalysis}{Chapter~1, \S1-3, Exercise~1.3.4}.

\vspace{0.75em}

\noindent
\textbf{Goal.}
Let $A_1,A_2,A_3,\dots$ be a collection of nonempty sets, each of which is bounded above.
\begin{enumerate}[label=(\alph*)]
\item Find a formula for $\sup(A_1\cup A_2)$, and extend to $\sup\!\Big(\bigcup_{k=1}^n A_k\Big)$.
\item Consider $\sup\!\Big(\bigcup_{k=1}^\infty A_k\Big)$. Does the formula in (a) extend to the infinite case?
\end{enumerate}

\vspace{0.75em}

\noindent
\textbf{Proof.}



\begin{enumerate}[label=(\alph*)]

\item
Let $A_1,A_2 \subseteq \mathbb{R}$ be nonempty and bounded above.
Define
\[
s_1 := \sup A_1, \qquad s_2 := \sup A_2,
\]
and set
\[
m := \max\{s_1,s_2\}.
\]

We show that $m = \sup(A_1 \cup A_2)$.

First, we verify that $m$ is an upper bound of $A_1 \cup A_2$.
Let $x \in A_1 \cup A_2$. Then either $x \in A_1$ or $x \in A_2$.
If $x \in A_1$, then $x \le s_1 \le m$.
If $x \in A_2$, then $x \le s_2 \le m$.
Hence $x \le m$ in all cases, and $m$ is an upper bound of $A_1 \cup A_2$.

Next, let $u$ be any upper bound of $A_1 \cup A_2$.
Because $A_1 \subseteq A_1 \cup A_2$ and $A_2 \subseteq A_1 \cup A_2$,
the number $u$ is an upper bound of both $A_1$ and $A_2$.
Therefore,
\[
s_1 \le u \quad \text{and} \quad s_2 \le u,
\]
which implies $m = \max\{s_1,s_2\} \le u$.

Thus $m$ is the least upper bound of $A_1 \cup A_2$, and
\[
\sup(A_1 \cup A_2) = \max\{\sup A_1,\sup A_2\}.
\]

The same argument extends to finitely many sets.
If $A_1,\dots,A_n$ are nonempty and bounded above, then
\[
\sup\!\Big(\bigcup_{k=1}^n A_k\Big)
= \max\{\sup A_1,\dots,\sup A_n\}.
\]

\item
We now consider the infinite union $\bigcup_{k=1}^\infty A_k$.

In general, the formula in part (a) does \emph{not} extend to the infinite case.
The set $\{\sup A_k : k \in \mathbb{N}\}$ need not have a maximum, and the union
$\bigcup_{k=1}^\infty A_k$ may fail to be bounded above.

However, if $\bigcup_{k=1}^\infty A_k$ \emph{is} bounded above, then the correct
relationship is
\[
\sup\!\Big(\bigcup_{k=1}^\infty A_k\Big)
= \sup_{k \in \mathbb{N}} (\sup A_k).
\]

To justify this, let $s_k := \sup A_k$ and define
\[
S := \sup_{k \in \mathbb{N}} s_k.
\]

First, we show that $S$ is an upper bound of $\bigcup_{k=1}^\infty A_k$.
If $x \in \bigcup_{k=1}^\infty A_k$, then $x \in A_j$ for some $j$.
Thus $x \le s_j \le S$, so $x \le S$.

Next, let $u$ be any upper bound of $\bigcup_{k=1}^\infty A_k$.
Then $u$ is an upper bound of each $A_k$, so $s_k \le u$ for all $k$.
Hence $u$ is an upper bound of the set $\{s_k\}$, which implies $S \le u$.

Therefore, $S$ is the least upper bound of $\bigcup_{k=1}^\infty A_k$, and
\[
\sup\!\Big(\bigcup_{k=1}^\infty A_k\Big)
= \sup_{k \in \mathbb{N}} (\sup A_k).
\]

\end{enumerate}



\vspace{1em}
\noindent
\hyperlink{ws-RA-ABB-C01-S1-3-E04}{\textbf{← Back to worksheet}}

