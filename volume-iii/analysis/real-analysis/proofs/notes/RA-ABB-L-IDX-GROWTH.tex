% --------------------------------------
% Proof: RA-ABB-L-IDX-GROWTH
% Lemma — Index Growth: n_k >= k for strictly increasing (n_k)
% Format: Sketch (compact)
% --------------------------------------
\clearpage
\phantomsection
\label{prf:index-growth}
\hypertarget{proof-RA-ABB-L-IDX-GROWTH}{}

\subsubsection[Index Growth Lemma]{Proof --- RA-ABB-L-IDX-GROWTH}

\begin{remark}[Return]
\hyperref[lem:index-growth]{$\leftarrow$ Back to Lemma (Index Growth) in Notes}
\end{remark}

\bigskip

\begin{tcolorbox}[colback=gray!6, colframe=gray!40, arc=2pt,
  left=6pt, right=6pt, top=4pt, bottom=4pt,
  title={\small\textbf{Proof Sketch --- Index Growth Lemma}},
  fonttitle=\small\bfseries]

\textbf{Statement.}
If $(n_k)$ is a strictly increasing sequence in $\mathbb{N}$, then $n_k \ge k$ for all $k$.
In particular, $n_k \to \infty$.

\textbf{Key move.}
Induction on $k$. Base: $n_1 \ge 1$ since $n_1 \in \mathbb{N}$.
Step: assuming $n_k \ge k$, strict increase gives $n_{k+1} > n_k \ge k$,
so $n_{k+1} \ge k+1$ (since $n_{k+1} \in \mathbb{N}$).
Hence $n_k \ge k \to \infty$ as $k \to \infty$.

\textbf{Depends on.}
Peano structure of $\mathbb{N}$ (integers are discrete: $m > k$ and $m \in \mathbb{N}$ implies $m \ge k+1$).
\end{tcolorbox}

\vspace{0.75em}

\noindent
\textbf{Remarks.}
\begin{itemize}
  \item \textbf{Why this matters.} Every subsequence argument that claims
    ``$n_k \to \infty$'' implicitly uses this lemma.
    It is the formal justification for the informal statement that ``subsequence
    indices run off to infinity.''

  \item \textbf{Use in BW and Cauchy.} In the Cauchy Criterion proof (Step 3),
    the phrase ``pick $k_0$ with $n_{k_0} \ge N_1$'' is valid precisely because
    $n_k \to \infty$ (from this lemma), so eventually $n_k$ exceeds any given bound.
\end{itemize}
