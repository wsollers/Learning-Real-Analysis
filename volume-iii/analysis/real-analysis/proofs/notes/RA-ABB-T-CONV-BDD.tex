% --------------------------------------
% Proof: RA-ABB-T-CONV-BDD
% Theorem — Every convergent sequence is bounded
% Format: Deep study (3-column)
% --------------------------------------
\clearpage
\phantomsection
\label{prf:conv-bounded}
\hypertarget{proof-RA-ABB-T-CONV-BDD}{}

\subsubsection[Convergent Sequences are Bounded]{Proof --- RA-ABB-T-CONV-BDD}

\begin{remark}[Return]
\hyperref[thm:conv-bounded]{$\leftarrow$ Back to Theorem (Convergent $\Rightarrow$ Bounded) in Notes}
\end{remark}

\bigskip

\noindent
\textbf{Source.}
\srccite{AbbottUnderstandingAnalysis}{Chapter~2, \S2.3, Theorem~2.3.2}.

\vspace{0.75em}

\noindent
\textbf{Goal.}
If $(x_n)$ converges in $\mathbb{R}$, then $(x_n)$ is bounded.

\vspace{0.75em}

\noindent
\textbf{Logical form.}
$\bigl(\exists L \in \mathbb{R},\ x_n \to L\bigr) \Rightarrow \bigl(\exists M > 0,\ \forall n \in \mathbb{N},\ |x_n| \le M\bigr)$.

\vspace{0.75em}

\noindent
\textbf{Key background.}
$\varepsilon$-$N$ definition of convergence.
Triangle inequality: $|x_n| = |(x_n - L) + L| \le |x_n - L| + |L|$.

\vspace{0.75em}

\noindent
\textbf{Proof strategy.}
Fix $\varepsilon = 1$. The convergence definition gives $N$ such that all
terms for $n \ge N$ lie within $1$ of $L$, hence are bounded by $|L| + 1$.
The finite set of terms $\{x_1, \ldots, x_{N-1}\}$ has a finite maximum.
Take $M$ to be the larger of the two bounds.

\vspace{0.75em}

\noindent
\textbf{Proof.}
\begin{proof}
Let $(x_n) \to L$.

\begin{longtable}{p{0.28\textwidth} p{0.35\textwidth} p{0.28\textwidth}}
\toprule
\textbf{Step / Claim} & \textbf{Justification} & \textbf{Annotation} \\
\midrule
Apply the convergence definition with $\varepsilon := 1$.
& Definition of $x_n \to L$.
& Choose a concrete $\varepsilon$ to get a concrete $N$.
\\
\addlinespace
$\exists N \in \mathbb{N}$ such that $n \ge N \Rightarrow |x_n - L| < 1$.
& Convergence with $\varepsilon = 1$.
& All terms from index $N$ onward are within distance $1$ of $L$.
\\
\addlinespace
For $n \ge N$: $|x_n| \le |x_n - L| + |L| < 1 + |L|$.
& Triangle inequality.
& Bound on the tail of the sequence.
\\
\addlinespace
Let $M_{\text{tail}} := |L| + 1$.
& Explicit bound for $n \ge N$.
& Covers all tail terms.
\\
\addlinespace
Let $M_{\text{head}} := \max\{|x_1|, |x_2|, \ldots, |x_{N-1}|\}$.
& Finite maximum of a finite set.
& Covers all head terms (indices $1$ through $N-1$).
\\
\addlinespace
Set $M := \max(M_{\text{head}},\ M_{\text{tail}})$.
& Take the larger bound.
& This $M$ covers all $n$.
\\
\addlinespace
$\forall n \in \mathbb{N},\ |x_n| \le M$.
& By cases: $n < N$ uses $M_{\text{head}}$; $n \ge N$ uses $M_{\text{tail}}$.
& $(x_n)$ is bounded.
\\
\bottomrule
\end{longtable}
\end{proof}

\vspace{0.75em}

\noindent
\textbf{Remarks.}
\begin{itemize}
  \item \textbf{The $\varepsilon = 1$ trick.} Choosing $\varepsilon = 1$ is a
    standard tactic when you need a concrete bound rather than an arbitrarily
    small distance. It gives one explicit $N$ to split the sequence into
    head and tail.

  \item \textbf{Head-tail decomposition.} The proof structure --- bound the
    finite head separately from the infinite tail, then take the max --- is
    a recurring pattern. It appears again in: uniqueness of limits, algebra
    of limits, and any proof that needs to handle finitely many exceptional terms.

  \item \textbf{Converse is false.} Bounded does not imply convergent.
    The sequence $x_n = (-1)^n$ is bounded by $1$ but does not converge.
    Proving non-convergence requires the $\varepsilon/2$ argument: no single $L$
    can be within $1/2$ of both $1$ and $-1$.

  \item \textbf{Common error.} Treating $N$ as a bound on \emph{all} indices.
    $N$ is a threshold: only indices $n \ge N$ satisfy the tail bound.
    Indices $n < N$ need separate treatment (hence $M_{\text{head}}$).
\end{itemize}
