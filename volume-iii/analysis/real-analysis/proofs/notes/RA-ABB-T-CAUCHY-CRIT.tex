% --------------------------------------
% Proof: RA-ABB-T-CAUCHY-CRIT
% Theorem — Cauchy Criterion in R
% Format: Deep study (3-column)
% --------------------------------------
\clearpage
\phantomsection
\label{prf:cauchy-criterion}
\hypertarget{proof-RA-ABB-T-CAUCHY-CRIT}{}

\subsubsection[Cauchy Criterion for Sequences]{Proof --- RA-ABB-T-CAUCHY-CRIT}

\begin{remark}[Return]
\hyperref[thm:cauchy-criterion]{$\leftarrow$ Back to Theorem (Cauchy Criterion) in Notes}
\end{remark}

\bigskip

\noindent
\textbf{Source.}
\srccite{AbbottUnderstandingAnalysis}{Chapter~2, \S2.6, Theorem~2.6.4}.

\vspace{0.75em}

\noindent
\textbf{Goal.}
A sequence $(a_n)$ in $\mathbb{R}$ converges if and only if it is a Cauchy sequence.

\vspace{0.75em}

\noindent
\textbf{Logical form.}
$\bigl(\exists L,\ a_n \to L\bigr) \iff \bigl(\forall \varepsilon > 0,\ \exists N,\ m,n \ge N \Rightarrow |a_m - a_n| < \varepsilon\bigr)$.

\vspace{0.75em}

\noindent
\textbf{Key background.}
$(\Rightarrow)$: triangle inequality. $(\Leftarrow)$: Cauchy $\Rightarrow$ bounded (direct);
BW gives convergent subsequence; subsequence limit is the full sequence limit.

\vspace{0.75em}

\noindent
\textbf{Proof strategy.}
$(\Rightarrow)$ is a standard $\varepsilon/2$ argument.
$(\Leftarrow)$ is the hard direction: (1) show Cauchy sequences are bounded,
(2) apply BW to extract a convergent subsequence $(a_{n_k}) \to L$,
(3) show the full sequence converges to $L$ using the Cauchy condition.

\vspace{0.75em}

\noindent
\textbf{Proof.}
\begin{proof}
\begin{longtable}{p{0.28\textwidth} p{0.35\textwidth} p{0.28\textwidth}}
\toprule
\textbf{Step / Claim} & \textbf{Justification} & \textbf{Annotation} \\
\midrule
\multicolumn{3}{l}{\textbf{($\Rightarrow$)} Assume $a_n \to L$.} \\
\addlinespace
Let $\varepsilon > 0$. $\exists N$ with $n \ge N \Rightarrow |a_n - L| < \varepsilon/2$.
& Convergence definition.
& Standard: get $N$ from $\varepsilon/2$.
\\
For $m, n \ge N$: $|a_m - a_n| \le |a_m - L| + |L - a_n| < \varepsilon$.
& Triangle inequality.
& $(a_n)$ is Cauchy.
\\
\midrule
\multicolumn{3}{l}{\textbf{($\Leftarrow$)} Assume $(a_n)$ is Cauchy.} \\
\addlinespace
\textbf{Step 1.} $(a_n)$ is bounded.
& (Detailed argument): take $\varepsilon = 1$, get $N$; all tail terms within $1$ of $a_N$; head is finite. $M = \max(\text{head max},\ |a_N| + 1)$.
& Same head-tail decomposition as the convergent-implies-bounded proof.
\\
\addlinespace
\textbf{Step 2.} $\exists$ convergent subsequence $(a_{n_k}) \to L$.
& Bolzano--Weierstrass: $(a_n)$ is bounded, so it has a convergent subsequence.
& The subsequence limit $L$ is our candidate for the full sequence limit.
\\
\addlinespace
\textbf{Step 3.} $(a_n) \to L$.
& &
\\
Let $\varepsilon > 0$.
By Cauchy, $\exists N_1$ s.t.\ $m, n \ge N_1 \Rightarrow |a_m - a_n| < \varepsilon/2$.
& Cauchy condition with $\varepsilon/2$.
& Choose $N_1$ for the Cauchy condition.
\\
By $(a_{n_k}) \to L$: $\exists K$ s.t.\ $k \ge K \Rightarrow |a_{n_k} - L| < \varepsilon/2$.
& Convergence of subsequence with $\varepsilon/2$.
& Choose $K$ for the subsequence.
\\
Pick $k_0 \ge K$ with $n_{k_0} \ge N_1$.
& Such $k_0$ exists since $n_k \to \infty$.
& Get a subsequence index that is both far enough for BW and for Cauchy.
\\
For $n \ge N_1$:
$|a_n - L| \le |a_n - a_{n_{k_0}}| + |a_{n_{k_0}} - L| < \varepsilon/2 + \varepsilon/2 = \varepsilon$.
& Triangle inequality; both terms bounded by $\varepsilon/2$.
& Full sequence converges to $L$.
\\
\bottomrule
\end{longtable}
\end{proof}

\vspace{0.75em}

\noindent
\textbf{Remarks.}
\begin{itemize}
  \item \textbf{The BW step is the key.} The $(\Leftarrow)$ proof requires BW as
    a black box to extract a candidate limit. Without completeness (which underlies BW),
    Cauchy sequences need not converge (they do not converge in $\mathbb{Q}$,
    e.g.\ the decimal truncations of $\sqrt{2}$).

  \item \textbf{Step 3 structure.} The argument in Step 3 is a ``bridge'' argument:
    we cannot directly compare $a_n$ to $L$, so we go via a common subsequence term
    $a_{n_{k_0}}$ that is simultaneously close to $a_n$ (by Cauchy) and close to $L$
    (by subsequence convergence).

  \item \textbf{Why two separate $N$'s?} The Cauchy condition gives $N_1$; the subsequence
    convergence gives $K$. We need a single index that satisfies both, hence the
    ``pick $k_0 \ge K$ with $n_{k_0} \ge N_1$'' step.

  \item \textbf{Significance.} The Cauchy Criterion is the characterisation of
    convergence that does not mention the limit. It is essential in contexts
    where the limit is not known in advance (e.g.\ proving a series converges
    before computing its sum).
\end{itemize}
