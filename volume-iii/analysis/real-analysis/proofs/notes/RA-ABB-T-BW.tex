% --------------------------------------
% Proof: RA-ABB-T-BW
% Theorem — Bolzano--Weierstrass
% Format: Deep study (3-column)
% --------------------------------------
\clearpage
\phantomsection
\label{prf:bolzano-weierstrass}
\hypertarget{proof-RA-ABB-T-BW}{}

\subsubsection[Bolzano--Weierstrass Theorem]{Proof --- RA-ABB-T-BW}

\begin{remark}[Return]
\hyperref[thm:bolzano-weierstrass]{$\leftarrow$ Back to Theorem (Bolzano--Weierstrass) in Notes}
\end{remark}

\bigskip

\noindent
\textbf{Source.}
\srccite{AbbottUnderstandingAnalysis}{Chapter~2, \S2.5, Theorem~2.5.5}.

\vspace{0.75em}

\noindent
\textbf{Goal.}
Every bounded sequence in $\mathbb{R}$ has a convergent subsequence.

\vspace{0.75em}

\noindent
\textbf{Logical form.}
$\bigl(\exists M > 0,\ \forall n,\ |x_n| \le M\bigr)
\Rightarrow
\bigl(\exists\ \text{subsequence}\ (x_{n_k}),\ \exists L,\ x_{n_k} \to L\bigr)$.

\vspace{0.75em}

\noindent
\textbf{Key background.}
Monotone Subsequence Theorem (Tao / Abbott): every sequence has a monotone subsequence.
Monotone Convergence Theorem (MCT).
A subsequence of a bounded sequence is bounded.

\vspace{0.75em}

\noindent
\textbf{Proof strategy.}
Every sequence has a monotone subsequence (Monotone Subsequence Theorem).
A monotone subsequence of a bounded sequence is itself bounded and monotone.
By MCT, it converges.

\vspace{0.75em}

\noindent
\textbf{Proof.}
\begin{proof}
Let $(x_n)$ be bounded: $|x_n| \le M$ for all $n$.

\begin{longtable}{p{0.28\textwidth} p{0.35\textwidth} p{0.28\textwidth}}
\toprule
\textbf{Step / Claim} & \textbf{Justification} & \textbf{Annotation} \\
\midrule
$(x_n)$ has a monotone subsequence $(x_{n_k})$.
& Monotone Subsequence Theorem.
& Every sequence (bounded or not) has a monotone subsequence.
\\
\addlinespace
$(x_{n_k})$ is bounded: $|x_{n_k}| \le M$ for all $k$.
& $(x_{n_k})$ selects terms from $(x_n)$; $|x_n| \le M$ for all $n$.
& Boundedness passes to subsequences.
\\
\addlinespace
$(x_{n_k})$ is monotone and bounded.
& From the two steps above.
& MCT hypotheses are satisfied.
\\
\addlinespace
$(x_{n_k})$ converges by MCT.
& Monotone Convergence Theorem.
& The convergent subsequence is $(x_{n_k})$.
\\
\bottomrule
\end{longtable}
\end{proof}

\vspace{0.75em}

\noindent
\textbf{The Monotone Subsequence Theorem (proof sketch).}
A \emph{peak} of $(x_n)$ is an index $n$ such that $x_n \ge x_k$ for all $k \ge n$.
\textbf{Case 1:} If $(x_n)$ has infinitely many peaks $n_1 < n_2 < \cdots$, then
$(x_{n_k})$ is decreasing (each peak dominates all later terms).
\textbf{Case 2:} If peaks are finite, there exists $N$ after which no peaks occur.
Starting from any $n_1 > N$: since $n_1$ is not a peak, $\exists n_2 > n_1$ with $x_{n_2} > x_{n_1}$.
By induction, obtain a strictly increasing subsequence.

\vspace{0.75em}

\noindent
\textbf{Remarks.}
\begin{itemize}
  \item \textbf{Proof is a composition.} BW = Monotone Subsequence Theorem + MCT.
    Recognising the proof as a two-step composition is essential: BW itself
    provides very little new machinery; it assembles what is already there.

  \item \textbf{Peak argument.} The Monotone Subsequence Theorem (via peaks)
    is an elegant combinatorial trick. Peaks are the indices from which the
    sequence is eventually non-increasing. Their abundance determines which
    type of monotone subsequence is extracted.

  \item \textbf{Uses of BW.} Bolzano--Weierstrass is the backbone of:
    Cauchy Criterion (every Cauchy sequence has a convergent subsequence $\to$
    the subsequence limit is the full sequence limit), Heine--Borel theorem
    (compact $\Leftrightarrow$ closed and bounded in $\mathbb{R}^n$), and
    sequential compactness arguments throughout analysis.

  \item \textbf{Completeness dependency.} BW requires MCT, which requires
    the Completeness Axiom. The theorem fails in $\mathbb{Q}$ (bounded sequences
    in $\mathbb{Q}$ may not have convergent subsequences in $\mathbb{Q}$).
\end{itemize}
