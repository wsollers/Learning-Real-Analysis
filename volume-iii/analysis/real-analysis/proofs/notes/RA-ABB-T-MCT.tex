% --------------------------------------
% Proof: RA-ABB-T-MCT
% Theorem — Monotone Convergence Theorem
% Format: Deep study (3-column)
% --------------------------------------
\clearpage
\phantomsection
\label{prf:mct}
\hypertarget{proof-RA-ABB-T-MCT}{}

\subsubsection[Monotone Convergence Theorem]{Proof --- RA-ABB-T-MCT}

\begin{remark}[Return]
\hyperref[thm:mct]{$\leftarrow$ Back to Theorem (Monotone Convergence) in Notes}
\end{remark}

\bigskip

\noindent
\textbf{Source.}
\srccite{AbbottUnderstandingAnalysis}{Chapter~2, \S2.4, Theorem~2.4.2};
\srccite{TaoAnalysis1}{Chapter~6, Proposition~6.3.8}.

\vspace{0.75em}

\noindent
\textbf{Goal.}
If $(x_n)$ is increasing and bounded above, then $(x_n)$ converges to $\sup\{x_n : n \in \mathbb{N}\}$.

\vspace{0.75em}

\noindent
\textbf{Logical form.}
$\bigl((x_n)\ \text{increasing} \land \exists M,\ \forall n,\ x_n \le M\bigr) \Rightarrow x_n \to \sup\{x_n\}$.

\vspace{0.75em}

\noindent
\textbf{Key background.}
Completeness Axiom: the set $\{x_n\}$ is nonempty and bounded above, so $L = \sup\{x_n\}$ exists.
$\varepsilon$-characterization of supremum.

\vspace{0.75em}

\noindent
\textbf{Proof strategy.}
Set $L = \sup\{x_n\}$ (exists by completeness).
For any $\varepsilon > 0$, the $\varepsilon$-characterization gives some $x_N > L - \varepsilon$.
Since $(x_n)$ is increasing, all subsequent terms $x_n \ge x_N > L - \varepsilon$.
Since $L$ is the supremum, $x_n \le L$. So $|x_n - L| < \varepsilon$ for all $n \ge N$.

\vspace{0.75em}

\noindent
\textbf{Proof.}
\begin{proof}
Let $(x_n)$ be increasing and bounded above. Let $L := \sup\{x_n : n \in \mathbb{N}\}$ (exists by Completeness).

\begin{longtable}{p{0.28\textwidth} p{0.35\textwidth} p{0.28\textwidth}}
\toprule
\textbf{Step / Claim} & \textbf{Justification} & \textbf{Annotation} \\
\midrule
Let $\varepsilon > 0$ be arbitrary.
& Given in the convergence goal.
& We need to find $N$ depending on $\varepsilon$.
\\
\addlinespace
$\exists N \in \mathbb{N}$ with $x_N > L - \varepsilon$.
& $\varepsilon$-characterization of $\sup$: $L - \varepsilon$ is not an upper bound.
& This is the key use of completeness.
\\
\addlinespace
For all $n \ge N$: $x_n \ge x_N > L - \varepsilon$.
& $(x_n)$ is increasing: $n \ge N \Rightarrow x_n \ge x_N$.
& The tail of an increasing sequence stays above $x_N$.
\\
\addlinespace
For all $n$: $x_n \le L$.
& $L = \sup\{x_n\}$ is an upper bound.
& Every term is at most $L$.
\\
\addlinespace
For all $n \ge N$: $L - \varepsilon < x_n \le L$.
& Combining the two bounds above.
& This is equivalent to $|x_n - L| < \varepsilon$.
\\
\addlinespace
Hence $|x_n - L| < \varepsilon$ for all $n \ge N$.
& $L - \varepsilon < x_n \le L$ rearranges to $0 \le L - x_n < \varepsilon$.
& $(x_n) \to L$ by the $\varepsilon$-$N$ definition.
\\
\bottomrule
\end{longtable}
\end{proof}

\vspace{0.75em}

\noindent
\textbf{Remarks.}
\begin{itemize}
  \item \textbf{Completeness is the engine.} The proof has only one moving part:
    the Completeness Axiom gives $L = \sup\{x_n\}$, and the
    $\varepsilon$-characterization then provides the required $N$.
    Without completeness, neither step works.

  \item \textbf{Limit is the supremum.} The proof not only shows convergence
    but identifies the limit as the supremum of the range.
    This is the geometric content: the sequence \emph{climbs toward} its
    least upper bound.

  \item \textbf{Decreasing version.} If $(x_n)$ is decreasing and bounded below,
    then $x_n \to \inf\{x_n\}$. The proof is symmetric (replace
    $\sup$ with $\inf$ throughout).

  \item \textbf{Consequence.} MCT is the workhorse behind:
    Bolzano--Weierstrass (via monotone subsequence), the $e = \sum 1/n!$
    construction, and the proof that Cauchy sequences converge.
    Recognising when a sequence is monotone and bounded is a key skill.
\end{itemize}
