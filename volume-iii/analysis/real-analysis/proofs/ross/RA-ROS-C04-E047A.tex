% --------------------------------------
% Proof: RA-ROS-C04-E047A
% --------------------------------------
\clearpage
\phantomsection
\hypertarget{proof-RA-ROS-C04-E047A}{}

\section*{Proof — RA-ROS-C04-E047A}

\noindent
\hyperlink{ws-RA-ROS-C04-E047A}{\textbf{← Back to worksheet}}

\noindent
\textbf{Source.}
\srccite{RossElementaryAnalysis}{Chapter~4, Exercise~4.7(a)}.

\vspace{0.75em}

\noindent
\textbf{Goal.}
Let $S$ and $T$ be nonempty bounded subsets of $\mathbb{R}$.
Prove: if $S\subseteq T$, then
\[
\inf T \le \inf S \le \sup S \le \sup T.
\]

\vspace{0.75em}

\noindent
\textbf{Proof.}

Let $S$ and $T$ be nonempty bounded subsets of $\mathbb{R}$. Assume $S\subseteq T$.

Define
\[
\begin{aligned}
s_s := \sup S. \\
s_i := \inf S. \\
t_s := \sup T. \\
t_i := \inf T. \\
\end{aligned}
\]

 By the definition of infimum and supremum, we know 
\[
\begin{aligned}
s_i <= s_s \\
t_i <= t_s. \\
\end{aligned}
\]

Let $s$ be arbitrary element $\in S$. Since $S\subseteq T, s \in T$. Therefore, $t_i <= s <= t_s$. \\

Since $s <= t_s$ holds for all $s \in S$, $t_s$ is an upper bound of $S$ meaning $\sup S <= \sup T.$

Similarly since $t_i <= s$  holds for all $s \in S$, $t_i$ is a lower bound for $S$ meaning $\inf T <= \inf S.$

Since $\inf S <= \sup S$, 

\[
\inf T <= \inf S<= \sup S <= \sup T.
\]


\qed


\vspace{6cm}

\noindent
\hyperlink{ws-RA-ROS-C04-E047A}{\textbf{← Back to worksheet}}

